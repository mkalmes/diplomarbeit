% \documentclass[pdftex,a4paper,twoside,12pt,final,toc=bib]{scrbook}
\documentclass[pdftex,a4paper,twoside,12pt,draft,toc=bib]{scrbook}
\usepackage[ngerman]{babel}
\usepackage[T1]{fontenc}

\usepackage{amssymb,amsmath}
\providecommand{\e}[1]{\ensuremath{\times 10^{#1}}}

\usepackage{algorithm,algpseudocode}
\floatname{algorithm}{Algorithmus}
% \newcommand*\Cost[2]{\hfill \mbox{#1} \mbox{#2}}
\newcommand*\Cost[2]{\hfill \begin{tabular}{l r} #1 & #2 \\ \end{tabular}}
%\newcommand*\Cost[2]{\begin{tabular}{l r} #1 & #2 \\ \end{tabular}}


\usepackage{color}

\usepackage{setspace}
%\doublespacing		% doppelzeilig oder
\onehalfspacing		% anderthalbzeilig

% Umlaut können direkt als UTF-8 Zeichen eingegeben werden
\usepackage[utf8]{inputenc}

% Setup for fullpage use
\usepackage{fullpage}

% Surround parts of graphics with box
\usepackage{boxedminipage}

% Package for including code in the document
\usepackage{listings}

% Package for my comments
\usepackage{verbatim}
\usepackage{fancyvrb,relsize}

\DefineVerbatimEnvironment%
      {VerbatimProg}%
      {Verbatim}%
      {fontsize=\relsize{-1}}

% If you want to generate a toc for each chapter (use with book)
% \usepackage{minitoc}

% This is now the recommended way for checking for PDFLaTeX:
\usepackage{ifpdf}

\ifpdf
\usepackage[pdftex]{graphicx}
\else
\usepackage{graphicx}
\fi
\usepackage{subfigure}


\usepackage[bibstyle=authoryear,citestyle=authoryear,language=german,block=ragged,doi=false]{biblatex}
\DeclareFieldFormat[article]{citetitle}{\mkbibemph{#1\isdot}}
\DeclareFieldFormat[inbook]{citetitle}{\mkbibemph{#1\isdot}}
\DeclareFieldFormat[incollection]{citetitle}{\mkbibemph{#1\isdot}}
\DeclareFieldFormat[inproceedings]{citetitle}{\mkbibemph{#1\isdot}}

\usepackage{csquotes}
\bibliography{bibliography/bibliography}	% 7) vorläufige Literaturangabe


\usepackage[ngerman]{hyperref}
\hypersetup{
	a4paper,
	bookmarksnumbered = true,
	pdftitle = Marker-basierte Tracking-Verfahren für Augmented Reality Anwendungen unter iOS 4,
	pdfauthor = Marc Kalmes,
	pdfsubject = Diplomarbeit
}
\addto\extrasngerman{%
	\def\algorithmautorefname{Alg.}%
	\def\equationautorefname{Gl.}%
	\def\figureautorefname{Abb.}
	\def\subfigureautorefname{Abb.}
	\def\subsectionautorefname{Abschnitt}%
}

\usepackage[acronym]{glossaries}
% Acronyme
\newacronym{AR}{AR}{Augmented Reality}
\newacronym{dpi}{DPI}{dots per pixel}
\newacronym{pixel}{Pixel}{Bildelement}
\newacronym{chrominanz}{Chrominanz}{Farbigkeit}
\newacronym{edgels}{edgels}{Kantenpixel}
% Glossar
% Acronyme
\newacronym{AR}{AR}{Augmented Reality}
\newacronym{dpi}{DPI}{dots per pixel}
\newacronym{pixel}{Pixel}{Bildelement}
\newacronym{chrominanz}{Chrominanz}{Farbigkeit}
\newacronym{luminanz}{Luminanz}{Intensität}
\newacronym{edgel}{Edgel}{Kantenpixel}
% \newglossaryentry{pixel}{
% 	name={Pixel},
% 	description={Bildelement},
% 	plural=Pixels
% }
% \newglossaryentry{chrominanz}{
% 	name={Chrominanz},
% 	description={Farbigkeit}
% }
% \newglossaryentry{edgel}{
% 	name={Edgel},
% 	description={Kantenpixel},
% 	plural=Edgels
% }

\makeglossaries
%
%  Created by Marc Kalmes on 2010-06-19.
%

% \documentclass[pdftex,a4paper,twoside,12pt,final,toc=bib]{scrbook}
\documentclass[pdftex,a4paper,twoside,12pt,draft,toc=bib]{scrbook}
\usepackage[ngerman]{babel}
\usepackage[T1]{fontenc}

\usepackage{amssymb,amsmath}
\providecommand{\e}[1]{\ensuremath{\times 10^{#1}}}

\usepackage{algorithm,algpseudocode}
\floatname{algorithm}{Algorithmus}
\newcommand{\algorithmautorefname}{Alg.}
% \newcommand*\Cost[2]{\hfill \mbox{#1} \mbox{#2}}
\newcommand*\Cost[2]{\hfill \begin{tabular}{l r} #1 & #2 \\ \end{tabular}}
%\newcommand*\Cost[2]{\begin{tabular}{l r} #1 & #2 \\ \end{tabular}}


\usepackage{color}

\usepackage{setspace}
%\doublespacing		% doppelzeilig oder
\onehalfspacing		% anderthalbzeilig

% Umlaut können direkt als UTF-8 Zeichen eingegeben werden
\usepackage[utf8]{inputenc}

% Setup for fullpage use
\usepackage{fullpage}

% Surround parts of graphics with box
\usepackage{boxedminipage}

% Package for including code in the document
\usepackage{listings}

% Package for my comments
\usepackage{verbatim}
\usepackage{fancyvrb,relsize}

\DefineVerbatimEnvironment%
      {VerbatimProg}%
      {Verbatim}%
      {fontsize=\relsize{-1}}

% If you want to generate a toc for each chapter (use with book)
% \usepackage{minitoc}

% This is now the recommended way for checking for PDFLaTeX:
\usepackage{ifpdf}

\ifpdf
\usepackage[pdftex]{graphicx}
\else
\usepackage{graphicx}
\fi


\usepackage[bibstyle=authoryear,citestyle=authoryear,language=german,block=ragged,doi=false]{biblatex}
\DeclareFieldFormat[article]{citetitle}{\mkbibemph{#1\isdot}}
\DeclareFieldFormat[inbook]{citetitle}{\mkbibemph{#1\isdot}}
\DeclareFieldFormat[incollection]{citetitle}{\mkbibemph{#1\isdot}}
\DeclareFieldFormat[inproceedings]{citetitle}{\mkbibemph{#1\isdot}}

\usepackage{csquotes}
\bibliography{bibliography/bibliography}	% 7) vorläufige Literaturangabe


\usepackage{hyperref}

\renewcommand {\chapterautorefname}{Kapitel}

\hypersetup{
	a4paper,
	bookmarksnumbered = true,
	pdftitle = Tracking Verfahren für Augmented Reality Anwendungen unter iOS 4,
	pdfauthor = Marc Kalmes ,
	pdfsubject = Exposé
}

\usepackage{glossaries}
% Acronyme
\newacronym{AR}{AR}{Augmented Reality}
\newacronym{dpi}{DPI}{dots per pixel}
\newacronym{pixel}{Pixel}{Bildelement}
\newacronym{chrominanz}{Chrominanz}{Farbigkeit}
\newacronym{edgels}{edgels}{Kantenpixel}
% Glossar
\makeglossaries

\begin{document}
\frontmatter

\ifpdf
\DeclareGraphicsExtensions{.pdf, .jpg, .tif}
\else
\DeclareGraphicsExtensions{.eps, .jpg}
\fi

% %\thispagestyle{empty}
%\begin{center}
%\Large{Privatheit und Öffentlichkeit am Modell des Internet}
%\end{center}
%Projektarbeit
%vorgelegt an der Fachhochschule Köln
%Campus Gummersbach
%im Studiengang Allgemeine Informatik
%ausgearbeitet von:
%Marc Kalmes
%\end{document}
\begin{titlepage}
 
\begin{center}
 
 
% Upper part of the page
\includegraphics[width=0.15\textwidth]{./resources/logo}\\[1cm]
 
\textsc{\huge \bfseries Privatheit und Öffentlichkeit am Modell des Internet}\\[2.0cm]
 
\textsc{\Large Projektarbeit}\\[1.5cm]
\Large vorgelegt an der Fachhochschule Köln\\[0.2cm]
\Large Campus Gummersbach\\[0.2cm]
\Large im Studiengang Allgemeine Informatik\\[1.5cm]
 
 
% Title
%\HRule %\\[0.4cm]
%{ \huge \bfseries Lager brewing techniques}\\[0.4cm]
 
%\HRule \\[1.5cm]
 
% Author and supervisor
\begin{minipage}{0.4\textwidth}
\begin{flushleft} \large
\emph{ausgearbeitet von:}\\
Marc \textsc{Kalmes} -- 11025526
\end{flushleft}
\end{minipage}
\begin{minipage}{0.4\textwidth}
\begin{flushright} \large
\emph{Betreuerin:}\\
Prof. Dr. Gabriele \textsc{Koeppe}
\end{flushright}
\end{minipage}
 
\vfill
 
% Bottom of the page
{\large Gummersbach, im Mai 2009}
 
\end{center}
 
\end{titlepage}	% TODO: Titel als eigenes tex-dokument setzen
\titlehead{Fachhochschule Köln\\ Fakultät für Informatik und Ingenieurwissenschaften}
\subject{Diplomarbeit}
\title{Marker-basierte Tracking-Verfahren für Augmented Reality Anwendungen unter iOS 4}
\author{Marc Kalmes}
\date{Juli 2011}
\publishers{betreut durch Prof. Dr. Heiner Klocke}

\maketitle

\tableofcontents

\selectlanguage{UKenglish}
\chapter*{Abstract} % (fold)
\label{cha:abstract}
The diploma thesis \emph{Marker-basierte Tracking-Verfahren für Augmented Reality Anwendungen unter iOS 4} asks the question whether tracking systems are practical on mobile devices. A tracking system by \citeauthor{hirzer08} was implemented within this thesis and compared with ARToolKitPlus.

Tracking system were realized in the field of Augmented Reality research on powerful computers. Augmented Reality on mobile devices has to cope with limited memory and restricted processing power in contrast to a computer. The requirements of tracking systems will be explained. With the requirement definitions the question should be answered wether a tracking system is capable to run on the limited resources of a mobile device.

The algorithms of the system by \citeauthor{hirzer08} and the algorithms by ARToolKitPlus will be analyzed and the asymptotic run time determined and evaluated. An additionally rum time measurement creates another comparison feature.

The thesis explains finally how the described requirement of the methods were met and which pros and cons ARToolKitPlus and the system by \citeauthor{hirzer08} exhibit.
% chapter abstract (end)

\selectlanguage{ngerman}
\chapter*{Zusammenfassung} % (fold)
\label{cha:abstract-deu}
Die Diplomarbeit \emph{Marker-basierte Tracking-Verfahren für Augmented Reality Anwendungen unter iOS 4} beschäftigt
 sich mit der Frage, ob Tracking-Verfahren für mobile Endgeräte geeignet sind. Im Rahmen dieser Arbeit wurde ein
 Tracking-Verfahren nach \citeauthor{hirzer08} implementiert und mit dem Tracking-Verfahren von ARToolKitPlus
 verglichen.

Tracking-Verfahren wurden in der Forschung der Augmented Reality auf leistungsfähigen Computern durchgeführt. Der
 Einsatz von Augmented Reality auf Smartphones muss somit mit weniger Arbeitsspeicher und weniger Prozessorleistung
 auskommen, als auf einem Computer. Die Anforderungen eines Tracking-Verfahren wird in dieser Arbeit erläutert. Anhand
 der Anforderungen soll die Frage beantwortet werden, ob Tracking Verfahren mit den limitierten Resourcen eines
 mobilen Endgerätes ausgeführt werden können.

Die Algorithmen des Verfahrens nach \citeauthor{hirzer08} und die Algorithmen von ARToolKitPlus werden analysiert und
 ihre Laufzeitfunktionen, sowie das asymptotische Wachstum der Verfahren, bestimmt und bewertet. Zusätzlich wird durch
 die Messung der Laufzeit ein Vergleichsmermal erstellt.

Die Diplomarbeit erläutert abschließend, wie die vorgestellten Verfahren die Anforderungen an Tracking-Verfahren
 erfüllen und welche Stärken und Schwächen ARToolKitPlus und das Verfahren von \citeauthor{hirzer08} aufweisen.
% chapter zusammenfassung (end)

\mainmatter

\chapter{Einleitung} % (fold)
\label{cha:einleitung}

Heutzutage können wir zwei Phänomene beobachten: Das erste Phänomen ist die Zunahme von \gls{AR} Anwendungen außerhalb
 der Forschung. \gls{AR} verspricht neue Bedienkonzepte und Informationsdarstellung durch Überlagerung von realen und
 virtuellen Bil\-dern. Die dazu benötigte Ausrüstung bestand viele Jahre aus einem so genannten Head-mounted Display
 und einem Computer in einem Rucksack\footcite{azuma01}.

Das zweite Phänomen ist die weite Verbreitung von Smartphones und deren ge\-sell\-schaft\-liche Akzeptanz. Dabei wird
 die Leistungsfähigkeit von Smartphones stetig ge\-stei\-gert und Anwendungen sind als Folge nicht mehr auf den
 Computer beschränkt. Gerade Spiele profitieren von der Prozessor\-ge\-schwin\-dig\-keit und den heute üblichen
 Grafikprozessoren moderner Smartphones. Auch die mobile Nutzung von E-Mail und Internet hat durch Smartphones
 zugenommen. Aus diesen Gründen bietet es sich an, die Mög\-lich\-kei\-ten der \gls{AR} auf Smartphones zu untersuchen.

\section{Forschungsstand} % (fold)
\label{sec:forschungsstand}
\begin{comment}
	Forschungsstand: Alle untersuchten Arbeiten aufführen und kurz erklären.
\end{comment}

ARToolKit\footcite{artoolkit} wurde 1999 von Kato entwickelt und als Open Source Software einer breiten
 Entwicklergmeinde zur Verfügung gestellt. ARToolKit galt als Referenz der Forschung im Bereich der \gls{AR}.
 Allerdings war der Systementwurf nicht auf das Aufkommen von mobilen Computern und Smartphones vorbereitet.

Im Jahre 2004 wurde von \citeauthor{wagner04} ARToolKit für die Windows CE Plattform portiert\footcite{wagner04}. Diese
 Portierung führt zur Entwicklung von ARToolKitPlus\footcite{artoolkitplus}, das nicht nur für Workstations, sondern
 auch für mobile Geräte, entworfen wurde\footcite{wagner07b}. \citeauthor{wagner07a} machte diese Entwicklung zum Thema
 seiner Dissertation\footcite{wagner07a}. Die Schwächen von ARToolKit wurden überarbeitet und das System auf den
 aktuellen Stand der Forschung gebracht. Neue Verfahren zur robusten Mar\-ken\-er\-kennung in ARToolKitPlus waren aber
 nicht in der Lage mit der begrenzten Prozessorgeschwindigkeit mobiler Geräte ausgeführt zu werden.

Die Erfahrung von \citeauthor{wagner04} bei der Entwicklung von ARToolKitPlus wurde genutzt um
 Studierstube\footcite{studierstube} zu entwickeln. Studierstube ist das erste System, dass auf die Bedürfnisse von
 Smartphones und mobilen Geräten konzipiert wurde und von Grund auf neu entwickelt wurde. Anders als ARToolKit und
 ARToolKitPlus ist Studierstube nicht im Quellcode verfügbar.
\citeauthor{wagner09a} erläutern in \citetitle{wagner09a}\footcite{wagner09a} und
 \citetitle{wagner09b}\footcite{wagner09b} einen Überblick über die Strategien und Entwicklung von Studierstube.

In \citetitle{hirzer08}\footcite{hirzer08} wird von \citeauthor{hirzer08} ein Verfahren zur robosten Markenerkennung
 vorgestellt und basiert auf der Arbeit\footcite{clarke96} von \citeauthor{clarke96}. Die robuste Erkennung von Linien
 und die hohe Fehlertoleranz gegenüber schwankenden Lichtverhältnissen zeichnen dieses Verfahren aus um als Grundlage
 zur Markenerkennung verwendet zu werden.

% section forschungsstand (end)
\section{Definition} % (fold)
\label{sec:definition}
\begin{comment}
	Definiere Begriffe der Augmented Reality und Bildverarbeitung, die dem Leser nicht geläufig sind. Denke dabei an Prof. Klocke als Leser ohne besonderen Kenntnisstand in der AR/Bildverarbeitung.

	Azuma Definition der AR
	Milgram + Kishino MR oder nur Milgram Definition MR
\end{comment}

Bis heute ist \gls{AR} nicht eindeutig definiert, obwohl die ersten Untersuchung von \citeauthor{sutherland} bereits
 1968 vorgestellt wurden\footcite{sutherland}. \citeauthor{milgram94b} beschrieben in
 \citetitle{milgram94b}\footcite{milgram94b} \gls{AR} als einen Punkt des Reality-Virtuality (RV) Continuum
 (Vgl. \autoref{fig:mixedreality}), welches reale und virtuelle Objekte gemeinsam auf einem Bildschirm darstellt.
\begin{figure}[!ht]
	\centering
	\input{resources/mixedreality.pdf_tex}
	\caption{Reality-Virtual Continuum nach \citeauthor{milgram94b}}
	\label{fig:mixedreality}
\end{figure}

Nach der neueren Definition von \citeauthor{azuma97}\footcite{azuma97} muss ein \gls{AR}-System drei Kriterien
 erfüllen:
\begin{itemize}
	\item Reale und virtuelle Objekte werden kombiniert
	\item Bezug von realen und virtuellen Objekten im 3-dimensionalen Raum
	\item Interaktion in Echtzeit
\end{itemize}
Das erste Kriterium bedeutet, dass, im Gegensatz zu Computerspielen, virtuelle Gegenstände mit der realen Welt in
 Verbindung stehen. Durch die zweite Bedingung muss ein virtuelles Objekt in einem räumlichen Zusammenhang stehen. Als
 Beispiel muss eine virtuelle Tasse auf dem realen Schreibtisch platziert sein und nicht im Raum schweben. Zuletzt muss
 die Interaktion mit einem \gls{AR}-System in Echtzeit erfolgen. Dies ist eine Abgrenzung zu einer Computeranimation in
 einem Film.

Nach \citeauthor{moeller2008}\footcite[Vgl.][S.~1]{moeller2008} ist Interaktion in Echtzeit mit $15$ FPS erfüllt, wobei
 eine Geschwindigkeit von $6$ FPS noch als interaktiv gilt. Diese Definition wird von \citeauthor{wagner09b} in ihrer
 Untersuchung\footcite[Vgl.][S.~8--9]{wagner09b} unterstützt.

\subsection{Tracking Verfahren} % (fold)
\label{sec:tracking_verfahren}
Tracking Verfahren bezeichnet bei \gls{AR}-Systemen die Fähigkeit, die reale Position eines Benutzers oder eines
 Displays in Relation zur Position einer \gls{AR}-Umgebung zu setzen. Tracking Verfahren müssen, wie in der Definition
 von \citeauthor{azuma97} schon erwähnt, in Echtzeit agieren. Zusätzlich müssen Tracking Verfahren, je nach
 Anwendungsgebiet, unter unterschiedlichen Lichtverhältnissen arbeiten können und eine Marke zuverlässig erkennen
 können.

Ein Bereich der Tracking Verfahren ist das Fiducial Tracking, auch Markenbasiertes Tracking genannt. Bei diesem
 Tracking Verfahren wird durch Hilfe einer Marke die Position im realen und virtuellen Raum berechnet. Die bereits
 erwähnten Systeme ARToolKit, ARToolKitPlus und Studierstube, sind Markenbasierte Verfahren und suchen in einem
 Bildsignal nach einer bitonalen Marke. Das erste Markenbasierte \gls{AR}-System war
 Matrix\footcite{rekimoto1998matrix} von \citeauthor{rekimoto1998matrix}. \citeauthor{fiala2004artaga} entwickelte mit
 ARTag\footcite{fiala2004artaga} ein \gls{AR}-System, dass die Idee eines 2D-Codes zur Identifizierung von Matrix
 übernahm. In ARToolKitPlus wurde die Identifizierung von 2D-Codes ebenfalls implementiert.
% subsection tracking_verfahren (end)

\subsection{Bitonale Marken} % (fold)
\label{sub:bitonalemarken}
Bei Markenbasierten \gls{AR}-Verfahren werden bitonale Marken zur Erkennung und Verfolgung von realen Objekten
 eingesetzt. Eine Marke besteht aus einem schwarzen quadratischen Rahmen auf weißem Untergrund und trägt in der Mitte
 Informationen zur Identifizierung. ARToolKit verwendet simple Muster oder Schriftzeichen zur Identifizierung, wie in
 \autoref{fig:marke-artoolkit} gezeigt. ARTag hingegen verwendet eine binär kodierte Zahl als Identifizierungsmerkmal,
 die als Anordnung von $6 \times 6$ Pixeln dargestellt wird (s.~\autoref{fig:marke-artag}). ARToolKitPlus, der
 Nachfolger von ARToolKit nutzt ein ähnliches Verfahren wie ARTag\footcite[Vgl.][S.~142]{wagner07b} und unterscheidet
 sich in der Rahmenbreite (s.~\autoref{fig:marke-artoolkitplus}). Wenn in den folgenden Kapiteln von einer bitonalen
 Marke, oder einfach Marke, gesprochen wird, beziehe ich mich auf einen schwarzen quadratischen Rahmen auf weißem
 Untergrund.

\begin{figure}[!ht]
	\centering
	\subfigure[ARToolKit]{
		\label{fig:marke-artoolkit}
		\includegraphics[scale=1]{resources/Marker-ART.pdf}
	}
	\subfigure[ARTag]{
		\label{fig:marke-artag}
		\includegraphics[scale=1]{resources/Marker-ARTag.pdf}
	}
	\subfigure[ARToolKitPlus]{
		\label{fig:marke-artoolkitplus}
		\includegraphics[scale=1]{resources/Marker-ART+.pdf}
	}
	\caption{Verschiedene bitonale Marken.
	}
	\label{fig:bitonale-marken}
\end{figure}
% subsection bitonale_marken (end)

\begin{comment}
	(wagner/schmalstieg ARToolKitPlus fpr Pose Tracking on Mobile Devices S.4)

	Beim Fiducial Marker Tracking werden künstliche Marken zur Erkennung und Verfolgung von realen Objekten eingesetzt. Häufig werden für diese Marken quadratische schwarze Rahmen verwendet die innerhalb des Rahmens Schriftzeichen, Bilder oder 2D Codes enthalten. Diese Marken sind einfach herzustellen und können mit geringem Aufwand an Objekte angebracht werden.

	Diese bitonale Marken haben den Vorteil, dass sie Robust gegen Helligkeitsveränderung sind und die Entscheidung eines Pixels auf eine Schwellwert-Entscheidung reduziert werden kann. Marken für \gls{AR} müssen in einem großen Blickfeld erkannt werden können, was wiederum bei industriellen Anwendung nicht der Fall ist, da hier Marken den größten Teil des Bildes einnehmen können.
\end{comment}

% section definition (end)
\section{Problemstellung} % (fold)
\label{sec:problemstellung}
\begin{comment}
	Problemstellung: Problemstellung und Frage im Detail erläutern
\end{comment}

Wie in Kapitel \ref{sec:forschungsstand} schon erwähnt, gibt es keine aktuellen und im Quellcode verfügbaren Systeme zur Entwicklung von \gls{AR} Anwendungen. Somit war eine Eigenentwicklung unumgänglich.

Um eine bitonale Marke, üblicherweise Schwarz/Weiß, zu erkennen, werden Verfahren zur Erkennung von Linien eingesetzt, die aus dem Bereich der digitalen Bildverarbeitung stammen. Anders als bei der herkömmlichen Bildverarbeitung muss Bildverarbeitung für \gls{AR} die Verarbeitungsschritte so schnell ausführen, dass eine Analyse eines Bildes in Echtzeit stattfinden kann. Bei Bildverarbeitung ausserhalb der \gls{AR} ist dies kein notwendiges Kriterium. Als Grundlage für eine Eigenentwicklung kommt das Verfahren von \citeauthor{clarke96}\footcite{clarke96} zum Einsatz, dass in Kapitel \ref{sub:verfahren_nach_hirzer} vorgestellt wird.

Der Bezug von Bildern über den Kamerasensor eines Smartphones muss so schnell wie möglich erfolgen um für die Weiterverarbeitung verwendet werden zu können. Bei Smartphones werden in aller Regel Kamerasensoren aus dem Verbraucherbereich verwendet. Diese Kamerasysteme liefern nicht die Performanz wie professionelle Kameras. Daraus bedingt müssen die zur Verfügung stehenden Mittel so gut es geht ausgenutzt werden. Beim Bezug der Bilder ist darauf zu achten, dass die Geschwindigkeit bei Bildtypenkonvertierungen nicht darunter leidet.

Die Softwarearchitektur für \gls{AR} auf Smartphones unterscheidet sich von einer PC Architektur, abgesehen von der Prozessorgeschwindigkeit, grundsätzlich durch die Menge und Bandbreite des Arbeitsspeichers. Dies bedeutet, dass alle Arbeitsschritte durch eine ineffektive Ausnutzung des zur Verfügung stehenden Speichers verlangsamt werden.

Die Software kann nach gängigen Entwurfskriterien entworfen werden wobei die Vor- und Nachteile von dynamischen und statischen Bibliotheken berücksichtigt werden müs\-sen.

% section problemstellung (end)

% chapter einleitung (end)				% bis zu 10% der Arbeit
\chapter{Untersuchungsdesign} % (fold)
\label{cha:untersuchungsdesign}

\section{Digitale Bildverarbeitung} % (fold)
\label{sec:bildverarbeitung}
\begin{comment}
	Bildverarbeitung: Notwendige Verfahren und Konzepte erläutern.
\end{comment}

\begin{comment}
	YUV und RGB bzw. RGBA
	kCVPixelFormatType_420YpCbCr8BiPlanarVideoRange, kCVPixelFormatType_420YpCbCr8BiPlanarFullRange and kCVPixelFormatType_32BGRA, except on iPhone 3G, where the supported pixel formats are kCVPixelFormatType_422YpCbCr8 and kCVPixelFormatType_32BGRA..(AVCaptureVideoDataOutput Class Reference)

	\subsection{Grundlagen} % (fold)
	\label{sec:grundlagen}
	Ein digitales Bild ist definiert durch seine Bildgröße und der Auflösung. Die Bildgröße ist in Höhe und Breite angegeben und eine entsprechende Bildmatrix kann durch die Bildspalten $u$ und Bildzeilen $v$ angegeben werden. Die Auflösung eines Bildes bezeichnet die räumliche Ausdehnung in \gls{dpi}, die in den meisten Schritten der Bildverarbeitung vernachlässigt werden kann.

Um auf ein \gls{pixel} aus der Bildmatrix zugreifen zu können, benötigen wir ein Koordinatensystem. Bei digitalen
 Bildern unterscheidet sich das verwendete Koordinatensystem von einem kartesischen Koordinatensystem dadurch, dass der
 Ursprungspunkt bei Bildern links oben liegt. Die \(x\)-Achse verläuft von links nach rechts und die \(y\)-Achse von
 oben nach unten. TODO: vgl Abbildung.

	Die Information eines \gls{pixel} ist als binärer Wert mit der Länge \(k\) gespeichert. Der Wertebereich eines \gls{pixel} umfasst \(\left[0 \dotsc 2^k\right]\), wobei der genaue Wertebereich abhängig vom eingesetzten Typ ist.

	Man unterscheidet im Allgemeinen zwischen Farb-, Monochrom- und Binärbildern, die eine direkte Auswirkung auf den Wertebereich haben. Bei Farbbildern wird häufig eine Komponente für Rot, Grün und Blau verwendet und ist typischerweise in 8 Bits kodiert. Ein Pixel besteht somit aus \(3 \cdot 8 = 24\) Bits mit einem Wertebereich von \(\left[0 \dotsc 255\right]\) pro Farbkomponente.\\TODO: Bild mit Beschreibung des Speichers für RGB

	Monochrombilder bestehen nur aus einem Intensitätskanal der ebenfalls mit 8 Bits kodiert wird. Der Wertebereich eines Pixel entspricht \(\left[0 \dotsc 255\right]\).\\TODO: Bild mit Speicherbeschreibung für Monochrom

	Bei Binärbildern werden Informationen nur in einem Bit gespeichert und der Wert entspricht somit \(0\) oder \(1\) für Schwarz oder Weiß.\\TODO: Bild mit Speicherbeschreibung für Binär.
	% section grundlagen (end)
\end{comment}

Bevor ich mich den \gls{AR}-Verfahren widme, möchte ich die grundlegenden Aspekte der digitalen Bildverarbeitung
 benennen, die in den nachfolgenden Kapiteln benötigt werden. Dabei werde ich keinen kompletten Überblick über Digitale
 Bildverarbeitung vermitteln, sondern nur Bereiche besprechen die für die Untersuchung der \gls{AR} berücksichtigt
 werden müssen.

\subsection{Grundlagen} % (fold)
\label{sub:grundlagen}

Ein digitales Bild ist eine Ansammlung numerischer Werte, die in einem Array gespeichert sind. Durch die Höhe und Breite des Bildes kann das Array, wie in \autoref{fig:bildmatrix}, als Matrix dargestellt werden.
\begin{figure}[!ht]
	\centering
	\input{resources/Bildmatrix.pdf_tex}
	\caption{Digitales Bild in Matrixdarstellung.}
	\label{fig:bildmatrix}
\end{figure}
Die Höhe $N$ entspricht dann den Zeilen und die Breite $M$ den Spalten der Matrix. Ein Bild $I$ ist somit definiert als
 eine Funktion auf einer Menge von Bildwerten $\mathbb{P}$, wie in \autoref{eq:digitalesBild} dargestellt ist.
\begin{equation}
	I\left(u,v\right)\in\mathbb{P} \text{ und } u,v\in\mathbb{N}
	\label{eq:digitalesBild}
\end{equation}
Um auf ein \gls{pixel} aus der Bildmatrix zugreifen zu können, bedienen wir uns eines Koordinatensystems. Im
 Unterschied zu einem kartesischen Koordinatensystem ist bei Bildern der Ursprungspunkt links oben. Die \(x\)-Achse
 verläuft von links nach rechts und die \(y\)-Achse von oben nach unten. \begin{comment}\\TODO:vgl abbildung\end{comment}
Welche Werte in einem \gls{pixel} gespeichert sind und wie diese Werte zu interpretieren sind, ist abhängig vom
 verwendeten Bildtyp.

% section grundlagen (end)

\subsection{Bildtypen} % (fold)
\label{sub:bildtypen}

Der Wert eines \gls{pixel} ist als binärer Wert der Länge $k$ angegeben. Darüber hinaus bestimmt der verwendete Bildtyp
 über die weiteren Informationen eines \gls{pixel}.

Monochrome Bilder, die im allgemeinen Sprachgebrauch als Schwarzweiß-Bilder bezeichnet werden, besitzen nur eine
 Komponente, die als \gls{luminanz} bezeichnet wird. Die Information des \gls{pixel} wird im Allgemeinen
 mit 8 Bit kodiert, sodass sich ein Wertebereich von $2^8 = \left[0\dotsc255\right]$ ergibt.

Binäre Bilder sind eine spezielle Form der monochromen Bilder. In binären Bildern werden nur Schwarzweiß-Werte
 gespeichert. Die Information kann dadurch in nur einem Bit, mit $0$ für Schwarz und $1$ für Weiß, gespeichert werden.

Farbbilder speichern im Unterschied zu monochromen Bildern mehr Informationen in einem \gls{pixel} ab. In den meisten
 Fällen wird die Information eines Farbbildes in drei Komponenten für Rot, Grün und Blau, mit jeweils 8 Bit, kodiert.
 Somit besteht ein Pixel aus \(3 \cdot 8 = 24\) Bits mit einem Wertebereich von \(\left[0 \dotsc 255\right]\) pro
 Farbkomponente.

% Der Aufbau der Farbinformationen ist abhängig von der Anordnung und wird in \autoref{sub:pixelanordnung} genauer
%  definiert.

% subsection bildtypen (end)
% 
% \subsection{Farbräume} % (fold)
% \label{sub:farbraeume}
% 
% Bei digitaler Bildverarbeitung gibt es unterschiedliche Farbräume für verschiedene Einsatzgebiete. Für diese
%  Untersuchung werde ich den RGB-Farbraum, sowie YUV und YCbCr, vorstellen.
% 
% \subsubsection{RGB} % (fold)
% \label{sec:rgb}
% 
% Der RGB-Farbraum ist in in der digitalen Bildverarbeitung der am häufigsten eingesetzte Farbraum und besteht aus roten,
%  grünen und blauen Komponenten (\autoref{fig:rgbLenna}).
% \begin{figure}[!ht]
% 	\centering
% 	\subfigure[]{
% 		\label{fig:rgbLenna-Farbe}
% 		\includegraphics[width=.2\textwidth]{resources/Lenna.pdf}
% 	}
% 	\subfigure[]{
% 		\label{fig:rgbLenna-Rot}
% 		\includegraphics[width=.2\textwidth]{resources/Lenna_R.pdf}
% 	}
% 	\subfigure[]{
% 		\label{fig:rgbLenna-Gruen}
% 		\includegraphics[width=.2\textwidth]{resources/Lenna_G.pdf}
% 	}
% 	\subfigure[]{
% 		\label{fig:rgbLenna-Blau}
% 		\includegraphics[width=.2\textwidth]{resources/Lenna_B.pdf}
% 	}
% 	\caption{Ein farbiges Bild \subref{fig:rgbLenna-Farbe} wird im RGB-Farbraum in seine Komponenten Rot
% 		 \subref{fig:rgbLenna-Rot}, Grün \subref{fig:rgbLenna-Gruen} und Blau \subref{fig:rgbLenna-Blau} aufgeteilt.
% 		Von \cite{lenna}.
% 	}
% 	\label{fig:rgbLenna}
% \end{figure}
% 
% Bei RGB handelt es sich um ein additives Farbsystem, dass durch Mischung der Primärfarben einen Farbton und dessen
%  Helligkeit generiert\footcite[Vgl.][S.~233]{burger05}. Der RGB Farbraum kann als dreidimensionaler Würfel dargestellt
%  werden (\autoref{fig:rgbWuerfel}).
% \begin{figure}[!ht]
% 	\centering
% 	\def\svgwidth{.2\columnwidth}
% 	\input{resources/RGBWuerfel.pdf_tex}
% 	\caption{RGB-Farbwürfel}
% 	\label{fig:rgbWuerfel}
% \end{figure}
% Seine Koordinatenachsen stellen den Wertebereich für die Grundfarben Rot, Grün und Blau dar. Normiert man den Würfel zu
%  einem Einheitswürfel der Länge $1$, umfasst jede Achse einen Bereich von \(\left[0 \dotsc 1.0\right]\). Innerhalb
%  dieses Würfels kann durch Verschiebung der Koordinate jeder Farbton generiert werden. Die Grundfarben können in diesem
%  Würfel durch die Punkte $R = \left(1,0,0\right)$, $G = \left(0,1,0\right)$ und $B = \left(0,0,1\right)$ dargestellt
%  werden. (Vgl. \autoref{tbl:rgbwerte})
% \begin{table}[!ht]
% 	\begin{center}
% 	\begin{tabular}[]{r|c|c|c}
% 	Farbe & R & G & B \\ \hline\hline
% 	rot & 1.0 & 0.0 & 0.0 \\
% 	grün & 0.0 & 1.0 & 0.0 \\
% 	blau & 0.0 & 0.0 & 1.0 \\
% 	schwarz & 0.0 & 0.0 & 0.0 \\
% 	weiß & 1.0 & 1.0 & 1.0 \\
% 	grau & 0.5 & 0.5 & 0.5 \\
% 	\end{tabular}
% 	\caption{RGB-Werte}
%  	\label{tbl:rgbwerte}
% 	\end{center}
% \end{table}
% Schwarz ($S = \left(0,0,0\right)$) und Weiß ($W = \left(1,1,1\right)$) werden auf die gleiche Weise dargestellt.
% 
% % subsubsection rgb (end)
% 
% \subsubsection{YUV und YCbCr} % (fold)
% \label{sec:yuv_und_ycbcr}
% 
% YUV und YCbCr sind standardisierte Formate zur Aufnahme, Darstellung und Übertragung von Bildern im Fehrnsehbereich und
% stellen andere Farbräume dar. YUV findet Verwendung bei analogen PAL und NTSC Systemen während YCbCr bei digitalen
% Systemen verwendet wird. Beide Farbräume unterteilen die Informationen in eine Luminanzkomponente $Y$ und zwei
% Komponenten zur Darstellung unterschiedlicher \gls{chrominanz} (\autoref{fig:yuvLenna}).
% \begin{figure}[!ht]
% 	\centering
% 	\subfigure[]{
% 		\label{fig:yuvLenna-Farbe}
% 		\includegraphics[width=.2\textwidth]{resources/Lenna.pdf}
% 	}
% 	\subfigure[]{
% 		\label{fig:yuvLenna-Y}
% 		\includegraphics[width=.2\textwidth]{resources/Lenna_Y.pdf}
% 	}
% 	\subfigure[]{
% 		\label{fig:yuvLenna-U}
% 		\includegraphics[width=.2\textwidth]{resources/Lenna_U.pdf}
% 	}
% 	\subfigure[]{
% 		\label{fig:yuvLenna-V}
% 		\includegraphics[width=.2\textwidth]{resources/Lenna_V.pdf}
% 	}
% 	\caption{Im YUV-Farbraum wird ein Bild \subref{fig:yuvLenna-Farbe} in Luminanz $Y$ \subref{fig:yuvLenna-Y},
% 		Chroma $U$ \subref{fig:yuvLenna-U} und Chroma $V$ \subref{fig:yuvLenna-Y} aufgeteilt. Von
% 		 \cite{lenna}.
% 	}
% 	\label{fig:yuvLenna}
% \end{figure}
% 
% Bei YUV ist die Chromakomponente $U$ als Differenz zwischen der Luminanz $Y$ und dem Blauanteil definiert (Vgl.~\autoref{eq:uchroma}).
% \begin{equation}
% 	U = 0.492 \cdot \left(B-Y\right)
% 	\label{eq:uchroma}
% \end{equation}
% Die Chromakomponente $V$ definiert die Differenz von Luminanz $Y$ und dem Rotanteil (Vgl.~\autoref{eq:vchroma}).
% \begin{equation}
% 	V = 0.877 \cdot \left(R-Y\right)
% 	\label{eq:vchroma}
% \end{equation}
% Die Chromakomponenten $Cb$ und $Cr$ von YCbCr sind, wie bei YUV, Differenzwerte des Blau- und Rotanteil und der
%  \gls{luminanz}. Der Unterschied zwischen YUV und YCbCr besteht in der unterschiedlichen Berechnung der
%  \gls{chrominanz} und ist in \citeauthor{burger05}\footcite[][S.265--266]{burger05} beschrieben.
% 
% Durch die Trennung der gls{luminanz} und \gls{chrominanz} ist es möglich das Signal auf Schwarzweiß-Fernsehern zu
%  nutzen, indem nur die Luminanzkomponente berücksichtigt wird. Bei Farbfernsehern werden zusätzlich die
%  Chromakomponenten verwendet.
% 
% % subsubsection yuv_und_ycbcr (end)
% 
% % subsection farbräume (end)
% 
% \subsection{Pixel-Anordnung} % (fold)
% \label{sub:pixelanordnung}
% Bei Pixeln gibt es unterschiedliche Modelle der Anordnung von Farbkomponenten. Man unterscheidet die
%  Komponentenanordnung, auch planare Anordnung genannt, von der gepackten Anordnung. Der Zugriff auf die
%  Farbinformationen ist dabei abhängig vom eingesetzten Farbraum. Im weiteren Verlauf wird die gepackte Anordnung für den
%  RGB Farbraum erläutert. Für YCbCr sind die vorgestellten Modelle anzupassen.
% 
% \subsubsection{Planare Anordnung} % (fold)
% \label{sub:planare_anordnung}
% Bei der planaren Anordnung werden die Farbkomponenten in jeweils eigenen Arrays mit gleicher Größe gespeichert. Ein Bild
% \begin{equation}
% 	I = \left(I_R, I_G, I_B\right)
% 	\label{eq:planarImage}
% \end{equation}
% besteht aus den drei Luminanzbildern $I_R$, $I_G$ und $I_B$ (\autoref{fig:planareAnordnung}).
% \begin{figure}[!ht]
% 	\centering
% 	\input{resources/planareAnordnung.pdf_tex}
% 	\caption{Planare Anordnung. Information der Pixel sind in den Matritzen $I_R$, $I_G$ und $I_B$ hinterlegt.}
% 	\label{fig:planareAnordnung}
% \end{figure}
% Der Zugriff auf ein \gls{pixel} erfolgt über das Auslesen aller drei Arrays, wie in
%  \citeauthor{burger05}\footcite[Vgl.][S.~235--236]{burger05} beschrieben. Der Zugriff erfolgt durch
%  \autoref{eq:readPlanarImage}.
% \begin{equation}
% 	\begin{pmatrix}
% 		R\\
% 		G\\
% 		B
% 	\end{pmatrix}
% 	\leftarrow
% 	\begin{pmatrix}
% 		I_R\left(u,v\right)\\
% 		I_G\left(u,v\right)\\
% 		I_B\left(u,v\right)
% 	\end{pmatrix}
% 	\label{eq:readPlanarImage}
% \end{equation}
% % subsection planare_anordnung (end)
% 
% \subsubsection{Gepackte Anordnung} % (fold)
% \label{sub:gepackte_anordnung}
% Bei der gepackten Anordnung sind alle Farbkomponenten in einem \gls{pixel} gespeichert und in einem Array hinterlegt
%  (\autoref{fig:gepackteAnordnung}).
% \begin{figure}[!ht]
% 	\centering
% 	\input{resources/gepackteAnordnung.pdf_tex}
% 	\caption{Die gepackte Anordnung speichert die Farbinformationen eines Pixels an einer Stelle.}
% 	\label{fig:gepackteAnordnung}
% \end{figure}
% Die gepackte Anordnung ist mit
% \begin{equation}
% 	I\left(u,v\right) = \left(R,G,B\right)
% 	\label{eq:packedImage}
% \end{equation}
% definiert. Im Allgemeinen wird ein Element an Stelle $(u,v)$ eines Bildes $I$ durch \autoref{eq:readPackedRGB}
%  zugegriffen.
% \begin{equation}
% 	\begin{pmatrix}
% 		R\\
% 		G\\
% 		B
% 	\end{pmatrix}
% 	\leftarrow
% 	\begin{pmatrix}
% 		Red\bigl(I\left(u,v\right)\bigr)\\
% 		Green\bigl(I\left(u,v\right)\bigr)\\
% 		Blue\bigl(I\left(u,v\right)\bigr)
% 	\end{pmatrix}
% 	\label{eq:readPackedRGB}
% \end{equation}
% $Red()$, $Green()$ und $Blue()$ sind abhängig vom eingesetzten Format\footcite[Vgl.][S.~236--237]{burger05}.
% % subsection gepackte_anordnung (end)
% 
% % subsection pixelanordnung (end)

\subsection{Filter und Faltung} % (fold)
\label{sub:filter}
Ein Filter ist eine Operation, mit dessen Hilfe ein Eingangsbild $I$ durch eine mathematische Abbildung in ein
 Ausgangsbild $I'$ überführt wird. Im Gegensatz zu Punktoperationen, die nur einen \gls{pixel} betreffen, operieren
 Filter auf Regionen von \gls{pixel}, um zum Beispiel Bilder zu verwischen oder zu schärfen
 (Vgl.~\autoref{fig:filterbeispiel}).
\begin{figure}[!ht]
	\centering
	\subfigure[]{
		\label{fig:filter-original}
		\includegraphics[scale=1]{resources/Filter-Original.pdf}
	}
	\subfigure[]{
		\label{fig:filter-verwischen}
		\includegraphics[scale=1]{resources/Filter-Verwischen.pdf}
	}
	\subfigure[]{
		\label{fig:filter-schaerfen}
		\includegraphics[scale=1]{resources/Filter-Schaerfen.pdf}
	}
	\caption{Das orignial Bild \subref{fig:filter-original} wurde in
	 \subref{fig:filter-verwischen} verwischt und in \subref{fig:filter-schaerfen} geschärft.}
	\label{fig:filterbeispiel}
\end{figure}

Aus einer Region $R_{u,v}$ eines Eingangsbild $I$ wird der neue Pixel $I'(u,v)$ berechnet. Die Größe der
 Filterregion bestimmt die Anzahl der gls{pixel} aus $I$, die zur Berechnung des neuen Pixels $I'(u,v)$ verwendet
 werden (Vgl.~\autoref{fig:region-filter}).
\begin{figure}[!ht]
	\centering
	\input{resources/Region-Filter.pdf_tex}
	\caption{Anwendung eines Filters. Aus der Region $R_{u,v}$ wird der neue Wert $I'(u,v)$ berechnet.}
	\label{fig:region-filter}
\end{figure}
Üblicherweise werden $3 \times 3$ oder $5 \times 5$ Filter verwendet, aber auch größere Filter mit $21 \times 21$
 Pixeln sind möglich\footcite[Vgl.][S.~90--91]{burger05}. Anhand der Schreibweise der Filter erkennt man, das es sich
 bei den Filtern um Matrizen handelt.

Bei einer Filtermatrix wird ein eigenes Koordinatensystem verwendet, dessen Ursprung in der Mitte der Matrix liegt.
 Aufgrund dieses Koordinatensystem sind die Koordinaten sowohl positiv als auch negativ. Bei einer $3 \times 3$
 Filtermatrix $H$ sähen die absoluten Koordinaten an der Stelle $(i,j)$ wie folgt aus:
\begin{equation}
	H(i,j) =
	\begin{pmatrix}
		\left(i-1, j-1\right)&	\left(i, j-1\right)&	\left(i+1, j-1\right)\\
		\left(i-1, j\right)& 	\left(i, j\right)&		\left(i+1, j\right)\\
		\left(i-1, j+1\right)&	\left(i, j+1\right)&	\left(i+1, j+1\right)
	\end{pmatrix}
\end{equation}
Somit können alle Pixel von $I'$ durch
\begin{equation}
	I'\left(u,v\right) \gets
	\sum \limits_{\left(i = -1\right)}^{i = 1}
	\sum \limits_{\left(j = -1\right)}^{j = 1}
	I\left(u + i, v + j\right) \cdot H\left(i,j\right)
\end{equation}
berechnet werden. Für Filter mit einer anderen Größe als $3 \times 3$, lautet die Formel
\begin{equation}
	I'\left(u,v\right) \gets
	\sum_{\left(i,j\right)\in\mathbb{R}} I\left(u + i, v + j\right) \cdot H\left(i,j\right)
\end{equation}
wobei $\mathbb{R}$ die Größe des Filters angibt\footcite[Vgl.][S.~92--93]{burger05}.

Filter können an den Rändern nicht mathematisch korrekt berechnet werden\footcite[Vgl.][S.~113]{burger05}, da der
 Filter $H$ an den Bildrändern über die definierte Bildmenge hinausragt. Um dieses Problem zu umgehen, gibt es die
 Möglichkeit
\begin{enumerate}
	\item am Randbereich einen konstanten Wert zuzuweisen,\label{konstant}
	\item den ursprünglichen Bildwert beizubehalten,\label{nix}
	\item die Randwerte zu berechnen, indem
	\begin{enumerate}
		\item außerhalb des Bildbereichs konstante Werte angenommen werden,\label{berechneKonstant}
		\item die Randpixel fortgesetzt werden,\label{berechneRand}
		\item die Bildwerte wiederholt werden.\label{berechneBild}
	\end{enumerate}
\end{enumerate}

Bei \autoref{konstant} wird der sichtbare Bildbereich verkleinert und bei \autoref{nix} der Filter an den Rändern nicht
 angewendet. Bei \autoref{berechneKonstant} kann die Zuweisung eines konstanten Wertes, beispielsweise Schwarz, das
 Ergebnis verfälschen. \autoref{berechneRand} weist die geringste Verfälschung auf. Bei \autoref{berechneBild} wird
 durch die Wiederholung des Bildes eine periodische Funktion betrachtet. Die möglichen Randbehandlungen sind in
 \autoref{fig:randbeispiel} illustriert.
\begin{figure}[!ht]
	\centering
	\subfigure[]{
		\label{fig:randbeispiel-original}
		\includegraphics[scale=1]{resources/Rand-Original.pdf}
	}
	\subfigure[]{
		\label{fig:randbeispiel-1}
		\includegraphics[scale=1]{resources/Rand-1.pdf}
	}
	\subfigure[]{
		\label{fig:randbeispiel-2}
		\includegraphics[scale=1]{resources/Rand-2.pdf}
	}
	\subfigure[]{
		\label{fig:randbeispiel-3a}
		\includegraphics[scale=1]{resources/Rand-3a.pdf}
	}
	\subfigure[]{
		\label{fig:randbeispiel-3b}
		\includegraphics[scale=1]{resources/Rand-3b.pdf}
	}
	\subfigure[]{
		\label{fig:randbeispiel-3c}
		\includegraphics[scale=1]{resources/Rand-3c.pdf}
	}
	\caption{Filter und Randbehandlung. Das Originalbild \subref{fig:randbeispiel-original} wird je nach eingesetzen
	 Verfahren unterschiedlich behandelt. Bei \subref{fig:randbeispiel-1} werden die Ränder nicht betrachtet und ein
	 konstanter Wert zugewiesen (hier schwarz dargestellt).	Die Originalwerte des Eingangsbilds werden bei
	 \subref{fig:randbeispiel-2} beibehalten. In \subref{fig:randbeispiel-3a} werden außerhalb des Bildes konstante
	 Werte verwendet (hier schwarz dargestellt). Die Randpixel werden bei \subref{fig:randbeispiel-3b} fortgesetzt,
	 während bei \subref{fig:randbeispiel-3c} das Eingangsbild an den Rändern wiederholt wird.}
	\label{fig:randbeispiel}
\end{figure}
Welche dieser Randbetrachtung eingesetzt wird ist abhängig von den benötigten Ergebnissen und Filtern und muss für jede
 Anwendung gesondert betrachtet werden.

In der digitalen Bildverarbeitung basieren Filter auf der mathematischen Operation der Faltung
\footcite[Vgl.][S.~101--104]{burger05}, bei der für zwei Funktionen ein dritte Funktion erzeugt wird. Die diskrete
 lineare Faltung ist definiert als
\begin{equation}
	I'\left(u,v\right) =
	\sum \limits_{i = -\infty}^{\infty}
	\sum \limits_{j = -\infty}^{\infty}
	I\left(u - i, v - j\right) \cdot H\left(i,j\right)
\end{equation}
und gekürzt
\begin{equation}
	I' = I * H,
\end{equation}
wobei $I$ dem Bildsignal und $H$ dem Faltungskern entspricht. Die Operation der Faltung ist die Grundlage aller
 linearen Filter in der digitalen Bildverarbeitung. Unterschiedliche Filter sind nur durch den Filterkern (Kernel)
 definiert. Die Faltungsoperation besitzt die algebraischen Eigenschaften der Kommutativität, Assoziativität und
 Distributivität.
\begin{align}
	&\text{Kommutativität: } &I * H = H * I\\
	&\text{Assoziativität: } &A * (B * C) = (A * B) * C\\
	&\text{Distributivität: } &H * \left(I_1 + I_2\right) = \left(H * I_1\right) + \left(H * I_2\right)
\end{align}
Durch die Assoziativität kann ein Kernel $H$ als Produkt mehrerer kleiner Kernel ausgedrückt werden, was wiederum
 Rechenoperationen einspart. Ein $3 \times 3$ Filter kann in eine $x$-Richtung und $y$-Richtung aufgeteilt werden.
\begin{align}
	H_{xy} = & H_x * H_y\\
	\begin{pmatrix}
		1& 1& 1\\
		1& 1& 1\\
		1& 1& 1
	\end{pmatrix} = &
	\begin{pmatrix}
		1& 1& 1
	\end{pmatrix}
	*
	\begin{pmatrix}
		1\\
		1\\
		1
	\end{pmatrix}
\end{align}
Bei der Berechnung von $I * H_{xy}$ entspricht dies $3 \cdot 3 = 9$ Multiplikationen und bei der Berechnung von
 $I * H_x * H_y = 3 + 3 = 6$ Multiplikationen.

% subsection filter (end)

\subsection{Regionen in Binärbilder} % (fold)
\label{sec:regionen_in_binärbilder}

Binärbilder enthalten, wie in \autoref{sub:bildtypen} bereits beschrieben, nur zwei Werte, die wir auch als Vordergrund
 und Hintergrund bezeichnen können. Das Interesse bei einem Binärbild gilt dementsprechend den Informationen im
 Vordergrund. Eine zusammenhängende Struktur von Vordergrundpixeln wird als Bildregion bezeichnet. Um diese
 Bildregionen in einem Binärbild zu erkennen, werden zusammenhängende \gls{pixel} markiert, was auch als
 \textit{region labeling}\footcite[Vgl.][S.~196]{burger05} bekannt ist. Zur Markierung von \gls{pixel} werden folgende
 numerischen Werte verwendet:
\begin{equation*}
	I(u,v) = \begin{cases}
	0 & \textrm{Hintergrund}\\
	1 & \textrm{Vordergrund}\\
	2,3,\ldots,n & \textrm{Regionenmarkierung}
	\end{cases}
\end{equation*}

Bei der sequentiellen Regionenmarkierung\footcite[Vgl.][S.~200--206]{burger05} wird ein Binärbild in zwei Schritten
 untersucht. Im ersten Schritt werden im Binärbild vorläufige Markierungen für Regionen gespeichert und im zweiten
 Schritt werden mehrfache Markierungen, sogenannte Kollisionen, für eine Region aufgelöst.

\subsubsection{Vorläufige Markierung} % (fold)
\label{sec:vorläufige_makierung}

Zuerst muss das Binärbild zeilenweise von oben nach unten auf das Vorhandensein von Vordergrundpixel untersucht werden.
Die Nachbarn eines angrenzenden \glspl{pixel} werden, je nach Definition der Nachbarschaftsbeziehung, ebenfalls
 untersucht. Zwei gebräuchliche Definitionen von Nachbarschaftsbeziehungen sind zum einen die 4er-Nachbarschaft und zum
 anderen die 8er-Nachbarschaft. Für ein \gls{pixel} $I(u,v)$ werden bei der 4er"=Nachbarschaft die beiden angrenzenden
 \gls{pixel} $I(u-1,v)$ und $I(u,v+1)$ untersucht. Bei der 8er"=Nachbarschaft werden die vier Nachbarn $I(u-1,v)$,
 $I(u-1,v+1)$, $I(u,v+1)$ und $I(u+1,v+1)$ untersucht. Beide Nachbarschaftsbeziehungen sind in
 \autoref{fig:nachbarschaft} abgebildet.
\begin{figure}[!ht]
	\centering
	\subfigure[]{
		\label{fig:4er-nachbarschaft}
		\input{resources/4er-Nachbarschaft.pdf_tex}
	}
	\subfigure[]{
		\label{fig:8er-nachbarschaft}
		\input{resources/8er-Nachbarschaft.pdf_tex}
	}
	\caption{Nachbarschaftsbeziehung. \subref{fig:4er-nachbarschaft} 4er-Nachbarschaft mit $N_1 = I(u-1,v)$ und
	 $N_2 = I(u,v+1)$. \subref{fig:8er-nachbarschaft} 8er-Nachbarschaft mit $N_1 = I(u-1,v)$, $N_2 = I(u-1,v+1)$,
	 $N_3 = (u,v+1)$ und $N_4 = I(u+1,v+1)$.}
	\label{fig:nachbarschaft}
\end{figure}
Die Zuweisung einer Markierung für einen \gls{pixel} ist davon abhängig ob
\begin{enumerate}
	\item alle Nachbarn Hintergrundpixel sind, \label{labeling-all-background}
	\item genau ein Nachbar eine Markierung hat oder \label{labeling-one-neighbour}
	\item mehrere Nachbarn eine Markierung haben. \label{labeling-many-neighbours}
\end{enumerate}

Wenn alle Nachbarn Hintergrundpixel sind (\autoref{labeling-all-background}), wird eine Markierung an Position $(u,v)$
 geschrieben. Bei \autoref{labeling-one-neighbour} wird die Markierung des Nachbarn übernommen. Im letzten Fall müssen
 die Markierungen miteinander verglichen werden. Wenn alle Nachbarn die gleiche Markierung besitzen, wird sie für
 Position $(u,v)$ übernommen. Falls es sich aber um verschiedene Markierungen handelt, spricht man von einer Kollision
 der Markierungen. Eine Kollision bedeutet, dass eine zusammenhängende Region durch zwei unterschiedliche
 Markierungen dargestellt wird (Vgl.~\autoref{fig:kollision}).
\begin{figure}[!ht]
	\centering
	\input{resources/Kollision.pdf_tex}
	\caption{Beispiel einer Kollision zwischen Markierung $2$ und Markierung $3$.}
	\label{fig:kollision}
\end{figure}
Dem \gls{pixel} an Position $(u,v)$ wird eine Markierung eines Nachbarn zugewiesen und die Kollision wird vermerkt. Nach
 \citeauthor{burger05} werden diese Kollisionen in einer dynamischen Datenstruktur gespeichert und zu einem späteren
 Zeitpunkt bearbeitet\footcite[Vgl.][S.~203--204]{burger05}.

Nach diesem ersten Schritt ist das Binärbild verarbeitet und allen Vordergrundpixeln wurde eine vorläufige Markierung
 zugeteilt. Dabei wurden auftretende Kollisionen der Markierungen gespeichert. Der Ablauf des ersten Schritts ist in
 \autoref{fig:markierung} dargestellt.
\begin{figure}[!ht]
	\centering
	\subfigure[]{
		\label{fig:markierung-binaer}
		\input{resources/Binaerbild.pdf_tex}
	}
	\subfigure[]{
		\label{fig:markierung-1}
		\input{resources/Regionenmarkierung-1.pdf_tex}
	}
	\subfigure[]{
		\label{fig:markierung-2}
		\input{resources/Regionenmarkierung-2.pdf_tex}
	}
	\subfigure[]{
		\label{fig:markierung-3}
		\input{resources/Regionenmarkierung-3.pdf_tex}
	}
	\caption{Vorläufige Regionenmarkierung. Ein Binärbild \subref{fig:markierung-binaer} wird zeilenweise von
	 oben nach unten durchlaufen. Eine 8er-Nachbarschaft vergleicht die angrenzenden Pixel \subref{fig:markierung-1}.
	 Eine Kollision zwischen zwei Markierungen wird registriert und der kleinere Wert zugewiesen bis das Binärbild
	 vollständig verarbeitet wurde \subref{fig:markierung-2}--\subref{fig:markierung-3}.}
	\label{fig:markierung}
\end{figure}

% subsubsectionsection vorläufige_makierung (end)

\subsubsection{Auflösung von Kollisionen} % (fold)
\label{sec:auflösung_von_kollisionen}

In diesem Schritt müssen nun die gespeicherten Kollisionen der Markierungen aufgelöst werden, damit eine
 zusammenhängende Bildregion nur durch eine Markierung repräsentiert wird. \citeauthor{burger05} beschreiben diese
 Aufgabe  als nicht trivial, da kollidierende Regionen auch durch weitere Regionen zusammenhängen
 können\footcite[Vgl.][S.~205]{burger05} (Vgl.~\autoref{fig:markierung-3}).

Die Menge der im letzten Schritt verwendeten vorläufigen Markierungen wird verwendet, um die Menge der Kollisionen
 zu vereinen. Dazu wird eine Kollision der Markierung von $a$ und $b$ zu einer Menge zusammengeführt. Das bedeutet,
 dass alle \gls{pixel} mit Markierung $b$ zu $a$ gehören. Danach wird das vorläufig markierte Binärbild erneut
 durchlaufen. Jede vorläufige Markierung wird nun mit der Menge der neuen Markierung verglichen und eine eindeutige
 Markierung zugewiesen.

Betrachten wir \autoref{alg:resolve-label-collision} an folgendem Beispiel:
\begin{algorithm}[!ht]\small
\caption{Kollisionen auflösen}
\label{alg:resolve-label-collision}
\begin{algorithmic}[1]
	\Require $I_b,L,C$
	\State $L = \{2,3,\ldots,n\}$ sind die vorläufigen Markierungen
	\State $R \gets [\{2\},\{3\},\ldots,\{n\}]$, sodass $R_i = {i}$ für alle $\{i\} \in L$
	\For{alle Kollisionen $(a,b)$ in $C$}
	\label{resolve-label-collsion-partition-start}
		\State $R_a \gets$ die Menge mit Markierung $a$
		\State $R_b \gets$ die Menge mit Markierung $b$
		\If{$R_a \neq R_b$}
		\label{resolve-label-collsion-set-start}
			\State $R_a \gets R_a \cup R_b$
			\State $R_b \gets \{\}$
		\EndIf
		\label{resolve-label-collsion-set-end}
	\EndFor
	\label{resolve-label-collsion-partition-end}
	\For{alle Pixel $(u,v)$ in $I_b$}
	\label{resolve-label-collsion-relabel-start}
		\If{$I_b(u,v) > 1$}
			\State Suche in $R$ die Menge $R_i$ die Markierung $I_b(u,v)$ enthält
			\State Ersetze Markierung in $I_b(u,v)$ mit $\textrm{min}(R_i)$
			\label{resolve-label-collsion-set-new-label}
		\EndIf
	\EndFor
	\label{resolve-label-collsion-relabel-end}
\end{algorithmic}
\end{algorithm}

In Zeile \ref{resolve-label-collsion-partition-start}--\ref{resolve-label-collsion-partition-end} wird die Menge $L$
 der vorläufigen Markierungen aufgeteilt und Kollisionen aufgelöst. Nehmen wir an, dass wir eine Kollision
 $C_i = (2,3)$ auflösen wollen, wird die Mengen $R_2 = \{2\}$ und $R_3 = \{3\}$  zu $R_2 = \{2,3\}$ vereint
 (Zeile \ref{resolve-label-collsion-set-start}--\ref{resolve-label-collsion-set-end}). Die Menge $R_3$ ist danach leer.
 Nun werden in Zeile \ref{resolve-label-collsion-relabel-start}--\ref{resolve-label-collsion-relabel-end} alle
 Markierungen untersucht. Wird dabei eine Markierung an Position $(u,v)$ gefunden, die zur Menge $R_2 = \{2,3\}$ gehört,
 wird der kleinste Wert von $R_2$ an Position $(u,v)$ geschrieben (Zeile \ref{resolve-label-collsion-set-new-label}).

Nach diesem letzten Schritt sind alle zusammenhängende Regionen in einem Binärbild eindeutig gekennzeichnet und können
 nun mit anderen Verfahren weiterverarbeitet werden, um beispielsweise die Form einer Region zu erkennen.

% subsubsectionsection auflösung_von_kollisionen (end)

% subsection regionen_in_binärbilder (end)

% section bildverarbeitung (end)

\section{Untersuchungsgegenstand: Tracking Verfahren und Tracking Algorithmen} % (fold)
\label{sec:untersuchungsgegenstand}
\begin{comment}
	Untersuchungsgegenstand: Verfahren und Algorithmen präzise vorstellen und ihre Unterschiede hervorheben.
	Notwendige Kriterien der Algorithmen bestimmen

	Grober Ablauf der Verfahren:
	* Wer hats erfunden?
	* Wie ist das Verfahren aufgebaut (Algo in grob)
	* Welche Kriterien müssen erfüllt sein (monochrom, rgb eingabe)?
\end{comment}

\subsection{Verfahren nach Hirzer} % (fold)
\label{sub:verfahren_nach_hirzer}

Der Algorithmus von \citeauthor{hirzer08}\footcite{hirzer08} ist nach dem Vorbild der \textit{pixel connectivity edge
 linking based algorithms} entworfen und ist in drei Hauptteile aufgebaut. Zuerst werden Liniensegmente erstellt,
 indem \glspl{edgel} auf einem Suchraster gefunden und zusammengeführt werden. Die kurzen Liniensegmente werden dann zu
 längeren Linien zusammengeführt. Anschließend werden im zweiten Schritt alle gefundenen Linien erweitert um die
 Gesamtlänge einer Linie zu erhalten. Im letzten Schritt werden die Linien zu Vierecken verbunden. Im weiteren Verlauf
 werden diese Schritte als Line Detection, Line Extension und Quadrangle Detection bezeichnet.

\subsubsection{Line Detection} % (fold)
\label{sub:line_detection}
Die Linienerkennung basiert auf dem Verfahren von \citeauthor{clarke96}\footcite{clarke96} und besteht aus zwei
 Schritten. Im ersten Schritt wird das Bildsignal grob abgetastet um im zweiten Schritt durch das RANSAC Verfahren eine
 Linienhypothese zu erstellen und zu bewerten.

Im ersten Schritt wird zuerst das monochrome Eingabesignal $I_m$ in $40 \times 40$ \gls{pixel} große Regionen
 unterteilt. Jede nachfolgende Operation erfolgt innerhalb einer Region. Eine Region wird wiederum unterteilt in
 horizontale und vertikale Scanlines, die jeweils $5$ \gls{pixel} Abstand zueinander haben. Jedes \gls{pixel} auf den
 Scanlines wird mit einem Gauß-Kernel gefaltet um die Komponente des Gradienten zu bestimmen. Ein lokales Maximum des
 Gradienten, dass größer als ein festgelegter Schwellwert ist, wird als \gls{edgel} betrachtet und seine Orientierung
 berechnet.

Im zweiten Schritt wird das RANSAC Verfahren verwendet, um aus der Menge der \glspl{edgel} Liniensegmente zu bestimmen.
 Eine Linienhypothese wird durch die zufällige Auswahl zweier \glspl{edgel} erstellt, deren Orientierung innerhalb
 eines Grenzwert liegen müssen. Ein \gls{edgel} dient als Startpunkt und das andere \gls{edgel} als Endpunkt der Linie.
 Im Anschluss wird die Anzahl der \glspl{edgel} betrachtet, die in der Nähe dieser Linie liegen und eine kompatible
 Orientierung mit der Linie aufweisen. Diese \glspl{edgel} unterstützen die Hypothese einer Linie im Bildsignal, wenn
 die Anzahl größer ist als die minimal geforderte Anzahl von Unterstützungsedgels. Die zufällige Auswahl zweier
 \glspl{edgel} um eine Linie zu erstellen und deren Edgelunterstützung zu ermitteln wird mehrmals wiederholt um die
 Linie mit der meisten Edgelunterstützung zu finden. Wenn eine solche dominante Linie gefunden wurde, gilt die
 Hypothese als bestätigt und die Linie wird als vorhanden betrachtet. Die Edgels die zur Unterstüztung der Hypothese
 der Linie galten, werden aus der Menge der Edgels entfernt und das Verfahren wird solange wiederholt, bis alle
 Liniensegmente gefunden wurden oder nicht mehr genügend Edgels vorhanden sind.

Das Verfahren ist in \autoref{alg:linedetection-clarke} dargestellt.

\begin{algorithm}
	\caption{Line Detection nach \citeauthor{clarke96}}
	\label{src:lineDetection}
	\begin{algorithmic}[1]
	\Procedure{LineDetection}{$I$}
		\State $I$ in $40 \times 40$ Pixel große Regionen unterteilen
		\For{$i \gets 0$, alle Regionen}
			\State $Liste \gets \infty$
			\State $i$ in horizontale und vertikale Scanlines unterteilen mit $5$ Pixel Abstand
			\For{$j \gets 0$, alle Pixel auf allen Scanlines}
				\State $x \gets$ Falte $I_m\left(j\right)$ mit Gauß-Kernel
				\If{$x > Schwellwert$}
					\Comment Edgel gefunden
					\State Orientierung des Edgels berechnen
					\State $Liste \gets edgel$
				\EndIf
			\EndFor
			\While{$Liste \not=0 \lor Liste <$ min. Support-Edgels}
				\State $erkannteLinie \gets 0$
				\For{$j \gets 1, 25$}
					\State $Linie \gets$ zwei zufällige Edgels mit kompatibler Orientierung aus $Liste$ wählen
					\State Anzahl von Support-Edgels in der Nähe der Linie bestimmen
					\If{$Linie > erkannteLinie$}
						\State $erkannteLinie \gets Linie$
					\EndIf
				\EndFor
				\If{Anzahl von Support-Edgels > min. Support-Edgels}
					\Comment Linie wurde erkannt
					\State $Liste \gets Liste -$ Support-Edgels der Linie
				\EndIf
			\EndWhile
		\EndFor
	\EndProcedure
	\end{algorithmic}
\end{algorithm}



\citeauthor{hirzer08} hat das Verfahren von \citeauthor{clarke96} abgewandelt, um es zur Erkennung einer bitonalen
 Marke zu nutzen. Dazu verwendet \citeauthor{hirzer08}, anstatt eines monochromen Bildsignals, ein farbiges Bildsignal
 und untersucht zuerst einen der drei Farbkanäle. Wenn ein \gls{edgel} in einem Kanal gefunden wird, werden die
 verbleibenden Kanäle untersucht, um sicherzustellen, dass auch hier ein \gls{edgel} vorliegt. Ist der Gradient in
 allen drei Kanälen höher als der festgelegte Schwellwert, handelt es sich um einen Übergang  von Schwarz nach Weiß.
 Ist dies nicht der Fall, handelt es sich um einen farbigen Übergang und ist somit zur Erkennung einer Marke
 uninteressant. Da ein monochromes Signal wie in \autoref{sub:bildtypen} beschrieben nur ein Kanal besitzt, kann hier
 diese Unterscheidung nicht getroffen werden, was zu einer größeren Anzahl von \glspl{edgel}
 führt\footcite[Vgl.][S.~6--7]{hirzer08}.

Das vorgestellte Verfahren von \citeauthor{clarke96} liefert als Ergebnis nur kurze Liniensegmente. Um eine Kante
 entlang einer Marke zu erkennen, müssen die kurzen Liniensegmente zusammengeführt werden. Dazu werden alle
 Liniensegmente miteinander verglichen um jede Kombinattionsmöglichkeit zu testen. Ob zwei Liniensegmente zu einer
 Linie zusammengeführt werden können, ist von drei Kriterien abhängig. Zuerst müssen zwei Liniensegmente eine
 kompatible Orientierung aufweisen, die nur geringfügig abweichen darf um als Ergebnis eine gerade Linien zu erhalten.
 Als zweites Kriterium muss eine Verbindungslinie zwischen den Liniensegmenten ebenfalls eine kompatible Orientierung
 aufweisen. Dadurch wird sichergestellt, dass keine Liniensegmente zusammengeführt werden, die zwar eine kompatible
 Orientierung besitzen aber parallel zueinander liegen. Als letztes Kriterium wird der Gradient der Verbindungslinie
 Punkt für Punkt untersucht. Dieses Kriterium dient dazu nebeneinanderliegende Marken zu unterscheiden. Würde man dies
 Unterscheidung vernachlässigen, würden mehrere Kanten unterschiedlicher Marken zusammengeführt (Vgl. \autoref{fig:}).

\citeauthor{hirzer08} verwendet keinen Schwellwert um einen minimalen oder maximalen Abstand zwischen zwei
 Liniensegmenten festzulegen. \citeauthor{hirzer08} begründet dies Entscheidung damit, dass bei einem zu kleinen
 Schwellwert Liniensegmente, die zuweit auseinander liegen, nicht zusammengeführt werden, obwohl sie zusammengehören.
 Wird der Abstand des Schwellwerts aber zu groß gewählt, werden Liniensegmente zusammengeführt, die nicht
 zusammengehören. Um auf einen Distanzschwellwert verzichten zukönnen, werden die Liniensegmente sortiert, sodaß
 Liniensegmente mit kurzen Verbindungslinien zuerst zusammengeführt werden. Dadurch kann sichergestellt werden, dass
 Liniensegmente zusammengeführt werden die nah beieinander liegen.

Das Zusammenführen der Liniensegmente wird zweimal durchgeführt. Zuerst innerhalb einer Region um alle kurzen
 Liniensegmente zu verbinden. Nachdem innerhalb aller Regionen Liniensegmente zusammengeführt wurden, wird der Vorgang
 auf dem gesammten Bildsignal wiederholt um größere Liniensegmente zuvereinen. Dadurch müssen nicht alle
 Liniensegmente-Kombinationen im gesammten Bildsignal verglichen werden, was die Laufzeit
 reduziert\footcite[Vgl.][S.~10]{hirzer08}.

% subsubsection line_detection (end)

\subsubsection{Line Extension} % (fold)
\label{sub:line_extension}

Nach der Anwendung von Line Detection erhalten wir als Ergebnis nur kurze Liniensegmente. Diese kurzen Segmente
 repräsentieren aber nicht eine tatsächliche vorhandene Kante im Bildsignal. In den meisten Fällen fehlt am Anfang oder
 Ende eines Segments ein Stück. Mit Line Extension wird versucht die fehlenden Stücke am Anfang und am Ende eines
 Segments zu erweitern, um damit eine gerade Linie abzubilden.

Um eine Linie zu erweitern, wird jedes Liniensegment \gls{pixel} für \gls{pixel} untersucht. Dabei wird an einem Ende
 ein \gls{pixel} weitergerückt und die Orientierung dieses \gls{pixel} mit der Orientierung des Liniensegments
 verglichen. Falls die Orientierung kompatibel ist, wird der Endpunkt des Segments durch das hinzugefügte \gls{pixel}
 ersetzt. Das Verfahren wird solange wiederholt, bis die Orientierung nicht mehr übereinstimmt. Das bedeutet, dass sich
 der Gradient zu stark verändert hat und dadurch das Ende einer Kante gefunden wurde. Somit ist eine Seite des Segments
 erweitert und das Verfahren wird für das andere Ende des Segments wiederholt.

Zusätzlich wird überprüft, ob über das Ende der erweiterten Linie hinaus ein heller \gls{pixel} liegt. Wenn dem so ist,
 eignet sich diese Linie zur Erkennung von Eckpunkten und somit zur Erkennung von Marken. Sollte sich über das Ende der
 Linie hinaus kein heller \gls{pixel} befinden, liegt eine Ecke der Marke entweder ausserhalb des Bildes oder ist
 verdeckt und eignet sich somit nicht zu Erkennung einer Marke. Das Verfahren wird auf beiden Enden einer Linie
 angewendet. Falls beide Enden sich nicht zur Erkennung von Eckpunkten eignen, wird die Linie entfernt.
% subsubsection line_extension (end)

\subsubsection{Quadrangle Detection} % (fold)
\label{sub:quadrangle_detection}
% subsubsection quadrangle_detection (end)

\begin{comment}
	Der RANSAC Grouper wird verwendet um gerade Liniensegmente zu finden. Dazu werden zwei zufällige edgels ausgewählt
	 und ihre kompatible Orientierung überprüft. Die Anzahl der supporting edgels wird durch die Distanz des
	 Liniensegments und der Orientierung bestimmt. Durch wiederholung dieses Prozesses wird die dominante Linie in der
	 Region bestimmt. Die supporting edgels werden entfernt und der Prozess wird wiederholt bis keine edgels mehr
	 vorhanden sind oder eine maximale Anzahl von wiederholungen erreicht wurde. Durch dieses Wiederholung wird
	 sichergestellt, dass alle dominanten Linien in einer Region erkannt werden.

	Vorteil: Der Algorithmus ist sehr schnell und lässt sich für gewünschte Liniensegmente anpassen.
	Nachteil: Durch sein antisotropic detection verhalten diskriminiert das verfahren diagonale liniensegmente. Dies
	 ist durch ein rechteckiges samplingrid bedingt.

	Hirzer hat das Verfahren um zwei Punkte erweitert und angepasst.
	Wird in einem RGB Bild ein Kanal untersucht und ein Edgel gefunden, werden in den restlichen zwei Kanälen an der
	 gleichen Position nach einem Edgel gesucht. Falls in allen drei Kanälen ein Edgel gefunden wird, handelt es sich
	 um ein Linie (Schwarz/Weiß) und keine Farblinie.

	Nur das erste Frame wird vollständig untersucht und die Position der gefundenen Marken notiert. In den folgenden
	 frames wird nur in den Regionen der gefundenen Marke das Verfahren benutzt. Erst nach einer festgelegten Anzahl
	 von frames wird wieder ein vollständiger Durchlauf des Verfahrens durchgeführt.
\end{comment}

% subsection verfahren_nach_hirzer (end)

% section section_name (end)
\section{Vorgehen} % (fold)
\label{sec:vorgehen}
\begin{comment}
	Vorgehen: Analysemethoden vorstellen wie Algorithmen untersucht werden.
	Vergleich O-Notation
	Laufzeitanalyse
	Gleiche Kriterien (selbes Bild, selbes Video)
\end{comment}

Um das Verfahren ARToolKitPlus und das Verfahren von \citeauthor{hirzer08} zu analysieren werden die Kosten der
 Algorithmen berechent und ihre Wachstumsrate ermittlet. Elementare Operationen, wie Addition, Zuweisung und
 Arrayzugriffe, werden in Abhängigkeit von der Größe der Eingabemenge berechnet. Daraus wird wiederum das aymptotische
 Wachstum ermittelt.

Bei Verfahren, deren Algorithmen nicht im Quellcode vorliegen, wird durch eine experimentelle Analyse die Laufzeit des
 Verfahrens für unterschiedliche Eingabemengen untersucht. Die Durchführung der experimentellen Analyse erfolgt dabei
 auf einem iPod touch (4. Generation) mit einem ARMv7 Befehlssatz. Alle Programme der Analyse wurden mit LLVM GCC 4.2
 und Kompileroptimierung $\mathit{-Os}$\footcite[Vgl.][]{cc} erstellt.

\paragraph{Bilddatenbeschaffung:} % (fold)
\label{par:bilddatenbeschaffung}
Das Videosignal wird unter iOS durch das AVFoundation Framework zur Verfügung gestellt und ermöglicht durch Delegation
 den Zugriff auf einzelne Frames\footcite{avfoundation}. AVFoundation erlaubt die Einstellung von fünf verschiedenen
 Auflösungen für Videosignale, von Low, Medium, High über $640 \times 480$ bis zu $1280 \times 720$. Ferner unterstützt
 AVFoundation die Bildformate $YCbCr$ mit $8$-Bit und $BGRA$ mit $32$-Bit.
% paragraph bilddatenbeschaffung (end)

\paragraph{Vergleichsanalyse:} % (fold)
\label{par:vergleichsanalyse}
Um beide Verfahren miteinander zu vergleichen wird die Laufzeit der Programme gemessen. Dazu wird durch hinzufügen von
 Anwendungsschritten, von der Bildbeschaffung hin zur Erkennung einer Marke, eine Laufzeit erstellt. Dieses Verfahren
 wurd von \citeauthor{wagner09b}\footcite[Vgl.][]{wagner09b} verwendet um Studierstube 4 auf unterschiedlichen Geräten
 zu beurteilen. In meiner Analyse wird mit der gleichen Technik ARToolKitPlus und das Verfahren von
 \citeauthor{hirzer08} auf einem iPod touch untersucht.
% paragraph vergleichsanalyse (end)

\paragraph{ARToolKitPlus:} % (fold)
\label{par:artoolkitplus}
Die Untersuchung von ARToolKitPlus unter iOS 4 erfolgt mit VRToolKit\footcite[Vgl.][]{vrtoolkit} von \citeauthor{vrtoolkit}. VRToolKit ist ein Obj-C Wrapper für ARToolKitPlus. VRToolKit konfigueriert das ARToolKitPlus System zur Verarbeitung von Marken und sorgt für die Bereitstellung von Bildsignalen der Kamera.
% paragraph artoolkitplus (end)

\paragraph{Verfahren nach \citeauthor{hirzer08}:} % (fold)
\label{par:verfahren_nach_hirzer}
Das Verfahren von \citeauthor{hirzer08} ist eine Eigenentwicklung in Obj-C und C, die nach dem Konzept von VRToolKit und
 dem Softwarehersteller infi\footcite[Vgl.][]{infi} entworfen wurde. Die Implementierung sorgt für die Bereitstellung
 und Verarbeitung des Bildsignals.
% paragraph verfahren_nach_hirzer (end)

% section vorgehen (end)
% \section{Material} % (fold)
\label{sec:material}
\begin{comment}
	Material:
	Vorstellung der Geräte, Frameworks, Libs, Opensource Code, etc. inkl. Referenzangabe wie ein Glossar.
\end{comment}

\begin{comment}
	\gls{AR}-Anwendungen für Smartphones werden durch einen geeigneten Softwareentwurf speziell für Smartphones ermöglicht\footcite[Vgl.][]{wagner09a}.

	Die Software Architektur für AR auf Smartphones unterscheidet sich von einer PC Architektur grundsätzlich durch die
	 Menge und Bandbreite des Arbeitsspeichers. Die Software kann nach gängigen Entwurfskriterien entworfen werden
	 wobei die Vor- und Nachteile von dynamic und static Libraries berücksichtigt werden müssen.

	Bei der Entwicklung ist zu beachten, dass Emulatoren für mobile Geräte ungeeignet sind um Algorithmen zu
	 entwickeln, da ein Emulator nicht die tatsächliche Geschwindigkeit ermöglicht. Dadurch bedingt sollte eine
	 Anwendung besser als native PC Anwendung entwickelt werden und im letzten Schritt auf der Zielplattform getestet
	 werden. Bei der Entwicklung ist darauf zu achten, dass die Software speziell für Smartphones entwickelt wird um
	 hohe Performanz und Robustheit zu erreichen.

	Durch die Verwendung von multithreading oder interleaving kann auch auf Geräten ohne Multicore oder Hyperthreading
	 eine parallele Ausführung von Verarbeitungsschritte ermöglicht werden. Der Bezug von Kameradaten und der
	 Posenberechnung erfolgt dann in zwei Threads.

	Durch die Verwendung von kompakten Pixelformaten wird nicht nur die Speicherbandbreite effektiver ausgenutzt
	 sondern auch die digitale Bildverarbeitung erleichtert\footcite[Vgl.][]{wagner09b}.

	Durch die gerätespezifischen Rendering Engines muss eine Gerätekonfiguration sorgfältig ausgewählt werden. Wird das
	 rendering für eine Library abstrahiert muss es für alle unterstützten Geräte angepasst werden. Dies wird nur zur
	 Vollständigkeit erwähnt und ist kein Bestandteil dieser Arbeit.

	Da die meisten Smartphones im Gegensatz zu PCs keine floating point einheit besitzen, werden die Operation in
	 Software emuliert. Daraus resultiert eine geringe Geschwindigkeit. Um das Problem zu umgehen sollte wo mögliche
	 fixed point oder integer verwendet werden, oder auch lookup tables und interpolation/annährung.

	Statistik-basierte Verfahren die mittels einer Hypotese (hypostesize) ein Liniensegment schätzen.
	Gradienten-basierte Verfahren die durch gradient magnitude und Orientierung linien erkennen.
	Pixel-Conectivity Verfahren erkennen Linien durch die Nähe von Pixeln und ihrer Orientierung.
	Verfahren nach Hough erkennen geometrische Figuren durch ihre parameter darstellung. (Linien durch Polar-Darstellung)\footcite[Vgl.][]{hirzer08}
\end{comment}

Die Entwicklung der Software wurde auf einem MacBook Pro und einem Mac Mini mit OS X 10.6 und Xcode 3.2 durchgeführt.
 Als Zielplattform kam ein iPhone 4 mit iOS 4 zum Einsatz.

Die Softwarebibliothek MKVideoIO abstrahiert Apple's AVFoundation (iPhone) und QTKit (OS X) Bibliotheken um
 Videosignale anzufordern und zu verarbeiten. Die Bibliothek wurde so entworfen, dass ein Einsatz auf dem iPhone und
 einem Mac möglich ist. Der Entwurf wurde nach dem Observer Muster\footcite[Vgl.][S.~287--300]{gamma96} angefertigt.

Für das iPhone gibt es von Apple keine Bibliothek um Aufgaben aus dem Bereich der Bildverarbeitung durchzuführen. Um
 diese Aufgaben durchzuführen, wurde die Softwarebibliothek MKImageProcessing entworfen. Die Aufgaben der Bibliothek
 umfasst das Auslesen und Schreiben einer Bildmatrix, sowie der Konvertierung von Pixeln. Bei der Konvertierung wurde
 besonderer Wert darauf gelegt, dass die Speicherbandbreite des iPhones ausgenutzt wird um Geschwindigkeitseinbußen zu
 vermeiden. Die Bibliothek umfasst einen Testmodus, indem es möglich ist, Ergebnisse einer Bildmanipulation auszugeben.
 Dies erleichterte die Entwicklung erheblich.

Um die Linienerkennung durchzuführen, verwendet MKImageProcessing die vorgestelle Method aus
 Kap~\ref{sub:verfahren_nach_hirzer}. Die Implementierung umfasst neben dem Verfahren auch Klassen um Linien und Edgels
 zu beschreiben. Die Implementierung verzichtet auf einen objektorientierten Entwurf um die Geschwindigkeit zu erhöhen.
 Ferner wurde eine eigene Speicherverwaltung entwickelt, die dafür sorgt, dass während der Ausführung kein neuer
 Speicher vom System angefordert werden muss. Auch dies wurde aus Gründen der Performanz durchgeführt um geeignete
 Ergebnisse zu erzielen.

% section material (end)

% chapter untersuchungsdesign (end)
		% Löwenanteil	85% inkl. Analyse + Ergebnisse + Diskussion
\section{Analyse} % (fold)
\label{sec:analyse}
\begin{comment}
	Detailierte Beschreibung der Algorithmen inkl. O-Notation (Nitty-Gritty Darstellung der Algos)
	1. ARToolKitPlus
	2. Zissermann/Clarke
	Analyse: Die auswertung nach den Kriterein aus Kap. Vorgehen OHNE WERTUNG! Nur die Daten erheben und auswerten.
\end{comment}

\subsection{Hirzer} % (fold)
\label{sub:hirzer}

\citeauthor{clarke96} verwenden in ihrem Verfahren ein monochromes Bildsignal $I_m$\footcite[Vgl.][S.~417]{clarke96}.
 Die Konvertierung des Bildsignals $I$ von YCbCr in $I_m$ erfolgt durch \autoref{alg:convertmonochrome}. Wie in
 \autoref{sub:farbräume} beschrieben, besteht ein YCbCr Signal aus einem Luminaz-Kanal $Y$ und den Chroma Abweichungen
 $Cb$ und $Cr$. Um ein monochromes Signal $I_m$ zu erstellen, muss der Luminanz Kanal ausgelesen und in einen Buffer
 kopiert werden.

\begin{algorithm}[ht]
\caption{Konvertierung zu monochromen Bildsignal}
\label{alg:convertmonochrome}
	\begin{algorithmic}[1]
		\Require $I, I_m$
		\State $Y \gets$ \Call{baseAddress}{$I$}
		\label{alg:convertmonochrome-baseaddress}
		\State $w \gets$ \Call{width}{$I$}
		\State $h \gets$ \Call{height}{$I$}
		\State $l \gets w \cdot h$
		\State $I_m \gets$ \Call{copy}{$I, Y, l$}
	\end{algorithmic}
\end{algorithm}


Der Algorithmus verwendet als Parameter das Bildsignal $I$ und einen Pointer $I_m$ auf einen Buffer für das monochrome
 Signal. Der Monochromebuffer $I_m$ ist ein Array mit fester Größe, das beim initialisieren einmalig angelegt wird und
 danach wiederverwendet werden kann. In Zeile \ref{alg:convertmonochrome-baseaddress} wird die Adresse des
 Luminanz-Kanals $Y$ ausgelesen. Die Funktionen \textproc{width} und \textproc{height} liefern die Breite und Höhe des
 Signals in Pixeln, mit denen die Länge der Daten berechnet wird. Anschließend werden die Daten in den Buffer kopiert.
 Die Laufzeit des Algorithmus entspricht $\Theta(1)$. (Vorsicht: Nur Zeile 5 verwendet keine Funktionen, deren Laufzeit
 dir nicht bekannt sind. baseaddress, width und height greifen evtl. auf metadaten zurück und wären damit ein einfacher
 lookup mit $\Theta(1)$. Dann wäre nur noch memcpy zu bestimmen, was im schlimmsten Fall $\Theta(n)$ wäre.)

Um auf \gls{pixel} zugreifen zu können, verwende ich \autoref{alg:getpixel}. Es wird der Buffer $I_m$ als Pointer
 übergeben und die Position $x$ und $y$ des gewünschten \gls{pixel}. $w$ und $h$ entsprechen der Breite und Höhe von
 $I_m$. Zeile \ref{alg:getpixel-startcheck} bis Zeile \ref{alg:getpixel-stopcheck} sorgen dafür, dass keine Werte
 außerhalb des Buffers gelesen werden können. Dies ist für die Randbehandlung bei Faltungsoperationen
 (Vgl. \autoref{sub:filter}) wichtig und wiederholt den \gls{pixel}.

\begin{algorithm}
\caption{Pixelwert auslesen}
\label{alg:getpixel}
	\begin{algorithmic}[1]
		\Require $I_m, x, y, w, h$
		\Ensure $I_m[i]$
		\If{$x < 0$}
		\label{alg:getpixel-startcheck}
			\State $x \gets 0$
		\EndIf
		\If{$y < 0$}
			\State $y \gets 0$
		\EndIf
		\If{$x \geq w$}
			\State $x \gets w - 1$
		\EndIf
		\If{$y \geq h$}
			\State $y \gets h -1$
		\EndIf
		\label{alg:getpixel-stopcheck}
		\State $i \gets x + \left(y \cdot w\right)$
		\State \textbf{return} $I_m[i]$
	\end{algorithmic}	
\end{algorithm}


Die Laufzeit von \autoref{alg:getpixel} ist im worst-case und im best-case konstant und somit $\Theta(1)$.

Der Algorithmus von \citeauthor{clarke96} ist in \autoref{alg:linedetection-analyze} aufgeführt. Zuerst wird die Breite
 $w$ und Höhe $h$ des Signals $I_m$ festgehalten. Die doppelte For-Schleife in Zeile
 \ref{alg:linedetection-analyze-start} bis \ref{alg:linedetection-analyze-end} unterteilt das Signal in Regionen der
 Größe $40 \times 40$ \gls{pixel}, indem die Koordinate der oberen linken Ecke berechnet wird.

\begin{algorithm}[ht]
\caption{Line Detection}
\label{alg:linedetection-analyze}
	\begin{algorithmic}[1]
		\Require $I_m$
		\State $w \gets$ \Call{width}{$I_m$} \Cost{$c_1$}{$1$}
		\State $h \gets$ \Call{height}{$I_m$} \Cost{$c_2$}{$1$}
		\For{$y \gets 0$ \textbf{to} $y < h$} \Cost{$c_3$}{$\tfrac{h}{40} + 1$}
		\label{alg:linedetection-analyze-start}
			\For{$x \gets 0$ \textbf{to} $x < w$} \Cost{$c_4$}{$\sum_{y=0}^{\tfrac{h}{40}} \left(\tfrac{w}{40} + 1\right)$}
				\State \Call{findedgels}{$I_m,E,x,y,40,40,w,h$} \Cost{$c_5$}{$\sum_{y = 0}^{\frac{h}{40}} \sum_{x = 0}^{\frac{w}{40}} t_y t_x$}
				\State \Call{findlinesegments}{$E,L$} \Cost{$c_6$}{$\sum_{y = 0}^{\frac{h}{40}} \sum_{x = 0}^{\frac{w}{40}} t_y t_x$}
				\State $ x \gets x + 40$ \Cost{$c_7$}{$\sum_{y = 0}^{\frac{h}{40}} \sum_{x = 0}^{\frac{w}{40}} t_y t_x$}
			\EndFor
			\State $y \gets y + 40$ \Cost{$c_9$}{$\sum_{y = 0}^{\frac{h}{40}} t_y$}
		\EndFor
		\label{alg:linedetection-analyze-end}
	\end{algorithmic}
\end{algorithm}


In \citeauthor{clarke96} ist keine Angabe zu den Abmessungen der untersuchten Signale angegeben. Auch der Grund warum
 eine Region $40 \times 40$ \gls{pixel} gross sein muss, fehlt. Zur Analyse der Videosignale verwendeten
 \citeauthor{clarke96} einen Framegrabber 2000 der eine Auflösung von 640x480px schafft. Betrachtet man

$640 mod 40 = 0$ und $480 mod 40 = 0$

ist ersichtlich, dass die Größe der Region in der Aufteilung des Bildsignals in Zusammenhang steht.

Der \autoref{alg:linedetection-analyze} ist der zentrale Algorithmus von \citeauthor{clarke96}. Der Algorithmus
 ist verantwortlich für die Unterteilung des Bildsignals in Regionen von jeweils $40 \times 40$ \gls{pixel} (Vgl. Zeile
~\ref{alg:linedetection-analyze-start}--\ref{alg:linedetection-analyze-end}). Die Kosten des Algorithmus sind in \autoref{eq:linedetection-analyze1} aufgeführt. Um die Gleichung zu vereinfachen führe ich in \autoref{eq:linedetection-analyze2} $n = \tfrac{h}{40}$ und $k = \tfrac{w}{40}$ ein. Sowohl bei worst-case als auch bei best-case werden die Summen immer vollständig durchlaufen. Damit kann die Gleichung zu \autoref{eq:linedetection-analyze3} vereinfacht werden, was eine Laufzeit von $\Theta(nk)$ ergibt.

\begin{subequations}
\label{eq:linedetection-analyze}
\begin{multline}
	T(I) = c_1
	+ c_2
	+ c_3 \left(\frac{h}{40} + 1\right)
	+ c_4 \sum \limits_{y = 0}^{\frac{h}{40}} t_y \left(\frac{w}{40} + 1 \right)
	+ c_5 \sum \limits_{y = 0}^{\frac{h}{40}} \sum \limits_{x = 0}^{\frac{w}{40}} t_y t_x\\
	+ c_6 \sum \limits_{y = 0}^{\frac{h}{40}} \sum \limits_{x = 0}^{\frac{w}{40}} t_y t_x
	+ c_7 \sum \limits_{y = 0}^{\frac{h}{40}} \sum \limits_{x = 0}^{\frac{w}{40}} t_y t_x
	+ c_9 \sum \limits_{y = 0}^{\frac{h}{40}} t_y
	\label{eq:linedetection-analyze1}
\end{multline}
\begin{multline}
	T(I) = c_1
	+ c_2
	+ c_3 \left(n + 1\right)
	+ c_4 \sum \limits_{y = 0}^{n} t_y \left(k + 1 \right)
	+ c_5 \sum \limits_{y = 0}^{n} \sum \limits_{x = 0}^{k} t_y t_x\\
	+ c_6 \sum \limits_{y = 0}^{n} \sum \limits_{x = 0}^{k} t_y t_x
	+ c_7 \sum \limits_{y = 0}^{n} \sum \limits_{x = 0}^{k} t_y t_x
	+ c_9 \sum \limits_{y = 0}^{n} t_y
	\label{eq:linedetection-analyze2}
\end{multline}
\begin{multline}
	T(I) =
	c_1
	+ c_2
	+ c_3 \left(n + 1\right)
	+ c_4 \left[n \left(k + 1 \right)\right]
	+ c_5 n k
	+ c_6 n k
	+ c_7 n k
	+ c_9 n\\
	= c_1 + c_2 + c_3 + \left(c_3 + c_4 + c_9\right) n + \left(c_4 + c_5 + c_6 + c_7\right) n k
	\label{eq:linedetection-analyze3}
\end{multline}
\end{subequations}

Das Verfahren zur Bestimmung der Edgels (\autoref{alg:findedgels-horizontal}) benötigt das monochrome Bildsignal $I_m$,
 sowie die Position der oberen linken Ecke der Region, die durch oben $t$ und links $l$ definiert ist. Die Breite und
 Höhe der Region ist durch $\mathit{rw}$ und $\mathit{rh}$ angegeben. Die Abmessung des Bildsignals werden als $w$ und
 $h$ bezeichnet. Der Pointer $E$ wird zur Speicherung der gefundenen \gls{edgels} verwendet.

\begin{algorithm}
\caption{Edgels bestimmen}
\label{alg:findedgels-horizontal}
	\begin{algorithmic}[1]
		\Require $I_m, E, t, l, \mathit{rw}, \mathit{rh}, w, h$
		\For{$y \gets t$ \textbf{to} $y < t + \mathit{rh}$}
		\label{alg:findedgels-horizontal-scanlinestart}
		\Comment Horizontale Scanlines
			\State $p_1 \gets 0$
			\State $p_2 \gets 0$
			\For{$x \gets l$ \textbf{to} $x < l + \mathit{rw}$}
			\label{alg:findedgels-horizontal-loopstart}
				\State $currentEdgel \gets$ \Call{Convolute}{$I_m,x,y,w,h$}
				\label{alg:findedgels-horizontal-convolute}
				\If{$currentEdgel > threshold$}
				\label{alg:findedgels-horizontal-foundedgel}
					\Comment Edgel gefunden
				\Else
					\State $currentEdgel \gets 0$
				\EndIf
				\If{$p_1 > 0 \land p_1 > p_2 \land p_1 > currentEdgel$}
				\label{alg:findedgels-horizontal-maxima}
					\Comment $p_1$ ist lokales Maximum
					\State $edgel \gets \infty$
					\State $edgel.x \gets x - 1$
					\State $edgel.y \gets y$
					\State $edgel.orientation \gets$ \Call{Orientation}{$x - 1, y$}
					\State $E \gets edgel$
				\EndIf
				\State $p_2 \gets p_1$
				\label{alg:findedgels-horizontal-copy-prev1}
				\State $p_1 \gets currentEdgel$
				\label{alg:findedgels-horizontal-copy-edgel}
			\EndFor
			\label{alg:findedgels-horizontal-loopend}
			\State $y \gets y + 5$
			\label{alg:findedgels-horizontal-increment}
		\EndFor
		\label{alg:findedgels-horizontal-scanlineend}
	\algstore{brkfindedgels}
	\end{algorithmic}
\end{algorithm}
\begin{algorithm}[ht]
	\caption{Edgels bestimmen (Fortsetzung)}
	\label{alg:findedgels-vertical}
	\begin{algorithmic}[1]
	\algrestore{brkfindedgels}
		\For{$x \gets l$ \textbf{to} $x < l + \mathit{rw}$}
		\Comment Vertikale Scanlines
		\label{alg:findedgels-vertical-scanlinestart}
			\State $p_1 \gets 0$
			\State $p_2 \gets 0$
			\For{$y \gets t$ \textbf{to} $y < t + \mathit{rh}$}
				\State $currentEdgel \gets$ \Call{Convolute}{$I_m,x,y,w,h$}
				\If{$currentEdgel > threshold$}
					\Comment Edgel gefunden
				\Else
					\State $currentEdgel \gets 0$
				\EndIf
				\If{$p_1 > 0 \land p_1 > p_2 \land p_1 > currentEdgel$}
					\Comment $p_1$ ist lokales Maximum
					\State $edgel \gets \infty$
					\State $edgel.x \gets x$
					\State $edgel.y \gets y - 1$
					\State $edgel.orientation \gets$ \Call{Orientation}{$x, y - 1$}
					\State $E \gets edgel$
				\EndIf
				\State $p_2 \gets p_1$
				\State $p_1 \gets currentEdgel$
			\EndFor
			\State $x \gets x + 5$
		\EndFor
		\label{alg:findedgels-vertical-scanlineend}
	\end{algorithmic}
\end{algorithm}


Zeile~\ref{alg:findedgels-horizontal-scanlinestart}--\ref{alg:findedgels-horizontal-scanlineend} ist für den Aufbau der
 horizontalen Scanlines verantwortlich. Die Überprüfung sorgt dafür, dass die Scanlines bis zum Ende der Region im
 Abstand von $5$ Pixeln untersucht werden. Nach der Initialisierung der Variablen wird in der Schleife von
 Zeile~\ref{alg:findedgels-horizontal-loopstart}--\ref{alg:findedgels-horizontal-loopend} jeder Pixel auf der Scanline
 untersucht. Zuerst wird in Zeile~\ref{alg:findedgels-horizontal-convolute} die Faltung mit einem Gauß-Kernel
 vorgenommen (Vgl. \autoref{alg:derivativeofgauss-horizontal}, S.~\pageref{alg:derivativeofgauss-horizontal}). Der Test
 in Zeile~\ref{alg:findedgels-horizontal-foundedgel} überprüft anschließend das Ergebnis der Faltung. Wenn der
 Schwellwert nicht überschritten wird, gibt es keinen genügend großen Anstieg des Gradienten und das Ergbnis wird auf
 $0$ gesetzt. Wird der Schwellwert überschritten, handelt es sich um einen Edgel und das Ergbnis wird in den
 Bedingungen von Zeile~\ref{alg:findedgels-horizontal-maxima} weiter untersucht, ob es sich um ein lokales Maximum
 handelt. Ein lokales Maximum bedeutet, dass ein Edgel einen größeren Gradienten besitzt als seine beiden Nachbarn.

Die Bedingung in Zeile~\ref{alg:findedgels-horizontal-maxima} wird bei der ersten Überprüfung immer fehlschlagen.
 Dadurch wird sichergestellt, dass kein Maxium an den Rändern existiert, da hier nicht genügend Nachbarn vorhanden sind
 um eine verlässliche Aussage zu treffen. Zeile~\ref{alg:findedgels-horizontal-copy-prev1} und
 Zeile~\ref{alg:findedgels-horizontal-copy-edgel} kopieren die Werte für den nächsten Durchlauf. Durch das kopieren der
 Werte werden die Nachbarn für den nächsten Durchlauf um eine Position weiterverschoben. Nur bei einem lokalen Maximum
 wird die Position des Edgels gespeichert, und seine Orientierung (Vlg. \autoref{alg:sobel},
 S.~\pageref{alg:sobel}) berechnet. Der Edgel wird in (einer Liste|einem Memorypool) zu weiteren Verarbeitung
 gespeichert.

Sind alle Pixel auf einer Scanline untersucht, wird in Zeile~\ref{alg:findedgels-horizontal-increment} die nächste
 Scanline ausgewählt. Das Verfahren wird solange wiederholt, bis alle Scanlines innerhalb der Region untersucht wurden.

\autoref{alg:findedgels-vertical} untersucht die vertikalen Scanlines in Zeile
 \ref{alg:findedgels-vertical-scanlinestart}--\ref{alg:findedgels-vertical-scanlineend} analog zu
 \autoref{alg:findedgels-horizontal} Zeile
 \ref{alg:findedgels-horizontal-scanlinestart}--\ref{alg:findedgels-horizontal-scanlineend}.

\autoref{alg:derivativeofgauss-horizontal} und \autoref{alg:derivativeofgauss-vertical} berechnen den Gradienten durch Faltung mit dem Gauß-Kernel
$\left( \begin{smallmatrix}
-3& -5& 0& 5& 3
\end{smallmatrix} \right)$
 auf der horizontalen und vertikalen Scanline. Als Parameter benötigt der Algorithmus den Pointer des monochromen
 Bildsignals $I_m$, die Position des Pixels ($x$ und $y$), sowie die Breite $w$ und Höhe $h$ von $I_m$. In Zeile
 \ref{alg:derivativeofgauss-horizontal-readstart}--\ref{alg:derivativeofgauss-horizontal-readend} werden durch die
 Funktion \textproc{getpixel} (Vgl. \autoref{alg:getpixel}, S. \pageref{alg:getpixel}) die benötigten Pixelwerte
 ausgelsen und den Variablen zugewiesen. Im Anschluss werden die Werte mit dem Gauß-Kernel
$\left( \begin{smallmatrix}
-3& -5& 0& 5& 3
\end{smallmatrix} \right)$
berechnet um den Gradienten zu bestimmen.

\begin{algorithm}
	\caption{Faltung mit Gauß-Kernel (horizontale Scanline)}
	\label{src:analyseConvolution}
	\begin{algorithmic}[1]
		\Procedure{convolute}{$I_m,x,y,w,h$}
		\State $pos_1 \gets$ \Call{getPixel}{$I_m, x - 2, y, w, h$}
		\State $pos_2 \gets$ \Call{getPixel}{$I_m, x - 1, y, w, h$}
		\State $pos_4 \gets$ \Call{getPixel}{$I_m, x + 1, y, w, h$}
		\State $pos_5 \gets$ \Call{getPixel}{$I_m, x + 2, y, w, h$}
		\State $value \gets 0$
		\State $value \gets value + \left( -3 \cdot pos_1 \right)$
		\State $value \gets value + \left( -5 \cdot pos_2 \right)$
		\State $value \gets value + \left( 5 \cdot pos_4 \right)$
		\State $value \gets value + \left( 3 \cdot pos_5 \right)$
		\State \textbf{return} $value$
		\EndProcedure
	\end{algorithmic}
\end{algorithm}


Durch die Multiplikation mit $\tfrac{1}{16}$ wird sichergestellt, dass der maximale Wert

\begin{equation}
	\frac{1}{16}
	\cdot
	\begin{pmatrix}
		-3& -5& 0& 5& 3
	\end{pmatrix}
	\cdot
	\begin{pmatrix}
		0& 0& 0& 255& 255
	\end{pmatrix}
	= 127.5
\end{equation}

und der minimale Wert

\begin{equation}
	\frac{1}{16}
	\cdot
	\begin{pmatrix}
		-3& -5& 0& 5& 3
	\end{pmatrix}
	\cdot
	\begin{pmatrix}
		255& 255& 0& 0& 0
	\end{pmatrix}
	= -127.5
\end{equation}

für ein monochromes Bild eingehalten werden.

Bei genauer Betrachtung von \autoref{alg:derivativeofgauss-horizontal} und \autoref{alg:derivativeofgauss-vertical}
 fällt auf, dass der Wert $p_3$ in der Berechnung nicht mit einfliesst. Dies ist darauf zurückzuführen, dass bei der
 Multiplikation des Gauß-Kernels an der dritten Stelle des Filter mit $0$ definiert ist. Eine Multiplikation mit $0$
 ergibt immer $0$ und kann somit vernachlässigt werden. Die Laufzeit von \autoref{alg:derivativeofgauss-horizontal} und
 \autoref{alg:derivativeofgauss-vertical} ist konstant.

In \autoref{alg:sobel} wird, wie in \autoref{alg:derivativeofgauss-horizontal}, mittels Faltung die Orientierung eines \gls{edgels} bestimmt. Als Eingabeparameter wird das monochrome Bildsignal $I_m$, dessen Breite $w$ und Höhe $h$, sowie die Position des \gls{edgels} ($x,y$) benötigt.

\begin{algorithm}[ht]
\caption{Orientierung berechnen}
\label{alg:sobel}
\begin{algorithmic}[1]
	\Require $I_m, x, y, w, h$
	\Ensure $-\pi < v \leq \pi$
	\State $p_{x_1} \gets$ \Call{getPixel}{$I_m, x -1, y - 1, w, h$}
	\label{alg:sobel-readstart}
	\State $p_{x_2} \gets$ \Call{getPixel}{$I_m, x, y - 1, w, h$}
	\State $p_{x_3} \gets$ \Call{getPixel}{$I_m, x + 1, y - 1, w, h$}
	\State $p_{x_4} \gets$ \Call{getPixel}{$I_m, x - 1, y + 1, w, h$}
	\State $p_{x_5} \gets$ \Call{getPixel}{$I_m, x, y + 1, w, h$}
	\State $p_{x_6} \gets$ \Call{getPixel}{$I_m, x + 1, y + 1, w, h$}
	\State $p_{y_1} \gets$ \Call{getPixel}{$I_m, x - 1, y - 1, w, h$}
	\State $p_{y_2} \gets$ \Call{getPixel}{$I_m, x - 1, y, w, h$}
	\State $p_{y_3} \gets$ \Call{getPixel}{$I_m, x - 1, y + 1, w, h$}
	\State $p_{y_4} \gets$ \Call{getPixel}{$I_m, x + 1, y - 1, w, h$}
	\State $p_{y_5} \gets$ \Call{getPixel}{$I_m, x + 1, y, w, h$}
	\State $p_{y_6} \gets$ \Call{getPixel}{$I_m, x + 1, y + 1, w, h$}
	\label{alg:sobel-readend}
	\State $g_x \gets 0$
	\State $g_y \gets 0$
	\State $g_x \gets g_x + p_{x_1}$
	\label{alg:sobel-convolutestart}
	\State $g_x \gets g_x + \left(p_{x_2} \cdot 2\right)$
	\State $g_x \gets g_x + p_{x_3}$
	\State $g_x \gets g_x - p_{x_4}$
	\State $g_x \gets g_x - \left(p_{x_5} \cdot 2\right)$
	\State $g_x \gets g_x - p_{x_6}$
	\State $g_y \gets g_y + p_{y_1}$
	\State $g_y \gets g_y + \left(p_{y_2} \cdot 2\right)$
	\State $g_y \gets g_y + p_{y_3}$
	\State $g_y \gets g_y - p_{y_4}$
	\State $g_y \gets g_y - \left(p_{y_5} \cdot 2\right)$
	\State $g_y \gets g_y - p_{y_6}$
	\label{alg:sobel-convoluteend}
	\State $v \gets \arctan{\left(gy, gx\right)}$
	\label{alg:sobel-arctan}
	\State \textbf{return} $v$
\end{algorithmic}
\end{algorithm}


In Zeile \ref{alg:sobel-readstart}--\ref{alg:sobel-readend} werden die Pixelwerte ausgelsen und den Variablen zugewiesen. In Zeile \ref{alg:sobel-convolutestart}--\ref{alg:sobel-convoluteend} erfolgt die Faltung mit dem Sobel-Operator\footcite[Vgl.][S.~120--123]{burger05}, dessen Filter

\begin{subequations}
\begin{align}
	H_x =&
	\begin{pmatrix}
		1& 0& -1&\\
		2& 0& -2&\\
		1& 0& -1
	\end{pmatrix}
\end{align}
\begin{align}
	H_y =&
	\begin{pmatrix}
		1& 2& 1&\\
		0& 0& 0&\\
		-1& -2& -1
	\end{pmatrix}
\end{align}
\end{subequations}

den Gradienten $G_x$ und $G_y$ bestimmen. Wie in \autoref{alg:derivativeofgauss-horizontal} werden Multiplikationen von $0$-Werten des Filters vernachlässigt. Mit

\begin{equation}
	\label{eq:orientation}
	\Phi(x,y) = \arctan{\left(\tfrac{G_y}{G_x}\right)}
\end{equation}

wird die Orientierung in Zeile \autoref{alg:sobel-arctan} berechnet. Die Orientierung unterscheidet sich um $180^\circ$
 wenn anstatt von weiß nach schwarz ein verlauf von schwarz nach weiß erfolgt. Das Ergebnis liegt im Bereich
 $-\pi < v \leq \pi$. $\arctan$ in \autoref{eq:orientation} kann in C durch \textproc{atan2} zur Berechnung verwendet
 werden. Die Laufzeit von \autoref{alg:sobel} ist konstant.

Die Datenstruktur eines \gls{edgels} besteht aus der $x$- und $y$-Koordinate und der Orientierung
 (Vgl. \autoref{alg:datastructure-edgel}). Lese- und Schreibzugriffe auf die Elemente eines \gls{edgels} sind konstant.

\begin{algorithm}[ht]
\caption{Datenstruktur eines edgels}
\label{alg:datastructure-edgel}
	\begin{algorithmic}[1]
		\State $x$
		\State $y$
		\State $o$
		\Comment Orientierung
	\end{algorithmic}
\end{algorithm}


Der Vergleich, ob \gls{edgels} kompatibel sind, wird mit \autoref{alg:compatibleedgel} bewerkstelligt. Als Parameter
 werden zwei zu vergleichende \gls{edgels} $e_1$ und $e_2$ übergeben. In Zeile~$6$ und $13$ wird sichergestellt, dass
 $e_1.o$ und $e_2.o$ innerhalb von $67.5^\circ$\footcite[Vgl.][S.~417]{clarke96} liegen und damit kompatibel wären.
 Dies wird durch

\begin{equation}
	d = 2 \pi \left( \frac{ \frac{67.5}{2} }{360} \right) = 0.589
\end{equation}

überprüft. Es muss sichergestellt werden, dass die Orientierung in Bogenmaß erfolgt.


Die Datenstruktur eines Liniensegments ist in \autoref{alg:datastructure-line} definiert. Eine Linie besteht aus den
 \gls{edgels} $s$ und $e$, die den Start- und End-Punkt der Linie darstellen. Variable $c$ speichert die Anzahl der
 unterstützenden Edgels der Linie. Die Lese- und Schreibzugriffe auf die Datenstruktur ist konstant.

Der Edgelspeicher in \autoref{alg:datastructure-edgelpool} verwendet ein Array von \gls{edgels}
 (Vgl. \autoref{alg:datastructure-edgel}) mit fester Größe $N$ und einer Zählvariable um die die nächste freie Position
 im Array zu makieren. Der Speicherpool wiederum ist ein Array der Größe $S$ und die Adresse wird in einem Pointer
 $\mathit{pool}$ gespeichert.

\begin{algorithm}[ht]
\caption{Datenstruktur des Edgel-Speichers}
\label{alg:datastructure-edgelpool}
\begin{algorithmic}[1]
	\State $\mathit{data}[512]$
	\Comment Anzahl der Einträge
	\State $\mathit{count}$
\end{algorithmic}
\end{algorithm}

\autoref{alg:edgelpool-getmemorypools} basiert auf einem einfachen Stack Allocator von
 \citeauthor{kr}\footcite[Vgl.][S.~100--104]{kr}. Die Variable $n$ gibt die Anzahl der angeforderten Pools an. In Zeile
 \ref{alg:edgelpool-getmemorypools-checkpoolsize} wird überprüft, ob genügend Pools zur Verfügung stehen und liefert im
 Erfolgsfall die Adresse zu einem Speicher (\autoref{alg:datastructure-edgelpool}) zurück. Falls kein Pool mehr zur
 Verfügung steht, wird $\mathit{NULL}$ zurückgegeben. \autoref{alg:edgelpool-getmemorypool} vereinfacht die Anforderung
 eines Pools, da in den meisten Fällen nur ein Pool benötigt wird. Bei einem Aufruf kann somit auf einen Parameter
 verzichetet werden. Sowohl \autoref{alg:edgelpool-getmemorypools} als auch \autoref{alg:edgelpool-getmemorypool} haben
 eine konstante Laufzeit.

\begin{algorithm}[ht]
\caption{Hole Edgelpools}
\label{alg:edgelpool-getmemorypools}
\begin{algorithmic}[1]
	\Require $n$
	\If{$\mathit{data} + S - {pool} \geq n$}
		\State $pool = pool + n$
		\State \textbf{return} $\mathit{pool} - n$
	\Else
		\State \textbf{return} $\mathit{NULL}$
	\EndIf
\end{algorithmic}
\end{algorithm}

\begin{algorithm}[ht]
\caption{Hole Edgelpool}
\label{alg:edgelpool-getmemorypool}
\begin{algorithmic}[1]
	\State $p \gets$ \Call{getmemorypools}{1}
	\State \textbf{return} $p$
\end{algorithmic}
\end{algorithm}

Um \gls{edgels} in einem Pool zu speichern, verwende ich \autoref{alg:edgelpool-addedgel}. Der Algorithmus benötigt
 einen Pointer $p$ auf einen Pool und einen \gls{edgels} $e$. In Zeile
 \ref{alg:edgelpool-addedgel-validpointer-start}--\ref{alg:edgelpool-addedgel-validpointer-end} wird geprüft, ob der
 Pointer auf eine Adresse verweist. Falls $p$ null ist, wird der Algorithmus verlassen. In Zeile
 \ref{alg:edgelpool-addedgel-checkspace-start}--\ref{alg:edgelpool-addedgel-checkspace-end} wird geprüft, ob im Array
 genügend Platz für einen weiteren Eintrag vorhanden ist. Die Größe von $N$ Einträgen richtet sich nach der in \autoref{alg:datastructure-edgelpool} festgelegten Arraygröße $N$. Wenn genügend Platz vorhanden ist, wird in Zeile
 \ref{alg:edgelpool-addedgel-add-start}--\ref{alg:edgelpool-addedgel-add-end} der \gls{edgels} $e$ an die freie
 Position $c$ geschrieben. Danach wird $\mathit{count}$ inkrementiert. Das hinzufügen eines \gls{edgels} ist konstant.

\begin{algorithm}[ht]
\caption{Edgel hinzufügen}
\label{alg:edgelpool-addedgel}
\begin{algorithmic}[1]
	\Require $p,e$
	\If{$!p$}
		\State \textbf{return}
	\EndIf
	\If{$!p(\mathit{count}) < N$}
		\State \textbf{return} \Comment Speicher voll
	\EndIf
	\State $c \gets p(\mathit{count})$
	\State $p(\mathit{data}[c]) \gets e$
	\State $p(\mathit{count}) \gets c + 1$
\end{algorithmic}
\end{algorithm}

\gls{edgels} werden mittels \autoref{alg:edgelpool-getedgel} gelesen. Dazu wird der Pointer $p$ auf den Pool und der
 Index $i$ übergeben. In Zeile
 \ref{alg:edgelpool-getedgel-validpointer-start}--\ref{alg:edgelpool-getedgel-validpointer-end} wird geprüft, ob es
 sich um einen gesetzten Pointer handelt. Anschliessend wird in Zeile
 \ref{alg:edgelpool-getedgel-validrange-start}--\ref{alg:edgelpool-getedgel-validrange-end} geprüft, ob der Index $i$
 innerhalb des gespeicherten Bereichs der \gls{edgels} liegt. Danach wird in Zeile
 \ref{alg:edgelpool-getedgel-returnedgel} der Wert des \gls{edgels} an Position $i$ zurückgegeben. Der Zugriff auf
 einen \gls{edgels} ist konstant.

\begin{algorithm}[ht]
\caption{Edgel lesen}
\label{alg:edgelpool-getedgel}
\begin{algorithmic}[1]
	\Require $p,i$
	\If{$!p$}
	\label{alg:edgelpool-getedgel-validpointer-start}
		\State \textbf{return}
	\EndIf
	\label{alg:edgelpool-getedgel-validpointer-end}
	\State $c \gets p(\mathit{count})$
	\If{$! c > i$}
	\label{alg:edgelpool-getedgel-validrange-start}
		\State \textbf{return}
	\EndIf
	\label{alg:edgelpool-getedgel-validrange-end}
	\State \textbf{return} $p(\mathit{data}[i])$
	\label{alg:edgelpool-getedgel-returnedgel}
\end{algorithmic}
\end{algorithm}

Damit \gls{edgels} aus dem Array entfernt werden können, wird \autoref{alg:edgelpool-removeedgel} verwendet. Es wird der
 Pool-Pointer $p$ und die Position $i$ des zu löschenden \gls{edgels} übergeben. Nach der Überprüfung des Pointers $p$
 in Zeile \ref{alg:edgelpool-removeedgel-validpointer-start}--\ref{alg:edgelpool-removeedgel-validpointer-end} und der
 Überprüfung des zulässigen Bereichs in Zeile
 \ref{alg:edgelpool-removeedgel-validrange-start}--\ref{alg:edgelpool-removeedgel-validrange-end}, gibt es zwei zu
 behandelten Fälle um einen \gls{edgels} zu löschen.

\begin{algorithm}[ht]
\caption{Edgel löschen}
\label{alg:edgelpool-removeedgel}
\begin{algorithmic}[1]
	\Require $p,i$
	\If{$!p$}
		\State \textbf{return}
	\EndIf
	\State $c \gets p(\mathit{count})$
	\If{$! c > i$}
		\State \textbf{return}
	\EndIf
	\If{$c > i + 1$}
		\State \Call{memmove}{$p(\mathit{data}[i]), p(\mathit{data}[i + 1]), c - (i + 1) * \textproc{sizeof}(e)$}
	\EndIf
	\State $p(\mathit{count}) = c - 1$
\end{algorithmic}
\end{algorithm}

Der Edgel liegt
\begin{enumerate}
	\item nicht am Ende des Arrays oder \label{removeedgel-worst}
	\item liegt am Ende des Arrays. \label{removeedgel-best}
\end{enumerate}

Bei \ref{removeedgel-best} muss lediglich $\mathit{count}$ dekrementiert werden um auf den vorigen Wert zu verweisen
 (Vgl. \autoref{fig:decrementcounter}). Das dekrementieren der Zählvaribale $p(\mathit{count})$ ist eine Zuweisung in
 konstanter Zeit. % TODO: Grafik -> [1][2]..[n-1][n|c] wird zu [1][2]..[n-1|c][n]

Bei \autoref{removeedgel-worst} wird das Array an der Position $i$ geteilt und der Wertebereich von $[i+1 \dotsc i-n]$
 wird an die Position $i$ verschoben (Vgl. \autoref{fig:memmove}). % TODO: Grafik [1][2]..[i-1][i][i+1]..[i-n]
In Zeile \autoref{alg:edgelpool-removeedgel-memmove} gibt die Funktion \textproc{sizeof}($e$) die Speichergröße eines
 \gls{edgels} an, welche zum verschieben der Daten notwendig ist. Mit $c - (i + 1)$ wird die Anzahl der zu
 verschiebenden Einträge ermittelt. Im worst-case werden $N-1$ Einträge an Position $0$ des Arrays verschoben.

Um die Laufzeit der Funktion \textproc{memmove} zu bestimmen, wurde ein Testprogramm geschrieben, dass die Zeit misst,
 die benötigt wird, um Einträge zu verschieben. Anahand der Daten wurde mittels einer Regressionsanalyse untersucht, ob
 die gemessenen Daten einen linearen Zusammenhang aufweisen. Die erfassten $2000$ Datenpunkte wurde nach dem Vorbild von
 \textproc{time} % TODO: Referenz auf den Katalog für /usr/bin/time und man-page
ermittelt um Real-, User- und Sys-Zeit zu bestimmen. Aus User- und Sys-Zeit wurde die CPU-Zeit bestimmt, die zur
 Analyse benutzt wurde. Die Kovarianz für $X = \mathit{BYTES}$ und $Y = \mathit{CPU}$ beträgt $r = 0.9937538$. An
 dieser Stelle sei darauf hingewiesen, dass die Kovarianz für $X = \mathit{BYTES}$ und
 $Y = \mathit{REAL}$ mit $r = 0.9981969$ zwar größer ist, aber nicht die tatsächlichen Operationen des Testprogramms
 untersucht. Aus diesem Grund wurde $Y = \mathit{CPU}$ untersucht. Der Interzept beträgt $\beta_0 = -71.89\e{-06}$ und
 die Steigung $\beta_1 = 4.410\e{-09}$, sodass

\begin{subequations}
\begin{multline}
	y =\\ \beta_0 + \beta_1n
\end{multline}
\begin{multline}
	y =\\ -71.89\e{-06} + 4.410\e{-09}n
\end{multline}
\end{subequations} % TODO: Sauber formatieren

Daraus ergibt sich eine Laufzeit von $\Theta(n)$. In \autoref{fig:regression-memmove} ist der Plot der Daten angegeben.

\begin{figure}[!ht]
	\centering
	\includegraphics[width=.8\textwidth]{resources/Regression-memmove.png}
	\caption{Regressionsanalyse von $2000$ Datenpunkten.}
	\label{fig:regression-memmove}
\end{figure}

Um einen Pool für einen neuen Durchlauf zu löschen, kommt \autoref{alg:edgelpool-resetmemorypool} zum Einsatz. Als
 Parameter wird der Pointer $p$ übergeben und in Zeile
 \ref{alg:edgelpool-resetmemorypool-validpointer-start}--\ref{alg:edgelpool-resetmemorypool-validpointer-end}
 überprüft. Um alle Daten als gelöscht zu makieren, wird lediglich die Zählvariable in Zeile
 \ref{alg:edgelpool-resetmemorypool-reset} auf $0$ gesetzt. Die Zuweisung erfolgt in konstanter Zeit.

\begin{algorithm}[ht]
\caption{Lösche Daten in Edgelpool}
\label{alg:edgelpool-resetmemorypool}
\begin{algorithmic}[1]
	\Require $p$
	\If{$!p$}
	\label{alg:edgelpool-resetmemorypool-validpointer-start}
		\State \textbf{return}
	\EndIf
	\label{alg:edgelpool-resetmemorypool-validpointer-end}
	\State $p(\mathit{count}) \gets 0$
	\label{alg:edgelpool-resetmemorypool-reset}
\end{algorithmic}
\end{algorithm}

Wenn ein Speicherpool nicht mehr benötigt wird, kann er mit \autoref{alg:edgelpool-freememorypool} freigegeben werden.
 Der Pointer $p$ wird in Zeile
 \ref{alg:edgelpool-freememorypool-validpointer-start}--\ref{alg:edgelpool-freememorypool-validpointer-end} überprüft.
 Zeile \ref{alg:edgelpool-freememorypool-resetmemory} werden die Daten des Pools gelöscht
 (Vgl. \autoref{alg:edgelpool-resetmemorypool}). Im Anschluss wird in Zeile
 \ref{alg:edgelpool-freememorypool-checkpointer} überprüft, ob $p$ zu dem Array $\mathit{data}$ gehört und nicht größer
 als die definierte Speichergröße ist. Wenn der Test positiv ausfällt, wird der Pointer $p$ zur weiteren Verwendung in
 $\mathit{pool}$ gespeichert. Das freigeben eines Pools erfolgt in konstanter Zeit.

\begin{algorithm}[ht]
\caption{Speicher des Edgelpool freigeben}
\label{alg:edgelpool-freememorypool}
\begin{algorithmic}[1]
	\Require $p$
	\If{$!p$}
	\label{alg:edgelpool-freememorypool-validpointer-start}
		\State \textbf{return}
	\EndIf
	\label{alg:edgelpool-freememorypool-validpointer-end}
	\State \Call{resetmemorypool}{$p$}
	\label{alg:edgelpool-freememorypool-resetmemory}
	\If{$p \geq \mathit{data} \land p \leq \mathit{data} + S$}
	\label{alg:edgelpool-freememorypool-checkpointer}
		\State $\mathit{pool} \gets p$
	\EndIf
\end{algorithmic}
\end{algorithm}

Die Anzahl der \gls{edgels} in einem Pool werden durch \autoref{alg:edgelpool-count} ermittelt. Als Parameter wird
 Pointer $p$ übergeben und in Zeile
 \ref{alg:edgelpool-count-validpointer-start}--\ref{alg:edgelpool-count-validpointer-end} überprüft. Die Anzahl der
 Einträge wird in Zeile \ref{alg:edgelpool-count-counter} über die Zählvariable $p(\mathit{count})$ ermittelt. Der
 Zugriff auf die Variable, und somit die Laufzeit des Algorithmus, erfolgt in konstanter Zeit.

\begin{algorithm}[ht]
\caption{Anzahl der Einträge}
\label{alg:edgelpool-count}
\begin{algorithmic}[1]
	\Require $p$
	\If{$!p$}
		\State \textbf{return}
	\EndIf
	\State \textbf{return} $p(\mathit{count})$
\end{algorithmic}
\end{algorithm}

\begin{algorithm}[ht]
\caption{Datenstruktur eines Liniensegments}
\label{alg:datastructure-line}
	\begin{algorithmic}[1]
		\State $s$
		\Comment Start des Liniensegments
		\State $e$
		\Comment Ende des Liniensegments
		\State $c$
		\Comment Anzahl der Support-Edgels
	\end{algorithmic}
\end{algorithm}

\begin{algorithm}[ht]
\caption{Orientierung von zwei edgels überprüfen}
\label{alg:compatibleedgel}
\begin{algorithmic}[1]
	\Require $e_1, e_2$
	\State $c \gets 0$
	\If{$e_1.o == e_2.o$}
		\State \textbf{return TRUE}
	\ElsIf{$e_1.o < e_2.o$}
		\State $c \gets e_2.o - e_1.o$
		\If{$c < d$}
			\State \textbf{return TRUE}
		\Else
			\State \textbf{return FALSE}
		\EndIf
	\Else
		\Comment $e_1.o > e_2.o$
		\State $c \gets e_1.o - e_2.o$
		\If{$c < d$}
			\State \textbf{return TRUE}
		\Else
			\State \textbf{return FALSE}
		\EndIf
	\EndIf
\end{algorithmic}
\end{algorithm}


\begin{algorithm}[ht]
\caption{Auffinden von Liniensegmenten}
\label{alg:findlinesegments}
\begin{algorithmic}[1]
	\Require $E, L$
	\State $linecount \gets 0$
	\State $edgelcount \gets$ \Call{numberofedgels}{E}
	\While{$linecount > minEdgels \land edgelcount > minEdgels$}
	\State $linecount \gets numberoflinecount$
	\State $edgelcount \gets numberofedgels$
	\EndWhile
\end{algorithmic}
\end{algorithm}

% subsection hirzer (end)

% section analyse (end)					% ausführlich
\chapter{Ergebnisse} % (fold)
\label{cha:ergebnisse}
\begin{comment}
	Ergebnisse: Die gewonnen Daten aus Kap. Analyse bewerten.
\end{comment}

In diesem Kapitel werden die gewonnen Daten aus \autoref{cha:analyse} bewertet. Dazu wird das asymptotische Wachstum
 der Verfahren beschrieben und ihre Eingabemengen erläutert. Danach werden die Verfahren durch Messung ihrer
 Verarbeitungszeit miteinander verglichen.

\section{ARToolKitPlus} % (fold)
\label{sec:ergebnisse-artoolkitplus}
ARToolKitPlus benötigt zur Erkennung einer Marke das Verfahren \textproc{regionLabeling} und
 \textproc{arDetectMarker2}. In \autoref{sec:artoolkitplus} wurde das asymptotische Wachstum des Verfahrens für den
 schlechtesten Fall auf
\begin{equation*}
O(\mathit{lxsize}\cdot\mathit{lysize})  + \Theta(\mathit{label\_num}\cdot n \log n)
\end{equation*}
bestimmt. Die Eingabemenge für das Verfahren ist das Bildsignal $I$, dass in der Untersuchung aus $640 \times 480$
 Pixeln besteht. Als Optimierung des Verfahrens werden nur die Hälfte der horizontalen \gls{pixel} ($\mathit{lxsize}$)
 und die Hälfte der vertikalen \gls{pixel} ($\mathit{lysize}$) prozessiert (Vgl. \autoref{par:artoolkitplus}).
 $\mathit{label\_num}$ ist durch $\tfrac{\mathit{lxsize}\cdot\mathit{lysize}}{4}$ begrenzt und beschreibt die maximal
 mögliche Anzahl von Regionen in einem Bildsignal mit der Größe $\mathit{lxsize}\cdot\mathit{lysize}$. Variable $n$ in
 $n \log n$ beschreibt die Anzahl der Verarbeitungsschritte in der Konturverfolgung. Die Konstante
 $\mathit{AR\_CHAIN\_MAX} = 10000$ schränkt $n$ ein, sodass eine Konturverfolgung abgebrochen wird, falls nach $10000$
 untersuchten Pixeln keine geschlossene Kontur gefunden werden konnte.
% section section_name (end)

\section{Verfahren nach Hirzer} % (fold)
\label{sec:ergebnisse-hirzer}
Das Verfahren von \citeauthor{hirzer08} besteht aus dem Algorithmus \textproc{lineDetection}
 (\autoref{alg:linedetection-hirzerquaddetection}), um eine Marke zu erkennen. Das asymptotische Wachstum des
 Verfahrens wurde, für den schlechtesten Fall, in \autoref{sec:hirzer} auf
\begin{equation*}
\Theta(hwmn^2+hwl^2\cdot\mathit{length})
\end{equation*}
bestimmt. Auch hier ist die Eingabemenge des Verfahrens das Bildsignal $I$, mit einer Bildgröße von $640 \times 480$
 Pixeln. Die Anzahl der horizontalen \gls{pixel} wird durch die Variable $w$ und die Anzahl der vertikalen
 \gls{pixel} durch die Variable $h$ beschrieben. Variable $n$ bezeichnet die Anzahl der maximal möglichen Edgel und ist
 begrenzt auf $\tfrac{1}{5}(\mathit{regionSize}^2 - 2\mathit{regionSize})$, mit $\mathit{regionSize} = 40$. Die Anzahl
 der maximal möglichen Wiederholungen des RANSAC Verfahrens (\autoref{sub:linienerkennung_nach_clarke96}) ist durch
 $m = \tfrac{1}{2}(\sqrt{8n + 25} - 5)$ begrenzt. Die Variable $l$ bezeichnet die Menge der Liniensegmente und
 $\mathit{length}$ bezeichnet die Länge des zu verarbeitenden Liniensegments.
% section section_name (end)

\section{Vergleich der Verfahren} % (fold)
\label{sec:vergleich_der_verfahren}
Beide Tracking Verfahren verarbeiten die gleiche Eingabemenge $I$ eines Bildsignals und berechnen für jede im
 Bildsignal vorhandene Marke vier Eckpunkte. Beide Verfahren sind von der Höhe und der Breite des Bildsignals abhängig.
 Durch die unterschiedlichen Arbeitsweisen der Verfahren wird das Bildsignal $I$ in nicht vergleichbare Eingabemengen
 aufgeteilt. Dadurch bedingt ist eine Analyse anhand des asymptotischen Wachstums nicht eindeutig. Da beide Verfahren
 ein BGRA Signal mit $640 \times 480$ Pixeln zu vier Eckpunkten einer Marke verarbeiten, kann durch die Messung
 der Verarbeitungszeit die Verfahren verglichen werden.

Für die Untersuchung der Laufzeit wurden zuerst $1800$ Bilder vom Kamerasensor angefordert ohne sie von den Verfahren
 prozessieren zu lassen. Die Ergebnisse sind in \autoref{tab:image} angegeben.
\begin{table}[!ht]
	\begin{center}
	\begin{tabular}[]{r..}
	\toprule
	& \multicolumn{1} {>{\centering\arraybackslash}m{4cm}}{ARToolKitPlus}
	& \multicolumn{1} {>{\centering\arraybackslash}m{4cm}}{\citeauthor{hirzer08}} \\
	\midrule
	Median		& 30.00  & 30.00  \\
	Mittelwert	& 30.12  & 30.09  \\
	Max.		& 252.60 & 218.00 \\
	Min.		& 5.10   & 9.60   \\
	\bottomrule
	\end{tabular}
	\caption{Messergebnisse der Bilddaten im Überblick.}
	\label{tab:image}
	\end{center}
\end{table}
Bei beiden Verfahren liegt der Median bei $30$ Bildwiederholungen. Dies stimmt mit der Untersuchung der
 Bildwiederholrate aus \autoref{sec:allgemein} überein. Mit diesen Ergebnissen kann sichergestellt werden, dass bei
 beiden Verfahren die Anforderung eines Bildsignals vom Bildsensor mit der vollen Geschwindigkeit erfolgt. Die
 Ausreißer bei beiden Verfahren, wirken sich nur geringfügig auf den Mittelwert aus und können vernachlässigt werden.

In einer zweiten Messung wurden von beiden Verfahren $1800$ Bilder zu vier Eckpunkten einer Marke verarbeitet. Für die
 Untersuchung der Markenerkennung wurde eine Marke auf einem hellen Untergrund platziert und mittig ausgerichtet. Die
 Erstellung der Messpunkte wurde für beide Verfahren unter gleichen Bedingungen durchgeführt. Die Ergebnisse der
 Markenerkennung sind in \autoref{tab:marker} aufgeführt.
\begin{table}[!ht]
	\begin{center}
	\begin{tabular}[]{r..}
	\toprule
	& \multicolumn{1} {>{\centering\arraybackslash}m{4cm}}{ARToolKitPlus}
	& \multicolumn{1} {>{\centering\arraybackslash}m{4cm}}{\citeauthor{hirzer08}} \\
	\midrule
	Median		& 19.10  & 15.70 \\
	Mittelwert	& 18.29  & 15.81 \\
	Max.		& 20.50  & 20.10 \\
	Min.		& 1.40   & 5.10  \\
	\bottomrule
	\end{tabular}
	\caption{Messergebnisse der Markenerkennung im Überblick.}
	\label{tab:marker}
	\end{center}
\end{table}
Durch die Verarbeitungszeit der Verfahren fällt die Bildwiederholrate, im Gegensatz zur ersten Untersuchung, bei beiden
 Verfahren ab. Bei ARToolKitPlus liegt die Median der Laufzeit bei $19.10$ Bildwiederholungen, während der Median der
 Laufzeit bei dem Verfahren von \citeauthor{hirzer08} bei $15.81$ Bildwiederholungen liegt. Das arithmetische Mittel
 liegt mit $18.29$ Bildwiederholungen bei ARToolKitPlus unterhalb des Median. Dies ist durch den kleinsten gemessenen
 Wert von $1.4$ Bildwiederholungen zurückzuführen. Bei dem Verfahren von \citeauthor{hirzer08} liegt das arithmetische
 Mittel hingegen mit $15.81$ Bildwiederholungen leicht über dem Median, da der kleinste gemessene Wert mit $5.1$
 Bildwiederholungen schneller war, als bei ARToolKitPlus. Aus diesen Messergebnissen wird deutlich, dass ARToolKitPlus
 in der gleichen Zeit mehr Bilder verarbeitet als das Verfahren von \citeauthor{hirzer08}.
% section vergleich_der_verfahren (end)

% chapter ergebnisse (end)				% knapp
\chapter{Fazit} % (fold)
\label{cha:fazit}

Wie in \autoref{cha:ergebnisse} gezeigt, erfüllen beide Verfahren die Echtzeitbedingung von \citeauthor{azuma97}. Die Verwendung der Verfahren unter iOS auf einem iPod touch ist durch die experimentelle Analyse bewiesen worden.

Wie die Analyse in \autoref{cha:analyse} gezeigt hat, sind beide Verfahren von der Eingabemenge des Bildsignals abhängig. Da ARToolKitPlus und das Verfahren von \citeauthor{hirzer} unterschiedliche Verfahren zur Erkennung einer Marke verwenden, ist eine Aussage über das asymptotische Wachstum schwierig. Da beide Verfahren Optimierungsstrategien anwenden, um die Menge der zu verarbeitenden Daten zu reduzieren, fällt dieser Optimierungsschritt in der Untersuchung des Wachstums weg. Erst die experimentelle Analyse offenbarte die Leistungsfähigkeit der Tracking Verfahren.

Betrachtet man lediglich die Geschwindigkeit der beiden Verfahren, so müsste ARToolKitPlus gegenüber dem Verfahren von \citeauthor{hirzer08} besser abschneiden.

Die Stärke von ARToolKitPlus liegt in der Geschwindigkeit der Verarbeitung eines Bildsignals, jedoch wird diese Geschwindigkeit auf Kosten der Erkennungsrate gewonnen. Da ARToolKitPlus eine Marke durch eine Region in einem Binärbild erkennt, wird bei ungünstigen Lichtverhältnissen keine Marke erkannt, da das Verfahren einen Schatten im Bereich der Marke als große Region betrachtet.

Diese Schwäche wird durch die Betrachtung von Liniensegmenten im Verfahren von \citeauthor{hirzer08} vermieden. Auch wenn beispielsweise eine Ecke einer Marke verdeckt ist, erkennt das Verfahren eine Marke. Der Schwachpunkt des Verfahrens nach \citeauthor{hizer08} ist bei vielen Hell/Dunkel Wechsel zu erkennen. Bei einer Jalousie, durch die Sonnenlicht eintritt, erkennt das Verfahren alle Wechsel als möglicher \gls{edgel}, der durch die parallele Ausrichtung der Jalousie zu einer Linie erweitert wird.


% Die Optimierung bei ARToolKitPlus, nur die Hälfte der vertikalen \gls{pixel} und die Hälfte der horizontalen \gls{pixel} zu untersuchen, wird durch die Betrachtung der Wachstumsrate nicht ersichtlich. Das gleiche gilt für die Optimierung bei dem Verfahren nach \citeauthor{hirzer08}.
% Hier muss durch die Verarbeitung der Scanlines im Abstand von $5$ \gls{pixel} weniger Daten verarbeitet werden als Eingabedaten vorhanden sind.


% chapter fazit (end)			% bis zu 5% der Arbeit

% Bei Abage auskommentieren
% \nocite{*}

%\bibliographystyle{bib}
% 7) vorläufige Literaturangabe ist im header
\appendix
\chapter{Anhang} % (fold)
\label{cha:anhang}
\begin{algorithm}[!ht]\small
\caption{MarkerInfo2}
\label{alg:datastructure-markerinfo2}
\begin{algorithmic}[1]
	\State $\mathit{area}$
	\State $\mathit{pos}[2]$
	\State $\mathit{coord\_num}$
	\State $\mathit{x\_coord}[\mathit{AR\_CHAIN\_MAX}]$
	\State $\mathit{y\_coord}[\mathit{AR\_CHAIN\_MAX}]$
	\State $\mathit{vertex}[5]$
\end{algorithmic}
\end{algorithm}

\begin{algorithm}[!ht]
\caption{\textproc{arDetectMarker}}
\label{alg:detectmarker}
\begin{algorithmic}[1]
	\Require $I,\mathit{thresh},\mathit{marker\_info},\mathit{marker\_num}$
	\State $I_l \gets$ \textproc{NULL}
	\label{alg:detectmarker-init-start}
	\State $\mathit{label\_num},\mathit{area},\mathit{clip},\mathit{label\_ref},\mathit{pos} \gets \infty$
	\label{alg:detectmarker-init-end}
	\State \Call{autoThreshold.reset}{}
	\label{alg:detectmarker-call-autothreshold}
	\State \Call{checkImageBuffer}{}
	\label{alg:detectmarker-call-imagebuffer}
	\State $\mathit{marker\_num} \gets 0$
	\State $I_l \gets$ \Call{arLabeling}{$I,\mathit{thresh},\mathit{label\_num},\mathit{area},\mathit{pos},\mathit{clip},\mathit{label\_ref}$}
	\label{alg:detectmarker-call-labeling}
	% AR_AREA_MAX (100000) und AR_AREA_MIN (70) sind defines
	% wmarker_num globale variable
	\If{$I_l$}
	\label{alg:detectmarker-check-il-start}
		\State $\mathit{marker\_info2} \gets$ \textproc{arDetectMarker2}$\left(
		\begin{aligned}
				& I_l,\mathit{thresh},\mathit{label\_ref},\mathit{area},\mathit{pos},\mathit{clip},\\
				& \mathit{AR\_AREA\_MAX},\\
				& \mathit{AR\_AREA\_MIN},\\
				& 1.0, \mathit{wmarker\_num}
		\end{aligned}\right)$
		\label{alg:detectmarker-call-method}
		\If{$\mathit{marker\_info2}$}
		\label{alg:detectmarker-check-marker-start}
			\State \ldots \Comment{Weitere Anweisungen zur Identifikation einer Marke.}
		\EndIf
		\label{alg:detectmarker-check-marker-end}
	\EndIf
	\label{alg:detectmarker-check-il-end}
	\State \ldots \Comment{Weitere Anweisungen zur Identifikation einer Marke.}
\end{algorithmic}
\end{algorithm}

\begin{algorithm}[ht]
\caption{\textproc{arDetectMarker} (Fortsetzung)}
\label{alg:detectmarker-1}
\begin{algorithmic}[1]
	\Require $I,\mathit{thresh},\mathit{marker\_info},\mathit{marker\_num}$
	\State $I_l \gets$ \textproc{NULL}
	\State $\mathit{label\_num},\mathit{area},\mathit{clip},\mathit{label\_ref},\mathit{pos} \gets \infty$
	\State \Call{autoThreshold.reset}{}
	\State \Call{checkImageBuffer}{}
	\State $\mathit{marker\_num} \gets 0$
	\State $I_l \gets$ \Call{arLabeling}{$I,\mathit{thresh},\mathit{label\_num},\mathit{area},\mathit{pos},\mathit{clip},\mathit{label\_ref}$}
	\label{alg:detectmarker-1-call-labeling}
	% AR_AREA_MAX (100000) und AR_AREA_MIN (70) sind defines
	% wmarker_num globale variable
	\If{$I_l$}
	\label{alg:detectmarker-1-check-il-start}
		\State $\mathit{marker\_info2} \gets$ \textproc{arDetectMarker2}$\left(
		\begin{aligned}
				& I_l,\mathit{thresh},\mathit{label\_ref},\mathit{area},\mathit{pos},\mathit{clip},\\
				& \mathit{AR\_AREA\_MAX},\\
				& \mathit{AR\_AREA\_MIN},\\
				& 1.0, \mathit{wmarker\_num}
		\end{aligned}\right)$
		\label{alg:detectmarker-1-call-method}
		\If{$\mathit{marker\_info2}$}
			\State \ldots \Comment{Weitere Anweisungen zur Identifikation einer Marke.}
		\EndIf
	\EndIf
	\label{alg:detectmarker-1-check-il-end}
	\State \ldots \Comment{Weitere Anweisungen zur Identifikation einer Marke.}
\end{algorithmic}
\end{algorithm}

% chapter anhang (end)
\backmatter

\printbibliography[]



\end{document}
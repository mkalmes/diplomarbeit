%
%  Created by Marc Kalmes on 2010-06-19.
%

% \documentclass[pdftex,a4paper,twoside,12pt,final,toc=bib]{scrbook}
\documentclass[pdftex,a4paper,twoside,12pt,draft,toc=bib]{scrbook}
\usepackage[ngerman]{babel}
\usepackage[T1]{fontenc}

\usepackage{amssymb,amsmath}

\usepackage{algorithm,algpseudocode}
\floatname{algorithm}{Algorithmus}
\newcommand{\algorithmautorefname}{Alg.}
% \newcommand*\Cost[2]{\hfill \mbox{#1} \mbox{#2}}
\newcommand*\Cost[2]{\hfill \begin{tabular}{l r} #1 & #2 \\ \end{tabular}}
%\newcommand*\Cost[2]{\begin{tabular}{l r} #1 & #2 \\ \end{tabular}}


\usepackage{color}

\usepackage{setspace}
%\doublespacing		% doppelzeilig oder
\onehalfspacing		% anderthalbzeilig

% Umlaut können direkt als UTF-8 Zeichen eingegeben werden
\usepackage[utf8]{inputenc}

% Setup for fullpage use
\usepackage{fullpage}

% Surround parts of graphics with box
\usepackage{boxedminipage}

% Package for including code in the document
\usepackage{listings}

% Package for my comments
\usepackage{verbatim}
\usepackage{fancyvrb,relsize}

\DefineVerbatimEnvironment%
      {VerbatimProg}%
      {Verbatim}%
      {fontsize=\relsize{-1}}

% If you want to generate a toc for each chapter (use with book)
% \usepackage{minitoc}

% This is now the recommended way for checking for PDFLaTeX:
\usepackage{ifpdf}

\ifpdf
\usepackage[pdftex]{graphicx}
\else
\usepackage{graphicx}
\fi


\usepackage[bibstyle=authoryear,citestyle=authoryear,language=german,block=ragged,doi=false]{biblatex}
\DeclareFieldFormat[article]{citetitle}{\mkbibemph{#1\isdot}}
\DeclareFieldFormat[inbook]{citetitle}{\mkbibemph{#1\isdot}}
\DeclareFieldFormat[incollection]{citetitle}{\mkbibemph{#1\isdot}}
\DeclareFieldFormat[inproceedings]{citetitle}{\mkbibemph{#1\isdot}}

\usepackage{csquotes}
\bibliography{bibliography/bibliography}	% 7) vorläufige Literaturangabe


\usepackage{hyperref}

\renewcommand {\chapterautorefname}{Kapitel}

\hypersetup{
	a4paper,
	bookmarksnumbered = true,
	pdftitle = Tracking Verfahren für Augmented Reality Anwendungen unter iOS 4,
	pdfauthor = Marc Kalmes ,
	pdfsubject = Exposé
}

\usepackage{glossaries}
% Glossar
\newglossaryentry{pixel}{
	name={Pixel},
	description={Bildelement},
	plural=Pixels
}
\newglossaryentry{chrominanz}{
	name={Chrominanz},
	description={Farbigkeit}
}
\newglossaryentry{edgel}{
	name={Edgel},
	description={Kantenpixel},
	plural=Edgels
}

\makeglossaries

\begin{document}
\frontmatter

\ifpdf
\DeclareGraphicsExtensions{.pdf, .jpg, .tif}
\else
\DeclareGraphicsExtensions{.eps, .jpg}
\fi

% %\thispagestyle{empty}
%\begin{center}
%\Large{Privatheit und Öffentlichkeit am Modell des Internet}
%\end{center}
%Projektarbeit
%vorgelegt an der Fachhochschule Köln
%Campus Gummersbach
%im Studiengang Allgemeine Informatik
%ausgearbeitet von:
%Marc Kalmes
%\end{document}
\begin{titlepage}
 
\begin{center}
 
 
% Upper part of the page
\includegraphics[width=0.15\textwidth]{./resources/logo}\\[1cm]
 
\textsc{\huge \bfseries Privatheit und Öffentlichkeit am Modell des Internet}\\[2.0cm]
 
\textsc{\Large Projektarbeit}\\[1.5cm]
\Large vorgelegt an der Fachhochschule Köln\\[0.2cm]
\Large Campus Gummersbach\\[0.2cm]
\Large im Studiengang Allgemeine Informatik\\[1.5cm]
 
 
% Title
%\HRule %\\[0.4cm]
%{ \huge \bfseries Lager brewing techniques}\\[0.4cm]
 
%\HRule \\[1.5cm]
 
% Author and supervisor
\begin{minipage}{0.4\textwidth}
\begin{flushleft} \large
\emph{ausgearbeitet von:}\\
Marc \textsc{Kalmes} -- 11025526
\end{flushleft}
\end{minipage}
\begin{minipage}{0.4\textwidth}
\begin{flushright} \large
\emph{Betreuerin:}\\
Prof. Dr. Gabriele \textsc{Koeppe}
\end{flushright}
\end{minipage}
 
\vfill
 
% Bottom of the page
{\large Gummersbach, im Mai 2009}
 
\end{center}
 
\end{titlepage}	% TODO: Titel als eigenes tex-dokument setzen
\titlehead{Fachhochschule Köln\\ Fakultät für Informatik und Ingenieurwissenschaften}
\subject{Diplomarbeit}
\title{Marker-basierte Tracking-Verfahren für Augmented Reality Anwendungen unter iOS 4}
\author{Marc Kalmes}
\date{Juli 2011}
\publishers{betreut durch Prof. Dr. Heiner Klocke}

\maketitle

\tableofcontents

\selectlanguage{UKenglish}
\chapter*{Abstract} % (fold)
\label{cha:abstract}
The diploma thesis \emph{Marker-basierte Tracking-Verfahren für Augmented Reality Anwendungen unter iOS 4} asks the question whether tracking systems are practical on mobile devices. A tracking system by \citeauthor{hirzer08} was implemented within this thesis and compared with ARToolKitPlus.

Tracking system were realized in the field of Augmented Reality research on powerful computers. Augmented Reality on mobile devices has to cope with limited memory and restricted processing power in contrast to a computer. The requirements of tracking systems will be explained. With the requirement definitions the question should be answered wether a tracking system is capable to run on the limited resources of a mobile device.

The algorithms of the system by \citeauthor{hirzer08} and the algorithms by ARToolKitPlus will be analyzed and the asymptotic run time determined and evaluated. An additionally rum time measurement creates another comparison feature.

The thesis explains finally how the described requirement of the methods were met and which pros and cons ARToolKitPlus and the system by \citeauthor{hirzer08} exhibit.
% chapter abstract (end)

\selectlanguage{ngerman}
\chapter*{Zusammenfassung} % (fold)
\label{cha:abstract-deu}
Die Diplomarbeit \emph{Marker-basierte Tracking-Verfahren für Augmented Reality Anwendungen unter iOS 4} beschäftigt
 sich mit der Frage, ob Tracking-Verfahren für mobile Endgeräte geeignet sind. Im Rahmen dieser Arbeit wurde ein
 Tracking-Verfahren nach \citeauthor{hirzer08} implementiert und mit dem Tracking-Verfahren von ARToolKitPlus
 verglichen.

Tracking-Verfahren wurden in der Forschung der Augmented Reality auf leistungsfähigen Computern durchgeführt. Der
 Einsatz von Augmented Reality auf Smartphones muss somit mit weniger Arbeitsspeicher und weniger Prozessorleistung
 auskommen, als auf einem Computer. Die Anforderungen eines Tracking-Verfahren wird in dieser Arbeit erläutert. Anhand
 der Anforderungen soll die Frage beantwortet werden, ob Tracking Verfahren mit den limitierten Resourcen eines
 mobilen Endgerätes ausgeführt werden können.

Die Algorithmen des Verfahrens nach \citeauthor{hirzer08} und die Algorithmen von ARToolKitPlus werden analysiert und
 ihre Laufzeitfunktionen, sowie das asymptotische Wachstum der Verfahren, bestimmt und bewertet. Zusätzlich wird durch
 die Messung der Laufzeit ein Vergleichsmermal erstellt.

Die Diplomarbeit erläutert abschließend, wie die vorgestellten Verfahren die Anforderungen an Tracking-Verfahren
 erfüllen und welche Stärken und Schwächen ARToolKitPlus und das Verfahren von \citeauthor{hirzer08} aufweisen.
% chapter zusammenfassung (end)

\mainmatter

\chapter{Einleitung} % (fold)
\label{cha:einleitung}

Heutzutage können wir zwei Phänomene beobachten: Das erste Phänomen ist die Zunahme von \gls{AR} Anwendungen außerhalb
 der Forschung. \gls{AR} verspricht neue Bedienkonzepte und Informationsdarstellung durch Überlagerung von realen und
 virtuellen Bil\-dern. Die dazu benötigte Ausrüstung bestand viele Jahre aus einem so genannten Head-mounted Display
 und einem Computer in einem Rucksack\footcite{azuma01}.

Das zweite Phänomen ist die weite Verbreitung von Smartphones und deren ge\-sell\-schaft\-liche Akzeptanz. Dabei wird
 die Leistungsfähigkeit von Smartphones stetig ge\-stei\-gert und Anwendungen sind als Folge nicht mehr auf den
 Computer beschränkt. Gerade Spiele profitieren von der Prozessor\-ge\-schwin\-dig\-keit und den heute üblichen
 Grafikprozessoren moderner Smartphones. Auch die mobile Nutzung von E-Mail und Internet hat durch Smartphones
 zugenommen. Aus diesen Gründen bietet es sich an, die Mög\-lich\-kei\-ten der \gls{AR} auf Smartphones zu untersuchen.

\section{Forschungsstand} % (fold)
\label{sec:forschungsstand}
\begin{comment}
	Forschungsstand: Alle untersuchten Arbeiten aufführen und kurz erklären.
\end{comment}

ARToolKit\footcite{artoolkit} wurde 1999 von Kato entwickelt und war das erste \gls{AR}-System, dass als Open Source einer breiten Entwicklergemeinde zur Verfügung stand. ARToolKit galt als die Referenz für Forschung im Bereich der \gls{AR}. Allerdings war der Systementwurf nicht auf das Aufkommen von mobilen Computern und Smartphones vorbereitet, was dafür sorgte, dass das System in dieser Form für weitere Forschung uninteressant wurde.

ARToolKitPlus\footcite{artoolkitplus} ist eine Optimierung von ARToolKit und wurde für mobile Geräte angepasst. Die Schwächen von ARToolKit wurden entfernt und das System auf den neusten Stand der Forschung aktualisiert. Besondere Eigenschaften zur robusten Mar\-ken\-er\-kennung waren aber nicht in der Lage auf limitierten mobilen Geräten ausgeführt zu werden.

Die Erfahrung aus der Entwicklung von ARToolKitPlus wurde genutzt um Studierstube\footcite{studierstube} zu entwickeln. Studierstube ist das erste System, dass auf die Bedürfnisse von Smartphones und mobilen Geräten konzipiert wurde und von Grund auf neu entwickelt wurde. Studierstube ist nicht wie ARToolKit und ARToolKitPlus im Quellcode verfügbar.

Wagner und Schmalstieg waren an der Weiterentwicklung von ARToolKitPlus beteiligt und haben aus dieser Erfahrung heraus Studierstube entwickelt. Mit ihrer Ver\-öf\-fent\-li\-chung von \citetitle{wagner03}\footcite{wagner03} wird die frühe Entwicklung von mobilen \gls{AR}-Systemen beleuchtet. In \citetitle{wagner09a}\footcite{wagner09a} und \citetitle{wagner09b}\footcite{wagner09b} geben sie einen Einblick in ihre Arbeit, die dieser Untersuchung zu Grunde liegt.

In \citetitle{clarke96}\footcite{clarke96} wird ein Verfahren zur schnellen Linienerkennung vorgestellt. Die robuste Erkennung von Linien und die hohe Fehlertoleranz gegenüber schwankenden Lichtverhältnissen zeichnen dieses Verfahren aus um als Grundlage zur Markenerkennung verwendet zu werden.

% section forschungsstand (end)
\section{Definition} % (fold)
\label{sec:definition}
\begin{comment}
	Definiere Begriffe der Augmented Reality und Bildverarbeitung, die dem Leser nicht geläufig sind. Denke dabei an Prof. Klocke als Leser ohne besonderen Kenntnisstand in der AR/Bildverarbeitung.

	Azuma Definition der AR
	Milgram + Kishino MR oder nur Milgram Definition MR
\end{comment}

Bis heute ist \gls{AR} nicht eindeutig definiert, obwohl die ersten Untersuchung von \citeauthor{sutherland} bereits
 1968 vorgestellt wurden\footcite{sutherland}. \citeauthor{milgram94b} beschrieben in
 \citetitle{milgram94b}\footcite{milgram94b} \gls{AR} als einen Punkt des Reality-Virtuality (RV) Continuum
 (Vgl. \autoref{fig:mixedreality}), welches reale und virtuelle Objekte gemeinsam auf einem Bildschirm darstellt.
\begin{figure}[!ht]
	\centering
	\input{resources/mixedreality.pdf_tex}
	\caption{Reality-Virtual Continuum nach \citeauthor{milgram94b}}
	\label{fig:mixedreality}
\end{figure}

Nach der neueren Definition von \citeauthor{azuma97}\footcite{azuma97} muss ein \gls{AR}-System drei Kriterien
 erfüllen:
\begin{itemize}
	\item Reale und virtuelle Objekte werden kombiniert
	\item Bezug von realen und virtuellen Objekten im 3-dimensionalen Raum
	\item Interaktion in Echtzeit
\end{itemize}
Das erste Kriterium bedeutet, dass, im Gegensatz zu Computerspielen, virtuelle Gegenstände mit der realen Welt in
 Verbindung stehen. Durch die zweite Bedingung muss ein virtuelles Objekt in einem räumlichen Zusammenhang stehen. Als
 Beispiel muss eine virtuelle Tasse auf dem realen Schreibtisch platziert sein und nicht im Raum schweben. Zuletzt muss
 die Interaktion mit einem \gls{AR}-System in Echtzeit erfolgen. Dies ist eine Abgrenzung zu einer Computeranimation in
 einem Film.

Nach \citeauthor{moeller2008}\footcite[Vgl.][S.~1]{moeller2008} ist Interaktion in Echtzeit mit $15$ FPS erfüllt, wobei
 eine Geschwindigkeit von $6$ FPS noch als interaktiv gilt. Diese Definition wird von \citeauthor{wagner09b} in ihrer
 Untersuchung\footcite[Vgl.][S.~8--9]{wagner09b} unterstützt.

\subsection{Tracking-Verfahren} % (fold)
\label{sec:tracking_verfahren}
Tracking-Verfahren bezeichnet bei \gls{AR}-Systemen die Fähigkeit, die reale Position eines Benutzers oder eines
 Displays in Relation zur Position einer \gls{AR}-Umgebung zu setzen. Tracking-Verfahren müssen, wie in der Definition
 von \citeauthor{azuma97} schon erwähnt, in Echtzeit agieren. Zusätzlich müssen Tracking-Verfahren, je nach
 Anwendungsgebiet, unter unterschiedlichen Lichtverhältnissen arbeiten können und eine Marke zuverlässig erkennen
 können.

Ein Bereich der Tracking-Verfahren ist das Fiducial-Tracking, auch Marken-basiertes Tracking genannt. Bei diesem
 Tracking-Verfahren wird durch Hilfe einer Marke die Position im realen und virtuellen Raum berechnet. Die bereits
 erwähnten Systeme ARToolKit, ARToolKitPlus und Studierstube, sind Marken-basierte Verfahren und suchen in einem
 Bildsignal nach einer bitonalen Marke. Das erste Marken-basierte \gls{AR}-System war
 Matrix\footcite{rekimoto1998matrix} von \citeauthor{rekimoto1998matrix}. \citeauthor{fiala2004artaga} entwickelte mit
 ARTag\footcite{fiala2004artaga} ein \gls{AR}-System, dass die Idee eines 2D-Codes zur Identifizierung von Matrix
 übernahm. In ARToolKitPlus wurde die Identifizierung von 2D-Codes ebenfalls implementiert.
% subsection tracking_verfahren (end)

\subsection{Bitonale Marken} % (fold)
\label{sub:bitonalemarken}
Bei Marken-basierten \gls{AR}-Verfahren werden bitonale Marken zur Erkennung und Verfolgung von realen Objekten
 eingesetzt. Eine Marke besteht aus einem schwarzen quadratischen Rahmen auf weißem Untergrund und trägt in der Mitte
 Informationen zur Identifizierung. ARToolKit verwendet simple Muster oder Schriftzeichen zur Identifizierung, wie in
 \autoref{fig:marke-artoolkit} gezeigt. ARTag hingegen verwendet eine binär kodierte Zahl als Identifizierungsmerkmal,
 die als Anordnung von $6 \times 6$ Pixeln dargestellt wird (s.~\autoref{fig:marke-artag}). ARToolKitPlus, der
 Nachfolger von ARToolKit nutzt ein ähnliches Verfahren wie ARTag\footcite[Vgl.][S.~142]{wagner07b} und unterscheidet
 sich in der Rahmenbreite (s.~\autoref{fig:marke-artoolkitplus}). Wenn in den folgenden Kapiteln von einer bitonalen
 Marke, oder einfach Marke, gesprochen wird, beziehe ich mich auf einen schwarzen quadratischen Rahmen auf weißem
 Untergrund.

\begin{figure}[!ht]
	\centering
	\subfigure[ARToolKit]{
		\label{fig:marke-artoolkit}
		\includegraphics[scale=1]{resources/Marker-ART.pdf}
	}
	\subfigure[ARTag]{
		\label{fig:marke-artag}
		\includegraphics[scale=1]{resources/Marker-ARTag.pdf}
	}
	\subfigure[ARToolKitPlus]{
		\label{fig:marke-artoolkitplus}
		\includegraphics[scale=1]{resources/Marker-ART+.pdf}
	}
	\caption{Verschiedene bitonale Marken.
	}
	\label{fig:bitonale-marken}
\end{figure}
% subsection bitonale_marken (end)

\begin{comment}
	(wagner/schmalstieg ARToolKitPlus fpr Pose Tracking on Mobile Devices S.4)

	Beim Fiducial Marker Tracking werden künstliche Marken zur Erkennung und Verfolgung von realen Objekten eingesetzt. Häufig werden für diese Marken quadratische schwarze Rahmen verwendet die innerhalb des Rahmens Schriftzeichen, Bilder oder 2D Codes enthalten. Diese Marken sind einfach herzustellen und können mit geringem Aufwand an Objekte angebracht werden.

	Diese bitonale Marken haben den Vorteil, dass sie Robust gegen Helligkeitsveränderung sind und die Entscheidung eines Pixels auf eine Schwellwert-Entscheidung reduziert werden kann. Marken für \gls{AR} müssen in einem großen Blickfeld erkannt werden können, was wiederum bei industriellen Anwendung nicht der Fall ist, da hier Marken den größten Teil des Bildes einnehmen können.
\end{comment}

% section definition (end)
\section{Problemstellung} % (fold)
\label{sec:problemstellung}
\begin{comment}
	Problemstellung: Problemstellung und Frage im Detail erläutern
\end{comment}

Die Erforschung von \gls{AR} Verfahren und Anwendungen wurde in der Vergangenheit auf leistungsfähigen Computern
 durchgeführt. Die Untersuchung der Verfahren auf mobilen Geräten ist ein noch neuer Bereich, der durch die Limitierung
 der Hardware eine Herausforderung darstellt.

Um eine bitonale Marke zu erkennen, werden Verfahren zur Erkennung von Linien eingesetzt, die aus dem Bereich der
 digitalen Bildverarbeitung stammen. Anders als bei der herkömmlichen Bildverarbeitung muss Bildverarbeitung für
 \gls{AR} die Verarbeitungsschritte so schnell ausführen, dass eine Analyse eines Bildes in Echtzeit stattfinden kann.
 Bei Bildverarbeitung außerhalb der \gls{AR} ist dies kein notwendiges Kriterium.

Der Bilddatenbeschaffung über den Kamerasensor eines Smartphones muss so schnell wie möglich erfolgen um für die
 Weiterverarbeitung verwendet werden zu können. Bei Smartphones werden in aller Regel Kamerasensoren aus dem
 Verbraucherbereich verwendet. Diese Kamerasysteme liefern nicht die Performanz einer professionellen Kamera. Dadurch
 bedingt müssen die zur Verfügung stehenden Mittel so gut es geht ausgenutzt werden.

Die Softwarearchitektur für \gls{AR} auf Smartphones unterscheidet sich von einer PC Architektur, abgesehen von der
 Prozessorgeschwindigkeit, grundsätzlich durch die Menge und Bandbreite des Arbeitsspeichers. Dies bedeutet, dass alle
 Arbeitsschritte durch eine ineffektive Ausnutzung des zur Verfügung stehenden Speichers verlangsamt werden.

Wie in \autoref{sec:forschungsstand} erwähnt, ist ARToolKitPlus das einzige aktuelle Tracking Verfahren, das im
 Quellcode verfügbar ist. Wie \citeauthor{wagner09a} festgestellt haben, eignet sich ARToolKitPlus nur noch bedingt für
 mobile Plattformen\footcite{wagner09a}, sodass sie mit Studierstube ein neues Tracking Verfahren entworfen haben. Ein
 modernes Verfahren zur Markenerkennung wurde von \citeauthor{hirzer08} vorgestellt\footcite{hirzer08}, dass im Rahmen
 dieser Arbeit implementiert wurde.

Diese Arbeit vergleicht die unterschiedlichen Tracking Verfahren von ARToolKitPlus und dem Verfahren von
 \citeauthor{hirzer08} zur Erkennung einer bitonalen Marke auf einem iPod touch (4. Generation) unter iOS 4.

Der Fokus dieser Arbeit liegt dabei auf die Einhaltung der Echtzeitbedinung von \citeauthor{azuma97}.


% section problemstellung (end)

% chapter einleitung (end)				% bis zu 10% der Arbeit
\chapter{Untersuchungsdesign} % (fold)
\label{cha:untersuchungsdesign}

\section{Digitale Bildverarbeitung} % (fold)
\label{sec:bildverarbeitung}
\begin{comment}
	Bildverarbeitung: Notwendige Verfahren und Konzepte erläutern.
\end{comment}

\begin{comment}
	YUV und RGB bzw. RGBA
	kCVPixelFormatType_420YpCbCr8BiPlanarVideoRange, kCVPixelFormatType_420YpCbCr8BiPlanarFullRange and kCVPixelFormatType_32BGRA, except on iPhone 3G, where the supported pixel formats are kCVPixelFormatType_422YpCbCr8 and kCVPixelFormatType_32BGRA..(AVCaptureVideoDataOutput Class Reference)

	\subsection{Grundlagen} % (fold)
	\label{sec:grundlagen}
	Ein digitales Bild ist definiert durch seine Bildgröße und der Auflösung. Die Bildgröße ist in Höhe und Breite angegeben und eine entsprechende Bildmatrix kann durch die Bildspalten $u$ und Bildzeilen $v$ angegeben werden. Die Auflösung eines Bildes bezeichnet die räumliche Ausdehnung in \gls{dpi}, die in den meisten Bildverarbeitungsschritten vernachlässigt werden kann.

Um auf ein \gls{pixel} aus der Bildmatrix zugreifen zu können, benötigen wir ein Koordinatensystem. Bei digitalen
 Bildern unterscheidet sich das verwendete Koordinatensystem von einem kartesischen Koordinatensystem dadurch, dass der
 Ursprungspunkt bei Bildern links oben liegt. Die \(x\)-Achse verläuft von links nach rechts und die \(y\)-Achse von
 oben nach unten. TODO: vgl Abbildung.

	Die Information eines \gls{pixel} ist als binärer Wert mit der Länge \(k\) gespeichert. Der Wertebereich eines \gls{pixel} umfasst \(\left[0 \dotsc 2^k\right]\), wobei der genaue Wertebereich abhängig vom eingesetzten Typ ist.

	Man unterscheidet im Allgemeinen zwischen Farb-, Monochrom- und Binärbildern, die eine direkte Auswirkung auf den Wertebereich haben. Bei Farbbildern wird häufig eine Komponente für Rot, Grün und Blau verwendet und ist typischerweise in 8 Bits kodiert. Ein Pixel besteht somit aus \(3 \cdot 8 = 24\) Bits mit einem Wertebereich von \(\left[0 \dotsc 255\right]\) pro Farbkomponente.\\TODO: Bild mit Speicherbeschreibung für RGB

	Monochrombilder bestehen nur aus einem Intensitätskanal der ebenfalls mit 8 Bits kodiert wird. Der Wertebereich eines Pixel entspricht \(\left[0 \dotsc 255\right]\).\\TODO: Bild mit Speicherbeschreibung für Monochrom

	Bei Binärbildern werden Informationen nur in einem Bit gespeichert und der Wert entspricht somit \(0\) oder \(1\) für Schwarz oder Weiß.\\TODO: Bild mit Speicherbeschreibung für Binär.
	% section grundlagen (end)
\end{comment}

Bevor ich mich den \gls{AR}-Verfahren widme, möchte ich die grundlegenden Aspekte der digitalen Bildverarbeitung benennen, die in den nachfolgenden Kapiteln benötigt werden. Dabei werde ich keinen kompletten Überblick über Digitale Bildverarbeitung vermitteln, sondern nur Bereiche besprechen die für die Untersuchung der \gls{AR} berücksichtigt werden müssen.


\subsection{Grundlagen} % (fold)
\label{sub:grundlagen}

Ein digitales Bild ist eine Ansammlung numerischer Werte, die in einem Array gespeichert sind. Durch die Höhe und Breite des Bildes kann das Array, wie in \autoref{fig:bildmatrix}, als Matrix dargestellt werden. Die Höhe $N$ entspricht dann den Zeilen und die Breite $M$ den Spalten der Matrix. Ein Bild $I$ ist somit definiert als eine Funktion auf einer Menge von Bildwerten $\mathbb{P}$, wie in \autoref{eq:digitalesBild} dargestellt ist.

\begin{figure}
	\centering
	\input{resources/Bildmatrix.pdf_tex}
	\caption{Digitales Bild in Matrixdarstellung.}
	\label{fig:bildmatrix}
\end{figure}

\begin{equation}
	I\left(u,v\right)\in\mathbb{P} \text{ und } u,v\in\mathbb{N}
	\label{eq:digitalesBild}
\end{equation}

Um auf ein \gls{pixel} aus der Bildmatrix zugreifen zu können, bedienen wir uns eines Koordinatensystems. Im Unterschied zu einem kartesischen Koordinatensystem ist bei Bildern der Ursprungspunkt links oben. Die \(x\)-Achse verläuft von links nach rechts und die \(y\)-Achse von oben nach unten. \begin{comment}\\TODO:vgl abbildung\end{comment} Welche Werte in einem \gls{pixel} gespeichert sind und wie diese Werte zu interpretieren sind, ist abhängig vom verwendeten Bildtyp.

% section grundlagen (end)

\subsection{Bildtypen} % (fold)
\label{sub:bildtypen}

Der Wert eines \gls{pixel} ist als binärer Wert der Länge $k$ angegeben. Darüber hinaus bestimmt der verwendete Bildtyp
 über die weiteren Informationen eines \gls{pixel}.

Monochrome Bilder, die im allgemeinen Sprachgebrauch als Schwarzweiß-Bilder bezeichnet werden, besitzen nur eine
 Komponente, die als Luminanz oder auch Intensität bezeichnet wird. Die Information des \gls{pixel} wird im Allgemeinen
 mit 8 Bit kodiert, so dass sich ein Wertebereich von $2^8 = \left[0\dotsc255\right]$ ergibt.

Binäre Bilder sind eine spezielle Form der monochromen Bilder. In binären Bildern werden nur Schwarzweiß-Werte
 gespeichert. Die Information kann dadurch in nur einem Bit, mit $0$ für Schwarz und $1$ für Weiß, gespeichert werden.

Farbbilder speichern im Unterschied zu monochromen Bildern mehr Informationen in einem \gls{pixel} ab. Der Aufbau der
Informationen von Farbbildern ist abhängig von dem eingesetzten Farbraum, der in \autoref{sub:farbräume} erläutert
wird. In den meisten Fällen wird die Information eines Farbbildes in drei Komponenten für Rot, Grün und Blau, mit
jeweils 8 Bit, kodiert. Somit besteht ein Pixel aus \(3 \cdot 8 = 24\) Bits mit einem Wertebereich von
\(\left[0 \dotsc 255\right]\) pro Farbkomponente.

Der Aufbau der Farbinformationen ist abhängig von der Anordnung, die in \autoref{sub:pixelanordnung} genauer definiert
 wird.

% subsection bildtypen (end)

\subsection{Farbräume} % (fold)
\label{sub:farbräume}

Bei digitaler Bildverarbeitung gibt es unterschiedliche Farbräume für verschiedene Einsatzgebiete. Für diese
 Untersuchung werde ich den RGB-Farbraum, sowie YUV und YCbCr, vorstellen.

\subsubsection{RGB} % (fold)
\label{sec:rgb}

Der RGB-Farbraum ist in in der digitalen Bildverarbeitung der am häufigsten eingesetzte Farbraum und besteht aus roten,
 grünen und blauen Komponenten (\autoref{fig:rgbLenna}).

\begin{figure}[!ht]
	\centering
	\subfigure[]{
		\label{fig:rgbLenna-Farbe}
		\includegraphics[width=.2\textwidth]{resources/Lenna.pdf}
	}
	\subfigure[]{
		\label{fig:rgbLenna-Rot}
		\includegraphics[width=.2\textwidth]{resources/Lenna_R.pdf}
	}
	\subfigure[]{
		\label{fig:rgbLenna-Gruen}
		\includegraphics[width=.2\textwidth]{resources/Lenna_G.pdf}
	}
	\subfigure[]{
		\label{fig:rgbLenna-Blau}
		\includegraphics[width=.2\textwidth]{resources/Lenna_B.pdf}
	}
	\caption{Ein farbiges Bild \subref{fig:rgbLenna-Farbe} wird im RGB-Farbraum in seine Komponenten Rot
		 \subref{fig:rgbLenna-Rot}, Grün \subref{fig:rgbLenna-Gruen} und Blau \subref{fig:rgbLenna-Blau} aufgeteilt.
		Von \cite{lenna}.
	}
	\label{fig:rgbLenna}
\end{figure}

Bei RGB handelt es sich um ein additives Farbsystem, dass durch Mischung der Primärfarben einen Farbton und dessen
 Helligkeit generiert\footcite[Vgl.][S.~233]{burger05}. Der RGB Farbraum kann als dreidimensionaler Würfel dargestellt
 werden (\autoref{fig:rgbWuerfel}). Seine Koordinatenachsen stellen den Wertebereich für die Grundfarben Rot, Grün und
 Blau dar. Normiert man den Würfel zu einem Einheitswürfel der Länge $1$, umfasst jede Achse einen Bereich von
 \(\left[0 \dotsc 1.0\right]\). Innerhalb dieses Würfels kann durch Verschiebung der Koordinate jeder Farbton generiert
 werden. Die Grundfarben können in diesem Würfel durch die Punkte $R = \left(1,0,0\right)$, $G = \left(0,1,0\right)$
 und $B = \left(0,0,1\right)$ dargestellt werden. Schwarz ($S = \left(0,0,0\right)$) und Weiß
 ($W = \left(1,1,1\right)$) werden auf die gleiche Weise dargestellt.

\begin{figure}[!ht]
	\centering
	\def\svgwidth{.2\columnwidth}
	\input{resources/RGBWuerfel.pdf_tex}
	\caption{RGB-Farbwürfel}
	\label{fig:rgbWuerfel}
\end{figure}

\begin{table}[!ht]
	\begin{center}
	\begin{tabular}[]{r|c|c|c}
	Farbe & R & G & B \\ \hline\hline
	rot & 1.0 & 0.0 & 0.0 \\
	grün & 0.0 & 1.0 & 0.0 \\
	blau & 0.0 & 0.0 & 1.0 \\
	schwarz & 0.0 & 0.0 & 0.0 \\
	weiß & 1.0 & 1.0 & 1.0 \\
	grau & 0.5 & 0.5 & 0.5 \\
	\end{tabular}
	\caption{RGB-Werte}
 	\label{tbl:rgbwerte}
	\end{center}
\end{table}

% subsubsection rgb (end)

\subsubsection{YUV und YCbCr} % (fold)
\label{sec:yuv_und_ycbcr}

YUV und YCbCr sind standardisierte Formate zur Aufnahme, Darstellung und Übertragung von Bildern im Fehrnsehbereich und
stellen andere Farbräume dar. YUV findet Verwendung bei analogen PAL und NTSC Systemen während YCbCr bei digitalen
Systemen verwendet wird. Beide Farbräume unterteilen die Informationen in eine Luminanzkomponente $Y$ und zwei
Komponenten zur Darstellung unterschiedlicher \gls{chrominanz} (\autoref{fig:yuvLenna}).

\begin{figure}[!ht]
	\centering
	\subfigure[]{
		\label{fig:yuvLenna-Farbe}
		\includegraphics[width=.2\textwidth]{resources/Lenna.pdf}
	}
	\subfigure[]{
		\label{fig:yuvLenna-Y}
		\includegraphics[width=.2\textwidth]{resources/Lenna_Y.pdf}
	}
	\subfigure[]{
		\label{fig:yuvLenna-U}
		\includegraphics[width=.2\textwidth]{resources/Lenna_U.pdf}
	}
	\subfigure[]{
		\label{fig:yuvLenna-V}
		\includegraphics[width=.2\textwidth]{resources/Lenna_V.pdf}
	}
	\caption{Im YUV-Farbraum wird ein Bild \subref{fig:yuvLenna-Farbe} in Luminanz $Y$ \subref{fig:yuvLenna-Y},
		Chroma $U$ \subref{fig:yuvLenna-U} und Chroma $V$ \subref{fig:yuvLenna-Y} aufgeteilt. Von
		 \cite{lenna}.
	}
	\label{fig:yuvLenna}
\end{figure}

Bei YUV ist die Chromakomponente $U$ als Differenz zwischen der Luminanz $Y$ und dem Blauanteil definiert (Vgl.~\autoref{eq:uchroma}).

\begin{equation}
	U = 0.492 \cdot \left(B-Y\right)
	\label{eq:uchroma}
\end{equation}

Die Chromakomponente $V$ definiert die Differenz von Luminanz $Y$ und dem Rotanteil (Vgl.~\autoref{eq:vchroma}).

\begin{equation}
	V = 0.877 \cdot \left(R-Y\right)
	\label{eq:vchroma}
\end{equation}

Die Chromakomponenten $Cb$ und $Cr$ von YCbCr sind, wie bei YUV, Differenzwerte des Blau- und Rotanteil und der
 Luminanz. Der Unterschied zwischen YUV und YCbCr besteht in der unterschiedlichen Berechnung der \gls{chrominanz} und
 ist in \citeauthor{burger05}\footcite[][S.265--266]{burger05} beschrieben.

Durch die Trennung der Luminanz und \gls{chrominanz} ist es möglich das Signal auf Schwarzweiß-Fernsehern zu nutzen,
 indem nur die Luminanzkomponente berücksichtigt wird. Bei Farbfernsehern werden zusätzlich die Chromakomponenten
 verwendet.

% subsubsection yuv_und_ycbcr (end)

% subsection farbräume (end)

\subsection{Pixel-Anordnung} % (fold)
\label{sub:pixelanordnung}
Bei Pixeln gibt es unterschiedliche Modelle der Anordnung von Farbkomponenten. Man unterscheidet die
 Komponentenanordnung, auch planare Anordnung genannt, von der gepackten Anordnung. Der Zugriff auf die
 Farbinformationen ist dabei abhängig vom eingesetzten Farbraum. Im weiteren Verlauf wird die gepackte Anordnung für den
 RGB Farbraum erläutert. Für YCbCr sind die vorgestellten Modelle anzupassen.

Bei der planaren Anordnung werden die Farbkomponenten in jeweils eigenen Arrays mit gleicher Größe gespeichert. Ein Bild

\begin{equation}
	I = \left(I_R, I_G, I_B\right)
	\label{eq:planarImage}
\end{equation}

besteht aus den drei Luminanzbildern $I_R$, $I_G$ und $I_B$ (\autoref{fig:planareAnordnung}). Der Zugriff auf ein
 \gls{pixel} erfolgt über das Auslesen aller drei Arrays, wie in
 \citeauthor{burger05}\footcite[Vgl.][S.~235--236]{burger05} beschrieben. Der Zugriff erfolgt durch

\begin{equation}
	\begin{pmatrix}
		R\\
		G\\
		B
	\end{pmatrix}
	\leftarrow
	\begin{pmatrix}
		I_R\left(u,v\right)\\
		I_G\left(u,v\right)\\
		I_B\left(u,v\right)
	\end{pmatrix}.
	\label{eq:readPlanarImage}
\end{equation}

\begin{figure}[!ht]
	\centering
	\input{resources/planareAnordnung.pdf_tex}
	\caption{Planare Anordnung. Information der Pixel sind in den Matritzen $I_R$, $I_G$ und $I_B$ hinterlegt.}
	\label{fig:planareAnordnung}
\end{figure}

Bei der gepackten Anordnung sind alle Farbkomponenten in einem \gls{pixel} gespeichert und in einem Array hinterlegt (\autoref{fig:gepackteAnordnung}). Die gepackte Anordnung ist mit

\begin{equation}
	I\left(u,v\right) = \left(R,G,B\right)
	\label{eq:packedImage}
\end{equation}

definiert.

\begin{figure}[!ht]
	\centering
	\input{resources/gepackteAnordnung.pdf_tex}
	\caption{Die gepackte Anordnung speichert die Farbinformationen eines Pixels an einer Stelle.}
	\label{fig:gepackteAnordnung}
\end{figure}

Im Allgemeinen wird ein Element an Stelle $(u,v)$ eines Bildes $I$ durch \autoref{eq:readPackedRGB} zugegriffen.

\begin{equation}
	\begin{pmatrix}
		R\\
		G\\
		B
	\end{pmatrix}
	\leftarrow
	\begin{pmatrix}
		Red\bigl(I\left(u,v\right)\bigr)\\
		Green\bigl(I\left(u,v\right)\bigr)\\
		Blue\bigl(I\left(u,v\right)\bigr)
	\end{pmatrix}
	\label{eq:readPackedRGB}
\end{equation}

$Red()$, $Green()$ und $Blue()$ sind abhängig vom eingesetzten Format\footcite[Vgl.][S.~236--237]{burger05}.
% subsection pixelanordnung (end)

\subsection{Filter und Faltung} % (fold)
\label{sub:filter}
Ein Filter ist eine Operation, mit dessen Hilfe ein Eingangsbild $I$ durch eine mathematische Abbildung in ein
 Ausgangsbild $I'$ überführt wird. Im Gegensatz zu Punktoperationen operieren Filter auf Regionen, um zum Beispiel
 Bilder zu glätten oder zu schärfen (Vgl.~\autoref{fig:filterbeispiel}).

\begin{figure}[!ht]
	\centering
	\subfigure[]{
		\label{fig:filter-original}
		\includegraphics[scale=1]{resources/Filter-Original.pdf}
	}
	\subfigure[]{
		\label{fig:filter-verwischen}
		\includegraphics[scale=1]{resources/Filter-Verwischen.pdf}
	}
	\subfigure[]{
		\label{fig:filter-schaerfen}
		\includegraphics[scale=1]{resources/Filter-Schaerfen.pdf}
	}
	\caption{Das orignial Bild \subref{fig:filter-original} wurde in
	 \subref{fig:filter-verwischen} verwischt und in \subref{fig:filter-schaerfen} geschärft.}
	\label{fig:filterbeispiel}
\end{figure}

Aus einer Region $R_{u,v}$ eines Eingangsbild $I$ wird der neue Pixelwert $I'(u,v)$ berechnet. Die Größe der
 Filterregion bestimmt die Anzahl der Pixel aus $I$, die zur Berechnung des neuen Pixelwerts $I'(u,v)$ verwendet
 werden (Vgl.~\autoref{fig:region-filter}). Üblicherweise werden $3 \times 3$ oder $5 \times 5$ Filter verwendet, aber
 auch größere Filter mit $21 \times 21$ Pixeln sind möglich. Anhand der Schreibweise der Filter erkennt man, das es
 sich bei den Filtern um Matrizen handelt.

\begin{figure}[!ht]
	\centering
	\input{resources/Region-Filter.pdf_tex}
	\caption{Anwendung eines Filters. Aus der Region $R_{u,v}$ wird der neue Wert $I'(u,v)$ berechnet.}
	\label{fig:region-filter}
\end{figure}

Bei einer Filtermatrix wird ein eigenes Koordinatensystem verwendet, dessen Ursprung in der Mitte der Matrix liegt.
 Aufgrund dieses Koordinatensystem sind die Koordinaten sowohl positiv als auch negativ. Bei einer $3 \times 3$
 Filtermatrix $H$ sähen die absoluten Koordinaten an der Stelle $(i,j)$ wie folgt aus:

\begin{equation}
	H(i,j) =
	\begin{pmatrix}
		\left(i-1, j-1\right)&	\left(i, j-1\right)&	\left(i+1, j-1\right)\\
		\left(i-1, j\right)& 	\left(i, j\right)&		\left(i+1, j\right)\\
		\left(i-1, j+1\right)&	\left(i, j+1\right)&	\left(i+1, j+1\right)
	\end{pmatrix}
\end{equation}

Somit können alle Pixel von $I'$ durch

\begin{equation}
	I'\left(u,v\right) \gets
	\sum \limits_{\left(i = -1\right)}^{i = 1}
	\sum \limits_{\left(j = -1\right)}^{j = 1}
	I\left(u + i, v + j\right) \cdot H\left(i,j\right)
\end{equation}

berechnet werden. Für Filter mit einer anderen Größe als $3 \times 3$, lautet die Formel

\begin{equation}
	I'\left(u,v\right) \gets
	\sum_{\left(i,j\right)\in\mathbb{R}} I\left(u + i, v + j\right) \cdot H\left(i,j\right)
\end{equation}

wobei $\mathbb{R}$ die Größe des Filters angibt\footcite[Vgl.][S.~92--93]{burger05}.

Filter können an den Rändern nicht mathematisch korrekt berechnet werden\footcite[Vgl.][S.~113]{burger05}, da der
 Filter $H$ an den Bildrändern über die definierte Bildmenge hinausragt. Um dieses Problem zu umgehen, gibt es die
 Möglichkeit

\begin{enumerate}
	\item am Randbereich einen konstanten Wert zuzuweisen,\label{konstant}
	\item den ursprünglichen Bildwert beizubehalten,\label{nix}
	\item die Randwerte zu berechnen, indem
	\begin{enumerate}
		\item außerhalb des Bildbereichs konstante Werte angenommen werden,\label{berechneKonstant}
		\item die Randpixel fortgesetzt werden,\label{berechneRand}
		\item die Bildwerte wiederholt werden.\label{berechneBild}
	\end{enumerate}
\end{enumerate}

Bei \autoref{konstant} wird der sichtbare Bildbereich verkleinert und bei \autoref{nix} der Filter an den Rändern nicht
 angewendet. Bei \autoref{berechneKonstant} kann die Zuweisung eines konstanten Wertes, beispielsweise Schwarz, das
 Ergebnis verfälschen. \autoref{berechneRand} weist die geringste Verfälschung auf. Bei \autoref{berechneBild} wird
 durch die Wiederholung des Bildes eine periodische Funktion betrachtet. Die möglichen Randbehandlungen sind in
 \autoref{fig:randbeispiel} illustriert. Welche dieser Randbetrachtung eingesetzt wird ist abhängig von den
 benötigten Ergebnissen und Filtern und muss für jede Anwendung gesondert betrachtet werden.

\begin{figure}[!ht]
	\centering
	\subfigure[]{
		\label{fig:randbeispiel-original}
		\includegraphics[scale=1]{resources/Rand-Original.pdf}
	}
	\subfigure[]{
		\label{fig:randbeispiel-1}
		\includegraphics[scale=1]{resources/Rand-1.pdf}
	}
	\subfigure[]{
		\label{fig:randbeispiel-2}
		\includegraphics[scale=1]{resources/Rand-2.pdf}
	}
	\subfigure[]{
		\label{fig:randbeispiel-3a}
		\includegraphics[scale=1]{resources/Rand-3a.pdf}
	}
	\subfigure[]{
		\label{fig:randbeispiel-3b}
		\includegraphics[scale=1]{resources/Rand-3b.pdf}
	}
	\subfigure[]{
		\label{fig:randbeispiel-3c}
		\includegraphics[scale=1]{resources/Rand-3c.pdf}
	}
	\caption{Filter und Randbehandlung. Das Originalbild \subref{fig:randbeispiel-original} wird je nach eingesetzen
	 Verfahren unterschiedlich behandelt. Bei \subref{fig:randbeispiel-1} werden die Ränder nicht betrachtet und ein
	 konstanter Wert zugewiesen (hier schwarz dargestellt).	Die Originalwerte des Eingangsbilds werden bei
	 \subref{fig:randbeispiel-2} beibehalten. In \subref{fig:randbeispiel-3a} werden ausserhalb des Bildes konstante
	 Werte verwendet (hier schwarz dargestellt). Die Randpixel werden bei \subref{fig:randbeispiel-3b} fortgesetzt,
	 während bei \subref{fig:randbeispiel-3c} das Eingangsbild an den Rändern wiederholt wird.}
	\label{fig:randbeispiel}
\end{figure}

In der digitalen Bildverarbeitung basieren Filter auf der mathematischen Operation der Faltung
\footcite[Vgl.][S.~101--104]{burger05}, bei der für zwei Funktionen ein dritte Funktion erzeugt wird. Die diskrete
 lineare Faltung ist definiert als

\begin{equation}
	I'\left(u,v\right) =
	\sum \limits_{i = -\infty}^{\infty}
	\sum \limits_{j = -\infty}^{\infty}
	I\left(u - i, v - j\right) \cdot H\left(i,j\right)
\end{equation}

und gekürzt

\begin{equation}
	I' = I * H,
\end{equation}

wobei $I$ dem Bildsignal und $H$ dem Faltungskern entspricht. Die Operation der Faltung ist die Grundlage aller
 linearen Filter in der digitalen Bildverarbeitung. Unterschiedliche Filter sind nur durch den Filterkern (Kernel)
 definiert. Die Faltungsoperation besitzt die algebraischen Eigenschaften der Kommutativität, Assoziativität und
 Distributivität.

\begin{align}
	&\text{Kommutativität: } &I * H = H * I\\
	&\text{Assoziativität: } &A * (B * C) = (A * B) * C\\
	&\text{Distributivität: } &H * \left(I_1 + I_2\right) = \left(H * I_1\right) + \left(H * I_2\right)
\end{align}

Durch die Assoziativität kann ein Kernel $H$ als Produkt mehrerer kleiner Kernel ausgedrückt werden was wiederum
 Rechenoperationen einspart. Ein $3 \times 3$ Filter kann in eine $x$-Richtung und $y$-Richtung aufgeteilt werden.

\begin{align}
	H_{xy} = & H_x * H_y\\
	\begin{pmatrix}
		1& 1& 1\\
		1& 1& 1\\
		1& 1& 1
	\end{pmatrix} = &
	\begin{pmatrix}
		1& 1& 1
	\end{pmatrix}
	*
	\begin{pmatrix}
		1\\
		1\\
		1
	\end{pmatrix}
\end{align}

Bei der Berechnung von $I * H_{xy}$ entspricht dies $3 \cdot 3 = 9$ Multiplikationen und bei der Berechnung von
 $I * H_x * H_y = 3 + 3 = 6$ Multiplikationen.

% subsection filter (end)

\subsection{Regionen in Binärbilder} % (fold)
\label{sec:regionen_in_binärbilder}

Binärbilder enthalten, wie in \autoref{sub:bildtypen} bereits beschrieben, nur zwei Werte, die wir auch als Vordergrund
 und Hintergrund bezeichnen können. Das Interesse bei einem Binärbild gilt dementsprechend den Informationen im
 Vordergrund. Eine zusammenhängende Struktur von Vordergrundpixeln wird als Bildregion bezeichnet. Um diese
 Bildregionen in einem Binärbild zu erkennen, werden zusammenhängende \gls{pixel} makiert, was auch als
 \textit{region labeling}\footcite[Vgl.][S.~196]{burger05} bekannt ist. Zur Makierung von \gls{pixel} werden folgende
 numerischen Werte verwendet:

\begin{equation*}
	I(u,v) = \begin{cases}
	0 & \textrm{Hintergrund}\\
	1 & \textrm{Vordergrund}\\
	2,3,\ldots,n & \textrm{Regionenmakierung}
	\end{cases}
\end{equation*}

Bei der sequentiellen Regionenmakierung\footcite[Vgl.][S.~200--206]{burger05} wird ein Binärbild in zwei Schritten
 untersucht. Im ersten Schritt werden im Binärbild vorläufige Markierungen für Regionen gespeichert und im zweiten
 Schritt werden mehrfache Makierungen für eine Region aufgelöst.

\subsubsection{Vorläufige Makierung} % (fold)
\label{sec:vorläufige_makierung}

Zuerst muss das Binärbild zeilenweise von oben nach unten auf das Vorhandensein von Vordergrundpixel untersucht werden.
Die Nachbarn eines angrenzenden \glspl{pixel} werden, je nach Definition der Nachbarschaftsbeziehung, ebenfalls
 untersucht. Zwei gebräuchliche Definitonen von Nachbarschaftsbeziehungen sind zum einen die 4er-Nachbarschaft und zum
 anderen die 8er-Nachbarschaft. Für ein \gls{pixel} $I(u,v)$ werden bei der 4er-Nachbarschaft die beiden angrenzenden
 \gls{pixel} $I(u-1,v)$ und $I(u,v+1)$ untersucht. Bei der 8er-Nachbarschaft werden die vier Nachbarn $I(u-1,v)$,
 $I(u-1,v+1)$, $I(u,v+1)$ und $I(u+1,v+1)$ untersucht. Beide Nachbarschaftsbeziehungen sind in
 \autoref{fig:nachbarschaft} abgebildet.

\begin{figure}[!ht]
	\centering
	\subfigure[]{
		\label{fig:4er-nachbarschaft}
		\input{resources/4er-Nachbarschaft.pdf_tex}
	}
	\subfigure[]{
		\label{fig:8er-nachbarschaft}
		\input{resources/8er-Nachbarschaft.pdf_tex}
	}
	\caption{Nachbarschaftsbeziehung. \subref{fig:4er-nachbarschaft} 4er-Nachbarschaft mit $N_1 = I(u-1,v)$ und
	 $N_2 = I(u,v+1)$. \subref{fig:8er-nachbarschaft} 8er-Nachbarschaft mit $N_1 = I(u-1,v)$, $N_2 = I(u-1,v+1)$,
	 $N_3 = (u,v+1)$ und $N_4 = I(u+1,v+1)$.}
	\label{fig:nachbarschaft}
\end{figure}

Die Zuweisung einer Markierung für einen \gls{pixel} ist davon abhängig ob

\begin{enumerate}
	\item alle Nachbarn Hintergrundpixel sind, \label{labeling-all-background}
	\item genau ein Nachbar hat eine Makierung hat oder \label{labeling-one-neighbour}
	\item mehrere Nachbarn eine Makierung haben. \label{labeling-many-neighbours}
\end{enumerate}

Wenn alle Nachbarn Hintergrundpixel sind (\autoref{labeling-all-background}), wird eine Makierung an Position $(u,v)$
 geschrieben. Bei \autoref{labeling-one-neighbour} wird die Makierung des Nachbarn übernommen. Im letzten Fall müssen
 die Makierungen miteinander verglichen werden. Wenn alle Nachbarn die gleiche Makierung besitzen wird sie für Position
 $(u,v)$ übernommen. Falls es sich aber um verschiedene Makierungen handelt, spricht man von einer Kollision der
 Makierungen. Eine Kollision bedeutet, dass eine zusammenhängende Regionen durch zwei unterschiedliche Makierungen
 dargestellt wird (Vgl.~\autoref{fig:kollision}). Dem \gls{pixel} an Position $(u,v)$ wird eine Makierung eines Nachbarn
 zugewiesen und die Kollision wird vermerkt. Nach \citeauthor{burger05} werden diese Kollisionen in einer dynamischen
 Datenstruktur gespeichert und zu einem späteren Zeitpunkt bearbeitet\footcite[Vgl.][S.~203--204]{burger05}.

\begin{figure}[!ht]
	\centering
	\input{resources/Kollision.pdf_tex}
	\caption{Beispiel einer Kollision zwischen Markierung $2$ und Markierung $3$.}
	\label{fig:kollision}
\end{figure}

Nach diesem ersten Schritt ist das Binärbild verarbeitet und allen Vordergrundpixeln wurde eine vorläufige Markierung
 zugeteilt. Dabei wurden auftretende Kollisionen der Markierungen gespeichert. Der Ablauf des ersten Schritts ist in
 \autoref{fig:markierung} dargestellt.

\begin{figure}[!ht]
	\centering
	\subfigure[]{
		\label{fig:markierung-binaer}
		\input{resources/Binaerbild.pdf_tex}
	}
	\subfigure[]{
		\label{fig:markierung-1}
		\input{resources/Regionenmarkierung-1.pdf_tex}
	}
	\subfigure[]{
		\label{fig:markierung-2}
		\input{resources/Regionenmarkierung-2.pdf_tex}
	}
	\subfigure[]{
		\label{fig:markierung-3}
		\input{resources/Regionenmarkierung-3.pdf_tex}
	}
	\caption{Vorläufige Regionenmarkierung. Ein Binärbild \subref{fig:markierung-binaer} wird zeilenweise von
	 oben nach unten durchlaufen. Eine 8er-Nachbarschaft vergleicht die angrenzenden Pixel \subref{fig:markierung-1}.
	 Eine Kollision zwischen zwei Markierungen wird registriert und der kleinere Wert zugewiesen bis das Binärbild
	 vollständig verarbeitet wurde \subref{fig:markierung-2}--\subref{fig:markierung-3}.}
	\label{fig:markierung}
\end{figure}

% subsubsectionsection vorläufige_makierung (end)

\subsubsection{Auflösung von Kollisionen} % (fold)
\label{sec:auflösung_von_kollisionen}

In diesem Schritt müssen nun die gespeicherten Kollisionen der Markierungen aufgelöst werden, damit eine
 zusammenhängende Bildregion nur durch eine Markierung repräsentiert wird. \citeauthor{burger05} beschreiben diese
 Aufgabe  als nicht trivial, da kollidierende Regionen auch durch weitere Region zusammenhängen
 können\footcite[Vgl.][S.~205]{burger05} (Vgl.~\autoref{fig:markierung-3}).

Die Menge der im letzten Schritt verwendeten vorläufigen Makierungen wird verwendet, um die Menge der Kollisionen
 zu vereinen. Dazu wird eine Kollision der Markierung von $a$ und $b$ zu einer Menge zusammengeführt. Das bedeutet,
 dass alle \gls{pixel} mit Markierung $b$ zu $a$ gehören. Danach wird das vorläufig markierte Binärbild erneut
 durchlaufen. Jede vorläufige Makierung wird nun mit der Menge der neuen Markierung verglichen und eine eindeutige
 Markierung zugewiesen.

Betrachten wir \autoref{alg:resolve-label-collision} an folgendem Beispiel: In Zeile
 \ref{resolve-label-collsion-partition-start}--\ref{resolve-label-collsion-partition-end} wird die Menge $L$ der
 vorläufigen Markierungen aufgeteilt und Kollisionen aufgelöst. Nehmen wir an, dass wir eine Kollision $C_i = (2,3)$
 auflösen wollen, wird die Mengen $R_2 = \{2\}$ und $R_3 = \{3\}$  zu $R_2 = \{2,3\}$ vereint
 (Zeile \ref{resolve-label-collsion-set-start}--\ref{resolve-label-collsion-set-end}). Die Menge $R_3$ ist danach leer.
 Nun werden in Zeile \ref{resolve-label-collsion-relabel-start}--\ref{resolve-label-collsion-relabel-end} alle
 Makierungen untersucht. Wird dabei eine Makierung an Position $(u,v)$ gefunden, die zur Menge $R_2 = \{2,3\}$ gehört,
 wird der kleinste Wert von $R_2$ an Position $(u,v)$ geschrieben (Zeile \ref{resolve-label-collsion-set-new-label}).

\begin{algorithm}[!ht]\small
\caption{Kollisionen auflösen}
\label{alg:resolve-label-collision}
\begin{algorithmic}[1]
	\Require $I_b,L,C$
	\State $L = \{2,3,\ldots,n\}$ sind die vorläufigen Markierungen
	\State $R \gets [\{2\},\{3\},\ldots,\{n\}]$, sodass $R_i = {i}$ für alle $\{i\} \in L$
	\For{alle Kollisionen $(a,b)$ in $C$}
	\label{resolve-label-collsion-partition-start}
		\State $R_a \gets$ die Menge mit Markierung $a$
		\State $R_b \gets$ die Menge mit Markierung $b$
		\If{$R_a \neq R_b$}
		\label{resolve-label-collsion-set-start}
			\State $R_a \gets R_a \cup R_b$
			\State $R_b \gets \{\}$
		\EndIf
		\label{resolve-label-collsion-set-end}
	\EndFor
	\label{resolve-label-collsion-partition-end}
	\For{alle Pixel $(u,v)$ in $I_b$}
	\label{resolve-label-collsion-relabel-start}
		\If{$I_b(u,v) > 1$}
			\State Suche in $R$ die Menge $R_i$ die Markierung $I_b(u,v)$ enthält
			\State Ersetze Markierung in $I_b(u,v)$ mit $\textrm{min}(R_i)$
			\label{resolve-label-collsion-set-new-label}
		\EndIf
	\EndFor
	\label{resolve-label-collsion-relabel-end}
\end{algorithmic}
\end{algorithm}


Nach diesem letzten Schritt sind alle zusammenhängende Regionen in einem Binärbild eindeutig gekennzeichnet und können
 nun mit anderen Verfahren weiterverarbeitet werden, um beispielsweise die Form einer Region zu erkennen.

% subsubsectionsection auflösung_von_kollisionen (end)

% subsection regionen_in_binärbilder (end)

% section bildverarbeitung (end)

\section{Untersuchungsgegenstand: Tracking Verfahren und Tracking Algorithmen} % (fold)
\label{sec:untersuchungsgegenstand}
\begin{comment}
	Untersuchungsgegenstand: Verfahren und Algorithmen präzise vorstellen und ihre Unterschiede hervorheben.
	Notwendige Kriterien der Algorithmen bestimmen

	Grober Ablauf der Verfahren:
	* Wer hats erfunden?
	* Wie ist das Verfahren aufgebaut (Algo in grob)
	* Welche Kriterien müssen erfüllt sein (monochrom, rgb eingabe)?
\end{comment}

\subsection{Verfahren nach Hirzer} % (fold)
\label{sub:verfahren_nach_hirzer}

Der Algorithmus von \citeauthor{hirzer08}\footcite{hirzer08} ist nach dem Vorbild der \textit{pixel connectivity edge
 linking based algorithms} entworfen und ist in drei Hauptteile aufgebaut. Zuerst werden Liniensegmente erstellt,
 indem \glspl{edgel} auf einem Suchraster gefunden und zusammengeführt werden. Die kurzen Liniensegmente werden dann zu
 längeren Linien zusammengeführt. Anschließend werden im zweiten Schritt alle gefundenen Linien erweitert um die
 Gesamtlänge einer Linie zu erhalten. Im letzten Schritt werden die Linien zu Vierecken verbunden. Im weiteren Verlauf
 werden diese Schritte als Line Detection, Line Extension und Quadrangle Detection bezeichnet.

\subsubsection{Line Detection} % (fold)
\label{sub:line_detection}
Die Linienerkennung basiert auf dem Verfahren von \citeauthor{clarke96}\footcite{clarke96} und besteht aus zwei
 Schritten. Im ersten Schritt wird das Bildsignal grob abgetastet um im zweiten Schritt durch das RANSAC Verfahren eine
 Linienhypothese zu erstellen und zu bewerten.

Im ersten Schritt wird zuerst das monochrome Eingabesignal $I_m$ in $40 \times 40$ \gls{pixel} große Regionen
 unterteilt. Jede nachfolgende Operation erfolgt innerhalb einer Region. Eine Region wird wiederum unterteilt in
 horizontale und vertikale Scanlines, die jeweils $5$ \gls{pixel} Abstand zueinander haben. Jedes \gls{pixel} auf den
 Scanlines wird mit einem Gauß-Kernel gefaltet um die Komponente des Gradienten zu bestimmen. Ein lokales Maximum des
 Gradienten, dass größer als ein festgelegter Schwellwert ist, wird als \gls{edgel} betrachtet und seine Orientierung
 berechnet.

Im zweiten Schritt wird das RANSAC Verfahren verwendet, um aus der Menge der \glspl{edgel} Liniensegmente zu bestimmen.
 Eine Linienhypothese wird durch die zufällige Auswahl zweier \glspl{edgel} erstellt, deren Orientierung innerhalb
 eines Grenzwert liegen müssen. Ein \gls{edgel} dient als Startpunkt und das andere \gls{edgel} als Endpunkt der Linie.
 Im Anschluss wird die Anzahl der \glspl{edgel} betrachtet, die in der Nähe dieser Linie liegen und eine kompatible
 Orientierung mit der Linie aufweisen. Diese \glspl{edgel} unterstützen die Hypothese einer Linie im Bildsignal, wenn
 die Anzahl größer ist als die minimal geforderte Anzahl von Unterstützungsedgels. Die zufällige Auswahl zweier
 \glspl{edgel} um eine Linie zu erstellen und deren Edgelunterstützung zu ermitteln wird mehrmals wiederholt um die
 Linie mit der meisten Edgelunterstützung zu finden. Wenn eine solche dominante Linie gefunden wurde, gilt die
 Hypothese als bestätigt und die Linie wird als vorhanden betrachtet. Die Edgels die zur Unterstüztung der Hypothese
 der Linie galten, werden aus der Menge der Edgels entfernt und das Verfahren wird solange wiederholt, bis alle
 Liniensegmente gefunden wurden oder nicht mehr genügend Edgels vorhanden sind.

Das Verfahren ist in \autoref{alg:linedetection-clarke} dargestellt.

\begin{algorithm}[!ht]\small
\caption{Line Detection nach \citeauthor{clarke96}}
\label{alg:linedetection-clarke-overview}
\begin{algorithmic}[1]
	\Require $I_m$
	\State $I_m$ in Regionen von $40 \times 40$ Pixeln unterteilen
		\For{$i \gets 0$, alle Regionen}
			\State $Liste \gets \infty$
			\State $i$ in horizontale und vertikale Scanlines unterteilen mit $5$ Pixeln Abstand
			\For{$j \gets 0$, alle Pixel auf allen Scanlines}
				\State $x \gets$ Falte $I_m\left(j\right)$ mit Gauß-Kernel
				\If{$x > Schwellwert$}
					\Comment Edgel gefunden
					\State Orientierung des Edgels berechnen
					\State $Liste \gets edgel$
				\EndIf
			\EndFor
			\While{$Liste > $ min. Support-Edgel $\land$ Support-Edgel > min. Support-Edgel}
				\State $erkannteLinie \gets 0$
				\For{$j \gets 1, 25$}
					\State $Linie \gets$ zwei zufällige Edgel mit kompatibler Orientierung aus $Liste$ wählen
					\State Anzahl von Support-Edgeln in der Nähe der Linie bestimmen
					\If{$Linie > erkannteLinie$}
						\State $erkannteLinie \gets Linie$
					\EndIf
				\EndFor
				\If{Anzahl von Support-Edgeln > min. Support-Edgel}
					\Comment Linie wurde erkannt
					\State $Liste \gets Liste -$ Support-Edgel der Linie
				\EndIf
			\EndWhile
		\EndFor
	\end{algorithmic}
\end{algorithm}



\citeauthor{hirzer08} hat das Verfahren von \citeauthor{clarke96} abgewandelt, um es zur Erkennung einer bitonalen
 Marke zu nutzen. Dazu verwendet \citeauthor{hirzer08}, anstatt eines monochromen Bildsignals, ein farbiges Bildsignal
 und untersucht zuerst einen der drei Farbkanäle. Wenn ein \gls{edgel} in einem Kanal gefunden wird, werden die
 verbleibenden Kanäle untersucht, um sicherzustellen, dass auch hier ein \gls{edgel} vorliegt. Ist der Gradient in
 allen drei Kanälen höher als der festgelegte Schwellwert, handelt es sich um einen Übergang  von Schwarz nach Weiß.
 Ist dies nicht der Fall, handelt es sich um einen farbigen Übergang und ist somit zur Erkennung einer Marke
 uninteressant. Da ein monochromes Signal wie in \autoref{sub:bildtypen} beschrieben nur ein Kanal besitzt, kann hier
 diese Unterscheidung nicht getroffen werden, was zu einer größeren Anzahl von \glspl{edgel}
 führt\footcite[Vgl.][S.~6--7]{hirzer08}.

Das vorgestellte Verfahren von \citeauthor{clarke96} liefert als Ergebnis nur kurze Liniensegmente. Um eine Kante
 entlang einer Marke zu erkennen, müssen die kurzen Liniensegmente zusammengeführt werden. Dazu werden alle
 Liniensegmente miteinander verglichen um jede Kombinattionsmöglichkeit zu testen. Ob zwei Liniensegmente zu einer
 Linie zusammengeführt werden können, ist von drei Kriterien abhängig. Zuerst müssen zwei Liniensegmente eine
 kompatible Orientierung aufweisen, die nur geringfügig abweichen darf um als Ergebnis eine gerade Linien zu erhalten.
 Als zweites Kriterium muss eine Verbindungslinie zwischen den Liniensegmenten ebenfalls eine kompatible Orientierung
 aufweisen. Dadurch wird sichergestellt, dass keine Liniensegmente zusammengeführt werden, die zwar eine kompatible
 Orientierung besitzen aber parallel zueinander liegen. Als letztes Kriterium wird der Gradient der Verbindungslinie
 Punkt für Punkt untersucht. Dieses Kriterium dient dazu nebeneinanderliegende Marken zu unterscheiden. Würde man dies
 Unterscheidung vernachlässigen, würden mehrere Kanten unterschiedlicher Marken zusammengeführt (Vgl. \autoref{fig:}).

\citeauthor{hirzer08} verwendet keinen Schwellwert um einen minimalen oder maximalen Abstand zwischen zwei
 Liniensegmenten festzulegen. \citeauthor{hirzer08} begründet dies Entscheidung damit, dass bei einem zu kleinen
 Schwellwert Liniensegmente, die zuweit auseinander liegen, nicht zusammengeführt werden, obwohl sie zusammengehören.
 Wird der Abstand des Schwellwerts aber zu groß gewählt, werden Liniensegmente zusammengeführt, die nicht
 zusammengehören. Um auf einen Distanzschwellwert verzichten zukönnen, werden die Liniensegmente sortiert, sodaß
 Liniensegmente mit kurzen Verbindungslinien zuerst zusammengeführt werden. Dadurch kann sichergestellt werden, dass
 Liniensegmente zusammengeführt werden die nah beieinander liegen.

Das Zusammenführen der Liniensegmente wird zweimal durchgeführt. Zuerst innerhalb einer Region um alle kurzen
 Liniensegmente zu verbinden. Nachdem innerhalb aller Regionen Liniensegmente zusammengeführt wurden, wird der Vorgang
 auf dem gesammten Bildsignal wiederholt um größere Liniensegmente zuvereinen. Dadurch müssen nicht alle
 Liniensegmente-Kombinationen im gesammten Bildsignal verglichen werden, was die Laufzeit
 reduziert\footcite[Vgl.][S.~10]{hirzer08}.

% subsubsection line_detection (end)

\subsubsection{Line Extension} % (fold)
\label{sub:line_extension}
% subsubsection line_extension (end)

\subsubsection{Quadrangle Detection} % (fold)
\label{sub:quadrangle_detection}
% subsubsection quadrangle_detection (end)

\begin{comment}
	Der RANSAC Grouper wird verwendet um gerade Liniensegmente zu finden. Dazu werden zwei zufällige edgels ausgewählt
	 und ihre kompatible Orientierung überprüft. Die Anzahl der supporting edgels wird durch die Distanz des
	 Liniensegments und der Orientierung bestimmt. Durch wiederholung dieses Prozesses wird die dominante Linie in der
	 Region bestimmt. Die supporting edgels werden entfernt und der Prozess wird wiederholt bis keine edgels mehr
	 vorhanden sind oder eine maximale Anzahl von wiederholungen erreicht wurde. Durch dieses Wiederholung wird
	 sichergestellt, dass alle dominanten Linien in einer Region erkannt werden.

	Vorteil: Der Algorithmus ist sehr schnell und lässt sich für gewünschte Liniensegmente anpassen.
	Nachteil: Durch sein antisotropic detection verhalten diskriminiert das verfahren diagonale liniensegmente. Dies
	 ist durch ein rechteckiges samplingrid bedingt.

	Hirzer hat das Verfahren um zwei Punkte erweitert und angepasst.
	Wird in einem RGB Bild ein Kanal untersucht und ein Edgel gefunden, werden in den restlichen zwei Kanälen an der
	 gleichen Position nach einem Edgel gesucht. Falls in allen drei Kanälen ein Edgel gefunden wird, handelt es sich
	 um ein Linie (Schwarz/Weiß) und keine Farblinie.

	Nur das erste Frame wird vollständig untersucht und die Position der gefundenen Marken notiert. In den folgenden
	 frames wird nur in den Regionen der gefundenen Marke das Verfahren benutzt. Erst nach einer festgelegten Anzahl
	 von frames wird wieder ein vollständiger Durchlauf des Verfahrens durchgeführt.
\end{comment}

% subsection verfahren_nach_hirzer (end)

% section section_name (end)
\section{Vorgehen} % (fold)
\label{sec:vorgehen}
\begin{comment}
	Vorgehen: Analysemethoden vorstellen wie Algorithmen untersucht werden.
	Vergleich O-Notation
	Laufzeitanalyse
	Gleiche Kriterien (selbes Bild, selbes Video)
\end{comment}

Um das Verfahren ARToolKitPlus und das Verfahren von \citeauthor{hirzer08} zu analysieren, werden die Kosten der
 Algorithmen betrachtet und ihre Wachstumsrate bestimmt. Elementare Operationen, wie Addition, Zuweisung und
 Zugriffe auf ein Array, werden in Abhängigkeit von der Größe der Eingabemenge berechnet. Daraus wird wiederum das
 asymptotische Wachstum ermittelt.

Bei Verfahren, deren Algorithmen nicht im Quellcode vorliegen, wird durch eine experimentelle Analyse die Laufzeit des
 Verfahrens für unterschiedliche Eingabemengen untersucht. Die Durchführung der experimentellen Analyse erfolgt dabei
 auf einem iPod touch (4. Generation) mit einem ARMv7 Befehlssatz. Alle Programme der Analyse wurden mit
 LLVM GCC 4.2 und Kompiliereroptimierung $\mathit{Os}$\footcite[Vgl.][]{cc} erstellt.

\paragraph{Bilddatenbeschaffung:} % (fold)
\label{par:bilddatenbeschaffung}
Das Videosignal wird unter iOS durch das AVFoundation Framework zur Verfügung gestellt und ermöglicht durch Delegation
 den Zugriff auf einzelne Frames\footcite{avfoundation}. AVFoundation erlaubt die Einstellung von fünf verschiedenen
 Auflösungen für Videosignale, von Low, Medium, High über $640 \times 480$ bis zu $1280 \times 720$. Ferner unterstützt
 AVFoundation die Bildformate YCbCr mit $8$-Bit und BGRA mit $32$-Bit.
% paragraph bilddatenbeschaffung (end)

\paragraph{Vergleichsanalyse:} % (fold)
\label{par:vergleichsanalyse}
Um beide Verfahren miteinander zu vergleichen, wird die Laufzeit der Programme gemessen. Dazu wird durch hinzufügen von
 Anwendungsschritten, von der Bildbeschaffung hin zur Erkennung einer Marke, eine Laufzeit erstellt. Dieses Verfahren
 wurde von \citeauthor{wagner09b}\footcite[Vgl.][]{wagner09b} verwendet um Studierstube 4 auf unterschiedlichen Geräten
 zu beurteilen. In meiner Analyse wird mit der gleichen Technik ARToolKitPlus und das Verfahren von
 \citeauthor{hirzer08} auf einem iPod touch untersucht.
% paragraph vergleichsanalyse (end)

\paragraph{ARToolKitPlus:} % (fold)
\label{par:artoolkitplus}
Die Untersuchung von ARToolKitPlus unter iOS 4 erfolgt mit VRToolKit\footcite[Vgl.][]{vrtoolkit} von
 \citeauthor{vrtoolkit}. VRToolKit ist ein Obj-C Wrapper für ARToolKitPlus. VRToolKit konfiguriert das ARToolKitPlus
 System zur Verarbeitung von Marken und sorgt für die Bereitstellung von Bildsignalen der Kamera.

ARToolKitPlus bietet die Möglichkeit einer automatischen Schwellwertanpassung, um auch bei wechselnden
 Lichtverhältnissen bessere Ergebnisse zu erzielen\footcite[Vgl.][S.~142]{wagner07b}. Diese Option wurde zur Analyse
 des Verfahrens nicht aktiviert. Zusätzlich erlaubt ARToolKitPlus die Verarbeitung eines Bildsignals, indem nur die
 Hälfte der horizontalen und die Hälfte der vertikalen Pixeln untersucht wird. Dies Option wurde explizit in der
 Analyse untersucht (Vgl. \autoref{sec:fiducial_detection}).
% paragraph artoolkitplus (end)

\paragraph{Verfahren nach \citeauthor{hirzer08}:} % (fold)
\label{par:verfahren_nach_hirzer}
Das Verfahren von \citeauthor{hirzer08} ist eine Eigenentwicklung in Obj-C und C, die nach dem Konzept von VRToolKit und
 dem Softwarehersteller infi\footcite[Vgl.][]{infi} entworfen wurde. Die Implementierung sorgt für die Bereitstellung
 und Verarbeitung des Bildsignals.
% paragraph verfahren_nach_hirzer (end)

% section vorgehen (end)
\section{Material} % (fold)
\label{sec:material}
\begin{comment}
	Material:
	Vorstellung der Geräte, Frameworks, Libs, Opensource Code, etc. inkl. Referenzangabe wie ein Glossar.
\end{comment}

\begin{comment}
	\gls{AR}-Anwendungen für Smartphones werden durch einen geeigneten Softwareentwurf speziell für Smartphones ermöglicht\footcite[Vgl.][]{wagner09a}.

	Die Software Architektur für AR auf Smartphones unterscheidet sich von einer PC Architektur grundsätzlich durch die
	 Menge und Bandbreite des Arbeitsspeichers. Die Software kann nach gängigen Entwurfskriterien entworfen werden
	 wobei die Vor- und Nachteile von dynamic und static Libraries berücksichtigt werden müssen.

	Bei der Entwicklung ist zu beachten, dass Emulatoren für mobile Geräte ungeeignet sind um Algorithmen zu
	 entwickeln, da ein Emulator nicht die tatsächliche Geschwindigkeit ermöglicht. Dadurch bedingt sollte eine
	 Anwendung besser als native PC Anwendung entwickelt werden und im letzten Schritt auf der Zielplattform getestet
	 werden. Bei der Entwicklung ist darauf zu achten, dass die Software speziell für Smartphones entwickelt wird um
	 hohe Leistung und Robustheit zu erreichen.

	Durch die Verwendung von multithreading oder interleaving kann auch auf Geräten ohne Multicore oder Hyperthreading
	 eine parallele Ausführung von Verarbeitungsschritte ermöglicht werden. Der Bezug von Kameradaten und der
	 Posenberechnung erfolgt dann in zwei Threads.

	Durch die Verwendung von kompakten Pixelformaten wird nicht nur die Speicherbandbreite effektiver ausgenutzt
	 sondern auch die digitale Bildverarbeitung erleichtert\footcite[Vgl.][]{wagner09b}.

	Durch die gerätespezifischen Rendering Engines muss eine Gerätekonfiguration sorgfältig ausgewählt werden. Wird das
	 rendering für eine Library abstrahiert muss es für alle unterstützten Geräte angepasst werden. Dies wird nur zur
	 Vollständigkeit erwähnt und ist kein Bestandteil dieser Arbeit.

	Da die meisten Smartphones im Gegensatz zu PCs keine floating point einheit besitzen, werden die Operation in
	 Software emuliert. Daraus resultiert eine geringe Geschwindigkeit. Um das Problem zu umgehen sollte wo mögliche
	 fixed point oder integer verwendet werden, oder auch lookup tables und interpolation/annährung.

	Statistik-basierte Verfahren die mittels einer Hypotese (hypostesize) ein Liniensegment schätzen.
	Gradienten-basierte Verfahren die durch gradient magnitude und Orientierung linien erkennen.
	Pixel-Conectivity Verfahren erkennen Linien durch die Nähe von Pixeln und ihrer Orientierung.
	Verfahren nach Hough erkennen geometrische Figuren durch ihre parameter darstellung. (Linien durch Polar-Darstellung)\footcite[Vgl.][]{hirzer08}
\end{comment}

Die Entwicklung der Software wurde auf einem MacBook Pro und einem Mac Mini mit OS X 10.6 und Xcode 3.2 durchgeführt.
 Als Zielplattform kam ein iPhone 4 mit iOS 4 zum Einsatz.

Die Softwarebibliothek MKVideoIO abstrahiert Apple's AVFoundation (iPhone) und QTKit (OS X) Bibliotheken um
 Videosignale anzufordern und zu verarbeiten. Die Bibliothek wurde so entworfen, dass ein Einsatz auf dem iPhone und
 einem Mac möglich ist. Der Entwurf wurde nach dem Observer Muster\footcite[Vgl.][S.~287--300]{gamma96} angefertigt.

Für das iPhone gibt es von Apple keine Bibliothek um Aufgaben aus dem Bereich der Bildverarbeitung durchzuführen. Um
 diese Aufgaben durchzuführen, wurde die Softwarebibliothek MKImageProcessing entworfen. Die Aufgaben der Bibliothek
 umfasst das Auslesen und Schreiben einer Bildmatrix, sowie der Konvertierung von Pixeln. Bei der Konvertierung wurde
 besonderer Wert darauf gelegt, dass die Speicherbandbreite des iPhones ausgenutzt wird um Geschwindigkeitseinbußen zu
 vermeiden. Die Bibliothek umfasst einen Testmodus, indem es möglich ist, Ergebnisse einer Bildmanipulation auszugeben.
 Dies erleichterte die Entwicklung erheblich.

Um die Linienerkennung durchzuführen, verwendet MKImageProcessing die vorgestelle Method aus
 \autoref{sub:verfahren_nach_hirzer}. Die Implementierung umfasst neben dem Verfahren auch Klassen um Linien und Edgels
 zu beschreiben. Die Implementierung verzichtet auf einen objektorientierten Entwurf um die Geschwindigkeit zu erhöhen.
 Ferner wurde eine eigene Speicherverwaltung entwickelt, die dafür sorgt, dass während der Ausführung kein neuer
 Speicher vom System angefordert werden muss. Auch dies wurde aus Gründen der Performanz durchgeführt um geeignete
 Ergebnisse zu erzielen.

% section material (end)
\chapter{Analyse} % (fold)
\label{cha:analyse}
\begin{comment}
	Detaillierte Beschreibung der Algorithmen inkl. O-Notation (Detailierte Darstellung der Algorithmen)
	1. ARToolKitPlus
	2. Zissermann/Clarke
	Analyse: Die Auswertung nach den Kriterien aus Kap. Vorgehen OHNE WERTUNG! Nur die Daten erheben und auswerten.
\end{comment}

In diesem Kapitel wird zuerst die Laufzeit des Bildsignals unter iOS 4 analysiert, um festzustellen, ob es Unterschiede
 bei den Kombinationsmöglichkeiten von Bildgröße und Bildformat gibt. Danach wird ARToolKitPlus und das Verfahren
 nach \citeauthor{hirzer08} detailliert beschrieben und ihre Laufzeit bestimmt.

\section{Allgemein} % (fold)
\label{sec:allgemein}

Die in diesem Abschnitt vorgestellten Algorithmen sind sowohl in ARToolKit als auch in dem Verfahren nach Hirzer
 anzutreffen.

\ldots

\citeauthor{clarke96} verwenden in ihrem Verfahren ein monochromes Bildsignal $I_m$\footcite[Vgl.][S.~417]{clarke96}.
 Die Konvertierung des Bildsignals $I$ von YCbCr in $I_m$ erfolgt durch \autoref{alg:convertmonochrome}. Wie in
 \autoref{sub:farbräume} beschrieben, besteht ein YCbCr Signal aus einem Luminanz Kanal $Y$ und den Chroma Abweichungen
 $Cb$ und $Cr$. Um ein monochromes Signal $I_m$ zu erstellen, muss der Luminanz Kanal ausgelesen und in einen Puffer
 kopiert werden.

\input{alg/convertmonochrome}

Der Algorithmus verwendet als Parameter das Bildsignal $I$ und einen Pointer $I_m$ auf einen Puffer für das monochrome
 Signal. Der Monochrompuffer $I_m$ ist ein Array mit fester Größe, das beim initialisieren einmalig angelegt wird und
 danach wiederverwendet werden kann. In Zeile~\ref{alg:convertmonochrome-baseaddress} wird die Adresse des
 Luminanz-Kanals $Y$ ausgelesen. Die Funktionen \textproc{width} und \textproc{height} liefern die Breite und Höhe des
 Signals in Pixeln, mit denen die Länge der Daten berechnet wird. Anschließend werden die Daten in den Puffer kopiert.
 Die Laufzeit des Algorithmus entspricht $\Theta(1)$. (Vorsicht: Nur Zeile 5 verwendet keine Funktionen, deren Laufzeit
 dir nicht bekannt sind. baseaddress, width und height greifen evtl. auf metadaten zurück und wären damit ein einfacher
 lookup mit $\Theta(1)$. Dann wäre nur noch memcpy zu bestimmen, was im schlimmsten Fall $\Theta(n)$ wäre.)

Um auf \gls{pixel} zugreifen zu können, verwende ich \autoref{alg:getpixel}. Es wird der Puffer $I_m$ als Pointer
 übergeben und die Position $x$ und $y$ des gewünschten \gls{pixel}. $w$ und $h$ entsprechen der Breite und Höhe von
 $I_m$. Zeile~\ref{alg:getpixel-startcheck} bis Zeile~\ref{alg:getpixel-stopcheck} sorgen dafür, dass keine Werte
 außerhalb des Puffers gelesen werden können. Dies ist für die Randbehandlung bei Faltungsoperationen
 (Vgl. \autoref{sub:filter}) wichtig und wiederholt den \gls{pixel}.

\input{alg/getpixel}

Die Laufzeit von \autoref{alg:getpixel} ist im worst-case und im best-case konstant und somit $\Theta(1)$.

% section allgemein (end)


\section{ARToolKitPlus} % (fold)
\label{sec:artoolkitplus}

% Die \textit{Runtime Tracking Pipeline} (Vgl. \autoref{sub:artoolkitplus}, S.~\pageref{sub:artoolkitplus}) von
%  ARToolKitPlus wird mit der Methode \textproc{calc} (Vgl. \autoref{alg:calc}) gestartet und benötigt das Bildsignal
%  $I$, sowie den Schwellwert $\mathit{tresh}$. Die Methode \textproc{calc} dient in erster Linie als Einstiegspunkt der
%  \textit{Runtime Tracking Pipeline}. Verarbeitungsschritte die nicht zur Fiducial Detection oder Rectangle
%  Fitting gehören werden nicht aufgeführt.
% \input{alg/analyse-artplus/calc}
% Die Variablen $\mathit{marker\_info}$ und $\mathit{marker\_num}$ werden in Zeile
%  \ref{alg:calc-init-start}--\ref{alg:calc-init-end} initialisiert. Bei $\mathit{marker\_info}$ handelt es sich um die
%  Datenstruktur \textproc{MarkerInfo} (Vgl. \autoref{alg:datastructure-markerinfo}), die der Identifikation einer Marke
%  dient. In Zeile \ref{alg:calc-check-image-start}--\ref{alg:calc-check-image-end} wird überprüft, ob das Bildsignal
%  nicht $0$ ist und es sich somit um einen gesetzten Zeiger auf einen Speicherbereich handelt. Falls dies nicht der Fall
%  ist, wird das Verfahren abgebrochen und ein Fehlerwert von $0$ zurückgeliefert. Anschließend wird in Zeile
%  \ref{alg:calc-start-detection-start}--\ref{alg:calc-start-detection-end} die Methode \textproc{arDetectMarker}
%  aufgerufen und ihr Rückgabewert überprüft. Falls während der Ausführung von \textproc{arDetectMarker} einen Fehler
%  aufgetreten ist, wird die weitere Verarbeitung gestoppt und ein Fehlerwert von $-1$ zurückgegeben. Wenn
%  \textproc{arDetectMarker} erfolgreich war und eine Marke identifizieren konnte, wird die Verarbeitung fortgesetzt und
%  letztendlich in Zeile \ref{alg:calc-identified-marker} eine Markenidentifizierung zurückgegeben. Die Laufzeit des
%  Algorithmus ist konstant.

\subsection{Datenstrukturen} % (fold)
\label{sec:datenstrukturen}
\input{chapters/analyse-artoolkitplus-markerinfo}

\subsubsection{Automatischer Schwellwert} % (fold)
\label{sec:automatischer_schwellwert}

Die automatische Schwellwertanalyse von ARToolKitPlus wird durch \autoref{alg:autothresholdreset} vor ihrer Verwendung
 auf bekannte Startwerte gesetzt. Dabei wird in Zeile \ref{alg:autothresholdreset-minLum} die globale Variable
 $\mathit{minLum}$ gesetzt und in Zeile \ref{alg:autothresholdreset-maxLum} die globale Variable $\mathit{maxLum}$
 gesetzt. Die statischen Größen $\mathit{MINLUM0}$ und $\mathit{MAXLUM0}$ geben dabei die Werte an. Die Laufzeit von
 \autoref{alg:autothresholdreset} ist konstant.

\input{alg/analyse-artplus/autothresholdreset}

% subsubsection automatischer_schwellwert (end)

\input{chapters/analyse-artoolkitplus-checkimagebuffer}

% subsection datenstrukturen (end)

\subsection{Fiducial Detection} % (fold)
\label{sec:fiducial_detection}
\input{chapters/analyse-artoolkitplus-fiducial}
% subsection fiducial_detection (end)

\clearpage

\subsection{Rectangle Fitting} % (fold)
\label{sec:rectangle_fitting}
\input{chapters/analyse-artoolkitplus-rectanglefitting}
% subsection rectangle_fitting (end)

% section artoolkitplus (end)


\section{Verfahren nach Hirzer} % (fold)
\label{sec:hirzer}

\subsection{Datenstrukturen} % (fold)
\label{sub:datenstrukturen}

\input{chapters/analyse-hirzer-edgels}

\input{chapters/analyse-hirzer-liniensegmente}

\input{chapters/analyse-hirzer-speicherpools}

% subsection datenstrukturen (end)

\subsection{Linienerkennung nach \citeauthor{clarke96}} % (fold)
\label{sub:linienerkennung_nach_clarke96}

\input{chapters/analyse-hirzer-linedetection-clarke}

% subsection linienerkennung_nach_clarke96 (end)

% section hirzer (end)


% chapter analyse (end)


% chapter untersuchungsdesign (end)
		% Löwenanteil	85% inkl. Analyse + Ergebnisse + Diskussion
\chapter{Analyse} % (fold)
\label{cha:analyse}
\begin{comment}
	Detaillierte Beschreibung der Algorithmen inkl. O-Notation (Detailierte Darstellung der Algorithmen)
	1. ARToolKitPlus
	2. Zissermann/Clarke
	Analyse: Die Auswertung nach den Kriterien aus Kap. Vorgehen OHNE WERTUNG! Nur die Daten erheben und auswerten.
\end{comment}

In diesem Kapitel wird zuerst die Laufzeit des Bildsignals unter iOS 4 analysiert, um festzustellen, ob es Unterschiede
 bei den Kombinationsmöglichkeiten von Bildgröße und Bildformat gibt. Danach wird ARToolKitPlus und das Verfahren
 nach \citeauthor{hirzer08} detailliert beschrieben und ihre Laufzeit bestimmt.

\section{Allgemein} % (fold)
\label{sec:allgemein}

Die in diesem Abschnitt vorgestellten Algorithmen sind sowohl in ARToolKit als auch in dem Verfahren nach Hirzer
 anzutreffen.

\ldots

\citeauthor{clarke96} verwenden in ihrem Verfahren ein monochromes Bildsignal $I_m$\footcite[Vgl.][S.~417]{clarke96}.
 Die Konvertierung des Bildsignals $I$ von YCbCr in $I_m$ erfolgt durch \autoref{alg:convertmonochrome}. Wie in
 \autoref{sub:farbräume} beschrieben, besteht ein YCbCr Signal aus einem Luminanz Kanal $Y$ und den Chroma Abweichungen
 $Cb$ und $Cr$. Um ein monochromes Signal $I_m$ zu erstellen, muss der Luminanz Kanal ausgelesen und in einen Puffer
 kopiert werden.

\begin{algorithm}[ht]
\caption{Konvertierung zu monochromen Bildsignal}
\label{alg:convertmonochrome}
	\begin{algorithmic}[1]
		\Require $I, I_m$
		\Ensure $I_m$
		\State $Y \gets$ \Call{baseAddress}{$I$}
		\label{src:analyseConvertMonochromeBaseaddress}
		\State $w \gets$ \Call{width}{$I$}
		\State $h \gets$ \Call{height}{$I$}
		\State $l \gets w \cdot h$
		\State $I_m \gets$ \Call{copy}{$I, Y, l$}
	\end{algorithmic}
\end{algorithm}


Der Algorithmus verwendet als Parameter das Bildsignal $I$ und einen Pointer $I_m$ auf einen Puffer für das monochrome
 Signal. Der Monochrompuffer $I_m$ ist ein Array mit fester Größe, das beim initialisieren einmalig angelegt wird und
 danach wiederverwendet werden kann. In Zeile~\ref{alg:convertmonochrome-baseaddress} wird die Adresse des
 Luminanz-Kanals $Y$ ausgelesen. Die Funktionen \textproc{width} und \textproc{height} liefern die Breite und Höhe des
 Signals in Pixeln, mit denen die Länge der Daten berechnet wird. Anschließend werden die Daten in den Puffer kopiert.
 Die Laufzeit des Algorithmus entspricht $\Theta(1)$. (Vorsicht: Nur Zeile 5 verwendet keine Funktionen, deren Laufzeit
 dir nicht bekannt sind. baseaddress, width und height greifen evtl. auf metadaten zurück und wären damit ein einfacher
 lookup mit $\Theta(1)$. Dann wäre nur noch memcpy zu bestimmen, was im schlimmsten Fall $\Theta(n)$ wäre.)

Um auf \gls{pixel} zugreifen zu können, verwende ich \autoref{alg:getpixel}. Es wird der Puffer $I_m$ als Pointer
 übergeben und die Position $x$ und $y$ des gewünschten \gls{pixel}. $w$ und $h$ entsprechen der Breite und Höhe von
 $I_m$. Zeile~\ref{alg:getpixel-startcheck} bis Zeile~\ref{alg:getpixel-stopcheck} sorgen dafür, dass keine Werte
 außerhalb des Puffers gelesen werden können. Dies ist für die Randbehandlung bei Faltungsoperationen
 (Vgl. \autoref{sub:filter}) wichtig und wiederholt den \gls{pixel}.

\begin{algorithm}[!ht]\small
\caption{\textproc{getPixel}}
\label{alg:getpixel}
	\begin{algorithmic}[1]
		\Require $I_m, x, y, w, h$
		\If{$x < 0$}
		\Cost{$c_{1}$}{$1$}
		\label{alg:getpixel-startcheck}
			\State $x \gets 0$
			\Cost{$c_{2}$}{$1$}
		\EndIf
		\If{$y < 0$}
		\Cost{$c_{4}$}{$1$}
			\State $y \gets 0$
			\Cost{$c_{5}$}{$1$}
		\EndIf
		\If{$x \geq w$}
		\Cost{$c_{7}$}{$1$}
			\State $x \gets w - 1$
			\Cost{$c_{8}$}{$2$}
		\EndIf
		\If{$y \geq h$}
		\Cost{$c_{10}$}{$1$}
			\State $y \gets h -1$
			\Cost{$c_{11}$}{$2$}
		\EndIf
		\label{alg:getpixel-stopcheck}
		\State $i \gets x + \left(y \cdot w\right)$
		\Cost{$c_{13}$}{$3$}
		\State \textbf{return} $I_m[i]$
		\Cost{$c_{14}$}{$2$}
	\end{algorithmic}
\end{algorithm}


Die Laufzeit von \autoref{alg:getpixel} ist im worst-case und im best-case konstant und somit $\Theta(1)$.

% section allgemein (end)


\section{ARToolKitPlus} % (fold)
\label{sec:artoolkitplus}

% Die \textit{Runtime Tracking Pipeline} (Vgl. \autoref{sub:artoolkitplus}, S.~\pageref{sub:artoolkitplus}) von
%  ARToolKitPlus wird mit der Methode \textproc{calc} (Vgl. \autoref{alg:calc}) gestartet und benötigt das Bildsignal
%  $I$, sowie den Schwellwert $\mathit{tresh}$. Die Methode \textproc{calc} dient in erster Linie als Einstiegspunkt der
%  \textit{Runtime Tracking Pipeline}. Verarbeitungsschritte die nicht zur Fiducial Detection oder Rectangle
%  Fitting gehören werden nicht aufgeführt.
% \begin{algorithm}[!ht]
\caption{\textproc{calc}}
\label{alg:calc}
\begin{algorithmic}[1]
	\Require $I, \mathit{thresh}$
	\State $\mathit{marker\_info} \gets \infty$
	\label{alg:calc-init-start}
	\State $\mathit{marker\_num} \gets \infty$
	\label{alg:calc-init-end}
	\If{$I = 0$}
	\label{alg:calc-check-image-start}
		\State \textbf{return} $0$
	\EndIf
	\label{alg:calc-check-image-end}
	\If{\Call{arDetectMarker}{$I,\mathit{thresh},\mathit{marker\_info},\mathit{marker\_num}$} $<0$}
	\label{alg:calc-start-detection-start}
		\State \textbf{return} $-1$
	\EndIf
	\label{alg:calc-start-detection-end}
	\State \ldots \Comment{Weitere Anweisungen zur Identifikation einer Marke.}
	\State \textbf{return} $\mathit{marker\_info}.\mathit{id}$
	\label{alg:calc-identified-marker}
\end{algorithmic}
\end{algorithm}

% Die Variablen $\mathit{marker\_info}$ und $\mathit{marker\_num}$ werden in Zeile
%  \ref{alg:calc-init-start}--\ref{alg:calc-init-end} initialisiert. Bei $\mathit{marker\_info}$ handelt es sich um die
%  Datenstruktur \textproc{MarkerInfo} (Vgl. \autoref{alg:datastructure-markerinfo}), die der Identifikation einer Marke
%  dient. In Zeile \ref{alg:calc-check-image-start}--\ref{alg:calc-check-image-end} wird überprüft, ob das Bildsignal
%  nicht $0$ ist und es sich somit um einen gesetzten Zeiger auf einen Speicherbereich handelt. Falls dies nicht der Fall
%  ist, wird das Verfahren abgebrochen und ein Fehlerwert von $0$ zurückgeliefert. Anschließend wird in Zeile
%  \ref{alg:calc-start-detection-start}--\ref{alg:calc-start-detection-end} die Methode \textproc{arDetectMarker}
%  aufgerufen und ihr Rückgabewert überprüft. Falls während der Ausführung von \textproc{arDetectMarker} einen Fehler
%  aufgetreten ist, wird die weitere Verarbeitung gestoppt und ein Fehlerwert von $-1$ zurückgegeben. Wenn
%  \textproc{arDetectMarker} erfolgreich war und eine Marke identifizieren konnte, wird die Verarbeitung fortgesetzt und
%  letztendlich in Zeile \ref{alg:calc-identified-marker} eine Markenidentifizierung zurückgegeben. Die Laufzeit des
%  Algorithmus ist konstant.

\subsection{Datenstrukturen} % (fold)
\label{sec:datenstrukturen}
\subsubsection{Marken} % (fold)
\label{sub:artoolkitplus-marken}
ARToolKitPlus speichert die Informationen einer Marke in zwei einfachen Datenstrukturen. Informationen zur
 Identifizierung einer Marke werden in \textproc{MarkerInfo} gespeichert und Informationen zur Erkennung einer Marke
 werden in \textproc{MarkerInfo2} gespeichert.
% Das Identifikationsmerkmal $\mathit{id}$ wird in \autoref{alg:calc} als Rückgabewert verwendet.

\paragraph{MarkerInfo:} % (fold)
\label{par:artoolkitplus-markerinfo}
Die Datenstruktur in \autoref{alg:datastructure-markerinfo} wird nur zur Identifizierung einer Marke verwendet und wird
 zur Vollständigkeit erwähnt.
\input{alg/analyse-artplus/datastructure-markerinfo}
Der Zugriff auf die Datenstruktur ist konstant.
% paragraph markerinfo (end)

\paragraph{MarkerInfo2:} % (fold)
\label{par:artoolkitplus-markerinfo2}
Die Variable $\mathit{area}$ in \autoref{alg:datastructure-markerinfo2} speichert den Flächeninhalt einer Marke,
 während $\mathit{pos}[2]$ die Position des Zentrums der Marke enthält.
\input{alg/analyse-artplus/datastructure-markerinfo2}
$\mathit{coord\_num}$ enthält die Anzahl der gefundenen Konturpixel, die in $\mathit{x\_coord}$ als $x$-Koordinate und
 in $\mathit{y\_coord}$ als $y$-Koordinate gespeichert sind. Die konstante Größe $\mathit{AR\_CHAIN\_MAX}$ des
 Speichers für die Koordinaten wird zur Laufzeit nicht verändert. ARToolKitPlus erlaubt maximal $10000$ Koordinaten pro
 Marke. Die Eckpunkte einer Marke sind in $\mathit{x\_coord}$ und $\mathit{y\_coord}$ enthalten und werden durch einen
 Index in $\mathit{vertex}$ referenziert. Hierbei ist zu beachten, dass $\mathit{vertex}$ fünf Einträge speichert, wobei
 der erste und letzte Eintrag auf die gleiche Koordinate verweisen. Dadurch kann bei der Grafikprogrammierung mit OpenGL
 sehr einfach ein Rahmen um eine Marke gezeichnet werden. Der Zugriff auf die Variablen \textproc{MarkerInfo2} ist
 konstant.
% paragraph markerinfo2 (end)
% subsubsection marken (end)


\subsubsection{Automatischer Schwellwert} % (fold)
\label{sec:automatischer_schwellwert}

Die automatische Schwellwertanalyse von ARToolKitPlus wird durch \autoref{alg:autothresholdreset} vor ihrer Verwendung
 auf bekannte Startwerte gesetzt. Dabei wird in Zeile \ref{alg:autothresholdreset-minLum} die globale Variable
 $\mathit{minLum}$ gesetzt und in Zeile \ref{alg:autothresholdreset-maxLum} die globale Variable $\mathit{maxLum}$
 gesetzt. Die statischen Größen $\mathit{MINLUM0}$ und $\mathit{MAXLUM0}$ geben dabei die Werte an. Die Laufzeit von
 \autoref{alg:autothresholdreset} ist konstant.

\begin{algorithm}[ht]
\caption{\textproc{autothreshold.reset}}
\label{alg:autothresholdreset}
\begin{algorithmic}[1]
	\State $\mathit{minLum} \gets \mathit{MINLUM0}$
	\label{alg:autothresholdreset-minLum}
	\State $\mathit{maxLum} \gets \mathit{MAXLUM0}$
	\label{alg:autothresholdreset-maxLum}
\end{algorithmic}
\end{algorithm}


% subsubsection automatischer_schwellwert (end)

\subsubsection{Bildspeicher} % (fold)
\label{sub:artoolkitplus-bildspeicher}

Die Algorithmen zur Verwaltung des Bildspeichers werden zur Vollständigkeit erwähnt, da sie einmalig vor den Verfahren
 der Markenerkennung ausgeführt werden. Die Verfahren des Bildspeichers haben auf die Analyse keinen Einfluss.

% \autoref{alg:checkimagebuffer} ist beim Initialisieren der \textit{Runtime Tracking Pipeline} für die Bereitstellung
% des Bildspeichers zuständig.
% In Zeile \ref{alg:checkimagebuffer-size} wird die Größe des Eingangssignals mit Hilfe der globalen Variablen errechnet.
%  Danach wird in Zeile \ref{alg:checkimagebuffer-checksize-start}--\ref{alg:checkimagebuffer-checksize-end} die
%  errechnete Größe mit der Größe der globalen Variable verglichen. Wenn die Größe nicht übereinstimmt, wird in Zeile
%  \ref{alg:checkimagebuffer-checkmem-start}--\ref{alg:checkimagebuffer-checkmem-end} überprüft, ob der Speicher für ein
%  Bild bereits gesetzt ist. Falls der Speicherbereich schon gesetzt ist, muss er mit \autoref{alg:artkpfree} gelöscht
%  werden. In Zeile \ref{alg:checkimagebuffer-sizeglobal} wird der globalen Variable die berechnete Größe des
%  Bildspeichers zugewiesen. Zuletzt wird der globalen Variable des Bildspeichers ein neuer Speicherbereich in Zeile
%  \ref{alg:checkimagebuffer-newmem} zugewiesen.
\textproc{checkImageBuffer} (\autoref{alg:checkimagebuffer}) wird beim ersten Aufruf Speicher für das Bildsignal anlegen
 und den Algorithmus vollständig durchlaufen.
\input{alg/analyse-artplus/checkimagebuffer}
In allen weiteren Schritten, wenn der Bildspeicher angelegt ist, wird der Algorithmus lediglich die Größe des
 Bildspeichers berechnen und vergleichen
(Zeile \ref{alg:checkimagebuffer-size}--\ref{alg:checkimagebuffer-checksize-end}), was in konstanter Zeit erfolgt.
\autoref{alg:artkpfree} überprüft in Zeile \ref{alg:artkpfree-checkmem-start}--\ref{alg:artkpfree-checkmem-end} ob der
 Speicherbereich gültig ist.
\input{alg/analyse-artplus/artkpfree}
% Falls nicht wird die weitere Ausführung abgebrochen.
Nur im Falle, dass es sich um einen gültigen Speicherbereich handelt, wird in Zeile
 \ref{alg:artkpfree-deletemem-start}--\ref{alg:artkpfree-deletemem-end} der Speicher gelöscht und $\mathit{NULL}$
 zugewiesen. \autoref{alg:artkpalloc} alloziert den Speicherbereich für die benötigte Größe, die in Zeile
 \ref{alg:checkimagebuffer-size} von \autoref{alg:checkimagebuffer} berechnet wurde.
\input{alg/analyse-artplus/artkpalloc}
Nachdem der Speicher das erste Mal angelegt wurde, werden weder \autoref{alg:artkpfree} noch \autoref{alg:artkpalloc}
 aufgerufen. Somit ist die Laufzeit, die zur Überprüfung der Größe des Bildspeichers verwendet wird, konstant.
% subsubsection bildspeicher (end)


% subsection datenstrukturen (end)

\subsection{Fiducial Detection} % (fold)
\label{sec:fiducial_detection}
\textproc{arDetectMarker} (Vgl. \autoref{alg:detectmarker}) ist der Einstiegspunkt der Markenerkennung und benötigt das
 Bildsignal $I$, den Schwellwert $\mathit{tresh}$ sowie $\mathit{marker\_info}$ und $\mathit{marker\_num}$.
\input{alg/analyse-artplus/detectmarker}
In Zeile \ref{alg:detectmarker-init-start}--\ref{alg:detectmarker-init-end} werden die lokalen Variablen initialisiert.
 Die Variablen werden als Parameter für den Aufruf der Methode \textproc{arDetectMarker2} in Zeile
 \ref{alg:detectmarker-call-method} verwendet. In Zeile \ref{alg:detectmarker-call-autothreshold} wird der Schwellwert
 auf seine Startwerte zurückgesetzt (Vgl. \autoref{alg:autothresholdreset}) und der Bildspeicher in Zeile
 \ref{alg:detectmarker-call-imagebuffer} überprüft (Vgl. \autoref{alg:checkimagebuffer}). Die Regionenmarkierung $I_l$
 wird durch den Rückgabewert von \textproc{arLabeling} in Zeile \ref{alg:detectmarker-call-labeling} gesetzt. Im
 Anschluss wird in Zeile \ref{alg:detectmarker-check-il-start}--\ref{alg:detectmarker-check-il-end} geprüft, ob der
 Speicher der Regionenmarkierung erfolgreich gesetzt wurde. Andernfalls wird die Untersuchung für das aktuelle
 Bildsignal $I$ beendet. Nur wenn die Regionenmarkierung erfolgreich war, wird in Zeile
 \ref{alg:detectmarker-call-method} die Methode \textproc{arDetectMarker2} aufgerufen. Der Rückgabewert von
 \textproc{arDetectMarker2} wird in der Membervariable $\mathit{marker\_info2}$ gespeichert. Es wird anschließend in
 Zeile \ref{alg:detectmarker-check-marker-start}--\ref{alg:detectmarker-check-marker-end} geprüft, ob der Zeiger von
 $\mathit{marker\_info2}$ auf einen gültigen Speicherbereich verweist. Falls die Überprüfung erfolgreich war, sind in
 $\mathit{marker\_info2}$ die Koordinaten der Eckpunkte der Marke gespeichert und das Verfahren beendet.
 Die Laufzeitfunktion von \autoref{alg:detectmarker} ist $T(I) = t_{1} + t_{2} + \Theta(1)$.

\input{chapters/analyse-artoolkitplus-labeling}

\input{chapters/analyse-artoolkitplus-contour}

% subsection fiducial_detection (end)

\clearpage

\subsection{Rectangle Fitting} % (fold)
\label{sec:rectangle_fitting}
In der Rectangle Fitting Phase wird überprüft, ob eine Kontur ein Rechteck ist. Dazu wird in Rectangle Fitting die
 Methode \textproc{checkSquare} verwendet (\autoref{alg:checksquare-1}--\autoref{alg:checksquare-7}). Doch zuerst
 müssen die Schritte zum Aufruf der Methode in \textproc{arDetectMarker2} erläutert werden
 (\autoref{alg:detectmarker2-3}).

\input{alg/analyse-artplus/detectmarker2-3}

Nachdem die Kontur in \textproc{arGetContour} ermittelt wurde, wird das Ergebnis in $\mathit{ret}$ gespeichtert und
 anschließend in Zeile \ref{alg:detectmarker2-3-retvalue} geprüft, ob \textproc{arGetContour} nicht mit einem
 Fehlerwert beendet wurde. Sonst wird in Zeile \ref{alg:detectmarker2-3-continue} die nächste Iteration der Schleife
 eingeleitet. Danach wird in Zeile \ref{alg:detectmarker2-3-checksquare} die Methode \textproc{checkSquare} aufgerufen.
 Auch hier wird der Rückgabewert überprüft, um im Fehlerfall einen neuen Durchlauf der Schleife in Zeile
 \ref{alg:detectmarker2-3-continue2} einzuleiten. Wenn \textproc{checkSquare} erfolgreich beendet wurde, wird in Zeile
 \ref{alg:detectmarker2-3-save-start}--\ref{alg:detectmarker2-3-save-end} der Flächeninhalt und die Position des
 Zentrums der Marke aktualisiert, und die Anzahl der gefundenen Marken wird erhöht. In Zeile
 \ref{alg:detectmarker2-3-maxpatterns} wird die Anzahl der gefundenen Marken mit einem festgelegten Wert verglichen, der
 die maximale Anzahl gleichzeitiger Marken festlegt. Wenn die Werte übereinstimmen, wird die Schleife von Zeile
 \ref{alg:detectmarker2-3-loop-start} bis Zeile \ref{alg:detectmarker2-3-loop-end} abgebrochen. Andernfalls wird die
 Laufvariable $i$ erhöht und ein neuer Schleifendurchgang begonnen.

Das Verfahren \textproc{checksquare} benötigt als Eingabeparameter die Fläche der Region ($\mathit{area}$), die
 Datenstruktur $\mathit{marker\_infoTwo}$ mit Informationen der Region und Kontur und einen Faktor ($\mathit{factor}$)
 zur Berechnung eines Distanzschwellwerts.

\input{alg/analyse-artplus/checksquare-1}

Die lokalen Variablen werden in \autoref{alg:checksquare-1} initialisiert und ihre Bedeutung bei der ersten Verwendung erläutert.

\input{alg/analyse-artplus/checksquare-2}

In \autoref{alg:checksquare-2} wird der größte Abstand von dem Punkt $(\mathit{sx},\mathit{sy})$ für alle anderen
 Konturpunkte berechnet. In Zeile \ref{alg:checksquare-2-x}--\ref{alg:checksquare-2-y} wird aus der Koordinatenliste
 von $\mathit{marker\_infoTWO}$ die $x$-Koordinate in $\mathit{sx}$ und die $y$-Koordinate in $\mathit{sy}$
 gespeichert. In \textproc{arGetContour} (\autoref{alg:argetcontour-3}) wurde an der ersten Position von
 $\mathit{marker\_infoTWO}$ eine Koordinate mit dem größten Abstand zum Anfang der Kontur gespeichert, die hier als
 Startpunkt der Rechteckerkennung dient. In der Schleife in Zeile
 \ref{alg:checksquare-2-loop-start}--\ref{alg:checksquare-2-loop-end} wird für alle Konturpunkte in
 $\mathit{marker\_infoTWO}$ der Abstand $d$ berechnet. Wenn der Abstand $d$ größer als $\mathit{dmax}$ ist (Zeile
 \ref{alg:checksquare-2-isdbigger}), wird $\mathit{dmax}$ mit dem neuen Abstandswert $d$ überschrieben und der Index
 $i$ in $\mathit{v1}$ notiert. Am Ende von \autoref{alg:checksquare-2} enthält $\mathit{v1}$ den Index der Koordinaten
 mit dem größten Abstand zum Punkt $(\mathit{sx},\mathit{sy})$ und ist der Index zu einem Eckpunkt der Kontur. Die
 Menge der Konturpixel in $\mathit{marker\_infoTWO}$ wird mit $\mathit{v1}$ in \autoref{alg:checksquare-3} geteilt und
 die Teilmengen einzeln untersucht.

\input{alg/analyse-artplus/checksquare-3}

Bevor die Konturpixel untersucht werden, wird in Zeile \ref{alg:checksquare-3-thresh} der Distanzschwellwert
 $\mathit{thresh}$ mit Hilfe der Fläche der Region und $\mathit{factor}$ erstellt. Danach wird die Anzahl der Vektoren
 $\mathit{vnum}$ initialisiert und der Index zum Punkt $(\mathit{sx}, \mathit{sy})$ in der Liste der Vektoren
 ($\mathit{vertex}$) gespeichert. $wvnum1$ und $wvnum2$ speichern die Anzahl der gefundenen Eckpunkte und werden in
 Zeile \ref{alg:checksquare-3-wvnum1}--\ref{alg:checksquare-3-wvnum2} initialisiert. In Zeile
 \ref{alg:checksquare-3-getvertex1} wird die Methode \textproc{getVertex} aufgerufen, die einen Eckpunkt in der Menge
 der Konturpixel zwischen Position $0$ und $\mathit{v1}$ in $\mathit{marker\_infoTWO}$ finden soll. Danach wird
 \textproc{getVertex} in Zeile \ref{alg:checksquare-3-getvertex2} aufgerufen, um einen Eckpunkt in der Menge zwischen
 $\mathit{v1}$ und dem letzten Eintrag in $\mathit{marker\_infoTWO}$ zu finden. Die Methode \textproc{getVertex} wird
 am Ende des Abschnitts beschrieben.

Die Ergebnisse aus \autoref{alg:checksquare-3} können in drei Fälle kategorisiert werden. Entweder wurde

\begin{enumerate}
	\item ein Eckpunkt in jeder Teilmenge gefunden, \label{enum:checksquare-case1}
	\item in der ersten Teilmenge mehr als ein Eckpunkt gefunden oder \label{enum:checksquare-case2}
	\item in der zweiten Teilmenge mehr als ein Eckpunkt gefunden. \label{enum:checksquare-case3}
\end{enumerate}

Jeder dieser Fälle wird in \autoref{alg:checksquare-4}--\autoref{alg:checksquare-6} untersucht.

\paragraph{1. Fall:} % (fold)
\label{par:1_fall}

Das Verfahren \autoref{alg:checksquare-4} behandelt den Fall, dass die Methode \textproc{getVertex} jeweils einen
 Eckpunkt für jede Teilmenge gefunden hat. Somit sind in diesem Fall die Indizes der beiden Eckpunkte $\mathit{wv1}$
 und $\mathit{wv2}$ bekannt. Der Index $\mathit{v1}$ eines Eckpunktes wurde in \autoref{alg:checksquare-2} berechnet.
 Die Indizes werden in Zeile \ref{alg:checksquare-4-save-start}--\ref{alg:checksquare-4-save-end} in die Liste der
 Vektoren gespeichert.

\input{alg/analyse-artplus/checksquare-4}

% paragraph  (end)

\paragraph{2. Fall} % (fold)
\label{par:2_fall}

In diesem Fall wurde in der ersten Teilmenge mehr als ein Eckpunkt gefunden und in der zweiten Teilmenge kein Eckpunkt.
 \autoref{alg:checksquare-5} versucht in diesem Fall die Eckpunkte zu korrigieren. Dazu wird ein neuer Index
 $\mathit{v2}$ in Zeile \ref{alg:checksquare-5-v2} generiert, indem der Index $\mathit{v1}$ verwendet wird. Danach wird
 die Anzahl der Eckpunkte in Zeile \ref{alg:checksquare-5-delwvnum1}--\ref{alg:checksquare-5-delwvnum2} gelöscht. In
 Zeile \ref{alg:checksquare-5-vertex1-start}--\ref{alg:checksquare-5-vertex1-end} wird mit \textproc{getVertex} die
 erste Teilmenge von $0$--$\mathit{v2}$ untersucht, und in Zeile
 \ref{alg:checksquare-5-vertex2-start}--\ref{alg:checksquare-5-vertex2-end} die zweite Teilmenge von
 $\mathit{v2}$--$\mathit{v1}$. Danach wird in Zeile \ref{alg:checksquare-5-hascorners} überprüft, ob ein Eckpunkt in
 jeder Teilmenge gefunden wurde. Falls dem so ist, werden die Indizes in Zeile
 \ref{alg:checksquare-5-save-start}--\ref{alg:checksquare-5-save-end} in die Liste der Vektoren geschrieben. Andernfalls
 handelt es sich bei der Kontur nicht um ein Rechteck und das Verfahren wird in Zeile \ref{alg:checksquare-5-error}
 abgebrochen.

\input{alg/analyse-artplus/checksquare-5}

% paragraph 2_fall (end)

\paragraph{3. Fall} % (fold)
\label{par:3_fall}

Der dritte Fall bedeutet, dass in der ersten Teilmenge kein Eckpunkt und in der zweiten Teilmenge mehr als ein Eckpunkt
 gefunden wurde. Das Verfahren \autoref{alg:checksquare-6} ähnelt dem Verfahren \autoref{alg:checksquare-5} (2. Fall).
 Auch hier wird ein neuer Index $\mathit{v2}$ erstellt (Zeile \ref{alg:checksquare-6-v2}). In diesem Fall wird der
 Index $\mathit{v2}$ erstellt, indem der Index $\mathit{v1}$ und die Anzahl der gespeicherten Konturpunkte verwendet
 werden. Der Rest des Verfahrens entspricht dem Verfahren \autoref{alg:checksquare-5}. Lediglich der Aufruf von
 \textproc{getVertex} in Zeile \ref{alg:checksquare-6-vertex1} und Zeile \ref{alg:checksquare-6-vertex2} verwendet
 andere Teilmengen der Konturpunkte aus $\mathit{marker\_infoTWO}$.

\input{alg/analyse-artplus/checksquare-6}

% paragraph 3_fall (end)

Wenn \autoref{alg:checksquare-4}--\autoref{alg:checksquare-6} erfolgreich waren, und Eckpunkte in die Liste der
 Vektoren schreiben konnten, wird mit \autoref{alg:checksquare-7} die Daten in $\mathit{marker\_infoTWO}$ gespeichert.

\input{alg/analyse-artplus/checksquare-7}

In Zeile \ref{alg:checksquare-7-savesxsy} wird der Punkt $(\mathit{sx},\mathit{sy})$ als erster Index in die Liste der
 Vektoren von $\mathit{marker\_infoTWO}$ gespeichert. Je nachdem welcher der drei Fälle behandelt wurde, werden die
 Indizes $\mathit{v1}$, $\mathit{wv1}$ und $\mathit{wv2}$ in Zeile
 \ref{alg:checksquare-7-save1}--\ref{alg:checksquare-7-save3} gespeichert. Die Indizes sind danach im Uhrzeigersinn in
 $\mathit{marker\_infoTWO}$ hinterlegt\footcite[Vgl.][S.~44]{wagner07a}. In Zeile \ref{alg:checksquare-7-savelast} wird
 der letzte Index aus $\mathit{marker\_infoTWO}$ in die Liste der Vektoren geschrieben. Da der erste und der letzte
 Index identisch ist (Vgl. \autoref{alg:argetcontour-3}), sind die Eckpunkte so angeordnet, dass sie grafisch einfach
 dargestellt werden können, indem mit nur einer Linie, vom Startpunkt aus über jeden Eckpunkt zurück zum Startpunkt,
 ein Rahmen gezeichnet wird.

\textproc{getvertex} (\autoref{alg:getvertex-1}--\autoref{alg:getvertex-2}) sucht in der Menge der Koordinaten den
 Punkt, der von der Linie zwischen der Start- und Endkoordinate am weitesten entfernt liegt. Dazu benötigt
 \autoref{alg:getvertex-1} die Liste der $x$- und $y$-Koordinaten, den Startpunkt $\mathit{st}$ und Endpunkt
 $\mathit{ed}$, sowie den Schwellwert zur Überprüfung der Distanz. In der Liste $\mathit{vertex}$ werden die Indizes
 der Koordinaten gespeichert. Die Variable $\mathit{vnum}$ enthält die Anzahl der gefundenen Eckpunkte.

\input{alg/analyse-artplus/getvertex-1}

In Zeile \ref{alg:getvertex-1-init-start}--\ref{alg:getvertex-1-init-end} werden die lokalen Variablen initialisiert.
 Die Variable $a$, $b$ und $c$ werden in Zeile \ref{alg:getvertex-1-line-start}--\ref{alg:getvertex-1-line-end}
 initialisiert und werden zur Berechnung des Abstand zwischen Punkt $i$ und der Geraden $(\mathit{st},\mathit{ed})$
 verwendet. Dazu wird in der Schleife in Zeile \ref{alg:getvertex-1-loop-start}--\ref{alg:getvertex-1-loop-end} jede
 Koordinate $i$ in die Geradengleichung \autoref{eq:punktgeradenabstand} eingesetzt, um den Abstand des Punktes zur
 Geraden zu berechnen.

\begin{equation}
	\label{eq:punktgeradenabstand}
	d = \frac{\mathit{ax} + \mathit{by} + c}{\sqrt{(a^2 + b^2)}}
\end{equation}

Wenn der Abstand größer als $\mathit{dmax}$ ist, wird $d$ in Zeile \ref{alg:getvertex-1-savedmax} in $\mathit{dmax}$
 gespeichert und die Position $i$ in $\mathit{v1}$ hinterlegt. Somit ist am Ende von \autoref{alg:getvertex-1} die
 Position $i$ des Punktes mit dem größten Abstand zur Linie gespeichert.

In \autoref{alg:getvertex-2} wird in Zeile \ref{alg:getvertex-2-isdmaxgreater} der Abstand mit dem Distanzschwellwert
 verglichen. Nur wenn der Abstand größer als der Distanzschwellwert ist, wird das Verfahren fortgesetzt. Ansonsten ist
 das Verfahren in Zeile \ref{alg:getvertex-2-done} beendet.

\input{alg/analyse-artplus/getvertex-2}

In Zeile \ref{alg:getvertex-2-recursiv1} wird die Methode \textproc{getVertex} rekursiv aufgerufen, um die Teilmenge
 von der Startposition $\mathit{st}$ bis zur Position $\mathit{v1}$ zu untersuchen. Da \textproc{getVertex} pro
 Teilmenge nur den größten Abstand als Eckpunkt betrachet, kann durch den rekursiven Aufruf für eine kleinere Menge von
 Koordinaten ein weitere Eckpunkt gefunden werden, solange der Abstand größer als der Distanzschwellert ist.

In Zeile \ref{alg:getvertex-2-isvnumgreater} wird überprüft, ob nicht mehr als fünf Eckpunkte gefunden wurde. Falls
 mehr als fünf Eckpunkte gefunden wurden, wird das Verfahren in Zeile \ref{alg:getvertex-2-error} mit einem Fehlerwert
 beendet. Falls jedoch weniger als fünf Eckpunkte gefunden wurden, wird der Index $\mathit{v1}$ in der Liste
 $\mathit{vertex}$ gespeichert und die Anzahl der gefundenen Eckpunkte erhöht. Danach wird \textproc{getVertex} für die
 zweite Teilmenge, von der Position $\mathit{v1}$ bis zum Endpunkt $\mathit{ed}$, in Zeile
 \ref{alg:getvertex-2-recursiv2-start}--\ref{alg:getvertex-2-recursiv2-end} rekursiv aufgerufen.

Nachdem \textproc{arGetCountour} abgeschlossen ist, wird in \textproc{arDetectMarker2} die Marke mit der größten Fläche
 gesucht. Dazu wird in \autoref{alg:detectmarker2-4} in der Schleife von Zeile \ref{alg:detectmarker2-4-loop1-start}
 bis Zeile \ref{alg:detectmarker2-4-loop1-end} jede Marke mit allen anderen Marken in der Schleife von Zeile
 \ref{alg:detectmarker2-4-loop2-start}--\ref{alg:detectmarker2-4-loop2-end} verglichen.

\input{alg/analyse-artplus/detectmarker2-4}

In Zeile \ref{alg:detectmarker2-4-length} wird die quadratische Länge des Abstands zwischen dem Mittelpunkt der beiden
 Marken berechnet. Anschließend wird die Anzahl der Konturpixel in $\mathit{area}$ der Marke $i$ und $j$ verglichen.
 Wenn $i$ größer als $j$ ist, wird in Zeile \ref{alg:detectmarker2-4-di} der Abstand $d$ mit $\frac{\mathit{area}}{4}$
 von Marke $i$ verglichen. Wenn der Abstand kleiner ist, wird die Anzahl der Konturpixel der Marke $j$ auf $0$ gesetzt.
 Falls jedoch die Anzahl der Konturpixel von Marke $j$ größer sein sollte als die von $i$, wird in Zeile
 \ref{alg:detectmarker2-4-dj} der Abstand $d$ mit $\frac{\mathit{area}}{4}$ von Marke $j$ verglichen. Ist der Abstand
 kleiner, wird die Anzahl von Marke $i$ auf $0$ gesetzt.

In \autoref{alg:detectmarker2-5} werden nun alle Marken gelöscht, deren Variable $\mathit{area}$ in
 \autoref{alg:detectmarker2-4} auf $0$ gesetzt worden sind. Dazu werden alle Marken in Zeile
 \ref{alg:detectmarker2-5-loop-start}--\ref{alg:detectmarker2-5-loop-end} untersucht. Wenn die Überprüfung in Zeile
 \ref{alg:detectmarker2-5-shoulddelete} eine zu löschende Marke findet, wird in der Schleife von Zeile
 \ref{alg:detectmarker2-5-move-start} bis Zeile \ref{alg:detectmarker2-5-move-end} alle nachfolgende Marken um eine
 Position verschoben. Danach wird in Zeile \ref{alg:detectmarker2-5-decrease} die Anzahl der Marken verringert.

\input{alg/analyse-artplus/detectmarker2-5}

In \autoref{sub:regionenmarkierung} wurde zur Optimierung des Verfahrens nur ein Teil des Bildsignals analysiert.
 Dadurch sind Daten entstanden, die nicht mit den tatsächlichen Daten im Bildsignal übereinstimmen.
 \autoref{alg:detectmarker2-6} sorgt dafür, dass diese Daten wieder aufbereitet werden.

\input{alg/analyse-artplus/detectmarker2-6}

In Zeile \ref{alg:detectmarker2-6-address} wird die Adresse der ersten Speicherstelle der Markeninformationen in der
 Variable $\mathit{pm}$ hinterlegt. In der Schleife in Zeile
 \ref{alg:detectmarker2-6-loop-start}--\ref{alg:detectmarker2-6-loop-end} werden nun alle verbliebenen Marken
 aufbereitet. Dazu wird in Zeile \ref{alg:detectmarker2-6-area} bis Zeile \ref{alg:detectmarker2-6-pos} die Anzahl der
 Konturpixel erhöht und die Koordinaten des Zentrums der Marke korrigiert. Danach wird in der Schleife von Zeile
 \ref{alg:detectmarker2-6-coord-start}--\ref{alg:detectmarker2-6-coord-end} die Koordinaten aller Konturpixel
 korrigiert. Anschließend wird in Zeile \ref{alg:detectmarker2-6-incpm} die Adresse inkrementiert und das Verfahren mit
 der nächsten Marke wiederholt. Abschließend wird die Anzahl der Marken in $\mathit{marker\_num}$ gespeichert und die
 Speicheradresse der ersten Markeninformation an \textproc{arDetectMarker} (\autoref{alg:detectmarker},
 S.~\pageref{alg:detectmarker}) zurückgegeben.
% subsection rectangle_fitting (end)

% section artoolkitplus (end)


\section{Verfahren nach Hirzer} % (fold)
\label{sec:hirzer}

\subsection{Datenstrukturen} % (fold)
\label{sub:datenstrukturen}

\subsubsection{Edgels} % (fold)
\label{sub:datenstruktur-edgels}

Die Datenstruktur eines \glspl{edgel} besteht aus den beiden Variablen $\mathit{coordinate}$ und $\mathit{slope}$
 (\autoref{alg:datastructure-edgel}). Beide Variablen sind vom Datentyp \textproc{vector} (Vgl.
 \autoref{alg:vector}), sodass Lese- und Schreibzugriffe auf die Elemente eines \glspl{edgel} konstant sind.
\input{alg/analyse-hirzer/datastructure-edgel}

Wie in \autoref{sub:line_detection} beschrieben, muss im Verfahren von \citeauthor{hirzer08} die Orientierung von zwei
 \glspl{edgel} überprüft werden, was mit \autoref{alg:edgeliscompatible} bewerkstelligt wird. Als Parameter werden
 zwei zu vergleichende \gls{edgel} $\mathit{left}$ und $\mathit{right}$ übergeben.
\input{alg/analyse-hirzer/datastructure-edgeliscompatible}
Mithilfe von \textproc{dotProduct} (\autoref{alg:vectordotproduct}) kann die Orientierung der beiden \gls{edgel}
 überprüft werden. Zwei \gls{edgel} sind dann kompatibel, wenn der Winkel zwischen den Vektoren nicht größer als
 $67.5^\circ$ ist\footcite[Vgl.][S.~417]{clarke96}. Dies wird durch \autoref{eq:edgeliscompatible} in
 \autoref{alg:edgeliscompatible} sichergestellt. Es ist dabei darauf zu achten, dass die Berechnung in Bogenmaß
 erfolgt. Die Laufzeitfunktion ist $T(n) = 8$.

\begin{equation}
	\label{eq:edgeliscompatible}
	\cos \left(67.5\right) \approx 0.38
\end{equation}

Der Edgelspeicher in \autoref{alg:datastructure-edgelpool} verwendet ein Array von \glspl{edgel}
 (Vgl. \autoref{alg:datastructure-edgel}) mit fester Größe $N$ und einer Zählvariable, um die nächste freie Position im
 Array zu markieren.
\input{alg/analyse-hirzer/datastructure-edgelpool}
Der Speichervorrat aus \autoref{alg:datastructure-poolimplementation} ist ein Array vom Typ
 \textproc{edgelPool} mit der Größe $S$, dessen Adresse im Zeiger $\mathit{pool}$ gespeichert wird.
\input{alg/analyse-hirzer/datastructure-edgelpoolimpl}
\autoref{alg:edgelpool-getmemorypools} basiert auf einem einfachen Stack Allocator von
 \citeauthor{kr}\footcite[Vgl.][S.~100--104]{kr}.
\input{alg/analyse-hirzer/datastructure-edgelpool-getmemorypools}
Die Variable $n$ gibt die Anzahl der angeforderten Blöcke aus dem Speichervorrat an. In Zeile
 \ref{alg:edgelpool-getmemorypools-checkpoolsize} wird überprüft, ob genügend Blöcke zur Verfügung stehen und liefert
 im Erfolgsfall die Adresse zu einem Block (\autoref{alg:datastructure-edgelpool}) zurück. Falls keine Blöcke mehr zur
 Verfügung stehen, wird $\mathit{NULL}$ zurückgegeben. Die Laufzeitfunktion ist im schlechtesten Fall $T(n)=7$.
 \autoref{alg:edgelpool-getmemorypool}
\input{alg/analyse-hirzer/datastructure-edgelpool-getmemorypool}
 vereinfacht die Anforderung von Speicherblöcken, da in den meisten Fällen nur ein Block benötigt wird. Bei einem
 Aufruf kann somit auf einen Parameter verzichtet werden. Die Laufzeitfunktion ist im schlechtesten Fall $T(n)=9$.
 Sowohl \autoref{alg:edgelpool-getmemorypools} als auch \autoref{alg:edgelpool-getmemorypool} haben eine konstante
 Laufzeit.

Um \glspl{edgel} in einem Block zu speichern, wird \autoref{alg:edgelpool-addedgel} verwendet.
\input{alg/analyse-hirzer/datastructure-edgelpool-addedgel}
Der Algorithmus benötigt einen Zeiger $p$ auf einen Speicherblock und einen \gls{edgel} $e$. In Zeile
 \ref{alg:edgelpool-addedgel-validpointer-start}--\ref{alg:edgelpool-addedgel-validpointer-end} wird geprüft, ob der
 Zeiger auf eine Adresse verweist. Falls $p$ null ist, wird der Algorithmus verlassen. In Zeile
 \ref{alg:edgelpool-addedgel-checkspace-start}--\ref{alg:edgelpool-addedgel-checkspace-end} wird geprüft, ob im Array
 genügend Platz für einen weiteren Eintrag vorhanden ist. Die Größe von $N$ Einträgen richtet sich nach der in
 \autoref{alg:datastructure-edgelpool} festgelegten Arraygröße $N$. Wenn genügend Platz vorhanden ist, wird in Zeile
 \ref{alg:edgelpool-addedgel-add-start}--\ref{alg:edgelpool-addedgel-add-end} der \gls{edgel} $e$ an die freie
 Position $c$ geschrieben. Danach wird $\mathit{count}$ inkrementiert. Das Hinzufügen eines \glspl{edgel} ist konstant
 und die Laufzeitfunktion entspricht im schlechtesten Fall $T(n) = 12$.

Die Position eines \glspl{edgel} kann mit \autoref{alg:edgelpool-edgelposition} gesucht werden.
\input{alg/analyse-hirzer/datastructure-edgelpool-edgelposition}
Dazu wird der Zeiger $p$ auf den Speicherblock und das zu suchende \gls{edgel} übergeben. In Zeile
 \ref{alg:edgelpool-edgelposition-validpointer-start}--\ref{alg:edgelpool-edgelposition-validpointer-end} wird
 überprüft, ob der Zeiger $p$ auf einen gültigen Speicherblock verweist. In Zeile
 \ref{alg:edgelpool-edgelposition-count} wird die Anzahl der eingetragenen \gls{edgel} ausgelesen. In Zeile
 \ref{alg:edgelpool-edgelposition-e-start}--\ref{alg:edgelpool-edgelposition-e-end} werden die Daten des \gls{edgel}
 ausgelesen. Die Suche des \gls{edgel} erfolgt in Zeile
 \ref{alg:edgelpool-edgelposition-search-start}--\ref{alg:edgelpool-edgelposition-search-end}. Dazu wird in Zeile
 \ref{alg:edgelpool-edgelposition-isequal-start}--\ref{alg:edgelpool-edgelposition-isequal-end} ein \gls{edgel} an der
 aktuellen Position $i$ mit den lokalen Variablen verglichen. Stimmen die Werte überein, wird die Position $i$
 zurückgegeben (Zeile \ref{alg:edgelpool-edgelposition-returni}). Andernfalls wird $i$ inkrementiert und die Suche
 fortgesetzt. Wenn das \gls{edgel} nicht im Speicherblock hinterlegt ist, wird in Zeile
 \ref{alg:edgelpool-edgelposition-returnerror} ein Fehlerwert zurückgegeben. Die Laufzeit des Algorithmus ist im besten
 Fall konstant, wenn das zu suchende \gls{edgel} an der ersten Position gespeichert ist. Im schlimmsten Fall, wenn ein
 \gls{edgel} nicht gefunden wird, entspricht die Laufzeitfunktion $T(n) = 21n + 18 $ für
 $n = \text{Anzahl der \gls{edgel}}$. Die Wachstumsrate ist $21n + 18 = \Theta(n)$ für $c_{1}=20$, $c_{2}=21$ und
 $n_{0} = 18$.

\glspl{edgel} werden mittels \autoref{alg:edgelpool-getedgel} aus einem Speicherblock gelesen.
\input{alg/analyse-hirzer/datastructure-edgelpool-getedgel}
Dazu wird der Zeiger $p$ auf den Speicherblock und der Index $i$ übergeben. In Zeile
 \ref{alg:edgelpool-getedgel-validpointer-start}--\ref{alg:edgelpool-getedgel-validpointer-end} wird geprüft, ob es
 sich um einen gesetzten Zeiger handelt. Anschließend wird in Zeile
 \ref{alg:edgelpool-getedgel-validrange-start}--\ref{alg:edgelpool-getedgel-validrange-end} geprüft, ob der Index $i$
 innerhalb des gespeicherten Bereichs der \glspl{edgel} liegt. Danach wird in Zeile
 \ref{alg:edgelpool-getedgel-returnedgel} der Wert des \glspl{edgel} an Position $i$ zurückgegeben. Der Zugriff auf
 einen \gls{edgel} ist konstant. Die Laufzeitfunktion ist im schlechtesten Fall $T(n) = 7$.

Damit \glspl{edgel} aus einem Array entfernt werden können, wird \autoref{alg:edgelpool-removeedgel} verwendet.
\input{alg/analyse-hirzer/datastructure-edgelpool-removeedgel}
Es wird der Zeiger $p$ auf einen Speicherblock und das zu löschenden \glspl{edgel} übergeben. In Zeile
 \ref{alg:edgelpool-removeedgel-position} wird mit \textproc{edgelPosition} (\autoref{alg:edgelpool-edgelposition}) die
 Position des \gls{edgel} gesucht. Falls das \gls{edgel} nicht gefunden wurde, wird in Zeile
 \ref{alg:edgelpool-removeedgel-error} das Verfahren abgebrochen. Ansonsten gibt es zwei zu behandelnde Fälle um ein
 \gls{edgel} zu löschen. Das \gls{edgel} liegt
\begin{enumerate}
	\item nicht am Ende des Arrays oder \label{removeedgel-worst}
	\item liegt am Ende des Arrays. \label{removeedgel-best}
\end{enumerate}
Bei \autoref{removeedgel-best} muss lediglich $\mathit{count}$ dekrementiert werden um auf den vorigen Wert zu
 verweisen (Vgl. \autoref{fig:decrementcounter}). Das dekrementieren der Zählvariable $\mathit{p.count}$ ist eine
 Zuweisung in konstanter Zeit.
\begin{figure}[!ht]
	\centering
	\subfigure[]{
		\input{resources/Memory-Decrement-Before.pdf_tex}
		\label{fig:decrementcounter-before}
	}
	\subfigure[]{
		\input{resources/Memory-Decrement-After.pdf_tex}
		\label{fig:decrementcounter-after}
	}
	\caption{Dekrementieren von $\mathit{count}$. In \subref{fig:decrementcounter-before} soll Position $i$ gelöscht
	 werden. $c$ verweist auf die nächste freie Speicherstelle. In \subref{fig:decrementcounter-after} wird $c$
	 dekrementiert und verweist auf die neue freie Speicherstelle.}
	\label{fig:decrementcounter}
\end{figure}
Bei \autoref{removeedgel-worst} wird das Array an der Position $\mathit{position}$ geteilt und der Wertebereich von
 $[\mathit{position}+1 \dotsc \mathit{position}-n]$ wird an die Position $\mathit{position}$ verschoben
 (Vgl. \autoref{fig:memmove}).
\begin{figure}[!ht]
	\centering
	\subfigure[]{
		\input{resources/Memory-Move-Before.pdf_tex}
		\label{fig:memmove-before}
	}
	\subfigure[]{
		\input{resources/Memory-Move-After.pdf_tex}
		\label{fig:memmove-after}
	}
	\caption{Verschieben des Speicherinhalts. In \subref{fig:memmove-before} soll Position $i$ gelöscht werden. Die
	 grau schattierten Einträge werden von ihrer Position an Position $i$ verschoben \subref{fig:memmove-after}.}
	\label{fig:memmove}
\end{figure}
In Zeile \ref{alg:edgelpool-removeedgel-memmove} gibt der Operator \textproc{sizeof}($e$) die Speichergröße eines
 \glspl{edgel} an, welche zum verschieben der Daten notwendig ist. Mit $c - \mathit{position} + 1$ wird die Anzahl der
 zu verschiebenden Einträge ermittelt. Im worst-case werden $N-1$ Einträge an Position $0$ des Arrays verschoben.

Um die Laufzeit der Funktion \textproc{memmove} zu bestimmen, wurde ein Testprogramm geschrieben, dass die Zeit misst,
\label{sub:datenstruktur-edgels-memmove}
 die benötigt wird, um Einträge zu verschieben. Anhand der Daten wurde mittels einer Regressionsanalyse untersucht, ob
 die gemessenen Daten einen linearen Zusammenhang aufweisen. Die erfassten $2200$ Datenpunkte wurden nach dem Vorbild
 von \textproc{time}\footcite{time-1} ermittelt um Real-, User- und Sys-Zeit zu bestimmen. Dabei wurde ein Bereich von
 $4$ Byte bis $8388608$ Byte $= 8$ MByte als Eingabe für \textproc{memmove} verwendet. Aus User- und Sys-Zeit wurde
 die CPU-Zeit bestimmt, die zur Analyse benutzt wurde. Der Korrelationskoeffizient für $X = \mathit{BYTES}$ und
 $Y = \mathit{CPU}$ beträgt $r = 0.9754288$ und das Bestimmungsmaß $r^2 = 0.9515$. Der Interzept beträgt
 $\beta_0 = -37.77\e{-06}$ (Abweichung von $58.55\e{-06}$) und die Steigung $\beta_1 = 5.885\e{-09}$
 (Abweichung von $28.35\e{-12}$). Daraus ergibt sich eine Laufzeit von $T(n) =\Theta(n)$
 (Vgl.~\autoref{eq:analyse-removeedgel-worst}).
\begin{equation}
	\label{eq:analyse-removeedgel-worst}
	\begin{split}
		y& = \beta_0 + \beta_1 \cdot n\\
		 & = -37.77\e{-06} + 5.885\e{-09} \cdot n\\
		T(n)& = -37.77\e{-06} + 5.885\e{-09} \cdot n\\
		 & = \Theta(n)
	\end{split}
\end{equation}
In \autoref{fig:regression-memmove} sind die Daten grafisch dargestellt.
\begin{figure}[!ht]
	\centering
	\input{resources/Regression-memmove.pdf_tex}
	\caption{Lineares Modell der CPU Zeit mit $2200$ Datenpunkten. In der Darstellung sind die Konfidenzintervalle für
	 $95\%$ (grüne Linie) und für $99\%$ (rote Linie) für die vorhergesagten Werte enthalten. Die Regressionsgerade ist
	 als blaue Linie eingezeichnet.}
	\label{fig:regression-memmove}
\end{figure}

In dem Verfahren nach \citeauthor{hirzer08} ist die Menge der \gls{edgel}, und somit der Speicher, begrenzt
 (Vgl. \autoref{alg:datastructure-edgelpool}). In der Implementierung des Verfahrens nach \citeauthor{hirzer08} werden
 maximal $N = 8192$ \gls{edgel} gespeichert. Die Regressionsanalyse wurde mit einer Eingabemenge von $4$ \gls{edgel}
 bis $8192$ \gls{edgel} wiederholt. Dies entspricht einer Speichergröße von $64$ Byte bis $131072$ Byte. Der
 Korrelationskoeffizient $r = 0.1332313$ und das Bestimmungsmaß $r^2 = 0.01775$ zeigen, dass eine lineare Abhängigkeit
 in diesem Bereich unwahrscheinlich ist. Wie in \autoref{fig:regression-memmove2} zusehen ist, sind Mittelwert und
 Median parallel zur $x$-Achse. Somit ist die Laufzeit von \textproc{memmove}, für die Untersuchung des Verfahrens nach
 Hirzer mit maximal $8192$ Einträgen, konstant.
\begin{figure}[!ht]
	\centering
	\input{resources/Regression-memmove2.pdf_tex}
	\caption{Regressionsanalyse mit $1200$ Datenpunkten. Der Mittelwert ist als grüne Linie eingezeichnet und der
	 Median als rote Linie.}
	\label{fig:regression-memmove2}
\end{figure}
Im schlechtesten Fall wird ein zu löschender \gls{edgel} in \autoref{alg:edgelpool-removeedgel} am Ende der Liste
 gefunden. Dadurch beträgt die Laufzeit $T(n) = 21n + 37$. Die Wachstumsrate ist, für $c_{1} = 20$, $c_{2} = 21$ und
 $n_{0} = 37$, $21n + 37= \Theta(n)$.

Um einen Speicherblock für einen neuen Durchlauf zu löschen, kommt \autoref{alg:edgelpool-resetmemorypool} zum Einsatz.
\input{alg/analyse-hirzer/datastructure-edgelpool-resetmemorypool}
 Als Parameter wird der Zeiger $p$ übergeben und in Zeile
 \ref{alg:edgelpool-resetmemorypool-validpointer-start}--\ref{alg:edgelpool-resetmemorypool-validpointer-end}
 überprüft. Um alle Daten als gelöscht zu markieren, wird lediglich die Zählvariable in Zeile
 \ref{alg:edgelpool-resetmemorypool-reset} auf $0$ gesetzt. Die Zuweisung erfolgt in konstanter Zeit. Die
 Laufzeitfunktion ist im schlechtesten Fall $T(n) = 3$.

Wenn ein Speicherblock nicht mehr benötigt wird, kann er mit \autoref{alg:edgelpool-freememorypool} freigegeben werden.
\input{alg/analyse-hirzer/datastructure-edgelpool-freememorypool}
 Der Zeiger $p$ wird in Zeile
 \ref{alg:edgelpool-freememorypool-validpointer-start}--\ref{alg:edgelpool-freememorypool-validpointer-end} überprüft.
 In Zeile \ref{alg:edgelpool-freememorypool-resetmemory} werden die Daten des Blocks als gelöscht markiert.
 (Vgl. \autoref{alg:edgelpool-resetmemorypool}). Im Anschluss wird in Zeile
 \ref{alg:edgelpool-freememorypool-checkpointer} überprüft, ob $p$ zu dem Array $\mathit{data}$ gehört und nicht größer
 als die definierte Speichergröße ist. Wenn der Test positiv ausfällt, wird der Zeiger $p$ zur weiteren Verwendung in
 $\mathit{pool}$ gespeichert. Das Freigeben eines Speicherblocks erfolgt in konstanter Zeit. Im schlechtesten Fall ist
 die Laufzeitfunktion $T(n) = 9$.

Die Anzahl der \glspl{edgel} in einem Speicherblock werden durch \autoref{alg:edgelpool-count} ermittelt.
\input{alg/analyse-hirzer/datastructure-edgelpool-count}
Als Parameter wird der Zeiger $p$ übergeben und in Zeile
 \ref{alg:edgelpool-count-validpointer-start}--\ref{alg:edgelpool-count-validpointer-end} überprüft. Die Anzahl der
 Einträge wird in Zeile \ref{alg:edgelpool-count-counter} über die Zählvariable $\mathit{p.count}$ ermittelt. Der
 Zugriff auf die Variable, und somit die Laufzeit des Algorithmus, erfolgt in konstanter Zeit. Die Laufzeitfunktion ist
 im schlechtesten Fall $T(n) = 3$.

% subsection datenstruktur-edgels (end)


\subsubsection{Liniensegmente} % (fold)
\label{sub:datenstruktur-liniensegmente}

Die Datenstruktur eines Liniensegments und die Methoden zum hinzufügen, löschen und freigeben des Speichers sind nach
 dem Vorbild des Edgelspeichers aufgebaut. Die Datenstruktur eines Liniensegments ist in
 \autoref{alg:datastructure-linesegment} definiert.
\input{alg/analyse-hirzer/datastructure-linesegment}
Eine Linie besteht aus den \glspl{edgel} $\mathit{start}$ und $\mathit{end}$, die den Start- und Endpunkt der Linie
 darstellen. Die Variable $\mathit{slope}$ enthält die Orientierung des Liniensegments, während die Variable
 $\mathit{supportCount}$ die Anzahl der unterstützenden \glspl{edgel} der Linie speichert. $\mathit{remove}$,
 $\mathit{startCorner}$ und $\mathit{endCorner}$ sind boolesche Variablen. $\mathit{remove}$ dient im späteren Verlauf
 zur Erkennung, ob ein Liniensegment gelöscht werden muss. Wenn eine Linie einen Eckpunkt am Anfang oder am Ende
 besitzt, wird dies in den Variablen $\mathit{startCorner}$ und $\mathit{endCorner}$ festgehalten.Die letzte Variable
 $\mathit{support}$ dient zur Speicherung von \glspl{edgel}, die eine Linienhypothese unterstützen. Die Lese- und
 Schreibzugriffe auf die Datenstruktur ist konstant.

Mit \autoref{alg:linesegmentaddedgel} wird ein Unterstüzungsedgel zu einem Liniensegment hinzugefügt. Das Verfahren
 benötigt dazu das Liniensegment $l$, den \gls{edgel} $e$ und die Position $\mathit{pos}$, an die der \gls{edgel}
 gespeichert wird.
\input{alg/analyse-hirzer/datastructure-linesegmentaddedgel}
In Zeile \ref{alg:linesegmentaddedgel-hasvalidrange} wird überprüft, ob genügend Speicherplatz für ein \gls{edgel} zur
 Verfügung steht. Falls dem nicht so ist, wird in Zeile \ref{alg:linesegmentaddedgel-notvalidrange} das Verfahren
 beendet. Andernfalls, wenn genügend Speicherplatz vorhanden ist, wird in Zeile
 \ref{alg:linesegmentaddedgel-storeedgel} der \gls{edgel} im Liniensegment gespeichert. Die Laufzeit des Verfahrens ist
 konstant.

Die Methode \textproc{isOrientationCompatible} untersucht, ob zwei Liniensegmente $\mathit{left}$ und $\mathit{right}$
 fast parallel zueinander stehen (\autoref{alg:linesegmentisorientationcompatible}).
\input{alg/analyse-hirzer/datastructure-linesegmentisorientationcompatible}
Dazu wird mithilfe von \textproc{dotProduct} die Orientierung berechnet. Wenn die Orientierung im Bereich von
 $(0.92,1]$ liegt, wird als Ergebnis wahr zurückgeliefert. Das bedeutet, dass die Orientierung der Linien im Bereich
 von $0^\circ$ bis $\sim 23^\circ$ liegt und die Linien als parallel betrachtet werden. Ansonsten wird als Ergebnis
 falsch zurückgegeben, was bedeutet, dass die Linien nicht parallel sind. Die Laufzeit von
 \autoref{alg:linesegmentisorientationcompatible} ist konstant.

Mit \autoref{alg:isedgelnearline} wird der Abstand eines \gls{edgel} zu einem Liniensegment berechnet.
\input{alg/analyse-hirzer/datastructure-linesegmentisedgelnearline}
Das Verfahren benötigt dazu ein Liniensegment $l$ und ein \gls{edgel} $e$. In Zeile
 \ref{alg:isedgelnearline-iscompatible} wird überprüft, ob die Orientierung des \gls{edgel} kompatibel mit der
 Orientierung des Liniensegments ist. Wenn dies der Fall ist, wird das Verfahren fortgesetz. Andrenfalls wird das
 Verfahren in Zeile \ref{alg:isedgelnearline-notcompatible} abgebrochen. In Zeile
 \ref{alg:isedgelnearline-distance-start}--\ref{alg:isedgelnearline-distance-end} wird die Länge des Abstands der
 Endpunkte der Linie berechnet und in lokalen Variablen gespeichert. Im Anschluß wird in Zeile
 \ref{alg:isedgelnearline-pointline-start}--\ref{alg:isedgelnearline-pointline-end} der Abstand des \gls{edgel} zur
 Linie berechnet. Das Präprozessor Makro \textproc{ABS} berechnet den absoluten Betrag der Distanz in konstanter Zeit
 (Vgl. \autoref{alg:abs}).
\input{alg/analyse-hirzer/abs}
In Zeile \ref{alg:isedgelnearline-return} wird als Rückgabewert, abhängig vom Vergleich der Distanz, wahr oder falsch
 zurückgegebn. Bleibt der Abstand des \gls{edgel} zur Linie unter $0.75$ wird wahr and die aufrufende Methode
 zurückgeben. Ansonsten, wenn der Abstand größer ist, wird falsch zurückgegeben. Die Laufzeit von
 \autoref{alg:isedgelnearline} ist konstant.

Mit \textproc{intersection} wird der Schnittpunkt zweier Linien berechnet. Dazu benötigt das Verfahren in
 \autoref{alg:linesegmenintersection} eine linke und eine rechte Linie.
\input{alg/analyse-hirzer/datastructure-linesegmentintersection}
 In Zeile \ref{alg:linesegmenintersection-var-start}--\ref{alg:linesegmenintersection-var-end} werden die Punkte der
 Linienkoordinaten in lokalen Variablen gespeichert. Danach wird in Zeile
 \ref{alg:linesegmenintersection-intersect-start}--\ref{alg:linesegmenintersection-intersect-end} der Schnittpunkt der
 beiden Linie berechnet und in Zeile \ref{alg:linesegmenintersection-return} an die aufrufende Methode zurückgegeben.
 Die Berechnung des Schnittpunktes zweier Linien erfolgt in konstanter Zeit.

Die Datenstruktur eines Speichervorrats für Linien in \autoref{alg:datastructure-linesegmentpool} besteht aus einem
 Array $\mathit{data}$ mit der festen Größe $N$ und einer Zählvariablen $\mathit{count}$.
\input{alg/analyse-hirzer/datastructure-linesegmentpool}
Der Speichervorrat für Linien in \autoref{alg:datastructure-linesegmentpoolimplementation} besteht wiederum aus einem
 Array $\mathit{data}$ mit der Anzahl $S$ der zur Verfügung stehenden Speicherblöcke.
\input{alg/analyse-hirzer/datastructure-linesegmentpoolimpl}
Der Zeiger von $\mathit{data}$ wird in der Variablen $\mathit{pool}$ gespeichert. Der Zugriff auf die Datenstruktur
 erfolgt in konstanter Zeit.

Mehrere Speicherblöcke können mit \autoref{alg:linepool-getmemorypools} angefordert werden und mit
\input{alg/analyse-hirzer/datastructure-linesegmentpool-getmemorypools}
 \autoref{alg:linepool-getmemorypool} wird ein Speicherblock angefordert.
\input{alg/analyse-hirzer/datastructure-linesegmentpool-getmemorypool}
Der Aufbau der Verfahren entspricht den Verfahren des Speichervorrats für \glspl{edgel}
 (Vgl. \autoref{alg:edgelpool-getmemorypools} und \autoref{alg:edgelpool-getmemorypool}). Der Zugriff erfolgt in
 konstanter Zeit.

Um eine Linie zu einem Speicherblock hinzuzufügen, wird \autoref{alg:linesegmentpool-addline} verwendet.
\input{alg/analyse-hirzer/datastructure-linesegmentpool-addlinesegment}
Es wird ein Zeiger $p$ auf den Speicherblock, sowie eine Linie $l$ übergeben. Wenn es sich um einen gültigen Zeiger $p$
 handelt, und genügend freier Speicherplatz für eine weitere Linie vorhanden ist, wird in Zeile
 \ref{alg:linesegmentpool-addline-add-start}--\ref{alg:linesegmentpool-addline-add-end} die Linie hinzugefügt und die
 Zählvariable inkrementiert. Das Hinzufügen einer Linie ist konstant.

Zum auslesen einer Linie aus dem Speicherblock, wird \autoref{alg:linepool-getline} verwendet.
\input{alg/analyse-hirzer/datastructure-linesegmentpool-getlinesegment}
Als Parameter werden ein Zeiger $p$ und ein Index $i$ übergeben. Der Index gibt an, welche Linie aus dem Block
 ausgelesen werden soll. In Zeile \ref{alg:linepool-getline-validrange-start} wird geprüft, ob der Index sich innerhalb
 der Grenzen der gespeicherten Linien befindet. Wenn dies der Fall ist, wird in Zeile
 \ref{alg:linepool-getline-returnline} die Linie in konstanter Zeit zurückgegeben.

Mit \autoref{alg:linepool-resetmemorypool} werden die Einträge im Speicherblock gelöscht.
\input{alg/analyse-hirzer/datastructure-linesegmentpool-resetmemorypool}
Dazu wird der Zeiger $p$ auf den Speicherblock übergeben und in Zeile
 \ref{alg:linepool-resetmemorypool-validpointer-start}--\ref{alg:linepool-resetmemorypool-validpointer-end} überprüft.
 Wenn es sich um einen gültigen Zeiger handelt, wird die Zählvariable auf $0$ gesetzt. Da es sich um einen direkten
 Zugriff handelt, erfolgt das Löschen in konstanter Zeit.

Durch \autoref{alg:linepool-freememorypool} kann ein Speicherblock wieder freigegeben werden.
\input{alg/analyse-hirzer/datastructure-linesegmentpool-freememorypool}
Dazu wird der Zeiger $p$ auf Gültigkeit geprüft. Danach wird der Speicher durch \textproc{resetMemoryPool}
 (\autoref{alg:linepool-resetmemorypool}) gelöscht. In Zeile \ref{alg:linepool-freememorypool-checkpointer} wird
 überprüft, ob der Zeiger $p$ zu dem entsprechenden Block gehört, um danach die Adresse in Zeile
 \ref{alg:linepool-freememorypool-savepointer} in $\mathit{pool}$ zu speichern. Auch hier erfolgt das Freigeben des
 Speichers wieder in konstanter Zeit.

Die Anzahl der Einträge in einem Pool werden durch \autoref{alg:linepool-count} bestimmt, indem die Zählvariable
 $\mathit{count}$ zurückgegeben wird.
\input{alg/analyse-hirzer/datastructure-linesegmentpool-count}
Der Zugriff auf die Variable erfolgt in konstanter Zeit.

Im Verfahren nach \citeauthor{clarke96} werden Liniensegmente nicht aus dem Speicherpool gelöscht. Darum kann auf
 einen Algorithmus zum löschen der Einträge, wie \autoref{alg:edgelpool-removeedgel} bei \glspl{edgel}, verzichtet
 werden. Alle Operationen für Linien erfolgen somit in konstanter Zeit $T(n) = \Theta(1)$.

% subsection datenstruktur-liniensegmente (end)


\subsubsection{Speicherpools} % (fold)
\label{sub:datenstruktur-speicherpools}

% subsection datenstruktur-speicherpools (end)


% subsection datenstrukturen (end)

\subsection{Linienerkennung nach \citeauthor{clarke96}} % (fold)
\label{sub:linienerkennung_nach_clarke96}

\citeauthor{clarke96} verwenden in ihrem Verfahren ein monochromes Bildsignal $I_m$\footcite[Vgl.][S.~417]{clarke96}.
 Die Konvertierung des Bildsignals $I$ von YCbCr in $I_m$ erfolgt durch \autoref{alg:convertmonochrome}. Wie in
 \autoref{sub:farbräume} beschrieben, besteht ein YCbCr Signal aus einem Luminanz Kanal $Y$ und den Chroma Abweichungen
 $Cb$ und $Cr$. Um ein monochromes Signal $I_m$ zu erstellen, muss der Luminanz Kanal ausgelesen und in einen Puffer
 kopiert werden.
\input{alg/analyse-hirzer/convertmonochrome}
Der Algorithmus verwendet als Parameter das Bildsignal $I$ und einen Zeiger $I_m$ auf einen Puffer für das monochrome
 Signal. Der Monochrompuffer $I_m$ ist ein Array mit fester Größe, das beim initialisieren einmalig angelegt wird und
 danach wiederverwendet werden kann. In Zeile~\ref{alg:convertmonochrome-baseaddress} wird die Adresse des
 Luminanz-Kanals $Y$ ausgelesen. Die Funktionen \textproc{width} und \textproc{height} liefern die Breite und Höhe des
 Signals in Pixeln, mit denen die Länge der Daten berechnet wird. Anschließend werden die Daten in den Puffer kopiert.
 Die Konvertierung des Bildsignals ist Verarbeitungsschritt der vor dem Verfahren von \citeauthor{clarke96} durchgeführt
 wird und wird nur der Vollständigkeit erwähnt.

Um auf \gls{pixel} zugreifen zu können, wird \autoref{alg:getpixel} verwendet.
\input{alg/analyse-hirzer/getpixel}
Es wird der Puffer $I_m$ als Zeiger übergeben und die Position $x$ und $y$ des gewünschten \gls{pixel}. $w$ und $h$
 entsprechen der Breite und Höhe von $I_m$. Zeile~\ref{alg:getpixel-startcheck} bis Zeile~\ref{alg:getpixel-stopcheck}
 sorgen dafür, dass keine Werte außerhalb des Puffers gelesen werden können. Dies ist für die Randbehandlung bei
 Faltungsoperationen (Vgl. \autoref{sub:filter}) wichtig und wiederholt den \gls{pixel}.Die Laufzeit von
 \autoref{alg:getpixel} ist konstant und somit $T(n)=\Theta(1)$.

Der Algorithmus von \citeauthor{clarke96} ist in \autoref{alg:linedetection-clarke} aufgeführt und benötigt das
 monochrome Bildsignal $I_m$.
\input{alg/analyse-hirzer/linedetection-clarke}
In der doppelten Schleife in Zeile \ref{alg:linedetection-clarke-start} bis \ref{alg:linedetection-clarke-end} wird
 $I_m$ in Regionen der Größe $40 \times 40$ \gls{pixel} unterteilt. Die globalen Variablen $\mathit{imageWidth}$ und
 $\mathit{imageHeight}$ enthalten die Breite und Höhe des Bildsignals $I_m$. Die Regionengröße von $40$ \gls{pixel} ist
 in der globalen Variable $\mathit{regionSize}$ gespeichert. In \citeauthor{clarke96}\footcite{clarke96} sind keine
 Hintergrundinformationen zu der Größe einer Region angegeben. Betrachtet man $640 \bmod 40 = 0$ und $480 \bmod 40 = 0$
 ist ersichtlich, dass die Größe der Region und der Aufteilung des Bildsignals in Zusammenhang steht. In Zeile
 \ref{alg:linedetection-clarke-call-start}--\ref{alg:linedetection-clarke-call-end} werden zuerst \glspl{edgel}
 ermittelt, um im Anschluss daraus Liniensegmente zu erstellen. Wenn für eine Region Liniensegmente erstellt und
 gespeichert wurden, wird der Speicherblock der \gls{edgel} und Liniensegmente in Zeile
 \ref{alg:linedetection-clarke-reset-start}--\ref{alg:linedetection-clarke-reset-end} gelöscht.

% \begin{subequations}
% \begin{align}
% \label{eq:linedetection-analyze1}
% T(I)& =
% c_1
% + c_2
% + c_3 \left(\frac{h}{40} + 1\right)
% + c_4 \sum \limits_{y = 0}^{\frac{h}{40}} t_y \left(\frac{w}{40} + 1 \right)
% + c_5 \sum \limits_{y = 0}^{\frac{h}{40}} \sum \limits_{x = 0}^{\frac{w}{40}} t_y t_x\\
% & \quad + c_6 \sum \limits_{y = 0}^{\frac{h}{40}} \sum \limits_{x = 0}^{\frac{w}{40}} t_y t_x
% + c_7 \sum \limits_{y = 0}^{\frac{h}{40}} \sum \limits_{x = 0}^{\frac{w}{40}} t_y t_x
% + c_9 \sum \limits_{y = 0}^{\frac{h}{40}} t_y \nonumber \\
% \label{eq:linedetection-analyze2}
% T(I)& =
% c_1
% + c_2
% + c_3 \left(n + 1\right)
% + c_4 \sum \limits_{y = 0}^{n} t_y \left(k + 1 \right)
% + c_5 \sum \limits_{y = 0}^{n} \sum \limits_{x = 0}^{k} t_y t_x\\
% & \quad + c_6 \sum \limits_{y = 0}^{n} \sum \limits_{x = 0}^{k} t_y t_x
% + c_7 \sum \limits_{y = 0}^{n} \sum \limits_{x = 0}^{k} t_y t_x
% + c_9 \sum \limits_{y = 0}^{n} t_y \nonumber \\
% \label{eq:linedetection-analyze3}
% T(I)& =
% c_1
% + c_2
% + c_3 \left(n + 1\right)
% + c_4 \left[n \left(k + 1 \right)\right]
% + c_5 n k
% + c_6 n k
% + c_7 n k
% + c_9 n\\
% \label{eq:linedetection-analyze4}
% T(I)& = c_1 + c_2 + c_3 + \left(c_3 + c_4 + c_9\right) n + \left(c_4 + c_5 + c_6 + c_7\right) n k\\
% \label{eq:linedetection-analyze5}
% T(I)& = \Theta(nk)
% \end{align}
% \end{subequations}

Das Verfahren zur Bestimmung der Edgels (\autoref{alg:findedgels-horizontal} und \autoref{alg:findedgels-vertical})
 benötigt das monochrome Bildsignal $I_m$, sowie die Position der oberen linken Ecke der Region, die durch oben $t$ und
 links $l$ definiert ist.
\input{alg/analyse-hirzer/findedgelsclarke1}
Der Zeiger $E$ wird zur Speicherung der gefundenen \glspl{edgel} verwendet. Zeile
 \ref{alg:findedgels-horizontal-scanlinestart}--\ref{alg:findedgels-horizontal-scanlineend} ist für den Aufbau der
 horizontalen Scanlines verantwortlich. Die Überprüfung sorgt dafür, dass die Scanlines bis zum Ende der Region im
 Abstand von $5$ Pixeln untersucht werden. Nach der Initialisierung der Variablen wird in der Schleife von
 Zeile~\ref{alg:findedgels-horizontal-loopstart}--\ref{alg:findedgels-horizontal-loopend} jeder Pixel auf der Scanline
 untersucht. Zuerst wird in Zeile \ref{alg:findedgels-horizontal-convolute} die Faltung mit einem Gauß-Kernel
 vorgenommen (Vgl. \autoref{alg:convolutekernel-horizontal}, S. \pageref{alg:convolutekernel-horizontal}). Der Test
 in Zeile \ref{alg:findedgels-horizontal-foundedgel} überprüft anschließend das Ergebnis der Faltung. Wenn der
 Schwellwert nicht überschritten wird, gibt es keinen genügend großen Anstieg des Gradienten und das Ergebnis wird auf
 $0$ gesetzt. Wird der Schwellwert überschritten, handelt es sich um einen Edgel und das Ergebnis wird in den
 Bedingungen von Zeile \ref{alg:findedgels-horizontal-maxima} weiter untersucht, ob es sich um ein lokales Maximum
 handelt. Ein lokales Maximum bedeutet, dass ein Edgel einen größeren Gradienten besitzt als seine beiden Nachbarn. Die
 Bedingung in Zeile \ref{alg:findedgels-horizontal-maxima} wird bei der ersten Überprüfung immer fehlschlagen.
 Dadurch wird sichergestellt, dass kein Maximum an den Rändern existiert, da hier nicht genügend Nachbarn vorhanden sind
 um eine verlässliche Aussage zu treffen. Zeile \ref{alg:findedgels-horizontal-copy-prev1} und
 Zeile \ref{alg:findedgels-horizontal-copy-edgel} kopieren die Werte für den nächsten Durchlauf. Durch das kopieren der
 Werte werden die Nachbarn für den nächsten Durchlauf um eine Position weiterverschoben. Nur bei einem lokalen Maximum
 (Zeile \ref{alg:findedgels-horizontal-maxima}--\ref{alg:findedgels-horizontal-maxima-end}) wird die Position des
 Edgels gespeichert (Vgl. \autoref{alg:vectorsetcoordinate}, S. \pageref{alg:vectorsetcoordinate}), und seine
 Orientierung berechnet (Vgl. \autoref{alg:gradientintensity}, S. \pageref{alg:gradientintensity}). Der Edgel wird mit
 \textproc{addEdgel} (\autoref{alg:edgelpool-addedgel}, S. \pageref{alg:edgelpool-addedgel}) in einem Speicherblock zu
 weiteren Verarbeitung gespeichert. Sind alle Pixel auf einer Scanline untersucht, wird in Zeile
 \ref{alg:findedgels-horizontal-increment} die nächste Scanline ausgewählt. Das Verfahren wird solange wiederholt, bis
 alle Scanlines innerhalb der Region untersucht wurden. \autoref{alg:findedgels-vertical} untersucht die vertikalen
 Scanlines in Zeile \ref{alg:findedgels-vertical-scanlinestart}--\ref{alg:findedgels-vertical-scanlineend} analog zu
 \autoref{alg:findedgels-horizontal} Zeile
 \ref{alg:findedgels-horizontal-scanlinestart}--\ref{alg:findedgels-horizontal-scanlineend}.
\input{alg/analyse-hirzer/findedgelsclarke2}

In \autoref{alg:findedgels-horizontal-analyse} und \autoref{alg:findedgels-vertical-analyse} sind die Kosten von
 \textproc{findEdgels} aufgeführt.
\input{alg/analyse-hirzer/findedgelsclarke3}
Die bereits vorgestellten Methoden \textproc{vectorSetCoordinate} (\autoref{alg:vectorsetcoordinate}) und
 \textproc{addEdgel} (\autoref{alg:edgelpool-addedgel}) haben eine konstante Laufzeit. Die Methoden
 \textproc{convoluteKernelX}, \textproc{convoluteKernelY} und \textproc{gradientIntensity} haben ebenfalls eine
 konstante Laufzeit, die zu einem späteren Zeitpunkt bewiesen wird.  Die Laufzeit von
 \autoref{alg:findedgels-horizontal-analyse} lässt sich wie in \autoref{eq:findedgels1} zusammenfassen.
\input{eq/hirzer/findedgels1}
Der Algorithmus ist von den Variablen $t$, $l$ und $\mathit{regionSize}$ abhängig. Der Bereich
 $[t,t+\mathit{regionSize})$ und $[l,l+\mathit{regionSize})$ wird in der Analyse als $[t,t+\mathit{regionSize}) = n$
 und $[l,l+\mathit{regionSize}) = m$ bezeichnet. Durch Umformung in \autoref{eq:findedgels2} werden die Konstanten
 isoliert, was zu einer Laufzeit von $T(n,m) = \Theta(\tfrac{nm}{5})$ für \autoref{alg:findedgels-horizontal-analyse}
 führt (\autoref{eq:findedgels3}). Die Kosten für \textproc{findEdgels} zur Untersuchung der vertikalen Scanline sind in
 \autoref{alg:findedgels-vertical-analyse} aufgeführt und entsprechen den Kosten von
 \autoref{alg:findedgels-horizontal-analyse}.
\input{alg/analyse-hirzer/findedgelsclarke4}
Die Kosten des Algorithmus sind in \autoref{eq:findedgels4} aufgeführt und durch Umformung in \autoref{eq:findedgels5}
 werden die Konstanten isoliert.
\input{eq/hirzer/findedgels2}
Dies führt zu einer Laufzeit von $T(n,m) = \Theta(\tfrac{nm}{5})$ für \autoref{alg:findedgels-vertical-analyse}
 (\autoref{eq:findedgels6}).
Um die Laufzeit von \textproc{findEdgels} zu bestimmen, werden die Laufzeiten von
 \autoref{alg:findedgels-horizontal-analyse} und \autoref{alg:findedgels-vertical-analyse} in \autoref{eq:findedgels7}
 zusammengefasst.
\input{eq/hirzer/findedgels3}
Durch Umformung in \autoref{eq:findedgels8} kann die Laufzeit des Algorithmus in \autoref{eq:findedgels9} ermittelt
 werden. Die Laufzeit von \textproc{findEdgels} entspricht demnach $T(n,m) = \Theta(\tfrac{nm}{5})$.

\autoref{alg:convolutekernel-horizontal} und \autoref{alg:convolutekernel-vertical} berechnen den Gradienten durch Faltung mit dem Gauß-Kernel
$\left( \begin{smallmatrix}
-3& -5& 0& 5& 3
\end{smallmatrix} \right)$
 auf der horizontalen und vertikalen Scanline. Als Parameter benötigt der Algorithmus den Zeiger des monochromen
 Bildsignals $I_m$, die Position des Pixels ($x$ und $y$), sowie die Breite $w$ und Höhe $h$ von $I_m$. In Zeile
 \ref{alg:convolutekernel-horizontal-readstart}--\ref{alg:convolutekernel-horizontal-readend} werden durch die
 Funktion \textproc{getpixel} (Vgl. \autoref{alg:getpixel}, S. \pageref{alg:getpixel}) die benötigten Pixelwerte
 ausgelesen und den Variablen zugewiesen. Im Anschluss werden die Werte mit dem Gauß-Kernel
$\left( \begin{smallmatrix}
-3& -5& 0& 5& 3
\end{smallmatrix} \right)$
berechnet um den Gradienten zu bestimmen.
\input{alg/analyse-hirzer/convolutekernelx}
Bei genauer Betrachtung von \autoref{alg:convolutekernel-horizontal} und \autoref{alg:convolutekernel-vertical}
 fällt auf, dass der Wert $p_3$ in der Berechnung nicht vorkommt.
\input{alg/analyse-hirzer/convolutekernely}
Dies ist darauf zurückzuführen, dass der Gauß-Kernel an der dritten Stelle mit $0$ definiert ist. Somit kann die
 Multiplikation vernachlässigt werden. Die Laufzeit von \autoref{alg:convolutekernel-horizontal} und
 \autoref{alg:convolutekernel-vertical} ist konstant.

In \autoref{alg:gradientintensity} wird mittels Faltung die Orientierung eines \glspl{edgel} bestimmt. Als
 Eingabeparameter wird das monochrome Bildsignal $I_m$, dessen Breite $w$ und Höhe $h$, sowie die Position des
 \glspl{edgel} ($x,y$) benötigt.
\input{alg/analyse-hirzer/gradientintensity}
In Zeile \ref{alg:gradientintensity-readstart}--\ref{alg:gradientintensity-readend} werden die Pixelwerte ausgelesen
 und den Variablen zugewiesen. In Zeile
 \ref{alg:gradientintensity-convolutestart}--\ref{alg:gradientintensity-convoluteend} erfolgt die Faltung mit dem
 Sobel-Operator\footcite[Vgl.][S.~120--123]{burger05}, dessen Filter
\begin{subequations}
\begin{align}
	H_x =&
	\begin{pmatrix}
		1& 0& -1\\
		2& 0& -2\\
		1& 0& -1
	\end{pmatrix}
\end{align}
\begin{align}
	H_y =&
	\begin{pmatrix}
		1& 2& 1\\
		0& 0& 0\\
		-1& -2& -1
	\end{pmatrix}
\end{align}
\end{subequations}
den Gradienten $G_x$ und $G_y$ bestimmen. Wie in \autoref{alg:convolutekernel-horizontal} werden Multiplikationen von
 $0$-Werten des Filters vernachlässigt. Mit
\begin{equation}
	\label{eq:orientation}
	\Phi(x,y) = \arctan{\left(\tfrac{G_y}{G_x}\right)}
\end{equation}
kann die Orientierung berechnet werden. $G_x$ und $G_y$ werden mit \autoref{alg:vectorsetcoordinate} als
 \textproc{vector} gespeichert und normalisiert
 (Zeile \ref{alg:gradientintensity-vector-start}--\ref{alg:gradientintensity-vector-end}). Die Laufzeit von
 \autoref{alg:gradientintensity} ist konstant.

Um aus den gefundenen \glspl{edgel} Liniensegmente zu erzeugen, wird \autoref{alg:findlinesegments1} verwendet. Das
 Verfahren benötigt den Speicherblock $E$, in dem die \glspl{edgel} vorliegen, und den Speicherblock $L$, der zur
 Speicherung der gefundenen Liniensegmente dient.

\input{alg/analyse-hirzer/findlinesegments-1}

In Zeile \ref{alg:findlinesegments1-init-start}--\ref{alg:findlinesegments1-init-end} wird die Variable
 $\mathit{presentLine}$ initialisiert. Diese Variable enthält das zu speichernde Liniensegment im Verfahren.
 Die Zählvariable $\mathit{edgelsInRegion}$ wird mit $0$ initialisiert und speichert die Anzahl der \gls{edgel} in
 einer Region. In der Schleife von Zeile \ref{alg:findlinesegments1-loop-start} bis Zeile
 \ref{alg:findlinesegments1-loop-end} wird das RANSAC-Verfahren wiederholt, solange das zu untersuchende Liniensegment
 genügend Unterstützung durch \gls{edgel} besitzt und genügend \gls{edgel} in der Region vorhanden sind. Die Anzahl der
 unterstützenden \gls{edgel} wird in Zeile \ref{alg:findlinesegments1-clearsupport} für jeden Durchlauf gelöscht.
 Danach wird in Zeile \ref{alg:findlinesegments1-line-start}--\ref{alg:findlinesegments1-line-end} ein zufälliges
 Liniensegment in der Region erstellt und untersucht
 (Vgl. \autoref{alg:findlinesegments2}--\autoref{alg:findlinesegments4}). Die Erstellung eines Liniensegments wird
 mehrmahls wiederholt, um das Liniensegment mit den meisten Unterstützungsedgels zu finden. In Zeile
 \ref{alg:findlinesegments1-hasenoughsupport} wird die Anzahl der der unterstützenden \gls{edgel} mit der Anzahl der
 benötigten \gls{edgel} verglichen. Nur wenn genügend \gls{edgel} das Liniensegment unterstützen, wird in
 \autoref{alg:findlinesegments5}--\autoref{alg:findlinesegments7} das Verfahren fortgesetzt und die
 Unterstützungsedgels aus dem Speicherblock $E$ entfernt.

In \autoref{alg:findlinesegments2} wird die Initialisierung der Variablen vorgenommen, die zur Erstellung eines
 Liniensegments benötigt werden. Die Variablen $\mathit{start}$ und $\mathit{end}$ werden als Indizes für \gls{edgel}
 benutzt. Die Variable $\mathit{maxIteration}$ gibt die maximale Anzahl der Interationen an, die $\mathit{iterator}$
 durchlaufen kann. $\mathit{first}$ ist der Startedgel und $\mathit{last}$ der Endedgel des Liniensegments. In
 $\mathit{numberOfEdgels}$ wird die Anzahl der \gls{edgel} in $E$ gespeichert.

\input{alg/analyse-hirzer/findlinesegments-2}

Die Auswahl eines Liniensegments erfolgt mit \autoref{alg:findlinesegments3}. Zeile
 \ref{alg:findlinesegments3-loop-start}--\ref{alg:findlinesegments3-loop-end} ist dafür verantwortlich, solange nach
 einem Liniensegment zu suchen, bis der erste und letzte \gls{edgel} sich unterscheiden oder ihre Orientierung
 zueinander kompatibel ist. Als letzte Bedingung muss die Anzahl der Iterationen unterhalb des festgelegten
 Schwellwerts $\mathit{maxIteration}$ bleiben.

\input{alg/analyse-hirzer/findlinesegments-3}

In Zeile \ref{alg:findlinesegments3-first} und Zeile \ref{alg:findlinesegments3-last} werden zufällig zwei Indizes aus
 der Menge der vorhandenen \gls{edgel} errechnet. Diese werden in Zeile
 \ref{alg:findlinesegments3-start}--\ref{alg:findlinesegments3-end} verwendet, um die beiden \gls{edgel}
 $\mathit{start}$ und $\mathit{end}$ auszuwählen. Der Iterator wird anschliessend in Zeile
 \ref{alg:findlinesegments3-inc} inkrementiert.

Wenn in \autoref{alg:findlinesegments3} ein Liniensegment erstellt wurde, werden in \autoref{alg:findlinesegments4} die
 \gls{edgel} zur unterstüztung der Hypothese hinzugefügt. Dazu wird in Zeile \ref{alg:findlinesegments4-isbelowmax}
 überprüft, ob die maximale Anzahl der Iterationen nicht überschritten wurde. Falls dem so ist wird
 \autoref{alg:findlinesegments4} nicht weiter ausgeführt. Andernfalls wird in Zeile
 \ref{alg:findlinesegments4-init-start}--\ref{alg:findlinesegments4-init-end} die Variable $\mathit{segment}$ zur
 Speicherung des Liniensegments vorbereitet und der Start- und Endedgel, sowie die Orientierung zugewiesen. Die Anzahl
 der Unterstützungedgels beträgt zu diesem Zeitpunkt noch $0$.

\input{alg/analyse-hirzer/findlinesegments-4}

In der Schleife in Zeile \ref{alg:findlinesegments4-loop-start}--\ref{alg:findlinesegments4-loop-end} werden die
 \gls{edgel} gezählt, die die Linienhypothese unterstützen. Dazu wird in Zeile \ref{alg:findlinesegments4-edgel} ein
 \gls{edgel} ausgewählt und in Zeile \ref{alg:findlinesegments4-isedgelnearline} untersucht, ob der Abstand zur Linie
 klein genug ist. Wenn nicht, wird in Zeile \ref{alg:findlinesegments4-inc} die Laufvariable $j$ inkrementiert und das
 nächste \gls{edgel} ausgewählt. Wenn ein \gls{edgel} nahe genug an dem Liniensegment liegt, wird es in Zeile
 \ref{alg:findlinesegments4-addedgel} dem Liniensegment hinzugefügt und die Anzahl der Unterstützungsedgel wird in
 Zeile \ref{alg:findlinesegments4-count} erhöht. Nachdem alle \gls{edgel} untersucht wurden, wird in Zeile
 \ref{alg:findlinesegments4-hasmoresupport} überprüft, ob die Anzahl der \gls{edgel} des Liniensegments
 $\mathit{segment}$ größer ist als die Anzahl der \gls{edgel} in $\mathit{presentLine}$. Im Falle, dass
 $\mathit{segment}$ mehr Unterstützungsedgel besitzt, wird das Liniensegment in $\mathit{presentLine}$ gespeichert.
 Durch die Wiederholung in \autoref{alg:findlinesegments1} wird sichergestellt, dass das Liniensegment mit der
 größten Unterstützung ausgewählt wird.

Nachdem ein Liniensegment mit genügend Unterstützung ausgewählt wurde, kann mit \autoref{alg:findlinesegments5} die
 Start- und Endposition des Liniensegments bestimmt werden. Da ein Liniensegment aus zwei zufällig ausgewählten
 \glspl{edgel} besteht, können diese \gls{edgel} und die tatsächlichen Start- und Endpunkte voneinander abweichen
 (Vgl. \autoref{fig:})

\input{alg/analyse-hirzer/findlinesegments-5}

In Zeile \ref{alg:findlinesegments5-start}--\ref{alg:findlinesegments5-end} wird dazu die Variable $\mathit{start}$ mit
 einem kleinen Wert, und die Variable $\mathit{end}$ mit einem großen Wert, initialisiert. Die Steigung des
 Liniensegments und die Orhogonale werden in Zeile
 \ref{alg:findlinesegments5-slope-start}--\ref{alg:findlinesegments5-slope-end} berechnet. In Zeile
 \ref{alg:findlinesegments5-isxsmaller} wird geprüft, ob der Absolutwert der Steigung an Punkt $x$ kleiner ist als der
 Punkt $y$. Falls dem so ist, wird in Zeile
 \ref{alg:findlinesegments5-newstart-start}--\ref{alg:findlinesegments5-newstart-end} ein neuer Start- und Endpunkt
 gesucht, indem die $y$-Koordinate aller Unterstützungsedgels des Liniensegments verglichen werden. Andernfalls wird in
 \autoref{alg:findlinesegments6} in Zeile
 \ref{alg:findlinesegments6-newstart-start}--\ref{alg:findlinesegments6-newstart-end} ein neuer Start- und Endpunkt
 gesucht, indem die $x$-Koordiante aller \glspl{edgel} des Liniensegments untersucht und verglichen werden. Am Ende von
 \autoref{alg:findlinesegments6} ist in $\mathit{presentLine}$ ein neuer Start- und Endpunkt gespeichert.

\input{alg/analyse-hirzer/findlinesegments-6}

In \autoref{alg:findlinesegments7} wird nun geprüft, ob der Start- und Endpunkt vertauscht ist. Dazu wird der Winkel
 zwischen dem Liniensegment und seiner Orthogonalen gebildet. Wenn der Winkel kleiner als $0$ ist, werden Start- und
 Endpunkt in Zeile \ref{alg:findlinesegments7-newstart-start}--\ref{alg:findlinesegments7-newstart-end} getauscht. Im
 Anschluss daran, wird in Zeile \ref{alg:findlinesegments7-save-start}--\ref{alg:findlinesegments7-save-end} die
 Orientierung des Liniensegments berechnet und gespeichert. Danach wird das Liniensegment in Zeile
 \ref{alg:findlinesegments7-addtomempool} in den Speicherblock $L$ hinterlegt. Jetzt müssen alle Unterstützungsedgel
 des Liniensegments in Zeile \ref{alg:findlinesegments7-removeedgel-start}--\ref{alg:findlinesegments7-removeedgel-end}
 aus dem Speicherblock $E$ entfernt werden. Das entfernen der Unterstützungsedgel bewirkt, dass entweder das
 RANSAC-Verfahren wiederholt werden kann ohne die gleichen \gls{edgel} erneut zu betrachten, oder, wenn nicht mehr
 genügend \gls{edgel} in der Region vorhanden sind, das Verfahren abzubrechen und die Linienerkennung zu beenden.

\input{alg/analyse-hirzer/findlinesegments-7}


% subsection linienerkennung_nach_clarke96 (end)

% section hirzer (end)


% chapter analyse (end)
					% ausführlich
\chapter{Ergebnisse} % (fold)
\label{cha:ergebnisse}
\begin{comment}
	Ergebnisse: Die gewonnen Daten aus Kap. Analyse bewerten.
\end{comment}

In diesem Kapitel werden die gewonnen Daten aus \autoref{cha:analyse} bewertet. Dazu wird das asymptotische Wachstum
 der Verfahren beschrieben und ihre Eingabemengen erläutert. Danach werden die Verfahren durch Messung ihrer
 Verarbeitungszeit miteinander verglichen.

\section{ARToolKitPlus} % (fold)
\label{sec:ergebnisse-artoolkitplus}
ARToolKitPlus benötigt zur Erkennung einer Marke das Verfahren \textproc{regionLabeling} und
 \textproc{arDetectMarker2}. In \autoref{sec:artoolkitplus} wurde das asymptotische Wachstum des Verfahrens für den
 schlechtesten Fall auf
\begin{equation*}
O(\mathit{lxsize}\cdot\mathit{lysize})  + \Theta(\mathit{label\_num}\cdot n \log n)
\end{equation*}
bestimmt. Die Eingabemenge für das Verfahren ist das Bildsignal $I$, dass in der Untersuchung aus $640 \times 480$
 Pixeln besteht. Als Optimierung des Verfahrens werden nur die Hälfte der horizontalen \gls{pixel} ($\mathit{lxsize}$)
 und die Hälfte der vertikalen \gls{pixel} ($\mathit{lysize}$) prozessiert (Vgl. \autoref{par:artoolkitplus}).
 $\mathit{label\_num}$ ist durch $\tfrac{\mathit{lxsize}\cdot\mathit{lysize}}{4}$ begrenzt und beschreibt die maximal
 mögliche Anzahl von Regionen in einem Bildsignal mit der Größe $\mathit{lxsize}\cdot\mathit{lysize}$. Variable $n$ in
 $n \log n$ beschreibt die Anzahl der Verarbeitungsschritte in der Konturverfolgung. Die Konstante
 $\mathit{AR\_CHAIN\_MAX} = 10000$ schränkt $n$ ein, sodass eine Konturverfolgung abgebrochen wird, falls nach $10000$
 untersuchten Pixeln keine geschlossene Kontur gefunden werden konnte.
% section section_name (end)

\section{Verfahren nach Hirzer} % (fold)
\label{sec:ergebnisse-hirzer}
Das Verfahren von \citeauthor{hirzer08} besteht aus dem Algorithmus \textproc{lineDetection}
 (\autoref{alg:linedetection-hirzerquaddetection}), um eine Marke zu erkennen. Das asymptotische Wachstum des
 Verfahrens wurde, für den schlechtesten Fall, in \autoref{sec:hirzer} auf
\begin{equation*}
\Theta(hwmn^2+hwl^2\cdot\mathit{length})
\end{equation*}
bestimmt. Auch hier ist die Eingabemenge des Verfahrens das Bildsignal $I$, mit einer Bildgröße von $640 \times 480$
 Pixeln. Die Anzahl der horizontalen \gls{pixel} wird durch die Variable $w$ und die Anzahl der vertikalen
 \gls{pixel} durch die Variable $h$ beschrieben. Variable $n$ bezeichnet die Anzahl der maximal möglichen Edgel und ist
 begrenzt auf $\tfrac{1}{5}(\mathit{regionSize}^2 - 2\mathit{regionSize})$, mit $\mathit{regionSize} = 40$. Die Anzahl
 der maximal möglichen Wiederholungen des RANSAC Verfahrens (\autoref{sub:linienerkennung_nach_clarke96}) ist durch
 $m = \tfrac{1}{2}(\sqrt{8n + 25} - 5)$ begrenzt. Die Variable $l$ bezeichnet die Menge der Liniensegmente und
 $\mathit{length}$ bezeichnet die Länge des zu verarbeitenden Liniensegments.
% section section_name (end)

\section{Vergleich der Verfahren} % (fold)
\label{sec:vergleich_der_verfahren}
Beide Tracking Verfahren verarbeiten die gleiche Eingabemenge $I$ eines Bildsignals und berechnen für jede im
 Bildsignal vorhandene Marke vier Eckpunkte. Beide Verfahren sind von der Höhe und der Breite des Bildsignals abhängig.
 Durch die unterschiedlichen Arbeitsweisen der Verfahren wird das Bildsignal $I$ in nicht vergleichbare Eingabemengen
 aufgeteilt. Dadurch bedingt ist eine Analyse anhand des asymptotischen Wachstums nicht eindeutig. Da beide Verfahren
 ein BGRA Signal mit $640 \times 480$ Pixeln zu vier Eckpunkten einer Marke verarbeiten, kann durch die Messung
 der Verarbeitungszeit die Verfahren verglichen werden.

Für die Untersuchung der Laufzeit wurden zuerst $1800$ Bilder vom Kamerasensor angefordert ohne sie von den Verfahren
 prozessieren zu lassen. Die Ergebnisse sind in \autoref{tab:image} angegeben.
\begin{table}[!ht]
	\begin{center}
	\begin{tabular}[]{r..}
	\toprule
	& \multicolumn{1} {>{\centering\arraybackslash}m{4cm}}{ARToolKitPlus}
	& \multicolumn{1} {>{\centering\arraybackslash}m{4cm}}{\citeauthor{hirzer08}} \\
	\midrule
	Median		& 30.00  & 30.00  \\
	Mittelwert	& 30.12  & 30.09  \\
	Max.		& 252.60 & 218.00 \\
	Min.		& 5.10   & 9.60   \\
	\bottomrule
	\end{tabular}
	\caption{Messergebnisse der Bilddaten im Überblick.}
	\label{tab:image}
	\end{center}
\end{table}
Bei beiden Verfahren liegt der Median bei $30$ Bildwiederholungen. Dies stimmt mit der Untersuchung der
 Bildwiederholrate aus \autoref{sec:allgemein} überein. Mit diesen Ergebnissen kann sichergestellt werden, dass bei
 beiden Verfahren die Anforderung eines Bildsignals vom Bildsensor mit der vollen Geschwindigkeit erfolgt. Die
 Ausreißer bei beiden Verfahren, wirken sich nur geringfügig auf den Mittelwert aus und können vernachlässigt werden.

In einer zweiten Messung wurden von beiden Verfahren $1800$ Bilder zu vier Eckpunkten einer Marke verarbeitet. Für die
 Untersuchung der Markenerkennung wurde eine Marke auf einem hellen Untergrund platziert und mittig ausgerichtet. Die
 Erstellung der Messpunkte wurde für beide Verfahren unter gleichen Bedingungen durchgeführt. Die Ergebnisse der
 Markenerkennung sind in \autoref{tab:marker} aufgeführt.
\begin{table}[!ht]
	\begin{center}
	\begin{tabular}[]{r..}
	\toprule
	& \multicolumn{1} {>{\centering\arraybackslash}m{4cm}}{ARToolKitPlus}
	& \multicolumn{1} {>{\centering\arraybackslash}m{4cm}}{\citeauthor{hirzer08}} \\
	\midrule
	Median		& 19.10  & 15.70 \\
	Mittelwert	& 18.29  & 15.81 \\
	Max.		& 20.50  & 20.10 \\
	Min.		& 1.40   & 5.10  \\
	\bottomrule
	\end{tabular}
	\caption{Messergebnisse der Markenerkennung im Überblick.}
	\label{tab:marker}
	\end{center}
\end{table}
Durch die Verarbeitungszeit der Verfahren fällt die Bildwiederholrate, im Gegensatz zur ersten Untersuchung, bei beiden
 Verfahren ab. Bei ARToolKitPlus liegt die Median der Laufzeit bei $19.10$ Bildwiederholungen, während der Median der
 Laufzeit bei dem Verfahren von \citeauthor{hirzer08} bei $15.81$ Bildwiederholungen liegt. Das arithmetische Mittel
 liegt mit $18.29$ Bildwiederholungen bei ARToolKitPlus unterhalb des Median. Dies ist durch den kleinsten gemessenen
 Wert von $1.4$ Bildwiederholungen zurückzuführen. Bei dem Verfahren von \citeauthor{hirzer08} liegt das arithmetische
 Mittel hingegen mit $15.81$ Bildwiederholungen leicht über dem Median, da der kleinste gemessene Wert mit $5.1$
 Bildwiederholungen schneller war, als bei ARToolKitPlus. Aus diesen Messergebnissen wird deutlich, dass ARToolKitPlus
 in der gleichen Zeit mehr Bilder verarbeitet als das Verfahren von \citeauthor{hirzer08}.
% section vergleich_der_verfahren (end)

% chapter ergebnisse (end)				% knapp
\chapter{Fazit} % (fold)
\label{cha:fazit}

Wie in \autoref{cha:ergebnisse} gezeigt, erfüllen beide Verfahren die Echtzeitbedingung von \citeauthor{azuma97}. Die Verwendung der Verfahren unter iOS auf einem iPod touch ist durch die experimentelle Analyse bewiesen worden.

Wie die Analyse in \autoref{cha:analyse} gezeigt hat, sind beide Verfahren von der Eingabemenge des Bildsignals abhängig. Da ARToolKitPlus und das Verfahren von \citeauthor{hirzer08} unterschiedliche Verfahren zur Erkennung einer Marke verwenden, ist eine Aussage über das asymptotische Wachstum suboptimal. Da beide Verfahren Optimierungsstrategien anwenden, um die Menge der zu verarbeitenden Daten zu reduzieren, fällt dieser Optimierungsschritt in der Untersuchung des Wachstums weg. Erst die experimentelle Analyse offenbarte die Leistungsfähigkeit der Tracking Verfahren.

Betrachtet man lediglich die Geschwindigkeit der beiden Verfahren, so müsste ARToolKitPlus gegenüber dem Verfahren von \citeauthor{hirzer08} besser abschneiden.

Die Stärke von ARToolKitPlus liegt in der Geschwindigkeit der Verarbeitung eines Bildsignals, jedoch wird diese Geschwindigkeit auf Kosten der Erkennungsrate gewonnen. Da ARToolKitPlus eine Marke durch eine Region in einem Binärbild erkennt, wird bei ungünstigen Lichtverhältnissen keine Marke erkannt, da das Verfahren einen Schatten im Bereich der Marke als große Region betrachtet.

Diese Schwäche wird durch die Betrachtung von Liniensegmenten im Verfahren von \citeauthor{hirzer08} vermieden. Auch wenn beispielsweise eine Ecke einer Marke verdeckt ist, erkennt das Verfahren eine Marke. Der Schwachpunkt des Verfahrens nach \citeauthor{hirzer08} ist bei vielen Hell/Dunkel Wechsel zu erkennen. Bei einer Jalousie, durch die Sonnenlicht eintritt, erkennt das Verfahren alle Wechsel als möglicher \gls{edgel}, der durch die parallele Ausrichtung der Jalousie zu einer Linie erweitert wird.

Beide Verfahren sind unter Laborbedingungen mit wenigen Störelemente und guter Ausleuchtung der Marke immer in einer optimalen Verarbeitungszeit. Doch bei Anwendungen die unter Umständen auch bei schlechten Lichtverhältnissen eingesetzt werden müssen sich die Tracking Verfahren bewähren. Somit sind die Verfahren im Zweifelsfall unter realen Bedingungen mit der schlechtesten Laufzeit konfrontiert.

Welches Verfahren eingesetzt wird, muss sorgfältig evaluiert werden, da beide Systeme Schwächen besitzen, die in einer praktischen Anwendung zu Problemen führen kann.

Die Geschwindigkeit eines Tracking Verfahrens alleine kann nicht als alleiniger Entscheidungsgrund genommen werden.

% Die Optimierung bei ARToolKitPlus, nur die Hälfte der vertikalen \gls{pixel} und die Hälfte der horizontalen \gls{pixel} zu untersuchen, wird durch die Betrachtung der Wachstumsrate nicht ersichtlich. Das gleiche gilt für die Optimierung bei dem Verfahren nach \citeauthor{hirzer08}.
% Hier muss durch die Verarbeitung der Scanlines im Abstand von $5$ \gls{pixel} weniger Daten verarbeitet werden als Eingabedaten vorhanden sind.


% chapter fazit (end)			% bis zu 5% der Arbeit

% Bei Abage auskommentieren
% \nocite{*}

%\bibliographystyle{bib}
% 7) vorläufige Literaturangabe ist im header
\appendix
\chapter{Inhalt der CD} % (fold)
\label{cha:anhang}
Dieser Anhang ist eine Auflistung und Beschreibung der beigelegten CD und dient als Überblick.
\\
\begin{center}
    \begin{tabular}{p{5cm}p{8.5cm}}
	\toprule
	Ordner & Beschreibung\\
	\midrule
	ARVideo &
	Implementierung des Verfahrens von \citeauthor{hirzer08} für iOS 4.\\
	iOSVideoSpeed &
	Testprogramm zur Ermittlung der Bildwiederholrate auf einem iPod touch.\\
	LibTest &
	Testprogramm zur experimentelle Analyse von Funktionen ohne Quellcode.\\
	VRToolKit &
	ARToolKitPlus Wrapper in Obj-C von \citeauthor{vrtoolkit} für iOS 4.\\
	Messdaten-und-Programme &
	Messdaten zur Analyse der Untersuchungen und Programme in R.\\
	\bottomrule
	\end{tabular}
\end{center}
% chapter anhang (end)
\backmatter

\printbibliography[]



\end{document}
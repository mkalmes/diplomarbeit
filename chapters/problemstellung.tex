\section{Problemstellung} % (fold)
\label{sec:problemstellung}
\begin{comment}
	Problemstellung: Problemstellung und Frage im Detail erläutern
\end{comment}

Wie in \autoref{sec:forschungsstand} schon erwähnt, gibt es keine aktuellen und im Quellcode verfügbaren Systeme zur Entwicklung von \gls{AR} Anwendungen. Somit war eine Eigenentwicklung unumgänglich.

Um eine bitonale Marke, üblicherweise Schwarz/Weiß, zu erkennen, werden Verfahren zur Erkennung von Linien eingesetzt,
 die aus dem Bereich der digitalen Bildverarbeitung stammen. Anders als bei der herkömmlichen Bildverarbeitung muss
 Bildverarbeitung für \gls{AR} die Verarbeitungsschritte so schnell ausführen, dass eine Analyse eines Bildes in
 Echtzeit stattfinden kann. Bei Bildverarbeitung außerhalb der \gls{AR} ist dies kein notwendiges Kriterium. Als
 Grundlage für eine Eigenentwicklung kommt das Verfahren von \citeauthor{clarke96}\footcite{clarke96} zum Einsatz,
 dass in \autoref{sub:verfahren_nach_hirzer} vorgestellt wird.

Der Bezug von Bildern über den Kamerasensor eines Smartphones muss so schnell wie möglich erfolgen um für die Weiterverarbeitung verwendet werden zu können. Bei Smartphones werden in aller Regel Kamerasensoren aus dem Verbraucherbereich verwendet. Diese Kamerasysteme liefern nicht die Performanz wie professionelle Kameras. Daraus bedingt müssen die zur Verfügung stehenden Mittel so gut es geht ausgenutzt werden. Beim Bezug der Bilder ist darauf zu achten, dass die Geschwindigkeit bei Bildtypenkonvertierungen nicht darunter leidet.

Die Softwarearchitektur für \gls{AR} auf Smartphones unterscheidet sich von einer PC Architektur, abgesehen von der Prozessorgeschwindigkeit, grundsätzlich durch die Menge und Bandbreite des Arbeitsspeichers. Dies bedeutet, dass alle Arbeitsschritte durch eine ineffektive Ausnutzung des zur Verfügung stehenden Speichers verlangsamt werden.

Die Software kann nach gängigen Entwurfskriterien entworfen werden wobei die Vor- und Nachteile von dynamischen und statischen Bibliotheken berücksichtigt werden müs\-sen.

% section problemstellung (end)
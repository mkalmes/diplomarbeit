\section{Problemstellung} % (fold)
\label{sec:problemstellung}
\begin{comment}
	Problemstellung: Problemstellung und Frage im Detail erläutern
\end{comment}

Die Erforschung von \gls{AR} Verfahren und Anwendungen wurde in der Vergangenheit auf leistungsfähigen Computern
 durchgeführt. Die Untersuchung der Verfahren auf mobilen Geräten ist ein noch neuer Bereich, der durch die Limitierung
 der Hardware eine Herausforderung darstellt.

Um eine bitonale Marke zu erkennen, werden Verfahren zur Erkennung von Linien eingesetzt, die aus dem Bereich der
 digitalen Bildverarbeitung stammen. Anders als bei der herkömmlichen Bildverarbeitung muss Bildverarbeitung für
 \gls{AR} die Verarbeitungsschritte so schnell ausführen, dass eine Analyse eines Bildes in Echtzeit stattfinden kann.
 Bei Bildverarbeitung außerhalb der \gls{AR} ist dies kein notwendiges Kriterium.

Der Bilddatenbeschaffung über den Kamerasensor eines Smartphones muss so schnell wie möglich erfolgen um für die
 Weiterverarbeitung verwendet werden zu können. Bei Smartphones werden in aller Regel Kamerasensoren aus dem
 Verbraucherbereich verwendet. Diese Kamerasysteme liefern nicht die Performanz einer professionellen Kamera. Dadurch
 bedingt müssen die zur Verfügung stehenden Mittel so gut es geht ausgenutzt werden.

Die Softwarearchitektur für \gls{AR} auf Smartphones unterscheidet sich von einer PC Architektur, abgesehen von der
 Prozessorgeschwindigkeit, grundsätzlich durch die Menge und Bandbreite des Arbeitsspeichers. Dies bedeutet, dass alle
 Arbeitsschritte durch eine ineffektive Ausnutzung des zur Verfügung stehenden Speichers verlangsamt werden.

Wie in \autoref{sec:forschungsstand} erwähnt, ist ARToolKitPlus das einzige aktuelle Tracking Verfahren, das im
 Quellcode verfügbar ist. Wie \citeauthor{wagner09a} festgestellt haben, eignet sich ARToolKitPlus nur noch bedingt für
 mobile Plattformen\footcite{wagner09a}, sodass sie mit Studierstube ein neues Tracking Verfahren entworfen haben. Ein
 modernes Verfahren zur Markenerkennung wurde von \citeauthor{hirzer08} vorgestellt\footcite{hirzer08}, dass im Rahmen
 dieser Arbeit implementiert wurde.

Diese Arbeit vergleicht die unterschiedlichen Tracking Verfahren von ARToolKitPlus und dem Verfahren von
 \citeauthor{hirzer08} zur Erkennung einer bitonalen Marke auf einem iPod touch (4. Generation) unter iOS 4.

Der Fokus dieser Arbeit liegt dabei auf die Einhaltung der Echtzeitbedinung von \citeauthor{azuma97}.


% section problemstellung (end)
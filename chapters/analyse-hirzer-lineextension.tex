$\mathit{extendLinesInPool}$ erweitert, wie in \autoref{sub:line_extension} beschrieben, die Linien am Anfang und am
 Ende. Das Verfahren ist in \autoref{alg:extendlinesinpool} dargestellt und benötigt als Parameter den Speicherblock
 $L$ mit Linien.

\begin{algorithm}[!ht]
\caption{\textproc{extendLinesInPool}}
\label{alg:extendlinesinpool}
\begin{algorithmic}[1]
	\Require $L$
	\State $\mathit{lineCount} \gets$ \Call{getLineCount}{$L$}
	\Cost{$c_{1}$}{$1 + \Theta(1)$}
	\For{$i \gets 0$ \textbf{to} $i < \mathit{lineCount}$}
	\Cost{$c_{2}$}{$(l + 1)$}
	\label{alg:extendlinesinpool-loop-start}
		\State $l \gets \mathit{L.data}[i]$
		\Cost{$c_{3}$}{$3l$}
		\label{alg:extendlinesinpool-line}
		\State $\mathit{slope} \gets \mathit{l.slope}$
		\Cost{$c_{4}$}{$2l$}
		\label{alg:extendlinesinpool-slope}
		\State \Call{extendLine}{$\mathit{l.end.coordinate}, \mathit{slope}, \mathit{l.end.slope},
		 \mathit{l.end.coordinate},999,I$}
		\Cost{$c_{5}$}{$l\cdot\Theta(999)$}
		\State $\mathit{slope.x} \gets \mathit{slope.x} \cdot - 1$
		\Cost{$c_{6}$}{$4l$}
		\label{alg:extendlinesinpool-slopex-invert}
		\State $\mathit{slope.y} \gets \mathit{slope.y} \cdot - 1$
		\Cost{$c_{7}$}{$4l$}
		\label{alg:extendlinesinpool-slopey-invert}
		\State \Call{extendLine}{$\mathit{l.start.coordinate}, \mathit{slope}, \mathit{l.end.slope},
		 \mathit{l.start.coordinate},999,I$}
		\Cost{$c_{8}$}{$l\cdot\Theta(999)$}
		\label{alg:extendlinesinpool-extend-start}
		\State $i \gets i + 1$
		\Cost{$c_{9}$}{$l$}
	\EndFor
	\label{alg:extendlinesinpool-loop-end}
\end{algorithmic}
\end{algorithm}


In der Schleife in Zeile \ref{alg:extendlinesinpool-loop-start}--\ref{alg:extendlinesinpool-loop-end} wird jede Linie im
 Speicherblock $L$ erweitert. Dazu wird zuerst in Zeile \ref{alg:extendlinesinpool-line} und
 \ref{alg:extendlinesinpool-slope} die Linie an Position $i$ ausgewählt und ihre Richtung ausgelesen. Danach wird durch
 \textproc{extendLine} die Linie erweitert. Die Methode \textproc{extendLine} wurde in
 \autoref{sub:linienerkennung_nach_hirzer08} beschrieben. Der Parameter $\mathit{maxLength}$ wird mit einem großen Wert
 verwendet, um die Linie soweit wie möglich zu erweitern. Danach wird die Orientierung der Linie in Zeile
 \ref{alg:extendlinesinpool-slopex-invert}--\ref{alg:extendlinesinpool-slopey-invert} umgekehrt und in Zeile
 \ref{alg:extendlinesinpool-extend-start} der Anfang der Linie erweitert.

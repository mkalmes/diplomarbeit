$\mathit{extendLinesInPool}$ erweitert, wie in \autoref{sub:line_extension} beschrieben, die Linien am Anfang und am
 Ende. Das Verfahren ist in \autoref{alg:extendlinesinpool} dargestellt und benötigt als Parameter den Speicherblock
 $L$ mit Linien.
\begin{algorithm}[!ht]\small
\caption{\textproc{extendLinesInPool}}
\label{alg:extendlinesinpool}
\begin{algorithmic}[1]
	\Require $L$
	\State $\mathit{lineCount} \gets$ \Call{getLineCount}{$L$}
	\Cost{$c_{1}$}{$1 + \Theta(1)$}
	\For{$i \gets 0$ \textbf{to} $i < \mathit{lineCount}$}
	\Cost{$c_{2}$}{$(l + 1)$}
	\label{alg:extendlinesinpool-loop-start}
		\State $l \gets \mathit{L.data}[i]$
		\Cost{$c_{3}$}{$3l$}
		\label{alg:extendlinesinpool-line}
		\State $\mathit{slope} \gets \mathit{l.slope}$
		\Cost{$c_{4}$}{$2l$}
		\label{alg:extendlinesinpool-slope}
		\State \Call{extendLine}{$\mathit{l.end.coordinate}, \mathit{slope}, \mathit{l.end.slope},
		 \mathit{l.end.coordinate},999,I$}
		\Cost{$c_{5}$}{$l\cdot\Theta(999)$}
		\State $\mathit{slope.x} \gets \mathit{slope.x} \cdot - 1$
		\Cost{$c_{6}$}{$4l$}
		\label{alg:extendlinesinpool-slopex-invert}
		\State $\mathit{slope.y} \gets \mathit{slope.y} \cdot - 1$
		\Cost{$c_{7}$}{$4l$}
		\label{alg:extendlinesinpool-slopey-invert}
		\State \Call{extendLine}{$\mathit{l.start.coordinate}, \mathit{slope}, \mathit{l.end.slope},
		 \mathit{l.start.coordinate},999,I$}
		\Cost{$c_{8}$}{$l\cdot\Theta(999)$}
		\label{alg:extendlinesinpool-extend-start}
		\State $i \gets i + 1$
		\Cost{$c_{9}$}{$l$}
	\EndFor
	\label{alg:extendlinesinpool-loop-end}
\end{algorithmic}
\end{algorithm}

In der Schleife in Zeile \ref{alg:extendlinesinpool-loop-start}--\ref{alg:extendlinesinpool-loop-end} wird jede Linie im
 Speicherblock $L$ erweitert. Dazu wird zuerst in Zeile \ref{alg:extendlinesinpool-line} und
 \ref{alg:extendlinesinpool-slope} die Linie an Position $i$ ausgewählt und ihre Richtung ausgelesen. Danach wird durch
 \textproc{extendLine} die Linie erweitert. Die Methode \textproc{extendLine} wurde in
 \autoref{sub:linienerkennung_nach_hirzer08} beschrieben. Der Parameter $\mathit{maxLength}$ wird mit einem großen Wert
 verwendet, um die Linie soweit wie möglich zu erweitern. Die Wachstumsrate $\Theta(\mathit{length})$ von
 \textproc{extendLine} wird in diesem Verfahren mit dem konstanten Wert $\mathit{maxLength} = 999$ verwendet. Danach
 wird die Orientierung der Linie in Zeile
 \ref{alg:extendlinesinpool-slopex-invert}--\ref{alg:extendlinesinpool-slopey-invert} umgekehrt und in Zeile
 \ref{alg:extendlinesinpool-extend-start} der Anfang der Linie erweitert. Die Laufzeitfunktion des Verfahrens ist für
 den schlechtesten Fall in \autoref{eq:extendlinesinpool-1} angegeben. Die Wachstumsrate ist $2013l + 3 = \Theta(l)$,
 für $c_{1} = 2012$, $c_{2} = 2013$ und $l_{0} = 1$.
\begin{subequations}
\label{eq:extendlinesinpool}
\begin{align}
\label{eq:extendlinesinpool-1}
T_{worst}(l)& =
c_{1}\bigl(1 + \Theta(1)\bigr)
+ c_{2}(l+1)
+ c_{3}3l
+ c_{4}2l
+ c_{5}\bigl(l \cdot \Theta(999)\bigr)
\\
& \quad
+ c_{6}4l
+ c_{7}4l
+ c_{8}\bigl(l \cdot \Theta(999)\bigr)
+ c_{9}l
\nonumber \\
\label{eq:extendlinesinpool-2}
T_{worst}(l)& =
\Bigl(c_{1}\bigl(1 + \Theta(1)\bigr) + c_{2}\Bigr)
+ l(c_{2} + c_{3}3 + c_{4}2 + c_{6}4 + c_{7}4 + c_{9})
\\
& \quad
+ l\bigl(c_{5}\Theta(999) + c_{8}\Theta(999)\bigr)
\nonumber
\end{align}
\end{subequations}


Ob eine erweiterte Linie zur Markenerkennung geeignet ist, wird dadurch bestimmt, ob über das Linienende hinaus ein
 heller Pixel liegt. Dazu wird das Verfahren $\mathit{findLinesWithCornersInLinePool}$ verwedent
 (\autoref{alg:findlineswithcornersinlinepool1}--\autoref{alg:findlineswithcornersinlinepool2}).

\begin{algorithm}[!ht]
\caption{\textproc{findLinesWithCornersInLinePool}}
\label{alg:findlineswithcornersinlinepool1}
\begin{algorithmic}[1]
	\Require $L,C$
	\State \Call{resetMemoryPool}{$C$}
	\label{alg:findlineswithcornersinlinepool1-clear}
	\State $w \gets \mathit{imageWidth}$ 
	\State $h \gets \mathit{imageHeight}$
	\State $n \gets$ \Call{getLineCount}{L}
	\For{$i \gets 0$ \textbf{to} $i < n$}
	\label{alg:findlineswithcornersinlinepool1-loop-start}
		\State $l \gets \mathit{pool.data}[i]$
		\label{alg:findlineswithcornersinlinepool1-line}
		\State $\mathit{dx} \gets l\mathit{.slope.x} \cdot 4$
		\label{alg:findlineswithcornersinlinepool1-dx}
		\State $\mathit{dy} \gets l\mathit{.slope.y} \cdot 4$
		\label{alg:findlineswithcornersinlinepool1-dy}
		\State $x \gets l\mathit{.start.coordinate.x} - dx$
		\label{alg:findlineswithcornersinlinepool1-x}
		\State $y \gets l\mathit{.start.coordinate.y} - dy$
		\label{alg:findlineswithcornersinlinepool1-y}
		\State $r \gets$ \Call{getRGBValue}{$I,x,y,red,w,h$}
		\State $g \gets$ \Call{getRGBValue}{$I,x,y,green,w,h$}
		\State $b \gets$ \Call{getRGBValue}{$I,x,y,blue,w,h$}
	\algstore{brk-findlineswithcornersinlinepool1}
\end{algorithmic}
\end{algorithm}

\begin{algorithm}[!ht]\small
\caption{\textproc{findLinesWithCornersInLinePool} (Fortsetzung)}
\label{alg:findlineswithcornersinlinepool2}
\begin{algorithmic}[1]
	\algrestore{brk-findlineswithcornersinlinepool1}
		\If{$r > 10 \land g > 10 \land b > 10$}
		\Cost{$c_{14}$}{$5l$}
		\label{alg:findlineswithcornersinlinepool2-iswhite-start}
			\State $l\mathit{.startCorner} \gets$ \textbf{true}
			\Cost{$c_{15}$}{$2l$}
		\EndIf
		\State $x \gets l\mathit{.end.coordinate.x} + dx$
		\Cost{$c_{17}$}{$5l$}
		\label{alg:findlineswithcornersinlinepool2-lineend-start}
		\State $y \gets l\mathit{.end.coordinate.y} + dy$
		\Cost{$c_{18}$}{$5l$}
		\State $r \gets$ \Call{getRGBValue}{$I,x,y,red,w,h$}
		\Cost{$c_{19}$}{$l\bigl(1 + \Theta(1)\bigr)$}
		\State $g \gets$ \Call{getRGBValue}{$I,x,y,green,w,h$}
		\Cost{$c_{20}$}{$l\bigl(1 + \Theta(1)\bigr)$}
		\State $b \gets$ \Call{getRGBValue}{$I,x,y,blue,w,h$}
		\Cost{$c_{21}$}{$l\bigl(1 + \Theta(1)\bigr)$}
		\label{alg:findlineswithcornersinlinepool2-lineend-end}
		\If{$r > 10 \land g > 10 \land b > 10$}
		\Cost{$c_{22}$}{$5l$}
		\label{alg:findlineswithcornersinlinepool2-iswhite-end}
			\State $l\mathit{.endCorner} \gets$ \textbf{true}
			\Cost{$c_{23}$}{$2l$}
		\EndIf
		\If{$l\mathit{.startCorner} \lor l\mathit{.endCorner}$}
		\Cost{$c_{25}$}{$3l$}
		\label{alg:findlineswithcornersinlinepool2-hascorner}
			\State \Call{addLineSegment}{$C,l$}
			\Cost{$c_{26}$}{$l\bigl(\Theta(1)\bigr)$}
			\label{alg:findlineswithcornersinlinepool2-addline}
		\EndIf
		\State $i \gets i + 1$
		\Cost{$c_{28}$}{$l$}
	\EndFor
	\label{alg:findlineswithcornersinlinepool2-loop-end}
\end{algorithmic}
\end{algorithm}


Das Verfahren untersucht alle Linien in Speicherblock $L$ und speichert Linien, die sich zur Erkennung eignen, in Block
 $C$. Zu Beginn des Verfahrens wird in Zeile \ref{alg:findlineswithcornersinlinepool1-clear} der Speicherblock $C$
 gelöscht. Danach werden die lokalen Variablen initialisiert. Die Bildbreite und -höhe wird in Variable $w$ und $h$
 hinterlegt. Die Anzahl der Linien in $L$ wird in Variable $n$ gespeichert. In der Schleife in Zeile
 \ref{alg:findlineswithcornersinlinepool1-loop-start}--\ref{alg:findlineswithcornersinlinepool2-loop-end} wird jede
 Linie untersucht, indem in Zeile \ref{alg:findlineswithcornersinlinepool1-line} eine Linie an Position $i$ zuerst in
 $l$ gespeichert wird. Danach wird die Richtung der Linie in Zeile \ref{alg:findlineswithcornersinlinepool1-dx} und
 \ref{alg:findlineswithcornersinlinepool1-dy} verlängert. Die Variable $dx$ und $dy$ sind dienen als Abstand der
 Linienenden. In Zeile \ref{alg:findlineswithcornersinlinepool1-x}--\ref{alg:findlineswithcornersinlinepool1-y} wird
 die Position vor dem Startpunkt der Linie berechent. Danach werden die Farbkomponenten an dieser Position ausgelesen.

In Zeile \ref{alg:findlineswithcornersinlinepool2-iswhite-start} wird dann jeder Farbwert mit einem Schwellwert
 verglichen, um festzustellen ob an der Position ein heller gls{pixel} vorliegt. Falls dem so ist, eignet sich die
 Startposition zur Erkennung einer Linie. In Zeile
 \ref{alg:findlineswithcornersinlinepool2-lineend-start}--\ref{alg:findlineswithcornersinlinepool2-lineend-end} wird
 dann für das Linienende die Koordinate und die Farbkomponenten an dieser Position berechnet. In Zeile
 \ref{alg:findlineswithcornersinlinepool2-iswhite-end} werden dann die Farbwerte am Linienende untersucht. Zuletzt wird
 in Zeile \ref{alg:findlineswithcornersinlinepool2-hascorner} untersucht, ob der Startpunkt oder der Endpunkt zur
 Markenerkennung geeignet ist. Wenn die Untersuchung positiv ausfällt, wird in Zeile
 \ref{alg:findlineswithcornersinlinepool2-addline} die Linie $l$ in den Speicherblock $C$ hinterlegt.

Das Verfahren der Linienerweiterung wird in \textproc{lineDetection} integriert und ist in
 \autoref{alg:linedetection-hirzerextending} dargestellt.

\begin{algorithm}[!ht]\small
\caption{\textproc{lineDetection} mit Linienerweiterung}
\label{alg:linedetection-hirzerextending}
	\begin{algorithmic}[1]
		\Require $I$
		\For{$y \gets 0$ \textbf{to} $y < \mathit{imageHeight}$}
		\Cost{$c_{1}$}{$(\frac{h}{r} + 1)$}
			\For{$x \gets 0$ \textbf{to} $x < \mathit{imageWidth}$}
			\Cost{$c_{2}$}{$\frac{h}{r}(\frac{w}{r} + 1)$}
				\State \Call{findEdgels}{$I,E,x,y$}
				\Cost{$c_{3}$}{$\frac{hw}{r^2}\cdot\Theta(r^2)$}
				\State \Call{findLineSegments}{$E,L$}
				\Cost{$c_{4}$}{$\frac{hw}{r^2}\cdot\Theta(mn^2)$}
				\State \Call{mergeLines}{$L$}
				\Cost{$c_{5}$}{$\frac{hw}{r^2}\cdot\Theta(l^2\cdot\mathit{length})$}
				\State $n \gets$ \Call{getLineCount}{$L$}
				\Cost{$c_{6}$}{$\frac{hw}{r^2}\bigl(1 + \Theta(1)\bigr)$}
				\For{$i \gets 0$ \textbf{to} $i < n$}
				\Cost{$c_{7}$}{$\frac{hw}{r^2}(l + 1)$}
					\State \Call{addLineSegment}{$M,$\textproc{getLineSegment}$(L,i)$}
					\Cost{$c_{8}$}{$\frac{hw}{r^2}l\bigl(\Theta(1) + \Theta(1)\bigr)$}
					\State $i \gets i + 1$
					\Cost{$c_{9}$}{$\frac{hw}{r^2}l$}
				\EndFor
				\State \Call{resetMemoryPool}{$E$}
				\Cost{$c_{11}$}{$\frac{hw}{r^2}\cdot\Theta(1)$}
				\State \Call{resetMemoryPool}{$L$}
				\Cost{$c_{12}$}{$\frac{hw}{r^2}\cdot\Theta(1)$}
				\State $ x \gets x + 40$
				\Cost{$c_{13}$}{$2\frac{hw}{r^2}$}
			\EndFor
			\State $y \gets y + 40$
			\Cost{$c_{15}$}{$2\frac{h}{r}$}
		\EndFor
		\State \Call{mergeLines}{$M$}
		\Cost{$c_{17}$}{$\Theta(l^2\cdot\mathit{length})$}
		\State \Call{extendLinesInPool}{$M$}
		\Cost{$c_{18}$}{$\Theta(l)$}
		\label{alg:linedetection-hirzerextending-extend}
		\State \Call{findLinesWithCornersInPool}{$M,L$}
		\Cost{$c_{19}$}{$\Theta(l)$}
		\label{alg:linedetection-hirzerextending-corner}
	\end{algorithmic}
\end{algorithm}


In Zeile \ref{alg:linedetection-hirzerextending-extend} werden zuerst die Linien in $M$ erweitert. Im Anschluß daran
 werden in Zeile \ref{alg:linedetection-hirzerextending-corner} die Linien auf ihre Tauglichkeit zur Markenerkennung
 untersucht. Alle Linien die zur Erkennung geeignet sind, werden in Speicherblock $L$ gespeichert.

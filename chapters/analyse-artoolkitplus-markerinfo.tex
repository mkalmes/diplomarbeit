\subsubsection{Marken} % (fold)
\label{sub:artoolkitplus-marken}
ARToolKitPlus speichert die Informationen einer Marke in zwei einfachen Datenstrukturen. Informationen zur
 Identifizierung einer Marke werden in \textproc{MarkerInfo} gespeichert und Informationen zur Erkennung einer Marke
 werden in \textproc{MarkerInfo2} gespeichert.
% Das Identifikationsmerkmal $\mathit{id}$ wird in \autoref{alg:calc} als Rückgabewert verwendet.

\paragraph{MarkerInfo:} % (fold)
\label{par:artoolkitplus-markerinfo}
Die Datenstruktur in \autoref{alg:datastructure-markerinfo} wird nur zur Identifizierung einer Marke verwendet und wird
 zur Vollständigkeit erwähnt.
\begin{algorithm}[ht]
\caption{MarkerInfo}
\label{alg:datastructure-markerinfo}
\begin{algorithmic}[1]
	\State $\mathit{area}$
	\State $\mathit{id}$
	\State $\mathit{dir}$
	\State $\mathit{cf}$
	\State $\mathit{pos}[2]$
	\State $\mathit{line}[4][3]$
	\State $\mathit{vertex}[4][2]$
\end{algorithmic}
\end{algorithm}

Der Zugriff auf die Datenstruktur ist konstant.
% paragraph markerinfo (end)

\paragraph{MarkerInfo2:} % (fold)
\label{par:artoolkitplus-markerinfo2}
Die Variable $\mathit{area}$ in \autoref{alg:datastructure-markerinfo2} speichert den Flächeninhalt einer Marke,
 während $\mathit{pos}[2]$ die Position des Zentrums der Marke enthält.
\begin{algorithm}[!ht]
\caption{MarkerInfo2}
\label{alg:datastructure-markerinfo2}
\begin{algorithmic}[1]
	\State $\mathit{area}$
	\State $\mathit{pos}[2]$
	\State $\mathit{coord\_num}$
	\State $\mathit{x\_coord}[\mathit{AR\_CHAIN\_MAX}]$
	\State $\mathit{y\_coord}[\mathit{AR\_CHAIN\_MAX}]$
	\State $\mathit{vertex}[5]$
\end{algorithmic}
\end{algorithm}

$\mathit{coord\_num}$ enthält die Anzahl der gefundenen Konturpixel, die in $\mathit{x\_coord}$ als $x$-Koordinate und
 in $\mathit{y\_coord}$ als $y$-Koordinate gespeichert sind. Die konstante Größe $\mathit{AR\_CHAIN\_MAX}$ des
 Speichers für die Koordinaten wird zur Laufzeit nicht verändert. ARToolKitPlus erlaubt maximal $10000$ Koordinaten pro
 Marke. Die Eckpunkte einer Marke sind in $\mathit{x\_coord}$ und $\mathit{y\_coord}$ enthalten und werden durch einen
 Index in $\mathit{vertex}$ referenziert. Hierbei ist zu beachten, dass $\mathit{vertex}$ fünf Einträge speichert, wobei
 der erste und letzte Eintrag auf die gleiche Koordinate verweisen. Dadurch kann bei der Grafikprogrammierung mit OpenGL
 sehr einfach ein Rahmen um eine Marke gezeichnet werden. Der Zugriff auf die Variablen \textproc{MarkerInfo2} ist
 konstant.
% paragraph markerinfo2 (end)
% subsubsection marken (end)

\subsubsection{Vektor} % (fold)
\label{sub:vektor}

Die meisten Datenstrukturen in der Implementierung des Verfahrens nach \citeauthor{hirzer08}, benötigen als Basis die
 Datenstruktur $\textproc{vector}$. Bei der Datenstruktur handelt es sich um eine Vektor in $\mathbb{R}^2$ und benötigt
 die beiden Variablen $x$ und $y$ (Vgl. \autoref{alg:vector}). Operationen mit der Datenstruktur \textproc{vector}
 umfassen

\begin{itemize}
	\item Addition,
	\item Subtraktion,
	\item Skalarprodukt,
	\item Quadratische Länge eines Vektors,
	\item Länge eines Vektors und
	\item Normalisieren von Vektoren.
\end{itemize}

\begin{algorithm}[!ht]
\caption{\textproc{vector}}
\label{alg:vector}
	\begin{algorithmic}[1]
		\State $x$
		\State $y$
	\end{algorithmic}
\end{algorithm}


Auf die Variablen $x$ und $y$ der Datenstruktur kann direkt zugegriffen werden, sowohl lesend als auch schreibend. Zum
 schreiben der Variablen kann auch \autoref{alg:vectorsetcoordinate} verwendet werden.

\begin{algorithm}[ht]
\caption{\textproc{vectorSetCoordinate}}
\label{alg:vectorsetcoordinate}
	\begin{algorithmic}[1]
		\Require $\mathit{vector}, x, y$
		\State $\mathit{vector \to x} \gets x$
		\State $\mathit{vector \to y} \gets y$
	\end{algorithmic}
\end{algorithm}


\textproc{vectorSetCoordinate} benötigt einen Vektor, dem die Werte $x$ und $y$ hinzugefügt werden. Der Zugriff auf die
 Datenstruktur \textproc{vector} ist konstant.

\paragraph{Addition:} % (fold)
\label{par:addition}

Die Addition von zwei Vektoren ist in \autoref{alg:vectoraddition} aufgeführt. Das Verfahren benötigt zwei Vektoren
 $\mathit{left}$ und $\mathit{right}$, die addiert werden sollen. Dazu wird ein neuer Vektor $\mathit{vector}$
 initialisiert, der mit Hilfe von \textproc{vectorSetCoordinate} die addierten Werte von $\mathit{left}$ und
 $\mathit{right}$ zugewiesen bekommt, und zurückgegebn.

\begin{algorithm}[!ht]
\caption{\textproc{vectorAddition}}
\label{alg:vectoraddition}
	\begin{algorithmic}[1]
		\Require $\mathit{left}, \mathit{right}$
		\State $\mathit{vector} \gets \infty$
		\State \Call{vectorSetCoordinate}{$\mathit{vector},\mathit{left.x} + \mathit{right.x}, \mathit{left.y} + \mathit{right.y}$}
		\State \textbf{return} $\mathit{vector}$
	\end{algorithmic}
\end{algorithm}


% paragraph addition (end)

\paragraph{Subtraktion:} % (fold)
\label{par:subtraktion}

Die Subtraktion von zwei Vektoren (\autoref{alg:vectorsubtract}) ähnelt dem Verfahren der Addition. Auch hier wird
 durch die Parameter $\mathit{left}$ und $\mathit{right}$ und neuer Vektor initialisiert und an die aufrufende Funktion
 zurückgeliefert.

\begin{algorithm}[!ht]
\caption{\textproc{vectorSubtract}}
\label{alg:vectorsubtract}
	\begin{algorithmic}[1]
		\Require $\mathit{left}, \mathit{right}$
		\State $\mathit{vector} \gets \infty$
		\State \Call{vectorSetCoordinate}{$\mathit{vector},\mathit{left.x} - \mathit{right.x}, \mathit{left.y} - \mathit{right.y}$}
		\State \textbf{return} $\mathit{vector}$
	\end{algorithmic}
\end{algorithm}


% paragraph subtraktion (end)

\paragraph{Skalarprodukt:} % (fold)
\label{par:skalarprodukt}

Das Skalarprodukt der Vektoren $\mathit{left}$ und $\mathit{right}$ wird durch \autoref{alg:vectordotproduct} direkt
 zurückgegeben.

\begin{algorithm}[!ht]
\caption{\textproc{dotProduct}}
\label{alg:vectordotproduct}
	\begin{algorithmic}[1]
		\Require $\mathit{left}, \mathit{right}$
		\State \textbf{return} $\left(\mathit{left.x} \cdot \mathit{right.x}\right) + \left(\mathit{left.y} \cdot \mathit{right.y}\right)$
	\end{algorithmic}
\end{algorithm}


% paragraph skalarprodukt (end)

\paragraph{Länge eines Vektors:} % (fold)
\label{par:länge_eines_vektors}

Die quadratische Länge eines Vektors (\autoref{alg:vectorsquardlength}) benötigt als Parameter einen Vektor
 $\mathit{vector}$ und berechnet durch direkte Zugriff auf die Datenstruktur die quadratische Länge.

\begin{algorithm}[ht]
\caption{\textproc{squaredLength}}
\label{alg:vectorsquaredlength}
	\begin{algorithmic}[1]
		\Require $\mathit{vector}$
		\State \textbf{return} $\left(\mathit{vector \to x} \cdot \mathit{vector \to x}\right) + \left(\mathit{vector \to y} \cdot \mathit{vector \to y}\right)$
	\end{algorithmic}
\end{algorithm}


Das Verfahren \textproc{length} (\autoref{alg:vectorlength}) nutzt \autoref{alg:vectorsquaredlength} zur Berechnung der
 Länge eines Vektors.

\begin{algorithm}[!ht]
\caption{\textproc{length}}
\label{alg:vectorlength}
	\begin{algorithmic}[1]
		\Require $\mathit{vector}$
		\State \textbf{return} \Call{sqrt}{\textproc{squaredlength}$(\mathit{vector})$}
	\end{algorithmic}
\end{algorithm}


% paragraph länge_eines_vektors (end)

\paragraph{Normalisieren von Vektoren:} % (fold)
\label{par:normalisieren_von_vektoren}

Das Verfahren zum normalisieren von Vektoren ist in \autoref{alg:vectornormalized} beschrieben und benötigt als
 Parameter einen Vektor. Durch \textproc{length} (\autoref{alg:vectorlength}) wird die Länge des Vektors berechnet und
 direkt im Parameter gespeichert.

\begin{algorithm}[!ht]
\caption{\textproc{normalized}}
\label{alg:vectornormalized}
	\begin{algorithmic}[1]
		\Require $\mathit{vector}$
		\State $\mathit{invertedLength} \gets 1 /$ \Call{length}{$\mathit{vector}$}
		\Cost{$c_{1}$}{$2 + 9$}
		\State $\mathit{vector.x} \gets \mathit{vector.x} \cdot \mathit{invertedLength}$
		\Cost{$c_{2}$}{$4$}
		\State $\mathit{vector.y} \gets \mathit{vector.y} \cdot \mathit{invertedLength}$
		\Cost{$c_{3}$}{$4$}
	\end{algorithmic}
\end{algorithm}


% paragraph normalisieren_von_vektoren (end)

Die vorgestellten Operationen (\autoref{alg:vectorsetcoordinate}--\autoref{alg:vectornormalized}) arbeiten in konstanter Laufzeit durch den direkten Zugriff auf die Variablen der Datenstruktur \textproc{vector}.

% subsubsection vektor (end)

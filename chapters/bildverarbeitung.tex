\section{Digitale Bildverarbeitung} % (fold)
\label{sec:bildverarbeitung}
\begin{comment}
	Bildverarbeitung: Notwendige Verfahren und Konzepte erläutern.
\end{comment}

\begin{comment}
	YUV und RGB bzw. RGBA
	kCVPixelFormatType_420YpCbCr8BiPlanarVideoRange, kCVPixelFormatType_420YpCbCr8BiPlanarFullRange and kCVPixelFormatType_32BGRA, except on iPhone 3G, where the supported pixel formats are kCVPixelFormatType_422YpCbCr8 and kCVPixelFormatType_32BGRA..(AVCaptureVideoDataOutput Class Reference)

	\subsection{Grundlagen} % (fold)
	\label{sec:grundlagen}
	Ein digitales Bild ist definiert durch seine Bildgröße und der Auflösung. Die Bildgröße ist in Höhe und Breite angegeben und eine entsprechende Bildmatrix kann durch die Bildspalten $u$ und Bildzeilen $v$ angegeben werden. Die Auflösung eines Bildes bezeichnet die räumliche Ausdehnung in \gls{dpi}, die in den meisten Bildverarbeitungsschritten vernachlässigt werden kann.

Um auf ein \gls{pixel} aus der Bildmatrix zugreifen zu können, benötigen wir ein Koordinatensystem. Bei digitalen
 Bildern unterscheidet sich das verwendete Koordinatensystem von einem kartesischen Koordinatensystem dadurch, dass der
 Ursprungspunkt bei Bildern links oben liegt. Die \(x\)-Achse verläuft von links nach rechts und die \(y\)-Achse von
 oben nach unten. TODO: vgl Abbildung.

	Die Information eines \gls{pixel} ist als binärer Wert mit der Länge \(k\) gespeichert. Der Wertebereich eines \gls{pixel} umfasst \(\left[0 \dotsc 2^k\right]\), wobei der genaue Wertebereich abhängig vom eingesetzten Typ ist.

	Man unterscheidet im Allgemeinen zwischen Farb-, Monochrom- und Binärbildern, die eine direkte Auswirkung auf den Wertebereich haben. Bei Farbbildern wird häufig eine Komponente für Rot, Grün und Blau verwendet und ist typischerweise in 8 Bits kodiert. Ein Pixel besteht somit aus \(3 \cdot 8 = 24\) Bits mit einem Wertebereich von \(\left[0 \dotsc 255\right]\) pro Farbkomponente.\\TODO: Bild mit Speicherbeschreibung für RGB

	Monochrombilder bestehen nur aus einem Intensitätskanal der ebenfalls mit 8 Bits kodiert wird. Der Wertebereich eines Pixel entspricht \(\left[0 \dotsc 255\right]\).\\TODO: Bild mit Speicherbeschreibung für Monochrom

	Bei Binärbildern werden Informationen nur in einem Bit gespeichert und der Wert entspricht somit \(0\) oder \(1\) für Schwarz oder Weiß.\\TODO: Bild mit Speicherbeschreibung für Binär.
	% section grundlagen (end)
\end{comment}

Bevor ich mich den \gls{AR}-Verfahren widme, möchte ich die grundlegenden Aspekte der Digitalen Bildverarbeitung benennen, die in den nachfolgenden Kapiteln benötigt werden. Dabei werde ich keinen kompletten Überblick über Digitale Bildverarbeitung vermitteln, sondern nur Bereiche besprechen die für die Untersuchung der \gls{AR} berücksichtigt werden müssen.


\subsection{Grundlagen} % (fold)
\label{sub:grundlagen}

Ein digitales Bild ist eine Ansammlung numerischer Werte, die in einem Array gespeichert sind. Durch die Höhe und Breite des Bildes kann das Array, wie in \autoref{fig:bildmatrix}, als Matrix dargestellt werden. Die Höhe $N$ entspricht dann den Zeilen und die Breite $M$ den Spalten der Matrix. Ein Bild $I$ ist somit definiert als eine Funktion auf einer Menge von Bildwerten $\mathbb{P}$, wie in \autoref{eq:digitalesBild} dargestellt ist.

\begin{figure}
	\centering
	\input{resources/Bildmatrix.pdf_tex}
	\caption{Digitales Bild in Matrixdarstellung.}
	\label{fig:bildmatrix}
\end{figure}

\begin{equation}
	I\left(u,v\right)\in\mathbb{P} \text{ und } u,v\in\mathbb{N}
	\label{eq:digitalesBild}
\end{equation}

Um auf ein \gls{pixel} aus der Bildmatrix zugreifen zu können, bedienen wir uns eines Koordinatensystems. Im Unterschied zu einem kartesischen Koordinatensystem ist bei Bildern der Ursprungspunkt links oben. Die \(x\)-Achse verläuft von links nach rechts und die \(y\)-Achse von oben nach unten. \begin{comment}\\TODO:vgl abbildung\end{comment} Welche Werte in einem \gls{pixel} gespeichert sind und wie diese Werte zu interpretieren sind, ist abhängig vom verwendeten Bildtyp.

% section grundlagen (end)

\subsection{Bildtypen} % (fold)
\label{sub:bildtypen}

Der Wert eines \gls{pixel} ist als binärer Wert der Länge $k$ angegeben. Darüber hinaus bestimmt der verwendete Bildtyp über die weiteren Informationen eines \gls{pixel}.

Monochrome Bilder, die im allgemeinen Sprachgebrauch als Schwarz/Weiß Bilder bezeichnet werden, besitzen nur eine Komponente, die als Luminanz oder auch Intensität bezeichnet wird. Die Information des \gls{pixel} wird im Allgemeinen mit 8 Bit kodiert, sodass sich ein Wertebereich von $2^8 = \left[0\dotsc255\right]$ ergibt.

Binäre Bilder sind eine spezielle Form der monochromen Bilder. In binären Bildern werden nur Schwarz/Weiß Werte gespeichert. Die Information kann dadurch in nur einem Bit, mit $0$ für Schwarz und $1$ für Weiß, gespeichert werden.

Farbbilder speichern im Unterschied zu monochromen Bildern mehr Informationen in einem \gls{pixel} ab. Der Aufbau der
Informationen von Farbbildern ist abhängig von dem eingesetzten Farbraum, der in \autoref{sub:farbräume} erläutert
wird. In den meisten Fällen wird die Information eines Farbbildes in drei Komponenten für Rot, Grün und Blau, mit
jeweils 8 Bit, kodiert. Somit besteht ein Pixel aus \(3 \cdot 8 = 24\) Bits mit einem Wertebereich von
\(\left[0 \dotsc 255\right]\) pro Farbkomponente.

Der Aufbau der Farbinformationen ist abhängig von der Anordnung, die in \autoref{sub:pixelanordnung} genauer definiert wird.

% subsection bildtypen (end)

\subsection{Farbräume} % (fold)
\label{sub:farbräume}

Bei Digitaler Bildverarbeitung wird häufig der RGB Farbraum eingesetzt, der aus roten, grünen und blauen Komponenten besteht (\autoref{fig:rgbLenna}).

\begin{figure}[!ht]
	\centering
	\subfigure[]{
		\label{fig:rgbLenna-Farbe}
		\includegraphics[width=.2\textwidth]{resources/Lenna.pdf}
	}
	\subfigure[]{
		\label{fig:rgbLenna-Rot}
		\includegraphics[width=.2\textwidth]{resources/Lenna_R.pdf}
	}
	\subfigure[]{
		\label{fig:rgbLenna-Gruen}
		\includegraphics[width=.2\textwidth]{resources/Lenna_G.pdf}
	}
	\subfigure[]{
		\label{fig:rgbLenna-Blau}
		\includegraphics[width=.2\textwidth]{resources/Lenna_B.pdf}
	}
	\caption{Ein farbiges Bild \subref{fig:rgbLenna-Farbe} wird im RGB Farbraum in seine Komponenten Rot
		 \subref{fig:rgbLenna-Rot}, Grün \subref{fig:rgbLenna-Gruen} und Blau \subref{fig:rgbLenna-Blau} aufgeteilt.
		Von \cite{lenna}.
	}
	\label{fig:rgbLenna}
\end{figure}

Bei RGB handelt es sich um ein additives Farbsystem, dass durch Mischung der Primärfarben einen Farbton und dessen
 Helligkeit generiert. Der RGB Farbraum kann als dreidimensionaler Würfel dargestellt werden
 (\autoref{fig:rgbWuerfel}). Seine Koordinatenachsen stellen den Wertebereich für die Grundfarben Rot, Grün und Blau
 dar. Normiert man den Würfel zu einem Einheitswürfel der Länge $1$, umfasst jede Achse einen Bereich von
 \(\left[0 \dotsc 1.0\right]\). Innerhalb dieses Würfels kann durch Verschiebung der Koordinate jeder Farbton generiert
 werden. Die Grundfarben können in diesem Würfel durch die Punkte $R = \left(1,0,0\right)$, $G = \left(0,1,0\right)$
 und $B = \left(0,0,1\right)$ dargestellt werden. Schwarz ($S = \left(0,0,0\right)$) und Weiß
 ($W = \left(1,1,1\right)$) werden auf die gleiche Weise dargestellt.

\begin{figure}[!ht]
	\centering
	\def\svgwidth{.2\columnwidth}
	\input{resources/RGBWuerfel.pdf_tex}
	\caption{RGB Farbwürfel}
	\label{fig:rgbWuerfel}
\end{figure}

\begin{table}[!ht]
	\begin{center}
	\begin{tabular}[]{r|c|c|c}
	Farbe & R & G & B \\ \hline\hline
	rot & 1.0 & 0.0 & 0.0 \\
	grün & 0.0 & 1.0 & 0.0 \\
	blau & 0.0 & 0.0 & 1.0 \\
	schwarz & 0.0 & 0.0 & 0.0 \\
	weiß & 1.0 & 1.0 & 1.0 \\
	grau & 0.5 & 0.5 & 0.5 \\
	\end{tabular}
	\caption{RGB Werte}
 	\label{tbl:rgbwerte}
	\end{center}
\end{table}

YUV und YCbCr sind standardisierte Formate zur Aufnahme, Darstellung und Übertragung von Bildern im Fehrnsehbereich und
stellen andere Farbräume dar. YUV findet Verwendung bei analogen PAL und NTSC Systemen während YCbCr bei digitalen
Systemen verwendet wird. Beide Farbräume unterteilen die Informationen in eine Luminanzkomponente $Y$ und zwei
Komponenten zur Darstellung unterschiedlicher \gls{chrominanz} (\autoref{fig:yuvLenna}).

\begin{figure}[!ht]
	\centering
	\subfigure[]{
		\label{fig:yuvLenna-Farbe}
		\includegraphics[width=.2\textwidth]{resources/Lenna.pdf}
	}
	\subfigure[]{
		\label{fig:yuvLenna-Y}
		\includegraphics[width=.2\textwidth]{resources/Lenna_Y.pdf}
	}
	\subfigure[]{
		\label{fig:yuvLenna-U}
		\includegraphics[width=.2\textwidth]{resources/Lenna_U.pdf}
	}
	\subfigure[]{
		\label{fig:yuvLenna-V}
		\includegraphics[width=.2\textwidth]{resources/Lenna_V.pdf}
	}
	\caption{Im YUV Farbraum wird ein Bild \subref{fig:yuvLenna-Farbe} in Luminanz $Y$ \subref{fig:yuvLenna-Y},
		Chroma $U$ \subref{fig:yuvLenna-U} und Chroma $V$ \subref{fig:yuvLenna-Y} aufgeteilt. Von
		 \cite{lenna}.
	}
	\label{fig:yuvLenna}
\end{figure}

Bei YUV ist die Chromakomponente $U$ als Differenz zwischen der Luminanz $Y$ und dem Blauanteil definiert (Vgl.~\autoref{eq:uchroma}).

\begin{equation}
	U = 0.492 \cdot \left(B-Y\right)
	\label{eq:uchroma}
\end{equation}

Die Chromakomponente $V$ definiert die Differenz von Luminanz $Y$ und dem Rotanteil (Vgl.~\autoref{eq:vchroma}).

\begin{equation}
	V = 0.877 \cdot \left(R-Y\right)
	\label{eq:vchroma}
\end{equation}

Die Chromakomponenten $Cb$ und $Cr$ von YCbCr sind, wie bei YUV, Differenzwerte des Blau- und Rotanteil und der Luminanz. Der Unterschied zwischen YUV und YCbCr besteht in der unterschiedlichen Berechnung der \gls{chrominanz} und ist in \citeauthor{burger05}\footcite[][S.265--266]{burger05} beschrieben.

Durch die Trennung der Luminanz und \gls{chrominanz} ist es möglich das Signal auf Schwarz/Weiß Fernsehern zu nutzen, indem nur die Luminanzkomponente berücksichtigt wird. Bei Farbfernsehern werden die zusätzlichen Chromakomponenten verwendet.

% subsection farbräume (end)

\subsection{Pixel Anordnung} % (fold)
\label{sub:pixelanordnung}
Bei \gls{pixel} gibt es unterschiedliche Modelle der Anordnung von Farbkomponenten. Man unterscheidet die Komponentenanordnung, auch planare Anordnung genannt, von der gepackten Anordnung. Der Zugriff auf die Farbinformationen ist dabei abhängig vom eingesetzten Farbraum. Im weiteren Verlauf wird die gepackte Anordnung für den RGB Farbraum erläutert. Für YCbCr sind die vorgestellten Modelle anzupassen.

Bei der planaren Anordnung werden die Farbkomponenten in jeweils eigenen Arrays mit gleicher Größe gespeichert. Ein Bild

\begin{equation}
	I = \left(I_R, I_G, I_B\right)
	\label{eq:planarImage}
\end{equation}

besteht aus den den drei Luminanzbildern $I_R$, $I_G$ und $I_B$ (\autoref{fig:planareAnordnung}). Der Zugriff auf ein \gls{pixel} erfolgt über das Auslesen aller drei Arrays, wie in \citeauthor{burger05}\footcite[Vgl.][S.~235--236]{burger05} beschrieben. Der Zugriff erfolgt durch

\begin{equation}
	\begin{pmatrix}
		R\\
		G\\
		B
	\end{pmatrix}
	\leftarrow
	\begin{pmatrix}
		I_R\left(u,v\right)\\
		I_G\left(u,v\right)\\
		I_B\left(u,v\right)
	\end{pmatrix}.
	\label{eq:readPlanarImage}
\end{equation}

\begin{figure}[!ht]
	\centering
	\input{resources/planareAnordnung.pdf_tex}
	\caption{Planare Anordnung. Information der Pixel sind in den Matritzen $I_R$, $I_G$ und $I_B$ hinterlegt.}
	\label{fig:planareAnordnung}
\end{figure}

Bei der gepackten Anordnung sind alle Farbkomponenten in einem \gls{pixel} gespeichert und in einem Array hinterlegt (\autoref{fig:gepackteAnordnung}). Die gepackte Anordnung ist mit

\begin{equation}
	I\left(u,v\right) = \left(R,G,B\right)
	\label{eq:packedImage}
\end{equation}

definiert.

\begin{figure}[!ht]
	\centering
	\input{resources/gepackteAnordnung.pdf_tex}
	\caption{Die gepackte Anordnung speichert die Farbinformationen eines Pixels an einer Stelle.}
	\label{fig:gepackteAnordnung}
\end{figure}

Im allgemeinen wird ein Element an Stelle $(u,v)$ eines Bildes $I$ durch \eqref{eq:readPackedRGB} zugegriffen.

\begin{equation}
	\begin{pmatrix}
		R\\
		G\\
		B
	\end{pmatrix}
	\leftarrow
	\begin{pmatrix}
		Red\bigl(I\left(u,v\right)\bigr)\\
		Green\bigl(I\left(u,v\right)\bigr)\\
		Blue\bigl(I\left(u,v\right)\bigr)
	\end{pmatrix}
	\label{eq:readPackedRGB}
\end{equation}

$Red()$, $Green()$ und $Blue()$ sind abhängig vom eingesetzten Format\footcite[Vgl.][S.~236--237]{burger05}.
% subsection pixelanordnung (end)

\subsection{Filter und Faltung} % (fold)
\label{sub:filter}
Ein Filter ist eine Operation, mit dessen Hilfe ein Eingangsbild $I$ durch eine mathematische Abbildung in ein
 Ausgangsbild $I'$ überführt wird. Im Gegensatz zu Punktoperationen operieren Filter auf Regionen um, zum Beispiel,
 Bilder zu glätten oder zu schärfen (Vgl.~\autoref{fig:filterbeispiel}).

\begin{figure}[!ht]
	\centering
	\subfigure[]{
		\label{fig:filter-original}
		\includegraphics[scale=1]{resources/Filter-Original.pdf}
	}
	\subfigure[]{
		\label{fig:filter-verwischen}
		\includegraphics[scale=1]{resources/Filter-Verwischen.pdf}
	}
	\subfigure[]{
		\label{fig:filter-schaerfen}
		\includegraphics[scale=1]{resources/Filter-Schaerfen.pdf}
	}
	\caption{Das orignial Bild \subref{fig:filter-original} wurde in
	 \subref{fig:filter-verwischen} verwischt und in \subref{fig:filter-schaerfen} geschärft.}
	\label{fig:filterbeispiel}
\end{figure}

Aus einer Region $R_{u,v}$ eines Eingangsbild $I$ wird der neue Pixelwert $I'(u,v)$ berechnet. Die Größe der
 Filterregion bestimmt die Anzahl der Pixel aus $I$, die zur Berechnung des neuen Pixelwerts $I'(u,v)$ verwendet
 werden (Vgl.~\autoref{fig:region-filter}). Üblicherweise werden $3 \times 3$ oder $5 \times 5$ Filter verwendet, aber
 auch größere Filter mit $21 \times 21$ Pixeln sind möglich. Anhand der Schreibweise der Filter erkennt man, das es
 sich bei den Filtern um Matrizen handelt.

\begin{figure}[!ht]
	\centering
	\input{resources/Region-Filter.pdf_tex}
	\caption{Anwendung eines Filters. Aus der Region $R_{u,v}$ wird der neue Wert $I'(u,v)$ berechnet.}
	\label{fig:region-filter}
\end{figure}

Bei einer Filtermatrix wird ein eigenes Koordinatensystem verwendet, dessen Ursprung in der Mitte der Matrix liegt.
 Aufgrund dieses Koordinatensystem sind die Koordinaten sowohl positiv als auch negativ. Bei einer $3 \times 3$
 Filtermatrix $H$ sähen die absoluten Koordinaten an der Stelle $(i,j)$ wie folgt aus:

\begin{equation}
	H(i,j) =
	\begin{pmatrix}
		\left(i-1, j-1\right)&	\left(i, j-1\right)&	\left(i+1, j-1\right)\\
		\left(i-1, j\right)& 	\left(i, j\right)&		\left(i+1, j\right)\\
		\left(i-1, j+1\right)&	\left(i, j+1\right)&	\left(i+1, j+1\right)
	\end{pmatrix}
\end{equation}

Somit können alle Pixel von $I'$ durch

\begin{equation}
	I'\left(u,v\right) \gets
	\sum \limits_{\left(i = -1\right)}^{i = 1}
	\sum \limits_{\left(j = -1\right)}^{j = 1}
	I\left(u + i, v + j\right) \cdot H\left(i,j\right)
\end{equation}

berechnet werden. Für Filter mit einer anderen Größe als $3 \times 3$, lautet die Formel

\begin{equation}
	I'\left(u,v\right) \gets
	\sum_{\left(i,j\right)\in\mathbb{R}} I\left(u + i, v + j\right) \cdot H\left(i,j\right)
\end{equation}

wobei $\mathbb{R}$ die Größe des Filters angibt\footcite[Vgl.][S.~92--93]{burger05}.

Filter können an den Rändern nicht mathematisch korrekt berechnet werden\footcite[Vgl.][S.~113]{burger05}, da der Filter $H$ an den Bildrändern über die definierte Bildmenge hinausragt. Um dieses Problem zu umgehen, gibt es die Möglichkeit

\begin{enumerate}
	\item am Randbereich einen konstanten Wert zuzuweisen,\label{konstant}
	\item den ursprünglichen Bildwert beizubehalten,\label{nix}
	\item die Randwerte zu berechnen, indem
	\begin{enumerate}
		\item außerhalb des Bildbereichs konstante Werte angenommen werden,\label{berechneKonstant}
		\item die Randpixel fortgesetzt werden,\label{berechneRand}
		\item die Bildwerte wiederholt werden.\label{berechneBild}
	\end{enumerate}
\end{enumerate}

Bei \autoref{konstant} wird der sichtbare Bildbereich verkleinert und bei \autoref{nix} der Filter an den Rändern nicht
 angewendet. Bei \autoref{berechneKonstant} kann durch die Zuweisung eines konstanten Wertes, beispielsweise Schwarz,
 das Ergebnis verfälschen. \autoref{berechneRand} weist die geringste Verfälschung auf. Bei \autoref{berechneBild} wird
 durch die Wiederholung des Bildes implizit eine periodische Funktion betrachtet. Die möglichen Randbehandlungen sind
 in \autoref{fig:randbeispiel} illustriert. Welche dieser Randbetrachtung eingesetzt wird ist abhängig von den
 benötigten Ergebnissen und Filtern und muss für jede Anwendung gesondert betrachtet werden.

\begin{figure}[!ht]
	\centering
	\subfigure[]{
		\label{fig:randbeispiel-original}
		\includegraphics[scale=1]{resources/Rand-Original.pdf}
	}
	\subfigure[]{
		\label{fig:randbeispiel-1}
		\includegraphics[scale=1]{resources/Rand-1.pdf}
	}
	\subfigure[]{
		\label{fig:randbeispiel-2}
		\includegraphics[scale=1]{resources/Rand-2.pdf}
	}
	\subfigure[]{
		\label{fig:randbeispiel-3a}
		\includegraphics[scale=1]{resources/Rand-3a.pdf}
	}
	\subfigure[]{
		\label{fig:randbeispiel-3b}
		\includegraphics[scale=1]{resources/Rand-3b.pdf}
	}
	\subfigure[]{
		\label{fig:randbeispiel-3c}
		\includegraphics[scale=1]{resources/Rand-3c.pdf}
	}
	\caption{Filter und Randbehandlung. Das Originalbild \subref{fig:randbeispiel-original} wird je nach eingesetzen
	 Verfahren unterschiedlich behandelt. Bei \subref{fig:randbeispiel-1} werden die Ränder nicht betrachtet und ein
	 konstanter Wert zugewiesen (hier schwarz dargestellt).	Die Originalwerte des Eingangsbilds werden bei
	 \subref{fig:randbeispiel-2} beibehalten. In \subref{fig:randbeispiel-3a} werden ausserhalb des Bildes konstante
	 Werte verwendet (hier schwarz dargestellt). Die Randpixel werden bei \subref{fig:randbeispiel-3b} fortgesetzt,
	 während bei \subref{fig:randbeispiel-3c} das Eingangsbild an den Rändern wiederholt wird.}
	\label{fig:randbeispiel}
\end{figure}

In der digitalen Bildverarbeitung basieren Filter auf der mathematischen Operation der Faltung
\footcite[Vgl.][S.~101--104]{burger05}, bei der für zwei Funktionen ein dritte Funktion erzeugt wird. Die diskrete
 lineare Faltung ist definiert als

\begin{equation}
	I'\left(u,v\right) =
	\sum \limits_{i = -\infty}^{\infty}
	\sum \limits_{j = -\infty}^{\infty}
	I\left(u - i, v - j\right) \cdot H\left(i,j\right)
\end{equation}

und gekürzt

\begin{equation}
	I' = I * H,
\end{equation}

wobei $I$ dem Bildsignal und $H$ dem Faltungskern entspricht. Die Operation der Faltung ist die Grundlage aller
 linearen Filter in der digitalen Bildverarbeitung. Unterschiedliche Filter sind nur durch den Filterkern (Kernel)
 definiert. Die Faltungsoperation besitzt die algebraischen Eigenschaften der Kommutativität, Assoziativität und
 Distributivität.

\begin{align}
	&\text{Kommutativität: } &I * H = H * I\\
	&\text{Assoziativität: } &A * (B * C) = (A * B) * C\\
	&\text{Distributivität: } &H * \left(I_1 + I_2\right) = \left(H * I_1\right) + \left(H * I_2\right)
\end{align}

Durch die Assoziativität kann ein Kernel $H$ als Produkt mehrerer kleiner Kernel ausgedrückt werden was wiederum
 Rechenoperationen einspart. Ein $3 \times 3$ Filter kann in eine $x$-Richtung und $y$-Richtung aufgeteilt werden.

\begin{align}
	H_{xy} = & H_x * H_y\\
	\begin{pmatrix}
		1& 1& 1\\
		1& 1& 1\\
		1& 1& 1
	\end{pmatrix} = &
	\begin{pmatrix}
		1& 1& 1
	\end{pmatrix}
	*
	\begin{pmatrix}
		1\\
		1\\
		1
	\end{pmatrix}
\end{align}

Bei der Berechnung von $I * H_{xy}$ entspricht dies $3 \cdot 3 = 9$ Multiplikationen und bei der Berechnung von
 $I * H_x * H_y = 3 + 3 = 6$ Multiplikationen.

% subsection filter (end)

\subsection{Regionen in Binärbilder} % (fold)
\label{sec:regionen_in_binärbilder}

Binärbilder enthalten, wie in \autoref{sub:bildtypen} bereits beschrieben, nur zwei Werte, die wir auch als Vordergrund
 und Hintergrund bezeichnen können. Das Interesse bei einem Binärbild gilt dementsprechend den Informationen im
 Vordergrund. Eine zusammenhängende Struktur von Vordergrundpixeln wird als Bildregion bezeichnet. Um diese
 Bildregionen in einem Binärbild zu erkennen, werden zusammenhängende \gls{pixel} makiert, was auch als
 \textit{region labeling}\footcite[Vgl.][S.~196]{burger05} bekannt ist. Zur Makierung von \gls{pixel} werden folgende
 numerischen Werte verwendet:

\begin{equation*}
	I(u,v) = \begin{cases}
	0 & \textrm{Hintergrund}\\
	1 & \textrm{Vordergrund}\\
	2,3,\ldots,n & \textrm{Regionenmakierung}
	\end{cases}
\end{equation*}

Bei der sequentiellen Regionenmakierung\footcite[Vgl.][S.~200--206]{burger05} wird ein Binärbild in zwei Schrittten
 untersucht. Im ersten Schritt werden im Binärbild vorläufige Markierungen für Regionen gespeichert und im zweiten
 Schritt werden mehrfache Makierungen für eine Region aufgelöst.

\subsubsection{Vorläufige Makierung} % (fold)
\label{sec:vorläufige_makierung}

Zuerst muss das Binärbild zeilenweise von oben nach unten auf das Vorhandensein von Vordergrundpixel untersucht werden.
Die Nachbarn eines angrenzenden \glspl{pixel} werden, je nach Definition der Nachbarschaftsbeziehung, ebenfalls
 untersucht. Zwei gebräuchliche Definitonen von Nachbarschaftsbeziehungen sind zum einen die 4er-Nachbarschaft und zum
 anderen die 8er-Nachbarschaft. Für ein \gls{pixel} $I(u,v)$ werden bei der 4er-Nachbarschaft die beiden angrenzenden
 \gls{pixel} $I(u-1,v)$ und $I(u,v+1)$ untersucht. Bei der 8er-Nachbarschaft werden die vier Nachbarn $I(u-1,v)$,
 $I(u-1,v+1)$, $I(u,v+1)$ und $I(u+1,v+1)$ untersucht. Beide Nachbarschaftsbeziehungen sind in
 \autoref{fig:nachbarschaft} abgebildet.

\begin{figure}[!ht]
	\centering
	\subfigure[]{
		\label{fig:4er-nachbarschaft}
		\input{resources/4er-Nachbarschaft.pdf_tex}
	}
	\subfigure[]{
		\label{fig:8er-nachbarschaft}
		\input{resources/8er-Nachbarschaft.pdf_tex}
	}
	\caption{Nachbarschaftsbeziehung. \subref{fig:4er-nachbarschaft} 4er-Nachbarschaft mit $N_1 = I(u-1,v)$ und
	 $N_2 = I(u,v+1)$. \subref{fig:8er-nachbarschaft} 8er-Nachbarschaft mit $N_1 = I(u-1,v)$, $N_2 = I(u-1,v+1)$,
	 $N_3 = (u,v+1)$ und $N_4 = I(u+1,v+1)$.}
	\label{fig:nachbarschaft}
\end{figure}

Die Zuweisung einer Markierung für einen \gls{pixel} ist davon abhängig ob

\begin{enumerate}
	\item alle Nachbarn Hintergrundpixel sind, \label{labeling-all-background}
	\item genau ein Nachbar hat eine Makierung hat oder \label{labeling-one-neighbour}
	\item mehrere Nachbarn eine Makierung haben. \label{labeling-many-neighbours}
\end{enumerate}

Wenn alle Nachbarn Hintergrundpixel sind (\autoref{labeling-all-background}), wird eine Makierung an Position $(u,v)$
 geschrieben. Bei \autoref{labeling-one-neighbour} wird die Makierung des Nachbarn übernommen. Im letzten Fall müssen
 die Makierungen miteinander verglichen werden. Wenn alle Nachbarn die gleiche Makierung besitzen wird sie für Position
 $(u,v)$ übernommen. Falls es sich aber um verschiedene Makierungen handelt, spricht man von einer Kollision der
 Makierungen. Eine Kollision bedeutet, dass eine zusammenhängende Regionen durch zwei unterschiedliche Makierungen
 dargestellt wird (Vgl.~\autoref{fig:kollision}). Dem \gls{pixel} an Position $(u,v)$ wird eine Makierung eines Nachbarn
 zugewiesen und die Kollision wird vermerkt. Nach \citeauthor{burger05} werden diese Kollisionen in einer dynamischen
 Datenstruktur gespeichert und zu einem späteren Zeitpunkt bearbeitet\footcite[Vgl.][S.~203--204]{burger05}.

\begin{figure}[!ht]
	\centering
	\input{resources/Kollision.pdf_tex}
	\caption{Beispiel einer Kollision zwischen Markierung $2$ und Markierung $3$.}
	\label{fig:kollision}
\end{figure}

Nach diesem ersten Schritt ist das Binärbild verarbeitet und allen Vordergrundpixeln wurde eine vorläufige Markierung
 zugeteilt. Dabei wurden auftretende Kollisionen der Markierungen gespeichert. Der Ablauf des ersten Schritts ist in
 \autoref{fig:markierung} dargestellt.

\begin{figure}[!ht]
	\centering
	\subfigure[]{
		\label{fig:markierung-binaer}
		\input{resources/Binaerbild.pdf_tex}
	}
	\subfigure[]{
		\label{fig:markierung-1}
		\input{resources/Regionenmarkierung-1.pdf_tex}
	}
	\subfigure[]{
		\label{fig:markierung-2}
		\input{resources/Regionenmarkierung-2.pdf_tex}
	}
	\subfigure[]{
		\label{fig:markierung-3}
		\input{resources/Regionenmarkierung-3.pdf_tex}
	}
	\caption{Vorläufige Regionenmarkierung. Ein Binärbild \subref{fig:markierung-binaer} wird zeilenweise von
	 oben nach unten durchlaufen. Eine 8er-Nachbarschaft vergleicht die angrenzenden Pixel \subref{fig:markierung-1}.
	 Eine Kollision zwischen zwei Markierungen wird regestriert und der kleinere Wert zugewiesen bis das Binärbild
	 vollständig verarbeitet wurde \subref{fig:markierung-2}--\subref{fig:markierung-3}.}
	\label{fig:markierung}
\end{figure}

% subsubsectionsection vorläufige_makierung (end)

\subsubsection{Auflösung von Kollisionen} % (fold)
\label{sec:auflösung_von_kollisionen}

In diesem Schritt müssen nun die gespeicherten Kollisionen der Markierungen aufgelöst werden, damit eine
 zusammenhängende Bildregion nur durch eine Markierung repräsentiert wird. \citeauthor{burger05} beschreiben diese
 Aufgabe  als nicht trivial, da kollidierende Regionen auch durch weitere Region zusammenhängen
 können\footcite[Vgl.][S.~205]{burger05} (Vgl.~\autoref{fig:markierung-3}).

Die Menge der im letzten Schritt verwendeten vorläufigen Makierungen wird verwendet, um die Menge der Kollisionen
 zu vereinen. Dazu wird eine Kollision der Markierung von $a$ und $b$ zu einer Menge zusammengeführt. Das bedeutet,
 dass alle \gls{pixel} mit Markierung $b$ zu $a$ gehören. Danach wird das vorläufig markierte Binärbild erneut
 durchlaufen. Jede vorläufige Makierung wird nun mit der Menge der neuen Markierung verglichen und eine eindeutige
 Markierung zugewiesen.

Betrachten wir \autoref{alg:resolve-label-collision} an folgendem Beispiel: In Zeile
 \ref{resolve-label-collsion-partition-start}--\ref{resolve-label-collsion-partition-end} wird die Menge $L$ der
 vorläufigen Markierungen aufgeteilt und Kollisionen aufgelöst. Nehmen wir an, dass wir eine Kollision $C_i = (2,3)$
 auflösen wollen, wird die Mengen $R_2 = \{2\}$ und $R_3 = \{3\}$  zu $R_2 = \{2,3\}$ vereint
 (Zeile \ref{resolve-label-collsion-set-start}--\ref{resolve-label-collsion-set-end}). Die Menge $R_3$ ist danach leer.
 Nun werden in Zeile \ref{resolve-label-collsion-relabel-start}--\ref{resolve-label-collsion-relabel-end} alle
 Makierungen untersucht. Wird dabei eine Makierung an Position $(u,v)$ gefunden, die zur Menge $R_2 = \{2,3\}$ gehört,
 wird der kleinste Wert von $R_2$ an Position $(u,v)$ geschrieben (Zeile \ref{resolve-label-collsion-set-new-label}).

\begin{algorithm}[!ht]\small
\caption{Kollisionen auflösen}
\label{alg:resolve-label-collision}
\begin{algorithmic}[1]
	\Require $I_b,L,C$
	\State $L = \{2,3,\ldots,n\}$ sind die vorläufigen Markierungen
	\State $R \gets [\{2\},\{3\},\ldots,\{n\}]$, sodass $R_i = {i}$ für alle $\{i\} \in L$
	\For{alle Kollisionen $(a,b)$ in $C$}
	\label{resolve-label-collsion-partition-start}
		\State $R_a \gets$ die Menge mit Markierung $a$
		\State $R_b \gets$ die Menge mit Markierung $b$
		\If{$R_a \neq R_b$}
		\label{resolve-label-collsion-set-start}
			\State $R_a \gets R_a \cup R_b$
			\State $R_b \gets \{\}$
		\EndIf
		\label{resolve-label-collsion-set-end}
	\EndFor
	\label{resolve-label-collsion-partition-end}
	\For{alle Pixel $(u,v)$ in $I_b$}
	\label{resolve-label-collsion-relabel-start}
		\If{$I_b(u,v) > 1$}
			\State Suche in $R$ die Menge $R_i$ die Markierung $I_b(u,v)$ enthält
			\State Ersetze Markierung in $I_b(u,v)$ mit $\textrm{min}(R_i)$
			\label{resolve-label-collsion-set-new-label}
		\EndIf
	\EndFor
	\label{resolve-label-collsion-relabel-end}
\end{algorithmic}
\end{algorithm}


Nach diesem letzten Schritt sind alle zusammenhängende Regionen in einem Binärbild eindeutig gekennzeichnet und können nun mit anderen Verfahren weiterverarbeitet werden um, beispielsweise, die Form einer Region zu erkennen.

% subsubsectionsection auflösung_von_kollisionen (end)

% subsection regionen_in_binärbilder (end)

% section bildverarbeitung (end)

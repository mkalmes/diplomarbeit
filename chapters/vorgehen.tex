\section{Vorgehen} % (fold)
\label{sec:vorgehen}
\begin{comment}
	Vorgehen: Analysemethoden vorstellen wie Algorithmen untersucht werden.
	Vergleich O-Notation
	Laufzeitanalyse
	Gleiche Kriterien (selbes Bild, selbes Video)
\end{comment}

Um das Verfahren ARToolKitPlus und das Verfahren von \citeauthor{hirzer08} zu analysieren werden die Kosten der
 Algorithmen berechent und ihre Wachstumsrate ermittlet. Elementare Operationen, wie Addition, Zuweisung und
 Arrayzugriffe, werden in Abhängigkeit von der Größe der Eingabemenge berechnet. Daraus wird wiederum das aymptotische
 Wachstum ermittelt.

Bei Verfahren, deren Algorithmen nicht im Quellcode vorliegen, wird durch eine experimentelle Analyse die Laufzeit des
 Verfahrens für unterschiedliche Eingabemengen untersucht. Die Durchführung der experimentellen Analyse erfolgt dabei
 auf einem iPod touch (4. Generation) mit einem ARMv7 Befehlssatz. Alle Programme der Analyse wurden mit LLVM GCC 4.2
 und Kompileroptimierung $\mathit{-Os}$\footcite[Vgl.][]{cc} erstellt.

\paragraph{Bilddatenbeschaffung:} % (fold)
\label{par:bilddatenbeschaffung}
Das Videosignal wird unter iOS durch das AVFoundation Framework zur Verfügung gestellt und ermöglicht durch Delegation
 den Zugriff auf einzelne Frames\footcite{avfoundation}. AVFoundation erlaubt die Einstellung von fünf verschiedenen
 Auflösungen für Videosignale, von Low, Medium, High über $640 \times 480$ bis zu $1280 \times 720$. Ferner unterstützt
 AVFoundation die Bildformate $YCbCr$ mit $8$-Bit und $BGRA$ mit $32$-Bit.
% paragraph bilddatenbeschaffung (end)

\paragraph{Vergleichsanalyse:} % (fold)
\label{par:vergleichsanalyse}
Um beide Verfahren miteinander zu vergleichen wird die Laufzeit der Programme gemessen. Dazu wird durch hinzufügen von
 Anwendungsschritten, von der Bildbeschaffung hin zur Erkennung einer Marke, eine Laufzeit erstellt. Dieses Verfahren
 wurd von \citeauthor{wagner09b}\footcite[Vgl.][]{wagner09b} verwendet um Studierstube 4 auf unterschiedlichen Geräten
 zu beurteilen. In meiner Analyse wird mit der gleichen Technik ARToolKitPlus und das Verfahren von
 \citeauthor{hirzer08} auf einem iPod touch untersucht.
% paragraph vergleichsanalyse (end)

\paragraph{ARToolKitPlus:} % (fold)
\label{par:artoolkitplus}
Die Untersuchung von ARToolKitPlus unter iOS 4 erfolgt mit VRToolKit\footcite[Vgl.][]{vrtoolkit} von \citeauthor{vrtoolkit}. VRToolKit ist ein Obj-C Wrapper für ARToolKitPlus. VRToolKit konfigueriert das ARToolKitPlus System zur Verarbeitung von Marken und sorgt für die Bereitstellung von Bildsignalen der Kamera.
% paragraph artoolkitplus (end)

\paragraph{Verfahren nach \citeauthor{hirzer08}:} % (fold)
\label{par:verfahren_nach_hirzer}
Das Verfahren von \citeauthor{hirzer08} ist eine Eigenentwicklung in Obj-C und C, die nach dem Konzept von VRToolKit und
 dem Softwarehersteller infi\footcite[Vgl.][]{infi} entworfen wurde. Die Implementierung sorgt für die Bereitstellung
 und Verarbeitung des Bildsignals.
% paragraph verfahren_nach_hirzer (end)

% section vorgehen (end)
\textproc{arDetectMarker} (Vgl. \autoref{alg:detectmarker}) ist der Einstiegspunkt der Markenerkennung und benötigt das
 Bildsignal $I$, den Schwellwert $\mathit{tresh}$ sowie $\mathit{marker\_info}$ und $\mathit{marker\_num}$.
\begin{algorithm}[!ht]
\caption{\textproc{arDetectMarker}}
\label{alg:detectmarker}
\begin{algorithmic}[1]
	\Require $I,\mathit{thresh},\mathit{marker\_info},\mathit{marker\_num}$
	\State $I_l \gets$ \textproc{NULL}
	\label{alg:detectmarker-init-start}
	\State $\mathit{label\_num},\mathit{area},\mathit{clip},\mathit{label\_ref},\mathit{pos} \gets \infty$
	\label{alg:detectmarker-init-end}
	\State \Call{autoThreshold.reset}{}
	\label{alg:detectmarker-call-autothreshold}
	\State \Call{checkImageBuffer}{}
	\label{alg:detectmarker-call-imagebuffer}
	\State $\mathit{marker\_num} \gets 0$
	\State $I_l \gets$ \Call{arLabeling}{$I,\mathit{thresh},\mathit{label\_num},\mathit{area},\mathit{pos},\mathit{clip},\mathit{label\_ref}$}
	\label{alg:detectmarker-call-labeling}
	% AR_AREA_MAX (100000) und AR_AREA_MIN (70) sind defines
	% wmarker_num globale variable
	\If{$I_l$}
	\label{alg:detectmarker-check-il-start}
		\State $\mathit{marker\_info2} \gets$ \textproc{arDetectMarker2}$\left(
		\begin{aligned}
				& I_l,\mathit{thresh},\mathit{label\_ref},\mathit{area},\mathit{pos},\mathit{clip},\\
				& \mathit{AR\_AREA\_MAX},\\
				& \mathit{AR\_AREA\_MIN},\\
				& 1.0, \mathit{wmarker\_num}
		\end{aligned}\right)$
		\label{alg:detectmarker-call-method}
		\If{$\mathit{marker\_info2}$}
		\label{alg:detectmarker-check-marker-start}
			\State \ldots \Comment{Weitere Anweisungen zur Identifikation einer Marke.}
		\EndIf
		\label{alg:detectmarker-check-marker-end}
	\EndIf
	\label{alg:detectmarker-check-il-end}
	\State \ldots \Comment{Weitere Anweisungen zur Identifikation einer Marke.}
\end{algorithmic}
\end{algorithm}

In Zeile \ref{alg:detectmarker-init-start}--\ref{alg:detectmarker-init-end} werden die lokalen Variablen initialisiert.
 Die Variablen werden als Parameter für den Aufruf der Methode \textproc{arDetectMarker2} in Zeile
 \ref{alg:detectmarker-call-method} verwendet. In Zeile \ref{alg:detectmarker-call-autothreshold} wird der Schwellwert
 auf seine Startwerte zurückgesetzt (Vgl. \autoref{alg:autothresholdreset}) und der Bildspeicher in Zeile
 \ref{alg:detectmarker-call-imagebuffer} überprüft (Vgl. \autoref{alg:checkimagebuffer}). Die Regionenmarkierung $I_l$
 wird durch den Rückgabewert von \textproc{arLabeling} in Zeile \ref{alg:detectmarker-call-labeling} gesetzt. Im
 Anschluss wird in Zeile \ref{alg:detectmarker-check-il-start}--\ref{alg:detectmarker-check-il-end} geprüft, ob der
 Speicher der Regionenmarkierung erfolgreich gesetzt wurde. Andernfalls wird die Untersuchung für das aktuelle
 Bildsignal $I$ beendet. Nur wenn die Regionenmarkierung erfolgreich war, wird in Zeile
 \ref{alg:detectmarker-call-method} die Methode \textproc{arDetectMarker2} aufgerufen. Der Rückgabewert von
 \textproc{arDetectMarker2} wird in der Membervariable $\mathit{marker\_info2}$ gespeichert. Es wird anschließend in
 Zeile \ref{alg:detectmarker-check-marker-start}--\ref{alg:detectmarker-check-marker-end} geprüft, ob der Zeiger von
 $\mathit{marker\_info2}$ auf einen gültigen Speicherbereich verweist. Falls die Überprüfung erfolgreich war, sind in
 $\mathit{marker\_info2}$ die Koordinaten der Eckpunkte der Marke gespeichert und das Verfahren beendet.
 \autoref{alg:detectmarker} verarbeitet die Anweisungen in konstanter Zeit.

\subsubsection{Regionenmarkierung} % (fold)
\label{sec:regionenmarkierung}

Die Methode \textproc{arLabeling} (\autoref{alg:arlabeling-init1}--\autoref{alg:arlabeling-calcregiondata}) wird zur
 Markierung der Regionen in einem Bildsignal $I$ verwendet. Wie in \autoref{sub:fiducial_detection} beschrieben,
 verzichet ARToolKitPlus beim Aufruf des Verfahrens auf ein Binärbild. Stattdessen wird ein Bildsignal $I$ während der
 Verarbeitung durch eine Schwellwertanalyse untersucht. \textproc{arLabeling} kann in drei Abschnitte unterteilt werden
 (Vgl. \autoref{alg:arlabeling-overview}):

\begin{enumerate}
	\item Initialisierung der Variablen und des Speichers, \label{label-init}
	\item Regionenmarkierung und Auflösen von Kollisionen und \label{label-region}
	\item Aufbereiten der Regionenmarkierung zur Speicherung. \label{label-cleaning}
\end{enumerate}

\begin{algorithm}[!ht]
\caption{\textproc{arLabeling} (Übersicht)}
\label{alg:arlabeling-overview}
\begin{algorithmic}[1]
	\Require $I,\mathit{thresh},\mathit{label\_num},\mathit{area},\mathit{pos},\mathit{clip},\mathit{label\_ref}$

	\State Initialisieren der Variablen und des Speichers
	\State Regionenmarkierung
	\State Regionenmarkierung aufbereiten und speichern

\end{algorithmic}
\end{algorithm}


\autoref{label-init} ist in \autoref{alg:arlabeling-init1} aufgeführt. In Zeile
 \ref{alg:arlabeling-init1-local-start}--\ref{alg:arlabeling-init1-local-end} werden die lokalen Variablen deklariert,
 deren Bedeutung bei ihrem ersten Auftreten erklärt werden. Speicheradressen werden in Zeile
 \ref{alg:arlabeling-init1-address-start}--\ref{alg:arlabeling-init1-address-end} initialisiert. Der
 Schwellwertparameter wird verdreifacht und in lokal gespeichert (Zeile \ref{alg:arlabeling-init1-threshold}). Die
 Schwellwertanalyse in \autoref{alg:arlabeling-regionlabeling} benutzt einen dreifachen Wert als Optimierung
 (Vgl. S.~\pageref{sub:arlabel-threshold}). Zum Schluss wird in Zeile
 \ref{alg:arlabeling-init1-size-start}--\ref{alg:arlabeling-init1-size-end} die Größe des Bildsignals festgelegt. Die
 Variablen $\mathit{arImXsize}$ und $\mathit{arImYsize}$ enthalten die Breite und Höhe des Bildsignals $I$. Das
 halbieren der Bildhöhe und -breite ist ebenfalls eine Optimierung. Da auf alle Variablen und Adressen direkt
 zugegriffen wird, ist die Laufzeit des Algorithmus konstant.

\begin{algorithm}[!hb]\small
\caption{\textproc{arLabeling} (Initialisierung)}
\label{alg:arlabeling-init1}
\begin{algorithmic}[1]
	\Require $I,\mathit{thresh},\mathit{label\_num},\mathit{area},\mathit{pos},\mathit{clip},\mathit{label\_ref}$

	\State $\mathit{pnt}, \mathit{pnt1}, \mathit{pnt2} \gets \infty$
	\label{alg:arlabeling-init1-local-start}
	\State $\mathit{wk}, \mathit{wk\_max}, m, n, i, j, k, \mathit{lxsize}, \mathit{lysize}, \mathit{poff} \gets \infty$
	\State $\mathit{l\_image}, \mathit{work}, \mathit{work2}, \mathit{wlabel\_num}, \mathit{warea}, \mathit{wclip},
	 \mathit{wpos} \gets \infty$
	\label{alg:arlabeling-init1-local-end}

	\State $\mathit{l\_image} \gets \mathit{l\_imageL}[0]$
	\label{alg:arlabeling-init1-address-start}
	\State $\mathit{work} \gets \mathit{workL}[0]$
	\State $\mathit{work2} \gets \mathit{work2L}[0]$
	\State $\mathit{wlabel\_num} \gets \mathit{wlabel\_numL}$
	\State $\mathit{warea} \gets \mathit{wareaL}[0]$
	\State $\mathit{wclip} \gets \mathit{wclipL}[0]$
	\State $\mathit{wpos} \gets \mathit{wposL}[0]$
	\label{alg:arlabeling-init1-address-end}

	\State $\mathit{thresh} \gets \mathit{thresh} \cdot 3$
	\label{alg:arlabeling-init1-threshold}
	\State $\mathit{lxsize} \gets \tfrac{arImXsize}{2}$
	\label{alg:arlabeling-init1-size-start}
	\State $\mathit{lysize} \gets \tfrac{arImYsize}{2}$
	\label{alg:arlabeling-init1-size-end}

	\algstore{brk-arkabelinginit}
\end{algorithmic}
\end{algorithm}


Die Initialisierung von \textproc{arLabeling} wird in \autoref{alg:arlabeling-init2} fortgesetzt. $\mathit{pnt1}$ wird
 in Zeile \ref{alg:arlabeling-init2-address1-start} auf die erste Speicherstelle der ersten Zeile des Regionenbildes
 gesetzt. $\mathit{pnt2}$ erhält in Zeile \ref{alg:arlabeling-init2-address1-end} die erste Speicherstelle der
 letzten Zeile des Regionenbildes. In der Schleife in Zeile
 \ref{alg:arlabeling-init2-loop1-start}--\ref{alg:arlabeling-init2-loop1-end} wird über die Breite des Regionenbildes
 iteriert, um die erste und die letzte Zeile des Regionenbildes zu löschen. Dazu wird an den Adressen von
 $\mathit{pnt1}$ und $\mathit{pnt2}$ der Wert $0$ gespeichert. Danach werden die Adressen von $\mathit{pnt1}$ und
 $\mathit{pnt2}$ inkrementiert. In Zeile \ref{alg:arlabeling-init2-inc-1} wird die Laufvariable $i$ inkrementiert.

\begin{algorithm}[!ht]
\caption{\textproc{arLabeling} (Fortsetzung der Initialisierung)}
\label{alg:arlabeling-init2}
\begin{algorithmic}[1]
	\algrestore{brk-arkabelinginit}

	\State $\mathit{pnt1} \gets \mathit{l\_image}[0]$
	\Cost{$c_{14}$}{$2$}
	\label{alg:arlabeling-init2-address1-start}
	\State $\mathit{pnt2} \gets \mathit{l\_image}[(\mathit{lysize} - 1) \cdot \mathit{lxsize}]$
	\Cost{$c_{15}$}{$4$}
	\label{alg:arlabeling-init2-address1-end}
	\For{$i \gets 1$ \textbf{to} $i < \mathit{lxsize}$}
	\Cost{$c_{16}$}{$(\mathit{lxsize} - 1) + 1$}
	\label{alg:arlabeling-init2-loop1-start}
		\State $\mathit{pnt1} \gets 0$
		\Cost{$c_{17}$}{$(\mathit{lxsize} - 1 )$}
		\State $\mathit{pnt2} \gets 0$
		\Cost{$c_{18}$}{$(\mathit{lxsize} - 1 )$}
		\State Inkrementiere ${pnt1}$
		\Cost{$c_{19}$}{$(\mathit{lxsize} - 1 )$}
		\State Inkrementiere ${pnt2}$
		\Cost{$c_{20}$}{$(\mathit{lxsize} - 1 )$}
		\State $i \gets i + 1$
		\Cost{$c_{21}$}{$(\mathit{lxsize} - 1 )$}
		\label{alg:arlabeling-init2-inc-1}
	\EndFor
	\label{alg:arlabeling-init2-loop1-end}

	\State $\mathit{pnt1} \gets \mathit{l\_image}[0]$
	\Cost{$c_{23}$}{$2$}
	\label{alg:arlabeling-init2-address2-start}
	\State $\mathit{pnt2} \gets \mathit{l\_image}[\mathit{lxsize} - 1]$
	\Cost{$c_{24}$}{$3$}
	\label{alg:arlabeling-init2-address2-end}
	\For{$i \gets 1$ \textbf{to} $i < \mathit{lysize}$}
	\Cost{$c_{25}$}{$(\mathit{lysize} - 1) + 1$}
	\label{alg:arlabeling-init2-loop2-start}
		\State $\mathit{pnt1} \gets 0$
		\Cost{$c_{26}$}{$\mathit{lysize} - 1$}
		\label{alg:arlabeling-init2-clearfirstrow}
		\State $\mathit{pnt2} \gets 0$
		\Cost{$c_{27}$}{$\mathit{lysize} - 1$}
		\label{alg:arlabeling-init2-clearlastrow}
		\State $\mathit{pnt1} \gets \mathit{pnt1} + \mathit{lxsize}$
		\Cost{$c_{28}$}{$(\mathit{lysize} - 1) 2$}
		\State $\mathit{pnt2} \gets \mathit{pnt2} + \mathit{lxsize}$
		\Cost{$c_{29}$}{$(\mathit{lysize} - 1) 2$}
		\State $i \gets i + 1$
		\Cost{$c_{30}$}{$\mathit{lysize} - 1$}
	\EndFor
	\label{alg:arlabeling-init2-loop2-end}

	\State $\mathit{wk\_max} \gets 0$
	\Cost{$c_{32}$}{$1$}
	\label{alg:arlabeling-init2-label}
	\State $\mathit{pnt2} \gets \mathit{l\_image}[\mathit{lxsize} + 1]$
	\Cost{$c_{33}$}{$3$}
	\State $\mathit{pnt} \gets I[\left(\mathit{arImXsize} \cdot 2 + 2 \right) \cdot \mathit{pixelSize}]$
	\Cost{$c_{34}$}{$5$}
	\State $\mathit{poff} \gets \mathit{pixelSize} \cdot 2$
	\Cost{$c_{35}$}{$2$}
\end{algorithmic}
\end{algorithm}


In Zeile \ref{alg:arlabeling-init2-address2-start}--\ref{alg:arlabeling-init2-address2-end} werden die Adressen
 von $\mathit{pnt1}$ und $\mathit{pnt2}$ erneut festgelegt. $\mathit{pnt1}$ wird die erste Speicherstelle des
 Regionenbildes zugewiesen. In $\mathit{pnt2}$ wird die Adresse der ersten Speicherstelle der letzten Zeile hinterlegt.
 In der Schleife von Zeile \ref{alg:arlabeling-init2-loop2-start}--\ref{alg:arlabeling-init2-loop2-end} wird die erste
 und letzte Spalte des Regionenbildes gelöscht, indem der Wert $0$ an die Speicherstelle von $\mathit{pnt1}$ und
 $\mathit{pnt2}$ geschrieben wird
 (Zeile \ref{alg:arlabeling-init2-clearfirstrow}--\ref{alg:arlabeling-init2-clearlastrow}). Im Anschluss daran werden
 die Adressen $\mathit{pnt1}$ und $\mathit{pnt2}$, sowie die Laufvariable $i$, inkrementiert.

In Zeile \ref{alg:arlabeling-init2-label} wird die Markierungsvariable $\mathit{wk\_max}$ mit dem Wert $0$
 initialisiert. Danach wird die Adresse der Startposition des Regionenbildes in $\mathit{pnt2}$ gespeichert. Dabei ist
 zu beachten, dass die Adresse auf den zweiten \gls{pixel} der zweiten Zeile verweist. $\mathit{pnt}$ wird daraufhin die
 Adresse des Bildsignals $I$ zugewiesen. Der Adresse von $I$ ist der vierte \gls{pixel} der zweiten Zeile. Die
 Variable $\mathit{poff}$ ist der Adressabstand der \gls{pixel} in $I$ und dient der Adressierung des nächsten
 \glspl{pixel}. $\mathit{poff}$ wird in der letzten Zeile von \autoref{alg:arlabeling-init2} gesetzt.

Die Kosten des Algorithmus sind in \autoref{alg:arlabeling-init2} angegeben und in \autoref{eq:analyse-arlabeling}
 aufgeführt. Da die Breite $\mathit{lxsize}$ des Regionenbildes größer ist als die Höhe $\mathit{lysize}$, ist die
 Laufzeit des Algorithmus abhängig von $\mathit{lxsize}$ und beträgt somit $\Theta{(\mathit{lxsize})}$.

\begin{equation}
	\label{eq:analyse-arlabeling}
	\begin{split}
		T(I) = &
		c_1
		+ c_2
		+ c_3 \left(\mathit{lxsize} + 1\right)\\
		& + c_4 \sum_{i=1}^{\mathit{lxsize}} 1
		+ c_5 \sum_{i=1}^{\mathit{lxsize}} 1
		+ c_6 \sum_{i=1}^{\mathit{lxsize}} 1
		+ c_7 \sum_{i=1}^{\mathit{lxsize}} 1
		+ c_8 \sum_{i=1}^{\mathit{lxsize}} 1\\
		& + c_9
		+ c_{10}
		+ c_{11} \left(\mathit{lysize} + 1\right)\\
		& + c_{12} \sum_{i=1}^{\mathit{lysize}} 1
		+ c_{13} \sum_{i=1}^{\mathit{lysize}} 1
		+ c_{14} \sum_{i=1}^{\mathit{lysize}} 1
		+ c_{15} \sum_{i=1}^{\mathit{lysize}} 1
		+ c_{16} \sum_{i=1}^{\mathit{lysize}} 1\\
		& + c_{17}
		+ c_{18}
		+ c_{19}
		+ c_{20}\\
		T(I) = &
		c_1
		+ c_2
		+ c_3
		+ \left(c_3 \mathit{lxsize}\right)\\
		& + c_4 \left[ \mathit{lxsize} \left(1\right) \right]
		+ c_5 \left[ \mathit{lxsize} \left(1\right) \right]
		+ c_6 \left[ \mathit{lxsize} \left(1\right) \right]
		+ c_7 \left[ \mathit{lxsize} \left(1\right) \right]
		+ c_8 \left[ \mathit{lxsize} \left(1\right) \right]\\
		& + c_9
		+ c_{10}
		+ c_{11}
		+ c_{11} \left[ \mathit{lysize} \left(1\right) \right]\\
		& + c_{12} \left[ \mathit{lysize} \left(1\right) \right]
		+ c_{13} \left[ \mathit{lysize} \left(1\right) \right]
		+ c_{14} \left[ \mathit{lysize} \left(1\right) \right]
		+ c_{15} \left[ \mathit{lysize} \left(1\right) \right]
		+ c_{16} \left[ \mathit{lysize} \left(1\right) \right]\\
		& + c_{17}
		+ c_{18}
		+ c_{19}
		+ c_{20}\\
		T(I) = &
		c_1
		+ c_2
		+ c_3
		+ \left( c_3 + c_4 + c_5 + c_6 + c_7 + c_8 \right) \mathit{lxsize}\\
		& + c_9
		+ c_{10}
		+ c_{11}
		+ \left( c_{11} + c_{12} + c_{13} + c_{14} + c_{15} + c_{16} \right) \mathit{lysize}\\
		& + c_{17}
		+ c_{18}
		+ c_{19}
		+ c_{20}\\
		T(I) = &
		\mathit{lxsize} + \mathit{lysize}\\
		T(I) = &
		\Theta\left( \mathit{lxsize} \right)
	\end{split}
\end{equation}

\autoref{label-region} von \autoref{alg:arlabeling-overview} ist für die Regionenmarkierung und das Auflösen von
 Kollisionen verantwortlich. Eine Übersich des Verfahrens ist in \autoref{alg:arlabeling-regionlabeling} dargestellt.
 \textproc{arLabeling} untersucht das Bildsignal $I$ zeilenweise von links oben nach rechts unten. Durch eine
 Schwellwertanalyse wird entschieden, ob ein \gls{pixel} an Position $I(u,v)$ ein Vordergrund- oder Hintergrundpixel
 ist. Die Regionenmarkierung wird dann in $\mathit{l\_image}$ gespeichert. Das Verfahren wird solange wiederholt, bis
 das Bildsignal $I$ vollständig prozessiert wurde und zusammenhängende Bildregionen in $\mathit{l\_image}$ markiert
 sind. Die Variable $\mathit{pnt}$ enthält die Adresse des zu untersuchenden \gls{pixel} aus $I$. Die nächste freie
 Speicherstelle in $\mathit{l\_image}$ ist in $\mathit{pnt2}$ hinterlegt.

\begin{algorithm}[!ht]\small
\caption{\textproc{arLabeling} (Regionenmarkierung)}
\label{alg:arlabeling-regionlabeling}
\begin{algorithmic}[1]
	\For{$j \gets 1$ \textbf{to} $j < \mathit{lysize} - 1$}
	\Cost{$c_{1}$}{$\mathit{lysize} - 1$}
	\label{alg:arlabeling-regionlabeling-loop1-start}

		\For{$i \gets 1$ \textbf{to} $i < \mathit{lxsize} - 1$}
		\Cost{$c_{2}$}{$(\mathit{lysize} - 2)(\mathit{lxsize} - 1)$}
		\label{alg:arlabeling-regionlabeling-loop2-start}

			\State $\mathit{coorTresh} \gets \mathit{tresh}$
			\Cost{$c_{3}$}{$(\mathit{lysize} - 2)(\mathit{lxsize} - 2)$}
			\label{alg:arlabeling-regionlabeling-threshold-start}
			\State $\mathit{isBlack} \gets \textbf{false}$
			\Cost{$c_{4}$}{$(\mathit{lysize} - 2)(\mathit{lxsize} - 2)$}
			\State $\mathit{isBlack} \gets \left(
			\begin{aligned}
				& \quad (\mathit{pnt} + 0) \\
				& \quad + (\mathit{pnt} + 1) \\
				& \quad + (\mathit{pnt} + 2) \\
				& \leq \mathit{coorTresh}
			\end{aligned}\right)$
			\Cost{$c_{5}$}{$(\mathit{lysize} - 2)(\mathit{lxsize} - 2)7$}
			\label{alg:arlabeling-regionlabeling-calcblack}

			\If{$\mathit{isBlack}$}
			\Cost{$c_{6}$}{$(\mathit{lysize} - 2)(\mathit{lxsize} - 2)$}
			\label{alg:arlabeling-regionlabeling-isblack?}
				\State Untersuche 8er-Nachbarschaft
				\Cost{$c_{7}$}{$(\mathit{lysize} - 2)(\mathit{lxsize} - 2)t_{1}$}
				\label{alg:arlabeling-regionlabeling-black}
			\Else
				\State $\mathit{pnt2} \gets 0$
				\Cost{$c_{9}$}{$(\mathit{lysize} - 2)(\mathit{lxsize} - 2)$}
				\label{alg:arlabeling-regionlabeling-notblack}
			\EndIf
			\label{alg:arlabeling-regionlabeling-threshold-end}

			\State $i \gets i + 1$
			\Cost{$c_{11}$}{$(\mathit{lysize} - 2)(\mathit{lxsize} - 2)$}
			\label{alg:arlabeling-regionlabeling-inc1-start}
			\State $\mathit{pnt} \gets \mathit{pnt} + \mathit{poff}$
			\Cost{$c_{12}$}{$(\mathit{lysize} - 2)(\mathit{lxsize} - 2)2$}
			\State Inkrementiere $\mathit{pnt2}$
			\Cost{$c_{13}$}{$(\mathit{lysize} - 2)(\mathit{lxsize} - 2)$}
			\label{alg:arlabeling-regionlabeling-inc1-end}
		\EndFor
		\label{alg:arlabeling-regionlabeling-loop2-end}

		\State $\mathit{pnt} \gets \mathit{pnt} + \mathit{arImXsize} \cdot \mathit{pixelSize}$
		\Cost{$c_{15}$}{$(\mathit{lysize} - 2)3$}
		\label{alg:arlabeling-regionlabeling-inc2-start}
		\State $j \gets j + 1$
		\Cost{$c_{16}$}{$(\mathit{lysize} - 2)$}
		\State $\mathit{pnt} \gets \mathit{pnt} + \mathit{poff} \cdot 2$
		\Cost{$c_{17}$}{$(\mathit{lysize} - 2)3$}
		\State $\mathit{pnt2} \gets \mathit{pnt2} + 2$
		\Cost{$c_{18}$}{$(\mathit{lysize} - 2)2$}
		\label{alg:arlabeling-regionlabeling-inc2-end}
	\EndFor
	\label{alg:arlabeling-regionlabeling-loop1-end}
\end{algorithmic}
\end{algorithm}

In den beiden Schleifen in Zeile
 \ref{alg:arlabeling-regionlabeling-loop1-start}--\ref{alg:arlabeling-regionlabeling-loop1-end} und Zeile
 \ref{alg:arlabeling-regionlabeling-loop2-start}--\ref{alg:arlabeling-regionlabeling-loop2-end} wird das Bildsignal
 zeilenweise, von oben links nach unten rechts, verarbeitet. Das Inkrementieren der Variablen in Zeile
 \ref{alg:arlabeling-regionlabeling-inc1-start}--\ref{alg:arlabeling-regionlabeling-inc1-end} und Zeile
 \ref{alg:arlabeling-regionlabeling-inc2-start}--\ref{alg:arlabeling-regionlabeling-inc2-end} sorgt dafür, dass nur die
 Hälfte der \gls{pixel} prozessiert werden.

Die Schwellwertanalyse wird in Zeile
 \ref{alg:arlabeling-regionlabeling-threshold-start}--\ref{alg:arlabeling-regionlabeling-threshold-end} durchgeführt.
 Dazu wird in Zeile \ref{alg:arlabeling-regionlabeling-calcblack} die RGB-Komponente des Bildsignals $I$ ausgelesen und
 addiert. Normalerweise würde man an dieser Stelle den Schwellwert mit jeder Komponente $R$, $G$ und $B$ einzeln
 vergleichen. Die Verdreifachung des Schwellwerts in \autoref{alg:arlabeling-init1} und die Addition der
 RGB-Komponenten ermöglichen hingegen eine Schwellwertanalyse mit nur einem Vergleich.\label{sub:arlabel-threshold}

In Zeile \ref{alg:arlabeling-regionlabeling-isblack?}--\ref{alg:arlabeling-regionlabeling-threshold-end} wird
 untersucht, ob ein Vordergrundpixel gefunden wurde. Falls nicht, wird in das Regionenbild $\mathit{l\_image}$ eine $0$
 geschrieben. Wenn die Schwellertanalyse ein Vordergrundpixel bestimmt hat, müssen in Zeile
 \ref{alg:arlabeling-regionlabeling-black} die Nachbarn des Vordergrundpixels mit einer 8er-Nachbarschaft untersucht
 werden (Vgl. \autoref{alg:arlabeling-neighbour}).

\begin{algorithm}[!ht]
\caption{\textproc{arLabeling} (Untersuchung der 8er-Nachbarschaft)}
\label{alg:arlabeling-neighbour}
\begin{algorithmic}[1]
	\State $\mathit{pnt1} \gets \mathit{pnt2}[-\mathit{lxsize}]$ \Comment Adresse zuweisen
	\Cost{$c_{1}$}{$3$}
	\label{alg:arlabeling-neighbour-n3}
	\If{$\mathit{pnt1} > 0$}
	\Cost{$c_{2}$}{$2$}
		\State \ldots \Comment 1. Fall
	\ElsIf{$\left(\mathit{pnt1} + 1\right) > 0$}
	\Cost{$c_{4}$}{$2$}
		\State \ldots \Comment 2. Fall
	\ElsIf{$\left(\mathit{pnt1} - 1\right) > 0$}
	\Cost{$c_{6}$}{$2$}
		\State \ldots \Comment 3. Fall
	\ElsIf{$\left(\mathit{pnt2} - 1\right) > 0$}
	\Cost{$c_{8}$}{$2$}
		\State \ldots \Comment 4. Fall
	\Else
		\State \ldots \Comment 5. Fall
	\EndIf
\end{algorithmic}
\end{algorithm}


Die Nachbarschafsuntersuchung in ARToolKitPlus untersucht die vier Nachbarn der Markierung an Position $I(u,v)$. Wie in
 \autoref{sec:vorläufige_makierung} bereits beschrieben, ist eine Regionenmarkierung davon abhängig ob alle Nachbarn
 Hintergrundpixel sind, genau ein Nachbar eine Markierung hat oder mehrere Nachbarn eine Markierung haben. Die Fälle
 1, 3 und 4 in \autoref{alg:arlabeling-neighbour} untersuchen, ob genau ein Nachbar eine Markierung hat. Bei Fall 5 sind
 alle Nachbarn Hintergrundpixel. Nur bei Fall 2 können  mehrere Nachbarn eine Markierung haben. In Zeile
 \ref{alg:arlabeling-neighbour-n3} wird der Variablen $\mathit{pnt1}$ der Nachbar $N_3 = (u,v+1)$ zugewiesen. Die
 Nachbarn sind in \autoref{fig:analyse-nachbarschaftsbeziehung} illustriert.

\begin{figure}[!ht]
	\centering
	\input{resources/8er-Nachbarschaft.pdf_tex}
	\caption{8er-Nachbarschaft mit $N_1 = I(u-1,v)$, $N_2 = I(u-1,v+1)$, $N_3 = (u,v+1)$ und $N_4 = I(u+1,v+1)$.}
	\label{fig:analyse-nachbarschaftsbeziehung}
\end{figure}

\paragraph{1. Fall:} % (fold)
\label{par:fall_1_}
Wir wissen durch \autoref{alg:arlabeling-regionlabeling}, dass $I(u,v)$ ein Vordergrundpixel ist, dem wir an dieser
 Stelle eine Markierung zuweisen wollen. Im ersten Fall wird die Markierung $N_3$ untersucht. Wenn $N_3$ ein
 Vordergrundpixel ist kann die Markierung für $I(u,v)$ übernommen werden. Falls es sich bei den Nachbarn
 $N_1 = I(u-1,v)$, $N_2 = I(u-1,v+1)$ und $N_4 = I(u+1,v+1)$ um Vordergrundpixel handelt, haben sie die gleiche
 Markierung wie $N_3$ und müssen nicht weiter untersucht werden. Das Verfahren ist in
 \autoref{alg:arlabeling-neighbour-case1} beschrieben.

% Fall 1: *pnt1 > 0
\begin{algorithm}[ht]
\caption{\textproc{arLabeling} (8er-Nachbarschaft: 1. Fall)}
\label{alg:arlabeling-neighbour-case1}
\begin{algorithmic}[1]

	\If{$\mathit{pnt1} > 0$}
		\State $\mathit{pnt2} \gets \mathit{pnt1}$
		\label{alg:arlabeling-neighbour-case1-save-label}
		\State $\mathit{pnt2\_index} \gets \left(\mathit{pnt2} - 1\right) \cdot 7$
		\label{alg:arlabeling-neighbour-case1-calc-offset}
		\State $\mathit{work2}\left[\mathit{pnt2\_index} + 0\right] \gets \mathit{work2}\left[\mathit{pnt2\_index} + 0\right] + 1$
		\label{alg:arlabeling-neighbour-case1-inc-region}
		\State $\mathit{work2}\left[\mathit{pnt2\_index} + 1\right] \gets \mathit{work2}\left[\mathit{pnt2\_index} + 1\right] + i$
		\State $\mathit{work2}\left[\mathit{pnt2\_index} + 2\right] \gets \mathit{work2}\left[\mathit{pnt2\_index} + 2\right] + j$
		\State $\mathit{work2}\left[\mathit{pnt2\_index} + 6\right] \gets j$
		\label{alg:arlabeling-neighbour-case1-save-j}
		\algstore{brk-case1}
\end{algorithmic}
\end{algorithm}


In Zeile \ref{alg:arlabeling-neighbour-case1-save-label} wird die Markierung von $N_3$ übernommen und in
 $\mathit{l\_image}$ an Position $(x,y)$ gespeichert. In der Variablen $\mathit{work2}$ werden Informationen der
 Regionenmarkierung gespeichert. Dazu wird zuerst in Zeile \ref{alg:arlabeling-neighbour-case1-calc-offset} aus
 $\mathit{pnt2}$ der Wert der Regionenmarkierung gelesen. Der Wert der Markierung wird zur Berechnung des
 Adressabstands benutzt, um die Werte für die Region an die richtige Stelle zu schreiben. An der Position $0$ von
 $\mathit{work2}$ (Zeile \ref{alg:arlabeling-neighbour-case1-inc-region}) wird die Anzahl der Vordergrundpixel der
 Region erhöht. Position 1 und Position 2 von $\mathit{work2}$ enthalten die akummulierten Werte von $i$ und $j$ für
 die u- und v-Koordinaten aller Vordergrundpixel der Region. An Position 6 von $\mathit{work2}$ in Zeile
 \ref{alg:arlabeling-neighbour-case1-save-j} wird die y-Koordinate gespeichert. Alle Anweisungen in
 \autoref{alg:arlabeling-neighbour-case1} werden in konstanter Zeit ausgeführt.

% paragraph fall_1_ (end)

\paragraph{2. Fall:} % (fold)
\label{par:fall_2_}
Beim zweiten Fall wird die Markierung $N_4$ betrachtet. Da $N_3$ keine Markierung aufweist, können nur $N_1$ und $N_2$
 Markierungen haben. Da die Markierungen nur durch $I(u,v)$ verbunden sind, kann es sich hier um eine Kollision
 handeln, die durch \autoref{alg:arlabeling-neighbour-case2-1} oder \autoref{alg:arlabeling-neighbour-case2-2}
 besonders behandelt wird. Das Verfahren ist in \autoref{alg:arlabeling-neighbour-case2} aufgeführt.

% Fall 2: *(pnt1+1) > 0
\begin{algorithm}[!ht]\small
\caption{\textproc{arLabeling} (8er-Nachbarschaft: 2. Fall)}
\label{alg:arlabeling-neighbour-case2}
\begin{algorithmic}[1]
	\algrestore{brk-case1}

	\ElsIf{$\left(\mathit{pnt1} + 1\right) > 0$}
	\Cost{$c_{8}$}{$2$}
		\If{$\left(\mathit{pnt1} - 1\right) > 0$}
		\Cost{$c_{9}$}{$2$}
		\Comment Ist in $N_2$ eine Markierung?
		\label{alg:arlabeling-neighbour-case2-n2}
			\State \ldots

		\ElsIf{$\left(\mathit{pnt2} - 1\right) > 0$}
		\Cost{$c_{11}$}{$2$}
		\Comment Ist in $N_1$ eine Markierung?
		\label{alg:arlabeling-neighbour-case2-n1}
			\State \ldots

		\Else \Comment Nur $N_4$ hat eine Markierung.
		\label{alg:arlabeling-neighbour-case2-4}
			\State \ldots
		\EndIf
		\algstore{brk-case2}
\end{algorithmic}
\end{algorithm}


In Zeile \ref{alg:arlabeling-neighbour-case2-n2} wird überprüft, ob $N_2$ eine Markierung enthält. Falls ja, wird
 \autoref{alg:arlabeling-neighbour-case2-1} untersuchen, ob eine Kollision vorliegt und sie gegebenfalls auflösen.
 Zeile \ref{alg:arlabeling-neighbour-case2-n1} überprüft $N_1$ auf eine vorhandene Markierung und fährt mit der
 Untersuchung einer evtl. Kollision in \autoref{alg:arlabeling-neighbour-case2-2} fort. Wenn weder $N_2$ noch $N_1$
 eine Markierung haben, ist nur $N_4$ ein Vordergrundpixel und wird mit \autoref{alg:arlabeling-neighbour-case2-3}
 markiert.

% Fall 2: *(pnt1+1) > 0
\begin{algorithm}[!ht]\small
\caption{\textproc{arLabeling} (8er-Nachbarschaft: 2. Fall, $N_4$ und $N_2$)}
\label{alg:arlabeling-neighbour-case2-1}
\begin{algorithmic}[1]
	% \State \Comment $N_4$ und $N_2$ haben eine Markierung.
	\State $m \gets \mathit{work}\left[\left(\mathit{pnt1} + 1\right) - 1\right]$
	\Cost{$c_{1}$}{$4$}
	\label{alg:arlabeling-neighbour-case2-1-m}
	\State $n \gets \mathit{work}\left[\left(\mathit{pnt1} - 1\right) - 1\right]$
	\Cost{$c_{2}$}{$4$}
	\label{alg:arlabeling-neighbour-case2-1-n}

	\If{$m > n$}
	\Cost{$c_{3}$}{$1$}
	\label{alg:arlabeling-neighbour-case2-mn-start}
		\State $\mathit{pnt2} \gets n$
		\Cost{$c_{4}$}{$1$}
		\label{alg:arlabeling-neighbour-case2-saven}
		\State $\mathit{wk} \gets \left(\mathit{work}\left[0\right]\right)$
		\Cost{$c_{5}$}{$2$}
		\label{alg:arlabeling-neighbour-case2-worklist}
		\For{$k \gets 0$ \textbf{to} $k < \mathit{wk\_max}$}
		\Cost{$c_{6}$}{$\mathit{wk\_max} + 1$}
		\label{alg:arlabeling-neighbour-case2-loop-start}
			\If{$\mathit{wk} = m$}
			\Cost{$c_{7}$}{$\mathit{wk\_max}$}
				\State $\mathit{wk} \gets n$
				\Cost{$c_{8}$}{$\mathit{wk\_max}$}
			\EndIf
			\State Inkrementiere $\mathit{wk}$
			\Cost{$c_{10}$}{$\mathit{wk\_max}$}
			\State $k \gets k + 1$
			\Cost{$c_{11}$}{$\mathit{wk\_max}$}
		\EndFor
		\label{alg:arlabeling-neighbour-case2-loop-end}
	\label{alg:arlabeling-neighbour-case2-mn-end}
	\ElsIf{$ m < n$}
	\Cost{$c_{13}$}{$1$}
	\label{alg:arlabeling-neighbour-case2-nm-start}
		\State $\mathit{pnt2} \gets m$
		\Cost{$c_{14}$}{$1$}
		\State $\mathit{wk} \gets \left(\mathit{work}\left[0\right]\right)$
		\Cost{$c_{15}$}{$2$}
		\For{$k \gets 0$ \textbf{to} $k < \mathit{wk\_max}$}
		\Cost{$c_{16}$}{$\mathit{wk\_max} + 1$}
			\If{$\mathit{wk} = n$}
			\Cost{$c_{17}$}{$\mathit{wk\_max}$}
				\State $\mathit{wk} \gets m$
				\Cost{$c_{18}$}{$\mathit{wk\_max}$}
			\EndIf
			\State Inkrementiere $\mathit{wk}$
			\Cost{$c_{20}$}{$\mathit{wk\_max}$}
			\State $k \gets k + 1$
			\Cost{$c_{21}$}{$\mathit{wk\_max}$}
		\EndFor
	\label{alg:arlabeling-neighbour-case2-nm-stop}
	\Else
		\State $\mathit{pnt2} \gets m$
		\Cost{$c_{24}$}{$1$}
		\label{alg:arlabeling-neighbour-case2-savem}
	\EndIf
\end{algorithmic}
\end{algorithm}


In \autoref{alg:arlabeling-neighbour-case2-1} wird in Zeile \ref{alg:arlabeling-neighbour-case2-1-m} der Wert der
 Markierung $N_4$ in Variable $m$ gespeichert. In Zeile \ref{alg:arlabeling-neighbour-case2-1-n} wird der Wert $N_2$
 in $n$ hinterlegt. In Zeile \ref{alg:arlabeling-neighbour-case2-mn-start}--\ref{alg:arlabeling-neighbour-case2-mn-end}
 wird überprüft, ob $m$ größer als $n$ ist. Falls dem so ist, wird der Wert $n$ in $\mathit{l\_image}$ gespeichert
 (Zeile \ref{alg:arlabeling-neighbour-case2-saven}). Danach wird in Zeile \ref{alg:arlabeling-neighbour-case2-worklist}
 die Adresse der ersten Stelle der Regionenmarkierungsliste $\mathit{work}$ in $wk$ hinterlegt. In der Schleife von
 Zeile \ref{alg:arlabeling-neighbour-case2-loop-start} bis Zeile \ref{alg:arlabeling-neighbour-case2-loop-end} wird die
 Liste durchlaufen und alle Werte von $m$ durch den Wert $n$ ersetzt. Falls der Wert $m$ kleiner als $n$ ist
 (Zeile \ref{alg:arlabeling-neighbour-case2-nm-start}--\ref{alg:arlabeling-neighbour-case2-nm-stop}) wird das gleiche
 Verfahren angewendet. Lediglich $m$ und $n$ werden getauscht. Wenn es sich bei $m$ und $n$ um den gleichen Wert
 handet, und somit $m$ und $n$ zur gleichen Region gehören, wird der Wert $m$ in $\mathit{l\_image}$ gespeichert
 (Zeile \ref{alg:arlabeling-neighbour-case2-savem}).

% Fall 2: *(pnt1+1) > 0
\begin{algorithm}[ht]
\caption{\textproc{arLabeling} (8er-Nachbarschaft: 2. Fall, $N_4$ und $N_1$)}
\label{alg:arlabeling-neighbour-case2-2}
\begin{algorithmic}[1]
	\State \Comment $N_4$ und $N_1$ haben eine Markierung.
	\State $m \gets \mathit{work}\left[\left(\mathit{pnt1} + 1\right) - 1\right]$ \Cost{$c_{1}$}{$1$}
	\State $n \gets \mathit{work}\left[\left(\mathit{pnt2} - 1\right) - 1\right]$ \Cost{$c_{2}$}{$1$}

	\If{$m > n$} \Cost{$c_{3}$}{$1$}
		\State $\mathit{pnt2} \gets n$ \Cost{$c_{4}$}{$1$}
		\State $\mathit{wk} \gets \left(\mathit{work}\left[0\right]\right)$ \Cost{$c_{5}$}{$1$}
		\For{$k \gets 0$ \textbf{to} $k < \mathit{wk\_max}$} \Cost{$c_{6}$}{$\mathit{wk\_max} + 1$}
			\If{$\mathit{wk} == m$} \Cost{$c_{7}$}{$\mathit{wk\_max}$}
				\State $\mathit{wk} \gets n$ \Cost{$c_{8}$}{$\mathit{wk\_max}$}
			\EndIf
			\State Inkrementiere $\mathit{wk}$ \Cost{$c_{9}$}{$\mathit{wk\_max}$}
			\State $k \gets k + 1$ \Cost{$c_{9}$}{$\mathit{wk\_max}$}
		\EndFor

	\ElsIf{$ m < n$}
		\State $\mathit{pnt2} \gets m$
		\State $\mathit{wk} \gets \left(\mathit{work}\left[0\right]\right)$
		\For{$k \gets 0$ \textbf{to} $k < \mathit{wk\_max}$}
			\If{$\mathit{wk} == n$}
				\State $\mathit{wk} \gets m$
			\EndIf
			\State Inkrementiere $\mathit{wk}$
			\State $k \gets k + 1$
		\EndFor

	\Else
		\State $\mathit{pnt2} \gets m$
	\EndIf
\end{algorithmic}
\end{algorithm}


\autoref{alg:arlabeling-neighbour-case2-2} behandelt den Fall, dass $N_4$ und $N_1$ eine Markierung aufweisen. Das Verfahren entspricht dem Verfahren in \autoref{alg:arlabeling-neighbour-case2-1}.

In \autoref{alg:arlabeling-neighbour-case2-3} ist der Fall beschrieben, dass $N_4$ der einzige Nachbar des
 Vordergrundpixels $(u,v)$ ist. In diesem Fall wird der Wert der Markierung von $N_4$ in $\mathit{l\_image}$
 gespeichert (Zeile \ref{alg:arlabeling-neighbour-case2-3-n4}). Danach wird der Adressabstand berechnet, um die
 Informationen der Regionenmarkierung in $\mathit{work2}$ zu aktualisieren. Zuerst wird in Zeile
 \ref{alg:arlabeling-neighbour-case2-3-incregion} die Anzahl der Vordergrundpixel der Region erhöht. An Position 1 und
 Position 2 von $\mathit{work2}$ werden die Werte von $i$ und $j$ aufaddiert. Falls in Zeile
 \ref{alg:arlabeling-neighbour-case2-3-isismaller} die Position $x$ des ersten Vordergrundpixels der Region größer ist,
 als der aktuelle Wert in $i$, wird die Position in Zeile \ref{alg:arlabeling-neighbour-case2-3-newi} aktualisiert. Zum
 Schluss wird die Position $y$ des letzten Vordergrundpixels der Region durch den aktuellen Wert $j$ ersetzt.

% Fall 2: *(pnt1+1) > 0
\begin{algorithm}[!ht]
\caption{\textproc{arLabeling} (8er-Nachbarschaft: 2. Fall, $N_4$)}
\label{alg:arlabeling-neighbour-case2-3}
\begin{algorithmic}[1]
	\State $\mathit{pnt2} \gets \left(\mathit{pnt1} + 1\right)$
	\Cost{$c_{1}$}{$2$}
	\label{alg:arlabeling-neighbour-case2-3-n4}
	\State $\mathit{pnt2\_index} \gets \left(\left(\mathit{pnt2}\right) - 1\right) \cdot 7$
	\Cost{$c_{2}$}{$4$}
	\State $\mathit{work2}\left[\mathit{pnt2\_index} + 0\right] \gets
	 \mathit{work2}\left[\mathit{pnt2\_index} + 0\right] + 1$
	\Cost{$c_{3}$}{$6$}
	\label{alg:arlabeling-neighbour-case2-3-incregion}
	\State $\mathit{work2}\left[\mathit{pnt2\_index} + 1\right] \gets
	 \mathit{work2}\left[\mathit{pnt2\_index} + 1\right] + i$
	\Cost{$c_{4}$}{$6$}
	\State $\mathit{work2}\left[\mathit{pnt2\_index} + 2\right] \gets
	 \mathit{work2}\left[\mathit{pnt2\_index} + 2\right] + j$
	\Cost{$c_{5}$}{$6$}
	\If{$\mathit{work2}\left[\mathit{pnt2\_index} + 3\right] > i$}
	\Cost{$c_{6}$}{$3$}
	\label{alg:arlabeling-neighbour-case2-3-isismaller}
		\State $\mathit{work2}\left[\mathit{pnt2\_index} + 3\right] \gets i$
		\Cost{$c_{7}$}{$3$}
		\label{alg:arlabeling-neighbour-case2-3-newi}
	\EndIf
	\State $\mathit{work2}\left[\mathit{pnt2\_index} + 6\right] \gets j$
	\Cost{$c_{9}$}{$3$}
\end{algorithmic}
\end{algorithm}


% paragraph fall_2_ (end)

\paragraph{3. Fall:} % (fold)
\label{par:fall_3_}
Beim dritten Fall wird die Markierung $N_2$ untersucht und wir wissen, dass $N_3$ und $N_4$ keine Markierungen haben
 können. Demnach kann es nur den Nachbarn $N_2$ geben, dessen Markierung an $I(u,v)$ weitergereicht wird. Falls $N_1$
 ebenfalls ein Vordergrundpixel ist, handelt es sich um die gleiche Markierung wie in $N_2$, da beide Vordergrundpixel
 direkt miteinander verbunden sind. Das Verfahren ist in \autoref{alg:arlabeling-neighbour-case3} dargestellt.

% Fall 3: *(pnt1-1) > 0
\begin{algorithm}[ht]
\caption{\textproc{arLabeling} (8er-Nachbarschaft: 3. Fall)}
\label{alg:arlabeling-neighbour-case3}
\begin{algorithmic}[1]
	\algrestore{brk-case2}
	\ElsIf{$\left(\mathit{pnt1} - 1\right) > 0$}
		\State $\mathit{pnt2} \gets \left(\mathit{pnt1} - 1\right)$
		\State $\mathit{pnt2\_index} \gets \left(\mathit{pnt2} - 1\right) \cdot 7$
		\label{alg:arlabeling-neighbour-case3-offset}
		\State $\mathit{work2}\left[\mathit{pnt2\_index} + 0\right] \gets \mathit{work2}\left[\mathit{pnt2\_index} + 0\right] + 1$
		\State $\mathit{work2}\left[\mathit{pnt2\_index} + 1\right] \gets \mathit{work2}\left[\mathit{pnt2\_index} + 1\right] + i$
		\State $\mathit{work2}\left[\mathit{pnt2\_index} + 2\right] \gets \mathit{work2}\left[\mathit{pnt2\_index} + 2\right] + j$
		\If{$\mathit{work2}\left[\mathit{pnt2\_index} + 4\right] < i$}
			\State $\mathit{work2}\left[\mathit{pnt2\_index} + 4\right] \gets i$
			\label{alg:arlabeling-neighbour-case3-newi}
		\EndIf
		\State $\mathit{work2}\left[\mathit{pnt2\_index} + 6\right] \gets j$
		\algstore{brk-case3}
\end{algorithmic}
\end{algorithm}


Für diesen Fall müssen nur die Daten in $\mathit{work2}$ gespeichert werden. Der Wert von $N_2$ wird im Regionenbild
 $\mathit{l\_image}$ gespeichert. Danach wird in Zeile \ref{alg:arlabeling-neighbour-case3-offset} der Adressabstand
 zur Speicherung der Daten in $\mathit{work2}$ berechnet. Danach wird die Anzahl der Vordergrundpixel der Region erhöht
 und $i$ und $j$ auf die bestehenden Werte in $\mathit{work2}$ addiert. Falls die $x$-Koordinate des letzten
 Vordergrundpixels der Region kleiner als der aktuelle Wert $i$ ist, wird die $x$-Koordinate in Zeile
 \ref{alg:arlabeling-neighbour-case3-newi} ersetzt. Zum Schluss wird die $y$-Koordinate des letzten Vordergrundpixels
 der Region mit dem Wert $j$ aktualisiert.

% paragraph fall_3_ (end)

\paragraph{4. Fall:} % (fold)
\label{par:fall_4_}
Der vierte Fall untersucht den letzten, und einzigen, Nachbarn $N_1$ (\autoref{alg:arlabeling-neighbour-case4}). Alle
 anderen Nachbarn sind keine Vordergrundpixel und es besteht keine Kollision. Die Markierung von $N_1$ wird für
 $I(u,v)$ übernommen (Zeile \ref{alg:arlabeling-neighbour-case4-n1}). Ansonsten sind
 \autoref{alg:arlabeling-neighbour-case4} und \autoref{alg:arlabeling-neighbour-case3} identisch.

% Fall 4: *(pnt2-1) > 0
\begin{algorithm}[ht]
\caption{\textproc{arLabeling} (8er-Nachbarschaft: 4. Fall)}
\label{alg:arlabeling-neighbour-case4}
\begin{algorithmic}[1]
	\algrestore{brk-case3}
	\ElsIf{$\left(\mathit{pnt2} - 1\right) > 0$}
		\State $\mathit{pnt2} \gets \left(\mathit{pnt2} - 1\right)$
		\label{alg:arlabeling-neighbour-case4-n1}
		\State $\mathit{pnt2\_index} \gets \left(\left(\mathit{pnt2}\right) - 1\right) \cdot 7$
		\State $\mathit{work2}\left[\mathit{pnt2\_index} + 0\right] \gets \mathit{work2}\left[\mathit{pnt2\_index} + 0\right] + 1$
		\State $\mathit{work2}\left[\mathit{pnt2\_index} + 1\right] \gets \mathit{work2}\left[\mathit{pnt2\_index} + 1\right] + i$
		\State $\mathit{work2}\left[\mathit{pnt2\_index} + 2\right] \gets \mathit{work2}\left[\mathit{pnt2\_index} + 2\right] + j$
		\If{$\mathit{work2}\left[\mathit{pnt2\_index} + 4\right] < i$}
			\State $\mathit{work2}\left[\mathit{pnt2\_index} + 4\right] \gets i$
		\EndIf
	\algstore{brk-case4}
\end{algorithmic}
\end{algorithm}


% paragraph fall_4_ (end)

\paragraph{5. Fall:} % (fold)
\label{par:fall_5_}
Im letzten Fall sind alle Nachbarn Hintergrundpixel und $\mathit{l\_image}$ wird eine neue Markierung für $I(u,v)$
 zugewiesen (\autoref{alg:arlabeling-neighbour-case5}).

% Fall 5: else
\begin{algorithm}[!ht]\small
\caption{\textproc{arLabeling} (8er-Nachbarschaft: 5. Fall)}
\label{alg:arlabeling-neighbour-case5}
\begin{algorithmic}[1]
	\algrestore{brk-case4}
	\Else
		\State $\mathit{wk\_max} \gets \mathit{wk\_max} + 1$
		\Cost{$c_{36}$}{$2$}
		\label{alg:arlabeling-neighbour-case5-incwk_max}
		\If{$\mathit{wk\_max} > \mathit{WORK\_SIZE}$}
		\Cost{$c_{37}$}{$1$}
			\State{\textbf{return} $0$}
			\Cost{$c_{38}$}{$1$}
		\EndIf
		\State $\mathit{pnt2} \gets \mathit{wk\_max}$
		\Cost{$c_{40}$}{$1$}
		\label{alg:arlabeling-neighbour-case5-save-uv}
		\State $\mathit{work}\left[\mathit{wk\_max} - 1\right] \gets \mathit{wk\_max}$
		\Cost{$c_{41}$}{$3$}
		\State $\mathit{wmax\_idx} \gets \left(\mathit{wk\_max} - 1\right) \cdot 7$
		\Cost{$c_{42}$}{$3$}
		\label{alg:arlabeling-neighbour-case5-offset}
		\State $\mathit{work2}\left[\mathit{wmax\_idx} + 0 \right] \gets 1$
		\Cost{$c_{43}$}{$3$}
		\label{alg:arlabeling-neighbour-case5-offset-0}
		\State $\mathit{work2}\left[\mathit{wmax\_idx} + 1 \right] \gets i$
		\Cost{$c_{44}$}{$3$}
		\label{alg:arlabeling-neighbour-case5-offset-1}
		\State $\mathit{work2}\left[\mathit{wmax\_idx} + 2 \right] \gets j$
		\Cost{$c_{45}$}{$3$}
		\State $\mathit{work2}\left[\mathit{wmax\_idx} + 3 \right] \gets i$
		\Cost{$c_{46}$}{$3$}
		\State $\mathit{work2}\left[\mathit{wmax\_idx} + 4 \right] \gets i$
		\Cost{$c_{47}$}{$3$}
		\State $\mathit{work2}\left[\mathit{wmax\_idx} + 5 \right] \gets j$
		\Cost{$c_{48}$}{$3$}
		\State $\mathit{work2}\left[\mathit{wmax\_idx} + 6 \right] \gets j$
		\Cost{$c_{49}$}{$3$}
		\label{alg:arlabeling-neighbour-case5-offset-6}
	\EndIf
\end{algorithmic}
\end{algorithm}


In Zeile \ref{alg:arlabeling-neighbour-case5-incwk_max} wird der aktuelle numerische Markierungswert erhöht. Falls der
 Wert in $\mathit{wk\_max}$ größer als ein festgelegter Wert ist, wird das Verfahren abgebrochen, da zuviele Regionen
 im Bildsignal $I$ vorkommen. Die Markierung wird in Zeile \ref{alg:arlabeling-neighbour-case5-save-uv} in
 $\mathit{l\_image}$ gespeichert. Danach wird die der Wert der Markierung in die Liste der Markierungen $\mathit{work}$
 eingetragen. Der Adressabstand für die Region mit der Markierung $\mathit{wk\_max}$ wird in Zeile
 \ref{alg:arlabeling-neighbour-case5-offset} berechnet. Anschliessend wird die neue Region in $\mathit{work2}$
 gespeichert. Der erste Vordergrundpixel wird in Zeile \ref{alg:arlabeling-neighbour-case5-offset-0} an Position $0$
 von $\mathit{work2}$ gespeichert. Danach werden in Zeile
 \ref{alg:arlabeling-neighbour-case5-offset-1}--\ref{alg:arlabeling-neighbour-case5-offset-1} die Position des ersten
 Vordergrundpixels gespeichert.

% paragraph fall_5_ (end)

Nachdem die Regionemarkierung abgeschlossen ist, enthält $\mathit{work}$ die numerischen Werte der Regionen. In
 $\mathit{work2}$ ist für jede Region die Anzahl der Vordergrundpixel, die summierten $x$- und $y$-Koordinaten aller
 Vordergrundpixel, sowie die $x$- und $y$-Koordinate für den ersten und letzten Vordergrundpixel, hinterlegt. Das
 Regionenbild ist in $\mathit{l\_image}$ gespeichert. Die erfassten Daten müssen nun in Schritt \ref{label-cleaning} von
 \autoref{alg:arlabeling-overview} aufbereitet werden. Zuerst werden mit \autoref{alg:sortlabels} die Werte
 in $\mathit{work}$ sortiert.

Die aufsteigenden Markierungswerte in $\mathit{work}$ sind duch Kollisionen in
 \autoref{alg:arlabeling-neighbour-case2-1} durch kleinere Werte ersetzt worden. Dadurch ist eine Lücke im Interval
 $\left[1,2,3..n\right]$ der numerischen Werte entstanden. Diese Lücken werden durch \autoref{alg:sortlabels}
 entfernt, um wieder ein aufsteigendes Interval zu erhalten.

\begin{algorithm}[ht]
\caption{\textproc{arLabeling} (Sortiere Markierungen)}
\label{alg:sortlabels}
\begin{algorithmic}[1]
	\State $j \gets 1$
	\label{alg:sortlabels-j}
	\State $\mathit{wk} \gets \mathit{work}\left[0\right]$
	\label{alg:sortlabels-address}
	\For{$i \gets 1$ \textbf{to} $i \leq \mathit{wk\_max}$}
	\label{alg:sortlabels-loop-start}
		\If{$\mathit{wk} == i$}
			\State $\mathit{wk} \gets j$
			\State $j \gets j + 1$
			\label{alg:sortlabels-savelabel}
		\Else
		\label{alg:sortlabels-collision-start}
			\State $\mathit{wk} \gets \mathit{work}\left[\mathit{wk} - 1\right]$ \Comment Letzten Markierungswert $\mathit{work}$ zuweisen.
		\EndIf
		\label{alg:sortlabels-collision-end}
		\State $i \gets i + 1$
		\label{alg:sortlabels-inci}
		\State Inkrementiere $\mathit{wk}$
		\label{alg:sortlabels-incwk}
	\EndFor
	\label{alg:sortlabels-loop-end}
\end{algorithmic}
\end{algorithm}


In Zeile \ref{alg:sortlabels-j} wird $j$ als Variable für die aktuelle Markierung initialisiert. In Zeile
 \ref{alg:sortlabels-address} wird die Adresse der ersten Markierung aus $\mathit{work}$ in $\mathit{wk}$ gespeichert.
 Die Schleife in Zeile \ref{alg:sortlabels-loop-start}--\ref{alg:sortlabels-loop-end} überprüft, ob die numerischen
 Werte ohne Lücken durch Kollisionen gespeichert sind. Dazu wird jeder Eintrag in $\mathit{work}$ mit der Laufvariable
 $i$ verglichen. Falls keine Kollisionen vorliegen, werden die Werte in $\mathit{work}$ immer mit $i$ übereinstimmen,
 da keine Lücken im Intervall vorhanden sind. In Zeile \ref{alg:sortlabels-savelabel} wird dann der Wert aus $j$ an die
 aktuelle Position von $\mathit{work}$ geschrieben und $j$ danach inkrementiert. Bei einer Kollision stimmt der Wert in
 $\mathit{work}$ nicht mit $i$ überein
 (Zeile \ref{alg:sortlabels-collision-start}--\ref{alg:sortlabels-collision-end}). In diesem Fall wird der Wert an der
 aktuellen Position von $\mathit{work}$ ausgelesen und dekrementiert. Dieser Wert wird als Index verwendet um den Wert
 des zuletzt zugewiesenen Markierungswert an die aktuelle Position von $\mathit{work}$ zu schreiben. Da die Variable
 $j$ in diesem Fall nicht inkrementiert wird, enthält sie den nächsten Wert des Markierungsintervals. In Zeile
 \ref{alg:sortlabels-inci} wird die Laufvariable $i$ um $1$ erhöht und in Zeile \ref{alg:sortlabels-incwk} die Adresse
 von $\mathit{work}$ inkrementiert. Das Beispiel in \autoref{fig:} illustriert das Verfahren.

Das Verfahren in \autoref{alg:arlabeling-initlabelmemory} initialisiert den Speicher zur Berechnung des Flächeninhalts
 einer Region und den Speicher der Koordinaten des Mittelpunkts einer Region. Dazu wird in Zeile
 \ref{alg:arlabeling-initlabelmemory-label-start}--\ref{alg:arlabeling-initlabelmemory-label-end} der größte
 Markierungswert aus $\mathit{work}$ in $\mathit{wlabel\_num}$ und $\mathit{label\_num}$ gespeichert. Falls in der
 Überprüfung in Zeile
 \ref{alg:arlabeling-initlabelmemory-islabel-start}--\ref{alg:arlabeling-initlabelmemory-islabel-end}
 $\mathit{label\_num}$ keine Markierung enthält, wird das Verfahren abgebrochen und das Regionenbild $\mathit{l\_image}$
 zurückgegeben. Andernfalls wird in Zeile
 \ref{alg:arlabeling-initlabelmemory-initwarea-start}--\ref{alg:arlabeling-initlabelmemory-initwarea-end} der Speicher
 des Flächeinhalts $\mathit{warea}$ mit dem Wert $0$ initialisiert, indem über die Größe des Speichers iteriert wird.
 Der Speicher zur Berechnung des Mittelpunkts der Regionen wird in Zeile
 \ref{alg:arlabeling-initlabelmemory-initwpos-start}--\ref{alg:arlabeling-initlabelmemory-initwpos-end} nach dem
 gleichen Prinzip initialisiert.

\begin{algorithm}[ht]
\caption{\textproc{arLabeling} (Regionenspeicher initialisieren)}
\label{alg:arlabeling-initlabelmemory}
\begin{algorithmic}[1]
	\State $\mathit{wlabel\_num} \gets j - 1$
	\label{alg:arlabeling-initlabelmemory-label-start}
	\State $\mathit{label\_num} \gets j - 1$
	\label{alg:arlabeling-initlabelmemory-label-end}
	\If{$\mathit{label\_num} == 0$}
	\label{alg:arlabeling-initlabelmemory-islabel-start}
		\State \textbf{return} $\mathit{l\_image}$
	\EndIf
	\label{alg:arlabeling-initlabelmemory-islabel-end}
	\State $\mathit{size} \gets \mathit{label\_num} \cdot$ \Call{sizeof}{$int$}
	\label{alg:arlabeling-initlabelmemory-initwarea-start}
	\For{$\mathit{size} > 0$}
		\State $\mathit{warea} \gets 0$
		\State Inkrementiere $\mathit{warea}$
		\State $\mathit{size} \gets \mathit{size} - 1$
	\EndFor
	\label{alg:arlabeling-initlabelmemory-initwarea-end}
	\State $\mathit{size} \gets \mathit{label\_num} \cdot 2 \cdot$ \Call{sizeof}{$float$}
	\label{alg:arlabeling-initlabelmemory-initwpos-start}
	\For{$\mathit{size} > 0$}
		\State $\mathit{wpos} \gets 0$
		\State Inkrementiere $\mathit{wpos}$
		\State $\mathit{size} \gets \mathit{size} - 1$
	\EndFor
	\label{alg:arlabeling-initlabelmemory-initwpos-end}
\end{algorithmic}
\end{algorithm}


In \autoref{alg:arlabeling-calcregiondata} werden die Daten aller Regionenmarkierungen aufbereitet. Das Verfahren
 berechnet den Flächeninhalt, den Mittelpunkt und die Start- und Endkoordinaten für alle Regionen.

\begin{algorithm}[ht]
\caption{\textproc{arLabeling} (Berechne Regionendaten)}
\label{alg:arlabeling-calcregiondata}
\begin{algorithmic}[1]
    % for(i = 0; i < *label_num; i++) {
    %     wclip[i*4+0] = lxsize;
    %     wclip[i*4+1] = 0;
    %     wclip[i*4+2] = lysize;
    %     wclip[i*4+3] = 0;
    % }
	\For{$i \gets 0$ \textbf{to} $i < \mathit{label\_num}$}
	\label{alg:arlabeling-calcregiondata-wclip-start}
		\State $\mathit{wclip}\left[i \cdot 4 + 0\right] \gets \mathit{lxsize}$
		\State $\mathit{wclip}\left[i \cdot 4 + 1\right] \gets 0$
		\State $\mathit{wclip}\left[i \cdot 4 + 2\right] \gets \mathit{lysize}$
		\State $\mathit{wclip}\left[i \cdot 4 + 3\right] \gets 0$
		\State $i \gets i + 1$
	\EndFor
	\label{alg:arlabeling-calcregiondata-wclip-end}
    % for(i = 0; i < wk_max; i++) {
    %     j = work[i] - 1;
    %     warea[j]    += work2[i*7+0];
    %     wpos[j*2+0] += work2[i*7+1];
    %     wpos[j*2+1] += work2[i*7+2];
    %     if( wclip[j*4+0] > work2[i*7+3] ) wclip[j*4+0] = work2[i*7+3];
    %     if( wclip[j*4+1] < work2[i*7+4] ) wclip[j*4+1] = work2[i*7+4];
    %     if( wclip[j*4+2] > work2[i*7+5] ) wclip[j*4+2] = work2[i*7+5];
    %     if( wclip[j*4+3] < work2[i*7+6] ) wclip[j*4+3] = work2[i*7+6];
    % }
	\For{$i \gets 0$ \textbf{to} $i < \mathit{wk\_max}$}
	\label{alg:arlabeling-calcregiondata-work-start}
		\State $j \gets \mathit{work}\left[i\right] - 1$
		\label{alg:arlabeling-calcregiondata-label}
		\State $\mathit{warea}\left[j\right] \gets \mathit{warea}\left[j\right] + \mathit{work2}\left[i \cdot 7 + 0\right]$
		\State $\mathit{wpos}\left[j \cdot 2 + 0\right] \gets \mathit{wpos}\left[j \cdot 2 + 0\right] + \mathit{work2}\left[i \cdot 7 + 1\right]$
		\label{alg:arlabeling-calcregiondata-sumx}
		\State $\mathit{wpos}\left[j \cdot 2 + 1\right] \gets \mathit{wpos}\left[j \cdot 2 + 1\right] + \mathit{work2}\left[i \cdot 7 + 2\right]$
		\label{alg:arlabeling-calcregiondata-sumy}
		\If{$\mathit{wclip}\left[i \cdot 4 + 0\right] > \mathit{work2}\left[i \cdot 7 + 3\right]$}
		\label{alg:arlabeling-calcregiondata-startx-start}
			\State $\mathit{wclip}\left[i \cdot 4 + 0\right] \gets \mathit{work2}\left[i \cdot 7 + 3\right]$
		\EndIf
		\label{alg:arlabeling-calcregiondata-startx-end}

		\If{$\mathit{wclip}\left[i \cdot 4 + 1\right] < \mathit{work2}\left[i \cdot 7 + 4\right]$}
		\label{alg:arlabeling-calcregiondata-endx-start}
			\State $\mathit{wclip}\left[i \cdot 4 + 1\right] \gets \mathit{work2}\left[i \cdot 7 + 4\right]$
		\EndIf
		\label{alg:arlabeling-calcregiondata-endx-end}

		\If{$\mathit{wclip}\left[i \cdot 4 + 2\right] > \mathit{work2}\left[i \cdot 7 + 5\right]$}
		\label{alg:arlabeling-calcregiondata-y-start}
			\State $\mathit{wclip}\left[i \cdot 4 + 2\right] \gets \mathit{work2}\left[i \cdot 7 + 5\right]$
		\EndIf

		\If{$\mathit{wclip}\left[i \cdot 4 + 3\right] < \mathit{work2}\left[i \cdot 7 + 6\right]$}
			\State $\mathit{wclip}\left[i \cdot 4 + 3\right] \gets \mathit{work2}\left[i \cdot 7 + 6\right]$
		\EndIf
		\label{alg:arlabeling-calcregiondata-y-end}

		\State $i \gets i + 1$
	\EndFor
	\label{alg:arlabeling-calcregiondata-work-end}
    % for( i = 0; i < *label_num; i++ ) {
    %     wpos[i*2+0] /= warea[i];
    %     wpos[i*2+1] /= warea[i];
    % }
	\For{$i \gets 0$ \textbf{to} $\mathit{label\_num}$}
	\label{alg:arlabeling-calcregiondata-pos-start}
		\State $\mathit{wpos}\left[i \cdot 2 + 0\right] \gets \mathit{wpos}\left[i \cdot 2 + 0\right] / \mathit{warea}\left[i\right]$
		\State $\mathit{wpos}\left[i \cdot 2 + 1\right] \gets \mathit{wpos}\left[i \cdot 2 + 1\right] / \mathit{warea}\left[i\right]$
		\State $i \gets i + 1$
	\EndFor
	\label{alg:arlabeling-calcregiondata-pos-end}
	% *label_ref = work;
	% *area      = warea;
	% *pos       = wpos;
	% *clip      = wclip;
	% return( l_image );
	\State $\mathit{label\_ref} \gets \mathit{work}$
	\State $\mathit{area} \gets \mathit{warea}$
	\State $\mathit{pos} \gets \mathit{wpos}$
	\State $\mathit{clip} \gets \mathit{wclip}$
	\State \textbf{return} $\mathit{l\_image}$
\end{algorithmic}
\end{algorithm}


Dazu wird in Zeile \ref{alg:arlabeling-calcregiondata-wclip-start}--\ref{alg:arlabeling-calcregiondata-wclip-end} der
 Speicher der Start- und Endkoordinaten initialisiert, indem über die Menge der Regionen itertiert wird und die
 Initialwerte, $\mathit{lxsize}$ und $0$ für die Startkoordiante und $\mathit{lysize}$ und $0$ für die Endkoordinate,
 gespeichert werden. In Zeile
 \ref{alg:arlabeling-calcregiondata-work-start}--\ref{alg:arlabeling-calcregiondata-work-end} werden die Werte aus
 $\mathit{work2}$ aufbereitet, indem über die Anzahl der Regionen iteriert wird. In Zeile
 \ref{alg:arlabeling-calcregiondata-label} wird die Regionemarkierung ausgelesen und in $j$ gespeichert. Danach wird
 die Anzahl der Vordergrundpixel für Region $j$ aus $\mathit{work2}$ ausgelesen und in $\mathit{warea}$ gespeichert. In
 Zeile \ref{alg:arlabeling-calcregiondata-sumx}--\ref{alg:arlabeling-calcregiondata-sumy} werden die summierten $x$- und
 $y$-Koordinaten aus $\mathit{work2}$ ausgelsen und in $\mathit{wpos}$ gespeichert. Die Überprüfung des $x$-Werts in
 Zeile \ref{alg:arlabeling-calcregiondata-startx-start}--\ref{alg:arlabeling-calcregiondata-startx-end} sorgt dafür,
 dass die Startkoordinate soweit links wie möglich beginnt. Der $y$-Wert der Startkoordinate wird in Zeile
 \ref{alg:arlabeling-calcregiondata-starty-start}--\ref{alg:arlabeling-calcregiondata-starty-end} überprüft. Bei dieser
 Überprüfung soll der $y$-Wert soweit oben wie möglich liegen. Zeile
 \ref{alg:arlabeling-calcregiondata-endcoord-start}--\ref{alg:arlabeling-calcregiondata-endcoord-end} wiederholen das
 Verfahren für den $x$- und $y$-Wert der Endkoordinate. Danach wird in Zeile
 \ref{alg:arlabeling-calcregiondata-pos-start}--\ref{alg:arlabeling-calcregiondata-pos-end} die Position
 des Mittelpunkts einer Region berechnet, indem die aufsummierten Koordinaten einer Region durch die Anzahl der
 Vordergrundpixel in $\mathit{warea}$ geteilt wird. Zuletzt werden die lokalen Variablen in den Übergabeparametern
 gespeichert und das Regionenbild $\mathit{l\_image}$ an die aufrufende Methode zurückgegeben. An dieser Stelle ist
 \autoref{alg:arlabeling-overview} beendet.

% subsubsection regionenmarkierung (end)


\clearpage

\subsubsection{Konturerzeugung} % (fold)
\label{sec:konturerzeugung}

Das zweite Verfahren der Fiducial Detection ermittelt aus einer Regionenmarkierung mit dem Verfahren
 \textproc{arGetContour} eine Kontur. Nachdem \textproc{arLabeling} (\autoref{alg:arlabeling-overview}) beendet ist,
 wird das Regionenbild an die Methode \textproc{arDetectMarker2} übergeben.

In \autoref{alg:detectmarker2-1} werden die lokalen Variablen initialisiert, deren Bedeutung bei ihrem ersten Einsatz
 erläutert wird.
\begin{algorithm}[!ht]\small
\caption{\textproc{arDetectMarker2} (Initialisierung)}
\label{alg:detectmarker2-1}
\begin{algorithmic}[1]
	\Require $
	\begin{aligned}
		& \mathit{limage}, \mathit{label\_num}, \mathit{label\_ref}, \mathit{warea}, \mathit{wpos}, \mathit{wclip}, \mathit{area\_max}, \mathit{area\_min},\\
		& \mathit{factor}, \mathit{marker\_num}
	\end{aligned}$
	\State $\mathit{pm}, i, j, d, \mathit{ret} \gets \infty$
	\State $\mathit{area\_min} \gets \mathit{area\_min} / 4$
	\State $\mathit{area\_max} \gets \mathit{area\_max} / 4$
	\State $\mathit{xsize} \gets \mathit{arImXsize} / 2$
	\State $\mathit{ysize} \gets \mathit{arImYsize} / 2$
	\algstore{brk-detectmarker2-init}
\end{algorithmic}
\end{algorithm}

Die Laufzeitfunktion von \autoref{alg:detectmarker2-1} ist $T(n) = 13$. \autoref{alg:detectmarker2-2} sorgt mit seiner
 Schleife in Zeile \ref{alg:detectmarker2-2-loop-start}--\ref{alg:detectmarker2-2-loop-end} für die Konturenermittlung.
\begin{algorithm}[ht]
\caption{\textproc{arDetectMarker2} (Fortsetzung)}
\label{alg:detectmarker2-2}
\begin{algorithmic}[1]
	\algrestore{brk-detectmarker2-init}
	\State $\mathit{marker\_num2} \gets 0$
	\label{alg:detectmarker2-2-markernum2}
	\For{$i \gets 0$ \textbf{to} $i < \mathit{label\_num}$}
	\label{alg:detectmarker2-2-loop-start}
		\If{$\mathit{warea}[i] < \mathit{area\_min} \lor \mathit{warea}[i] > \mathit{area\_max}$}
		\label{alg:detectmarker2-2-area-start}
			\State \textbf{continue}
		\EndIf
		\label{alg:detectmarker2-2-area-end}
		\If{$\mathit{wclip}[i \cdot 4 + 0] == 1 \lor \mathit{wclip}[i \cdot 4 + 1] == \mathit{xsize} - 2$}
		\label{alg:detectmarker2-2-checkx-start}
			\State \textbf{continue}
			\label{alg:detectmarker2-2-checkx-continue}
		\EndIf
		\label{alg:detectmarker2-2-checkx-end}
		\If{$\mathit{wclip}[i \cdot 4 + 2] == 1 \lor \mathit{wclip}[i \cdot 4 + 3] == \mathit{ysize} - 2$}
		\label{alg:detectmarker2-2-checky-start}
			\State \textbf{continue}
		\EndIf
		\label{alg:detectmarker2-2-checky-end}
		\State $\mathit{ret} \gets$ \textproc{arGetContour}$\left(
		\begin{aligned}
			& \mathit{limage}, \mathit{label\_ref}, i + 1,\\
			& \mathit{wclip}[i \cdot 4], \mathit{marker\_infoTWO}[\mathit{marker\_num2}]
		\end{aligned}\right)$
		\label{alg:detectmarker2-2-callcontour}
		\State \ldots
		\State $i \gets i + 1$
	\EndFor
	\label{alg:detectmarker2-2-loop-end}
\end{algorithmic}
\end{algorithm}

Dazu wird in Zeile \ref{alg:detectmarker2-2-markernum2} die Variable $\mathit{marker\_num2}$ initialisiert, die zur
 Identifizierung einer Region dient. In Zeile \ref{alg:detectmarker2-2-loop-start}--\ref{alg:detectmarker2-2-loop-end}
 wird jede Regionenmarkierung untersucht. Dazu wird in Zeile
 \ref{alg:detectmarker2-2-area-start}--\ref{alg:detectmarker2-2-area-end} die Fläche der Region $i$ mit dem festgelegten
 minimalen und maximalen Wert verglichen. Nur wenn die Fläche der Region sich innerhalb dieser Grenzen befindet, wird
 die Region $i$ weiter untersucht. Andernfalls wird mit der nächsten Region das Verfahren wiederholt. Zeile
 \ref{alg:detectmarker2-2-checkx-start}--\ref{alg:detectmarker2-2-checkx-end} überprüft die $x$-Koordinaten der
 Start- und Endposition, die in $\mathit{wclip}$ gespeichert sind. Wenn die Koordinaten an den Rändern des
 Regionenbildes $\mathit{limage}$ liegen, wird die weitere Untersuchung in Zeile
 \ref{alg:detectmarker2-2-checkx-continue} abgebrochen. Nur Regionen, die nicht an den Rändern liegen, eignen sich zur
 Konturenermittlung. In Zeile \ref{alg:detectmarker2-2-checky-start}--\ref{alg:detectmarker2-2-checky-end} wird das
 Verfahren für die $y$-Koordinaten wiederholt. Nur wenn eine Region eine festgelegte Größe nicht unter- oder
 überschreitet und die Region nicht an den Rändern liegt,
 eignet sie sich zur Konturenermittlung und wird mit der Methode \textproc{arGetContour} in Zeile
 \ref{alg:detectmarker2-2-callcontour} aufgerufen. Die Laufzeitfunktion ist in \autoref{eq:analyse-detectmarker2-1}
 angegeben.
\begin{subequations}
\label{eq:analyse-detectmarker2-1}
\begin{align}
\label{eq:analyse-detectmarker2-1-1}
T(\mathit{label\_num})& =
c_{6}
+ c_{7}(\mathit{label\_num} + 1)
+ 5c_{8}(\mathit{label\_num})
+ c_{9}(\mathit{label\_num})
\\
& \quad
+ 10c_{11}(\mathit{label\_num})
+ 10c_{14}(\mathit{label\_num})
+ tc_{17}(\mathit{label\_num})
\nonumber \\
& \quad
+ c_{19}(\mathit{label\_num})
\nonumber \\
\label{eq:analyse-detectmarker2-1-2}
T(\mathit{label\_num})& =
tc_{17}(\mathit{label\_num})
\\
& \quad
+ (c_{7} + 5c_{8} + c_{9} + 10c_{11} + 10c_{14} + c_{19}) (\mathit{label\_num})
+ (c_{6} + c_{7})
\nonumber
\end{align}
\end{subequations}


\textproc{arGetContour} (\autoref{alg:argetcontour-1}--\autoref{alg:argetcontour-3}) ermittelt für eine Region eine
 Kontur. In \autoref{alg:argetcontour-1} werden die Variablen in Zeile
 \ref{alg:argetcontour-1-initvar-start}--\ref{alg:argetcontour-1-initvar-end} initialisiert.
\begin{algorithm}[ht]
\caption{\textproc{arGetContour} (Initialisierung)}
\label{alg:argetcontour-1}
\begin{algorithmic}[1]
	\Require $\mathit{limage},\mathit{label\_ref},\mathit{label},\mathit{wclip}\left[4\right],\mathit{marker\_infoTWO}$ 
	% static const int      xdir[8] = { 0, 1, 1, 1, 0,-1,-1,-1};
	% static const int      ydir[8] = {-1,-1, 0, 1, 1, 1, 0,-1};
	\State $\mathit{xdir} \gets \left[0, 1, 1, 1, 0,-1,-1,-1\right]$
	\label{alg:argetcontour-1-initvar-start}
	\State $\mathit{ydir} \gets \left[-1,-1, 0, 1, 1, 1, 0,-1\right]$
	\label{alg:argetcontour-1-initvar-neighbour}
	% ARInt16         *p1;
	% int             xsize, ysize;
	% int             sx, sy, dir;
	% int             dmax, d, v1 = 0;
	% int             i, j;
	\State $\mathit{p1} \gets \infty$
	\State $\mathit{sx} \gets \infty$
	\State $\mathit{sy} \gets \infty$
	\State $\mathit{dir} \gets \infty$
	\State $\mathit{dmax} \gets \infty$
	\State $d \gets \infty$
	\State $\mathit{v1} \gets 0$
	\State $i \gets \infty$
	\State $j \gets \infty$
	%     xsize = arImXsize / 2;
	%     ysize = arImYsize / 2;
	\State $\mathit{xsize} \gets \mathit{arImXsize} / 2$
	\label{alg:argetcontour-1-initsizex}
	\State $\mathit{ysize} \gets \mathit{arImYsize} / 2$
	\label{alg:argetcontour-1-initvar-end}
	% j = clip[2];
	% p1 = &(limage[j*xsize+clip[0]]);
	\State $j \gets \mathit{clip}\left[2\right]$
	\State $\mathit{offset} \gets j \cdot \mathit{xsize} + \mathit{clip}\left[0\right]$
	\label{alg:argetcontour-1-offset}
	\State $\mathit{p1} \gets \mathit{limage}\left[\mathit{offset}\right]$
	\label{alg:argetcontour-1-p1}
	% for( i = clip[0]; i <= clip[1]; i++, p1++ ) {
	%     if( *p1 > 0 && label_ref[(*p1)-1] == label ) {
	%         sx = i; sy = j; break;
	%     }
	% }
	\For{$i \gets \mathit{clip}\left[0\right]$ \textbf{to} $i \leq \mathit{clip}\left[1\right]$}
	\label{alg:argetcontour-1-loop-start}
		\If{$\mathit{p1} > 0 \land \mathit{label\_ref}\left[\mathit{p1 - 1}\right] == \mathit{label}$}
		\label{alg:argetcontour-1-haslabel-start}
			\State $\mathit{sx} \gets i$
			\label{alg:argetcontour-1-savex}
			\State $\mathit{sy} \gets j$
			\label{alg:argetcontour-1-savey}
			\State \textbf{break}
		\EndIf
		\label{alg:argetcontour-1-haslabel-end}
		\State $i \gets i + 1$
		\label{alg:argetcontour-1-inci}
		\State Inkrementiere $\mathit{p1}$
		\label{alg:argetcontour-1-incp1}
	\EndFor
	\label{alg:argetcontour-1-loop-end}
	\algstore{brk-argetcontour-init}
\end{algorithmic}
\end{algorithm}

Die 8er-Nachbarschaft zur Verfolgung einer Kontur wird in Zeile
 \ref{alg:argetcontour-1-initvar-start}--\ref{alg:argetcontour-1-initvar-neighbour} initialisiert. Die Bildbreite und
 Bildhöhe werden in Zeile \ref{alg:argetcontour-1-initsizex}--\ref{alg:argetcontour-1-initvar-end} in $\mathit{xsize}$
 und $\mathit{ysize}$ gespeichert. Danach wird in Zeile \ref{alg:argetcontour-1-offset} ein Adressabstand berechnet, um
 den Startpunkt der ersten Regionenmarkierung in $\mathit{l\_image}$ in $\mathit{p1}$ zu speichern (Zeile
 \ref{alg:argetcontour-1-p1}). In der Schleife in Zeile
 \ref{alg:argetcontour-1-loop-start}--\ref{alg:argetcontour-1-loop-end} wird der Startpunkt der Region gesucht. Dazu
 wird in Zeile \ref{alg:argetcontour-1-haslabel-start}--\ref{alg:argetcontour-1-haslabel-end} geprüft, ob an der
 Adresse $\mathit{p1}$ eine Regionenmarkierung gespeichert ist und ob der Wert der Regionenmarkierung in
 $\mathit{label\_ref}$ mit der Markierung in $\mathit{label}$ übereinstimmt. Falls die Werte übereinstimmen, werden die
 Koordinaten in Zeile \ref{alg:argetcontour-1-savex}--\ref{alg:argetcontour-1-savey} in den Variablen $\mathit{sx}$ und
 $\mathit{sy}$ gespeichert und die Schleife wird abgebrochen. Ansonsten wird in Zeile
 \ref{alg:argetcontour-1-inci}--\ref{alg:argetcontour-1-incp1} die Laufvariable $i$ um $1$ erhöht und die Adresse
 $\mathit{p1}$ inkrementiert. Der Bereich $[\mathit{clip}[0],\mathit{clip}[1])$ hängt von $\mathit{xsize}$ ab und
 umfasst maximal $\mathit{xsize} - 4$ Einträge (Vgl. \autoref{alg:detectmarker2-2}, Zeile
 \ref{alg:detectmarker2-2-checkx-start}). Die Laufzeitfunktion ist in \autoref{eq:analyse-argetconout-init} aufgeführt.
\begin{subequations}
\label{eq:analyse-argetconout-init}
\begin{align}
\label{eq:analyse-argetconout-init-1}
T_{worst}(\mathit{xsize})& = (c_{1} + c_{2} + c_{3}6 + c_{4} + c_{5}2 + c_{6}2 + c_{7}2 + c_{8}2 + c_{9}4 + c_{10}2)
\\
& \quad
+ c_{11}(\mathit{xsize} - 3)
+ (c_{12}5 + c_{17} + c_{18})(\mathit{xsize} - 4)
\nonumber \\
\label{eq:analyse-argetconout-init-2}
T_{worst}(\mathit{xsize})& =
(c_{11} + c_{12}5 + c_{17} + c_{18})\mathit{xsize}
- (3c_{11} + 20c_{12} + 4c_{17} + 4c_{18})
\\
& \quad
+  (c_{1} + c_{2} + c_{3}6 + c_{4} + c_{5}2 + c_{6}2 + c_{7}2 + c_{8}2 + c_{9}4 + c_{10}2)
\nonumber
\end{align}
\end{subequations}

Die Wachstumsrate von \autoref{eq:analyse-argetconout-init-2} ist, für $c_{1} = 7$, $c_{2} = 8$ und
 $\mathit{xsize}_0 = 8$, $8\mathit{xsize} - 8 = \Theta(\mathit{xsize})$.

\autoref{alg:argetcontour-2} verfolgt nun die Kontur der Region. Dazu wird zuerst in der Datenstruktur
 $\mathit{marker\_infoTWO}$ die Anzahl der Koordinaten initialisiert und die $x$- und $y$-Werte der ersten
 Regionenmarkierung gespeichert (Zeile
 \ref{alg:argetcontour-2-markerinit-start}--\ref{alg:argetcontour-2-markerinit-end}).
\begin{algorithm}[!ht]
\caption{\textproc{arGetContour} (Verfolge Kontur)}
\label{alg:argetcontour-2}
\begin{algorithmic}[1]
	\algrestore{brk-argetcontour-init}
	% marker_infoTWO->coord_num = 1;
	% marker_infoTWO->x_coord[0] = sx;
	% marker_infoTWO->y_coord[0] = sy;
	\State $\mathit{marker\_infoTWO.coord\_num} \gets 1$
	\Cost{$c_{20}$}{$2$}
	\label{alg:argetcontour-2-markerinit-start}
	\State $\mathit{marker\_infoTWO.xcoord}\left[0\right] \gets \mathit{sx}$
	\Cost{$c_{21}$}{$3$}
	\State $\mathit{marker\_infoTWO.ycoord}\left[0\right] \gets \mathit{sy}$
	\Cost{$c_{22}$}{$3$}
	\label{alg:argetcontour-2-markerinit-end}
	% dir = 5;
	\State $\mathit{dir} \gets 5$
	\Cost{$c_{23}$}{$1$}
	\label{alg:argetcontour-2-orientation}
	% for(;;) {
	\For{\textbf{true}}
	\label{alg:argetcontour-2-contourloop-start}
	%     p1 = &(limage[marker_infoTWO->y_coord[marker_infoTWO->coord_num-1] * xsize
	%                 + marker_infoTWO->x_coord[marker_infoTWO->coord_num-1]]);
		\State $\mathit{tmp\_coord\_num} \gets \mathit{marker\_infoTWO.coord\_num} - 1$
		\Cost{$c_{25}$}{$(n-3) \cdot 3$}
		\label{alg:argetcontour-2-offset-start}
		\State $\mathit{offset} \gets \mathit{marker\_infoTWO.y\_coord}\left[\mathit{tmp\_coord\_num}\right]
		 \cdot \mathit{xsize}$
		\Cost{$c_{26}$}{$(n-3) \cdot 5$}
		\State $\mathit{offset} \gets \mathit{offset} +
		 \mathit{marker\_infoTWO.x\_coord}\left[\mathit{tmp\_coord\_num}\right]$
		\Cost{$c_{27}$}{$(n-3) \cdot 5$}
		\label{alg:argetcontour-2-offset-end}
		\State $\mathit{p1} \gets \mathit{l\_image}[\mathit{offset}]$
		\Cost{$c_{28}$}{$(n-3) \cdot 2$}
		\label{alg:argetcontour-2-address}
	%     dir = (dir+5)%8;
		\State $\mathit{dir} \gets \left(\mathit{dir} + 5\right)\mod 8$
		\Cost{$c_{29}$}{$(n-3) \cdot 3$}
		\label{alg:argetcontour-2-nextorientation}
	%     for(i=0;i<8;i++) {
	%         if( p1[ydir[dir]*xsize+xdir[dir]] > 0 ) break;
	%         dir = (dir+1)%8;
	%     }
		\For{$i \gets 0$ \textbf{to} $i < 8$}
		\Cost{$c_{30}$}{$(n-3) \cdot \bigl((8 - 1) + 1\bigr)$}
		\label{alg:argetcontour-2-neighbourloop-start}
			\State $\mathit{offset} \gets \mathit{ydir}\left[dir\right] \cdot \mathit{xsize} +
			 \mathit{xdir}\left[\mathit{dir}\right]$
			\Cost{$c_{31}$}{$(n-3) \cdot 5(8 - 1)$}
			\label{alg:argetcontour-2-offset}
			\If{$\mathit{p1}\left[\mathit{offset}\right] > 0$}
			\Cost{$c_{32}$}{$(n-3) \cdot 2(8 - 1)$}
			\label{alg:argetcontour-2-haslabel-start}
				\State \textbf{break}
				\Cost{$c_{33}$}{$n-3$}
			\EndIf
			\label{alg:argetcontour-2-haslabel-end}
			\State $\mathit{dir} \gets \left(\mathit{dir} + 1\right)\mod 8$
			\Cost{$c_{35}$}{$(n-3) \cdot 3(8 - 1)$}
			\label{alg:argetcontour-2-incdir}
			\State $i \gets i + 1$
			\Cost{$c_{36}$}{$(n-3)(8 - 1)$}
			\label{alg:argetcontour-2-inci}
		\EndFor
		\label{alg:argetcontour-2-neighbourloop-end}
		% if( i == 8 ) {
		%             printf("??? 2\n"); return(-1);
		%         }
		\If{$i = 8$}
		\Cost{$c_{38}$}{$(n-3) \cdot 1$}
		\label{alg:argetcontour-2-hasnolabel-start}
			\State \textbf{return}$(-1)$
			% \Cost{$c_{39}$}{$9997 \cdot 1$}
		\EndIf
		\label{alg:argetcontour-2-hasnolabel-end}
	%     marker_infoTWO->x_coord[marker_infoTWO->coord_num]
	%         = marker_infoTWO->x_coord[marker_infoTWO->coord_num-1] + xdir[dir];
		\State $\mathit{mkx} \gets \mathit{marker\_infoTWO.x\_coord}\left[\mathit{marker\_infoTwo.coord\_num}\right]$
		\Cost{$c_{41}$}{$(n-3) \cdot 4$}
		\label{alg:argetcontour-2-savex}
		\State $\mathit{mkx} \gets \mathit{marker\_infoTWO.x\_coord}\left[\mathit{marker\_infoTwo.coord\_num}
		 - 1\right] + \mathit{xdir}\left[\mathit{dir}\right]$
		\Cost{$c_{42}$}{$(n-3) \cdot 7$}
	%     marker_infoTWO->y_coord[marker_infoTWO->coord_num]
	%         = marker_infoTWO->y_coord[marker_infoTWO->coord_num-1] + ydir[dir];
		\State $\mathit{mky} \gets \mathit{marker\_infoTWO.y\_coord}
		\left[\mathit{marker\_infoTwo.coord\_num}\right]$
		\Cost{$c_{43}$}{$(n-3) \cdot 4$}
		\State $\mathit{mky} \gets \mathit{marker\_infoTWO.y\_coord}\left[\mathit{marker\_infoTwo.coord\_num} - 1\right]
		 + \mathit{ydir}\left[\mathit{dir}\right]$
		\Cost{$c_{44}$}{$(n-3) \cdot 7$}
		\label{alg:argetcontour-2-savey}
	%     if( marker_infoTWO->x_coord[marker_infoTWO->coord_num] == sx
	%      && marker_infoTWO->y_coord[marker_infoTWO->coord_num] == sy ) break;
		\If{$\mathit{mkx} = sx \land \mathit{mky} = sy$}
		\Cost{$c_{45}$}{$(n-3) \cdot 3$}
		\label{alg:argetcontour-2-iscontourclosed-start}
			\State \textbf{break}
			\Cost{$c_{46}$}{$1$}
		\EndIf
		\label{alg:argetcontour-2-iscontourclosed-end}
	%     marker_infoTWO->coord_num++;
		\State $\mathit{marker\_infoTWO.coord\_num} \gets \mathit{marker\_infoTWO.coord\_num} + 1$
		\Cost{$c_{48}$}{$(n-3) \cdot 4$}
		\label{alg:argetcontour-2-inccoordnum}
		% if( marker_infoTWO->coord_num == AR_CHAIN_MAX-1 ) {
		%             printf("??? 3\n"); return(-1);
		%         }
		\If{$\mathit{marker\_infoTWO.coord\_num} = \mathit{AR\_CHAIN\_MAX} - 1}$
		\Cost{$c_{49}$}{$(n-3) \cdot 3$}
		\label{alg:argetcontour-2-ismaxchainreached-start}
			\State \textbf{return}$(-1)$
			% \Cost{$c_{50}$}{$9997 \cdot 1$}
			\label{alg:argetcontour-2-error}
		\EndIf
		\label{alg:argetcontour-2-ismaxchainreached-end}
	% }
	\EndFor
	\label{alg:argetcontour-2-contourloop-end}
	\algstore{brk-argetcontour-contour}
\end{algorithmic}
\end{algorithm}

Danach wird die Orientiertung der 8er-Nachbarschaft in Zeile \ref{alg:argetcontour-2-orientation} festgelegt. Die
 Kontur wird in der Schleife in Zeile
 \ref{alg:argetcontour-2-contourloop-start}--\ref{alg:argetcontour-2-contourloop-end} verfolgt. Zuerst muss der
 Adressabstand der letzten Regionenmarkierung in $\mathit{l\_image}$ berechnet werden (Zeile
 \ref{alg:argetcontour-2-offset-start}--\ref{alg:argetcontour-2-offset-end}). Danach wird die Adresse der
 Regionenmarkierung in $\mathit{p1}$ hinterlegt (Zeile \ref{alg:argetcontour-2-address}). In Zeile
 \ref{alg:argetcontour-2-nextorientation} wird die Orientierung der 8er-Nachbarschaft auf die nächste zu untersuchende
 Richtung gedreht. Das drehen der 8er-Nachbarschaft erfolgt, indem von der letzten Position einer Kontur im
 Uhrzeigersinn auf den nächsten Nachbarn weitergerückt wird (Vgl. \autoref{fig:}). In Zeile
 \ref{alg:argetcontour-2-neighbourloop-start}--\ref{alg:argetcontour-2-neighbourloop-end} werden in der Schleife die
 Nachbarn im Uhrzeigersinn untersucht. Dazu wird mit Hilfe der 8er-Nachbarschaft in Zeile
 \ref{alg:argetcontour-2-offset} ein Adressabstand berechnet. Durch die Überprüfung einer Regionenmarkierung an der
 Adresse von $\mathit{p1}$ in Zeile \ref{alg:argetcontour-2-haslabel-start}--\ref{alg:argetcontour-2-haslabel-end} wird
 entschieden, ob die Schleife von Zeile
 \ref{alg:argetcontour-2-neighbourloop-start}--\ref{alg:argetcontour-2-neighbourloop-end} abgebrochen wird, weil eine
 Markierung gefunden wurde, oder ob ein weiterer Nachbar untersucht werden muss. Falls ein weiterer Nachbar untersucht
 werden muss, wird in Zeile \ref{alg:argetcontour-2-incdir} die Position der 8er-Nachbarschaft um eine Position
 weitergedreht und in Zeile \ref{alg:argetcontour-2-inci} die Laufvariable $i$ erhöht. Falls alle Nachbarn der
 Regionenmarkierung untersucht worden sind ohne eine weiter Markierung zu finden, wird das Verfahren in Zeile
 \ref{alg:argetcontour-2-hasnolabel-start}--\ref{alg:argetcontour-2-hasnolabel-end} mit einem Fehlerwert abgebrochen.
 Um eine erfolgreiche Iteration zu erlangen, wird die Schleife im schlechtesten Fall $8-1$-mal wiederholt, ohne in Zeile
 \ref{alg:argetcontour-2-hasnolabel-start} mit einem Fehlerwert beendet zu werden. Wenn eine weitere Markierung in
 Zeile \ref{alg:argetcontour-2-neighbourloop-start}--\ref{alg:argetcontour-2-neighbourloop-end} gefunden und die
 Schleife abgebrochen wurde, wird in Zeile \ref{alg:argetcontour-2-savex}--\ref{alg:argetcontour-2-savey} die
 Markierung gespeichert. Dazu werden die $x$- und $y$-Koordinate der letzten Markierung ausgelesen und die Richtung der
 8er-Nachbarschaft addiert. Danach wird in Zeile
 \ref{alg:argetcontour-2-iscontourclosed-start}--\ref{alg:argetcontour-2-iscontourclosed-end} überprüft, ob die
 gespeicherte Markierung mit den Anfangskoordinaten übereinstimmt und es sich somit um eine geschlossene Kontur
 handelt. Falls das Verfahren eine geschlossene Kontur gefunden hat, wird die Schleife von Zeile
 \ref{alg:argetcontour-2-contourloop-start}--\ref{alg:argetcontour-2-contourloop-end} abgebrochen. Falls es sich nicht
 um eine geschlossene Kontur handelt, wird die Kontur weiterverfolgt. In Zeile \ref{alg:argetcontour-2-inccoordnum}
 wird die Anzahl der Koordinaten in $\mathit{marker\_infoTWO}$ erhöht. Bevor die Kontur weiterverfolgt wird, wird in
 Zeile \ref{alg:argetcontour-2-ismaxchainreached-start}--\ref{alg:argetcontour-2-ismaxchainreached-end} überprüft, ob
 die Anzahl der Koordinaten $\mathit{AR\_CHAIN\_MAX} - 1 = 9999$ entspricht. Falls dem so ist, wird das Verfahren mit
 einem Fehlerwert in Zeile \ref{alg:argetcontour-2-error} abgebrochen. Dadurch wird sichergestellt, dass die Schleife
 in Zeile \ref{alg:argetcontour-2-contourloop-start}--\ref{alg:argetcontour-2-contourloop-end} auch bei einer nicht
 geschlossenen Kontur terminiert. Somit wird im schlechtesten Fall die Schleife $9997$-mal wiederholt, ohne das
 Verfahren mit einem Fehlerwert zu beenden. Die Laufzeitfunktion ist in \autoref{eq:analyse-argetcontour-contour},
 für $\mathit{AR\_CHAIN\_MAX} = n$, angegeben.
\begin{subequations}
\label{eq:analyse-argetconout-contour}
\begin{align}
\label{eq:analyse-argetconout-contour-1}
T_{worst}(n)& =
(c_{20}2 + c_{21}3 + c_{22}3 + c_{23})
\\
& \quad
+ (n-3)(c_{25}3 + c_{26}5 + c_{27}5 + c_{28}2 + c_{29}3)
\nonumber \\
& \quad
+ (n-3)(c_{30}8 + c_{31}35 + c_{32}14 + c_{33} + c_{35}21 + c_{36}7)
\nonumber \\
& \quad
+ (n-3)(c_{38} + c_{41}4 + c_{42}7 + c_{43}4 + c_{44}7 + c_{45}3)
+ c_{46}
\nonumber \\
& \quad
+ (n-3)(c_{48}4 + c_{49}3)
\nonumber \\
\label{eq:analyse-argetconout-contour-1}
T_{worst}(n)& =
+ (n-3)(c_{25}3 + c_{26}5 + c_{27}5 + c_{28}2 + c_{29}3)
\nonumber \\
& \quad
+ (n-3)(c_{30}8 + c_{31}35 + c_{32}14 + c_{33} + c_{35}21 + c_{36}7)
\nonumber \\
& \quad
+ (n-3)(c_{38} + c_{41}4 + c_{42}7 + c_{43}4 + c_{44}7 + c_{45}3)
\nonumber \\
& \quad
+ (n-3)(c_{48}4 + c_{49}3)
+ (c_{20}2 + c_{21}3 + c_{22}3 + c_{23} + c_{46})
\nonumber
\end{align}
\end{subequations}

Das Wachstum der Laufzeitfunktion ist $137n -401 = \Theta(n)$ für $c_{1} = 136$, $c_{2} = 137$ und $n_{0} = 401$.

In \autoref{alg:argetcontour-3} wird, wenn eine geschlossene Kontur vorliegt, die Koordinate mit dem größten Abstand
 zum Startpunkt der Region gesucht (Zeile
 \ref{alg:argetcontour-3-finddmax-start}--\ref{alg:argetcontour-3-finddmax-end}).
\begin{algorithm}[!ht]\small
\caption{\textproc{arGetContour} (Finde größten Abstand zu Punkt $(sx,sy)$)}
\label{alg:argetcontour-3}
\begin{algorithmic}[1]
	\algrestore{brk-argetcontour-contour}
	% dmax = 0;
	\State $\mathit{dmax} \gets 0$
	\Cost{$c_{53}$}{$1$}
	\label{alg:argetcontour-3-initdmax}
	% for(i=1;i<marker_infoTWO->coord_num;i++) {
	\For{$i \gets 1$ \textbf{to} $i < \mathit{marker\_infoTWO.coord\_num}$}
	\Cost{$c_{54}$}{$n - 2$}
	% \Cost{$c_{54}$}{$1$}
	\label{alg:argetcontour-3-finddmax-start}
	%     d = (marker_infoTWO->x_coord[i]-sx)*(marker_infoTWO->x_coord[i]-sx)
	%       + (marker_infoTWO->y_coord[i]-sy)*(marker_infoTWO->y_coord[i]-sy);
	%     if( d > dmax ) {
	%         dmax = d;
	%         v1 = i;
	%     }
	% }
		\State $a \gets (\mathit{marker\_infoTWO.x\_coord}\left[i\right] - sx)$
		\Cost{$c_{55}$}{$4(n-3)$}
		\label{alg:argetcontour-3-calcdistance-start}
		\State $b \gets (\mathit{marker\_infoTWO.y\_coord}\left[i\right] - sy)$
		\Cost{$c_{56}$}{$4(n-3)$}
		\State $d \gets (a \cdot a) + (b \cdot b)$
		\Cost{$c_{57}$}{$4(n-3)$}
		\label{alg:argetcontour-3-calcdistance-end}
		\If{$d > \mathit{dmax}$}
		\Cost{$c_{58}$}{$n-3$}
		\label{alg:argetcontour-3-isdmaxbigger}
			\State $\mathit{dmax} \gets d$
			\Cost{$c_{59}$}{$n-3$}
			\label{alg:argetcontour-3-savedmax}
			\State $\mathit{v1} \gets i$
			\Cost{$c_{60}$}{$n-3$}
		\EndIf
		\State $i \gets i + 1$
		\Cost{$c_{62}$}{$n-3$}
	\EndFor
	\label{alg:argetcontour-3-finddmax-end}
	% for(i=0;i<v1;i++) {
	%     arGetContour_wx[i] = marker_infoTWO->x_coord[i];
	%     arGetContour_wy[i] = marker_infoTWO->y_coord[i];
	% }
	\For{$i \gets 0$ \textbf{to} $i < \mathit{v1}$}
	\Cost{$c_{64}$}{$n-2$}
	\label{alg:argetcontour-3-dividev1-start}
		\State $\mathit{arGetContour\_wx}[i] \gets \mathit{marker\_infoTWO.x\_coord}\left[i\right]$
		\Cost{$c_{65}$}{$4(n-3)$}
		\State $\mathit{arGetContour\_wy}[i] \gets \mathit{marker\_infoTWO.y\_coord}\left[i\right]$
		\Cost{$c_{66}$}{$4(n-3)$}
		\State $i \gets i + 1$
		\Cost{$c_{67}$}{$(n-3)$}
	\EndFor
	\label{alg:argetcontour-3-dividev1-end}
	% for(i=v1;i<marker_infoTWO->coord_num;i++) {
	\For{$i \gets \mathit{v1}$ \textbf{to} $i < \mathit{marker\_infoTWO.coord\_num}$}
	\Cost{$c_{69}$}{$2$}
	\label{alg:argetcontour-3-dividecoordnum-start}
	%     marker_infoTWO->x_coord[i-v1] = marker_infoTWO->x_coord[i];
	%     marker_infoTWO->y_coord[i-v1] = marker_infoTWO->y_coord[i];
		\State $\mathit{marker\_infoTWO.x\_coord}\left[i - \mathit{v1}\right] \gets
		 \mathit{marker\_infoTWO.x\_coord}\left[i\right]$
		\Cost{$c_{70}$}{$6$}
		\State $\mathit{marker\_infoTWO.y\_coord}\left[i - \mathit{v1}\right] \gets
		 \mathit{marker\_infoTWO.y\_coord}\left[i\right]$
		\Cost{$c_{71}$}{$6$}
		\State $i \gets i + 1$
		\Cost{$c_{72}$}{$1$}
	% }
	\EndFor
	\label{alg:argetcontour-3-dividecoordnum-end}
	% for(i=0;i<v1;i++) {
	\For{$i \gets 0$ \textbf{to} $i < \mathit{v1}$}
	\Cost{$c_{74}$}{$(n-2)$}
	\label{alg:argetcontour-3-merge-start}
	%     marker_infoTWO->x_coord[i-v1+marker_infoTWO->coord_num] = arGetContour_wx[i];
	%     marker_infoTWO->y_coord[i-v1+marker_infoTWO->coord_num] = arGetContour_wy[i];
		\State $num \gets \mathit{marker\_infoTWO.coord\_num}$
		\Cost{$c_{75}$}{$2(n-3)$}
		\State $\mathit{marker\_infoTWO.x\_coord}\left[i - \mathit{v1} + num\right] \gets
		 \mathit{arGetContour\_wx}[i]$
		\Cost{$c_{76}$}{$6(n-3)$}
		\State $\mathit{marker\_infoTWO.y\_coord}\left[i - \mathit{v1} + num\right] \gets
		 \mathit{arGetContour\_wy}[i]$
		\Cost{$c_{77}$}{$6(n-3)$}
		\State $i \gets i + 1$
		\Cost{$c_{78}$}{$(n-3)$}
	% }
	\EndFor
	\label{alg:argetcontour-3-merge-end}
	% marker_infoTWO->x_coord[marker_infoTWO->coord_num] = marker_infoTWO->x_coord[0];
	% marker_infoTWO->y_coord[marker_infoTWO->coord_num] = marker_infoTWO->y_coord[0];
	% marker_infoTWO->coord_num++;
	% 
	% return 0;
	\State $num \gets \mathit{marker\_infoTWO.coord\_num}$
	\Cost{$c_{80}$}{$2$}
	\label{alg:argetcontour-3-savev1-start}
	\State $\mathit{marker\_infoTWO.x\_coord}[num] \gets \mathit{marker\_infoTWO.x\_coord}[0]$
	\Cost{$c_{81}$}{$5$}
	\State $\mathit{marker\_infoTWO.y\_coord}[num] \gets \mathit{marker\_infoTWO.y\_coord}[0]$
	\Cost{$c_{82}$}{$5$}
	\label{alg:argetcontour-3-savev1-end}
	\State $\mathit{marker\_infoTWO.coord\_num} \gets \mathit{marker\_infoTWO.coord\_num} + 1$
	\Cost{$c_{83}$}{$4$}
	\label{alg:argetcontour-3-inccoordnum}
	\State \textbf{return} $0$
	\Cost{$c_{84}$}{$1$}
\end{algorithmic}
\end{algorithm}

Dazu wird in Zeile \ref{alg:argetcontour-3-initdmax} die Variable $\mathit{dmax}$ initialisiert. Danach wird in Zeile
 \ref{alg:argetcontour-3-calcdistance-start}--\ref{alg:argetcontour-3-calcdistance-end} der Abstand zwischen dem
 aktuellen Punkt $i$ und der Koordinaten $\mathit{sx}$ und $\mathit{sy}$ berechnet. Falls der Abstandswert größer als
 $\mathit{dmax}$ ist (Zeile \ref{alg:argetcontour-3-isdmaxbigger}), wird der Abstandswert in $\mathit{dmax}$
 gespeichert (Zeile \ref{alg:argetcontour-3-savedmax}) und die Position $i$ in $\mathit{v1}$ hinterlegt. Nachdem die
 Schleife in Zeile \ref{alg:argetcontour-3-calcdistance-start}--\ref{alg:argetcontour-3-calcdistance-end} beendet ist,
 ist der größte Abstand in $\mathit{dmax}$ gespeichert und die Position der Koordinate in $\mathit{v1}$ gespeichert.
 Nun wird in Zeile \ref{alg:argetcontour-3-dividev1-start}--\ref{alg:argetcontour-3-dividev1-end} von der ersten
 Position bis zur Position $\mathit{v1}$ alle Koordinaten in einer temporären Liste gespeichert. In Zeile
 \ref{alg:argetcontour-3-dividecoordnum-start}--\ref{alg:argetcontour-3-dividecoordnum-end} werden alle Koordinaten ab
 Position $\mathit{v1}$ an den Anfang von $\mathit{marker\_infoTWO}$ verschoben. Die temporär gespeicherten Koordinaten
 werden in Zeile \ref{alg:argetcontour-3-merge-start}--\ref{alg:argetcontour-3-merge-end} an das Ende von
 $\mathit{marker\_infoTWO}$ angehängt. Zum Schluss wird in Zeile
 \ref{alg:argetcontour-3-savev1-start}--\ref{alg:argetcontour-3-savev1-end} die Koordinaten des Punktes mit dem größten
 Abstandswert an die letzte Stelle von marker\_infoTWO kopiert, sodass die Koordinaten am Anfang und am Ende
 gespeichert sind. Die Anzahl der gespeicherten Koordinaten wird in Zeile \ref{alg:argetcontour-3-inccoordnum} erhöht
 und die Konturverfolgung beendet. $\mathit{marker\_infoTWO.coord\_num}$ ist von $\mathit{AR\_CHAIN\_MAX}$ abhängig und
 kann maximal $\mathit{AR\_CHAIN\_MAX} - 3$ Einträge enthalten. Im schlechtesten Fall liegt der Punkt mit dem größten
 Abstand am Ende der Liste von $\mathit{marker\_infoTWO.coord\_num}$. Für diesen Fall sind die Kosten des Verfahrens
 in \autoref{alg:argetcontour-3} aufgelistet. Für die Kosten der Laufzeitfunktion in
 \autoref{eq:analyse-argetcontour-distance} ist $\mathit{AR\_CHAIN\_MAX} = n$.
\begin{subequations}
\label{eq:analyse-argetcontour-distance}
\begin{align}
\label{eq:analyse-argetcontour-distance-1}
T_{worst}(n)& =
c_{53}
 + c_{54}(n-2)
 + 4(c_{55} + c_{56} + c_{57})(n-3)
\\
& \quad
 + (c_{58} + c_{59} + c_{60} + c_{62})(n-3)
 + c_{64}(n-2)
 + 4(c_{65} + c_{66})(n-3)
\nonumber \\
& \quad
 + c_{67}(n-3)
 + (c_{69}2 + c_{70}6 + c_{71}6 + c_{72})
 + c_{74}(n-2)
\nonumber \\
& \quad
 + (c_{75}2 + c_{76}6 + c_{77}6 + c_{78})(n-3)
 + (c_{80}2 + c_{81}5 + c_{82}5 + c_{83}4 + c_{84})
\nonumber \\
\label{eq:analyse-argetcontour-distance-2}
T_{worst}(n)& =
n(c_{54} + c_{64} + c_{74} + c_{75}8 + c_{76}24 + c_{77}24 + c_{78}4
\\
& \quad \quad
 + c_{58} + c_{59} + c_{60} + c_{62} + c_{67})
\nonumber \\
& \quad
- 2(c_{54} + c_{64} + c_{74})
- (c_{75}24 + c_{76}72 + c_{77}72 + c_{78}12)
\nonumber \\
& \quad
- (c_{58}3 + c_{59}3 + c_{60}3 + c_{62}3 + c_{67}3)
+ 4(c_{55} + c_{56} + c_{57} + c_{65} + c_{66})
\nonumber \\
& \quad
+ (c_{53} + c_{69}2 + c_{70}6 + c_{71}6 + c_{72} + c_{80}2 + c_{81}5 + c_{82}5 + c_{83}4 + c_{84})
\nonumber
\end{align}
\end{subequations}

Für $c_{1} = 67$, $c_{2} = 68$ und $n_{0} = 163$ ist $68n - 163 = \Theta(n)$.

Die Laufzeitfunktion der Methode \textproc{arGetContour}, bestehend aus \autoref{alg:argetcontour-1},
 \autoref{alg:argetcontour-2} und \autoref{alg:argetcontour-3}, ist in \autoref{eq:analyse-argetcontour-all} angegeben.
 Die dominante Eingabegröße des Verfahrens ist $\mathit{AR\_CHAIN\_MAX} = n$ (Vgl. \autoref{alg:argetcontour-2},
 Zeile \ref{alg:argetcontour-2-contourloop-start}--\ref{alg:argetcontour-2-contourloop-end}). Die Variable
 $\mathit{xsize}$ kann durch die Breite des Bildsignals ersetzt werden und entspricht $\mathit{xsize} = \tfrac{640}{2}$.
\begin{align}
\label{eq:analyse-argetcontour-all}
T_{worst}(n)& =
8(320) - 8 + 137n - 401 + 68n - 163
\\
& =
205n - 3132
\nonumber
\end{align}

Die Wachstumsrate der Funktion aus \autoref{eq:analyse-argetcontour-all} ist im schlechtesten Fall
 $205n - 3132 = \Theta(n)$, für $c_{1} = 204$, $c_{2} = 205$ und $n_{0} = 3132$.
% subsubsection konturerzeugung (end)


\clearpage

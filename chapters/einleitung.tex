\chapter{Einleitung} % (fold)
\label{cha:einleitung}

Heutzutage können wir zwei Phänomene beobachten:\\Das erste Phänomen ist die Zunahme von \gls{AR} Anwendungen außerhalb
 der Forschung. \gls{AR} verspricht neue Bedienkonzepte und Informationsdarstellung durch Überlagerung von realen und
 virtuellen Bil\-dern. Die dazu benötigte Ausrüstung bestand viele Jahre aus einem so genannten Head-mounted Display
 und einem Computer in einem Rucksack.

Das zweite Phänomen ist die weite Verbreitung von Smartphones und deren ge\-sell\-schaft\-liche Akzeptanz. Dabei wird
 die Leistungsfähigkeit von Smartphones stetig ge\-stei\-gert und Anwendungen sind als Folge nicht mehr auf den
 Computer beschränkt. Gerade Spiele profitieren von der Prozessor\-ge\-schwin\-dig\-keit und den heute üblichen
 Grafikprozessoren moderner Smartphones. Auch die mobile Nutzung von E-Mail und Internet hat durch Smartphones
 zugenommen. Aus diesen Gründen bietet es sich an, die Mög\-lich\-kei\-ten der \gls{AR} auf Smartphones zu untersuchen.

\section{Forschungsstand} % (fold)
\label{sec:forschungsstand}
\begin{comment}
	Forschungsstand: Alle untersuchten Arbeiten aufführen und kurz erklären.
\end{comment}

ARToolKit\footcite{artoolkit} wurde 1999 von Kato entwickelt und als Open Source Software einer breiten
 Entwicklergmeinde zur Verfügung gestellt. ARToolKit galt als Referenz der Forschung im Bereich der \gls{AR}.
 Allerdings war der Systementwurf nicht auf das Aufkommen von mobilen Computern und Smartphones vorbereitet.

Im Jahre 2004 wurde von \citeauthor{wagner04} ARToolKit für die Windows CE Plattform portiert\footcite{wagner04}. Diese
 Portierung führt zur Entwicklung von ARToolKitPlus\footcite{artoolkitplus}, das nicht nur für Workstations, sondern
 auch für mobile Geräte, entworfen wurde\footcite{wagner07b}. \citeauthor{wagner07a} machte diese Entwicklung zum Thema
 seiner Dissertation\footcite{wagner07a}. Die Schwächen von ARToolKit wurden überarbeitet und das System auf den
 aktuellen Stand der Forschung gebracht. Neue Verfahren zur robusten Mar\-ken\-er\-kennung in ARToolKitPlus waren aber
 nicht in der Lage mit der begrenzten Prozessorgeschwindigkeit mobiler Geräte ausgeführt zu werden.

Die Erfahrung von \citeauthor{wagner04} bei der Entwicklung von ARToolKitPlus wurde genutzt um
 Studierstube\footcite{studierstube} zu entwickeln. Studierstube ist das erste System, dass auf die Bedürfnisse von
 Smartphones und mobilen Geräten konzipiert wurde und von Grund auf neu entwickelt wurde. Anders als ARToolKit und
 ARToolKitPlus ist Studierstube nicht im Quellcode verfügbar.
\citeauthor{wagner09a} erläutern in \citetitle{wagner09a}\footcite{wagner09a} und
 \citetitle{wagner09b}\footcite{wagner09b} einen Überblick über die Strategien und Entwicklung von Studierstube.

In \citetitle{hirzer08}\footcite{hirzer08} wird von \citeauthor{hirzer08} ein Verfahren zur robosten Markenerkennung
 vorgestellt und basiert auf der Arbeit\footcite{clarke96} von \citeauthor{clarke96}. Die robuste Erkennung von Linien
 und die hohe Fehlertoleranz gegenüber schwankenden Lichtverhältnissen zeichnen dieses Verfahren aus um als Grundlage
 zur Markenerkennung verwendet zu werden.

% section forschungsstand (end)
\section{Definition} % (fold)
\label{sec:definition}
\begin{comment}
	Definiere Begriffe der Augmented Reality und Bildverarbeitung, die dem Leser nicht geläufig sind. Denke dabei an Prof. Klocke als Leser ohne besonderen Kenntnisstand in der AR/Bildverarbeitung.

	Azuma Definition der AR
	Milgram + Kishino MR oder nur Milgram Definition MR
\end{comment}

Bis heute ist \gls{AR} nicht eindeutig definiert, obwohl die ersten Untersuchung von \citeauthor{sutherland} bereits
 1968 vorgestellt wurden\footcite{sutherland}. \citeauthor{milgram94b} beschrieben in
 \citetitle{milgram94b}\footcite{milgram94b} \gls{AR} als einen Punkt des Reality-Virtuality (RV) Continuum
 (Vgl. \autoref{fig:mixedreality}), welches reale und virtuelle Objekte gemeinsam auf einem Bildschirm darstellt.
\begin{figure}[!ht]
	\centering
	\input{resources/mixedreality.pdf_tex}
	\caption{Reality-Virtual Continuum nach \citeauthor{milgram94b}}
	\label{fig:mixedreality}
\end{figure}

Nach der neueren Definition von \citeauthor{azuma97}\footcite{azuma97} muss ein \gls{AR}-System drei Kriterien
 erfüllen:
\begin{itemize}
	\item Reale und virtuelle Objekte werden kombiniert
	\item Bezug von realen und virtuellen Objekten im 3-dimensionalen Raum
	\item Interaktion in Echtzeit
\end{itemize}
Das erste Kriterium bedeutet, dass, im Gegensatz zu Computerspielen, virtuelle Gegenstände mit der realen Welt in
 Verbindung stehen. Durch die zweite Bedingung muss ein virtuelles Objekt in einem räumlichen Zusammenhang stehen. Als
 Beispiel muss eine virtuelle Tasse auf dem realen Schreibtisch platziert sein und nicht im Raum schweben. Zuletzt muss
 die Interaktion mit einem \gls{AR}-System in Echtzeit erfolgen. Dies ist eine Abgrenzung zu einer Computeranimation in
 einem Film.

Nach \citeauthor{moeller2008}\footcite[Vgl.][S.~1]{moeller2008} ist Interaktion in Echtzeit mit $15$ FPS erfüllt, wobei
 eine Geschwindigkeit von $6$ FPS noch als interaktiv gilt. Diese Definition wird von \citeauthor{wagner09b} in ihrer
 Untersuchung\footcite[Vgl.][S.~8--9]{wagner09b} unterstützt.

\subsection{Tracking Verfahren} % (fold)
\label{sec:tracking_verfahren}
Tracking Verfahren bezeichnet bei \gls{AR}-Systemen die Fähigkeit, die reale Position eines Benutzers oder eines
 Displays in Relation zur Position einer \gls{AR}-Umgebung zu setzen. Tracking Verfahren müssen, wie in der Definition
 von \citeauthor{azuma97} schon erwähnt, in Echtzeit agieren. Zusätzlich müssen Tracking Verfahren, je nach
 Anwendungsgebiet, unter unterschiedlichen Lichtverhältnissen arbeiten können und eine Marke zuverlässig erkennen
 können.

Ein Bereich der Tracking Verfahren ist das Fiducial Tracking, auch Markenbasiertes Tracking genannt. Bei diesem
 Tracking Verfahren wird durch Hilfe einer Marke die Position im realen und virtuellen Raum berechnet. Die bereits
 erwähnten Systeme ARToolKit, ARToolKitPlus und Studierstube, sind Markenbasierte Verfahren und suchen in einem
 Bildsignal nach einer bitonalen Marke. Das erste Markenbasierte \gls{AR}-System war
 Matrix\footcite{rekimoto1998matrix} von \citeauthor{rekimoto1998matrix}. \citeauthor{fiala2004artaga} entwickelte mit
 ARTag\footcite{fiala2004artaga} ein \gls{AR}-System, dass die Idee eines 2D-Codes zur Identifizierung von Matrix
 übernahm. In ARToolKitPlus wurde die Identifizierung von 2D-Codes ebenfalls implementiert.
% subsection tracking_verfahren (end)

\subsection{Bitonale Marken} % (fold)
\label{sub:bitonalemarken}
Bei Markenbasierten \gls{AR}-Verfahren werden bitonale Marken zur Erkennung und Verfolgung von realen Objekten
 eingesetzt. Eine Marke besteht aus einem schwarzen quadratischen Rahmen auf weißem Untergrund und trägt in der Mitte
 Informationen zur Identifizierung. ARToolKit verwendet simple Muster oder Schriftzeichen zur Identifizierung, wie in
 \autoref{fig:marke-artoolkit} gezeigt. ARTag hingegen verwendet eine binär kodierte Zahl als Identifizierungsmerkmal,
 die als Anordnung von $6 \times 6$ Pixeln dargestellt wird (s.~\autoref{fig:marke-artag}). ARToolKitPlus, der
 Nachfolger von ARToolKit nutzt ein ähnliches Verfahren wie ARTag\footcite[Vgl.][S.~142]{wagner07b} und unterscheidet
 sich in der Rahmenbreite (s.~\autoref{fig:marke-artoolkitplus}). Wenn in den folgenden Kapiteln von einer bitonalen
 Marke, oder einfach Marke, gesprochen wird, beziehe ich mich auf einen schwarzen quadratischen Rahmen auf weißem
 Untergrund.

\begin{figure}[!ht]
	\centering
	\subfigure[ARToolKit]{
		\label{fig:marke-artoolkit}
		\includegraphics[scale=1]{resources/Marker-ART.pdf}
	}
	\subfigure[ARTag]{
		\label{fig:marke-artag}
		\includegraphics[scale=1]{resources/Marker-ARTag.pdf}
	}
	\subfigure[ARToolKitPlus]{
		\label{fig:marke-artoolkitplus}
		\includegraphics[scale=1]{resources/Marker-ART+.pdf}
	}
	\caption{Verschiedene bitonale Marken.
	}
	\label{fig:bitonale-marken}
\end{figure}
% subsection bitonale_marken (end)

\begin{comment}
	(wagner/schmalstieg ARToolKitPlus fpr Pose Tracking on Mobile Devices S.4)

	Beim Fiducial Marker Tracking werden künstliche Marken zur Erkennung und Verfolgung von realen Objekten eingesetzt. Häufig werden für diese Marken quadratische schwarze Rahmen verwendet die innerhalb des Rahmens Schriftzeichen, Bilder oder 2D Codes enthalten. Diese Marken sind einfach herzustellen und können mit geringem Aufwand an Objekte angebracht werden.

	Diese bitonale Marken haben den Vorteil, dass sie Robust gegen Helligkeitsveränderung sind und die Entscheidung eines Pixels auf eine Schwellwert-Entscheidung reduziert werden kann. Marken für \gls{AR} müssen in einem großen Blickfeld erkannt werden können, was wiederum bei industriellen Anwendung nicht der Fall ist, da hier Marken den größten Teil des Bildes einnehmen können.
\end{comment}

% section definition (end)
\section{Problemstellung} % (fold)
\label{sec:problemstellung}
\begin{comment}
	Problemstellung: Problemstellung und Frage im Detail erläutern
\end{comment}

Wie in Kapitel \ref{sec:forschungsstand} schon erwähnt, gibt es keine aktuellen und im Quellcode verfügbaren Systeme zur Entwicklung von \gls{AR} Anwendungen. Somit war eine Eigenentwicklung unumgänglich.

Um eine bitonale Marke, üblicherweise Schwarz/Weiß, zu erkennen, werden Verfahren zur Erkennung von Linien eingesetzt, die aus dem Bereich der digitalen Bildverarbeitung stammen. Anders als bei der herkömmlichen Bildverarbeitung muss Bildverarbeitung für \gls{AR} die Verarbeitungsschritte so schnell ausführen, dass eine Analyse eines Bildes in Echtzeit stattfinden kann. Bei Bildverarbeitung ausserhalb der \gls{AR} ist dies kein notwendiges Kriterium. Als Grundlage für eine Eigenentwicklung kommt das Verfahren von \citeauthor{clarke96}\footcite{clarke96} zum Einsatz, dass in Kapitel \ref{sub:verfahren_nach_hirzer} vorgestellt wird.

Der Bezug von Bildern über den Kamerasensor eines Smartphones muss so schnell wie möglich erfolgen um für die Weiterverarbeitung verwendet werden zu können. Bei Smartphones werden in aller Regel Kamerasensoren aus dem Verbraucherbereich verwendet. Diese Kamerasysteme liefern nicht die Performanz wie professionelle Kameras. Daraus bedingt müssen die zur Verfügung stehenden Mittel so gut es geht ausgenutzt werden. Beim Bezug der Bilder ist darauf zu achten, dass die Geschwindigkeit bei Bildtypenkonvertierungen nicht darunter leidet.

Die Softwarearchitektur für \gls{AR} auf Smartphones unterscheidet sich von einer PC Architektur, abgesehen von der Prozessorgeschwindigkeit, grundsätzlich durch die Menge und Bandbreite des Arbeitsspeichers. Dies bedeutet, dass alle Arbeitsschritte durch eine ineffektive Ausnutzung des zur Verfügung stehenden Speichers verlangsamt werden.

Die Software kann nach gängigen Entwurfskriterien entworfen werden wobei die Vor- und Nachteile von dynamischen und statischen Bibliotheken berücksichtigt werden müs\-sen.

% section problemstellung (end)

% chapter einleitung (end)
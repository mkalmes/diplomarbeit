\chapter{Einleitung} % (fold)
\label{cha:einleitung}

Heutzutage können wir zwei Phänomene beobachten: Das erste Phänomen ist die Zunahme von \gls{AR} Anwendungen außerhalb
 der Forschung. \gls{AR} verspricht neue Bedienkonzepte und Informationsdarstellung durch Überlagerung von realen und
 virtuellen Bil\-dern. Die dazu benötigte Ausrüstung bestand viele Jahre aus einem so genannten Head-mounted Display
 und einem Computer in einem Rucksack\footcite{azuma01}.

Das zweite Phänomen ist die weite Verbreitung von Smartphones und deren ge\-sell\-schaft\-liche Akzeptanz. Dabei wird
 die Leistungsfähigkeit von Smartphones stetig ge\-stei\-gert und Anwendungen sind als Folge nicht mehr auf den
 Computer beschränkt. Gerade Spiele profitieren von der Prozessor\-ge\-schwin\-dig\-keit und den heute üblichen
 Grafikprozessoren moderner Smartphones. Auch die mobile Nutzung von E-Mail und Internet hat durch Smartphones
 zugenommen. Aus diesen Gründen bietet es sich an, die Mög\-lich\-kei\-ten der \gls{AR} auf Smartphones zu untersuchen.

\section{Forschungsstand} % (fold)
\label{sec:forschungsstand}
\begin{comment}
	Forschungsstand: Alle untersuchten Arbeiten aufführen und kurz erklären.
\end{comment}

ARToolKit\footcite{artoolkit} wurde 1999 von Kato entwickelt und war das erste \gls{AR}-System, dass als Open Source einer breiten Entwicklergemeinde zur Verfügung stand. ARToolKit galt als die Referenz für Forschung im Bereich der \gls{AR}. Allerdings war der Systementwurf nicht auf das Aufkommen von mobilen Computern und Smartphones vorbereitet, was dafür sorgte, dass das System in dieser Form für weitere Forschung uninteressant wurde.

ARToolKitPlus\footcite{artoolkitplus} ist eine Optimierung von ARToolKit und wurde für mobile Geräte angepasst. Die Schwächen von ARToolKit wurden entfernt und das System auf den neusten Stand der Forschung aktualisiert. Besondere Eigenschaften zur robusten Mar\-ken\-er\-kennung waren aber nicht in der Lage auf limitierten mobilen Geräten ausgeführt zu werden.

Die Erfahrung aus der Entwicklung von ARToolKitPlus wurde genutzt um Studierstube\footcite{studierstube} zu entwickeln. Studierstube ist das erste System, dass auf die Bedürfnisse von Smartphones und mobilen Geräten konzipiert wurde und von Grund auf neu entwickelt wurde. Studierstube ist nicht wie ARToolKit und ARToolKitPlus im Quellcode verfügbar.

Wagner und Schmalstieg waren an der Weiterentwicklung von ARToolKitPlus beteiligt und haben aus dieser Erfahrung heraus Studierstube entwickelt. Mit ihrer Ver\-öf\-fent\-li\-chung von \citetitle{wagner03}\footcite{wagner03} wird die frühe Entwicklung von mobilen \gls{AR}-Systemen beleuchtet. In \citetitle{wagner09a}\footcite{wagner09a} und \citetitle{wagner09b}\footcite{wagner09b} geben sie einen Einblick in ihre Arbeit, die dieser Untersuchung zu Grunde liegt.

In \citetitle{clarke96}\footcite{clarke96} wird ein Verfahren zur schnellen Linienerkennung vorgestellt. Die robuste Erkennung von Linien und die hohe Fehlertoleranz gegenüber schwankenden Lichtverhältnissen zeichnen dieses Verfahren aus um als Grundlage zur Markenerkennung verwendet zu werden.

% section forschungsstand (end)
\section{Definition} % (fold)
\label{sec:definition}
\begin{comment}
	Definiere Begriffe der Augmented Reality und Bildverarbeitung, die dem Leser nicht geläufig sind. Denke dabei an Prof. Klocke als Leser ohne besonderen Kenntnisstand in der AR/Bildverarbeitung.

	Azuma Definition der AR
	Milgram + Kishino MR oder nur Milgram Definition MR
\end{comment}

Bis heute ist \gls{AR} nicht eindeutig definiert, obwohl die ersten Untersuchung von \citeauthor{sutherland} bereits
 1968 vorgestellt wurden\footcite{sutherland}. \citeauthor{milgram94b} beschrieben in
 \citetitle{milgram94b}\footcite{milgram94b} \gls{AR} als einen Punkt des Reality-Virtuality (RV) Continuum
 (Vgl. \autoref{fig:mixedreality}), welches reale und virtuelle Objekte gemeinsam auf einem Bildschirm darstellt.
\begin{figure}[!ht]
	\centering
	\input{resources/mixedreality.pdf_tex}
	\caption{Reality-Virtual Continuum nach \citeauthor{milgram94b}}
	\label{fig:mixedreality}
\end{figure}

Nach der neueren Definition von \citeauthor{azuma97}\footcite{azuma97} muss ein \gls{AR}-System drei Kriterien
 erfüllen:
\begin{itemize}
	\item Reale und virtuelle Objekte werden kombiniert
	\item Bezug von realen und virtuellen Objekten im 3-dimensionalen Raum
	\item Interaktion in Echtzeit
\end{itemize}
Das erste Kriterium bedeutet, dass, im Gegensatz zu Computerspielen, virtuelle Gegenstände mit der realen Welt in
 Verbindung stehen. Durch die zweite Bedingung muss ein virtuelles Objekt in einem räumlichen Zusammenhang stehen. Als
 Beispiel muss eine virtuelle Tasse auf dem realen Schreibtisch platziert sein und nicht im Raum schweben. Zuletzt muss
 die Interaktion mit einem \gls{AR}-System in Echtzeit erfolgen. Dies ist eine Abgrenzung zu einer Computeranimation in
 einem Film.

Nach \citeauthor{moeller2008}\footcite[Vgl.][S.~1]{moeller2008} ist Interaktion in Echtzeit mit $15$ FPS erfüllt, wobei
 eine Geschwindigkeit von $6$ FPS noch als interaktiv gilt. Diese Definition wird von \citeauthor{wagner09b} in ihrer
 Untersuchung\footcite[Vgl.][S.~8--9]{wagner09b} unterstützt.

\subsection{Tracking-Verfahren} % (fold)
\label{sec:tracking_verfahren}
Tracking-Verfahren bezeichnet bei \gls{AR}-Systemen die Fähigkeit, die reale Position eines Benutzers oder eines
 Displays in Relation zur Position einer \gls{AR}-Umgebung zu setzen. Tracking-Verfahren müssen, wie in der Definition
 von \citeauthor{azuma97} schon erwähnt, in Echtzeit agieren. Zusätzlich müssen Tracking-Verfahren, je nach
 Anwendungsgebiet, unter unterschiedlichen Lichtverhältnissen arbeiten können und eine Marke zuverlässig erkennen
 können.

Ein Bereich der Tracking-Verfahren ist das Fiducial-Tracking, auch Marken-basiertes Tracking genannt. Bei diesem
 Tracking-Verfahren wird durch Hilfe einer Marke die Position im realen und virtuellen Raum berechnet. Die bereits
 erwähnten Systeme ARToolKit, ARToolKitPlus und Studierstube, sind Marken-basierte Verfahren und suchen in einem
 Bildsignal nach einer bitonalen Marke. Das erste Marken-basierte \gls{AR}-System war
 Matrix\footcite{rekimoto1998matrix} von \citeauthor{rekimoto1998matrix}. \citeauthor{fiala2004artaga} entwickelte mit
 ARTag\footcite{fiala2004artaga} ein \gls{AR}-System, dass die Idee eines 2D-Codes zur Identifizierung von Matrix
 übernahm. In ARToolKitPlus wurde die Identifizierung von 2D-Codes ebenfalls implementiert.
% subsection tracking_verfahren (end)

\subsection{Bitonale Marken} % (fold)
\label{sub:bitonalemarken}
Bei Marken-basierten \gls{AR}-Verfahren werden bitonale Marken zur Erkennung und Verfolgung von realen Objekten
 eingesetzt. Eine Marke besteht aus einem schwarzen quadratischen Rahmen auf weißem Untergrund und trägt in der Mitte
 Informationen zur Identifizierung. ARToolKit verwendet simple Muster oder Schriftzeichen zur Identifizierung, wie in
 \autoref{fig:marke-artoolkit} gezeigt. ARTag hingegen verwendet eine binär kodierte Zahl als Identifizierungsmerkmal,
 die als Anordnung von $6 \times 6$ Pixeln dargestellt wird (s.~\autoref{fig:marke-artag}). ARToolKitPlus, der
 Nachfolger von ARToolKit nutzt ein ähnliches Verfahren wie ARTag\footcite[Vgl.][S.~142]{wagner07b} und unterscheidet
 sich in der Rahmenbreite (s.~\autoref{fig:marke-artoolkitplus}). Wenn in den folgenden Kapiteln von einer bitonalen
 Marke, oder einfach Marke, gesprochen wird, beziehe ich mich auf einen schwarzen quadratischen Rahmen auf weißem
 Untergrund.

\begin{figure}[!ht]
	\centering
	\subfigure[ARToolKit]{
		\label{fig:marke-artoolkit}
		\includegraphics[scale=1]{resources/Marker-ART.pdf}
	}
	\subfigure[ARTag]{
		\label{fig:marke-artag}
		\includegraphics[scale=1]{resources/Marker-ARTag.pdf}
	}
	\subfigure[ARToolKitPlus]{
		\label{fig:marke-artoolkitplus}
		\includegraphics[scale=1]{resources/Marker-ART+.pdf}
	}
	\caption{Verschiedene bitonale Marken.
	}
	\label{fig:bitonale-marken}
\end{figure}
% subsection bitonale_marken (end)

\begin{comment}
	(wagner/schmalstieg ARToolKitPlus fpr Pose Tracking on Mobile Devices S.4)

	Beim Fiducial Marker Tracking werden künstliche Marken zur Erkennung und Verfolgung von realen Objekten eingesetzt. Häufig werden für diese Marken quadratische schwarze Rahmen verwendet die innerhalb des Rahmens Schriftzeichen, Bilder oder 2D Codes enthalten. Diese Marken sind einfach herzustellen und können mit geringem Aufwand an Objekte angebracht werden.

	Diese bitonale Marken haben den Vorteil, dass sie Robust gegen Helligkeitsveränderung sind und die Entscheidung eines Pixels auf eine Schwellwert-Entscheidung reduziert werden kann. Marken für \gls{AR} müssen in einem großen Blickfeld erkannt werden können, was wiederum bei industriellen Anwendung nicht der Fall ist, da hier Marken den größten Teil des Bildes einnehmen können.
\end{comment}

% section definition (end)
\section{Problemstellung} % (fold)
\label{sec:problemstellung}
\begin{comment}
	Problemstellung: Problemstellung und Frage im Detail erläutern
\end{comment}

Die Erforschung von \gls{AR} Verfahren und Anwendungen wurde in der Vergangenheit auf leistungsfähigen Computern
 durchgeführt. Die Untersuchung der Verfahren auf mobilen Geräten ist ein noch neuer Bereich, der durch die Limitierung
 der Hardware eine Herausforderung darstellt.

Um eine bitonale Marke zu erkennen, werden Verfahren zur Erkennung von Linien eingesetzt, die aus dem Bereich der
 digitalen Bildverarbeitung stammen. Anders als bei der herkömmlichen Bildverarbeitung muss Bildverarbeitung für
 \gls{AR} die Verarbeitungsschritte so schnell ausführen, dass eine Analyse eines Bildes in Echtzeit stattfinden kann.
 Bei Bildverarbeitung außerhalb der \gls{AR} ist dies kein notwendiges Kriterium.

Der Bilddatenbeschaffung über den Kamerasensor eines Smartphones muss so schnell wie möglich erfolgen um für die
 Weiterverarbeitung verwendet werden zu können. Bei Smartphones werden in aller Regel Kamerasensoren aus dem
 Verbraucherbereich verwendet. Diese Kamerasysteme liefern nicht die Performanz einer professionellen Kamera. Dadurch
 bedingt müssen die zur Verfügung stehenden Mittel so gut es geht ausgenutzt werden.

Die Softwarearchitektur für \gls{AR} auf Smartphones unterscheidet sich von einer PC Architektur, abgesehen von der
 Prozessorgeschwindigkeit, grundsätzlich durch die Menge und Bandbreite des Arbeitsspeichers. Dies bedeutet, dass alle
 Arbeitsschritte durch eine ineffektive Ausnutzung des zur Verfügung stehenden Speichers verlangsamt werden.

Wie in \autoref{sec:forschungsstand} erwähnt, ist ARToolKitPlus das einzige aktuelle Tracking Verfahren, das im
 Quellcode verfügbar ist. Wie \citeauthor{wagner09a} festgestellt haben, eignet sich ARToolKitPlus nur noch bedingt für
 mobile Plattformen\footcite{wagner09a}, sodass sie mit Studierstube ein neues Tracking Verfahren entworfen haben. Ein
 modernes Verfahren zur Markenerkennung wurde von \citeauthor{hirzer08} vorgestellt\footcite{hirzer08}, dass im Rahmen
 dieser Arbeit implementiert wurde.

Diese Arbeit vergleicht die unterschiedlichen Tracking Verfahren von ARToolKitPlus und dem Verfahren von
 \citeauthor{hirzer08} zur Erkennung einer bitonalen Marke auf einem iPod touch (4. Generation) unter iOS 4.

Der Fokus dieser Arbeit liegt dabei auf die Einhaltung der Echtzeitbedinung von \citeauthor{azuma97}.


% section problemstellung (end)

% chapter einleitung (end)
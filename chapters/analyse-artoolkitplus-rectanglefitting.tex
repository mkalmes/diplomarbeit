In der Rectangle Fitting Phase wird überprüft, ob eine Kontur ein Rechteck ist. Dazu wird die Methode
 \textproc{checkSquare} verwendet (\autoref{alg:checksquare-1}--\autoref{alg:checksquare-7}). Doch zuerst müssen die
 Schritte zum Aufruf der Methode in \textproc{arDetectMarker2} erläutert werden
 (\autoref{alg:detectmarker2-3}).
\begin{algorithm}[ht]
\caption{\textproc{arDetectMarker2} (Rechtecküberprüfung)}
\label{alg:detectmarker2-3}
\begin{algorithmic}[1]
	\State $\mathit{marker\_num2} \gets 0$
	\For{$i \gets 0$ \textbf{to} $i < \mathit{label\_num}$}
	\label{alg:detectmarker2-3-loop-start}
		\State \ldots
		\State $\mathit{ret} \gets$ \textproc{arGetContour}$\left(
		\begin{aligned}
			& \mathit{limage}, \mathit{label\_ref}, i + 1,\\
			& \mathit{wclip}[i \cdot 4], \mathit{marker\_infoTWO}[\mathit{marker\_num2}]
		\end{aligned}\right)$
		\If{$\mathit{ret} < 0$}
		\label{alg:detectmarker2-3-retvalue}
			\State \textbf{continue}
			\label{alg:detectmarker2-3-continue}
		\EndIf
		\State $\mathit{ret} \gets$ \Call{checkSquare}{$\mathit{warea}[i],\mathit{marker\_infoTWO}[\mathit{marker\_num2}], \mathit{factor}$}
		\label{alg:detectmarker2-3-checksquare}
		\If{$\mathit{ret} < 0$}
			\State \textbf{continue}
			\label{alg:detectmarker2-3-continue2}
		\EndIf
		\State $\mathit{marker\_infoTWO}[marker\_num2].area \gets \mathit{warea}[i]$
		\label{alg:detectmarker2-3-save-start}
		\State $\mathit{marker\_infoTWO}[marker\_num2].pos[0] \gets \mathit{wpos}[i \cdot 2 + 0]$
		\State $\mathit{marker\_infoTWO}[marker\_num2].pos[1] \gets \mathit{wpos}[i \cdot 2 + 1]$
		\State $\mathit{marker\_num2} \gets \mathit{marker\_num2} + 1$
		\label{alg:detectmarker2-3-save-end}
		\If{$\mathit{marker\_num2} == \mathit{MAX\_IMAGE\_PATTERNS}$}
		\label{alg:detectmarker2-3-maxpatterns}
			\State \textbf{break}
		\EndIf
		\State $i \gets i + 1$
	\EndFor
	\label{alg:detectmarker2-3-loop-end}
	\algstore{brk-detectmarker2-process}
\end{algorithmic}
\end{algorithm}

Nachdem die Kontur in \textproc{arGetContour} ermittelt wurde, wird das Ergebnis in $\mathit{ret}$ gespeichtert und
 anschließend in Zeile \ref{alg:detectmarker2-3-retvalue} geprüft, ob \textproc{arGetContour} nicht mit einem
 Fehlerwert beendet wurde. Ansonsten wird in Zeile \ref{alg:detectmarker2-3-continue} die nächste Iteration der
 Schleife eingeleitet. Danach wird in Zeile \ref{alg:detectmarker2-3-checksquare} die Methode \textproc{checkSquare}
 aufgerufen. Auch hier wird der Rückgabewert überprüft, um im Fehlerfall einen neuen Durchlauf der Schleife in Zeile
 \ref{alg:detectmarker2-3-continue2} einzuleiten. Wenn \textproc{checkSquare} erfolgreich beendet wurde, wird in Zeile
 \ref{alg:detectmarker2-3-save-start}--\ref{alg:detectmarker2-3-save-end} der Flächeninhalt und die Position des
 Zentrums der Marke aktualisiert, und die Anzahl der gefundenen Marken wird erhöht. In Zeile
 \ref{alg:detectmarker2-3-maxpatterns} wird die Anzahl der gefundenen Marken mit einem festgelegten Wert verglichen,
 der die maximale Anzahl gleichzeitiger Marken festlegt. Wenn die Werte übereinstimmen, wird die Schleife von Zeile
 \ref{alg:detectmarker2-3-loop-start} bis Zeile \ref{alg:detectmarker2-3-loop-end} abgebrochen. Andernfalls wird die
 Laufvariable $i$ erhöht und ein neuer Schleifendurchgang begonnen. Die Laufzeitfunktion ist in
 \autoref{eq:analyse-detectmarker2-2} angegeben.
\begin{subequations}
\label{eq:analyse-detectmarker2-2}
\begin{align}
\label{eq:analyse-detectmarker2-2-1}
T(\mathit{label\_num})& =
c_{2}(\mathit{label\_num} + 1)
+ c_{4}\mathit{label\_num}\Theta(n)
+ c_{5}\mathit{label\_num}
\\
& \quad
+ c_{6}\mathit{label\_num}
+ c_{8}(\mathit{label\_num})t
+ c_{9}\mathit{label\_num}
+ c_{10}\mathit{label\_num}
\nonumber \\
& \quad
+ 4c_{12}\mathit{label\_num}
+ 7c_{13}\mathit{label\_num}
+ 7c_{14}\mathit{label\_num}
+ 2c_{15}\mathit{label\_num}
\nonumber \\
& \quad
+ c_{16}\mathit{label\_num}
+ c_{17}\mathit{label\_num}
+ c_{19}\mathit{label\_num}
\nonumber \\
\label{eq:analyse-detectmarker2-2-2}
T(\mathit{label\_num})& =
+ c_{4}\mathit{label\_num}\Theta(n)
+ c_{2}
\\
& \quad
+ (c_{2} + c_{5} + c_{6} + tc_{8} + c_{9} + c_{10} + 4c_{12} + 7c_{13} + 7c_{14} + 2c_{15} + c_{16}
\nonumber \\
& \quad \quad
 + c_{17} + c_{19})\mathit{label\_num}
\nonumber
\end{align}
\end{subequations}


Das Verfahren \textproc{checksquare} benötigt als Eingabeparameter die Fläche der Region ($\mathit{area}$), die
 Datenstruktur $\mathit{marker\_infoTwo}$ mit Informationen der Region und Kontur und einen Faktor ($\mathit{factor}$)
 zur Berechnung eines Distanzschwellwerts.
\begin{algorithm}[!ht]\small
\caption{\textproc{checkSquare} (Initialisierung)}
\label{alg:checksquare-1}
\begin{algorithmic}[1]
	\Require $\mathit{area}, \mathit{marker\_infoTWO}, \mathit{factor}$
	\State $\mathit{sx}, \mathit{sy}, d, \mathit{vertex}[10] \gets \infty$
	\Cost{$c_{1}$}{$4$}
	\State $\mathit{vnum}, \mathit{wv1}[10], \mathit{wvnum1}, \mathit{wv2}[10], \mathit{wvnum2} \gets \infty$
	\Cost{$c_{2}$}{$5$}
	\State $\mathit{v2}, \mathit{thresh}, i \gets \infty$
	\Cost{$c_{3}$}{$3$}
	\State $\mathit{dmax},\mathit{v1} \gets 0$
	\Cost{$c_{4}$}{$2$}
	\algstore{brk-checksquare-1-init}
\end{algorithmic}
\end{algorithm}

Die lokalen Variablen werden in \autoref{alg:checksquare-1} initialisiert und ihre Bedeutung bei der ersten Verwendung erläutert. Die Laufzeitfunktion ist $T(n) = 14$ und ist konstant.

In \autoref{alg:checksquare-2} wird der größte Abstand von dem Punkt $(\mathit{sx},\mathit{sy})$ für alle anderen
 Konturpunkte berechnet. In Zeile \ref{alg:checksquare-2-x}--\ref{alg:checksquare-2-y} wird aus der Koordinatenliste
 von $\mathit{marker\_infoTWO}$ die $x$-Koordinate in $\mathit{sx}$ und die $y$-Koordinate in $\mathit{sy}$
 gespeichert.
\begin{algorithm}[ht]
\caption{\textproc{checkSquare} (Fortsetzung)}
\label{alg:checksquare-2}
\begin{algorithmic}[1]
	\algrestore{brk-checksquare-1-init}
	\State $\mathit{sx} \gets \mathit{marker\_infoTWO \to x\_coord}[0]$
	\label{alg:checksquare-2-x}
	\State $\mathit{sy} \gets \mathit{marker\_infoTWO \to y\_coord}[0]$
	\label{alg:checksquare-2-y}
	\For{$i \gets 1$ \textbf{to} $\mathit{marker\_infoTWO \to coord\_num} - 1$}
	\label{alg:checksquare-2-loop-start}
		\State $a \gets \mathit{marker\_infoTWO \to x\_coord}[i] - \mathit{sx}$
		\State $b \gets \mathit{marker\_infoTWO \to y\_coord}[i] - \mathit{sy}$
		\State $d \gets \left(a \cdot a\right) + \left(b \cdot b\right)$
		\If{$d > \mathit{dmax}$}
		\label{alg:checksquare-2-isdbigger}
			\State $\mathit{dmax} \gets d$
			\State $\mathit{v1} \gets i$
		\EndIf
		\State $i \gets i + 1$
	\EndFor
	\label{alg:checksquare-2-loop-end}
	\algstore{brk-checksquare-2-distance}
\end{algorithmic}
\end{algorithm}

In \textproc{arGetContour} (\autoref{alg:argetcontour-3}) wurde an der ersten Position von $\mathit{marker\_infoTWO}$
 eine Koordinate mit dem größten Abstand zum Anfang der Kontur gespeichert, die hier als Startpunkt der
 Rechteckerkennung dient. In der Schleife in Zeile \ref{alg:checksquare-2-loop-start}--\ref{alg:checksquare-2-loop-end}
 wird für alle Konturpunkte in $\mathit{marker\_infoTWO}$ der Abstand $d$ berechnet. Wenn der Abstand $d$ größer als
 $\mathit{dmax}$ ist (Zeile \ref{alg:checksquare-2-isdbigger}), wird $\mathit{dmax}$ mit dem neuen Abstandswert $d$
 überschrieben und der Index $i$ in $\mathit{v1}$ notiert. Im schlechtesten Fall wird die Hälfte aller Punkte
 untersucht, um den größten Abstand zu finden.Am Ende von \autoref{alg:checksquare-2} enthält
 $\mathit{v1}$ den Index der Koordinaten mit dem größten Abstand zum Punkt $(\mathit{sx},\mathit{sy})$ und ist der
 Index zu einem Eckpunkt der Kontur. Die Laufzeitfunktion ist in \autoref{eq:analyse-checksquare-distance} angegeben.
 Die Eingabemenge $\mathit{marker\_infoTWO.coord\_num}$ ist von $\mathit{AR\_CHAIN\_MAX}$ abhängig und wird in
 \autoref{eq:analyse-checksquare-distance} durch $n$ substituiert. Die Wachstumsrate ist $16n - 25 = \Theta(n)$ für
 $c_{1} = 15$, $c_{2} = 16$ und $n_{0} = 25$.
\begin{subequations}
\label{eq:analyse-checksquare-distance}
\begin{align}
\label{eq:analyse-checksquare-distance-1}
T_{worst}(n)& =
(c_{5} + c_{6})3
+ c_{7}(n-1)
+ 4(c_{8} + c_{9} + c_{10})(n-2)
+ (c_{11} + c_{15})(n-2)
\\
& \quad
+ (c_{12} + c_{13})(\frac{n-2}{2})
\nonumber \\
\label{eq:analyse-checksquare-distance-2}
T_{worst}(n)& =
n(c_{7} + 4c_{8} + 4c_{9} + 4c_{10} + c_{11} + \frac{c_{12}}{2} + \frac{c_{13}}{2} + c_{15})
\\
& \quad
+ 3(c_{5} + c_{6})
- (c_{7} + 8c_{8} + 8c_{9} + 8c_{10} + 2c_{11} + c_{12} + c_{13} + 2c_{15})
\nonumber
\end{align}
\end{subequations}


Die Menge der Konturpixel in $\mathit{marker\_infoTWO}$ wird mit $\mathit{v1}$ in \autoref{alg:checksquare-3} geteilt
 und die Teilmengen einzeln untersucht.
\begin{algorithm}[!ht]\small
\caption{\textproc{checkSquare} (Eckpunkte bestimmen)}
\label{alg:checksquare-3}
\begin{algorithmic}[1]
	\algrestore{brk-checksquare-2-distance}
	\State $\mathit{thresh} \gets \left(\mathit{area}/0.75\right) \cdot 0.01 \cdot \mathit{factor}$
	\Cost{$c_{17}$}{$4$}
	\label{alg:checksquare-3-thresh}
	\State $\mathit{vnum} \gets 1$
	\Cost{$c_{18}$}{$1$}
	\State $\mathit{vertex}[0] \gets 0$
	\Cost{$c_{19}$}{$2$}
	\State $\mathit{wvnum1} \gets 0$
	\Cost{$c_{20}$}{$1$}
	\label{alg:checksquare-3-wvnum1}
	\State $\mathit{wvnum2} \gets 0$
	\Cost{$c_{21}$}{$1$}
	\label{alg:checksquare-3-wvnum2}
	\State $x \gets \mathit{marker\_infoTWO.x\_coord}$
	\Cost{$c_{22}$}{$2$}
	\State $y \gets \mathit{marker\_infoTWO.y\_coord}$
	\Cost{$c_{23}$}{$2$}
	\If{\Call{getVertex}{$x, y, 0, \mathit{v1}, \mathit{thresh}, \mathit{wv1}, \mathit{wvnum1}$} $< 0$}
	\Cost{$c_{24}$}{$\Theta(\frac{n}{2} \log \frac{n}{2})$}
	\label{alg:checksquare-3-getvertex1}
		\State \textbf{return}$(-1)$
	\EndIf
	\State $x \gets \mathit{marker\_infoTWO.x\_coord}$
	\Cost{$c_{27}$}{$2$}
	\State $y \gets \mathit{marker\_infoTWO.y\_coord}$
	\Cost{$c_{28}$}{$2$}
	\State $\mathit{num} \gets \mathit{marker\_infoTWO.coord\_num} - 1$
	\Cost{$c_{29}$}{$3$}
	\If{\Call{getVertex}{$x, y, \mathit{v1}, \mathit{num}, \mathit{thresh}, \mathit{wv2}, \mathit{wvnum2}$} $<$}
	\Cost{$c_{30}$}{$\Theta(\frac{n}{2} \log \frac{n}{2})$}
	\label{alg:checksquare-3-getvertex2}
		\State \textbf{return}$(-1)$
	\EndIf
	\algstore{brk-checksquare-3-getvertex}
\end{algorithmic}
\end{algorithm}

Bevor die Konturpixel untersucht werden, wird in Zeile \ref{alg:checksquare-3-thresh} der Distanzschwellwert
 $\mathit{thresh}$ mit Hilfe der Fläche der Region und $\mathit{factor}$ erstellt. Danach wird die Anzahl der Vektoren
 $\mathit{vnum}$ initialisiert und der Index zum Punkt $(\mathit{sx}, \mathit{sy})$ in der Liste der Vektoren
 ($\mathit{vertex}$) gespeichert. $wvnum1$ und $wvnum2$ speichern die Anzahl der gefundenen Eckpunkte und werden in
 Zeile \ref{alg:checksquare-3-wvnum1}--\ref{alg:checksquare-3-wvnum2} initialisiert. In Zeile
 \ref{alg:checksquare-3-getvertex1} wird die Methode \textproc{getVertex} aufgerufen, die einen Eckpunkt in der Menge
 der Konturpixel zwischen Position $0$ und $\mathit{v1}$ in $\mathit{marker\_infoTWO}$ finden soll. Danach wird
 \textproc{getVertex} in Zeile \ref{alg:checksquare-3-getvertex2} aufgerufen, um einen Eckpunkt in der Menge zwischen
 $\mathit{v1}$ und dem letzten Eintrag in $\mathit{marker\_infoTWO}$ zu finden. Die Methode \textproc{getVertex} wird
 am Ende des Abschnitts beschrieben. Die Laufzeitfunktion ist in \autoref{eq:analyse-checksquare-vertex} angegeben.
\begin{align}
\label{eq:analyse-checksquare-vertex}
T_{worst}(n)& =
(c_{17}4 + c_{18}1 + c_{19}2 + c_{20}1 + c_{21}1 + c_{22}2 + c_{23}2 + c_{27}2 + c_{28}2 + c_{29}3)
\\
& \quad
+ c_{24}\Theta(\frac{n}{2} \log \frac{n}{2})
+ c_{30}\Theta(\frac{n}{2} \log \frac{n}{2})
\nonumber
\end{align}

Die Wachstumsrate der Funktion $20 + \Theta(\frac{n}{2} \log \frac{n}{2}) + \Theta(\frac{n}{2} \log \frac{n}{2})$ ist
 $\Theta(n \log n)$.

Die Ergebnisse aus \autoref{alg:checksquare-3} können in drei Fälle kategorisiert werden. Entweder wurde
\begin{enumerate}
	\item ein Eckpunkt in jeder Teilmenge gefunden, \label{enum:checksquare-case1}
	\item in der ersten Teilmenge mehr als ein Eckpunkt gefunden oder \label{enum:checksquare-case2}
	\item in der zweiten Teilmenge mehr als ein Eckpunkt gefunden. \label{enum:checksquare-case3}
\end{enumerate}
Jeder dieser Fälle wird in \autoref{alg:checksquare-4}--\autoref{alg:checksquare-6} untersucht.

\paragraph{1. Fall:} % (fold)
\label{par:1_fall}
Das Verfahren \autoref{alg:checksquare-4} behandelt den Fall, dass die Methode \textproc{getVertex} jeweils einen
 Eckpunkt für jede Teilmenge gefunden hat.
\begin{algorithm}[!ht]\small
\caption{\textproc{checkSquare} (1. Fall)}
\label{alg:checksquare-4}
\begin{algorithmic}[1]
	\algrestore{brk-checksquare-3-getvertex}
	\If{$\mathit{wvnum1} = 1 \land \mathit{wvnum2} = 1$}
	\Cost{$c_{33}$}{$3$}
		\State $\mathit{vertex}[1] \gets \mathit{wv1}[0]$
		\Cost{$c_{34}$}{$3$}
		\label{alg:checksquare-4-save-start}
		\State $\mathit{vertex}[2] \gets \mathit{v1}$
		\Cost{$c_{35}$}{$2$}
		\State $\mathit{vertex}[3] \gets \mathit{wv2}[0]$
		\Cost{$c_{36}$}{$3$}
		\label{alg:checksquare-4-save-end}
	\algstore{brk-checksquare-4-case1}
\end{algorithmic}
\end{algorithm}

Somit sind in diesem Fall die Indizes der beiden Eckpunkte $\mathit{wv1}$ und $\mathit{wv2}$ bekannt. Der Index
 $\mathit{v1}$ eines Eckpunktes wurde in \autoref{alg:checksquare-2} berechnet. Die Indizes werden in Zeile
 \ref{alg:checksquare-4-save-start}--\ref{alg:checksquare-4-save-end} in die Liste der Vektoren gespeichert. Die Laufzeitfunktion ist $T(n) = 11$ und somit konstant.
% paragraph  (end)

\paragraph{2. Fall:} % (fold)
\label{par:2_fall}
In diesem Fall wurde in der ersten Teilmenge mehr als ein Eckpunkt gefunden und in der zweiten Teilmenge kein Eckpunkt.
 \autoref{alg:checksquare-5} versucht in diesem Fall die Eckpunkte zu korrigieren.
\begin{algorithm}[ht]
\caption{\textproc{checkSquare} (Fortsetzung)}
\label{alg:checksquare-5}
\begin{algorithmic}[1]
	\algrestore{brk-checksquare-4-case1}
	\ElsIf{$\mathit{wvnum1} > 1 \land \mathit{wvnum2} == 0$}
		\State $\mathit{v2} \gets \mathit{v1} / 2$
		\label{alg:checksquare-5-v2}
		\State $\mathit{wvnum1} \gets 0$
		\label{alg:checksquare-5-delwvnum1}
		\State $\mathit{wvnum2} \gets 0$
		\label{alg:checksquare-5-delwvnum2}
		\State $x \gets \mathit{marker\_infoTWO \to x\_coord}$
		\State $y \gets \mathit{marker\_infoTWO \to y\_coord}$
		\If{\Call{getVertex}{$x, y, 0, \mathit{v2}, \mathit{thresh}, \mathit{wv1}, \mathit{wvnum1}$} $< 0$}
		\label{alg:checksquare-5-vertex1-start}
			\State \textbf{return}$(-1)$
		\EndIf
		\label{alg:checksquare-5-vertex1-end}
		\State $x \gets \mathit{marker\_infoTWO \to x\_coord}$
		\State $y \gets \mathit{marker\_infoTWO \to y\_coord}$
		\If{\Call{getVertex}{$x, y, \mathit{v2}, \mathit{v1}, \mathit{thresh}, \mathit{wv2}, \mathit{wvnum2}$} $< 0$}
		\label{alg:checksquare-5-vertex2-start}
			\State \textbf{return}$(-1)$
		\EndIf
		\label{alg:checksquare-5-vertex2-end}
		\If{$\mathit{wvnum1} == 1 \land \mathit{wvnum2} == 1$}
		\label{alg:checksquare-5-hascorners}
			\State $\mathit{vertex}[1] = \mathit{wv1}[0]$
			\label{alg:checksquare-5-save-start}
			\State $\mathit{vertex}[2] = \mathit{wv2}[0]$
			\State $\mathit{vertex}[3] = \mathit{v1}$
			\label{alg:checksquare-5-save-end}
		\Else
			\State \textbf{return}$(-1)$
			\label{alg:checksquare-5-error}
		\EndIf
	\algstore{brk-checksquare-5-case2}
\end{algorithmic}
\end{algorithm}

Dazu wird ein neuer Index $\mathit{v2}$ in Zeile \ref{alg:checksquare-5-v2} generiert, indem der Index $\mathit{v1}$
 verwendet wird. Danach wird die Anzahl der Eckpunkte in Zeile
 \ref{alg:checksquare-5-delwvnum1}--\ref{alg:checksquare-5-delwvnum2} gelöscht. In Zeile
 \ref{alg:checksquare-5-vertex1-start}--\ref{alg:checksquare-5-vertex1-end} wird mit \textproc{getVertex} die erste
 Teilmenge von $0$--$\mathit{v2}$ untersucht, und in Zeile
  \ref{alg:checksquare-5-vertex2-start}--\ref{alg:checksquare-5-vertex2-end} die zweite Teilmenge von
 $\mathit{v2}$--$\mathit{v1}$. Danach wird in Zeile \ref{alg:checksquare-5-hascorners} überprüft, ob ein Eckpunkt in
 jeder Teilmenge gefunden wurde. Falls dem so ist, werden die Indizes in Zeile
 \ref{alg:checksquare-5-save-start}--\ref{alg:checksquare-5-save-end} in die Liste der Vektoren geschrieben. Andernfalls
 handelt es sich bei der Kontur nicht um ein Rechteck und das Verfahren wird in Zeile \ref{alg:checksquare-5-error}
 abgebrochen. Die Laufzeitfunktion ist in \autoref{eq:analyse-checksquare-case2} angegeben. Die Wachstumsrate von
 $\Theta(\frac{n}{4} \log \frac{n}{4}) + \Theta(\frac{n}{4} \log \frac{n}{4}) + 26$ ist $\Theta(n \log n)$.
\begin{align}
\label{eq:analyse-checksquare-case2}
T_{worst}(n)& =
c_{37}3 + c_{38}2 + c_{39}1 + c_{40}1 + c_{41}2 + c_{42}2 + c_{46}2 + c_{47}2 + c_{51}3 + c_{52}3
\\
& \quad
+ c_{53}3 + c_{54}2
+ c_{43}\Theta(\frac{n}{4} \log \frac{n}{4})
+ c_{48}\Theta(\frac{n}{4} \log \frac{n}{4})
\nonumber
\end{align}

% paragraph 2_fall (end)

\paragraph{3. Fall:} % (fold)
\label{par:3_fall}
Der dritte Fall bedeutet, dass in der ersten Teilmenge kein Eckpunkt und in der zweiten Teilmenge mehr als ein Eckpunkt
 gefunden wurde. Das Verfahren in \autoref{alg:checksquare-6} ähnelt dem Verfahren in \autoref{alg:checksquare-5}
 (2. Fall).
\begin{algorithm}[ht]
\caption{\textproc{checkSquare} (Fortsetzung)}
\label{alg:checksquare-6}
\begin{algorithmic}[1]
	\algrestore{brk-checksquare-5-case2}
	\ElsIf{$\mathit{wvnum1} == 0 \land \mathit{wvnum2} > 1$}
		\State $\mathit{v2} \gets (\mathit{v1} + \mathit{marker\_infoTWO \to coord\_num} - 1) / 2$
		\label{alg:checksquare-6-v2}
		\State $\mathit{wvnum1} \gets 0$
		\State $\mathit{wvnum2} \gets 0$
		\State $x \gets \mathit{marker\_infoTWO \to x\_coord}$
		\State $y \gets \mathit{marker\_infoTWO \to y\_coord}$
		\If{\Call{getVertex}{$x, y, \mathit{v1}, \mathit{v2}, \mathit{thresh}, \mathit{wv1}, \mathit{wvnum1}$} $< 0$}
		\label{alg:checksquare-6-vertex1}
			\State \textbf{return}$(-1)$
		\EndIf
		\State $x \gets \mathit{marker\_infoTWO \to x\_coord}$
		\State $y \gets \mathit{marker\_infoTWO \to y\_coord}$
		\State $\mathit{num} \gets \mathit{marker\_infoTWO \to coord\_num} - 1$
		\If{\Call{getVertex}{$x, y, \mathit{v2}, \mathit{num}, \mathit{thresh}, \mathit{wv2}, \mathit{wvnum2}$} $< 0$}
		\label{alg:checksquare-6-vertex2}
			\State \textbf{return}$(-1)$
		\EndIf
		\If{$\mathit{wvnum1} == 1 \land \mathit{wvnum2} == 1$}
			\State $\mathit{vertex}[1] \gets \mathit{v1}$
			\State $\mathit{vertex}[2] \gets \mathit{wv1}[0]$
			\State $\mathit{vertex}[3] \gets \mathit{wv2}[0]$
		\Else
			\State \textbf{return}$(-1)$
		\EndIf
	\Else
		\State \textbf{return}$(-1)$
	\EndIf
	\algstore{brk-checksquare-6-case3}
\end{algorithmic}
\end{algorithm}

 Auch hier wird ein neuer Index $\mathit{v2}$ erstellt (Zeile \ref{alg:checksquare-6-v2}). In diesem Fall wird der
 Index $\mathit{v2}$ erstellt, indem der Index $\mathit{v1}$ und die Anzahl der gespeicherten Konturpunkte verwendet
 werden. Der Rest des Verfahrens entspricht dem Verfahren \autoref{alg:checksquare-5}. Lediglich der Aufruf von
 \textproc{getVertex} in Zeile \ref{alg:checksquare-6-vertex1} und Zeile \ref{alg:checksquare-6-vertex2} verwendet
 eine andere Teilmengen der Konturpunkte aus $\mathit{marker\_infoTWO}$. Die Laufzeitfunktion ist in
 \autoref{eq:analyse-checksquare-case3} angegeben. Die Wachstumsrate von \autoref{alg:checksquare-6} ist
 $\Theta(\tfrac{n}{4} \log \tfrac{n}{4}) + \Theta(\tfrac{n}{4} \log \tfrac{n}{4}) + 32 = \Theta(n \log n)$.
\begin{align}
\label{eq:analyse-checksquare-case3}
T_{worst}(n)& =
c_{58}3 + c_{59}5 + c_{60}1 + c_{61}1 + c_{62}2 + c_{63}2 + c_{67}2 + c_{68}2 + c_{69}3 + c_{73}3
\\
& \quad
+ c_{74}2 + c_{75}3 + c_{76}3
+ c_{64}\Theta(\frac{n}{4} \log \frac{n}{4})
+ c_{70}\Theta(\frac{n}{4} \log \frac{n}{4})
\nonumber
\end{align}

% paragraph 3_fall (end)

Wenn \autoref{alg:checksquare-4}--\autoref{alg:checksquare-6} erfolgreich waren, und Eckpunkte in die Liste der
 Vektoren schreiben konnten, werden mit \autoref{alg:checksquare-7} die Daten in $\mathit{marker\_infoTWO}$
 gespeichert.
\begin{algorithm}[ht]
\caption{\textproc{checkSquare} (Fortsetzung)}
\label{alg:checksquare-7}
\begin{algorithmic}[1]
	\algrestore{brk-checksquare-6-case3}
	\State $\mathit{marker\_infoTWO \to vertex}[0] \gets \mathit{vertex}[0]$
	\label{alg:checksquare-7-savesxsy}
	\State $\mathit{marker\_infoTWO \to vertex}[1] \gets \mathit{vertex}[1]$
	\label{alg:checksquare-7-save1}
	\State $\mathit{marker\_infoTWO \to vertex}[2] \gets \mathit{vertex}[2]$
	\State $\mathit{marker\_infoTWO \to vertex}[3] \gets \mathit{vertex}[3]$
	\label{alg:checksquare-7-save3}
	\State $\mathit{marker\_infoTWO \to vertex}[4] \gets \mathit{marker\_infoTWO \to coord\_num} - 1$
	\label{alg:checksquare-7-savelast}
	\State \textbf{return} $0$
\end{algorithmic}
\end{algorithm}

In Zeile \ref{alg:checksquare-7-savesxsy} wird der Punkt $(\mathit{sx},\mathit{sy})$ als erster Index in die Liste der
 Vektoren von $\mathit{marker\_infoTWO}$ gespeichert. Je nachdem welcher der drei Fälle behandelt wurde, werden die
 Indizes $\mathit{v1}$, $\mathit{wv1}$ und $\mathit{wv2}$ in Zeile
 \ref{alg:checksquare-7-save1}--\ref{alg:checksquare-7-save3} gespeichert. Die Indizes sind danach im Uhrzeigersinn in
 $\mathit{marker\_infoTWO}$ hinterlegt\footcite[Vgl.][S.~44]{wagner07a}. In Zeile \ref{alg:checksquare-7-savelast} wird
 der letzte Index aus $\mathit{marker\_infoTWO}$ in die Liste der Vektoren geschrieben. Da der erste und der letzte
 Index identisch sind (Vgl. \autoref{alg:argetcontour-3}), sind die Eckpunkte so angeordnet, dass sie grafisch einfach
 dargestellt werden können, indem mit nur einer Linie, vom Startpunkt aus über jeden Eckpunkt zurück zum Startpunkt,
 ein Rahmen gezeichnet wird. Die Laufzeitfunktion ist $T(n) = 22$ und somit konstant.

Das Verfahren \textproc{checkSquare} besteht im schlechtesten Fall aus der Initialisierung
 (\autoref{alg:checksquare-1}), der Distanzberechnung (\autoref{alg:checksquare-2}), der Bestimmung von Eckpunkten
 (\autoref{alg:checksquare-3}), der Untersuchung des 3. Falls (\autoref{alg:checksquare-6}) und der Speicherung der
 Eckpunkte (\autoref{alg:checksquare-7}). Die Laufzeifunktion ist in \autoref{eq:analyse-checksquare-all} angegeben.
\begin{subequations}
\label{eq:analyse-checksquare-all}
\begin{align}
\label{eq:analyse-checksquare-all-1}
T_{worst}(n)& =
14 + 16n - 25 + \Theta(n \log n) + \Theta(n \log n) + 22
\\
\label{eq:analyse-checksquare-all-2}
T_{worst}(n)& =
2 \Theta(n \log n) + 16n + 11
\end{align}	
\end{subequations}

Die Wachstumsrate des Verfahrens ist $2\Theta(n \log n) + 16n = \Theta(n \log n)$.

\textproc{getvertex} (\autoref{alg:getvertex-1}--\autoref{alg:getvertex-2}) sucht in der Menge der Koordinaten den
 Punkt, der von der Linie zwischen der Start- und Endkoordinate am weitesten entfernt liegt. Dazu benötigt
 \autoref{alg:getvertex-1} die Liste der $x$- und $y$-Koordinaten, den Startpunkt $\mathit{st}$ und Endpunkt
 $\mathit{ed}$, sowie den Schwellwert zur Überprüfung der Distanz.
\begin{algorithm}[!ht]
\caption{\textproc{getVertex} (Abstand Punkt-Gerade)}
\label{alg:getvertex-1}
\begin{algorithmic}[1]
	\Require $\mathit{x\_coord}[], \mathit{y\_coord}[], \mathit{st}, \mathit{ed}, \mathit{thresh}, \mathit{vertex}[],
	 \mathit{vnum}$
	\State $d, \mathit{dmax}, i \gets \infty$
	\Cost{$c_{1}$}{$3$}
	\label{alg:getvertex-1-init-start}
	\State $\mathit{v1} \gets 0$
	\Cost{$c_{2}$}{$1$}
	\label{alg:getvertex-1-init-end}
	\State $a \gets \mathit{y\_coord}[ed] - \mathit{y\_coord}[st]$
	\Cost{$c_{3}$}{$4$}
	\label{alg:getvertex-1-line-start}
	\State $b \gets \mathit{x\_coord}[st] - \mathit{x\_coord}[ed]$
	\Cost{$c_{4}$}{$4$}
	\State $c \gets \left(\mathit{x\_coord}[ed] \cdot \mathit{y\_coord}[st]\right) - \left(\mathit{y\_coord}[ed] \cdot
	 \mathit{x\_coord}[st]\right)$
	\Cost{$c_{5}$}{$8$}
	\label{alg:getvertex-1-line-end}
	\State $\mathit{dmax} \gets 0$
	\Cost{$c_{6}$}{$1$}
	\For{$\mathit{i} \gets \mathit{st} + 1$ \textbf{to} $i < \mathit{ed}$}
	\Cost{$c_{7}$}{$n$}
	\label{alg:getvertex-1-loop-start}
		\State $d \gets \left(a \cdot \mathit{x\_coord}[i]\right) + \left(b \cdot \mathit{y\_coord}[i]\right) + c$
		\Cost{$c_{8}$}{$7(n - 1)$}
		\If{$d \cdot d > \mathit{dmax}$}
		\Cost{$c_{9}$}{$2(n - 1)$}
			\State $\mathit{dmax} \gets d \cdot d$
			\Cost{$c_{10}$}{$2(n - 1)$}
			\label{alg:getvertex-1-savedmax}
			\State $\mathit{v1} \gets i$
			\Cost{$c_{11}$}{$n - 1$}
		\EndIf
		\State $i \gets i + 1$
		\Cost{$c_{13}$}{$n - 1$}
	\EndFor
	\label{alg:getvertex-1-loop-end}
	\algstore{brk-getvertex1-distance}
\end{algorithmic}
\end{algorithm}

In der Liste $\mathit{vertex}$ werden die Indizes der Koordinaten gespeichert. Die Variable $\mathit{vnum}$ enthält die
 Anzahl der gefundenen Eckpunkte. In Zeile \ref{alg:getvertex-1-init-start}--\ref{alg:getvertex-1-init-end} werden die
 lokalen Variablen initialisiert. Die Variable $a$, $b$ und $c$ werden in Zeile
 \ref{alg:getvertex-1-line-start}--\ref{alg:getvertex-1-line-end} initialisiert und werden zur Berechnung des Abstand
 zwischen Punkt $i$ und der Geraden $(\mathit{st},\mathit{ed})$ verwendet. Dazu wird in der Schleife in Zeile
 \ref{alg:getvertex-1-loop-start}--\ref{alg:getvertex-1-loop-end} jede Koordinate $i$ in die Geradengleichung
 \autoref{eq:punktgeradenabstand} eingesetzt, um den Abstand des Punktes zur Geraden zu berechnen.
\begin{equation}
	\label{eq:punktgeradenabstand}
	d = \frac{\mathit{ax} + \mathit{by} + c}{\sqrt{(a^2 + b^2)}}
\end{equation}
Wenn der Abstand größer als $\mathit{dmax}$ ist, wird $d$ in Zeile \ref{alg:getvertex-1-savedmax} in $\mathit{dmax}$
 gespeichert und die Position $i$ in $\mathit{v1}$ hinterlegt. Somit ist am Ende von \autoref{alg:getvertex-1} die
 Position $i$ des Punktes mit dem größten Abstand zur Linie gespeichert. Die Laufzeitfunktion ist in
 \autoref{eq:analyse-getvertex-distance} aufgeführt.
\begin{subequations}
\label{eq:analyse-getvertex-distance}
\begin{align}
\label{eq:analyse-getvertex-distance-1}
T_{worst}(n)& =
(c_{1}3 + c_{2} + c_{3}4 + c_{4}4 + c_{5}8 + c_{6})
+ c_{7}n
\\
& \quad
+ (n-1) (c_{8}7 + c_{9}2 + c_{10}2 + c_{11}+ c_{13})
\nonumber \\
\label{eq:analyse-getvertex-distance-2}
T_{worst}(n)& =
n(c_{7} + c_{8}7 + c_{9}2 + c_{10}2 + c_{11}+ c_{13})
- (c_{8}7 + c_{9}2 + c_{10}2 + c_{11}+ c_{13})
\\
& \quad
+ (c_{1}3 + c_{2} + c_{3}4 + c_{4}4 + c_{5}8 + c_{6})
\nonumber
% n*(c_7 + 7*c_8 + 2*c_9 + 2*c_10 + c_11 + c_13)
% - (7*c_8 + 2*c_9 + 2*c_10 + c_11 + c_13)
% + (c_1*3 + c_2 + c_3*4 + c_4*4 + c_5*8 + c_6)
\end{align}
\end{subequations}

Die Eingabemenge $n$ entspricht der Anzahl der Einträge im Interval $[1,\mathit{ed}-\mathit{sd})$. Im schlechsten Fall
 liegt der Punkt mit dem größten Abstand am Ende der Liste. Die Wachstumsrate für $c_{1} = 13$, $c_{2} = 14$ und
 $n_{0} = 1$ ist $14n = \Theta(n)$.

In \autoref{alg:getvertex-2} wird in Zeile \ref{alg:getvertex-2-isdmaxgreater} der Abstand mit dem Distanzschwellwert
 verglichen. Nur wenn der Abstand größer als der Distanzschwellwert ist, wird das Verfahren fortgesetzt. Zur Analyse
 von \textproc{getVertex} (\autoref{alg:getvertex-2}) wird $\mathit{tresh} = 0$ verwendet. Somit wird das
 Verfahren solange ausgeführt, bis der Abstand $\mathit{dmax} = 0$ ist. Dies ist dann der Fall, wenn eine Linie von
 $\mathit{st}$ bis $\mathit{ed}$ genau über einer Kontur liegt. Das Verfahren wird dann in Zeile
 \ref{alg:getvertex-2-done} beendet.
\begin{algorithm}[!ht]
\caption{\textproc{getVertex} (Fortsetzung)}
\label{alg:getvertex-2}
\begin{algorithmic}[1]
	\algrestore{brk-getvertex1-distance}
	\If{$\mathit{dmax}/\left((a \cdot a) + (b \cdot b)\right) > \mathit{thresh}$}
	\Cost{$c_{15}$}{$5$}
	\label{alg:getvertex-2-isdmaxgreater}
			\If{\Call{getVertex}{$\mathit{x\_coord}, \mathit{y\_coord}, \mathit{st}, \mathit{v1}, \mathit{thresh},
			 \mathit{vertex}, \mathit{vnum}$} $< 0$}
			\Cost{$c_{16}$}{$1$}
			\label{alg:getvertex-2-recursiv1}
				\State \textbf{return}$(-1)$
			\EndIf
			\If{$\mathit{vnum} > 5$}
			\Cost{$c_{19}$}{$1$}
			\label{alg:getvertex-2-isvnumgreater}
				\State \textbf{return}$(-1)$
				\label{alg:getvertex-2-error}
			\EndIf
			\State $\mathit{vertex}[\mathit{vnum}] \gets \mathit{v1}$
			\Cost{$c_{22}$}{$2$}
			\State $\mathit{vnum} \gets \mathit{vnum} + 1$
			\Cost{$c_{23}$}{$1$}
			\If{\Call{getVertex}{$\mathit{x\_coord}, \mathit{y\_coord}, \mathit{v1}, \mathit{ed}, \mathit{thresh},
			 \mathit{vertex}, \mathit{vnum}$} $< 0$}
			\Cost{$c_{24}$}{$1$}
			\label{alg:getvertex-2-recursiv2-start}
				\State \textbf{return}$(-1)$
			\EndIf
			\label{alg:getvertex-2-recursiv2-end}
	\EndIf
	\State \textbf{return} $0$
	\Cost{$c_{28}$}{$1$}
	\label{alg:getvertex-2-done}
\end{algorithmic}
\end{algorithm}

Für alle Fälle in denen $\mathit{dmax} > 0$ ist, wird die Eingabemenge $n$ in zwei Problemmengen geteilt. In Zeile
 \ref{alg:getvertex-2-recursiv1} wird die Methode \textproc{getVertex} rekursiv aufgerufen, um die Teilmenge von der
 Startposition $\mathit{st}$ bis zur Position $\mathit{v1}$ zu untersuchen. Da \textproc{getVertex} pro Teilmenge nur
 den größten Abstand als Eckpunkt betrachtet, kann durch den rekursiven Aufruf für eine kleinere Menge von Koordinaten
 ein weitere Eckpunkt gefunden werden, solange der Abstand größer als der Distanzschwellwert ist. In Zeile
 \ref{alg:getvertex-2-isvnumgreater} wird überprüft, ob nicht mehr als fünf Eckpunkte gefunden wurde. Falls mehr als
 fünf Eckpunkte gefunden wurden, wird das Verfahren in Zeile \ref{alg:getvertex-2-error} mit einem Fehlerwert beendet.
 Falls jedoch weniger als fünf Eckpunkte gefunden wurden, wird der Index $\mathit{v1}$ in der Liste $\mathit{vertex}$
 gespeichert und die Anzahl der gefundenen Eckpunkte erhöht. Danach wird \textproc{getVertex} für die zweite Teilmenge,
 von der Position $\mathit{v1}$ bis zum Endpunkt $\mathit{ed}$, in Zeile
  \ref{alg:getvertex-2-recursiv2-start}--\ref{alg:getvertex-2-recursiv2-end} rekursiv aufgerufen. Die Teilung des
 Problems in Unterprobleme wird von \autoref{alg:getvertex-1} durchgeführt. Dazu gehört die Überprüfung, ob eine
 Rekursion möglich ist (\autoref{alg:getvertex-2} Zeile \ref{alg:getvertex-2-isdmaxgreater}), und das Beenden des
 Verfahrens (\autoref{alg:getvertex-2} Zeile \ref{alg:getvertex-2-done}). Somit ist die Laufzeitfunktion
 $T(n) = 14n + 14$ mit der Wachstumsrate $14n = \Theta(n)$ für $c_{1} = 13$, $c_{2} = 14$ und $n_{0} = 1$. Daraus
 folgt \autoref{eq:analyse-getvertex-all-1}.
\begin{align}
\label{eq:analyse-getvertex-all-1}
T(n)
\begin{cases}
	14n + 14 & \mbox{wenn } \mathit{dmax} = 0 \\
	2T(\frac{n}{2}) + 14n + 14 & \mbox{sonst, für } \mathit{vnum} < 6
\end{cases}
\end{align}

Durch Überprüfung von $T(n) = 2T(\tfrac{n}{2}) + 14n$ mit dem Master Theorem in \autoref{eq:analyse-getvertex-all-2}
\begin{subequations}
\label{eq:analyse-getvertex-all-2}
\begin{align}
\label{eq:analyse-getvertex-all-2-fall2}
f(n) &\in \Theta(n^{\log_b a})
\\
\label{eq:analyse-getvertex-all-2-values}
14n &\in \Theta(n^{\log_2 2})
\\
\label{eq:analyse-getvertex-all-2-true}
14n &\in \Theta(n)
\end{align}
\end{subequations}

folgt $T(n) = \Theta(n \log n)$ für das Verfahren \textproc{getVertex}.

Die Laufzeitfunktion von \textproc{arDetectMarker2} (\autoref{alg:detectmarker2-1}, \autoref{alg:detectmarker2-2} und
 \autoref{alg:detectmarker2-3}) ist in \autoref{eq:analyse-detectmarker2-3} angegeben. Die Variable
 $\mathit{label\_num}$ erlaubt maximal $\tfrac{\mathit{lxsize}\cdot\mathit{lysize}}{4}$ Einträge, während Variable $n$
 von $\mathit{AR\_CHAIN\_MAX} = 10000$ begrenzt ist.
\begin{subequations}
\label{eq:analyse-detectmarker2-3}
\begin{align}
\label{eq:analyse-detectmarker2-3-1}
T(\mathit{label\_num})& =
13
+ \Theta(n) \cdot \mathit{label\_num} + 28 \mathit{label\_num} + 2
\\
& \quad
+ 26 \mathit{label\_num}
+ \Theta(n \log n) \cdot \mathit{label\_num}
\nonumber \\
\label{eq:analyse-detectmarker2-3-2}
T(\mathit{label\_num})& =
\mathit{label\_num} \cdot \Theta(n \log n) + \mathit{label\_num} \cdot \Theta(n)
\\
& \quad
+ 54\mathit{label\_num} + 15
\nonumber \\
\label{eq:analyse-detectmarker2-3-3}
T(\mathit{label\_num})& =
\Theta(\mathit{label\_num} \cdot n \log n) + \Theta(\mathit{label\_num} \cdot n)
\\
& \quad
+ 54\mathit{label\_num} + 15
\nonumber
\end{align}
\end{subequations}

Die Wachstumsrate von \textproc{detectmarker2} entspricht $\Theta(\mathit{label\_num}\cdot n \log n)$.

% Nachdem \textproc{arGetCountour} abgeschlossen ist, wird in \textproc{arDetectMarker2} die Marke mit der größten Fläche
%  gesucht. Dazu wird in \autoref{alg:detectmarker2-4} in der Schleife von Zeile \ref{alg:detectmarker2-4-loop1-start}
%  bis Zeile \ref{alg:detectmarker2-4-loop1-end} jede Marke mit allen anderen Marken in der Schleife von Zeile
%  \ref{alg:detectmarker2-4-loop2-start}--\ref{alg:detectmarker2-4-loop2-end} verglichen.
% 
% \begin{algorithm}[!ht]\small
\caption{\textproc{arDetectMarker2} (Größte Marke)}
\label{alg:detectmarker2-4}
\begin{algorithmic}[1]
	\algrestore{brk-detectmarker2-process}
	\For{$i \gets 0$ \textbf{to} $i < \mathit{marker\_num2}$}
	\label{alg:detectmarker2-4-loop1-start}
		\For{$j \gets i + 1$ \textbf{to} $j < \mathit{marker\_num2}$}
		\label{alg:detectmarker2-4-loop2-start}
			\State $a \gets \mathit{marker\_infoTWO}[i].pos[0] - \mathit{marker\_infoTWO}[j].pos[0]$
			\State $b \gets \mathit{marker\_infoTWO}[i].pos[1] - \mathit{marker\_infoTWO}[j].pos[1]$
			\State $d \gets \left(a \cdot a\right) + \left(b \cdot b\right)$
			\label{alg:detectmarker2-4-length}
			\If{$\mathit{marker\_infoTWO}[i].area > \mathit{marker\_infoTWO}[j].area$}
				\If{$d < \mathit{marker\_infoTWO}[i].area / 4$}
				\label{alg:detectmarker2-4-di}
					\State $\mathit{marker\_infoTWO}[j].area \gets 0$
				\EndIf
			\Else
				\If{$d < \mathit{marker\_infoTWO}[j].area / 4$}
				\label{alg:detectmarker2-4-dj}
					\State $\mathit{marker\_infoTWO}[i].area \gets 0$
				\EndIf
			\EndIf
		\State $j \gets j + 1$
		\EndFor
		\label{alg:detectmarker2-4-loop2-end}
	\State $i \gets i + 1$
	\EndFor
	\label{alg:detectmarker2-4-loop1-end}
	\algstore{brk-detectmarker2-sort}
\end{algorithmic}
\end{algorithm}

% 
% In Zeile \ref{alg:detectmarker2-4-length} wird die quadratische Länge des Abstands zwischen dem Mittelpunkt der beiden
%  Marken berechnet. Anschließend wird die Anzahl der Konturpixel in $\mathit{area}$ der Marke $i$ und $j$ verglichen.
%  Wenn $i$ größer als $j$ ist, wird in Zeile \ref{alg:detectmarker2-4-di} der Abstand $d$ mit $\frac{\mathit{area}}{4}$
%  von Marke $i$ verglichen. Wenn der Abstand kleiner ist, wird die Anzahl der Konturpixel der Marke $j$ auf $0$ gesetzt.
%  Falls jedoch die Anzahl der Konturpixel von Marke $j$ größer sein sollte als die von $i$, wird in Zeile
%  \ref{alg:detectmarker2-4-dj} der Abstand $d$ mit $\frac{\mathit{area}}{4}$ von Marke $j$ verglichen. Ist der Abstand
%  kleiner, wird die Anzahl von Marke $i$ auf $0$ gesetzt.
% 
% In \autoref{alg:detectmarker2-5} werden nun alle Marken gelöscht, deren Variable $\mathit{area}$ in
%  \autoref{alg:detectmarker2-4} auf $0$ gesetzt worden sind. Dazu werden alle Marken in Zeile
%  \ref{alg:detectmarker2-5-loop-start}--\ref{alg:detectmarker2-5-loop-end} untersucht. Wenn die Überprüfung in Zeile
%  \ref{alg:detectmarker2-5-shoulddelete} eine zu löschende Marke findet, wird in der Schleife von Zeile
%  \ref{alg:detectmarker2-5-move-start} bis Zeile \ref{alg:detectmarker2-5-move-end} alle nachfolgende Marken um eine
%  Position verschoben. Danach wird in Zeile \ref{alg:detectmarker2-5-decrease} die Anzahl der Marken verringert.
% 
% \begin{algorithm}[!ht]\small
\caption{\textproc{arDetectMarker2} (Lösche Marken)}
\label{alg:detectmarker2-5}
\begin{algorithmic}[1]
	\algrestore{brk-detectmarker2-sort}
	\For{$i \gets 0$ \textbf{to} $\mathit{marker\_num2}$}
	\label{alg:detectmarker2-5-loop-start}
		\If{$\mathit{marker\_infoTWO}[i].area = 0$}
		\label{alg:detectmarker2-5-shoulddelete}
			\For{$j \gets i + 1$ \textbf{to} $\mathit{marker\_num2}$}
			\label{alg:detectmarker2-5-move-start}
				\State $\mathit{marker\_infoTWO}[j - 1] \gets \mathit{marker\_infoTWO}[j]$
				\State $j \gets j + 1$
			\EndFor
			\label{alg:detectmarker2-5-move-end}
			\State $\mathit{marker\_num2} \gets \mathit{marker\_num2} - 1$
			\label{alg:detectmarker2-5-decrease}
		\EndIf
		\State $i \gets i + 1$
	\EndFor
	\label{alg:detectmarker2-5-loop-end}
	\algstore{brk-detectmarker2-decrease}
\end{algorithmic}
\end{algorithm}

% 
% In \autoref{sub:regionenmarkierung} wurde zur Optimierung des Verfahrens nur ein Teil des Bildsignals analysiert.
%  Dadurch sind Daten entstanden, die nicht mit den tatsächlichen Daten im Bildsignal übereinstimmen.
%  \autoref{alg:detectmarker2-6} sorgt dafür, dass diese Daten wieder aufbereitet werden.
% 
% \begin{algorithm}[ht]
\caption{\textproc{arDetectMarker2} (Koordinaten aufbereiten)}
\label{alg:detectmarker2-6}
\begin{algorithmic}[1]
	\algrestore{brk-detectmarker2-decrease}
	\State $\mathit{pm} \gets \mathit{marker\_infoTWO}[0]$
	\label{alg:detectmarker2-6-address}
	\For{$i \gets 0$ \textbf{to} $\mathit{marker\_num2}$}
	\label{alg:detectmarker2-6-loop-start}
		\State $\mathit{pm.area} \gets \mathit{pm.area} \cdot 4$
		\label{alg:detectmarker2-6-area}
		\State $\mathit{pm.pos}[0] \gets \mathit{pm.pos}[0] \cdot 2$
		\State $\mathit{pm.pos}[1] \gets \mathit{pm.pos}[1] \cdot 2$
		\label{alg:detectmarker2-6-pos}
		\For{$j \gets 0$ \textbf{to} $\mathit{pm.coord\_num}$}
		\label{alg:detectmarker2-6-coord-start}
			\State $\mathit{pm.x\_coord}[j] \gets \mathit{pm.x\_coord}[j] \cdot 2$
			\State $\mathit{pm.y\_coord}[j] \gets \mathit{pm.y\_coord}[j] \cdot 2$
			\State $j \gets j + 1$
		\EndFor
		\label{alg:detectmarker2-6-coord-end}
		\State Inkrementiere $\mathit{pm}$
		\label{alg:detectmarker2-6-incpm}
		\State $i \gets i + 1$
	\EndFor
	\label{alg:detectmarker2-6-loop-end}
	\State $\mathit{marker\_num} \gets \mathit{marker\_num2}$
	\State \textbf{return} $\mathit{marker\_infoTWO}[0]$
\end{algorithmic}
\end{algorithm}

% 
% In Zeile \ref{alg:detectmarker2-6-address} wird die Adresse der ersten Speicherstelle der Markeninformationen in der
%  Variable $\mathit{pm}$ hinterlegt. In der Schleife in Zeile
%  \ref{alg:detectmarker2-6-loop-start}--\ref{alg:detectmarker2-6-loop-end} werden nun alle verbliebenen Marken
%  aufbereitet. Dazu wird in Zeile \ref{alg:detectmarker2-6-area} bis Zeile \ref{alg:detectmarker2-6-pos} die Anzahl der
%  Konturpixel erhöht und die Koordinaten des Zentrums der Marke korrigiert. Danach wird in der Schleife von Zeile
%  \ref{alg:detectmarker2-6-coord-start}--\ref{alg:detectmarker2-6-coord-end} die Koordinaten aller Konturpixel
%  korrigiert. Anschließend wird in Zeile \ref{alg:detectmarker2-6-incpm} die Adresse inkrementiert und das Verfahren mit
%  der nächsten Marke wiederholt. Abschließend wird die Anzahl der Marken in $\mathit{marker\_num}$ gespeichert und die
%  Speicheradresse der ersten Markeninformation an \textproc{arDetectMarker} (\autoref{alg:detectmarker},
%  S.~\pageref{alg:detectmarker}) zurückgegeben.

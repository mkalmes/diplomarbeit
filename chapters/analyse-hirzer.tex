\section{Verfahren nach Hirzer} % (fold)
\label{sec:hirzer}

\subsection{Datenstrukturen} % (fold)
\label{sub:datenstrukturen}

\subsubsection{Edgels} % (fold)
\label{sub:datenstruktur-edgels}

Die Datenstruktur eines \glspl{edgel} besteht aus den beiden Variablen $\mathit{coordinate}$ und $\mathit{slope}$
 (\autoref{alg:datastructure-edgel}). Beide Variablen sind vom Datentyp \textproc{vector} (Vgl.
 \autoref{alg:vector}), sodass Lese- und Schreibzugriffe auf die Elemente eines \glspl{edgel} konstant sind.
\begin{algorithm}[ht]
\caption{Datenstruktur eines Edgels}
\label{alg:datastructure-edgel}
	\begin{algorithmic}[1]
		\State $x$
		\State $y$
		\State $o$
		\Comment Orientierung
	\end{algorithmic}
\end{algorithm}


Wie in \autoref{sub:line_detection} beschrieben, muss im Verfahren von \citeauthor{hirzer08} die Orientierung von zwei
 \glspl{edgel} überprüft werden, was mit \autoref{alg:edgeliscompatible} bewerkstelligt wird. Als Parameter werden
 zwei zu vergleichende \gls{edgel} $\mathit{left}$ und $\mathit{right}$ übergeben.
\begin{algorithm}[ht]
\caption{\textproc{isCompatible}}
\label{alg:edgeliscompatible}
\begin{algorithmic}[1]
	\Require $\mathit{left}, \mathit{right}$
	\State \textbf{return} \Call{dotProduct}{$\mathit{left}, \mathit{right}$} $> 0.38$
\end{algorithmic}
\end{algorithm}

Mithilfe von \textproc{dotProduct} (\autoref{alg:vectordotproduct}) kann die Orientierung der beiden \gls{edgel}
 überprüft werden. Zwei \gls{edgel} sind dann kompatibel, wenn der Winkel zwischen den Vektoren nicht größer als
 $67.5^\circ$ ist\footcite[Vgl.][S.~417]{clarke96}. Dies wird durch \autoref{eq:edgeliscompatible} in
 \autoref{alg:edgeliscompatible} sichergestellt. Es ist dabei darauf zu achten, dass die Berechnung in Bogenmaß
 erfolgt. Die Laufzeitfunktion ist $T(n) = 8$.

\begin{equation}
	\label{eq:edgeliscompatible}
	\cos \left(67.5\right) \approx 0.38
\end{equation}

Der Edgelspeicher in \autoref{alg:datastructure-edgelpool} verwendet ein Array von \glspl{edgel}
 (Vgl. \autoref{alg:datastructure-edgel}) mit fester Größe $N$ und einer Zählvariable, um die nächste freie Position im
 Array zu markieren.
\begin{algorithm}[!ht]
\caption{\textproc{edgelPool} (Datenstruktur)}
\label{alg:datastructure-edgelpool}
\begin{algorithmic}[1]
	\State $\mathit{data}[N]$
	\Comment Anzahl der Einträge
	\State $\mathit{count}$
\end{algorithmic}
\end{algorithm}

Der Speichervorrat aus \autoref{alg:datastructure-poolimplementation} ist ein Array vom Typ
 \textproc{edgelPool} mit der Größe $S$, dessen Adresse im Zeiger $\mathit{pool}$ gespeichert wird.
\begin{algorithm}[!ht]\small
\caption{\textproc{edgelPool} (Speichervorrat)}
\label{alg:datastructure-poolimplementation}
\begin{algorithmic}[1]
	\State $\mathit{data}[S]$
	\Comment Anzahl der Pools
	\State $\mathit{pool}$
\end{algorithmic}
\end{algorithm}

\autoref{alg:edgelpool-getmemorypools} basiert auf einem einfachen Stack Allocator von
 \citeauthor{kr}\footcite[Vgl.][S.~100--104]{kr}.
\input{alg/analyse-hirzer/datastructure-edgelpool-getmemorypools}
Die Variable $n$ gibt die Anzahl der angeforderten Blöcke aus dem Speichervorrat an. In Zeile
 \ref{alg:edgelpool-getmemorypools-checkpoolsize} wird überprüft, ob genügend Blöcke zur Verfügung stehen und liefert
 im Erfolgsfall die Adresse zu einem Block (\autoref{alg:datastructure-edgelpool}) zurück. Falls keine Blöcke mehr zur
 Verfügung stehen, wird $\mathit{NULL}$ zurückgegeben. Die Laufzeitfunktion ist im schlechtesten Fall $T(n)=7$.
 \autoref{alg:edgelpool-getmemorypool}
\begin{algorithm}[ht]
\caption{\textproc{getMemoryPool}}
\label{alg:edgelpool-getmemorypool}
\begin{algorithmic}[1]
	\State $p \gets$ \Call{getmemorypools}{1}
	\State \textbf{return} $p$
\end{algorithmic}
\end{algorithm}
 vereinfacht die Anforderung von Speicherblöcken, da in den meisten Fällen nur ein Block benötigt wird. Bei einem
 Aufruf kann somit auf einen Parameter verzichtet werden. Die Laufzeitfunktion ist im schlechtesten Fall $T(n)=9$.
 Sowohl \autoref{alg:edgelpool-getmemorypools} als auch \autoref{alg:edgelpool-getmemorypool} haben eine konstante
 Laufzeit.

Um \glspl{edgel} in einem Block zu speichern, wird \autoref{alg:edgelpool-addedgel} verwendet.
\begin{algorithm}[!ht]
\caption{\textproc{addEdgel}}
\label{alg:edgelpool-addedgel}
\begin{algorithmic}[1]
	\Require $p,e$
	\If{$\lnot p$}
	\label{alg:edgelpool-addedgel-validpointer-start}
		\State \textbf{return}
	\EndIf
	\label{alg:edgelpool-addedgel-validpointer-end}
	\If{$\lnot \left(\mathit{p.count} < N\right)$}
	\label{alg:edgelpool-addedgel-checkspace-start}
		\State \textbf{return} \Comment Speicher voll
	\EndIf
	\label{alg:edgelpool-addedgel-checkspace-end}
	\State $c \gets \mathit{p.count}$
	\label{alg:edgelpool-addedgel-add-start}
	\State $\mathit{p.data}[c] \gets e$
	\State $\mathit{p.count} \gets c + 1$
	\label{alg:edgelpool-addedgel-add-end}
\end{algorithmic}
\end{algorithm}

Der Algorithmus benötigt einen Zeiger $p$ auf einen Speicherblock und einen \gls{edgel} $e$. In Zeile
 \ref{alg:edgelpool-addedgel-validpointer-start}--\ref{alg:edgelpool-addedgel-validpointer-end} wird geprüft, ob der
 Zeiger auf eine Adresse verweist. Falls $p$ null ist, wird der Algorithmus verlassen. In Zeile
 \ref{alg:edgelpool-addedgel-checkspace-start}--\ref{alg:edgelpool-addedgel-checkspace-end} wird geprüft, ob im Array
 genügend Platz für einen weiteren Eintrag vorhanden ist. Die Größe von $N$ Einträgen richtet sich nach der in
 \autoref{alg:datastructure-edgelpool} festgelegten Arraygröße $N$. Wenn genügend Platz vorhanden ist, wird in Zeile
 \ref{alg:edgelpool-addedgel-add-start}--\ref{alg:edgelpool-addedgel-add-end} der \gls{edgel} $e$ an die freie
 Position $c$ geschrieben. Danach wird $\mathit{count}$ inkrementiert. Das Hinzufügen eines \glspl{edgel} ist konstant
 und die Laufzeitfunktion entspricht im schlechtesten Fall $T(n) = 12$.

Die Position eines \glspl{edgel} kann mit \autoref{alg:edgelpool-edgelposition} gesucht werden.
\begin{algorithm}[!ht]
\caption{\textproc{edgelPosition}}
\label{alg:edgelpool-edgelposition}
\begin{algorithmic}[1]
	\Require $p,e$
	\If{$\lnot p$}
	\label{alg:edgelpool-edgelposition-validpointer-start}
		\State \textbf{return} $- 1$
	\EndIf
	\label{alg:edgelpool-edgelposition-validpointer-end}
	\State $n \gets \mathit{p.count}$
	\label{alg:edgelpool-edgelposition-count}
	\State $ \mathit{sx} \gets \mathit{e.slope.x}$
	\label{alg:edgelpool-edgelposition-e-start}
	\State $ \mathit{sy} \gets \mathit{e.slope.y}$
	\State $ x \gets \mathit{e.coordinate.x}$
	\State $ y \gets \mathit{e.coordinate.y}$
	\label{alg:edgelpool-edgelposition-e-end}
	\For{$i \gets 0$ \textbf{to} $i < n$}
	\label{alg:edgelpool-edgelposition-search-start}
		\State $c \gets \mathit{p.data}[i]$
		\If{$\mathit{sx} = \mathit{c.slope.x} \land \mathit{sy} = \mathit{c.slope.y} \land x =
		 \mathit{c.coordinate.x} \land y = \mathit{c.coordinate.y}$}
		\label{alg:edgelpool-edgelposition-isequal-start}
			\State \textbf{return} $i$
			\label{alg:edgelpool-edgelposition-returni}
		\EndIf
		\label{alg:edgelpool-edgelposition-isequal-end}
		\State $i \gets i + 1$
	\EndFor
	\label{alg:edgelpool-edgelposition-search-end}
	\State \textbf{return} $- 1$
	\label{alg:edgelpool-edgelposition-returnerror}
\end{algorithmic}
\end{algorithm}

Dazu wird der Zeiger $p$ auf den Speicherblock und das zu suchende \gls{edgel} übergeben. In Zeile
 \ref{alg:edgelpool-edgelposition-validpointer-start}--\ref{alg:edgelpool-edgelposition-validpointer-end} wird
 überprüft, ob der Zeiger $p$ auf einen gültigen Speicherblock verweist. In Zeile
 \ref{alg:edgelpool-edgelposition-count} wird die Anzahl der eingetragenen \gls{edgel} ausgelesen. In Zeile
 \ref{alg:edgelpool-edgelposition-e-start}--\ref{alg:edgelpool-edgelposition-e-end} werden die Daten des \gls{edgel}
 ausgelesen. Die Suche des \gls{edgel} erfolgt in Zeile
 \ref{alg:edgelpool-edgelposition-search-start}--\ref{alg:edgelpool-edgelposition-search-end}. Dazu wird in Zeile
 \ref{alg:edgelpool-edgelposition-isequal-start}--\ref{alg:edgelpool-edgelposition-isequal-end} ein \gls{edgel} an der
 aktuellen Position $i$ mit den lokalen Variablen verglichen. Stimmen die Werte überein, wird die Position $i$
 zurückgegeben (Zeile \ref{alg:edgelpool-edgelposition-returni}). Andernfalls wird $i$ inkrementiert und die Suche
 fortgesetzt. Wenn das \gls{edgel} nicht im Speicherblock hinterlegt ist, wird in Zeile
 \ref{alg:edgelpool-edgelposition-returnerror} ein Fehlerwert zurückgegeben. Die Laufzeit des Algorithmus ist im besten
 Fall konstant, wenn das zu suchende \gls{edgel} an der ersten Position gespeichert ist. Im schlimmsten Fall, wenn ein
 \gls{edgel} nicht gefunden wird, entspricht die Laufzeitfunktion $T(n) = 21n + 18 $ für
 $n = \text{Anzahl der \gls{edgel}}$. Die Wachstumsrate ist $21n + 18 = \Theta(n)$ für $c_{1}=20$, $c_{2}=21$ und
 $n_{0} = 18$.

\glspl{edgel} werden mittels \autoref{alg:edgelpool-getedgel} aus einem Speicherblock gelesen.
\begin{algorithm}[!ht]
\caption{\textproc{getEdgel}}
\label{alg:edgelpool-getedgel}
\begin{algorithmic}[1]
	\Require $p,i$
	\If{$\lnot p$}
	\Cost{$c_{1}$}{$1$}
	\label{alg:edgelpool-getedgel-validpointer-start}
		\State \textbf{return}
		\Cost{$c_{2}$}{$1$}
	\EndIf
	\label{alg:edgelpool-getedgel-validpointer-end}
	\State $c \gets \mathit{p.count}$
	\Cost{$c_{4}$}{$2$}
	\If{$\lnot \left(c > i\right)$}
	\Cost{$c_{5}$}{$2$}
	\label{alg:edgelpool-getedgel-validrange-start}
		\State \textbf{return}
		\Cost{$c_{6}$}{$1$}
	\EndIf
	\label{alg:edgelpool-getedgel-validrange-end}
	\State \textbf{return} $\mathit{p.data}[i]$
	\Cost{$c_{8}$}{$2$}
	\label{alg:edgelpool-getedgel-returnedgel}
\end{algorithmic}
\end{algorithm}
Dazu wird der Zeiger $p$ auf den Speicherblock und der Index $i$ übergeben. In Zeile
 \ref{alg:edgelpool-getedgel-validpointer-start}--\ref{alg:edgelpool-getedgel-validpointer-end} wird geprüft, ob es
 sich um einen gesetzten Zeiger handelt. Anschließend wird in Zeile
 \ref{alg:edgelpool-getedgel-validrange-start}--\ref{alg:edgelpool-getedgel-validrange-end} geprüft, ob der Index $i$
 innerhalb des gespeicherten Bereichs der \glspl{edgel} liegt. Danach wird in Zeile
 \ref{alg:edgelpool-getedgel-returnedgel} der Wert des \glspl{edgel} an Position $i$ zurückgegeben. Der Zugriff auf
 einen \gls{edgel} ist konstant. Die Laufzeitfunktion ist im schlechtesten Fall $T(n) = 7$.

Damit \glspl{edgel} aus einem Array entfernt werden können, wird \autoref{alg:edgelpool-removeedgel} verwendet.
\begin{algorithm}[!ht]\small
\caption{\textproc{removeEdgel}}
\label{alg:edgelpool-removeedgel}
\begin{algorithmic}[1]
	\Require $p,e$
	\State $\mathit{position} \gets$ \Call{edgelPosition}{$p,e$}
	\Cost{$c_{1}$}{$1 + 21n + 18$}
	\label{alg:edgelpool-removeedgel-position}
	\If{$\mathit{position} < 0$}
	\Cost{$c_{2}$}{$1$}
		\State \textbf{return}
		\Cost{$c_{3}$}{$1$}
		\label{alg:edgelpool-removeedgel-error}
	\EndIf
	\label{alg:edgelpool-removeedgel-best-end}
	\State $c \gets \mathit{p.count}$
	\Cost{$c_{5}$}{$2$}
	\If{$c > \mathit{position} + 1$}
	\Cost{$c_{6}$}{$2$}
	\label{alg:edgelpool-removeedgel-isvalid}
		\State \Call{memmove}{$\mathit{p.data}[\mathit{position}], \mathit{p.data}[\mathit{position} + 1],
		 \left(c - \mathit{position} + 1\right) \cdot \textproc{sizeof}(e)$}
		\Cost{$c_{7}$}{$9$}
		\label{alg:edgelpool-removeedgel-memmove}
	\EndIf
	\State $\mathit{p.count} \gets \mathit{p.count} - 1$
	\Cost{$c_{9}$}{$4$}
\end{algorithmic}
\end{algorithm}
Es wird der Zeiger $p$ auf einen Speicherblock und das zu löschenden \glspl{edgel} übergeben. In Zeile
 \ref{alg:edgelpool-removeedgel-position} wird mit \textproc{edgelPosition} (\autoref{alg:edgelpool-edgelposition}) die
 Position des \gls{edgel} gesucht. Falls das \gls{edgel} nicht gefunden wurde, wird in Zeile
 \ref{alg:edgelpool-removeedgel-error} das Verfahren abgebrochen. Ansonsten gibt es zwei zu behandelnde Fälle um ein
 \gls{edgel} zu löschen. Das \gls{edgel} liegt
\begin{enumerate}
	\item nicht am Ende des Arrays oder \label{removeedgel-worst}
	\item liegt am Ende des Arrays. \label{removeedgel-best}
\end{enumerate}
Bei \autoref{removeedgel-best} muss lediglich $\mathit{count}$ dekrementiert werden um auf den vorigen Wert zu
 verweisen (Vgl. \autoref{fig:decrementcounter}). Das dekrementieren der Zählvariable $\mathit{p.count}$ ist eine
 Zuweisung in konstanter Zeit.
\begin{figure}[!ht]
	\centering
	\subfigure[]{
		\input{resources/Memory-Decrement-Before.pdf_tex}
		\label{fig:decrementcounter-before}
	}
	\subfigure[]{
		\input{resources/Memory-Decrement-After.pdf_tex}
		\label{fig:decrementcounter-after}
	}
	\caption{Dekrementieren von $\mathit{count}$. In \subref{fig:decrementcounter-before} soll Position $i$ gelöscht
	 werden. $c$ verweist auf die nächste freie Speicherstelle. In \subref{fig:decrementcounter-after} wird $c$
	 dekrementiert und verweist auf die neue freie Speicherstelle.}
	\label{fig:decrementcounter}
\end{figure}
Bei \autoref{removeedgel-worst} wird das Array an der Position $\mathit{position}$ geteilt und der Wertebereich von
 $[\mathit{position}+1 \dotsc \mathit{position}-n]$ wird an die Position $\mathit{position}$ verschoben
 (Vgl. \autoref{fig:memmove}).
\begin{figure}[!ht]
	\centering
	\subfigure[]{
		\input{resources/Memory-Move-Before.pdf_tex}
		\label{fig:memmove-before}
	}
	\subfigure[]{
		\input{resources/Memory-Move-After.pdf_tex}
		\label{fig:memmove-after}
	}
	\caption{Verschieben des Speicherinhalts. In \subref{fig:memmove-before} soll Position $i$ gelöscht werden. Die
	 grau schattierten Einträge werden von ihrer Position an Position $i$ verschoben \subref{fig:memmove-after}.}
	\label{fig:memmove}
\end{figure}
In Zeile \ref{alg:edgelpool-removeedgel-memmove} gibt der Operator \textproc{sizeof}($e$) die Speichergröße eines
 \glspl{edgel} an, welche zum verschieben der Daten notwendig ist. Mit $c - \mathit{position} + 1$ wird die Anzahl der
 zu verschiebenden Einträge ermittelt. Im worst-case werden $N-1$ Einträge an Position $0$ des Arrays verschoben.

Um die Laufzeit der Funktion \textproc{memmove} zu bestimmen, wurde ein Testprogramm geschrieben, dass die Zeit misst,
\label{sub:datenstruktur-edgels-memmove}
 die benötigt wird, um Einträge zu verschieben. Anhand der Daten wurde mittels einer Regressionsanalyse untersucht, ob
 die gemessenen Daten einen linearen Zusammenhang aufweisen. Die erfassten $2200$ Datenpunkte wurden nach dem Vorbild
 von \textproc{time}\footcite{time-1} ermittelt um Real-, User- und Sys-Zeit zu bestimmen. Dabei wurde ein Bereich von
 $4$ Byte bis $8388608$ Byte $= 8$ MByte als Eingabe für \textproc{memmove} verwendet. Aus User- und Sys-Zeit wurde
 die CPU-Zeit bestimmt, die zur Analyse benutzt wurde. Der Korrelationskoeffizient für $X = \mathit{BYTES}$ und
 $Y = \mathit{CPU}$ beträgt $r = 0.9754288$ und das Bestimmungsmaß $r^2 = 0.9515$. Der Interzept beträgt
 $\beta_0 = -37.77\e{-06}$ (Abweichung von $58.55\e{-06}$) und die Steigung $\beta_1 = 5.885\e{-09}$
 (Abweichung von $28.35\e{-12}$). Daraus ergibt sich eine Laufzeit von $T(n) =\Theta(n)$
 (Vgl.~\autoref{eq:analyse-removeedgel-worst}).
\begin{equation}
	\label{eq:analyse-removeedgel-worst}
	\begin{split}
		y& = \beta_0 + \beta_1 \cdot n\\
		 & = -37.77\e{-06} + 5.885\e{-09} \cdot n\\
		T(n)& = -37.77\e{-06} + 5.885\e{-09} \cdot n\\
		 & = \Theta(n)
	\end{split}
\end{equation}
In \autoref{fig:regression-memmove} sind die Daten grafisch dargestellt.
\begin{figure}[!ht]
	\centering
	\input{resources/Regression-memmove.pdf_tex}
	\caption{Lineares Modell der CPU Zeit mit $2200$ Datenpunkten. In der Darstellung sind die Konfidenzintervalle für
	 $95\%$ (grüne Linie) und für $99\%$ (rote Linie) für die vorhergesagten Werte enthalten. Die Regressionsgerade ist
	 als blaue Linie eingezeichnet.}
	\label{fig:regression-memmove}
\end{figure}

In dem Verfahren nach \citeauthor{hirzer08} ist die Menge der \gls{edgel}, und somit der Speicher, begrenzt
 (Vgl. \autoref{alg:datastructure-edgelpool}). In der Implementierung des Verfahrens nach \citeauthor{hirzer08} werden
 maximal $N = 8192$ \gls{edgel} gespeichert. Die Regressionsanalyse wurde mit einer Eingabemenge von $4$ \gls{edgel}
 bis $8192$ \gls{edgel} wiederholt. Dies entspricht einer Speichergröße von $64$ Byte bis $131072$ Byte. Der
 Korrelationskoeffizient $r = 0.1332313$ und das Bestimmungsmaß $r^2 = 0.01775$ zeigen, dass eine lineare Abhängigkeit
 in diesem Bereich unwahrscheinlich ist. Wie in \autoref{fig:regression-memmove2} zusehen ist, sind Mittelwert und
 Median parallel zur $x$-Achse. Somit ist die Laufzeit von \textproc{memmove}, für die Untersuchung des Verfahrens nach
 Hirzer mit maximal $8192$ Einträgen, konstant.
\begin{figure}[!ht]
	\centering
	\input{resources/Regression-memmove2.pdf_tex}
	\caption{Regressionsanalyse mit $1200$ Datenpunkten. Der Mittelwert ist als grüne Linie eingezeichnet und der
	 Median als rote Linie.}
	\label{fig:regression-memmove2}
\end{figure}
Im schlechtesten Fall wird ein zu löschender \gls{edgel} in \autoref{alg:edgelpool-removeedgel} am Ende der Liste
 gefunden. Dadurch beträgt die Laufzeit $T(n) = 21n + 37$. Die Wachstumsrate ist, für $c_{1} = 20$, $c_{2} = 21$ und
 $n_{0} = 37$, $21n + 37= \Theta(n)$.

Um einen Speicherblock für einen neuen Durchlauf zu löschen, kommt \autoref{alg:edgelpool-resetmemorypool} zum Einsatz.
\begin{algorithm}[!ht]
\caption{\textproc{resetMemoryPool}}
\label{alg:edgelpool-resetmemorypool}
\begin{algorithmic}[1]
	\Require $p$
	\If{$\lnot p$}
	\label{alg:edgelpool-resetmemorypool-validpointer-start}
		\State \textbf{return}
	\EndIf
	\label{alg:edgelpool-resetmemorypool-validpointer-end}
	\State $\mathit{p.count} \gets 0$
	\label{alg:edgelpool-resetmemorypool-reset}
\end{algorithmic}
\end{algorithm}

 Als Parameter wird der Zeiger $p$ übergeben und in Zeile
 \ref{alg:edgelpool-resetmemorypool-validpointer-start}--\ref{alg:edgelpool-resetmemorypool-validpointer-end}
 überprüft. Um alle Daten als gelöscht zu markieren, wird lediglich die Zählvariable in Zeile
 \ref{alg:edgelpool-resetmemorypool-reset} auf $0$ gesetzt. Die Zuweisung erfolgt in konstanter Zeit. Die
 Laufzeitfunktion ist im schlechtesten Fall $T(n) = 3$.

Wenn ein Speicherblock nicht mehr benötigt wird, kann er mit \autoref{alg:edgelpool-freememorypool} freigegeben werden.
\begin{algorithm}[!ht]
\caption{\textproc{freeMemoryPool}}
\label{alg:edgelpool-freememorypool}
\begin{algorithmic}[1]
	\Require $p$
	\If{$\lnot p$}
	\label{alg:edgelpool-freememorypool-validpointer-start}
		\State \textbf{return}
	\EndIf
	\label{alg:edgelpool-freememorypool-validpointer-end}
	\State \Call{resetMemoryPool}{$p$}
	\label{alg:edgelpool-freememorypool-resetmemory}
	\If{$p \geq \mathit{data} \land p \leq \mathit{data} + S$}
	\label{alg:edgelpool-freememorypool-checkpointer}
		\State $\mathit{pool} \gets p$
	\EndIf
\end{algorithmic}
\end{algorithm}

 Der Zeiger $p$ wird in Zeile
 \ref{alg:edgelpool-freememorypool-validpointer-start}--\ref{alg:edgelpool-freememorypool-validpointer-end} überprüft.
 In Zeile \ref{alg:edgelpool-freememorypool-resetmemory} werden die Daten des Blocks als gelöscht markiert.
 (Vgl. \autoref{alg:edgelpool-resetmemorypool}). Im Anschluss wird in Zeile
 \ref{alg:edgelpool-freememorypool-checkpointer} überprüft, ob $p$ zu dem Array $\mathit{data}$ gehört und nicht größer
 als die definierte Speichergröße ist. Wenn der Test positiv ausfällt, wird der Zeiger $p$ zur weiteren Verwendung in
 $\mathit{pool}$ gespeichert. Das Freigeben eines Speicherblocks erfolgt in konstanter Zeit. Im schlechtesten Fall ist
 die Laufzeitfunktion $T(n) = 9$.

Die Anzahl der \glspl{edgel} in einem Speicherblock werden durch \autoref{alg:edgelpool-count} ermittelt.
\begin{algorithm}[!ht]
\caption{\textproc{getEdgelCount}}
\label{alg:edgelpool-count}
\begin{algorithmic}[1]
	\Require $p$
	\If{$\lnot p$}
	\Cost{$c_{1}$}{$1$}
	\label{alg:edgelpool-count-validpointer-start}
		\State \textbf{return}
		\Cost{$c_{2}$}{$1$}
	\EndIf
	\label{alg:edgelpool-count-validpointer-end}
	\State \textbf{return} $\mathit{p.count}$
	\Cost{$c_{4}$}{$2$}
	\label{alg:edgelpool-count-counter}
\end{algorithmic}
\end{algorithm}

Als Parameter wird der Zeiger $p$ übergeben und in Zeile
 \ref{alg:edgelpool-count-validpointer-start}--\ref{alg:edgelpool-count-validpointer-end} überprüft. Die Anzahl der
 Einträge wird in Zeile \ref{alg:edgelpool-count-counter} über die Zählvariable $\mathit{p.count}$ ermittelt. Der
 Zugriff auf die Variable, und somit die Laufzeit des Algorithmus, erfolgt in konstanter Zeit. Die Laufzeitfunktion ist
 im schlechtesten Fall $T(n) = 3$.

% subsection datenstruktur-edgels (end)


\subsubsection{Liniensegmente} % (fold)
\label{sub:datenstruktur-liniensegmente}

Die Datenstruktur eines Liniensegments und die Methoden zum hinzufügen, löschen und freigeben des Speichers sind nach
 dem Vorbild des Edgelspeichers aufgebaut. Die Datenstruktur eines Liniensegments ist in
 \autoref{alg:datastructure-linesegment} definiert.
\begin{algorithm}[!ht]
\caption{\textproc{lineSegment}}
\label{alg:datastructure-linesegment}
	\begin{algorithmic}[1]
		\State $\mathit{start}$
		\State $\mathit{end}$
		\State $\mathit{slope}$
		\State $\mathit{supportCount}$
		\State $\mathit{remove}$
		\State $\mathit{startCorner}$
		\State $\mathit{endCorner}$
		\State $\mathit{support}[\mathit{MAXEDGELS}]$
	\end{algorithmic}
\end{algorithm}

Eine Linie besteht aus den \glspl{edgel} $\mathit{start}$ und $\mathit{end}$, die den Start- und Endpunkt der Linie
 darstellen. Die Variable $\mathit{slope}$ enthält die Orientierung des Liniensegments, während die Variable
 $\mathit{supportCount}$ die Anzahl der unterstützenden \glspl{edgel} der Linie speichert. $\mathit{remove}$,
 $\mathit{startCorner}$ und $\mathit{endCorner}$ sind boolesche Variablen. $\mathit{remove}$ dient im späteren Verlauf
 zur Erkennung, ob ein Liniensegment gelöscht werden muss. Wenn eine Linie einen Eckpunkt am Anfang oder am Ende
 besitzt, wird dies in den Variablen $\mathit{startCorner}$ und $\mathit{endCorner}$ festgehalten.Die letzte Variable
 $\mathit{support}$ dient zur Speicherung von \glspl{edgel}, die eine Linienhypothese unterstützen. Die Lese- und
 Schreibzugriffe auf die Datenstruktur ist konstant.

Mit \autoref{alg:linesegmentaddedgel} wird ein Unterstüzungsedgel zu einem Liniensegment hinzugefügt. Das Verfahren
 benötigt dazu das Liniensegment $l$, den \gls{edgel} $e$ und die Position $\mathit{pos}$, an die der \gls{edgel}
 gespeichert wird.
\begin{algorithm}[!ht]
\caption{\textproc{addEdgel}}
\label{alg:linesegmentaddedgel}
\begin{algorithmic}[1]
	\Require $l, e, \mathit{pos}$
	\If{$\mathit{pos} > \mathit{MAXEDGELS} - 1$}
	\label{alg:linesegmentaddedgel-hasvalidrange}
		\State \textbf{return}
		\label{alg:linesegmentaddedgel-notvalidrange}
	\EndIf
	\State $\mathit{l.support}[\mathit{pos}] \gets e$
	\label{alg:linesegmentaddedgel-storeedgel}
\end{algorithmic}
\end{algorithm}

In Zeile \ref{alg:linesegmentaddedgel-hasvalidrange} wird überprüft, ob genügend Speicherplatz für ein \gls{edgel} zur
 Verfügung steht. Falls dem nicht so ist, wird in Zeile \ref{alg:linesegmentaddedgel-notvalidrange} das Verfahren
 beendet. Andernfalls, wenn genügend Speicherplatz vorhanden ist, wird in Zeile
 \ref{alg:linesegmentaddedgel-storeedgel} der \gls{edgel} im Liniensegment gespeichert. Die Laufzeit des Verfahrens ist
 konstant.

Die Methode \textproc{isOrientationCompatible} untersucht, ob zwei Liniensegmente $\mathit{left}$ und $\mathit{right}$
 fast parallel zueinander stehen (\autoref{alg:linesegmentisorientationcompatible}).
\begin{algorithm}[ht]
\caption{\textproc{isOrientationCompatible}}
\label{alg:linesegmentisorientationcompatible}
\begin{algorithmic}[1]
	\Require $\mathit{left}, \mathit{right}$
	\State \textbf{return} \Call{dotProduct}{$\mathit{left.slope}, \mathit{right.slope}$} $> 0.92$
\end{algorithmic}
\end{algorithm}

Dazu wird mithilfe von \textproc{dotProduct} die Orientierung berechnet. Wenn die Orientierung im Bereich von
 $(0.92,1]$ liegt, wird als Ergebnis wahr zurückgeliefert. Das bedeutet, dass die Orientierung der Linien im Bereich
 von $0^\circ$ bis $\sim 23^\circ$ liegt und die Linien als parallel betrachtet werden. Ansonsten wird als Ergebnis
 falsch zurückgegeben, was bedeutet, dass die Linien nicht parallel sind. Die Laufzeit von
 \autoref{alg:linesegmentisorientationcompatible} ist konstant.

Mit \autoref{alg:isedgelnearline} wird der Abstand eines \gls{edgel} zu einem Liniensegment berechnet.
\begin{algorithm}[!ht]\small
\caption{\textproc{isEdgelNearLine}}
\label{alg:isedgelnearline}
\begin{algorithmic}[1]
	\Require $l,e$
	\If{$\lnot$ \Call{isCompatible}{$\mathit{l.start,e}$}}
	\Cost{$c_{1}$}{$1+8$}
	\label{alg:isedgelnearline-iscompatible}
		\State \textbf{return FALSE}
		\Cost{$c_{2}$}{$1$}
		\label{alg:isedgelnearline-notcompatible}
	\EndIf
	\State $a \gets \mathit{l.end.coordinate.x} - \mathit{l.start.coordinate.x}$
	\Cost{$c_{4}$}{$8$}
	\label{alg:isedgelnearline-distance-start}
	\State $b \gets \mathit{l.end.coordinate.y} - \mathit{l.start.coordinate.y}$
	\Cost{$c_{5}$}{$8$}
	\State $c \gets$ \Call{sqrt}{$(a \cdot a)+(b \cdot b)$}
	\Cost{$c_{6}$}{$3+1$}
	\label{alg:isedgelnearline-distance-end}
	\State $\mathit{AB1} \gets \mathit{l.end.coordinate.x} - \mathit{l.start.coordinate.x}$
	\Cost{$c_{7}$}{$8$}
	\label{alg:isedgelnearline-pointline-start}
	\State $\mathit{AC2} \gets \mathit{e.coordinate.y} - \mathit{l.start.coordinate.y}$
	\Cost{$c_{8}$}{$7$}
	\State $\mathit{AC1} \gets \mathit{e.coordinate.x} - \mathit{l.start.coordinate.x}$
	\Cost{$c_{9}$}{$7$}
	\State $\mathit{AB2} \gets \mathit{l.end.coordinate.y} - \mathit{l.start.coordinate.y}$
	\Cost{$c_{10}$}{$8$}
	\State $\mathit{crossproduct} \gets (\mathit{AB1} \cdot \mathit{AC2}) - (\mathit{AC1} \cdot \mathit{AB2})$
	\Cost{$c_{11}$}{$4$}
	\State $\mathit{distance} \gets$ \Call{ABS}{$\mathit{crossproduct}/c$}
	\Cost{$c_{12}$}{$3+2$}
	\label{alg:isedgelnearline-pointline-end}
	\State \textbf{return} $\mathit{distance} < 0.75$
	\Cost{$c_{13}$}{$2$}
	\label{alg:isedgelnearline-return}
\end{algorithmic}
\end{algorithm}

Das Verfahren benötigt dazu ein Liniensegment $l$ und ein \gls{edgel} $e$. In Zeile
 \ref{alg:isedgelnearline-iscompatible} wird überprüft, ob die Orientierung des \gls{edgel} kompatibel mit der
 Orientierung des Liniensegments ist. Wenn dies der Fall ist, wird das Verfahren fortgesetz. Andrenfalls wird das
 Verfahren in Zeile \ref{alg:isedgelnearline-notcompatible} abgebrochen. In Zeile
 \ref{alg:isedgelnearline-distance-start}--\ref{alg:isedgelnearline-distance-end} wird die Länge des Abstands der
 Endpunkte der Linie berechnet und in lokalen Variablen gespeichert. Im Anschluß wird in Zeile
 \ref{alg:isedgelnearline-pointline-start}--\ref{alg:isedgelnearline-pointline-end} der Abstand des \gls{edgel} zur
 Linie berechnet. Das Präprozessor Makro \textproc{ABS} berechnet den absoluten Betrag der Distanz in konstanter Zeit
 (Vgl. \autoref{alg:abs}).
\begin{algorithm}[!ht]\small
\caption{\textproc{ABS}}
\label{alg:abs}
\begin{algorithmic}[1]
	\Require $a$
	\If{$a < 0$}
	\Cost{$c_{1}$}{$1$}
		\State \textbf{return} $-a$
		\Cost{$c_{2}$}{$1$}
	\Else
		\State \textbf{return} $a$
		\Cost{$c_{4}$}{$1$}
	\EndIf
\end{algorithmic}
\end{algorithm}

In Zeile \ref{alg:isedgelnearline-return} wird als Rückgabewert, abhängig vom Vergleich der Distanz, wahr oder falsch
 zurückgegebn. Bleibt der Abstand des \gls{edgel} zur Linie unter $0.75$ wird wahr and die aufrufende Methode
 zurückgeben. Ansonsten, wenn der Abstand größer ist, wird falsch zurückgegeben. Die Laufzeit von
 \autoref{alg:isedgelnearline} ist konstant.

Mit \textproc{intersection} wird der Schnittpunkt zweier Linien berechnet. Dazu benötigt das Verfahren in
 \autoref{alg:linesegmenintersection} eine linke und eine rechte Linie.
\begin{algorithm}[!ht]
\caption{\textproc{intersection}}
\label{alg:linesegmenintersection}
\begin{algorithmic}[1]
	\Require $\mathit{left}, \mathit{right}$
	\State $\mathit{intersection} \gets \infty$
	\Cost{$c_{1}$}{$1$}
	\State $\mathit{x1} \gets \mathit{left.start.coordinate.x}$
	\Cost{$c_{2}$}{$4$}
	\label{alg:linesegmenintersection-var-start}
	\State $\mathit{y1} \gets \mathit{left.start.coordinate.y}$
	\Cost{$c_{3}$}{$4$}
	\State $\mathit{x2} \gets \mathit{left.end.coordinate.x}$
	\Cost{$c_{4}$}{$4$}
	\State $\mathit{y2} \gets \mathit{left.end.coordinate.y}$
	\Cost{$c_{5}$}{$4$}
	\State $\mathit{x3} \gets \mathit{right.start.coordinate.x}$
	\Cost{$c_{6}$}{$4$}
	\State $\mathit{y3} \gets \mathit{right.start.coordinate.y}$
	\Cost{$c_{7}$}{$4$}
	\State $\mathit{x4} \gets \mathit{right.end.coordinate.x}$
	\Cost{$c_{8}$}{$4$}
	\State $\mathit{y4} \gets \mathit{right.end.coordinate.y}$
	\Cost{$c_{9}$}{$4$}
	\label{alg:linesegmenintersection-var-end}
	\State $\mathit{numerator} \gets \bigl((\mathit{x4} - \mathit{x3}) \cdot (\mathit{y1} - \mathit{y3})\bigr)
	 - \bigl((\mathit{y4} - \mathit{y3}) \cdot (\mathit{x1} - \mathit{x3})\bigr)$
	\Cost{$c_{10}$}{$8$}
	\label{alg:linesegmenintersection-intersect-start}
	\State $\mathit{denumerator} \gets \bigl((\mathit{y4} - \mathit{y3}) \cdot (\mathit{x2} - \mathit{x1})\bigr)
	 - \bigl((\mathit{x4} - \mathit{x3}) \cdot (\mathit{y2} - \mathit{y1})\bigr)$
	\Cost{$c_{11}$}{$8$}
	\State $\mathit{u\_a} \gets \mathit{numerator} / \mathit{denumerator}$
	\Cost{$c_{12}$}{$2$}
	\State $\mathit{intersection.x} \gets \mathit{x1} + \mathit{u\_a} \cdot (\mathit{x2} - \mathit{x1})$
	\Cost{$c_{13}$}{$5$}
	\State $\mathit{intersection.y} \gets \mathit{y1} + \mathit{u\_a} \cdot (\mathit{y2} - \mathit{y1})$
	\Cost{$c_{14}$}{$5$}
	\label{alg:linesegmenintersection-intersect-end}
	\State \textbf{return} $\mathit{intersection}$
	\Cost{$c_{15}$}{$1$}
	\label{alg:linesegmenintersection-return}
\end{algorithmic}
\end{algorithm}

 In Zeile \ref{alg:linesegmenintersection-var-start}--\ref{alg:linesegmenintersection-var-end} werden die Punkte der
 Linienkoordinaten in lokalen Variablen gespeichert. Danach wird in Zeile
 \ref{alg:linesegmenintersection-intersect-start}--\ref{alg:linesegmenintersection-intersect-end} der Schnittpunkt der
 beiden Linie berechnet und in Zeile \ref{alg:linesegmenintersection-return} an die aufrufende Methode zurückgegeben.
 Die Berechnung des Schnittpunktes zweier Linien erfolgt in konstanter Zeit.

Die Datenstruktur eines Speichervorrats für Linien in \autoref{alg:datastructure-linesegmentpool} besteht aus einem
 Array $\mathit{data}$ mit der festen Größe $N$ und einer Zählvariablen $\mathit{count}$.
\begin{algorithm}[!ht]\small
\caption{\textproc{lineSegmentPool} (Datenstruktur)}
\label{alg:datastructure-linesegmentpool}
\begin{algorithmic}[1]
	\State $\mathit{data}[N]$
	\Comment Anzahl der Einträge
	\State $\mathit{count}$
\end{algorithmic}
\end{algorithm}

Der Speichervorrat für Linien in \autoref{alg:datastructure-linesegmentpoolimplementation} besteht wiederum aus einem
 Array $\mathit{data}$ mit der Anzahl $S$ der zur Verfügung stehenden Speicherblöcke.
\begin{algorithm}[ht]
\caption{\textproc{lineSegmentPool} (Implementierung)}
\label{alg:datastructure-linesegmentpoolimplementation}
\begin{algorithmic}[1]
	\State $\mathit{data}[S]$
	\Comment Anzahl der Pools
	\State $\mathit{pool}$
\end{algorithmic}
\end{algorithm}

Der Zeiger von $\mathit{data}$ wird in der Variablen $\mathit{pool}$ gespeichert. Der Zugriff auf die Datenstruktur
 erfolgt in konstanter Zeit.

Mehrere Speicherblöcke können mit \autoref{alg:linepool-getmemorypools} angefordert werden und mit
\begin{algorithm}[!ht]
\caption{\textproc{getMemoryPools}}
\label{alg:linepool-getmemorypools}
\begin{algorithmic}[1]
	\Require $n$
	\If{$\mathit{data} + S - \mathit{pool} \geq n$}
	\Cost{$c_{1}$}{$3$}
	\label{alg:linepool-getmemorypools-checkpoolsize}
		\State $\mathit{pool} \gets \mathit{pool} + n$
		\Cost{$c_{2}$}{$2$}
		\State \textbf{return} $\mathit{pool} - n$
		\Cost{$c_{3}$}{$2$}
	\Else
		\State \textbf{return} $\mathit{NULL}$
		\Cost{$c_{5}$}{$1$}
	\EndIf
\end{algorithmic}
\end{algorithm}

 \autoref{alg:linepool-getmemorypool} wird ein Speicherblock angefordert.
\input{alg/analyse-hirzer/datastructure-linesegmentpool-getmemorypool}
Der Aufbau der Verfahren entspricht den Verfahren des Speichervorrats für \glspl{edgel}
 (Vgl. \autoref{alg:edgelpool-getmemorypools} und \autoref{alg:edgelpool-getmemorypool}). Der Zugriff erfolgt in
 konstanter Zeit.

Um eine Linie zu einem Speicherblock hinzuzufügen, wird \autoref{alg:linesegmentpool-addline} verwendet.
\begin{algorithm}[ht]
\caption{\textproc{addLineSegment}}
\label{alg:linesegmentpool-addline}
\begin{algorithmic}[1]
	\Require $p,l$
	\If{$\lnot p$}
	\label{alg:linesegmentpool-addline-validpointer-start}
		\State \textbf{return}
	\EndIf
	\label{alg:linesegmentpool-addline-validpointer-end}
	\If{$\lnot \left(p(\mathit{count}) < N\right)$}
	\label{alg:linepool-addline-checkspace-start}
		\State \textbf{return} \Comment Speicher voll
	\EndIf
	\label{alg:linesegmentpool-addline-checkspace-end}
	\State $c \gets p(\mathit{count})$
	\label{alg:linesegmentpool-addline-add-start}
	\State $p(\mathit{data}[c]) \gets l$
	\State $p(\mathit{count}) \gets c + 1$
	\label{alg:linesegmentpool-addline-add-end}
\end{algorithmic}
\end{algorithm}

Es wird ein Zeiger $p$ auf den Speicherblock, sowie eine Linie $l$ übergeben. Wenn es sich um einen gültigen Zeiger $p$
 handelt, und genügend freier Speicherplatz für eine weitere Linie vorhanden ist, wird in Zeile
 \ref{alg:linesegmentpool-addline-add-start}--\ref{alg:linesegmentpool-addline-add-end} die Linie hinzugefügt und die
 Zählvariable inkrementiert. Das Hinzufügen einer Linie ist konstant.

Zum auslesen einer Linie aus dem Speicherblock, wird \autoref{alg:linepool-getline} verwendet.
\begin{algorithm}[!ht]
\caption{\textproc{getLineSegment}}
\label{alg:linepool-getline}
\begin{algorithmic}[1]
	\Require $p,i$
	\If{$\lnot p$}
	\label{alg:linepool-getline-validpointer-start}
		\State \textbf{return}
	\EndIf
	\label{alg:linepool-getline-validpointer-end}
	\State $c \gets \mathit{p.count}$
	\If{$\lnot \left(c > i\right)$}
	\label{alg:linepool-getline-validrange-start}
		\State \textbf{return}
	\EndIf
	\label{alg:linepool-getline-validrange-end}
	\State \textbf{return} $\mathit{p.data}[i]$
	\label{alg:linepool-getline-returnline}
\end{algorithmic}
\end{algorithm}

Als Parameter werden ein Zeiger $p$ und ein Index $i$ übergeben. Der Index gibt an, welche Linie aus dem Block
 ausgelesen werden soll. In Zeile \ref{alg:linepool-getline-validrange-start} wird geprüft, ob der Index sich innerhalb
 der Grenzen der gespeicherten Linien befindet. Wenn dies der Fall ist, wird in Zeile
 \ref{alg:linepool-getline-returnline} die Linie in konstanter Zeit zurückgegeben.

Mit \autoref{alg:linepool-resetmemorypool} werden die Einträge im Speicherblock gelöscht.
\begin{algorithm}[!ht]\small
\caption{\textproc{resetMemoryPool}}
\label{alg:linepool-resetmemorypool}
\begin{algorithmic}[1]
	\Require $p$
	\If{$\lnot p$}
	\Cost{$c_{1}$}{$1$}
	\label{alg:linepool-resetmemorypool-validpointer-start}
		\State \textbf{return}
		\Cost{$c_{2}$}{$1$}
	\EndIf
	\label{alg:linepool-resetmemorypool-validpointer-end}
	\State $\mathit{p.count} \gets 0$
	\Cost{$c_{4}$}{$2$}
	\label{alg:linepool-resetmemorypool-reset}
\end{algorithmic}
\end{algorithm}

Dazu wird der Zeiger $p$ auf den Speicherblock übergeben und in Zeile
 \ref{alg:linepool-resetmemorypool-validpointer-start}--\ref{alg:linepool-resetmemorypool-validpointer-end} überprüft.
 Wenn es sich um einen gültigen Zeiger handelt, wird die Zählvariable auf $0$ gesetzt. Da es sich um einen direkten
 Zugriff handelt, erfolgt das Löschen in konstanter Zeit.

Durch \autoref{alg:linepool-freememorypool} kann ein Speicherblock wieder freigegeben werden.
\begin{algorithm}[!ht]\small
\caption{\textproc{freeMemoryPool}}
\label{alg:linepool-freememorypool}
\begin{algorithmic}[1]
	\Require $p$
	\If{$\lnot p$}
	\Cost{$c_{1}$}{$1$}
	\label{alg:linepool-freememorypool-validpointer-start}
		\State \textbf{return}
		\Cost{$c_{2}$}{$1$}
	\EndIf
	\label{alg:linepool-freememorypool-validpointer-end}
	\State \Call{resetmemorypool}{$p$}
	\Cost{$c_{4}$}{$3$}
	\label{alg:linepool-freememorypool-resetmemory}
	\If{$p \geq \mathit{data} \land p \leq \mathit{data} + S$}
	\Cost{$c_{5}$}{$4$}
	\label{alg:linepool-freememorypool-checkpointer}
		\State $\mathit{pool} \gets p$
		\Cost{$c_{6}$}{$1$}
		\label{alg:linepool-freememorypool-savepointer}
	\EndIf
\end{algorithmic}
\end{algorithm}

Dazu wird der Zeiger $p$ auf Gültigkeit geprüft. Danach wird der Speicher durch \textproc{resetMemoryPool}
 (\autoref{alg:linepool-resetmemorypool}) gelöscht. In Zeile \ref{alg:linepool-freememorypool-checkpointer} wird
 überprüft, ob der Zeiger $p$ zu dem entsprechenden Block gehört, um danach die Adresse in Zeile
 \ref{alg:linepool-freememorypool-savepointer} in $\mathit{pool}$ zu speichern. Auch hier erfolgt das Freigeben des
 Speichers wieder in konstanter Zeit.

Die Anzahl der Einträge in einem Pool werden durch \autoref{alg:linepool-count} bestimmt, indem die Zählvariable
 $\mathit{count}$ zurückgegeben wird.
\begin{algorithm}[ht]
\caption{\textproc{getLineCount}}
\label{alg:linepool-count}
\begin{algorithmic}[1]
	\Require $p$
	\If{$\lnot p$}
	\label{alg:linepool-count-validpointer-start}
		\State \textbf{return}
	\EndIf
	\label{alg:linepool-count-validpointer-end}
	\State \textbf{return} $p(\mathit{count})$
	\label{alg:linepool-count-counter}
\end{algorithmic}
\end{algorithm}

Der Zugriff auf die Variable erfolgt in konstanter Zeit.

Im Verfahren nach \citeauthor{clarke96} werden Liniensegmente nicht aus dem Speicherpool gelöscht. Darum kann auf
 einen Algorithmus zum löschen der Einträge, wie \autoref{alg:edgelpool-removeedgel} bei \glspl{edgel}, verzichtet
 werden. Alle Operationen für Linien erfolgen somit in konstanter Zeit $T(n) = \Theta(1)$.

% subsection datenstruktur-liniensegmente (end)


\subsubsection{Speicherpools} % (fold)
\label{sub:datenstruktur-speicherpools}

% subsection datenstruktur-speicherpools (end)


% subsection datenstrukturen (end)

\subsection{Linienerkennung nach \citeauthor{clarke96}} % (fold)
\label{sub:linienerkennung_nach_clarke96}

Der Algorithmus von \citeauthor{clarke96} ist in \autoref{alg:linedetection-analyze} aufgeführt. Zuerst wird die Breite
 $w$ und Höhe $h$ des Signals $I_m$ festgehalten. Die doppelte For-Schleife in Zeile
 \ref{alg:linedetection-analyze-start} bis \ref{alg:linedetection-analyze-end} unterteilt das Signal in Regionen der
 Größe $40 \times 40$ \gls{pixel}, indem die Koordinate der oberen linken Ecke berechnet wird.

\begin{algorithm}[ht]
\caption{Line Detection}
\label{alg:linedetection-analyze}
	\begin{algorithmic}[1]
		\Require $I_m$
		\State $w \gets$ \Call{width}{$I_m$} \Cost{$c_1$}{$1$}
		\State $h \gets$ \Call{height}{$I_m$} \Cost{$c_2$}{$1$}
		\For{$y \gets 0$ \textbf{to} $y < h$} \Cost{$c_3$}{$\tfrac{h}{40} + 1$}
		\label{alg:linedetection-analyze-start}
			\For{$x \gets 0$ \textbf{to} $x < w$} \Cost{$c_4$}{$\sum_{y=0}^{\tfrac{h}{40}} \left(\tfrac{w}{40} + 1\right)$}
				\State \Call{FindEdgels}{$I_m,E,x,y,40,40,w,h$} \Cost{$c_5$}{$\sum_{y = 0}^{\frac{h}{40}} \sum_{x = 0}^{\frac{w}{40}} t_y t_x$}
				\State \Call{findLineSegments}{$E,L$} \Cost{$c_6$}{$\sum_{y = 0}^{\frac{h}{40}} \sum_{x = 0}^{\frac{w}{40}} t_y t_x$}
				\State $ x \gets x + 40$ \Cost{$c_7$}{$\sum_{y = 0}^{\frac{h}{40}} \sum_{x = 0}^{\frac{w}{40}} t_y t_x$}
			\EndFor
			\State $y \gets y + 40$ \Cost{$c_9$}{$\sum_{y = 0}^{\frac{h}{40}} t_y$}
		\EndFor
		\label{alg:linedetection-analyze-end}
	\end{algorithmic}
\end{algorithm}


In \citeauthor{clarke96} ist keine Angabe zu den Abmessungen der untersuchten Signale angegeben. Auch der Grund warum
 eine Region $40 \times 40$ \gls{pixel} groß sein muss, fehlt. Zur Analyse der Videosignale verwendeten
 \citeauthor{clarke96} einen Framegrabber 2000 der eine Auflösung von 640x480px schafft. Betrachtet man

$640 mod 40 = 0$ und $480 mod 40 = 0$

ist ersichtlich, dass die Größe der Region in der Aufteilung des Bildsignals in Zusammenhang steht.

Der \autoref{alg:linedetection-analyze} ist der zentrale Algorithmus von \citeauthor{clarke96}. Der Algorithmus
 ist verantwortlich für die Unterteilung des Bildsignals in Regionen von jeweils $40 \times 40$ \gls{pixel}
 (Vgl. Zeile~\ref{alg:linedetection-analyze-start}--\ref{alg:linedetection-analyze-end}). Die Kosten des Algorithmus
 sind in \autoref{eq:linedetection-analyze1} aufgeführt. Um die Gleichung zu vereinfachen führe ich in
 \autoref{eq:linedetection-analyze2} $n = \tfrac{h}{40}$ und $k = \tfrac{w}{40}$ ein. Sowohl bei worst-case als auch
 bei best-case werden die Summen immer vollständig durchlaufen. Damit kann die Gleichung zu
 \autoref{eq:linedetection-analyze3} vereinfacht werden. Durch \autoref{eq:linedetection-analyze4} ergibt die Laufzeit
 $T(I) = \Theta(nk)$ (Vgl.~\autoref{eq:linedetection-analyze5}).

\begin{subequations}
\begin{align}
\label{eq:linedetection-analyze1}
T(I)& =
c_1
+ c_2
+ c_3 \left(\frac{h}{40} + 1\right)
+ c_4 \sum \limits_{y = 0}^{\frac{h}{40}} t_y \left(\frac{w}{40} + 1 \right)
+ c_5 \sum \limits_{y = 0}^{\frac{h}{40}} \sum \limits_{x = 0}^{\frac{w}{40}} t_y t_x\\
& \quad + c_6 \sum \limits_{y = 0}^{\frac{h}{40}} \sum \limits_{x = 0}^{\frac{w}{40}} t_y t_x
+ c_7 \sum \limits_{y = 0}^{\frac{h}{40}} \sum \limits_{x = 0}^{\frac{w}{40}} t_y t_x
+ c_9 \sum \limits_{y = 0}^{\frac{h}{40}} t_y \nonumber \\
\label{eq:linedetection-analyze2}
T(I)& =
c_1
+ c_2
+ c_3 \left(n + 1\right)
+ c_4 \sum \limits_{y = 0}^{n} t_y \left(k + 1 \right)
+ c_5 \sum \limits_{y = 0}^{n} \sum \limits_{x = 0}^{k} t_y t_x\\
& \quad + c_6 \sum \limits_{y = 0}^{n} \sum \limits_{x = 0}^{k} t_y t_x
+ c_7 \sum \limits_{y = 0}^{n} \sum \limits_{x = 0}^{k} t_y t_x
+ c_9 \sum \limits_{y = 0}^{n} t_y \nonumber \\
\label{eq:linedetection-analyze3}
T(I)& =
c_1
+ c_2
+ c_3 \left(n + 1\right)
+ c_4 \left[n \left(k + 1 \right)\right]
+ c_5 n k
+ c_6 n k
+ c_7 n k
+ c_9 n\\
\label{eq:linedetection-analyze4}
T(I)& = c_1 + c_2 + c_3 + \left(c_3 + c_4 + c_9\right) n + \left(c_4 + c_5 + c_6 + c_7\right) n k\\
\label{eq:linedetection-analyze5}
T(I)& = \Theta(nk)
\end{align}
\end{subequations}

Das Verfahren zur Bestimmung der Edgels (\autoref{alg:findedgels-horizontal}) benötigt das monochrome Bildsignal $I_m$,
 sowie die Position der oberen linken Ecke der Region, die durch oben $t$ und links $l$ definiert ist. Die Breite und
 Höhe der Region ist durch $\mathit{rw}$ und $\mathit{rh}$ angegeben. Die Abmessung des Bildsignals werden als $w$ und
 $h$ bezeichnet. Der Pointer $E$ wird zur Speicherung der gefundenen \glspl{edgel} verwendet.

\begin{algorithm}
\caption{Edgels bestimmen}
\label{alg:findedgels-horizontal}
	\begin{algorithmic}[1]
		\Require $I_m, E, t, l, \mathit{rw}, \mathit{rh}, w, h$
		\For{$y \gets t$ \textbf{to} $y < t + \mathit{rh}$}
		\label{alg:findedgels-horizontal-scanlinestart}
		\Comment Horizontale Scanlines
			\State $p_1 \gets 0$
			\State $p_2 \gets 0$
			\For{$x \gets l$ \textbf{to} $x < l + \mathit{rw}$}
			\label{alg:findedgels-horizontal-loopstart}
				\State $currentEdgel \gets$ \Call{Convolute}{$I_m,x,y,w,h$}
				\label{alg:findedgels-horizontal-convolute}
				\If{$currentEdgel > threshold$}
				\label{alg:findedgels-horizontal-foundedgel}
					\Comment Edgel gefunden
				\Else
					\State $currentEdgel \gets 0$
				\EndIf
				\If{$p_1 > 0 \land p_1 > p_2 \land p_1 > currentEdgel$}
				\label{alg:findedgels-horizontal-maxima}
					\Comment $p_1$ ist lokales Maximum
					\State $edgel \gets \infty$
					\State $edgel.x \gets x - 1$
					\State $edgel.y \gets y$
					\State $edgel.orientation \gets$ \Call{Orientation}{$x - 1, y$}
					\State $E \gets edgel$
				\EndIf
				\State $p_2 \gets p_1$
				\label{alg:findedgels-horizontal-copy-prev1}
				\State $p_1 \gets currentEdgel$
				\label{alg:findedgels-horizontal-copy-edgel}
			\EndFor
			\label{alg:findedgels-horizontal-loopend}
			\State $y \gets y + 5$
			\label{alg:findedgels-horizontal-increment}
		\EndFor
		\label{alg:findedgels-horizontal-scanlineend}
	\algstore{brkfindedgels}
	\end{algorithmic}
\end{algorithm}
\begin{algorithm}[ht]
	\caption{Edgels bestimmen (Fortsetzung)}
	\label{alg:findedgels-vertical}
	\begin{algorithmic}[1]
	\algrestore{brkfindedgels}
		\For{$x \gets l$ \textbf{to} $x < l + \mathit{rw}$}
		\Comment Vertikale Scanlines
		\label{alg:findedgels-vertical-scanlinestart}
			\State $p_1 \gets 0$
			\State $p_2 \gets 0$
			\For{$y \gets t$ \textbf{to} $y < t + \mathit{rh}$}
				\State $currentEdgel \gets$ \Call{convolute}{$I_m,x,y,w,h$}
				\If{$currentEdgel > threshold$}
					\Comment Edgel gefunden
				\Else
					\State $currentEdgel \gets 0$
				\EndIf
				\If{$p_1 > 0 \land p_1 > p_2 \land p_1 > currentEdgel$}
					\Comment $p_1$ ist lokales Maximum
					\State $edgel \gets \infty$
					\State $edgel.x \gets x$
					\State $edgel.y \gets y - 1$
					\State $edgel.orientation \gets$ \Call{orientation}{$x, y - 1$}
					\State $E \gets edgel$
				\EndIf
				\State $p_2 \gets p_1$
				\State $p_1 \gets currentEdgel$
			\EndFor
			\State $x \gets x + 5$
		\EndFor
		\label{alg:findedgels-vertical-scanlineend}
	\end{algorithmic}
\end{algorithm}


Zeile~\ref{alg:findedgels-horizontal-scanlinestart}--\ref{alg:findedgels-horizontal-scanlineend} ist für den Aufbau der
 horizontalen Scanlines verantwortlich. Die Überprüfung sorgt dafür, dass die Scanlines bis zum Ende der Region im
 Abstand von $5$ Pixeln untersucht werden. Nach der Initialisierung der Variablen wird in der Schleife von
 Zeile~\ref{alg:findedgels-horizontal-loopstart}--\ref{alg:findedgels-horizontal-loopend} jeder Pixel auf der Scanline
 untersucht. Zuerst wird in Zeile~\ref{alg:findedgels-horizontal-convolute} die Faltung mit einem Gauß-Kernel
 vorgenommen (Vgl. \autoref{alg:derivativeofgauss-horizontal}, S.~\pageref{alg:derivativeofgauss-horizontal}). Der Test
 in Zeile~\ref{alg:findedgels-horizontal-foundedgel} überprüft anschließend das Ergebnis der Faltung. Wenn der
 Schwellwert nicht überschritten wird, gibt es keinen genügend großen Anstieg des Gradienten und das Ergebnis wird auf
 $0$ gesetzt. Wird der Schwellwert überschritten, handelt es sich um einen Edgel und das Ergebnis wird in den
 Bedingungen von Zeile~\ref{alg:findedgels-horizontal-maxima} weiter untersucht, ob es sich um ein lokales Maximum
 handelt. Ein lokales Maximum bedeutet, dass ein Edgel einen größeren Gradienten besitzt als seine beiden Nachbarn.

Die Bedingung in Zeile~\ref{alg:findedgels-horizontal-maxima} wird bei der ersten Überprüfung immer fehlschlagen.
 Dadurch wird sichergestellt, dass kein Maximum an den Rändern existiert, da hier nicht genügend Nachbarn vorhanden sind
 um eine verlässliche Aussage zu treffen. Zeile~\ref{alg:findedgels-horizontal-copy-prev1} und
 Zeile~\ref{alg:findedgels-horizontal-copy-edgel} kopieren die Werte für den nächsten Durchlauf. Durch das kopieren der
 Werte werden die Nachbarn für den nächsten Durchlauf um eine Position weiterverschoben. Nur bei einem lokalen Maximum
 wird die Position des Edgels gespeichert, und seine Orientierung (Vlg. \autoref{alg:sobel},
 S.~\pageref{alg:sobel}) berechnet. Der Edgel wird in (einer Liste|einem Memorypool) zu weiteren Verarbeitung
 gespeichert.

Sind alle Pixel auf einer Scanline untersucht, wird in Zeile~\ref{alg:findedgels-horizontal-increment} die nächste
 Scanline ausgewählt. Das Verfahren wird solange wiederholt, bis alle Scanlines innerhalb der Region untersucht wurden.

\autoref{alg:findedgels-vertical} untersucht die vertikalen Scanlines in Zeile
 \ref{alg:findedgels-vertical-scanlinestart}--\ref{alg:findedgels-vertical-scanlineend} analog zu
 \autoref{alg:findedgels-horizontal} Zeile
 \ref{alg:findedgels-horizontal-scanlinestart}--\ref{alg:findedgels-horizontal-scanlineend}.

\autoref{alg:derivativeofgauss-horizontal} und \autoref{alg:derivativeofgauss-vertical} berechnen den Gradienten durch Faltung mit dem Gauß-Kernel
$\left( \begin{smallmatrix}
-3& -5& 0& 5& 3
\end{smallmatrix} \right)$
 auf der horizontalen und vertikalen Scanline. Als Parameter benötigt der Algorithmus den Pointer des monochromen
 Bildsignals $I_m$, die Position des Pixels ($x$ und $y$), sowie die Breite $w$ und Höhe $h$ von $I_m$. In Zeile
 \ref{alg:derivativeofgauss-horizontal-readstart}--\ref{alg:derivativeofgauss-horizontal-readend} werden durch die
 Funktion \textproc{getpixel} (Vgl. \autoref{alg:getpixel}, S. \pageref{alg:getpixel}) die benötigten Pixelwerte
 ausgelesen und den Variablen zugewiesen. Im Anschluss werden die Werte mit dem Gauß-Kernel
$\left( \begin{smallmatrix}
-3& -5& 0& 5& 3
\end{smallmatrix} \right)$
berechnet um den Gradienten zu bestimmen.

\begin{algorithm}[ht]
\caption{Faltung mit Gauß-Kernel (horizontale Scanline)}
\label{alg:derivativeofgauss-horizontal}
\begin{algorithmic}[1]
	\Require $I_m,x,y,w,h$
	\Ensure $-127.5 \leq value \leq 127.5$
	\State $p_1 \gets$ \Call{getPixel}{$I_m, x - 2, y, w, h$}
	\State $p_2 \gets$ \Call{getPixel}{$I_m, x - 1, y, w, h$}
	\State $p_3 \gets$ \Call{getPixel}{$I_m, x, y, w, h$}
	\State $p_4 \gets$ \Call{getPixel}{$I_m, x + 1, y, w, h$}
	\State $p_5 \gets$ \Call{getPixel}{$I_m, x + 2, y, w, h$}
	\State $value \gets 0$
	\State $value \gets value + \left( -3 \cdot p_1 \right)$
	\State $value \gets value + \left( -5 \cdot p_2 \right)$
	\State $value \gets value + \left( 0 \cdot p_3 \right)$
	\State $value \gets value + \left( 5 \cdot p_4 \right)$
	\State $value \gets value + \left( 3 \cdot p_5 \right)$
	\State $value \gets value \cdot \frac{1}{16}$
	\State \textbf{return} $value$
\end{algorithmic}
\end{algorithm}

\begin{algorithm}[ht]
\caption{Faltung mit Gauß-Kernel (vertikale Scanline)}
\label{alg:derivativeofgauss-vertical}
\begin{algorithmic}[1]
	\Require $I_m,x,y,w,h$
	\Ensure $-127.5 \leq value \leq 127.5$
	\State $p_1 \gets$ \Call{getPixel}{$I_m, x, y - 2, w, h$}
	\State $p_2 \gets$ \Call{getPixel}{$I_m, x, y - 1, w, h$}
	\State $p_3 \gets$ \Call{getPixel}{$I_m, x, y, w, h$}
	\State $p_4 \gets$ \Call{getPixel}{$I_m, x, y + 1, w, h$}
	\State $p_5 \gets$ \Call{getPixel}{$I_m, x, y + 2, w, h$}
	\State $value \gets 0$
	\State $value \gets value + \left( -3 \cdot p_1 \right)$
	\State $value \gets value + \left( -5 \cdot p_2 \right)$
	\State $value \gets value + \left( 0 \cdot p_3 \right)$
	\State $value \gets value + \left( 5 \cdot p_4 \right)$
	\State $value \gets value + \left( 3 \cdot p_5 \right)$
	\State $value \gets value \cdot \frac{1}{16}$
	\State \textbf{return} $value$
\end{algorithmic}
\end{algorithm}


Durch die Multiplikation mit $\tfrac{1}{16}$ wird sichergestellt, dass der maximale Wert

\begin{equation}
	\frac{1}{16}
	\cdot
	\begin{pmatrix}
		-3& -5& 0& 5& 3
	\end{pmatrix}
	\cdot
	\begin{pmatrix}
		0& 0& 0& 255& 255
	\end{pmatrix}
	= 127.5
\end{equation}

und der minimale Wert

\begin{equation}
	\frac{1}{16}
	\cdot
	\begin{pmatrix}
		-3& -5& 0& 5& 3
	\end{pmatrix}
	\cdot
	\begin{pmatrix}
		255& 255& 0& 0& 0
	\end{pmatrix}
	= -127.5
\end{equation}

für ein monochromes Bild eingehalten werden.

Bei genauer Betrachtung von \autoref{alg:derivativeofgauss-horizontal} und \autoref{alg:derivativeofgauss-vertical}
 fällt auf, dass der Wert $p_3$ in der Berechnung nicht [mit einfließt|vorkommt]. Dies ist darauf zurückzuführen, dass bei der
 Multiplikation des Gauß-Kernels an der dritten Stelle des Filter mit $0$ definiert ist. Eine Multiplikation mit $0$
 ergibt immer $0$ und kann somit vernachlässigt werden. Die Laufzeit von \autoref{alg:derivativeofgauss-horizontal} und
 \autoref{alg:derivativeofgauss-vertical} ist konstant.

In \autoref{alg:sobel} wird, wie in \autoref{alg:derivativeofgauss-horizontal}, mittels Faltung die Orientierung eines
 \glspl{edgel} bestimmt. Als Eingabeparameter wird das monochrome Bildsignal $I_m$, dessen Breite $w$ und Höhe $h$,
 sowie die Position des \glspl{edgel} ($x,y$) benötigt.

\begin{algorithm}[ht]
\caption{Orientierung berechnen}
\label{alg:sobel}
\begin{algorithmic}[1]
	\Require $I_m, x, y, w, h$
	\Ensure $-\pi < v \leq \pi$
	\State $p_{x_1} \gets$ \Call{getPixel}{$I_m, x -1, y - 1, w, h$}
	\State $p_{x_2} \gets$ \Call{getPixel}{$I_m, x, y - 1, w, h$}
	\State $p_{x_3} \gets$ \Call{getPixel}{$I_m, x + 1, y - 1, w, h$}
	\State $p_{x_4} \gets$ \Call{getPixel}{$I_m, x - 1, y + 1, w, h$}
	\State $p_{x_5} \gets$ \Call{getPixel}{$I_m, x, y + 1, w, h$}
	\State $p_{x_6} \gets$ \Call{getPixel}{$I_m, x + 1, y + 1, w, h$}
	\State $p_{y_1} \gets$ \Call{getPixel}{$I_m, x - 1, y - 1, w, h$}
	\State $p_{y_2} \gets$ \Call{getPixel}{$I_m, x - 1, y, w, h$}
	\State $p_{y_3} \gets$ \Call{getPixel}{$I_m, x - 1, y + 1, w, h$}
	\State $p_{y_4} \gets$ \Call{getPixel}{$I_m, x + 1, y - 1, w, h$}
	\State $p_{y_5} \gets$ \Call{getPixel}{$I_m, x + 1, y, w, h$}
	\State $p_{y_6} \gets$ \Call{getPixel}{$I_m, x + 1, y + 1, w, h$}
	\State $g_x \gets 0$
	\State $g_y \gets 0$
	\State $g_x \gets g_x + p_{x_1}$
	\State $g_x \gets g_x + \left(p_{x_2} \cdot 2\right)$
	\State $g_x \gets g_x + p_{x_3}$
	\State $g_x \gets g_x - p_{x_4}$
	\State $g_x \gets g_x - \left(p_{x_5} \cdot 2\right)$
	\State $g_x \gets g_x - p_{x_6}$
	\State $g_y \gets g_y + p_{y_1}$
	\State $g_y \gets g_y + \left(p_{y_2} \cdot 2\right)$
	\State $g_y \gets g_y + p_{y_3}$
	\State $g_y \gets g_y - p_{y_4}$
	\State $g_y \gets g_y - \left(p_{y_5} \cdot 2\right)$
	\State $g_y \gets g_y - p_{y_6}$
	\State $v \gets \arctan{\left(gy, gx\right)}$
	\State \textbf{return} $v$
\end{algorithmic}
\end{algorithm}


In Zeile~\ref{alg:sobel-readstart}--\ref{alg:sobel-readend} werden die Pixelwerte ausgelesen und den Variablen
 zugewiesen. In Zeile~\ref{alg:sobel-convolutestart}--\ref{alg:sobel-convoluteend} erfolgt die Faltung mit dem
 Sobel-Operator\footcite[Vgl.][S.~120--123]{burger05}, dessen Filter

\begin{subequations}
\begin{align}
	H_x =&
	\begin{pmatrix}
		1& 0& -1\\
		2& 0& -2\\
		1& 0& -1
	\end{pmatrix}
\end{align}
\begin{align}
	H_y =&
	\begin{pmatrix}
		1& 2& 1\\
		0& 0& 0\\
		-1& -2& -1
	\end{pmatrix}
\end{align}
\end{subequations}

den Gradienten $G_x$ und $G_y$ bestimmen. Wie in \autoref{alg:derivativeofgauss-horizontal} werden Multiplikationen von
 $0$-Werten des Filters vernachlässigt. Mit

\begin{equation}
	\label{eq:orientation}
	\Phi(x,y) = \arctan{\left(\tfrac{G_y}{G_x}\right)}
\end{equation}

wird die Orientierung in Zeile \autoref{alg:sobel-arctan} berechnet. Die Orientierung unterscheidet sich um $180^\circ$
 wenn anstatt von Weiß nach Schwarz ein Verlauf von Schwarz nach Weiß erfolgt. Das Ergebnis liegt im Bereich
 $-\pi < v \leq \pi$. $\arctan$ in \autoref{eq:orientation} kann in C durch \textproc{atan2} zur Berechnung verwendet
 werden. Die Laufzeit von \autoref{alg:sobel} ist konstant.

\begin{algorithm}[ht]
\caption{Auffinden von Liniensegmenten}
\label{alg:findlinesegments}
\begin{algorithmic}[1]
	\Require $E, L$
	\State $linecount \gets 0$
	\State $edgelcount \gets$ \Call{numberofedgels}{E}
	\While{$linecount > minEdgels \land edgelcount > minEdgels$}
	\State $linecount \gets numberoflinecount$
	\State $edgelcount \gets numberofedgels$
	\EndWhile
\end{algorithmic}
\end{algorithm}

% subsection linienerkennung_nach_clarke96 (end)

% section hirzer (end)

\section{Verfahren nach Hirzer} % (fold)
\label{sec:hirzer}

In diesem Kapitel wird das Verfahren von \citeauthor{hirzer08} untersucht. Die im Rahmen dieser Arbeit entstandene
 Implementierung verwendet eigene Datentypen, die in den ersten Abschnitten beschrieben werden. Danach wird das
 Verfahren von \citeauthor{clarke96} beschrieben, auf dem das Verfahren von \citeauthor{hirzer08} aufgebaut ist. Danach
 werden die Optimierungen von \citeauthor{hirzer08} in \autoref{sub:linienerkennung_nach_hirzer08} untersucht. In den
 restlichen Abschnitten des Kapitels werden die Verfahren zur Linien Erweiterung (Line Extension) und Quadraterkennung
 (Quadrangle Detection) beschrieben.

\subsection{Datenstrukturen} % (fold)
\label{sub:datenstrukturen}

In diesem Abschnitt werden die verwendeten Datenstrukturen für Vektoren, \gls{edgel}, Liniensegmente und Marken
 beschrieben und analysiert.

\subsubsection{Vektor} % (fold)
\label{sub:vektor}

Die meisten Datenstrukturen in der Implementierung des Verfahrens nach \citeauthor{hirzer08}, benötigen als Basis die
 Datenstruktur $\textproc{vector}$. Bei der Datenstruktur handelt es sich um eine Vektor in $\mathbb{R}^2$ und besteht
 aus den beiden Variablen $x$ und $y$ (Vgl. \autoref{alg:vector}).
\begin{algorithm}[ht]
\caption{\textproc{vector}}
\label{alg:vector}
	\begin{algorithmic}[1]
		\State $x$
		\State $y$
	\end{algorithmic}
\end{algorithm}

Operationen mit der Datenstruktur \textproc{vector} umfassen

\begin{enumerate}
	\item Addition,
	\item Subtraktion,
	\item Skalarprodukt,
	\item Quadratische Länge eines Vektors,
	\item Länge eines Vektors und
	\item Normalisieren von Vektoren.
\end{enumerate}

Auf die Variablen $x$ und $y$ der Datenstruktur kann direkt zugegriffen werden, sowohl lesend als auch schreibend. Zum
 schreiben der Variablen kann auch \autoref{alg:vectorsetcoordinate} verwendet werden.
\begin{algorithm}[!ht]
\caption{\textproc{vectorSetCoordinate}}
\label{alg:vectorsetcoordinate}
	\begin{algorithmic}[1]
		\Require $\mathit{vector}, x, y$
		\State $\mathit{vector.x} \gets x$
		\Cost{$c_{1}$}{$2$}
		\State $\mathit{vector.y} \gets y$
		\Cost{$c_{2}$}{$2$}
	\end{algorithmic}
\end{algorithm}

\textproc{vectorSetCoordinate} benötigt einen Vektor, dem die Werte $x$ und $y$ hinzugefügt werden. Der Zugriff auf die
 Datenstruktur \textproc{vector} ist konstant. Die Laufzeitfunktion ist $T(n) = 4$.

\paragraph{Addition:} % (fold)
\label{par:addition}

Die Addition von zwei Vektoren ist in \autoref{alg:vectoradd} aufgeführt.
\begin{algorithm}[!ht]
\caption{\textproc{vectorAdd}}
\label{alg:vectoradd}
	\begin{algorithmic}[1]
		\Require $\mathit{left}, \mathit{right}$
		\State $\mathit{vector} \gets \infty$
		\State \Call{vectorSetCoordinate}{$\mathit{vector},\mathit{left.x} + \mathit{right.x}, \mathit{left.y} + \mathit{right.y}$}
		\State \textbf{return} $\mathit{vector}$
	\end{algorithmic}
\end{algorithm}

Das Verfahren benötigt zwei Vektoren $\mathit{left}$ und $\mathit{right}$, die addiert werden sollen. Dazu wird ein
 neuer Vektor $\mathit{vector}$ initialisiert, der mit Hilfe von \textproc{vectorSetCoordinate} die addierten Werte von
 $\mathit{left}$ und $\mathit{right}$ zugewiesen bekommt. Danach wird der Vektor an die aufrufenden Funktion
 zurückgegeben. Die Laufzeitfunktion ist $T(n) = 12$.

% paragraph addition (end)

\paragraph{Subtraktion:} % (fold)
\label{par:subtraktion}

Die Subtraktion von zwei Vektoren (\autoref{alg:vectorsubtract}) ähnelt dem Verfahren der Addition.
\begin{algorithm}[!ht]
\caption{\textproc{vectorSubtract}}
\label{alg:vectorsubtract}
	\begin{algorithmic}[1]
		\Require $\mathit{left}, \mathit{right}$
		\State $\mathit{vector} \gets \infty$
		\Cost{$c_{1}$}{$1$}
		\State \Call{vectorSetCoordinate}{$\mathit{vector},\mathit{left.x} - \mathit{right.x}, \mathit{left.y} -
		 \mathit{right.y}$}
		\Cost{$c_{2}$}{$6 + 4$}
		\State \textbf{return} $\mathit{vector}$
		\Cost{$c_{3}$}{$1$}
	\end{algorithmic}
\end{algorithm}

Auch hier wird durch die Parameter $\mathit{left}$ und $\mathit{right}$ ein Vektor initialisiert und an die
 aufrufende Funktion zurückgeliefert. Die Laufzeitfunktion ist ebenfalls $T(n) = 12$.

% paragraph subtraktion (end)

\paragraph{Skalarprodukt:} % (fold)
\label{par:skalarprodukt}

Das Skalarprodukt der Vektoren $\mathit{left}$ und $\mathit{right}$ wird durch \autoref{alg:vectordotproduct} direkt
 zurückgegeben.
\begin{algorithm}[!ht]
\caption{\textproc{dotProduct}}
\label{alg:vectordotproduct}
	\begin{algorithmic}[1]
		\Require $\mathit{left}, \mathit{right}$
		\State \textbf{return} $\left(\mathit{left.x} \cdot \mathit{right.x}\right) + \left(\mathit{left.y}
		 \cdot \mathit{right.y}\right)$
		\Cost{$c_{1}$}{7}
	\end{algorithmic}
\end{algorithm}

Die Laufzeitfunktion ist $T(n) = 7$.

% paragraph skalarprodukt (end)

\paragraph{Länge eines Vektors:} % (fold)
\label{par:länge_eines_vektors}

Die quadratische Länge eines Vektors (\autoref{alg:vectorsquaredlength}) benötigt als Parameter einen Vektor
 $\mathit{vector}$ und berechnet durch direkten Zugriff auf die Datenstruktur die quadratische Länge. Die
 Laufzeitfunktion ist $T(n) = 7$.
\begin{algorithm}[ht]
\caption{\textproc{squaredLength}}
\label{alg:vectorsquaredlength}
	\begin{algorithmic}[1]
		\Require $\mathit{vector}$
		\State \textbf{return} $\left(\mathit{vector.x} \cdot \mathit{vector.x}\right) + \left(\mathit{vector.y} \cdot \mathit{vector.y}\right)$
	\end{algorithmic}
\end{algorithm}

Das Verfahren \textproc{length} (\autoref{alg:vectorlength}) nutzt \autoref{alg:vectorsquaredlength} zur Berechnung der
 Länge eines Vektors.
\begin{algorithm}[!ht]\small
\caption{\textproc{length}}
\label{alg:vectorlength}
	\begin{algorithmic}[1]
		\Require $\mathit{vector}$
		\State \textbf{return} \Call{sqrt}{\textproc{squaredlength}$(\mathit{vector})$}
		\Cost{$c_{1}$}{$1 + 1 + \Theta(1)$}
	\end{algorithmic}
\end{algorithm}

Die Laufzeit von \autoref{alg:vectorlength} ist abhängig von der Funktion \textproc{sqrt}\footcite[Vgl.][]{sqrtf},
 deren Laufzeit durch ein Testprogramm ermittlet wurde. Dazu wurden $300$ Datenpunkte erfasst und grafisch dargestellt.
 In \autoref{fig:regression-sqrtf} ist zu erkennen, dass die gemessenen Werte keinen linearen Zusammenhang
\begin{figure}[!ht]
	\centering
	\input{resources/Regression-sqrtf.pdf_tex}
	\caption{Regressionsanalyse von \textproc{sqrt}. $300$ Datenpunkte in logarithmischer Darstellung ($x$-Achse). Der
	 Mittelwert der Daten ist als grüne Linie eingezeichnet.}
	\label{fig:regression-sqrtf}
\end{figure}
 aufweisen. Der Median und Mittelwert liegen jeweils bei $0$. Damit ist die Laufzeit von \textproc{sqrt}, und somit von
 \textproc{length}, konstant. Die Laufzeitfunktion ist $T(n) = 9$.

% paragraph länge_eines_vektors (end)

\paragraph{Normalisieren von Vektoren:} % (fold)
\label{par:normalisieren_von_vektoren}

Das Verfahren zum normalisieren von Vektoren ist in \autoref{alg:vectornormalized} beschrieben und benötigt als
 Parameter einen Vektor.
\begin{algorithm}[!ht]\small
\caption{\textproc{normalized}}
\label{alg:vectornormalized}
	\begin{algorithmic}[1]
		\Require $\mathit{vector}$
		\State $\mathit{invertedLength} \gets 1 /$ \Call{length}{$\mathit{vector}$}
		\Cost{$c_{1}$}{$2 + 9$}
		\State $\mathit{vector.x} \gets \mathit{vector.x} \cdot \mathit{invertedLength}$
		\Cost{$c_{2}$}{$4$}
		\State $\mathit{vector.y} \gets \mathit{vector.y} \cdot \mathit{invertedLength}$
		\Cost{$c_{3}$}{$4$}
	\end{algorithmic}
\end{algorithm}

Durch \textproc{length} (\autoref{alg:vectorlength}) wird die Länge des Vektors berechnet und direkt im Parameter
 gespeichert. Die Laufzeit von \autoref{alg:vectornormalized} ist konstant. Die Laufzeitfunktion ist $T(n) = 19$.

% paragraph normalisieren_von_vektoren (end)

Die vorgestellten Operationen (\autoref{alg:vectorsetcoordinate}--\autoref{alg:vectornormalized}) arbeiten in konstanter
 Laufzeit durch den direkten Zugriff auf die Variablen der Datenstruktur \textproc{vector}.

% subsubsection vektor (end)


\subsubsection{Edgels} % (fold)
\label{sub:datenstruktur-edgels}

Die Datenstruktur eines \glspl{edgel} besteht aus der $x$- und $y$-Koordinate und der Orientierung
 (Vgl. \autoref{alg:datastructure-edgel}). Lese- und Schreibzugriffe auf die Elemente eines \glspl{edgel} sind konstant.

\begin{algorithm}[ht]
\caption{Datenstruktur eines edgels}
\label{alg:datastructure-edgel}
	\begin{algorithmic}[1]
		\State $x$
		\State $y$
		\State $o$
		\Comment Orientierung
	\end{algorithmic}
\end{algorithm}


Der Vergleich, ob \glspl{edgel} kompatibel sind, wird mit \autoref{alg:compatibleedgel} bewerkstelligt. Als Parameter
 werden zwei zu vergleichende \gls{edgel} $e_1$ und $e_2$ übergeben. In Zeile~$6$ und $13$ wird sichergestellt, dass
 $e_1.o$ und $e_2.o$ innerhalb von $67.5^\circ$\footcite[Vgl.][S.~417]{clarke96} liegen und damit kompatibel wären.
 Dies wird durch

\begin{equation}
	d = 2 \pi \left( \frac{ \frac{67.5}{2} }{360} \right) = 0.589
\end{equation}

überprüft. Es muss sichergestellt werden, dass die Orientierung in Bogenmaß erfolgt.

\begin{algorithm}[ht]
\caption{Orientierung von zwei edgels überprüfen}
\label{alg:compatibleedgel}
\begin{algorithmic}[1]
	\Require $e_1, e_2$
	\State $c \gets 0$
	\If{$e_1.o == e_2.o$}
		\State \textbf{return TRUE}
	\ElsIf{$e_1.o < e_2.o$}
		\State $c \gets e_2.o - e_1.o$
		\If{$c < d$}
			\State \textbf{return TRUE}
		\Else
			\State \textbf{return FALSE}
		\EndIf
	\Else
		\Comment $e_1.o > e_2.o$
		\State $c \gets e_1.o - e_2.o$
		\If{$c < d$}
			\State \textbf{return TRUE}
		\Else
			\State \textbf{return FALSE}
		\EndIf
	\EndIf
\end{algorithmic}
\end{algorithm}


Der Edgelspeicher in \autoref{alg:datastructure-edgelpool} verwendet ein Array von \glspl{edgel}
 (Vgl. \autoref{alg:datastructure-edgel}) mit fester Größe $N$ und einer Zählvariable um die nächste freie Position im
 Array zu markieren. Der Speicherpool aus \autoref{alg:datastructure-poolimplementation} ist ein Array vom Typ des
 Edgelspeichers mit der Größe $S$, dessen Adresse im Pointer $\mathit{pool}$ gespeichert wird.

\begin{algorithm}[ht]
\caption{Datenstruktur des Edgel-Speichers}
\label{alg:datastructure-edgelpool}
\begin{algorithmic}[1]
	\State $\mathit{data}[512]$
	\Comment Anzahl der Einträge
	\State $\mathit{count}$
\end{algorithmic}
\end{algorithm}

\autoref{alg:edgelpool-getmemorypools} basiert auf einem einfachen Stack Allocator von
 \citeauthor{kr}\footcite[Vgl.][S.~100--104]{kr}. Die Variable $n$ gibt die Anzahl der angeforderten Pools an. In Zeile
 \ref{alg:edgelpool-getmemorypools-checkpoolsize} wird überprüft, ob genügend Pools zur Verfügung stehen und liefert im
 Erfolgsfall die Adresse zu einem Speicher (\autoref{alg:datastructure-edgelpool}) zurück. Falls kein Pool mehr zur
 Verfügung steht, wird $\mathit{NULL}$ zurückgegeben. \autoref{alg:edgelpool-getmemorypool} vereinfacht die Anforderung
 eines Pools, da in den meisten Fällen nur ein Pool benötigt wird. Bei einem Aufruf kann somit auf einen Parameter
 verzichtet werden. Sowohl \autoref{alg:edgelpool-getmemorypools} als auch \autoref{alg:edgelpool-getmemorypool} haben
 eine konstante Laufzeit.

\begin{algorithm}[ht]
\caption{Hole Edgelpools}
\label{alg:edgelpool-getmemorypools}
\begin{algorithmic}[1]
	\Require $n$
	\If{$\mathit{data} + S - {pool} \geq n$}
		\State $pool = pool + n$
		\State \textbf{return} $\mathit{pool} - n$
	\Else
		\State \textbf{return} $\mathit{NULL}$
	\EndIf
\end{algorithmic}
\end{algorithm}

\begin{algorithm}[ht]
\caption{Hole Edgelpool}
\label{alg:edgelpool-getmemorypool}
\begin{algorithmic}[1]
	\State $p \gets$ \Call{getmemorypools}{1}
	\State \textbf{return} $p$
\end{algorithmic}
\end{algorithm}

Um \glspl{edgel} in einem Pool zu speichern, verwende ich \autoref{alg:edgelpool-addedgel}. Der Algorithmus benötigt
 einen Pointer $p$ auf einen Pool und einen \gls{edgel} $e$. In Zeile
 \ref{alg:edgelpool-addedgel-validpointer-start}--\ref{alg:edgelpool-addedgel-validpointer-end} wird geprüft, ob der
 Pointer auf eine Adresse verweist. Falls $p$ null ist, wird der Algorithmus verlassen. In Zeile
 \ref{alg:edgelpool-addedgel-checkspace-start}--\ref{alg:edgelpool-addedgel-checkspace-end} wird geprüft, ob im Array
 genügend Platz für einen weiteren Eintrag vorhanden ist. Die Größe von $N$ Einträgen richtet sich nach der in \autoref{alg:datastructure-edgelpool} festgelegten Arraygröße $N$. Wenn genügend Platz vorhanden ist, wird in Zeile
 \ref{alg:edgelpool-addedgel-add-start}--\ref{alg:edgelpool-addedgel-add-end} der \gls{edgel} $e$ an die freie
 Position $c$ geschrieben. Danach wird $\mathit{count}$ inkrementiert. Das Hinzufügen eines \glspl{edgel} ist konstant.

\begin{algorithm}[ht]
\caption{Edgel hinzufügen}
\label{alg:edgelpool-addedgel}
\begin{algorithmic}[1]
	\Require $p,e$
	\If{$!p$}
		\State \textbf{return}
	\EndIf
	\If{$!p(\mathit{count}) < N$}
		\State \textbf{return} \Comment Speicher voll
	\EndIf
	\State $c \gets p(\mathit{count})$
	\State $p(\mathit{data}[c]) \gets e$
	\State $p(\mathit{count}) \gets c + 1$
\end{algorithmic}
\end{algorithm}

\glspl{edgel} werden mittels \autoref{alg:edgelpool-getedgel} gelesen. Dazu wird der Pointer $p$ auf den Pool und der
 Index $i$ übergeben. In Zeile
 \ref{alg:edgelpool-getedgel-validpointer-start}--\ref{alg:edgelpool-getedgel-validpointer-end} wird geprüft, ob es
 sich um einen gesetzten Pointer handelt. Anschließend wird in Zeile
 \ref{alg:edgelpool-getedgel-validrange-start}--\ref{alg:edgelpool-getedgel-validrange-end} geprüft, ob der Index $i$
 innerhalb des gespeicherten Bereichs der \glspl{edgel} liegt. Danach wird in Zeile
 \ref{alg:edgelpool-getedgel-returnedgel} der Wert des \glspl{edgel} an Position $i$ zurückgegeben. Der Zugriff auf
 einen \gls{edgel} ist konstant.

\begin{algorithm}[ht]
\caption{Edgel lesen}
\label{alg:edgelpool-getedgel}
\begin{algorithmic}[1]
	\Require $p,i$
	\If{$!p$}
	\label{alg:edgelpool-getedgel-validpointer-start}
		\State \textbf{return}
	\EndIf
	\label{alg:edgelpool-getedgel-validpointer-end}
	\State $c \gets p(\mathit{count})$
	\If{$! c > i$}
	\label{alg:edgelpool-getedgel-validrange-start}
		\State \textbf{return}
	\EndIf
	\label{alg:edgelpool-getedgel-validrange-end}
	\State \textbf{return} $p(\mathit{data}[i])$
	\label{alg:edgelpool-getedgel-returnedgel}
\end{algorithmic}
\end{algorithm}

Damit \glspl{edgel} aus dem Array entfernt werden können, wird \autoref{alg:edgelpool-removeedgel} verwendet. Es wird
 der Pool-Pointer $p$ und die Position $i$ des zu löschenden \glspl{edgel} übergeben. Nach der Überprüfung des Pointers
 $p$ in Zeile~\ref{alg:edgelpool-removeedgel-validpointer-start}--\ref{alg:edgelpool-removeedgel-validpointer-end} und
 der Überprüfung des zulässigen Bereichs in Zeile
 \ref{alg:edgelpool-removeedgel-validrange-start}--\ref{alg:edgelpool-removeedgel-validrange-end}, gibt es zwei zu
 behandelnde Fälle um einen \gls{edgel} zu löschen.

\begin{algorithm}[ht]
\caption{Edgel löschen}
\label{alg:edgelpool-removeedgel}
\begin{algorithmic}[1]
	\Require $p,i$
	\If{$!p$}
		\State \textbf{return}
	\EndIf
	\State $c \gets p(\mathit{count})$
	\If{$! c > i$}
		\State \textbf{return}
	\EndIf
	\If{$c > i + 1$}
		\State \Call{memmove}{$p(\mathit{data}[i]), p(\mathit{data}[i + 1]), c - (i + 1) * \textproc{sizeof}(e)$}
	\EndIf
	\State $p(\mathit{count}) = c - 1$
\end{algorithmic}
\end{algorithm}

Der Edgel liegt
\begin{enumerate}
	\item nicht am Ende des Arrays oder \label{removeedgel-worst}
	\item liegt am Ende des Arrays. \label{removeedgel-best}
\end{enumerate}

Bei \autoref{removeedgel-best} muss lediglich $\mathit{count}$ dekrementiert werden um auf den vorigen Wert zu verweisen
 (Vgl. \autoref{fig:decrementcounter}). Das dekrementieren der Zählvariable $p(\mathit{count})$ ist eine Zuweisung in
 konstanter Zeit.

\begin{figure}[!ht]
	\centering
	\subfigure[]{
		\input{resources/Memory-Decrement-Before.pdf_tex}
		\label{fig:decrementcounter-before}
	}
	\subfigure[]{
		\input{resources/Memory-Decrement-After.pdf_tex}
		\label{fig:decrementcounter-after}
	}
	\caption{Dekrementieren von $\mathit{count}$. In \subref{fig:decrementcounter-before} soll Position $i$ gelöscht
	 werden. $c$ verweist auf die nächste freie Speicherstelle. In \subref{fig:decrementcounter-after} wird $c$
	 dekrementiert und verweist auf die neue freie Speicherstelle.}
	\label{fig:decrementcounter}
\end{figure}

Bei \autoref{removeedgel-worst} wird das Array an der Position $i$ geteilt und der Wertebereich von $[i+1 \dotsc i-n]$
 wird an die Position $i$ verschoben (Vgl. \autoref{fig:memmove}). % TODO: Grafik [1][2]..[i-1][i][i+1]..[i-n]
In Zeile \ref{alg:edgelpool-removeedgel-memmove} gibt die Funktion \textproc{sizeof}($e$) die Speichergröße eines
 \glspl{edgel} an, welche zum verschieben der Daten notwendig ist. Mit $c - (i + 1)$ wird die Anzahl der zu
 verschiebenden Einträge ermittelt. Im worst-case werden $N-1$ Einträge an Position $0$ des Arrays verschoben.

Um die Laufzeit der Funktion \textproc{memmove} zu bestimmen, wurde ein Testprogramm geschrieben, dass die Zeit misst,
 die benötigt wird, um Einträge zu verschieben. Anhand der Daten wurde mittels einer Regressionsanalyse untersucht, ob
 die gemessenen Daten einen linearen Zusammenhang aufweisen. Die erfassten $2000$ Datenpunkte wurden nach dem Vorbild von
 \textproc{time} % TODO: Referenz auf den Katalog für /usr/bin/time und man-page
ermittelt um Real-, User- und Sys-Zeit zu bestimmen. Aus User- und Sys-Zeit wurde die CPU-Zeit bestimmt, die zur
 Analyse benutzt wurde. Die Kovarianz für $X = \mathit{BYTES}$ und $Y = \mathit{CPU}$ beträgt $r = 0.9937538$ und
 $r^2 = 0.9875$. An dieser Stelle sei darauf hingewiesen, dass die Kovarianz für $X = \mathit{BYTES}$ und
 $Y = \mathit{REAL}$ mit $r = 0.9981969$ zwar größer ist, aber nicht die tatsächlichen Operationen des Testprogramms
 untersucht. Aus diesem Grund wurde $Y = \mathit{CPU}$ untersucht. Der Interzept beträgt $\beta_0 = -71.89\e{-06}$
 (Abweichung von $71.98\e{-06}$) und die Steigung $\beta_1 = 4.410\e{-09}$ (Abweichung von $11.08\e{-12}$), sodass

\begin{subequations}
\begin{multline}
	y =\\ \beta_0 + \beta_1n
\end{multline}
\begin{multline}
	y =\\ -71.89\e{-06} + 4.410\e{-09}n
\end{multline}
\end{subequations} % TODO: Sauber formatieren

Daraus ergibt sich eine Laufzeit von $\Theta(n)$. In \autoref{fig:regression-memmove} ist der Plot der Daten angegeben.

\begin{figure}[!ht]
	\centering
	\includegraphics[width=.8\textwidth]{resources/Regression-memmove.png}
	\caption{Regressionsanalyse von $2000$ Datenpunkten.}
	\label{fig:regression-memmove}
\end{figure}

Um einen Pool für einen neuen Durchlauf zu löschen, kommt \autoref{alg:edgelpool-resetmemorypool} zum Einsatz. Als
 Parameter wird der Pointer $p$ übergeben und in Zeile
 \ref{alg:edgelpool-resetmemorypool-validpointer-start}--\ref{alg:edgelpool-resetmemorypool-validpointer-end}
 überprüft. Um alle Daten als gelöscht zu markieren, wird lediglich die Zählvariable in Zeile
 \ref{alg:edgelpool-resetmemorypool-reset} auf $0$ gesetzt. Die Zuweisung erfolgt in konstanter Zeit.

\begin{algorithm}[ht]
\caption{Lösche Daten in Edgelpool}
\label{alg:edgelpool-resetmemorypool}
\begin{algorithmic}[1]
	\Require $p$
	\If{$!p$}
	\label{alg:edgelpool-resetmemorypool-validpointer-start}
		\State \textbf{return}
	\EndIf
	\label{alg:edgelpool-resetmemorypool-validpointer-end}
	\State $p(\mathit{count}) \gets 0$
	\label{alg:edgelpool-resetmemorypool-reset}
\end{algorithmic}
\end{algorithm}

Wenn ein Speicherpool nicht mehr benötigt wird, kann er mit \autoref{alg:edgelpool-freememorypool} freigegeben werden.
 Der Pointer $p$ wird in Zeile
 \ref{alg:edgelpool-freememorypool-validpointer-start}--\ref{alg:edgelpool-freememorypool-validpointer-end} überprüft.
 Zeile~\ref{alg:edgelpool-freememorypool-resetmemory} werden die Daten des Pools gelöscht
 (Vgl. \autoref{alg:edgelpool-resetmemorypool}). Im Anschluss wird in Zeile
 \ref{alg:edgelpool-freememorypool-checkpointer} überprüft, ob $p$ zu dem Array $\mathit{data}$ gehört und nicht größer
 als die definierte Speichergröße ist. Wenn der Test positiv ausfällt, wird der Pointer $p$ zur weiteren Verwendung in
 $\mathit{pool}$ gespeichert. Das Freigeben eines Pools erfolgt in konstanter Zeit.

\begin{algorithm}[ht]
\caption{Speicher des Edgelpool freigeben}
\label{alg:edgelpool-freememorypool}
\begin{algorithmic}[1]
	\Require $p$
	\If{$!p$}
	\label{alg:edgelpool-freememorypool-validpointer-start}
		\State \textbf{return}
	\EndIf
	\label{alg:edgelpool-freememorypool-validpointer-end}
	\State \Call{resetmemorypool}{$p$}
	\label{alg:edgelpool-freememorypool-resetmemory}
	\If{$p \geq \mathit{data} \land p \leq \mathit{data} + S$}
	\label{alg:edgelpool-freememorypool-checkpointer}
		\State $\mathit{pool} \gets p$
	\EndIf
\end{algorithmic}
\end{algorithm}

Die Anzahl der \glspl{edgel} in einem Pool werden durch \autoref{alg:edgelpool-count} ermittelt. Als Parameter wird
 Pointer $p$ übergeben und in Zeile
 \ref{alg:edgelpool-count-validpointer-start}--\ref{alg:edgelpool-count-validpointer-end} überprüft. Die Anzahl der
 Einträge wird in Zeile~\ref{alg:edgelpool-count-counter} über die Zählvariable $p(\mathit{count})$ ermittelt. Der
 Zugriff auf die Variable, und somit die Laufzeit des Algorithmus, erfolgt in konstanter Zeit.

\begin{algorithm}[ht]
\caption{Anzahl der Einträge}
\label{alg:edgelpool-count}
\begin{algorithmic}[1]
	\Require $p$
	\If{$!p$}
		\State \textbf{return}
	\EndIf
	\State \textbf{return} $p(\mathit{count})$
\end{algorithmic}
\end{algorithm}

% subsection datenstruktur-edgels (end)


\subsubsection{Liniensegmente} % (fold)
\label{sub:datenstruktur-liniensegmente}

Die Datenstruktur eines Liniensegments und die Methoden zum hinzufügen, löschen und freigeben des Speichers sind nach
 dem Vorbild des Edgelspeichers aufgebaut. Die Datenstruktur eines Liniensegments ist in
 \autoref{alg:datastructure-linesegment} definiert. Eine Linie besteht aus den \glspl{edgel} $\mathit{start}$ und
 $\mathit{end}$, die den Start- und Endpunkt der Linie darstellen. Die Variable $\mathit{slope}$ enthält die
 Orientierung des Liniensegments, während die Variable $\mathit{supportCount}$ die Anzahl der unterstützenden
 \glspl{edgel} der Linie speichert. $\mathit{remove}$, $\mathit{startCorner}$ und $\mathit{endCorner}$ sind boolesche
 Variablen. $\mathit{remove}$ dient im späteren Verlauf zur Erkennung, ob ein Liniensegment gelöscht werden muss,
 während $\mathit{startCorner}$ und $\mathit{endCorner}$ benutzt werden, ob eine Linie einen Eckpunkt am Anfang oder
 Ende besitzt. Die letzte Variable $\mathit{support}$ dient zur Speicherung von \glspl{edgel}, die eine Linienhypothese
 unterstützen. Die Lese- und Schreibzugriffe auf die Datenstruktur ist konstant.

\begin{algorithm}[!ht]\small
\caption{\textproc{lineSegment}}
\label{alg:datastructure-linesegment}
	\begin{algorithmic}[1]
		\State $\mathit{start}$
		\State $\mathit{end}$
		\State $\mathit{slope}$
		\State $\mathit{supportCount}$
		\State $\mathit{remove}$
		\State $\mathit{startCorner}$
		\State $\mathit{endCorner}$
		\State $\mathit{support}[\mathit{MAXEDGELS}]$
	\end{algorithmic}
\end{algorithm}


Die Datenstruktur eines Speichervorrats für Linien in \autoref{alg:datastructure-linesegmentpool} besteht aus einem
 Array $\mathit{data}$ mit der festen Größe $N$ und einer Zählvariablen $\mathit{count}$.

\begin{algorithm}[!ht]\small
\caption{\textproc{lineSegmentPool} (Datenstruktur)}
\label{alg:datastructure-linesegmentpool}
\begin{algorithmic}[1]
	\State $\mathit{data}[N]$
	\Comment Anzahl der Einträge
	\State $\mathit{count}$
\end{algorithmic}
\end{algorithm}


Der Speichervorrat für Linien in \autoref{alg:datastructure-linesegmentpool-implementation} besteht wiederum aus einem
 Array $\mathit{data}$ mit der Anzahl $S$ der zur Verfügung stehenden Speicherblöcke. Der Zeiger von $\mathit{data}$
 wird in der Variablen $\mathit{pool}$ gespeichert. Der Zugriff auf die Datenstruktur erfolgt in konstanter Zeit.

Mehrere Speicherblöcke können mit \autoref{alg:linepool-getmemorypools} angefordert werden und mit
 \autoref{alg:linepool-getmemorypool} wird ein Speicherblock angefordert. Der Aufbau der Verfahren entspricht den
 Verfahren des Speichervorrats für \glspl{edgel} (Vgl. \autoref{alg:edgelpool-getmemorypools} und
 \autoref{alg:edgelpool-getmemorypool}). Der Zugriff erfolgt in konstanter Zeit.

\begin{algorithm}[!ht]
\caption{\textproc{getMemoryPools}}
\label{alg:linepool-getmemorypools}
\begin{algorithmic}[1]
	\Require $n$
	\If{$\mathit{data} + S - \mathit{pool} \geq n$}
	\label{alg:linepool-getmemorypools-checkpoolsize}
		\State $\mathit{pool} \gets \mathit{pool} + n$
		\State \textbf{return} $\mathit{pool} - n$
	\Else
		\State \textbf{return} $\mathit{NULL}$
	\EndIf
\end{algorithmic}
\end{algorithm}


\begin{algorithm}[!ht]\small
\caption{\textproc{getMemoryPool}}
\label{alg:linepool-getmemorypool}
\begin{algorithmic}[1]
	\State $p \gets$ \Call{getmemorypools}{1}
	\Cost{$c_{1}$}{$1+7$}
	\State \textbf{return} $p$
	\Cost{$c_{2}$}{$1$}
\end{algorithmic}
\end{algorithm}


Um eine Linie zu einem Speicherblock hinzuzufügen, wird \autoref{alg:linesegmentpool-addline} verwendet. Es wird ein Zeiger
 $p$ auf den Speicherblock, sowie eine Linie $l$ übergeben. Wenn es sich um einen gültigen Zeiger $p$ handelt, und
 genügend freier Speicherplatz für eine weitere Linie vorhanden ist, wird in Zeile
 \ref{alg:linesegmentpool-addline-add-start}--\ref{alg:linesegmentpool-addline-add-end} die Linie hinzugefügt und die
 Zählvariable inkrementiert. Das Hinzufügen einer Linie ist konstant.

\begin{algorithm}[!ht]\small
\caption{\textproc{addLineSegment}}
\label{alg:linesegmentpool-addline}
\begin{algorithmic}[1]
	\Require $p,l$
	\If{$\lnot p$}
	\Cost{$c_{1}$}{$1$}
	\label{alg:linesegmentpool-addline-validpointer-start}
		\State \textbf{return}
		\Cost{$c_{2}$}{$1$}
	\EndIf
	\label{alg:linesegmentpool-addline-validpointer-end}
	\If{$\lnot \left(\mathit{p.count} < N\right)$}
	\Cost{$c_{4}$}{$2$}
	\label{alg:linepool-addline-checkspace-start}
		\State \textbf{return} \Comment Speicher voll
		\Cost{$c_{5}$}{$1$}
	\EndIf
	\label{alg:linesegmentpool-addline-checkspace-end}
	\State $c \gets \mathit{p.count}$
	\Cost{$c_{7}$}{$2$}
	\label{alg:linesegmentpool-addline-add-start}
	\State $\mathit{p.data}[c] \gets l$
	\Cost{$c_{8}$}{$3$}
	\State $\mathit{p.count} \gets c + 1$
	\Cost{$c_{9}$}{$3$}
	\label{alg:linesegmentpool-addline-add-end}
\end{algorithmic}
\end{algorithm}


Zum auslesen einer Linie aus dem Speicherblock, wird \autoref{alg:linepool-getline} verwendet. Als Parameter werden ein
 Zeiger $p$ und ein Index $i$ übergeben. Der Index gibt an, welche Linie aus dem Block ausgelesen werden soll. In Zeile
 \ref{alg:linepool-getline-validrange-start} wird geprüft, ob der Index sich innerhalb der Grenzen der gespeicherten
 Linien befindet. Wenn dies der Fall ist, wird in Zeile \ref{alg:linepool-getline-returnline} die Linie in konstanter
 Zeit zurückgegeben.

\begin{algorithm}[!ht]
\caption{\textproc{getLineSegment}}
\label{alg:linepool-getline}
\begin{algorithmic}[1]
	\Require $p,i$
	\If{$\lnot p$}
	\label{alg:linepool-getline-validpointer-start}
		\State \textbf{return}
	\EndIf
	\label{alg:linepool-getline-validpointer-end}
	\State $c \gets \mathit{p.count}$
	\If{$\lnot \left(c > i\right)$}
	\label{alg:linepool-getline-validrange-start}
		\State \textbf{return}
	\EndIf
	\label{alg:linepool-getline-validrange-end}
	\State \textbf{return} $\mathit{p.data}[i]$
	\label{alg:linepool-getline-returnline}
\end{algorithmic}
\end{algorithm}


Mit \autoref{alg:linepool-resetmemorypool} werden die Einträge im Speicherblock gelöscht. Dazu wird der Zeiger $p$ auf
 den Speicherblock übergeben und in Zeile
 \ref{alg:linepool-resetmemorypool-validpointer-start}--\ref{alg:linepool-resetmemorypool-validpointer-end} überprüft.
 Wenn es sich um einen gültigen Zeiger handelt, wird die Zählvariable auf $0$ gesetzt. Da es sich um einen direkten
 Zugriff handelt, erfolgt das Löschen in konstanter Zeit.

\begin{algorithm}[!ht]
\caption{\textproc{resetMemoryPool}}
\label{alg:linepool-resetmemorypool}
\begin{algorithmic}[1]
	\Require $p$
	\If{$\lnot p$}
	\Cost{$c_{1}$}{$1$}
	\label{alg:linepool-resetmemorypool-validpointer-start}
		\State \textbf{return}
		\Cost{$c_{2}$}{$1$}
	\EndIf
	\label{alg:linepool-resetmemorypool-validpointer-end}
	\State $\mathit{p.count} \gets 0$
	\Cost{$c_{4}$}{$2$}
	\label{alg:linepool-resetmemorypool-reset}
\end{algorithmic}
\end{algorithm}


Durch \autoref{alg:linepool-freememorypool} kann ein Speicherblock wieder freigegeben werden. Dazu wird der Zeiger $p$
 auf Gültigkeit geprüft. Danach wird der Speicher durch \textproc{resetMemoryPool}
 (\autoref{alg:linepool-resetmemorypool}) gelöscht. In Zeile \ref{alg:linepool-freememorypool-checkpointer} wird
 überprüft, ob der Zeiger $p$ zu dem entsprechenden Block gehört, um danach die Adresse in Zeile
 \ref{alg:linepool-freememorypool-savepointer} in $\mathit{pool}$ zu speichern. Auch hier erfolgt das Freigeben des
 Speichers wieder in konstanter Zeit.

\begin{algorithm}[!ht]\small
\caption{\textproc{freeMemoryPool}}
\label{alg:linepool-freememorypool}
\begin{algorithmic}[1]
	\Require $p$
	\If{$\lnot p$}
	\Cost{$c_{1}$}{$1$}
	\label{alg:linepool-freememorypool-validpointer-start}
		\State \textbf{return}
		\Cost{$c_{2}$}{$1$}
	\EndIf
	\label{alg:linepool-freememorypool-validpointer-end}
	\State \Call{resetmemorypool}{$p$}
	\Cost{$c_{4}$}{$3$}
	\label{alg:linepool-freememorypool-resetmemory}
	\If{$p \geq \mathit{data} \land p \leq \mathit{data} + S$}
	\Cost{$c_{5}$}{$4$}
	\label{alg:linepool-freememorypool-checkpointer}
		\State $\mathit{pool} \gets p$
		\Cost{$c_{6}$}{$1$}
		\label{alg:linepool-freememorypool-savepointer}
	\EndIf
\end{algorithmic}
\end{algorithm}


Die Anzahl der Einträge in einem Pool werden durch \autoref{alg:linepool-count} bestimmt, indem die Zählvariable
 $\mathit{count}$ zurückgegeben wird. Der Zugriff auf die Variable erfolgt in konstanter Zeit.

\begin{algorithm}[!ht]
\caption{\textproc{getLineCount}}
\label{alg:linepool-count}
\begin{algorithmic}[1]
	\Require $p$
	\If{$\lnot p$}
	\Cost{$c_{1}$}{$1$}
	\label{alg:linepool-count-validpointer-start}
		\State \textbf{return}
		\Cost{$c_{2}$}{$1$}
	\EndIf
	\label{alg:linepool-count-validpointer-end}
	\State \textbf{return} $\mathit{p.count}$
	\Cost{$c_{4}$}{$2$}
	\label{alg:linepool-count-counter}
\end{algorithmic}
\end{algorithm}


Im Verfahren nach \citeauthor{clarke96} werden Liniensegmente nicht aus dem Speicherpool gelöscht. Darum kann auf
 einen Algorithmus wie \autoref{alg:edgelpool-removeedgel} bei \glspl{edgel} verzichtet werden.

Alle Operationen für Linien erfolgen somit in konstanter Zeit $T(n) = \Theta(1)$.

% subsection datenstruktur-liniensegmente (end)


\subsubsection{Distanz} % (fold)
\label{sub:distanz}

Die Datenstruktur \textproc{distance} dient zur Speicherung der Länge von Liniensegmenten und der Speicherung eines
 Index, um eine Linie in einem Speicherblock anzusprechen (\autoref{alg:distance}).

\begin{algorithm}[!ht]\small
\caption{\textproc{distance}}
\label{alg:distance}
	\begin{algorithmic}[1]
		\State $\mathit{distance}$
		\State $\mathit{index}$
	\end{algorithmic}
\end{algorithm}


Wie bei \gls{edgel} oder Liniensegmenten, kann \textproc{distance} in einem Speicherbereich hinterlegt werden und ist in
 \autoref{alg:datastructure-distancepool} dargestellt. Auch bei diesem Verfahren wird ein Array mit festert Größe $N$
 und einer Zählvariable verwendet. Im Gegensatz zu den anderen Speicherbereichen wird bei \textproc{distance} auf einen
 Speichervorrat verzichtet. Die Verwaltung eines Speichervorrat kann aber manuel erfolgen.

\begin{algorithm}[!ht]\small
\caption{\textproc{distancePool} (Datenstruktur)}
\label{alg:datastructure-distancepool}
\begin{algorithmic}[1]
	\State $\mathit{data}[N]$
	\Comment Anzahl der Einträge
	\State $\mathit{count}$
\end{algorithmic}
\end{algorithm}


Mit \autoref{alg:distancepool-adddistance} und \autoref{alg:distancepool-removeline} können Distanzwerte dem
 Speicherbereich hinzugefügt und entfernt werden. Durch \autoref{alg:distancepool-distancecount} kann die Anzahl der
 gespeicherten Distanzwerte ausgelesen werden. \textproc{freePool} (\autoref{alg:distancepool-freepool}) löscht den
 Speicherbereich.

\begin{algorithm}[!ht]
\caption{\textproc{addDistance}}
\label{alg:distancepool-adddistance}
\begin{algorithmic}[1]
	\Require $p,d$
	\If{$\lnot p$}
	\Cost{$c_{1}$}{$1$}
		\State \textbf{return}
		\Cost{$c_{2}$}{$1$}
	\EndIf
	\If{$\lnot \left(\mathit{p.count} < N\right)$}
	\Cost{$c_{4}$}{$3$}
		\State \textbf{return} \Comment Speicher voll
		\Cost{$c_{5}$}{$1$}
	\EndIf
	\State $c \gets \mathit{p.count}$
	\Cost{$c_{7}$}{$2$}
	\State $\mathit{p.data}[c] \gets d$
	\Cost{$c_{8}$}{$3$}
	\State $\mathit{p.count} \gets c + 1$
	\Cost{$c_{9}$}{$3$}
\end{algorithmic}
\end{algorithm}

\begin{algorithm}[!ht]
\caption{\textproc{removeDistance}}
\label{alg:distancepool-removedistance}
\begin{algorithmic}[1]
	\Require $p,i$
	\If{$\lnot p$}
	\Cost{$c_{1}$}{$1$}
		\State \textbf{return}
		\Cost{$c_{2}$}{$1$}
	\EndIf
	\State $c \gets \mathit{p.count}$
	\Cost{$c_{4}$}{$2$}
	\If{$i < 0 \lor i > c$}
	\Cost{$c_{5}$}{$3$}
	\label{alg:distancepool-isvalid-start}
		\State \textbf{return}
		\Cost{$c_{6}$}{$1$}
	\EndIf
	\label{alg:distancepool-isvalid-end}
	\If{$c > i + 1$}
	\Cost{$c_{8}$}{$2$}
		\State \Call{memmove}{$\mathit{p.data}[\mathit{i}], \mathit{p.data}[\mathit{i} + 1], \left(c - \mathit{i}
		 + 1\right) \cdot \textproc{sizeof}(d)$}
		\Cost{$c_{9}$}{$8 + \Theta(1)$}
	\EndIf
\end{algorithmic}
\end{algorithm}

\begin{algorithm}[!ht]
\caption{\textproc{getDistanceCount}}
\label{alg:distancepool-distancecount}
\begin{algorithmic}[1]
	\Require $p$
	\If{$\lnot p$}
	\Cost{$c_{1}$}{$1$}
		\State \textbf{return}
		\Cost{$c_{2}$}{$1$}
	\EndIf
	\State \textbf{return} $\mathit{p.count}$
	\Cost{$c_{4}$}{$2$}
\end{algorithmic}
\end{algorithm}

\begin{algorithm}[!ht]\small
\caption{\textproc{freePool}}
\label{alg:distancepool-freepool}
\begin{algorithmic}[1]
	\Require $p$
	\If{$\lnot p$}
	\Cost{$c_{1}$}{$1$}
		\State \textbf{return}
		\Cost{$c_{2}$}{$1$}
	\EndIf
	\State $\mathit{p.count} \gets 0$
	\Cost{$c_{4}$}{$2$}
\end{algorithmic}
\end{algorithm}


% subsubsection distanz (end)


\subsubsection{Marken} % (fold)
\label{sub:marken}

Die Datenstruktur \textproc{marker}, der Aufbau des Speicherblocks und die Operationen zum hinzufügen und freigeben von
 Daten ist, wie bei der Datenstruktur der Liniensegmente auch, nach dem Vorbild des Edgelspeichers aufgebaut. Die
 Datenstruktur in \autoref{alg:marker} verwendet vier Variablen vom Typ \textproc{vector}, um die Eckpunkte einer Marke
 zu speichern.
\begin{algorithm}[!ht]\small
\caption{\textproc{marker}}
\label{alg:marker}
	\begin{algorithmic}[1]
		\State $c1$
		\State $c2$
		\State $c3$
		\State $c4$
	\end{algorithmic}
\end{algorithm}


Die Datenstruktur des Speichervorrats (\autoref{alg:datastructure-markerpool})
\begin{algorithm}[!ht]\small
\caption{\textproc{markerPool} (Datenstruktur)}
\label{alg:datastructure-markerpool}
\begin{algorithmic}[1]
	\State $\mathit{data}[N]$
	\Comment Anzahl der Einträge
	\State $\mathit{count}$
\end{algorithmic}
\end{algorithm}

und des Speicherblocks (\autoref{alg:datastructure-markerpoolimpl}) verwenden eine festgelegte Anzahl von Einträgen zum
\begin{algorithm}[!ht]\small
\caption{\textproc{markerPool} (Speichervorrat)}
\label{alg:datastructure-markerpoolimpl}
\begin{algorithmic}[1]
	\State $\mathit{data}[S]$
	\Comment Anzahl der Pools
	\State $\mathit{pool}$
\end{algorithmic}
\end{algorithm}

speichern der Einträge.

Mit \textproc{getMemoryPools} (\autoref{alg:markerpool-getmemorypools})
\begin{algorithm}[!ht]\small
\caption{\textproc{getMemoryPools}}
\label{alg:markerpool-getmemorypools}
\begin{algorithmic}[1]
	\Require $n$
	\If{$\mathit{data} + S - \mathit{pool} \geq n$}
	\Cost{$c_{1}$}{$3$}
		\State $\mathit{pool} \gets \mathit{pool} + n$
		\Cost{$c_{2}$}{$2$}
		\State \textbf{return} $\mathit{pool} - n$
		\Cost{$c_{3}$}{$2$}
	\Else
		\State \textbf{return} $\mathit{NULL}$
		\Cost{$c_{5}$}{$1$}
	\EndIf
\end{algorithmic}
\end{algorithm}

und \textproc{getMemoryPool} (\autoref{alg:markerpool-getmemorypool})
\begin{algorithm}[!ht]
\caption{\textproc{getMemoryPool}}
\label{alg:markerpool-getmemorypool}
\begin{algorithmic}[1]
	\State $p \gets$ \Call{getmemorypools}{1}
	\State \textbf{return} $p$
\end{algorithmic}
\end{algorithm}

werden Speicherblöcke in konstanter Zeit angefordert.

Um die Einträge in einem Speicherblock zu löschen, wird \autoref{alg:markerpool-resetmemorypool} verwendet. Die
 Laufzeit ist dabei konstant.
\begin{algorithm}[!ht]
\caption{\textproc{resetMemoryPool}}
\label{alg:markerpool-resetmemorypool}
\begin{algorithmic}[1]
	\Require $p$
	\If{$\lnot p$}
		\State \textbf{return}
	\EndIf
	\State $p(\mathit{count}) \gets 0$
\end{algorithmic}
\end{algorithm}


Mit \textproc{freeMemoryPool} (\autoref{alg:markerpool-freememorypool}) kann der Speicherblock an den Vorrat
 zurückgegeben werden. Auch hier ist die Laufzeit konstant.
\begin{algorithm}[!ht]
\caption{\textproc{freeMemoryPool}}
\label{alg:markerpool-freememorypool}
\begin{algorithmic}[1]
	\Require $p$
	\If{$\lnot p$}
		\State \textbf{return}
	\EndIf
	\State \Call{resetMemoryPool}{$p$}
	\If{$p \geq \mathit{data} \land p \leq \mathit{data} + S$}
		\State $\mathit{pool} \gets p$
	\EndIf
\end{algorithmic}
\end{algorithm}


Die Anzahl der gespeicherten Marken in einem Speicherblock können mit \autoref{alg:markerpool-count} ermittelt werden.
 Durch den direkten Zugriff auf die Zählvariable ist die Laufzeit konstant.
\begin{algorithm}[!ht]
\caption{\textproc{getMarkerCount}}
\label{alg:markerpool-count}
\begin{algorithmic}[1]
	\Require $p$
	\If{$\lnot p$}
	\Cost{$c_{1}$}{$1$}
		\State \textbf{return}
		\Cost{$c_{2}$}{$1$}
	\EndIf
	\State \textbf{return} $\mathit{p.count}$
	\Cost{$c_{4}$}{$2$}
\end{algorithmic}
\end{algorithm}


Um eine Marke zu einem Speicherblock hinzuzufügen, wird \autoref{alg:markerpool-addmarker} verwendet. Auch hier wird durch den direkten Zugriff auf die Variablen eine konstante Laufzeit erreicht.
\begin{algorithm}[!ht]
\caption{\textproc{addMarker}}
\label{alg:markerpool-addmarker}
\begin{algorithmic}[1]
	\Require $p,m$
	\If{$\lnot p$}
	\Cost{$c_{1}$}{$1$}
		\State \textbf{return}
		\Cost{$c_{2}$}{$1$}
	\EndIf
	\If{$\lnot \left(\mathit{p.count} < N\right)$}
	\Cost{$c_{4}$}{$3$}
		\State \textbf{return} \Comment Speicher voll
		\Cost{$c_{5}$}{$1$}
	\EndIf
	\State $c \gets \mathit{p.count}$
	\Cost{$c_{7}$}{$2$}
	\State $\mathit{p.data}[c] \gets m$
	\Cost{$c_{8}$}{$3$}
	\State $\mathit{p.count} \gets c + 1$
	\Cost{$c_{9}$}{$3$}
\end{algorithmic}
\end{algorithm}


Alle Methoden der Datenstruktur \textproc{marker} haben eine konstante Laufzeit $T(n)=\Theta(1)$.
% subsubsection marken (end)


% subsection datenstrukturen (end)

\subsection{Linienerkennung nach \texorpdfstring{\citeauthor{clarke96}}{Clarke, Carlsson und Zisserman}} % (fold)
\label{sub:linienerkennung_nach_clarke96}
\citeauthor{clarke96} verwenden in ihrem Verfahren ein monochromes Bildsignal $I_m$\footcite[Vgl.][S.~417]{clarke96}.
 Die Konvertierung des Bildsignals $I$ von YCbCr in $I_m$ erfolgt durch \autoref{alg:convertmonochrome}. Wie in
 \autoref{sub:farbräume} beschrieben, besteht ein YCbCr Signal aus einem Luminanz Kanal $Y$ und den Chroma Abweichungen
 $Cb$ und $Cr$. Um ein monochromes Signal $I_m$ zu erstellen, muss der Luminanz Kanal ausgelesen und in einen Puffer
 kopiert werden.
\begin{algorithm}[!ht]
\caption{Konvertierung zu monochromen Bildsignal}
\label{alg:convertmonochrome}
	\begin{algorithmic}[1]
		\Require $I, I_m$
		\State $Y \gets$ \Call{baseAddress}{$I$}
		\label{alg:convertmonochrome-baseaddress}
		\State $w \gets$ \Call{width}{$I$}
		\State $h \gets$ \Call{height}{$I$}
		\State $l \gets w \cdot h$
		\State $I_m \gets$ \Call{copy}{$I, Y, l$}
	\end{algorithmic}
\end{algorithm}

Der Algorithmus verwendet als Parameter das Bildsignal $I$ und einen Zeiger $I_m$ auf einen Puffer für das monochrome
 Signal. Der Monochrompuffer $I_m$ ist ein Array mit fester Größe, das beim initialisieren einmalig angelegt wird und
 danach wiederverwendet werden kann. In Zeile~\ref{alg:convertmonochrome-baseaddress} wird die Adresse des
 Luminanz-Kanals $Y$ ausgelesen. Die Funktionen \textproc{width} und \textproc{height} liefern die Breite und Höhe des
 Signals in Pixeln, mit denen die Länge der Daten berechnet wird. Anschließend werden die Daten in den Puffer kopiert.
 Die Konvertierung des Bildsignals ist Verarbeitungsschritt der vor dem Verfahren von \citeauthor{clarke96} durchgeführt
 wird und wird nur der Vollständigkeit erwähnt.

Um auf \gls{pixel} zugreifen zu können, wird \autoref{alg:getpixel} verwendet.
\begin{algorithm}[ht]
\caption{\textproc{getPixel}}
\label{alg:getpixel}
	\begin{algorithmic}[1]
		\Require $I_m, x, y, w, h$
		\If{$x < 0$}
		\label{alg:getpixel-startcheck}
			\State $x \gets 0$
		\EndIf
		\If{$y < 0$}
			\State $y \gets 0$
		\EndIf
		\If{$x \geq w$}
			\State $x \gets w - 1$
		\EndIf
		\If{$y \geq h$}
			\State $y \gets h -1$
		\EndIf
		\label{alg:getpixel-stopcheck}
		\State $i \gets x + \left(y \cdot w\right)$
		\State \textbf{return} $I_m[i]$
	\end{algorithmic}
\end{algorithm}

Es wird der Puffer $I_m$ als Zeiger übergeben und die Position $x$ und $y$ des gewünschten \gls{pixel}. $w$ und $h$
 entsprechen der Breite und Höhe von $I_m$. Zeile~\ref{alg:getpixel-startcheck} bis Zeile~\ref{alg:getpixel-stopcheck}
 sorgen dafür, dass keine Werte außerhalb des Puffers gelesen werden können. Dies ist für die Randbehandlung bei
 Faltungsoperationen (Vgl. \autoref{sub:filter}) wichtig und wiederholt den \gls{pixel}.Die Laufzeit von
 \autoref{alg:getpixel} ist konstant und somit $T(n)=\Theta(1)$.

Der Algorithmus von \citeauthor{clarke96} ist in \autoref{alg:linedetection-clarke} aufgeführt und benötigt das
 monochrome Bildsignal $I_m$.
\begin{algorithm}[!ht]\small
\caption{\textproc{lineDetection}}
\label{alg:linedetection-clarke}
	\begin{algorithmic}[1]
		\Require $I_m$
		\For{$y \gets 0$ \textbf{to} $y < \mathit{imageHeight}$}
		\Cost{$c_{1}$}{$\frac{h}{r} + 1$}
		\label{alg:linedetection-clarke-start}
			\For{$x \gets 0$ \textbf{to} $x < \mathit{imageWidth}$}
			\Cost{$c_{2}$}{$\frac{h}{r}(\frac{w}{r} + 1)$}
				\State \Call{findEdgels}{$I_m,E,x,y$}
				\Cost{$c_{3}$}{$\frac{h \cdot w}{r^2} t_{1}$}
				\label{alg:linedetection-clarke-call-start}
				\State \Call{findLineSegments}{$E,L$}
				\Cost{$c_{4}$}{$\frac{h \cdot w}{r^2} t_{2}$}
				\label{alg:linedetection-clarke-call-end}
				\State \ldots \Comment{Speichern der Liniensegmente zur weiteren Verarbeitung}
				\State \Call{resetMemoryPool}{$E$}
				\Cost{$c_{6}$}{$\frac{h \cdot w}{r^2} \Theta(1)$}
				\label{alg:linedetection-clarke-reset-start}
				\State \Call{resetMemoryPool}{$L$}
				\Cost{$c_{7}$}{$\frac{h \cdot w}{r^2} \Theta(1)$}
				\label{alg:linedetection-clarke-reset-end}
				\State $ x \gets x + \mathit{regionSize}$
				\Cost{$c_{8}$}{$\frac{h \cdot w}{r^2} 2$}
			\EndFor
			\State $y \gets y + \mathit{regionSize}$
			\Cost{$c_{10}$}{$\frac{h}{r} 2$}
		\EndFor
		\label{alg:linedetection-clarke-end}
	\end{algorithmic}
\end{algorithm}

In der doppelten Schleife in Zeile \ref{alg:linedetection-clarke-start} bis \ref{alg:linedetection-clarke-end} wird
 $I_m$ in Regionen der Größe $40 \times 40$ \gls{pixel} unterteilt. Die globalen Variablen $\mathit{imageWidth}$ und
 $\mathit{imageHeight}$ enthalten die Breite und Höhe des Bildsignals $I_m$. Die Regionengröße von $40$ \gls{pixel} ist
 in der globalen Variable $\mathit{regionSize}$ gespeichert. In \citeauthor{clarke96}\footcite{clarke96} sind keine
 Hintergrundinformationen zu der Größe einer Region angegeben. Betrachtet man $640 \bmod 40 = 0$ und $480 \bmod 40 = 0$
 ist ersichtlich, dass die Größe der Region und der Aufteilung des Bildsignals in Zusammenhang steht. In Zeile
 \ref{alg:linedetection-clarke-call-start}--\ref{alg:linedetection-clarke-call-end} werden zuerst \glspl{edgel}
 ermittelt, um im Anschluss daraus Liniensegmente zu erstellen. Wenn für eine Region Liniensegmente erstellt und
 gespeichert wurden, wird der Speicherblock der \gls{edgel} und Liniensegmente in Zeile
 \ref{alg:linedetection-clarke-reset-start}--\ref{alg:linedetection-clarke-reset-end} gelöscht.

% \begin{subequations}
% \begin{align}
% \label{eq:linedetection-analyze1}
% T(I)& =
% c_1
% + c_2
% + c_3 \left(\frac{h}{40} + 1\right)
% + c_4 \sum \limits_{y = 0}^{\frac{h}{40}} t_y \left(\frac{w}{40} + 1 \right)
% + c_5 \sum \limits_{y = 0}^{\frac{h}{40}} \sum \limits_{x = 0}^{\frac{w}{40}} t_y t_x\\
% & \quad + c_6 \sum \limits_{y = 0}^{\frac{h}{40}} \sum \limits_{x = 0}^{\frac{w}{40}} t_y t_x
% + c_7 \sum \limits_{y = 0}^{\frac{h}{40}} \sum \limits_{x = 0}^{\frac{w}{40}} t_y t_x
% + c_9 \sum \limits_{y = 0}^{\frac{h}{40}} t_y \nonumber \\
% \label{eq:linedetection-analyze2}
% T(I)& =
% c_1
% + c_2
% + c_3 \left(n + 1\right)
% + c_4 \sum \limits_{y = 0}^{n} t_y \left(k + 1 \right)
% + c_5 \sum \limits_{y = 0}^{n} \sum \limits_{x = 0}^{k} t_y t_x\\
% & \quad + c_6 \sum \limits_{y = 0}^{n} \sum \limits_{x = 0}^{k} t_y t_x
% + c_7 \sum \limits_{y = 0}^{n} \sum \limits_{x = 0}^{k} t_y t_x
% + c_9 \sum \limits_{y = 0}^{n} t_y \nonumber \\
% \label{eq:linedetection-analyze3}
% T(I)& =
% c_1
% + c_2
% + c_3 \left(n + 1\right)
% + c_4 \left[n \left(k + 1 \right)\right]
% + c_5 n k
% + c_6 n k
% + c_7 n k
% + c_9 n\\
% \label{eq:linedetection-analyze4}
% T(I)& = c_1 + c_2 + c_3 + \left(c_3 + c_4 + c_9\right) n + \left(c_4 + c_5 + c_6 + c_7\right) n k\\
% \label{eq:linedetection-analyze5}
% T(I)& = \Theta(nk)
% \end{align}
% \end{subequations}

Das Verfahren zur Bestimmung der Edgels (\autoref{alg:findedgels-horizontal} und \autoref{alg:findedgels-vertical})
 benötigt das monochrome Bildsignal $I_m$, sowie die Position der oberen linken Ecke der Region, die durch oben $t$ und
 links $l$ definiert ist.
\begin{algorithm}[!ht]
\caption{\textproc{findEdgels} (Horizontale Scanlines)}
\label{alg:findedgels-horizontal}
	\begin{algorithmic}[1]
		\Require $I_m, E, t, l$
		\For{$y \gets t$ \textbf{to} $y < t + \mathit{regionSize}$}
		\label{alg:findedgels-horizontal-scanlinestart}
			\State $p_1 \gets 0$
			\State $p_2 \gets 0$
			\For{$x \gets l$ \textbf{to} $x < l + \mathit{regionSize}$}
			\label{alg:findedgels-horizontal-loopstart}
				\State $\mathit{currentEdgel} \gets$ \Call{convoluteKernelX}{$I_m,x,y,\mathit{imageWidth},\mathit{imageHeight}$}
				\label{alg:findedgels-horizontal-convolute}
				\If{$\mathit{currentEdgel} > \mathit{threshold}$}
				\label{alg:findedgels-horizontal-foundedgel}
					\Comment Möglicherweise ein Egdel
				\Else
					\State $\mathit{currentEdgel} \gets 0$
				\EndIf
				\If{$p_1 > 0 \land p_1 > p_2 \land p_1 > \mathit{currentEdgel}$}
				\label{alg:findedgels-horizontal-maxima}
					\Comment $p_1$ ist lokales Maximum
					\State $\mathit{edgel} \gets \infty$
					\State \Call{vectorSetCoordinate}{$\mathit{edgel},x - 1,y$}
					\State $\mathit{edgel.slope} \gets$ \Call{gradientIntensity}{$I_m, \mathit{imageWidth}, \mathit{imageHeight}, x - 1,y$}
					\State \Call{addEdgel}{$E,\mathit{edgel}$}
				\EndIf
				\label{alg:findedgels-horizontal-maxima-end}
				\State $p_2 \gets p_1$
				\label{alg:findedgels-horizontal-copy-prev1}
				\State $p_1 \gets currentEdgel$
				\label{alg:findedgels-horizontal-copy-edgel}
				\State $x \gets x + 1$
			\EndFor
			\label{alg:findedgels-horizontal-loopend}
			\State $y \gets y + 5$
			\label{alg:findedgels-horizontal-increment}
		\EndFor
		\label{alg:findedgels-horizontal-scanlineend}
	\algstore{brkfindedgels}
	\end{algorithmic}
\end{algorithm}

Der Zeiger $E$ wird zur Speicherung der gefundenen \glspl{edgel} verwendet. Zeile
 \ref{alg:findedgels-horizontal-scanlinestart}--\ref{alg:findedgels-horizontal-scanlineend} ist für den Aufbau der
 horizontalen Scanlines verantwortlich. Die Überprüfung sorgt dafür, dass die Scanlines bis zum Ende der Region im
 Abstand von $5$ Pixeln untersucht werden. Nach der Initialisierung der Variablen wird in der Schleife von
 Zeile~\ref{alg:findedgels-horizontal-loopstart}--\ref{alg:findedgels-horizontal-loopend} jeder Pixel auf der Scanline
 untersucht. Zuerst wird in Zeile \ref{alg:findedgels-horizontal-convolute} die Faltung mit einem Gauß-Kernel
 vorgenommen (Vgl. \autoref{alg:convolutekernel-horizontal}, S. \pageref{alg:convolutekernel-horizontal}). Der Test
 in Zeile \ref{alg:findedgels-horizontal-foundedgel} überprüft anschließend das Ergebnis der Faltung. Wenn der
 Schwellwert nicht überschritten wird, gibt es keinen genügend großen Anstieg des Gradienten und das Ergebnis wird auf
 $0$ gesetzt. Wird der Schwellwert überschritten, handelt es sich um einen Edgel und das Ergebnis wird in den
 Bedingungen von Zeile \ref{alg:findedgels-horizontal-maxima} weiter untersucht, ob es sich um ein lokales Maximum
 handelt. Ein lokales Maximum bedeutet, dass ein Edgel einen größeren Gradienten besitzt als seine beiden Nachbarn. Die
 Bedingung in Zeile \ref{alg:findedgels-horizontal-maxima} wird bei der ersten Überprüfung immer fehlschlagen.
 Dadurch wird sichergestellt, dass kein Maximum an den Rändern existiert, da hier nicht genügend Nachbarn vorhanden sind
 um eine verlässliche Aussage zu treffen. Zeile \ref{alg:findedgels-horizontal-copy-prev1} und
 Zeile \ref{alg:findedgels-horizontal-copy-edgel} kopieren die Werte für den nächsten Durchlauf. Durch das kopieren der
 Werte werden die Nachbarn für den nächsten Durchlauf um eine Position weiterverschoben. Nur bei einem lokalen Maximum
 (Zeile \ref{alg:findedgels-horizontal-maxima}--\ref{alg:findedgels-horizontal-maxima-end}) wird die Position des
 Edgels gespeichert (Vgl. \autoref{alg:vectorsetcoordinate}, S. \pageref{alg:vectorsetcoordinate}), und seine
 Orientierung berechnet (Vgl. \autoref{alg:gradientintensity}, S. \pageref{alg:gradientintensity}). Der Edgel wird mit
 \textproc{addEdgel} (\autoref{alg:edgelpool-addedgel}, S. \pageref{alg:edgelpool-addedgel}) in einem Speicherblock zu
 weiteren Verarbeitung gespeichert. Sind alle Pixel auf einer Scanline untersucht, wird in Zeile
 \ref{alg:findedgels-horizontal-increment} die nächste Scanline ausgewählt. Das Verfahren wird solange wiederholt, bis
 alle Scanlines innerhalb der Region untersucht wurden. \autoref{alg:findedgels-vertical} untersucht die vertikalen
 Scanlines in Zeile \ref{alg:findedgels-vertical-scanlinestart}--\ref{alg:findedgels-vertical-scanlineend} analog zu
 \autoref{alg:findedgels-horizontal} Zeile
 \ref{alg:findedgels-horizontal-scanlinestart}--\ref{alg:findedgels-horizontal-scanlineend}.
\begin{algorithm}[!ht]
\caption{\textproc{findEdgels} (Vertikale Scanlines)}
\label{alg:findedgels-vertical}
\begin{algorithmic}[1]
\algrestore{brkfindedgels}
	\For{$x \gets l$ \textbf{to} $x < l + \mathit{regionSize}$}
	\Cost{$c_{21}$}{$\frac{40}{5} + 1$}
	\label{alg:findedgels-vertical-scanlinestart}
		\State $p_1, p_2 \gets 0$
		\Cost{$c_{22}$}{$8(2)$}
		\For{$y \gets t$ \textbf{to} $y < t + \mathit{regionSize}$}
		\Cost{$c_{23}$}{$8(40 + 1)$}
			\State $currentEdgel \gets$ \Call{convoluteKernelY}{$I_m,x,y,\mathit{imageWidth},\mathit{imageHeight}$}
			\Cost{$c_{24}$}{$320t_{3}$}
			\If{$currentEdgel > threshold$}
			\Comment Möglicherweise ein Egdel
			\Cost{$c_{25}$}{$320$}
			\Else
				\State $currentEdgel \gets 0$
				\Cost{$c_{27}$}{$320$}
			\EndIf
			\If{$p_1 > 0 \land p_1 > p_2 \land p_1 > currentEdgel$}
			\Comment $p_1$ ist lokales Maximum
			\Cost{$c_{29}$}{$320(5)$}
			\label{alg:findedgels-vertical-maxima}
				\State $edgel \gets \infty$
				\Cost{$c_{30}$}{$320$}
				\State \Call{vectorSetCoordinate}{$\mathit{edgel},x,y - 1$}
				\Cost{$c_{31}$}{$320 \cdot \Theta(1)$}
				\State $\mathit{edgel.slope} \gets$ \Call{gradientIntensity}{$I_m, \mathit{imageWidth},
				 \mathit{imageHeight}, x,y - 1$}
				\Cost{$c_{32}$}{$320t_{4}$}
				\State \Call{addEdgel}{$E,\mathit{edgel}$}
				\Cost{$c_{33}$}{$320 \cdot \Theta(1)$}
			\EndIf
			\State $p_2 \gets p_1$
			\Cost{$c_{35}$}{$320$}
			\State $p_1 \gets currentEdgel$
			\Cost{$c_{36}$}{$320$}
			\State $y \gets y + 1$
			\Cost{$c_{37}$}{$320$}
		\EndFor
		\State $x \gets x + 5$
		\Cost{$c_{39}$}{$8(2)$}
	\EndFor
	\label{alg:findedgels-vertical-scanlineend}
\end{algorithmic}
\end{algorithm}


In \autoref{alg:findedgels-horizontal-analyse} und \autoref{alg:findedgels-vertical-analyse} sind die Kosten von
 \textproc{findEdgels} aufgeführt.
\begin{algorithm}[!ht]
\caption{\textproc{findEdgels} (Analyse)}
\label{alg:findedgels-horizontal-analyse}
	\begin{algorithmic}[1]
		\Require $I_m, E, t, l$
		\For{$y \gets t$ \textbf{to} $y < t + \mathit{regionSize}$}
		\Cost{$c_1$}{$\tfrac{n}{5} + 1$}
			\State $p_1 \gets 0$
			\Cost{$c_2$}{$\sum_{y=0}^{\tfrac{n}{5}} 1$}
			\State $p_2 \gets 0$
			\Cost{$c_3$}{$\sum_{y=0}^{\tfrac{n}{5}} 1$}
			\For{$x \gets l$ \textbf{to} $x < l + \mathit{regionSize}$}
			\Cost{$c_4$}{$\sum_{y=0}^{\tfrac{n}{5}} (m + 1)$}
				\State Gradienten bestimmen
				\Cost{$c_5$}{$\sum_{y=0}^{\tfrac{n}{5}} \sum_{x=0}^{m} \Theta(1)$}
				\If{$\mathit{currentEdgel} > \mathit{threshold}$}
				\Cost{$c_6$}{$\sum_{y=0}^{\tfrac{n}{5}} \sum_{x=0}^{m} 1$}
				\Else
					\State $\mathit{currentEdgel} \gets 0$
					\Cost{$c_7$}{$\sum_{y=0}^{\tfrac{n}{5}} \sum_{x=0}^{m} 1$}
				\EndIf
				\If{$p_1 > 0 \land p_1 > p_2 \land p_1 > \mathit{currentEdgel}$}
				\Cost{$c_8$}{$\sum_{y=0}^{\tfrac{n}{5}} \sum_{x=0}^{m} 1$}
					\State $\mathit{edgel} \gets \infty$
					\Cost{$c_9$}{$\sum_{y=0}^{\tfrac{n}{5}} \sum_{x=0}^{m} 1$}
					\State Koordinaten speichern
					\Cost{$c_{10}$}{$\sum_{y=0}^{\tfrac{n}{5}} \sum_{x=0}^{m} \Theta(1)$}
					\State Orientierung berechnen
					\Cost{$c_{11}$}{$\sum_{y=0}^{\tfrac{n}{5}} \sum_{x=0}^{m} \Theta(1)$}
					\State \Call{addEdgel}{$E,\mathit{edgel}$}
					\Cost{$c_{12}$}{$\sum_{y=0}^{\tfrac{n}{5}} \sum_{x=0}^{m} \Theta(1)$}
				\EndIf
				\State $p_2 \gets p_1$
				\Cost{$c_{13}$}{$\sum_{y=0}^{\tfrac{n}{5}} \sum_{x=0}^{m} 1$}
				\State $p_1 \gets currentEdgel$
				\Cost{$c_{14}$}{$\sum_{y=0}^{\tfrac{n}{5}} \sum_{x=0}^{m} 1$}
				\State $x \gets x + 1$
				\Cost{$c_{15}$}{$\sum_{y=0}^{\tfrac{n}{5}} \sum_{x=0}^{m} 1$}
			\EndFor
			\State $y \gets y + 5$
			\Cost{$c_{16}$}{$\sum_{y=0}^{\tfrac{n}{5}} 1$}
		\EndFor
	\algstore{brk-findedgelsanalyse}
	\end{algorithmic}
\end{algorithm}

Die bereits vorgestellten Methoden \textproc{vectorSetCoordinate} (\autoref{alg:vectorsetcoordinate}) und
 \textproc{addEdgel} (\autoref{alg:edgelpool-addedgel}) haben eine konstante Laufzeit. Die Methoden
 \textproc{convoluteKernelX}, \textproc{convoluteKernelY} und \textproc{gradientIntensity} haben ebenfalls eine
 konstante Laufzeit, die zu einem späteren Zeitpunkt bewiesen wird.  Die Laufzeit von
 \autoref{alg:findedgels-horizontal-analyse} lässt sich wie in \autoref{eq:findedgels1} zusammenfassen.
\begin{subequations}
\begin{align}
\label{eq:findedgels1}
T(n,m)& =
c_1 \cdot (\frac{n}{5} + 1) + (c_2 + c_3 + c_{16}) \cdot \sum_{y=0}^{\frac{n}{5}} 1
 + c_4 \cdot \sum_{y=0}^{\frac{n}{5}} (m+1) \\
& \quad + (c_5 + c_6 + c_7 + c_8 + c_9 + c_{10} + c_{11} + c_{12} + c_{13} + c_{14} + c_{15})
 \cdot \sum_{y=0}^{\frac{n}{5}} \sum_{x=0}^{m} 1 \nonumber \\
\label{eq:findedgels2}
T(n,m)& = c_1 + (c_1 + c_2 + c_3 + c_4 + c_{16}) \cdot \frac{n}{5} + (c_4 + \ldots + c_{15})
\cdot (\frac{n}{5} \cdot m) \\
\label{eq:findedgels3}
T(n,m)& = \frac{n \cdot m}{5}
\end{align}
\end{subequations}

Der Algorithmus ist von der Eingabgröße $n = \mathit{regionSize} = 40$ abhängig. Durch Umformung in \autoref{eq:findedgels2} werden die Konstanten isoliert, was zu einer Laufzeit von $T(n) = \Theta(n^2)$ für \autoref{alg:findedgels-horizontal-analyse}
 führt (\autoref{eq:findedgels1-end}). Die Kosten für \textproc{findEdgels} zur Untersuchung der vertikalen Scanline
 sind in \autoref{alg:findedgels-vertical-analyse} aufgeführt und entsprechen den Kosten von
 \autoref{alg:findedgels-horizontal-analyse}.
\begin{algorithm}[!ht]\small
	\caption{\textproc{findEdgels} (Fortsetzung der Analyse)}
	\label{alg:findedgels-vertical-analyse}
	\begin{algorithmic}[1]
	\algrestore{brk-findedgelsanalyse}
		\For{$x \gets l$ \textbf{to} $x < l + \mathit{regionSize}$}
		\Cost{$c_{17}$}{$\tfrac{n}{5} + 1$}
			\State $p_1 \gets 0$
			\Cost{$c_{18}$}{$\sum_{x=0}^{\tfrac{n}{5}} 1$}
			\State $p_2 \gets 0$
			\Cost{$c_{19}$}{$\sum_{x=0}^{\tfrac{n}{5}} 1$}
			\For{$y \gets t$ \textbf{to} $y < t + \mathit{regionSize}$}
			\Cost{$c_{20}$}{$\sum_{x=0}^{\tfrac{n}{5}} (n + 1)$}
				\State Gradient bestimmen
				\Cost{$c_{21}$}{$\sum_{x=0}^{\tfrac{n}{5}} \sum_{y=0}^{n} \Theta(1)$}
				\If{$currentEdgel > threshold$}
				\Cost{$c_{22}$}{$\sum_{x=0}^{\tfrac{n}{5}} \sum_{y=0}^{n} 1$}
				\Else
					\State $currentEdgel \gets 0$
					\Cost{$c_{23}$}{$\sum_{x=0}^{\tfrac{n}{5}} \sum_{y=0}^{n} 1$}
				\EndIf
				\If{$p_1 > 0 \land p_1 > p_2 \land p_1 > currentEdgel$}
				\Cost{$c_{24}$}{$\sum_{x=0}^{\tfrac{n}{5}} \sum_{y=0}^{n} 1$}
					\State $edgel \gets \infty$
					\Cost{$c_{25}$}{$\sum_{x=0}^{\tfrac{n}{5}} \sum_{y=0}^{n} 1$}
					\State Koordinaten speichern
					\Cost{$c_{26}$}{$\sum_{x=0}^{\tfrac{n}{5}} \sum_{y=0}^{n} \Theta(1)$}
					\State Orientierung berechnen
					\Cost{$c_{27}$}{$\sum_{x=0}^{\tfrac{n}{5}} \sum_{y=0}^{n} \Theta(1)$}
					\State \Call{addEdgel}{$E,\mathit{edgel}$}
					\Cost{$c_{28}$}{$\sum_{x=0}^{\tfrac{n}{5}} \sum_{y=0}^{n} \Theta(1)$}
				\EndIf
				\State $p_2 \gets p_1$
				\Cost{$c_{29}$}{$\sum_{x=0}^{\tfrac{n}{5}} \sum_{y=0}^{n} 1$}
				\State $p_1 \gets currentEdgel$
				\Cost{$c_{30}$}{$\sum_{x=0}^{\tfrac{n}{5}} \sum_{y=0}^{n} 1$}
				\State $y \gets y + 1$
				\Cost{$c_{31}$}{$\sum_{x=0}^{\tfrac{n}{5}} \sum_{y=0}^{n} 1$}
			\EndFor
			\State $x \gets x + 5$
			\Cost{$c_{32}$}{$\sum_{x=0}^{\tfrac{n}{5}} 1$}
		\EndFor
	\end{algorithmic}
\end{algorithm}

Die Kosten des Algorithmus sind in \autoref{eq:findedgels4} aufgeführt und durch Umformung in \autoref{eq:findedgels5}
 werden die Konstanten isoliert.
\begin{subequations}
\label{eq:findedgels2}
\begin{align}
\label{eq:findedgels2-1}
T(r)& =
c_{21}(\frac{r}{5} + 1)
+ c_{22}2\frac{r}{5}
+ c_{23}\frac{r}{5}(r + 1)
+ c_{24}r\frac{r}{5} \cdot \Theta(1)
+ c_{25}r\frac{r}{5}
+ c_{27}r\frac{r}{5}
\\
& \quad
+ c_{29}r\frac{r}{5}5
+ c_{30}r\frac{r}{5}
+ c_{31}r\frac{r}{5} \cdot \Theta(1)
+ c_{32}r\frac{r}{5} \cdot \Theta(1)
+ c_{33}r\frac{r}{5} \cdot \Theta(1)
\nonumber \\
& \quad
+ c_{35}r\frac{r}{5}
+ c_{36}r\frac{r}{5}
+ c_{37}r\frac{r}{5}
+ c_{39}2\frac{r}{5}
\nonumber \\
\label{eq:findedgels2-2}
T(r)& =
\frac{1}{5} + (c_{21}r + 2c_{22}r + c_{23}r + 2c_{39}r + 5c_{21})
\\
& \quad
+ \frac{r^2}{5} (c_{23} + c_{25} + c_{27} + 5c_{29} + c_{30}+ c_{35} + c_{36} + c_{37})
\nonumber \\
& \quad
+ \frac{r^2}{5}(c_{24} + c_{31} + c_{32} + c_{33}) \Theta(1)
\nonumber
\end{align}
\end{subequations}

Dies führt zu einer Laufzeit von $T(n) = \Theta(n^2)$ für \autoref{alg:findedgels-vertical-analyse}
 (\autoref{eq:findedgels4-end}).
Um die Laufzeit von \textproc{findEdgels} zu bestimmen, werden die Laufzeiten von
 \autoref{alg:findedgels-horizontal-analyse} und \autoref{alg:findedgels-vertical-analyse} in \autoref{eq:findedgels7}
 zusammengefasst.
\begin{subequations}
\begin{align}
\label{eq:findedgels7}
T(n)& = n^2 + n^2\\
\label{eq:findedgels8}
T(n)& = 2 \cdot n^2\\
\label{eq:findedgels9}
T(n)& = n^2
\end{align}
\end{subequations}

Durch Umformung in \autoref{eq:findedgels8} kann die Laufzeit des Algorithmus in \autoref{eq:findedgels9} ermittelt
 werden. Die Laufzeit von \textproc{findEdgels} entspricht demnach $T(n) = \Theta(n^2)$.

\autoref{alg:convolutekernel-horizontal} und \autoref{alg:convolutekernel-vertical} berechnen den Gradienten durch Faltung mit dem Gauß-Kernel
$\left( \begin{smallmatrix}
-3& -5& 0& 5& 3
\end{smallmatrix} \right)$
 auf der horizontalen und vertikalen Scanline. Als Parameter benötigt der Algorithmus den Zeiger des monochromen
 Bildsignals $I_m$, die Position des Pixels ($x$ und $y$), sowie die Breite $w$ und Höhe $h$ von $I_m$. In Zeile
 \ref{alg:convolutekernel-horizontal-readstart}--\ref{alg:convolutekernel-horizontal-readend} werden durch die
 Funktion \textproc{getpixel} (Vgl. \autoref{alg:getpixel}, S. \pageref{alg:getpixel}) die benötigten Pixelwerte
 ausgelesen und den Variablen zugewiesen. Im Anschluss werden die Werte mit dem Gauß-Kernel
$\left( \begin{smallmatrix}
-3& -5& 0& 5& 3
\end{smallmatrix} \right)$
berechnet um den Gradienten zu bestimmen.
\begin{algorithm}[!ht]
\caption{\textproc{convoluteKernelX} (horizontale Scanline)}
\label{alg:convolutekernel-horizontal}
\begin{algorithmic}[1]
	\Require $I_m,x,y,w,h$
	\State $p_1 \gets$ \Call{getpixel}{$I_m, x - 2, y, w, h$}
	\Cost{$c_{1}$}{$2 + \Theta(1)$}
	\label{alg:convolutekernel-horizontal-readstart}
	\State $p_2 \gets$ \Call{getpixel}{$I_m, x - 1, y, w, h$}
	\Cost{$c_{2}$}{$2 + \Theta(1)$}
	% \State $p_3 \gets$ \Call{getpixel}{$I_m, x, y, w, h$}
	\State $p_4 \gets$ \Call{getpixel}{$I_m, x + 1, y, w, h$}
	\Cost{$c_{3}$}{$2 + \Theta(1)$}
	\State $p_5 \gets$ \Call{getpixel}{$I_m, x + 2, y, w, h$}
	\Cost{$c_{4}$}{$2 + \Theta(1)$}
	\label{alg:convolutekernel-horizontal-readend}
	\State $v \gets 0$
	\Cost{$c_{5}$}{$1$}
	\State $v \gets v + \left( -3 \cdot p_1 \right)$
	\Cost{$c_{6}$}{$3$}
	\State $v \gets v + \left( -5 \cdot p_2 \right)$
	\Cost{$c_{7}$}{$3$}
	% \State $v \gets v + \left( 0 \cdot p_3 \right)$
	\State $v \gets v + \left( 5 \cdot p_4 \right)$
	\Cost{$c_{8}$}{$3$}
	\State $v \gets v + \left( 3 \cdot p_5 \right)$
	\Cost{$c_{9}$}{$3$}
	\State \textbf{return} $v$
	\Cost{$c_{10}$}{$1$}
\end{algorithmic}
\end{algorithm}

Bei genauer Betrachtung von \autoref{alg:convolutekernel-horizontal} und \autoref{alg:convolutekernel-vertical}
 fällt auf, dass der Wert $p_3$ in der Berechnung nicht vorkommt.
\begin{algorithm}[!ht]
\caption{\textproc{convoluteKernelY} (vertikale Scanline)}
\label{alg:convolutekernel-vertical}
\begin{algorithmic}[1]
	\Require $I_m,x,y,w,h$
	\State $p_1 \gets$ \Call{getpixel}{$I_m, x, y - 2, w, h$}
	\State $p_2 \gets$ \Call{getpixel}{$I_m, x, y - 1, w, h$}
	% \State $p_3 \gets$ \Call{getpixel}{$I_m, x, y, w, h$}
	\State $p_4 \gets$ \Call{getpixel}{$I_m, x, y + 1, w, h$}
	\State $p_5 \gets$ \Call{getpixel}{$I_m, x, y + 2, w, h$}
	\State $v \gets 0$
	\State $v \gets v + \left( -3 \cdot p_1 \right)$
	\State $v \gets v + \left( -5 \cdot p_2 \right)$
	% \State $v \gets v + \left( 0 \cdot p_3 \right)$
	\State $v \gets v + \left( 5 \cdot p_4 \right)$
	\State $v \gets v + \left( 3 \cdot p_5 \right)$
	\State \textbf{return} $v$
\end{algorithmic}
\end{algorithm}

Dies ist darauf zurückzuführen, dass der Gauß-Kernel an der dritten Stelle mit $0$ definiert ist. Somit kann die
 Multiplikation vernachlässigt werden. Die Laufzeit von \autoref{alg:convolutekernel-horizontal} und
 \autoref{alg:convolutekernel-vertical} ist konstant.

In \autoref{alg:gradientintensity} wird mittels Faltung die Orientierung eines \glspl{edgel} bestimmt. Als
 Eingabeparameter wird das monochrome Bildsignal $I_m$, dessen Breite $w$ und Höhe $h$, sowie die Position des
 \glspl{edgel} ($x,y$) benötigt.
\begin{algorithm}[ht]
\caption{\textproc{gradientIntensity}}
\label{alg:gradientintensity}
\begin{algorithmic}[1]
	\Require $I_m, x, y, w, h$
	\State $p_{x_1} \gets$ \Call{getPixel}{$I_m, x -1, y - 1, w, h$}
	\label{alg:gradientintensity-readstart}
	\State $p_{x_2} \gets$ \Call{getPixel}{$I_m, x, y - 1, w, h$}
	\State $p_{x_3} \gets$ \Call{getPixel}{$I_m, x + 1, y - 1, w, h$}
	\State $p_{x_4} \gets$ \Call{getPixel}{$I_m, x - 1, y + 1, w, h$}
	\State $p_{x_5} \gets$ \Call{getPixel}{$I_m, x, y + 1, w, h$}
	\State $p_{x_6} \gets$ \Call{getPixel}{$I_m, x + 1, y + 1, w, h$}
	\State $p_{y_1} \gets$ \Call{getPixel}{$I_m, x - 1, y - 1, w, h$}
	\State $p_{y_2} \gets$ \Call{getPixel}{$I_m, x - 1, y, w, h$}
	\State $p_{y_3} \gets$ \Call{getPixel}{$I_m, x - 1, y + 1, w, h$}
	\State $p_{y_4} \gets$ \Call{getPixel}{$I_m, x + 1, y - 1, w, h$}
	\State $p_{y_5} \gets$ \Call{getPixel}{$I_m, x + 1, y, w, h$}
	\State $p_{y_6} \gets$ \Call{getPixel}{$I_m, x + 1, y + 1, w, h$}
	\label{alg:gradientintensity-readend}
	\State $g_x \gets 0$
	\State $g_y \gets 0$
	\State $g_x \gets g_x + p_{x_1}$
	\label{alg:gradientintensity-convolutestart}
	\State $g_x \gets g_x + \left(p_{x_2} \cdot 2\right)$
	\State $g_x \gets g_x + p_{x_3}$
	\State $g_x \gets g_x - p_{x_4}$
	\State $g_x \gets g_x - \left(p_{x_5} \cdot 2\right)$
	\State $g_x \gets g_x - p_{x_6}$
	\State $g_y \gets g_y + p_{y_1}$
	\State $g_y \gets g_y + \left(p_{y_2} \cdot 2\right)$
	\State $g_y \gets g_y + p_{y_3}$
	\State $g_y \gets g_y - p_{y_4}$
	\State $g_y \gets g_y - \left(p_{y_5} \cdot 2\right)$
	\State $g_y \gets g_y - p_{y_6}$
	\label{alg:gradientintensity-convoluteend}
	\State $\mathit{slope} \gets \infty$
	\State \Call{vectorSetCoordinate}{$\mathit{slope},\mathit{gy},\mathit{gx}$}
	\label{alg:gradientintensity-vector-start}
	\State \Call{normalized}{$\mathit{slope}$}
	\label{alg:gradientintensity-vector-end}
	\State \textbf{return} $\mathit{slope}$
\end{algorithmic}
\end{algorithm}

In Zeile \ref{alg:gradientintensity-readstart}--\ref{alg:gradientintensity-readend} werden die Pixelwerte ausgelesen
 und den Variablen zugewiesen. In Zeile
 \ref{alg:gradientintensity-convolutestart}--\ref{alg:gradientintensity-convoluteend} erfolgt die Faltung mit dem
 Sobel-Operator\footcite[Vgl.][S.~120--123]{burger05}, dessen Filter
\begin{subequations}
\begin{align}
	H_x =&
	\begin{pmatrix}
		1& 0& -1\\
		2& 0& -2\\
		1& 0& -1
	\end{pmatrix}
\end{align}
\begin{align}
	H_y =&
	\begin{pmatrix}
		1& 2& 1\\
		0& 0& 0\\
		-1& -2& -1
	\end{pmatrix}
\end{align}
\end{subequations}
den Gradienten $G_x$ und $G_y$ bestimmen. Wie in \autoref{alg:convolutekernel-horizontal} werden Multiplikationen von
 $0$-Werten des Filters vernachlässigt. Mit
\begin{equation}
	\label{eq:orientation}
	\Phi(x,y) = \arctan{\left(\tfrac{G_y}{G_x}\right)}
\end{equation}
kann die Orientierung berechnet werden. $G_x$ und $G_y$ werden mit \autoref{alg:vectorsetcoordinate} als
 \textproc{vector} gespeichert und normalisiert
 (Zeile \ref{alg:gradientintensity-vector-start}--\ref{alg:gradientintensity-vector-end}). Die Laufzeit von
 \autoref{alg:gradientintensity} ist konstant.

Um aus den gefundenen \glspl{edgel} Liniensegmente zu erzeugen, wird \autoref{alg:findlinesegments1} verwendet. Das
 Verfahren benötigt den Speicherblock $E$, in dem die \glspl{edgel} vorliegen, und den Speicherblock $L$, der zur
 Speicherung der gefundenen Liniensegmente dient.
\algblockdefx[DOWHILE]{DO}{WHILE}
   {\textbf{do}}
   [1]{\textbf{while} #1}
\begin{algorithm}[ht]
\caption{\textproc{findLineSegments}}
\label{alg:findlinesegments1}
\begin{algorithmic}[1]
	\Require $E, L$
	\State $\mathit{presentLine} \gets \infty$
	\State $\mathit{edgelsInRegion} \gets 0$
	\DO
		\State $\mathit{presentLine.supportCount} \gets 0$
		\For{$i \gets 0$ \textbf{to} $i < 25$}
			\State \ldots \Comment 1. Teil
			\State $i \gets i + 1$
		\EndFor
		\If{$\mathit{presentLine.supportCount} > \mathit{minEdgels}$}
			\State \ldots \Comment 2. Teil
		\EndIf
		\State $\mathit{edgelsInRegion} \gets$ \Call{getEdgelCount}{$E$}
	\WHILE{$\mathit{presentLine.supportCount} > \mathit{minEdgels} \land \mathit{edgelsInRegion} > \mathit{minEdgels}$}
\end{algorithmic}
\end{algorithm}

In Zeile \ref{alg:findlinesegments1-init-start}--\ref{alg:findlinesegments1-init-end} wird die Variable
 $\mathit{presentLine}$ initialisiert. Diese Variable enthält das zu speichernde Liniensegment im Verfahren.
 Die Zählvariable $\mathit{edgelsInRegion}$ wird mit $0$ initialisiert und speichert die Anzahl der \gls{edgel} in
 einer Region. In der Schleife von Zeile \ref{alg:findlinesegments1-loop-start} bis Zeile
 \ref{alg:findlinesegments1-loop-end} wird das RANSAC-Verfahren wiederholt, solange das zu untersuchende Liniensegment
 genügend Unterstützung durch \gls{edgel} besitzt und genügend \gls{edgel} in der Region vorhanden sind. Die Anzahl der
 unterstützenden \gls{edgel} wird in Zeile \ref{alg:findlinesegments1-clearsupport} für jeden Durchlauf gelöscht.
 Danach wird in Zeile \ref{alg:findlinesegments1-line-start}--\ref{alg:findlinesegments1-line-end} ein zufälliges
 Liniensegment in der Region erstellt und untersucht
 (Vgl. \autoref{alg:findlinesegments2}--\autoref{alg:findlinesegments4}). Die Erstellung eines Liniensegments wird
 mehrmahls wiederholt, um das Liniensegment mit den meisten Unterstützungsedgels zu finden. In Zeile
 \ref{alg:findlinesegments1-hasenoughsupport} wird die Anzahl der der unterstützenden \gls{edgel} mit der Anzahl der
 benötigten \gls{edgel} verglichen. Nur wenn genügend \gls{edgel} das Liniensegment unterstützen, wird in
 \autoref{alg:findlinesegments6}--\autoref{alg:findlinesegments8} das Verfahren fortgesetzt und die
 Unterstützungsedgels aus dem Speicherblock $E$ entfernt.

In \autoref{alg:findlinesegments2} wird die Initialisierung der Variablen vorgenommen, die zur Erstellung eines
 Liniensegments benötigt werden.
\algblockdefx[DOWHILE]{DO}{WHILE}
   {\textbf{do}}
   [1]{\textbf{while} #1}
\begin{algorithm}[ht]
\caption{\textproc{findLineSegments} (Initialisierung der Liniensegmentvariablen)}
\label{alg:findlinesegments2}
\begin{algorithmic}[1]
	\State $\mathit{start} \gets \infty$
	\State $\mathit{end} \gets \infty$
	\State $\mathit{maxIteration} \gets 64$
	\State $\mathit{iterator} \gets 0$
	\State $\mathit{first} \gets \infty$
	\State $\mathit{last} \gets \infty$
	\State $\mathit{numberOfEdgels} \gets$ \Call{getEdgelCount}{$E$}
	\algstore{brk-findlinesegments2-init}
\end{algorithmic}
\end{algorithm}

Die Variablen $\mathit{start}$ und $\mathit{end}$ werden als Indizes für \gls{edgel} benutzt. $\mathit{first}$ ist der
 Startedgel und $\mathit{last}$ der Endedgel des Liniensegments. Die Variable $\mathit{maxIteration}$ gibt die maximale
 Anzahl der Interationen an, die $\mathit{iterator}$ durchlaufen kann. In $\mathit{numberOfEdgels}$ wird die Anzahl der
 \gls{edgel} in $E$ gespeichert. Die Laufzeit der Initialisierung ist konstant.

Die Auswahl eines Liniensegments erfolgt mit \autoref{alg:findlinesegments3}. Zeile
 \ref{alg:findlinesegments3-loop-start}--\ref{alg:findlinesegments3-loop-end} ist dafür verantwortlich, solange nach
 einem Liniensegment zu suchen, bis der erste und letzte \gls{edgel} sich unterscheiden oder ihre Orientierung
 zueinander kompatibel ist.
\algblockdefx[DOWHILE]{DO}{WHILE}
   {\textbf{do}}
   [1]{\textbf{while} #1}
\begin{algorithm}[ht]
\caption{\textproc{findLineSegments} (Liniensegment suchen)}
\label{alg:findlinesegments3}
\begin{algorithmic}[1]
	\algrestore{brk-findlinesegments2-init}
	\DO
	\label{alg:findlinesegments3-loop-start}
		\State $\mathit{first} \gets$ \Call{rand} $ \% \mathit{numberOfEdgels}$
		\label{alg:findlinesegments3-first}
		\State $\mathit{last} \gets$ \Call{rand} $ \% \mathit{numberOfEdgels}$
		\label{alg:findlinesegments3-last}
		\State $\mathit{start} \gets$ \Call{getEdgel}{$E,\mathit{first}$}
		\label{alg:findlinesegments3-start}
		\State $\mathit{end} \gets$ \Call{getEdgel}{$E,\mathit{last}$}
		\label{alg:findlinesegments3-end}
		\State $\mathit{iterator} \gets \mathit{iterator} + 1$
		\label{alg:findlinesegments3-inc}
	\WHILE{$(\mathit{first} == \mathit{last} \lor \lnot \textproc{isCompatible}(\mathit{start},\mathit{end})) \land \mathit{iterator} < \mathit{maxIteration}$}
	\label{alg:findlinesegments3-loop-end}
	\algstore{brk-findlinesegments3-dowhile}
\end{algorithmic}
\end{algorithm}

Als letzte Bedingung muss die Anzahl der Iterationen unterhalb des festgelegten Schwellwerts $\mathit{maxIteration}$
 bleiben. In Zeile \ref{alg:findlinesegments3-first} und Zeile \ref{alg:findlinesegments3-last} werden zufällig zwei
 Indizes aus der Menge der vorhandenen \gls{edgel} errechnet. Diese werden in Zeile
 \ref{alg:findlinesegments3-start}--\ref{alg:findlinesegments3-end} verwendet, um die beiden \gls{edgel}
 $\mathit{start}$ und $\mathit{end}$ auszuwählen. Der Iterator wird anschliessend in Zeile
 \ref{alg:findlinesegments3-inc} inkrementiert.

Die Laufzeit der Funktion \textproc{rand} wurde durch eine Regressionsanalyse mit $300$ Testdaten ermittelt. Wenn diese
 Daten in der grafischen Darstellung in \autoref{fig:regression-rand} betrachtet werden, ist ersichtlicht, dass keine
 lineare Abhängigkeit besteht.
\begin{figure}[!ht]
	\centering
	\input{resources/Regression-rand.pdf_tex}
	\caption{Regressionsanalyse von \textproc{rand} mit $300$ Datenpunkte. Der Mittelwert der Daten ist als grüne Linie
	 eingezeichnet und der Median der Daten als rote Linie.}
	\label{fig:regression-rand}
\end{figure}
Dies ist auch dadurch begründet, dass \textproc{rand} keine Eingabedaten benötigt\footcite[Vgl.][]{rand}. Somit
 entspricht die Laufzeit von \textproc{rand} $T(n) = \Theta(1)$ und ist demnach konstant. Die Laufzeit von
 \textproc{getEdgel} (\autoref{alg:edgelpool-getedgel}) ist ebenfalls konstant. Die Laufzeit von
 \autoref{alg:findlinesegments3} ist somit abhängig von der Anzahl der Wiederholungen in Zeile
 \ref{alg:findlinesegments3-loop-end}. Im besten Fall werden bei der ersten Iteration zwei \gls{edgel} gefunden, deren
 Orientierung kompatibel ist. Dann ist die Laufzeit $T_{best}(n) = \Theta(1)$. Im schlimmsten Fall wird
 \autoref{alg:findlinesegments3} $\mathit{maxIteration} = 64$ mal ausgeführt. Die Laufzeit ist dann
 $T_{worst} (n) = \Theta(64)$.

Wenn in \autoref{alg:findlinesegments3} ein Liniensegment erstellt wurde, werden in \autoref{alg:findlinesegments4} die
 \gls{edgel} zur unterstüztung der Hypothese hinzugefügt.
\algblockdefx[DOWHILE]{DO}{WHILE}
   {\textbf{do}}
   [1]{\textbf{while} #1}
\begin{algorithm}[!ht]\small
\caption{\textproc{findLineSegments} (Unterstützungsedgel bestimmen)}
\label{alg:findlinesegments4}
\begin{algorithmic}[1]
	\algrestore{brk-findlinesegments3-dowhile}
	\If{$\mathit{iterator} < \mathit{maxIteration}$}
	\Cost{$c_{12}$}{$1$}
	\label{alg:findlinesegments4-isbelowmax}
		\State $\mathit{segment} \gets \infty$
		\Cost{$c_{13}$}{$1$}
		\label{alg:findlinesegments4-init-start}
		\State $\mathit{segment.start} \gets \mathit{start}$
		\Cost{$c_{14}$}{$2$}
		\State $\mathit{segment.end} \gets \mathit{end}$
		\Cost{$c_{15}$}{$2$}
		\State $\mathit{segment.slope} \gets \mathit{start.slope}$
		\Cost{$c_{16}$}{$3$}
		\State $\mathit{segment.supportCount} \gets 0$
		\Cost{$c_{17}$}{$2$}
		\State $\mathit{segment.remove} \gets$ \textbf{FALSE}
		\Cost{$c_{18}$}{$2$}
		\State $\mathit{segment.startCorner} \gets$ \textbf{FALSE}
		\Cost{$c_{19}$}{$2$}
		\State $\mathit{segment.endCorner} \gets$ \textbf{FALSE}
		\Cost{$c_{20}$}{$2$}
		\label{alg:findlinesegments4-init-end}
		\For{$j \gets 0$ \textbf{to} $j < \mathit{numberOfEdgels}$}
		\Cost{$c_{21}$}{$n + 1$}
		\label{alg:findlinesegments4-loop-start}
			\State $\mathit{supportEdgel} \gets$ \Call{getEdgel}{$E,j$}
			\Cost{$c_{22}$}{$n \cdot \bigl(1 + \Theta(1)\bigr)$}
			\label{alg:findlinesegments4-edgel}
			\If{\Call{isEdgelNearLine}{$\mathit{segment}, \mathit{supportEdgel}$}}
			\Cost{$c_{23}$}{$n \cdot \bigl(1 + \Theta(1)\bigr)$}
			\label{alg:findlinesegments4-isedgelnearline}
				\State \Call{addEdgel}{$\mathit{segment},\mathit{supportCount},\mathit{segment.supportCount}$}
				\Cost{$c_{24}$}{$n \cdot \Theta(1)$}
				\label{alg:findlinesegments4-addedgel}
				\State $\mathit{segment.supportCount} \gets \mathit{segment.supportCount} + 1$
				\Cost{$c_{25}$}{$4n$}
				\label{alg:findlinesegments4-count}
			\EndIf
			\State $j \gets j + 1$
			\Cost{$c_{27}$}{$n$}
			\label{alg:findlinesegments4-inc}
		\EndFor
		\label{alg:findlinesegments4-loop-end}
		\If{$\mathit{segment.supportCount} > \mathit{presentLine.supportCount}$}
		\Cost{$c_{29}$}{$3$}
		\label{alg:findlinesegments4-hasmoresupport}
			\State $\mathit{presentLine} \gets \mathit{segment}$
			\Cost{$c_{30}$}{$1$}
		\EndIf
	\EndIf
\end{algorithmic}
\end{algorithm}

Dazu wird in Zeile \ref{alg:findlinesegments4-isbelowmax} überprüft, ob die maximale Anzahl der Iterationen nicht
 überschritten wurde. Falls dem so ist wird \autoref{alg:findlinesegments4} nicht weiter ausgeführt und die Laufzeit
 wäre $T_{best}(n) = \Theta(1)$. Andernfalls wird in Zeile
 \ref{alg:findlinesegments4-init-start}--\ref{alg:findlinesegments4-init-end} die Variable $\mathit{segment}$ zur
 Speicherung des Liniensegments vorbereitet und der Start- und Endedgel, sowie die Orientierung zugewiesen. Die Anzahl
 der Unterstützungedgels beträgt zu diesem Zeitpunkt noch $0$. In der Schleife in Zeile
 \ref{alg:findlinesegments4-loop-start}--\ref{alg:findlinesegments4-loop-end} werden die \gls{edgel} gezählt, die die
 Linienhypothese unterstützen. Dazu wird in Zeile \ref{alg:findlinesegments4-edgel} ein \gls{edgel} ausgewählt und in
 Zeile \ref{alg:findlinesegments4-isedgelnearline} untersucht, ob der Abstand zur Linie klein genug ist. Wenn nicht,
 wird in Zeile \ref{alg:findlinesegments4-inc} die Laufvariable $j$ inkrementiert und das nächste \gls{edgel}
 ausgewählt. Wenn ein \gls{edgel} nahe genug an dem Liniensegment liegt, wird es in Zeile
 \ref{alg:findlinesegments4-addedgel} dem Liniensegment hinzugefügt und die Anzahl der Unterstützungsedgel wird in
 Zeile \ref{alg:findlinesegments4-count} erhöht. Nachdem alle \gls{edgel} untersucht wurden, wird in Zeile
 \ref{alg:findlinesegments4-hasmoresupport} überprüft, ob die Anzahl der \gls{edgel} des Liniensegments
 $\mathit{segment}$ größer ist als die Anzahl der \gls{edgel} in $\mathit{presentLine}$. Im Falle, dass
 $\mathit{segment}$ mehr Unterstützungsedgel besitzt, wird das Liniensegment in $\mathit{presentLine}$ gespeichert.
 Durch die Wiederholung in \autoref{alg:findlinesegments1} wird sichergestellt, dass das Liniensegment mit der
 größten Unterstützung ausgewählt wird. Die Kosten von \autoref{alg:findlinesegments4} sind in
 \autoref{alg:findlinesegments5} aufgeführt.
\algblockdefx[DOWHILE]{DO}{WHILE}
   {\textbf{do}}
   [1]{\textbf{while} #1}
\begin{algorithm}[!ht]
\caption{\textproc{findLineSegments} (Analyse)}
\label{alg:findlinesegments5}
\begin{algorithmic}[1]
	\If{$\mathit{iterator} < \mathit{maxIteration}$}
	\Cost{$c_{1}$}{$1$}
		\State Initialisierung lokaler Variablen
		\Cost{$c_{2}$}{$1$}
		\For{alle \gls{edgel} in $E$}
		\Cost{$c_{3}$}{$n$}
			\State $\mathit{supportEdgel} \gets$ \Call{getEdgel}{$E,j$}
			\Cost{$c_{4}$}{$\sum_{j=0}^{n - 1} \Theta(1)$}
			\If{\gls{edgel} liegt nahe am Liniensegment}
			\Cost{$c_{5}$}{$\sum_{j=0}^{n - 1} \Theta(1)$}
				\State \gls{edgel} zum Segment hinzufügen
				\Cost{$c_{6}$}{$\sum_{j=0}^{n - 1} \Theta(1)$}
				\State $\mathit{segment.supportCount} \gets \mathit{segment.supportCount} + 1$
				\Cost{$c_{7}$}{$\sum_{j=0}^{n - 1}1$}
			\EndIf
			\State $j \gets j + 1$
			\Cost{$c_{8}$}{$\sum_{j=0}^{n - 1}1$}
		\EndFor
		\label{alg:findlinesegments5-loop-end}
		\If{$\mathit{segment.supportCount} > \mathit{presentLine.supportCount}$}
		\Cost{$c_{9}$}{$1$}
		\label{alg:findlinesegments5-hasmoresupport}
			\State $\mathit{presentLine} \gets \mathit{segment}$
			\Cost{$c_{10}$}{$1$}
		\EndIf
	\EndIf
\end{algorithmic}
\end{algorithm}

Die Methoden \textproc{getEdgel} (\autoref{alg:edgelpool-getedgel}), \textproc{isEdgelNearLine}
 (\autoref{alg:isedgelnearline}) und \textproc{addEdgel} (\autoref{alg:linesegmentaddedgel}) haben eine konstante
 Laufzeit. Somit lassen sich die Kosten zu \autoref{eq:findlinesegments1} zusammenfassen.
\begin{subequations}
\begin{align}
\label{eq:findlinesegments1}
T(n)& = (c_{1} + c_{2} + c_{9} + c_{10}) \cdot 1 + c_{3} \cdot n + (c_{4} + c_{5} + c_{6} + c_{7} + c_{8}) \cdot
 \sum_{j=0}^{n-1} 1\\
\label{eq:findlinesegments2}
T(n)& = (c_{1} + c_{2} + c_{9} + c_{10}) \cdot 1 + c_{3} \cdot n + (c_{4} + c_{5} + c_{6} + c_{7} + c_{8}) \cdot
 n-1\\
\label{eq:findlinesegments3}
T(n)& = n
\end{align}
\end{subequations}

Durch Umformung in \autoref{eq:findlinesegments2} erhalten wir eine Laufzeit von $T_{worst}(n) = \Theta(n)$
 (\autoref{eq:findlinesegments3}).

Nachdem ein Liniensegment mit genügend Unterstützung ausgewählt wurde, kann mit \autoref{alg:findlinesegments6} die
 Start- und Endposition des Liniensegments bestimmt werden.
\algblockdefx[DOWHILE]{DO}{WHILE}
   {\textbf{do}}
   [1]{\textbf{while} #1}
\begin{algorithm}[!ht]
\caption{\textproc{findLineSegments} (Neuer Start- und Endpunkt für y)}
\label{alg:findlinesegments6}
\begin{algorithmic}[1]
	\State $\mathit{start} \gets 0$
	\Cost{$c_{1}$}{$1$}
	\label{alg:findlinesegments6-start}
	\State $\mathit{end} \gets \infty$
	\Cost{$c_{2}$}{$1$}
	\label{alg:findlinesegments6-end}
	\State $\mathit{slope} \gets$ \textproc{vectorSubtract}$\left(
	\begin{aligned}
		&\mathit{presentLine.start.coordinate},\\
		&\mathit{presentLine.end.coordinate}
	\end{aligned}\right)$
	\Cost{$c_{3}$}{$1$}
	\label{alg:findlinesegments6-slope-start}
	\State $\mathit{ortho.x} \gets - \mathit{presentLine.start.slope.y}$
	\Cost{$c_{4}$}{$1$}
	\State $\mathit{ortho.y} \gets \mathit{presentLine.start.slope.x}$
	\Cost{$c_{5}$}{$1$}
	\label{alg:findlinesegments6-slope-end}
	\If{\Call{abs}{$\mathit{slope.x}$} $\leq$ \Call{abs}{$\mathit{slope.y}$}}
	\Cost{$c_{6}$}{$1$}
	\label{alg:findlinesegments6-isxsmaller}
		\For{$k \gets 0$ \textbf{to} $k < \mathit{presentLine.supportCount}$}
		\Cost{$c_{7}$}{$n$}
		\label{alg:findlinesegments6-newstart-start}
			\State $\mathit{edgel} \gets \mathit{presentLine.support}[k]$
			\Cost{$c_{8}$}{$\sum_{k=0}^{n-1}1$}
			\If{$\mathit{edgel.coordinate.y} > start$}
			\Cost{$c_{9}$}{$\sum_{k=0}^{n-1}1$}
				\State $\mathit{start} \gets \mathit{edgel.coordinate.y}$
				\Cost{$c_{10}$}{$\sum_{k=0}^{n-1}1$}
				\State $\mathit{presentLine.start} \gets \mathit{edgel}$
				\Cost{$c_{11}$}{$\sum_{k=0}^{n-1}1$}
			\EndIf
			\If{$\mathit{edgel.coordinate.y} < \mathit{end}$}
			\Cost{$c_{12}$}{$\sum_{k=0}^{n-1}1$}
				\State $\mathit{end} \gets \mathit{edgel.coordinate.y}$
				\Cost{$c_{13}$}{$\sum_{k=0}^{n-1}1$}
				\State $\mathit{presentLine.end} \gets \mathit{edgel}$
				\Cost{$c_{14}$}{$\sum_{k=0}^{n-1}1$}
			\EndIf
			\State $k \gets k + 1$
			\Cost{$c_{15}$}{$\sum_{k=0}^{n-1}1$}
		\EndFor
		\label{alg:findlinesegments6-newstart-end}
	\algstore{brk-findlinesegments6-if}
\end{algorithmic}
\end{algorithm}

Da ein Liniensegment aus zwei zufällig ausgewählten \glspl{edgel} besteht, können diese \gls{edgel} und die
 tatsächlichen Start- und Endpunkte voneinander abweichen (Vgl. \autoref{fig:}) In Zeile
 \ref{alg:findlinesegments6-start}--\ref{alg:findlinesegments6-end} wird dazu die Variable $\mathit{start}$ mit einem
 kleinen Wert, und die Variable $\mathit{end}$ mit einem großen Wert, initialisiert. Die Steigung des Liniensegments
 und die Orhogonale werden in Zeile \ref{alg:findlinesegments6-slope-start}--\ref{alg:findlinesegments6-slope-end}
 berechnet. In Zeile \ref{alg:findlinesegments6-isxsmaller} wird geprüft, ob der Absolutwert der Steigung an Punkt $x$
 kleiner ist als der Punkt $y$. Falls dem so ist, wird in Zeile
 \ref{alg:findlinesegments6-newstart-start}--\ref{alg:findlinesegments6-newstart-end} ein neuer Start- und Endpunkt
 gesucht, indem die $y$-Koordinate aller Unterstützungsedgels des Liniensegments verglichen werden. Die Kosten von
 \textproc{findLineSegments} sind in \autoref{alg:findlinesegments6} aufgelistet. Die Laufzeit des Algorithmus ist
 abhängig von $[k=0,\mathit{presentLine.supportCount}) = n$ und beträgt demnach $T(n) = \Theta(n)$
 (Vgl. \autoref{eq:findlinesegments-startendy}).
\begin{subequations}
\label{eq:findlinesegments-startendy}
\begin{align}
\label{eq:findlinesegments4}
T(n)& = (c_{1} + c_{2} + c_{3} + c_{4} + c_{5} + c_{6}) \cdot 1 + c_{7} \cdot n\\
& \quad + (c_{8} + c_{9} + c_{10} + c_{11} + c_{12} + c_{13} + c_{14} + c_{15}) \cdot \sum_{j=0}^{n-1} 1 \nonumber \\
\label{eq:findlinesegments5}
T(n)& = (c_{1} + c_{2} + c_{3} + c_{4} + c_{5} + c_{6}) \cdot 1 \\
& \quad + (c_{7} + c_{8} + c_{9} + c_{10} + c_{11} + c_{12} + c_{13} + c_{14} + c_{15}) \cdot n \nonumber \\
& \quad - (c_{8} + c_{9} + c_{10} + c_{11} + c_{12} + c_{13} + c_{14} + c_{15}) \cdot 1 \nonumber \\
\label{eq:findlinesegments6}
T(n)& = n
\end{align}
\end{subequations}

Wenn in \autoref{alg:findlinesegments6} in Zeile \ref{alg:findlinesegments6-isxsmaller} der Absolutwert der Steigung an
 Punkt $x$ größer ist als an Punkt $y$, wird in \autoref{alg:findlinesegments7} in Zeile
 \ref{alg:findlinesegments7-newstart-start}--\ref{alg:findlinesegments7-newstart-end} ein neuer Start- und Endpunkt
 gesucht, indem die $x$-Koordiante aller \glspl{edgel} des Liniensegments untersucht und verglichen werden.
\algblockdefx[DOWHILE]{DO}{WHILE}
   {\textbf{do}}
   [1]{\textbf{while} #1}
\begin{algorithm}[!ht]\small
\caption{\textproc{findLineSegments} (Unterstützungsedgel entfernen)}
\label{alg:findlinesegments7}
\begin{algorithmic}[1]
	\algrestore{brk-findlinesegments6-else}
	\State $\mathit{m} \gets$ \textproc{vectorSubtract}$\left(
	\begin{aligned}
		&\mathit{presentLine.end.coordinate},\\
		&\mathit{presentLine.start.coordinate}
	\end{aligned}\right)$
	\Cost{$c_{33}$}{$\bigl(1 + \Theta(1)\bigr)$}
	\State $\mathit{angle} \gets$ \Call{dotProduct}{$\mathit{m},\mathit{ortho}$}
	\Cost{$c_{34}$}{$\bigl(1 + \Theta(1)\bigr)$}
	\If{$\mathit{angle} < 0$}
	\Cost{$c_{35}$}{$1$}
		\State $\mathit{newEnd} \gets \mathit{presentLine.start}$
		\Cost{$c_{36}$}{$2$}
		\label{alg:findlinesegments7-newstart-start}
		\State $\mathit{presentLine.start} \gets \mathit{presentLine.end}$
		\Cost{$c_{37}$}{$3$}
		\State $\mathit{presentLine.end} \gets \mathit{newEnd}$
		\Cost{$c_{38}$}{$2$}
		\label{alg:findlinesegments7-newstart-end}
	\EndIf
	\State $\mathit{m} \gets$ \textproc{vectorSubtract}$\left(
	\begin{aligned}
		&\mathit{presentLine.end.coordinate},\\
		&\mathit{presentLine.start.coordinate}
	\end{aligned}\right)$
	\Cost{$c_{40}$}{$\bigl(1 + \Theta(1)\bigr)$}
	\label{alg:findlinesegments7-save-start}
	\State \Call{normalized}{$\mathit{m}$}
	\Cost{$c_{41}$}{$\Theta(1)$}
	\State $\mathit{presentLine.slope} \gets \mathit{m}$
	\Cost{$c_{42}$}{$2$}
	\label{alg:findlinesegments7-save-end}
	\State \Call{addLineSegment}{$L,\mathit{presentLine}$}
	\Cost{$c_{43}$}{$\Theta(1)$}
	\label{alg:findlinesegments7-addtomempool}
	\State $\mathit{supportCount} \gets \mathit{presentLine.supportCount}$
	\Cost{$c_{44}$}{$1$}
	\For{$i \gets 0$ \textbf{to} $i < \mathit{supportCount}$}
	\Cost{$c_{45}$}{$(n + 1)$}
	\label{alg:findlinesegments7-removeedgel-start}
		\State $\mathit{remove} \gets \mathit{presentLine.support}[i]$
		\Cost{$c_{46}$}{$3n$}
		\State \Call{removeEdgel}{$E,\mathit{remove}$}
		\Cost{$c_{47}$}{$n \cdot \Theta(n)$}
		\State $i \gets i + 1$
		\Cost{$c_{48}$}{$n$}
	\EndFor
	\label{alg:findlinesegments7-removeedgel-end}
\end{algorithmic}
\end{algorithm}

Am Ende von \autoref{alg:findlinesegments7} ist in $\mathit{presentLine}$ ein neuer Start- und Endpunkt gespeichert.
 Auch hier ist die Laufzeit des Algorithmus abhängig von $[k=0,\mathit{presentLine.supportCount}) = n$ und die Laufzeit
 von \autoref{alg:findlinesegments7} somit $T(n) = \Theta(n)$ (Vgl. \autoref{eq:findlinesegments-startendx}).
\begin{subequations}
\label{eq:findlinesegments-startendx}
\begin{align}
\label{eq:findlinesegments7}
T(n)& = c_{16} \cdot n + (c_{17} + c_{18} + c_{19} + c_{20} + c_{21} + c_{22} + c_{23} + c_{24}) \cdot
 \sum_{k=0}^{n-1} 1 \\
\label{eq:findlinesegments8}
T(n)& = (c_{16} + c_{17} + c_{18} + c_{19} + c_{20} + c_{21} + c_{22} + c_{23} + c_{24}) \cdot n \\
& \quad - (c_{17} + c_{18} + c_{19} + c_{20} + c_{21} + c_{22} + c_{23} + c_{24}) \cdot 1 \nonumber \\
\label{eq:findlinesegments9}
T(n)& = n
\end{align}
\end{subequations}

Somit kann die Laufzeit von \autoref{alg:findlinesegments6} und \autoref{alg:findlinesegments7} in
 \autoref{eq:findlinesegments-all} zu $T(n) = \Theta(n)$ zusammengefasst werden.
\begin{subequations}
\label{eq:findlinesegments-all}
\begin{align}
\label{eq:findlinesegments10}
T(n)& = \Theta(n) + \Theta(n)\\
\label{eq:findlinesegments11}
T(n)& = \Theta(n)
\end{align}
\end{subequations}


In \autoref{alg:findlinesegments8} wird nun geprüft, ob der Start- und Endpunkt vertauscht ist. Dazu wird der Winkel
 zwischen dem Liniensegment und seiner Orthogonalen gebildet.
\algblockdefx[DOWHILE]{DO}{WHILE}
   {\textbf{do}}
   [1]{\textbf{while} #1}
\begin{algorithm}[!hb]
\caption{\textproc{findLineSegments} (Unterstützungsedgel entfernen)}
\label{alg:findlinesegments8}
\begin{algorithmic}[1]
	\algrestore{brk-findlinesegments7-else}
	\State $\mathit{m} \gets$ \Call{vectorSubtract}{$\mathit{presentLine.end.coordinate},\mathit{presentLine.start.coordinate}$}
	\State $\mathit{angle} \gets$ \Call{dotProduct}{$\mathit{m},\mathit{ortho}$}
	\If{$\mathit{angle} < 0$}
		\State $\mathit{newEnd} \gets \mathit{presentLine.start}$
		\label{alg:findlinesegments8-newstart-start}
		\State $\mathit{presentLine.start} \gets \mathit{presentLine.end}$
		\State $\mathit{presentLine.end} \gets \mathit{newEnd}$
		\label{alg:findlinesegments8-newstart-end}
	\EndIf
	\State $\mathit{m} \gets$ \Call{vectorSubtract}{$\mathit{presentLine.end.coordinate},\mathit{presentLine.start.coordinate}$}
	\label{alg:findlinesegments8-save-start}
	\State \Call{normalized}{$\mathit{m}$}
	\State $\mathit{presentLine.slope} \gets \mathit{m}$
	\label{alg:findlinesegments8-save-end}
	\State \Call{addLineSegment}{$L,\mathit{presentLine}$}
	\label{alg:findlinesegments8-addtomempool}
	\State $\mathit{supportCount} \gets \mathit{presentLine.supportCount}$
	\For{$i \gets 0$ \textbf{to} $i < \mathit{supportCount}$}
	\label{alg:findlinesegments8-removeedgel-start}
		\State $\mathit{remove} \gets \mathit{presentLine.support}[i]$
		\State \Call{removeEdgel}{$E,\mathit{remove}$}
		\State $i \gets i + 1$
	\EndFor
	\label{alg:findlinesegments7-removeedgel-end}
\end{algorithmic}
\end{algorithm}

Wenn der Winkel kleiner als $0$ ist, werden Start- und Endpunkt in Zeile
 \ref{alg:findlinesegments8-newstart-start}--\ref{alg:findlinesegments8-newstart-end} getauscht. Im Anschluss daran,
 wird in Zeile \ref{alg:findlinesegments8-save-start}--\ref{alg:findlinesegments8-save-end} die Orientierung des
 Liniensegments berechnet und gespeichert. Danach wird das Liniensegment in Zeile
 \ref{alg:findlinesegments8-addtomempool} in den Speicherblock $L$ hinterlegt. Jetzt müssen alle Unterstützungsedgel
 des Liniensegments in Zeile \ref{alg:findlinesegments8-removeedgel-start}--\ref{alg:findlinesegments8-removeedgel-end}
 aus dem Speicherblock $E$ entfernt werden. Das entfernen der Unterstützungsedgel bewirkt, dass entweder das
 RANSAC-Verfahren wiederholt werden kann ohne die gleichen \gls{edgel} erneut zu betrachten, oder, wenn nicht mehr
 genügend \gls{edgel} in der Region vorhanden sind, das Verfahren abzubrechen und die Linienerkennung zu beenden. Die
 Laufzeit von \autoref{alg:findlinesegments8} ist abhängig von dem Bereich $[i = 0,\mathit{supportCount}) = n$. Die
 Kosten lassen sich zu \autoref{eq:findlinesegments12} zusammen fassen. Die Methoden \textproc{vectorSubtract},
 \textproc{dotProduct}, \textproc{normalized} und \textproc{addLineSegment} haben eine konstante Laufzeit.
 \textproc{removeEdgel} hat eine Laufzeit von $T_{worst}(m)=\Theta(m)$, wenn der zu löschende \gls{edgel} am Ende des
 Speicherblocks liegt.
\begin{subequations}
\label{eq:findlinesegments-remove}
\begin{align}
\label{eq:findlinesegments12}
T(n)& = (c_{25} + c_{26} + c_{27} + c_{28} + c_{29} + c_{30} + c_{31} + c_{32} + c_{33} + c_{34} + c_{35}) \cdot 1\\
& \quad + c_{36} \cdot n + (c_{37} + c_{39}) \cdot \sum_{i = 0}^{n - 1} + c_{38} \cdot \sum_{i = 0}^{n-1} m \nonumber \\
\label{eq:findlinesegments13}
T(n)& = (c_{25} + c_{26} + c_{27} + c_{28} + c_{29} + c_{30} + c_{31} + c_{32} + c_{33} + c_{34} + c_{35}) \cdot 1 \\
& \quad + (c_{36} + c_{37} + c_{39}) \cdot n - (c_{37} + c_{39}) \cdot 1 + c_{38} \cdot (n \cdot m) -  c_{38} \cdot m
 \nonumber \\
\label{eq:findlinesegments14}
T(n)& = n + n \cdot m - m\\
\label{eq:findlinesegments15}
T(n)& = n \cdot m
\end{align}
\end{subequations}

Durch Umformung der Gleichung in \autoref{eq:findlinesegments13} und \autoref{eq:findlinesegments14} wird eine Laufzeit
 von $T_{worst}(n)=\Theta(nm)$ ermittelt.

Die Laufzeit des RANSAC Verfahrens (\autoref{alg:findlinesegments1}) ist abhängig von der Anzahl der \gls{edgel} in
 Speicherblock $E$. Für die Analyse des Algorithmus in \autoref{alg:findlinesegments9} wird $n = $ Anzahl der
 \gls{edgel} verwendet.
\algblockdefx[DOWHILE]{DO}{WHILE}
   {\textbf{do}}
   [1]{\textbf{while} #1}
\begin{algorithm}[!ht]
\caption{\textproc{findLineSegments} (Analyse im günstigsten Fall)}
\label{alg:findlinesegments9}
\begin{algorithmic}[1]
	\Require $E, L$
	\State $\mathit{presentLine} \gets \infty$
	\Cost{$c_{1}$}{$1$}
	\State $\mathit{edgelsInRegion} \gets 0$
	\Cost{$c_{2}$}{$1$}
	\DO
		\State $\mathit{presentLine.supportCount} \gets 0$
		\Cost{$c_{3}$}{$1$}
		\For{$i \gets 0$ \textbf{to} $i < 25$}
		\label{alg:findlinesegments9-loop-start}
		\Cost{$c_{4}$}{$25 + 1$}
			\State \autoref{alg:findlinesegments2}
			\Cost{$c_{5}$}{$\sum_{i=0}^{25}1$}
			\State \autoref{alg:findlinesegments3}
			\Cost{$c_{6}$}{$\sum_{i=0}^{25}64$}
			\State \autoref{alg:findlinesegments4}
			\Cost{$c_{7}$}{$\sum_{i=0}^{25}n$}
			\State $i \gets i + 1$
			\Cost{$c_{8}$}{$\sum_{i=0}^{25}1$}
		\EndFor
		\label{alg:findlinesegments9-loop-end}
		\If{$\mathit{presentLine.supportCount} > \mathit{minEdgels}$}
		\label{alg:findlinesegments9-failure}
		\Cost{$c_{9}$}{$1$}
			\State \autoref{alg:findlinesegments6}
			% \Cost{$c_{10}$}{$n$}
			\State \autoref{alg:findlinesegments7}
			% \Cost{$c_{11}$}{$n$}
			\State \autoref{alg:findlinesegments8}
			% \Cost{$c_{12}$}{$nm$}
		\EndIf
		\State $\mathit{edgelsInRegion} \gets$ \Call{getEdgelCount}{$E$}
		\label{alg:findlinesegments9-edgels}
		\Cost{$c_{13}$}{$1$}
	\WHILE{$\left(
	\begin{aligned}
		& \quad \mathit{presentLine.supportCount} > \mathit{minEdgels}\\
		& \land \mathit{edgelsInRegion} > \mathit{minEdgels}
	\end{aligned}\right)$}
	\Cost{$c_{14}$}{$1$}
	\label{alg:findlinesegments9-break}
\end{algorithmic}
\end{algorithm}

Die Methoden \textproc{vectorSubtract}, \textproc{dotProduct}, \textproc{normalized} und \textproc{addLineSegment}
 haben eine konstante Laufzeit. Die Abbruchbedingung der Schleife ist in Zeile \ref{alg:findlinesegments9-break} von
 \autoref{alg:findlinesegments9} angegeben. Sie besagt, dass die Anzahl der gefundenen Unterstützungsedgel größer sein
 muss als eine festgelegte Mindestanzahl. Zur Untersuchung wird die Mindestanzahl $\mathit{minEdgels} = 0$ festgelegt.
 Somit wird das Verfahren beendet, wenn ein Liniensegment keine Unterstüztungsedgel enthält. Im besten Fall wird das
 Verfahren in \autoref{alg:findlinesegments9} Zeile
 \ref{alg:findlinesegments9-loop-start}--\ref{alg:findlinesegments9-loop-end} ausführen und ein Liniensegment ohne
 Unterstützungsedgel finden. Dadurch wird die Überprüfung in Zeile \ref{alg:findlinesegments9-failure} einmalig
 ausgeführt und danach in Zeile \ref{alg:findlinesegments9-edgels} die Anzahl der \gls{edgel} in die lokale Variable
 schreiben. In Zeile \ref{alg:findlinesegments9-break} wird nach einmaliger Überprüfung das Verfahren beendet. Die
 Kosten des Verfahrens sind in \autoref{eq:findlinesegments16} aufgelistet.
\begin{subequations}
\label{eq:findlinesegments-best}
\begin{align}
\label{eq:findlinesegments16}
T_{best}(n)& = (c_{1} + c_{2} + c_{3} + c_{9} + c_{13} + c_{14}) \cdot 1 + c_{4} \cdot (25 + 1) + (c_{5} + c_{8})
 \cdot \sum_{i=0}^{25} 1\\
& \quad + c_{6} \cdot \sum_{i=0}^{25} \cdot 64 + c_{7} \cdot \sum_{i=0}^{25} n \nonumber \\
\label{eq:findlinesegments17}
T_{best}(n)& = (c_{1} + c_{2} + c_{3} + c_{4} + c_{9} + c_{13} + c_{14}) \cdot 1 + (c_{4} + c_{5} + c_{8}) \cdot 25 \\
& \quad + c_{6} \cdot (25 \cdot 64) + c_{7} \cdot (25 \cdot n) \nonumber \\
\label{eq:findlinesegments18}
T_{best}(n)& = 1 + 25 + (25 \cdot 64) + 25 \cdot n\\
\label{eq:findlinesegments19}
T_{best}(n)& = n
\end{align}
\end{subequations}

Die Laufzeit beträgt, nach der Umformung und Isolierung der Konstanten in \autoref{eq:findlinesegments17} und
 \autoref{eq:findlinesegments18}, $T_{best}(n) = n$ (Vgl. \autoref{eq:findlinesegments19}).

Für die Untersuchung des schlechtesten Falls (\autoref{alg:findlinesegments10}), muss die Anzahl der Wiederholungen der
 Schleife in Zeile \ref{alg:findlinesegments10-do}--\ref{alg:findlinesegments10-while} ermittelt werden.
\algblockdefx[DOWHILE]{DO}{WHILE}
   {\textbf{do}}
   [1]{\textbf{while} #1}
\begin{algorithm}[!ht]\small
\caption{\textproc{findLineSegments} (Analyse im schlechtesten Fall)}
\label{alg:findlinesegments10}
\begin{algorithmic}[1]
	\Require $E, L$
	\State $\mathit{presentLine} \gets \infty$
	\Cost{$c_{1}$}{$1$}
	\State $\mathit{edgelsInRegion} \gets 0$
	\Cost{$c_{2}$}{$1$}
	\DO
	\label{alg:findlinesegments10-do}
		\State $\mathit{presentLine.supportCount} \gets 0$
		\Cost{$c_{3}$}{$\sum_{j=1}^{m}1$}
		\For{$i \gets 0$ \textbf{to} $i < 25$}
		\Cost{$c_{4}$}{$\sum_{j=1}^{{m}} 25 + 1$}
			\State \autoref{alg:findlinesegments2}
			\Cost{$c_{5}$}{$\sum_{j=1}^{m}\sum_{i=0}^{25}1$}
			\State \autoref{alg:findlinesegments3}
			\Cost{$c_{6}$}{$\sum_{j=1}^{m}\sum_{i=0}^{25}64$}
			\State \autoref{alg:findlinesegments4}
			\Cost{$c_{7}$}{$\sum_{j=1}^{m}\sum_{i=0}^{25}n$}
			\State $i \gets i + 1$
			\Cost{$c_{8}$}{$\sum_{j=1}^{m}\sum_{i=0}^{25}1$}
		\EndFor
		\If{$\mathit{presentLine.supportCount} > \mathit{minEdgels}$}
		\label{alg:findlinesegments10-hasmoresupport}
		\Cost{$c_{9}$}{$\sum_{j=1}^{m}1$}
		\label{alg:findlinesegments10-remove-start}
			\State \autoref{alg:findlinesegments6}
			\Cost{$c_{10}$}{$\sum_{j=1}^{m}j$}
			\State \autoref{alg:findlinesegments7}
			\Cost{$c_{11}$}{$\sum_{j=1}^{m}j$}
			\State \autoref{alg:findlinesegments8}
			\Cost{$c_{12}$}{$\sum_{j=1}^{m}jn$}
		\EndIf
		\label{alg:findlinesegments10-remove-end}
		\State $\mathit{edgelsInRegion} \gets$ \Call{getEdgelCount}{$E$}
		\Cost{$c_{13}$}{$\sum_{j=1}^{m}1$}
	\WHILE{$\left(
	\begin{aligned}
		& \quad \mathit{presentLine.supportCount} > \mathit{minEdgels}\\
		& \land \mathit{edgelsInRegion} > \mathit{minEdgels}
	\end{aligned}\right)$}
	\Cost{$c_{14}$}{$m$}
	\label{alg:findlinesegments10-while}
\end{algorithmic}
\end{algorithm}

Zeile \ref{alg:findlinesegments10-hasmoresupport} überprüft, ob ein neues Liniensegment eine größere Anzahl von
 Unterstützungsedgel besitzt, als ein bereits gefundenes Liniensegment. Bei der ersten Iteratition der Schleife muss
 ein Liniensegment mit einem Unterstützungsedgel gefunden werden. Danach wird der Unterstützungsedgel und der Start-
 und Endedgel des Liniensegments von \autoref{alg:findlinesegments8} in Zeile
 \ref{alg:findlinesegments8-removeedgel-start}--\ref{alg:findlinesegments8-removeedgel-end} gelöscht. Bei der zweiten
 Iteration müssen mindestens zwei Unterstützungsedgel für ein Liniensegment gefunden werden. Danach werden vier
 \gls{edgel} entfernt. Somit kann durch
\begin{equation}
\label{eq:linesegments-nedgels}
\sum_{i=1}^{m} i + 2 = n
\end{equation}
die Anzahl $n$ der \gls{edgel} bestimmt werden, die für $m$ Iterationen benötigt werden. Durch Umformung von \autoref{eq:linesegments-nedgels} zu
\begin{equation}
\label{eq:linesegments-miteration}
m = \left\lfloor\frac{1}{2}(\sqrt{8n+25} - 5)\right\rfloor
\end{equation}
lässt sich die maximale Anzahl $m$ der Wiederholung ermitteln, die mit $n$ \gls{edgel} möglich sind. Die Laufzeit von
 \autoref{alg:findlinesegments6} und \autoref{alg:findlinesegments7} sind abhängig von dem Bereich
 $[k=0,supportCount)$. Dies entspricht $\sum_{j=1}^{m}j$ für $m$-Wiederholungen.  Die Laufzeit von
 \autoref{alg:findlinesegments8} ist, wie \autoref{alg:findlinesegments6} und \autoref{alg:findlinesegments7}, von dem
 Bereich $[k=0,supportCount)$ abhängig. Zusätzlich muss ein zu löschendes \gls{edgel} in der Menge $n$ gesucht werden.
 Die Kosten der Algorithmen sind zur Analyse des schlechtesten Falls in \autoref{alg:findlinesegments10} und
 \autoref{eq:findlinesegments20} aufgeführt.
\begin{subequations}
\label{eq:findlinesegments-worst}
\begin{align}
\label{eq:findlinesegments20}
T_{worst}(n)& = (c_{1} + c_{2}) \cdot 1 + (c_{3} + c_{9} + c_{13}) \cdot \sum_{j=1}^{m}1 + c_{4} \cdot \sum_{j=1}^{m}
 (25 + 1) \\
& \quad + (c_{5} + c_{8}) \cdot \sum_{j=1}^{m} \sum_{i=0}^{25} 1 + c_{6} \cdot \sum_{j=1}^{m} \sum_{i=0}^{25} 64 +
 c_{7} \cdot \sum_{j=1}^{m} \sum_{i=0}^{25} n \nonumber \\
& \quad + (c_{10} + c_{11})\cdot \sum_{j=1}^{m}j + c_{12} \cdot \sum_{j=1}^{m} j \cdot n + c_{14} \cdot m \nonumber \\
\label{eq:findlinesegments21}
T_{worst}(n)& = (c_{1} + c_{2}) \cdot 1 + (c_{3} + c_{4} + c_{9} + c_{13} + c_{14}) \cdot m \\
& \quad + (c_{4} + c_{5} + c_{8}) \cdot (25 \cdot m) + c_{6} \cdot \bigl(25 \cdot (64 \cdot m)\bigr) + c_{7} \cdot \bigl(25 \cdot (m \cdot
 n)\bigr) \nonumber \\
& \quad + (c_{10} + c_{11}) \cdot \bigl(\frac{1}{2} \cdot m \cdot (m + 1)\bigr) + c_{12} \cdot \bigl(\frac{1}{2} \cdot m \cdot (m + 1) \cdot n\bigr) \nonumber \\
\label{eq:findlinesegments22}
T_{worst}(n)& = 1 + m + 25 \cdot m + 25 \cdot (64 \cdot m) + 25 \cdot (m \cdot n) \\
& \quad + \frac{1}{2} \cdot m \cdot (m + 1) + \frac{1}{2} \cdot m \cdot (m + 1) \cdot n \nonumber \\
\label{eq:findlinesegments23}
T_{worst}(n)& = 4 \cdot m + m \cdot n + m^2 + m^2 \cdot n \\
\label{eq:findlinesegments24}
T_{worst}(n)& = m^2 + m^2 \cdot n
\end{align}
\end{subequations}

In \autoref{eq:findlinesegments21}--\autoref{eq:findlinesegments24} werden die Konstanten isoliert und Werte niedriger
 Ordnung vernachlässigt. Durch Substitution von \autoref{eq:linesegments-miteration} in \autoref{eq:findlinesegments24}
 zu \autoref{eq:findlinesegments25} kann die Laufzeit, durch Isolierung der Konstanten und Eliminierung von Werten
 niedriger Ordnung in \autoref{eq:findlinesegments26}, auf $T_{worst}=\Theta(n^2)$ bestimmt werden
 (\autoref{eq:findlinesegments27}).
\begin{subequations}
\begin{align}
\label{eq:findlinesegments25}
T_{worst}(n)& = \bigl(\frac{1}{2} \cdot (\sqrt{8 \cdot n + 25} - 5)\bigr)^2 + \bigl(\frac{1}{2} \cdot
 (\sqrt{8 \cdot n + 25} - 5)\bigr)^2 \cdot n \\
\label{eq:findlinesegments26}
T_{worst}(n)& = \frac{1}{2} \cdot \bigl(4 \cdot n^2 + 29 \cdot n - 5 \cdot (n + 1) \cdot
 \sqrt{8 \cdot n + 25} + 25\bigr) \\
\label{eq:findlinesegments27}
T_{worst}(n)& = n^2
\end{align}
\end{subequations}


\begin{algorithm}[!ht]\small
\caption{\textproc{lineDetection} (Analyse)}
\label{alg:linedetection-clarke-analyse}
	\begin{algorithmic}[1]
		\Require $I_m$
		\For{$y \gets 0$ \textbf{to} $y < \mathit{imageHeight}$}
		\label{alg:linedetection-clarke-analyse-loopy-start}
		\Cost{$c_{1}$}{$(\frac{h}{r} + 1)$}
			\For{$x \gets 0$ \textbf{to} $x < \mathit{imageWidth}$}
			\label{alg:linedetection-clarke-analyse-loopx-start}
			\Cost{$c_{2}$}{$\frac{h}{r}(\frac{w}{r} + 1)$}
				\State \Call{findEdgels}{$I_m,E,x,y$}
				\Cost{$c_{3}$}{$\frac{hw}{r^2} \cdot \Theta(r^2)$}
				\State \Call{findLineSegments}{$E,L$}
				\Cost{$c_{4}$}{$\frac{hw}{r^2} \cdot \Theta(mn^2)$}
				\State \Call{resetMemoryPool}{$E$}
				\Cost{$c_{5}$}{$\frac{hw}{r^2} \cdot \Theta(1)$}
				\State \Call{resetMemoryPool}{$L$}
				\Cost{$c_{6}$}{$\frac{hw}{r^2} \cdot \Theta(1)$}
				\State $ x \gets x + \mathit{r}$
				\Cost{$c_{7}$}{$2\frac{hw}{r^2}$}
			\EndFor
			\label{alg:linedetection-clarke-analyse-loopx-end}
			\State $y \gets y + \mathit{r}$
			\Cost{$c_{9}$}{$2\frac{h}{r}$}
		\EndFor
		\label{alg:linedetection-clarke-analyse-loopy-end}
	\end{algorithmic}
\end{algorithm}

Die Eingabemenge zur Analyse der Laufzeit von \autoref{alg:linedetection-clarke} entspricht
 $n = I_m = \mathit{imageWidth} \cdot \mathit{imageHeight}$. Die Eingabemenge für \textproc{findEdgels} ist $n$ für  Laufzeit beträgt $T(regionSize) = \Theta(regionSize^2)$. Es werden $\tfrac{2}{5}regionSize^2$ Pixel untersucht, sodass \textproc{findEdgels} maximal $\tfrac{1}{5}(regionSize^2 - 2)$ \gls{edgel} finden kann (Bedingung in Zeile \ref{} horizontal und Zeile \ref{} vertikal). Die Laufzeit von \textproc{findLineSegments} beträgt $T(n) = \Theta(n^2)$ für $n = $ Anzahl der \gls{edgel}. Somit beträgt die Laufzeit in der Untersuchung $T\bigl(\tfrac{1}{5}(regionSize^2 - 2)\bigr) = \Theta\bigl(\tfrac{1}{25}(regionSize^2 - 2)^2\bigr)$ Die Kosten sind in \autoref{} aufgeführt, wobei $\mathit{regionSize}$ durch $r$ zur besseren Lesbarkeit ersetzt wurde.
\begin{subequations}
\label{eq:linedetection-clarke}
\begin{align}
\label{eq:linedetection-clarke-1}
T_{worst}(wh)& =
c_{1}(\frac{h}{r} + 1)
+ c_{2}\frac{h}{r}(\frac{w}{r} + 1)
+ c_{3}\frac{hw}{r^2} \cdot \Theta(r^2)
+ c_{4}\frac{hw}{r^2} \cdot \Theta(mn^2)
\\
& \quad
+ c_{5}\frac{hw}{r^2} \cdot \Theta(1)
+ c_{6}\frac{hw}{r^2} \cdot \Theta(1)
+ c_{7}\frac{2hw}{r^2}
+ c_{9}\frac{2h}{r}
\nonumber \\
\label{eq:linedetection-clarke-2}
T_{worst}(wh)& =
c_{1}(\frac{h}{r} + 1)
+ c_{2}\frac{h}{r}(\frac{w}{r} + 1)
+ c_{3}\frac{hw\cdot\Theta(r^2)}{r^2}
+ c_{4}\frac{hw\cdot\Theta(mn^2)}{r^2}
\\
& \quad
+ (c_{5} + c_{6})\frac{hw}{r^2} \cdot \Theta(1)
+ c_{7}\frac{2hw}{r^2}
+ c_{9}\frac{2h}{r}
\nonumber \\
\label{eq:linedetection-clarke-3}
T_{worst}(wh)& = \frac{1}{r^2}(hwmn^2) + \frac{5}{r^2}(hw) + \frac{4}{r}(h) + hw + 1
\\
\label{eq:linedetection-clarke-4}
T_{worst}(wh)& = \frac{1}{1600}(hwmn^2) + \frac{5}{1600}(hw) + \frac{4}{40}(h) + hw + 1
\end{align}
\end{subequations}

% subsection linienerkennung_nach_clarke96 (end)

\subsection{Linienerkennung nach \texorpdfstring{\citeauthor{hirzer08}}{Hirzer}} % (fold)
\label{sub:linienerkennung_nach_hirzer08}
Das Verfahren zur Linienerkennung nach \citeauthor{hirzer08} basiert, wie in \autoref{sub:line_detection} bereits
 erwähnt, auf dem Verfahren von \citeauthor{clarke96}. \citeauthor{hirzer08} verwendet in seinem Verfahren anstatt
 eines monochromen Bildsignals ein RGB-Signal. Dadurch ändern sich einige Algorithmen geringfügig und werden im
 folgenden Abschnitt erläutert.

Das Verfahren in \autoref{alg:linedetection-hirzer} unterscheidet sich von \autoref{alg:linedetection-clarke}, dass
 anstatt dem monochromen Signal $I_m$ das RGB-Signal $I$ verwendet wird. Dadurch verändern sich auch die Verfahren
 \textproc{findEdgels} und \textproc{findLineSegments}.
\begin{algorithm}[!ht]\small
\caption{\textproc{lineDetection} nach Hirzer}
\label{alg:linedetection-hirzer}
	\begin{algorithmic}[1]
		\Require $I$
		\For{$y \gets 0$ \textbf{to} $y < \mathit{imageHeight}$}
			\For{$x \gets 0$ \textbf{to} $x < \mathit{imageWidth}$}
				\State \Call{findEdgels}{$I,E,x,y$}
				\State \Call{findLineSegments}{$E,L$}
				\State \ldots \Comment Speichern der Liniensegmente zur weiteren Verarbeitung
				\State \Call{resetMemoryPool}{$E$}
				\State \Call{resetMemoryPool}{$L$}
				\State $ x \gets x + 40$
			\EndFor
			\State $y \gets y + 40$
		\EndFor
	\end{algorithmic}
\end{algorithm}


In dem Verfahren zur Erstellung von \gls{edgel} in \autoref{alg:findedgelshirzer-1}--\autoref{alg:findedgelshirzer-2}
\begin{algorithm}[ht]
\caption{\textproc{findEdgels} (Hirzer)}
\label{alg:findedgelshirzer-1}
	\begin{algorithmic}[1]
		\Require $I, E,\mathit{left},\mathit{top}$
		\For{$y \gets \mathit{top}$ \textbf{to} $y < \mathit{top} + \mathit{regionSize}$}
			\State $\mathit{prev1} \gets 0$
			\State $\mathit{prev2} \gets 0$
			\For{$x \gets \mathit{left}$ \textbf{to} $x < \mathit{left} + \mathit{regionSize}$}
				\State $\mathit{currentEdgel} \gets$ \Call{convoluteKernelX}{$I,x,y,\mathit{blue},\mathit{imageWidth}, \mathit{imageHeight}$}
				\State $\mathit{red} \gets$ \Call{convoluteKernelX}{$I,x,y,\mathit{red},\mathit{imageWidth}, \mathit{imageHeight}$}
				\State $\mathit{green} \gets$ \Call{convoluteKernelX}{$I,x,y,\mathit{green},\mathit{imageWidth}, \mathit{imageHeight}$}
				\If{$\mathit{currentEdgel} > \mathit{threshold} \land \mathit{red} > \mathit{threshold} \land \mathit{green} > \mathit{threshold}$}
				\Comment Möglicherweise ein Edgel
				\Else
					\State $\mathit{currentEdgel} \gets 0$
				\EndIf
				\If{$\mathit{prev1} > 0 \land \mathit{prev2} > \mathit{prev1} \land \mathit{prev1} > \mathit{currentEdgel}$}
					\State $\mathit{edgel} \gets \infty$
					\State \Call{vectorSetCoordinate}{$\mathit{edgel},x - 1,y$}
					\State $\mathit{edgel.slope} \gets$ \Call{gradientIntensity}{$I_m, \mathit{imageWidth}, \mathit{imageHeight}, x - 1,y$}
					\State \Call{addEdgel}{$E,\mathit{edgel}$}
				\EndIf
				\State $\mathit{prev2} \gets \mathit{prev1}$
				\State $\mathit{prev1} \gets \mathit{currentEdgel}$
				\State $x \gets x + 1$
			\EndFor
			\State $y \gets y + 5$
		\EndFor
		\algstore{brk-findedgels-x}
	\end{algorithmic}
\end{algorithm}

wird im Gegensatz zu \autoref{alg:findedgels-horizontal} die Faltung für alle drei Farbkanäle durchgeführt (Vgl. Zeile
 \ref{alg:findedgelshirzer-1-color-start}--\ref{alg:findedgelshirzer-1-color-end} in \autoref{alg:findedgelshirzer-1}
 und Zeile \ref{alg:findedgelshirzer-2-color-start}--\ref{alg:findedgelshirzer-2-color-end} in
 \autoref{alg:findedgelshirzer-2}).
\begin{algorithm}[!ht]
\caption{\textproc{findEdgels} (Fortsetzung)}
\label{alg:findedgelshirzer-2}
	\begin{algorithmic}[1]
		\algrestore{brk-findedgels-x}
		\For{$x \gets \mathit{left}$ \textbf{to} $x < \mathit{left} + \mathit{regionSize}$}
			\State $\mathit{prev1} \gets 0$
			\State $\mathit{prev2} \gets 0$
			\For{$y \gets \mathit{top}$ \textbf{to} $y < \mathit{top} + \mathit{regionSize}$}
				\State $\mathit{currentEdgel} \gets$ \textproc{convoluteKernelY}
				$\left(
				\begin{aligned}
					& I,x,y,\mathit{blue},\\
					& \mathit{imageWidth},\mathit{imageHeight}
				\end{aligned}\right)$
				\label{alg:findedgelshirzer-2-color-start}
				\State $\mathit{red} \gets$ \Call{convoluteKernelY}{$I,x,y,\mathit{red},\mathit{imageWidth},
				 \mathit{imageHeight}$}
				\State $\mathit{green} \gets$ \Call{convoluteKernelY}{$I,x,y,\mathit{green},\mathit{imageWidth},
				 \mathit{imageHeight}$}
				\label{alg:findedgelshirzer-2-color-end}
				\If{$\mathit{currentEdgel} > \mathit{threshold} \land \mathit{red} > \mathit{threshold} \land
				 \mathit{green} > \mathit{threshold}$}
					\State Möglicherweise ein Edgel
				\Else
					\State $\mathit{currentEdgel} \gets 0$
				\EndIf
				\If{$\mathit{prev1} > 0 \land \mathit{prev2} > \mathit{prev1} \land \mathit{prev1} >
				 \mathit{currentEdgel}$}
				\State $\mathit{edgel} \gets \infty$
				\State \Call{vectorSetCoordinate}{$\mathit{edgel},x,y - 1$}
				\State $\mathit{edgel.slope} \gets$ \textproc{gradientIntensity}
					$\left(
					\begin{aligned}
						& I_m, \mathit{imageWidth},\\
						& \mathit{imageHeight}, x,y - 1
					\end{aligned}\right)$
				\State \Call{addEdgel}{$E,\mathit{edgel}$}
				\EndIf
				\State $\mathit{prev2} \gets \mathit{prev1}$
				\State $\mathit{prev1} \gets \mathit{currentEdgel}$
				\State $y \gets y + 1$
			\EndFor
			\State $y \gets y + 5$
		\EndFor
	\end{algorithmic}
\end{algorithm}


Betrachtet man die Methode \textproc{convoluteKernelX} und \textproc{convoluteKernelY} stellt man fest, dass hier eine
 Angabe zum Farbkanal erfolgt. In \autoref{alg:convolutekernelhirzer-horizontal} und
\begin{algorithm}[!ht]\small
\caption{\textproc{convoluteKernelX} (horizontale Scanline nach Hirzer)}
\label{alg:convolutekernelhirzer-horizontal}
\begin{algorithmic}[1]
	\Require $I,x,y,\mathit{color},w,h$
	\State $p_1 \gets$ \Call{getRGBValue}{$I,x-2,y,\mathit{color},w,h$}
	\label{alg:convolutekernelhirzer-horizontal-readstart}
	\State $p_2 \gets$ \Call{getRGBValue}{$I,x-1,y,\mathit{color},w,h$}
	\State $p_4 \gets$ \Call{getRGBValue}{$I,x+1,y,\mathit{color},w,h$}
	\State $p_5 \gets$ \Call{getRGBValue}{$I,x+2,y,\mathit{color},w,h$}
	\label{alg:convolutekernelhirzer-horizontal-readend}
	\State $v \gets 0$
	\State $v \gets v + \left( -3 \cdot p_1 \right)$
	\State $v \gets v + \left( -5 \cdot p_2 \right)$
	% \State $v \gets v + \left( 0 \cdot p_3 \right)$
	\State $v \gets v + \left( 5 \cdot p_4 \right)$
	\State $v \gets v + \left( 3 \cdot p_5 \right)$
	\State \textbf{return} $v$
\end{algorithmic}
\end{algorithm}

 \autoref{alg:convolutekernelhirzer-vertical} wird durch den Aufruf der Methode \textproc{getRGBValue} ein bestimmter
 Farbkanal betrachtet.
\begin{algorithm}[!ht]
\caption{\textproc{convoluteKernelY} (vertikale Scanline nach Hirzer)}
\label{alg:convolutekernelhirzer-vertical}
\begin{algorithmic}[1]
	\Require $I,x,y,\mathit{color},w,h$
	\State $p_1 \gets$ \Call{getRGBValue}{$I,x,y-2,\mathit{color},w,h$}
	\State $p_2 \gets$ \Call{getRGBValue}{$I,x,y-1,\mathit{color},w,h$}
	\State $p_4 \gets$ \Call{getRGBValue}{$I,x,y+1,\mathit{color},w,h$}
	\State $p_5 \gets$ \Call{getRGBValue}{$I,x,y+2,\mathit{color},w,h$}
	\State $v \gets 0$
	\State $v \gets v + \left( -3 \cdot p_1 \right)$
	\State $v \gets v + \left( -5 \cdot p_2 \right)$
	% \State $v \gets v + \left( 0 \cdot p_3 \right)$
	\State $v \gets v + \left( 5 \cdot p_4 \right)$
	\State $v \gets v + \left( 3 \cdot p_5 \right)$
	\State \textbf{return} $v$
\end{algorithmic}
\end{algorithm}


Auch das Verfahren \textproc{gradientIntensity} in \autoref{alg:gradientintensityhirzer} verwendet die Methode
 \textproc{getRGBValue}.

\input{alg/analyse-hirzer/gradientIntensityhirzer}

\textproc{getRGBValue} dient dem auslesen eines \gls{pixel} für einen angegeben Farbkanal. In \autoref{alg:getrgbvalue} ist das Verfahren beschrieben.

\begin{algorithm}[!ht]
\caption{\textproc{getRGBValue}}
\label{alg:getrgbvalue}
\begin{algorithmic}[1]
	\Require $I,x,y,\mathit{color},w,h$
	\If{$x < 0$}
	\Cost{$c_{1}$}{$1$}
	\label{alg:getrgbvalue-sanity-start}
		\State $x \gets 0$
		\Cost{$c_{2}$}{$1$}
	\EndIf
	\If{$y < 0$}
	\Cost{$c_{4}$}{$1$}
		\State $y \gets 0$
		\Cost{$c_{5}$}{$1$}
	\EndIf
	\If{$x \geq w$}
	\Cost{$c_{7}$}{$1$}
		\State $x \gets w - 1$
		\Cost{$c_{8}$}{$2$}
	\EndIf
	\If{$y \geq h$}
	\Cost{$c_{10}$}{$1$}
		\State $y \gets h -1$
		\Cost{$c_{11}$}{$2$}
	\EndIf
	\label{alg:getrgbvalue-sanity-end}
	\State $\mathit{offset} \gets (x + (y \cdot w)) \cdot \left(\textproc{sizeof}(\mathit{char}) \cdot 4\right)$
	\Cost{$c_{13}$}{$6$}
	\label{alg:getrgbvalue-offset}
	\State $\mathit{address} \gets I + \mathit{offset}$
	\Cost{$c_{14}$}{$2$}
	\label{alg:getrgbvalue-address}
	\State \textbf{return} $\mathit{address} + \mathit{color}$
	\Cost{$c_{13}$}{$2$}
	\label{alg:getrgbvalue-returncolor}
\end{algorithmic}
\end{algorithm}


Das Verfahren benötigt das RGB-Signal $I$, die Position des \gls{pixel} ($x$, $y$), den Farbkanal, sowie die Breite $w$
 und Höhe $h$ von $I$. In den Zeilen \ref{alg:getrgbvalue-sanity-start}--\ref{alg:getrgbvalue-sanity-end} wird
 sichergestellt, dass kein \gls{pixel} ausserhalb der Bildgrenzen gelesen werden können. In Zeile
 \ref{alg:getrgbvalue-offset} wird der Adressabstand für ein \gls{pixel} berechnet. Die Adresse des \gls{pixel} wird in
 Zeile \ref{alg:getrgbvalue-address} aus der Adresse von $I$ und dem Adressabstand berechnet. Zuletzt wird der Wert des
 \gls{pixel} in Zeile \ref{alg:getrgbvalue-returncolor} für den angegeben Farbkanal zurückgegeben.

Das Verfahren von \citeauthor{clarke96} (\autoref{sub:linienerkennung_nach_clarke96}) liefert als Ergebnis nur kurze
 Liniensegmente, die zur Erkennung einer Marke ungeeignet sind. Im Verfahren von \citeauthor{hirzer08} werden kurze
 Liniensegmente zusammengeführt. Dazu dient das Verfahren \textproc{mergeLines}, dass in \autoref{alg:mergelines1}
 vorgestellt wird.

\begin{algorithm}[!ht]
\caption{\textproc{mergeLines} (Überblick)}
\label{alg:mergelines1}
\begin{algorithmic}[1]
	\Require $L$
	\State $\mathit{distancepool} \gets \infty$
	\For{$i \gets 0$ \textbf{to} $i <$ \Call{getLineCount}}
		\State \Call{freeDistancePool}{$\mathit{distancepool}$}
		\State $\mathit{start} \gets$ \Call{getLineSegment}{$L,i$}
		\State \ldots \Comment Suche zwei kompatible Liniensegmente
		\If{$\lnot$ \Call{getDistanceCount}{$\mathit{distancepool}$}}
			\State \Call{continue}{}
		\EndIf
		\State \ldots \Comment Verbinde Liniensegmente mit kurzem Abstand zuerst
		\State $i \gets i + 1$
	\EndFor
\end{algorithmic}
\end{algorithm}


In der Schleife von Zeile \ref{alg:mergelines1-loop-start}--\ref{alg:mergelines1-loop-end} werden alle Liniensegmente
 untersucht. In Zeile \ref{alg:mergelines1-distancepool} wird zuerst ein Speicherbereich für Distanzwerte der Linien
 initialisiert. Für jeden Lauf der Schleife wird in Zeile \ref{alg:mergelines1-cleardistancepool} der Speicherbereich
 der Distanzwerte gelöscht. Danach wird in Zeile \ref{alg:mergelines1-getline} ein Liniensegment an Position $i$
 ausgewählt. Dieses Liniensegment wird in \autoref{alg:mergelines2}--\autoref{alg:mergelines3} untersucht, um ein
 zweites, kompatibles, Liniensegment zu finden. In Zeile
 \ref{alg:mergelines1-hasdistance-start}--\ref{alg:mergelines1-hasdistance-end} wird die Anzahl der Distanzwerte
 untersucht. Wenn keine Einträge in \textproc{distancepool} vorhanden sind, wird die Verarbeitung mit einem neuen
 Durchlauf der Schleife fortgesetzt. Andernfalls werden in Zeile \ref{alg:mergelines1-mergelines} mit
 \autoref{alg:mergelines4}--\autoref{alg:mergelines6} komaptible Liniensegmente zusammengeführt.

In \autoref{alg:mergelines2} wird in der Schleife \ref{alg:mergelines2-loop-start}--\ref{alg:mergelines3-loop-end} ein
 zweites Liniensegment gesucht, das kompatibel mit dem ersten Liniensegment ist.

\begin{algorithm}[!ht]
\caption{\textproc{mergeLines} (Kompatible Liniensegmente suchen)}
\label{alg:mergelines2}
\begin{algorithmic}[1]
	\For{$j \gets 0$ \textbf{to} $j <$ \Call{getLineCount}{$L$}}
	\label{alg:mergelines2-loop-start}
		\If{$i = j$}
		\label{alg:mergelines2-ignore-start}
			\State \textbf{continue}
		\EndIf
		\label{alg:mergelines2-ignore-end}
		\State $\mathit{tmp} \gets$ \Call{getLineSegment}{$L,j$}
		\label{alg:mergelines2-j}
		\State $\mathit{lineOrientation} \gets$ \Call{dotProduct}{$\mathit{tmp.slope}, \mathit{start.slope}$}
		\label{alg:mergelines2-orientation-start}
		\State $\mathit{connectionLine} \gets$ \textproc{vectorSubtract}
		$\left(
		\begin{aligned}
			& \mathit{tmp.end.coordinate},\\
			& \mathit{start.start.coordinate}
		\end{aligned}\right)$
		\State \Call{normalized}{$\mathit{connectionLine}$}
		\State $\mathit{connectionOrientation} \gets$ \Call{dotProduct}{$\mathit{connectionLine}, \mathit{start.slope}$}
		\label{alg:mergelines2-orientation-end}
	\algstore{brk-mergelines2}
\end{algorithmic}
\end{algorithm}


In Zeile \ref{alg:mergelines2-ignore-start}--\ref{alg:mergelines2-ignore-end} werden Liniensegmente an der gleichen
 Position ignoriert und die Schleife mit einem neuen Lauf wiederholt. Danach wird in Zeile \ref{alg:mergelines2-j} ein
 Liniensegment an Position $j$ ausgewählt. In Zeile
 \ref{alg:mergelines2-orientation-start}--\ref{alg:mergelines2-orientation-end} wird die Orientierung der beiden
 Liniensegmente und die Orientierung der Verbindungslinie berechnet.

In Zeile \ref{alg:mergelines3-isorientationcompatible} (\autoref{alg:mergelines3}) wird die Orientierung untersucht,
 die sowohl für die Lininesegmente, als auch für die Verbindungsline, nur geringfügig abweichen darf. Wenn die
 Überprüfung positiv ausfällt, wird in Zeile \ref{alg:mergelines3-cline-start}--\ref{alg:mergelines3-cline-end} die
 Verbindungslinie in \textproc{cLine} gespeichert. In Zeile \ref{alg:mergelines3-isbridged} wird anschließend die
 Orientierung der Verbindungslinie Punkt für Punkt untersucht (Vgl. \autoref{sub:line_detection}). Wenn der Gradient
 der Verbindungslinie sich nicht verändert hat, wird in Zeile
 \ref{alg:mergelines3-save-start}--\ref{alg:mergelines3-save-end} die Länge zum Quadrat und die Position der
 Verbindungsline in \textproc{distancepool} gespeichert.

\begin{algorithm}[!ht]
\caption{\textproc{mergeLines} (Kompatible Liniensegmente suchen)}
\label{alg:mergelines3}
\begin{algorithmic}[1]
	\algrestore{brk-mergelines2}
		\If{$\mathit{lineOrientation} > 0.99 \land \mathit{connectionOrientation} > 0.99$}
		\Cost{$c_{10}$}{$3l$}
		\label{alg:mergelines3-isorientationcompatible}
			\State $\mathit{cLine} \gets \infty$
			\Cost{$c_{11}$}{$l$}
			\label{alg:mergelines3-cline-start}
			\State $\mathit{cLine.start} \gets \mathit{start.end}$
			\Cost{$c_{12}$}{$2l$}
			\State $\mathit{cLine.end} \gets \mathit{tmp.start}$
			\Cost{$c_{13}$}{$3l$}
			\State $\mathit{cLine.slope} \gets \mathit{start.slope}$
			\Cost{$c_{14}$}{$3l$}
			\label{alg:mergelines3-cline-end}
			\If{$\lnot$ \Call{isConnectionLineBridged}{$cLine, I$}}
			\Cost{$c_{15}$}{$l \cdot \Theta(\mathit{length})$}
			\label{alg:mergelines3-isbridged}
				\State $\mathit{connectionLine} \gets$ \textproc{vectorSubtract}
				$\left(
				\begin{aligned}
					& \mathit{tmp.start.coordinate},\\
					& \mathit{start.end.coordinate}
				\end{aligned}\right)$
				\Cost{$c_{16}$}{$\bigl(5 + \Theta(1)\bigr)l$}
				\label{alg:mergelines3-save-start}
				\State $\mathit{squaredLength} \gets$ \Call{squaredLengt}{$\mathit{connectionLine}$}
				\Cost{$c_{17}$}{$\bigl(1 + \Theta(1)\bigr)l$}
				\State $\mathit{distance} \gets \infty$
				\Cost{$c_{18}$}{$l$}
				\State $\mathit{distance.distance} \gets \mathit{squaredLength}$
				\Cost{$c_{19}$}{$2l$}
				\State $\mathit{distance.index} \gets j$
				\Cost{$c_{20}$}{$2l$}
				\State \Call{addDistance}{$\mathit{distancepool}, \mathit{distance}$}
				\Cost{$c_{21}$}{$\Theta(1) \cdot l$}
				\label{alg:mergelines3-save-end}
			\EndIf
		\EndIf
		\State $j \gets j + 1$
		\Cost{$c_{24}$}{$l$}
	\EndFor
	\label{alg:mergelines3-loop-end}
\end{algorithmic}
\end{algorithm}


\textproc{isConnectionLineBrigded} benötigt zur Untersuchung die Verbindungsline $\mathit{connectionLine}$ und das
 Bildsignal $I$. Das Verfahren ist in \autoref{alg:isconnectionlinebridged} aufgeführt.

\begin{algorithm}[!ht]
\caption{\textproc{isConnectionLineBridged}}
\label{alg:isconnectionlinebridged}
\begin{algorithmic}[1]
	\Require $\mathit{connectionLine},I$
	\State $\mathit{lineLength} \gets$ \Call{vectorSubtract}{$\mathit{connectionLine.end.coordinate},\mathit{connectionLine.start.coordinate}$}
	\label{alg:isconnectionlinebridged-length-start}
	\State $\mathit{length} \gets$ \Call{getLength}{$\mathit{lineLength}$}
	\State $\mathit{point} \gets \mathit{connectionLine.start.coordinate}$
	\label{alg:isconnectionlinebridged-length-end}
	\For{$i \gets 0$ \textbf{to} $i < \mathit{length}$}
	\label{alg:isconnectionlinebridged-loop-start}
		\State $\mathit{point} \gets$ \Call{vectorAdd}{$\mathit{point},\mathit{connectionLine.slope}$}
		\State $\mathit{orientationEdgel}$
		\State $\mathit{orientationEdgel.slope} \gets$ \Call{gradientIntensity}{$I,\mathit{imageWidth},\mathit{imageHeight}, \mathit{point.x}, \mathit{point.y}$}
		\label{alg:isconnectionlinebridged-gradient}
		\If{\Call{isCompatible}{$\mathit{orientationEdgel},\mathit{connectionLine.start}$}}
		\label{alg:isconnectionlinebridged-iscompatible}
			\State \textbf{continue}
		\EndIf
		\State \textbf{return true}
		\label{alg:isconnectionlinebridged-bridged}
		\State $i \gets i + 1$
	\EndFor
	\label{alg:isconnectionlinebridged-loop-end}
	\State \textbf{return false}
	\label{alg:isconnectionlinebridged-notbridged}
\end{algorithmic}
\end{algorithm}


In Zeile \ref{alg:isconnectionlinebridged-length-start}--\ref{alg:isconnectionlinebridged-length-end} wird die Länge
 der Verbindungslinie berechnet und der Startpunkt festgelegt. In der Schleife in Zeile
 \ref{alg:isconnectionlinebridged-loop-start}--\ref{alg:isconnectionlinebridged-loop-end} wird jeder Punkt auf der
 Linie untersucht. In Zeile \ref{alg:isconnectionlinebridged-gradient} wird der Gradient an der Position $i$  der
 Verbindungslinie berechent. Wenn der Gradient in Zeile \ref{alg:isconnectionlinebridged-iscompatible} mit dem Gradient
 der Linie kompatibel ist, wird die Schleife fortgesetzt. Andernfalls, wenn die Orientierung des Punktes und der Linie
 nicht kompatibel sind, wird in Zeile \ref{alg:isconnectionlinebridged-bridged} das Verfahren abgebrochen. Wenn jeder
 Punkt auf der Verbindungslinie kompatibel zur Verbindungslinie ist, wird die Schleife in Zeile
 \ref{alg:isconnectionlinebridged-loop-start}--\ref{alg:isconnectionlinebridged-loop-end} vollständig durchlaufen und
 das Verfahren in Zeile \ref{alg:isconnectionlinebridged-notbridged} beendet.

Nachdem in \autoref{alg:mergelines2}--\autoref{alg:mergelines3} kompatible Liniensegmente ermittelt wurden, kann mit
 \autoref{alg:mergelines4} und \autoref{alg:mergelines5} die Liniensegmente verbunden werden.

\begin{algorithm}[!ht]
\caption{\textproc{mergeLines} (Liniensegmente verbinden)}
\label{alg:mergelines4}
\begin{algorithmic}[1]
	\State $\mathit{numberOfDistances} \gets$ \Call{getDistanceCount}{$\mathit{distancepool}$}
	\State \Call{qsort}{$\mathit{distancepool}, \mathit{numberOfDistances}, \mathit{size},$ \textproc{distanceCompare}} \Comment man-page qsort
		\For{$k \gets 0$ \textbf{to} $k < \mathit{numberOfDistances}$}
			\State $j \gets \mathit{distancepool.data}[k]\mathit{.index}$
			\State $\mathit{startpoint} \gets \mathit{start.end.coordinate}$
			\State $\mathit{endpoint} \gets$ \Call{getLineSegment}{$L,j$}$\mathit{.start.coordinate}$
			\State $\mathit{lineLength} \gets$ \Call{vectorSubtract}{$\mathit{endpoint},\mathit{startpoint}$}
			\State $\mathit{lineSlope} \gets$ \Call{vectorSubtract}{$\mathit{endpoint},\mathit{startpoint}$}
			\State $\mathit{length} \gets$ \Call{getLength}{$\mathit{lineLength}$}
			\State $\mathit{lineSlope} \gets$ \Call{normalized}{$\mathit{lineSlope}$}
			\algstore{mergelines4}
\end{algorithmic}
\end{algorithm}


In Zeile \ref{alg:mergelines4-sort-start}--\ref{alg:mergelines4-sort-end} werden die Liniensegmente anhand ihrers
 Abstands aufsteigend sortiert. Dazu wird die Funktion \textproc{qsort} verwendet\footcite{qsort}. Die Funktion
 \textproc{distanceCompare} wird von \textproc{qsort} zum sortieren verwendent (Vgl. \autoref{alg:distancecompare}).

\begin{algorithm}[!ht]\small
\caption{\textproc{distanceCompare}}
\label{alg:distancecompare}
\begin{algorithmic}[1]
	\Require{$a,b$}
	\State $\mathit{ad} \gets a$
	\State $\mathit{bd} \gets b$
	\State \textbf{return} $\mathit{ad.distance} - \mathit{bd.distance}$
\end{algorithmic}
\end{algorithm}


\textproc{distanceCompare} benötigt zwei Distanzwerte $a$ und $b$ deren Länge subtrahiert und als Ergebnis zurückgegebn wird.

In Zeile \ref{alg:mergelines4-loop-start}--\ref{alg:mergelines5-loop-end} von \autoref{alg:mergelines4} und
 \autoref{alg:mergelines5} werden die Liniensegmente vorbereitet. Dazu wird aus dem sortiereten Speicherblock
 $\mathit{distancepool}$ der Index des Liniensegments ausgelesen und gespeichert. Die Start- und Endedgel, sowie die
 Länge der Linie und ihre Orientierung werden berechnet
 (Zeile \ref{alg:mergelines4-merge-start}--\ref{alg:mergelines4-merge-end}).

\begin{algorithm}[!ht]
\caption{\textproc{mergeLines} (Liniensegmente verbinden)}
\label{alg:mergelines5}
\begin{algorithmic}[1]
	\algrestore{mergelines4}
	\If{\Call{extendLine}{$\mathit{startpoint},\mathit{lineSlope},\mathit{start.end.slope},\mathit{endpoint},\mathit{length},I$}}
	\label{alg:mergelines5-extendline}
		\State $\mathit{pool.data}[i]\mathit{.end} \gets \mathit{pool.data}[j]\mathit{.end}$
		\label{alg:mergelines5-merge-start}
		\State $\mathit{newLineSlope} \gets$ \Call{subtract}{$\mathit{pool.data}[i]\mathit{.end.coordinate},\mathit{pool.data}[i]\mathit{.start.coordinate}$}
		\State $\mathit{newLineSlope} \gets$ \Call{normalized}{$\mathit{newLineSlope}$}
		\State $\mathit{pool.data}[i]\mathit{.slope} \gets \mathit{newLineSlope}$
		\State $\mathit{pool.data}[j]\mathit{.remove} \gets$ \textbf{true}
		\label{alg:mergelines5-merge-end}
	\Else
		\State \textbf{break}
	\EndIf
	\State $k \gets k + 1$
	\EndFor
	\label{alg:mergelines5-loop-end}
	\algstore{brk-mergelines5}
\end{algorithmic}
\end{algorithm}


In Zeile \ref{alg:mergelines5-extendline} wird mit der Methode \textproc{extendLine} überprüft, ob sich die Linie
 erweitern lässt (Vgl. \autoref{}). Falls dem so ist, wird in Zeile
 \ref{alg:mergelines5-merge-start}--\ref{alg:mergelines5-merge-end} der Linie an Position $i$ der neue Endpunkt der
 Linie an Position $j$ zugewiesen. Danach wird die Orientierung der Linie berechnet und das Liniensegment an Position
 $j$ zum löschen markiert. Wenn sich eine Linie nicht durch \textproc{extendLine} erweitern lässt, wird die Schleife
 abgebrochen.

Mit \autoref{alg:mergelines6} werden die markierten Liniensegmente gelöscht. Dazu wird der Speicherblock $L$
 vollständig durchlaufen (Zeile \ref{alg:mergelines6-del-start}--\ref{alg:mergelines6-del-end}). Wenn die Überprüfung
 in Zeile \ref{alg:mergelines6-shouldremove} ein markiertes Lininesegment findet, wird es in Zeile
 \ref{alg:mergelines6-remove} entfernt und die Laufvariable dekrementiert. Zusätzlich wird in Zeile
 \ref{alg:mergelines6-merged} die boolsche Variable gesetzt. Wenn die Variable $\mathit{merged}$ gesetzt wurde, und
 somit Liniensegmente verschmolzen wurden, wird die Variable $i$ auf den Wert $-1$ gesetzt
 (Zeile \ref{alg:mergelines6-wasmerged-start}--\ref{alg:mergelines6-wasmerged-end}). Dies hat zur Folge, dass das
 Verfahren \textproc{mergeLines} in Zeile \ref{alg:mergelines1-loop-start} von \autoref{alg:mergelines1} wieder bei $0$
 beginnt und alle verbundenen Linien mit den restlichen Liniensegmenten verglichen werden.

\begin{algorithm}[!ht]
\caption{\textproc{mergeLines} (Liniensegmente löschen)}
\label{alg:mergelines6}
\begin{algorithmic}[1]
	\algrestore{brk-mergelines5}
	\State $\mathit{merged} \gets$ \textbf{false}
	\For{$j \gets 0$ \textbf{to} $j <$ \Call{getLineCount}{$L$}}
	\label{alg:mergelines6-del-start}
		\If{$\mathit{pool \to data}[j]\mathit{.remove} ==$ \textbf{true}}
		\label{alg:mergelines6-shouldremove}
			\State \Call{removeLine}{$L,j$}
			\label{alg:mergelines6-remove}
			\State $j \gets j - 1$
			\State $\mathit{merged} \gets$ \textbf{true}
			\label{alg:mergelines6-merged}
		\EndIf
		\State $j \gets j + 1$
	\EndFor
	\label{alg:mergelines6-del-end}
	\If{$\mathit{merged}$}
	\label{alg:mergelines6-wasmerged-start}
		\State $ i \gets -1$
	\EndIf
	\label{alg:mergelines6-wasmerged-end}
\end{algorithmic}
\end{algorithm}


\textproc{extendLine} (\autoref{alg:extendline1}--\autoref{alg:extendline2}) vergleicht jeden Punkt zwischen Start und
 Endpunkt auf Kompatibilität der Orientierung und benötigt dazu den Start- und Endpunkt der Linie, ihre Richtung
 $\mathit{slope}$ und die Orientierung. Desweiteren wird die zu untersuchende Länge der Linie und das Bildsignal $I$
 benötigt.

\begin{algorithm}[!ht]
\caption{\textproc{extendLine}}
\label{alg:extendline1}
\begin{algorithmic}[1]
	\Require $\mathit{start}, \mathit{slope}, \mathit{gradient}, \mathit{end}, \mathit{maxLength}, I$
	\State $\mathit{normal.x} \gets \mathit{slope.y}$
	\label{alg:extendline1-ortho-start}
	\State $\mathit{normal.y} \gets - \mathit{slope.x}$
	\label{alg:extendline1-ortho-end}
	\State $\mathit{merge} \gets$ \textbf{true}
	\label{alg:extendline1-boolean}
	\For{$i \gets 0$ \textbf{to} $i < \mathit{maxLength}$}
	\label{alg:extendline1-loop-start}
		\State $\mathit{start} \gets$ \Call{vectorAdd}{$\mathit{start},\mathit{slope}$}
		\label{alg:extendline1-next}
		\State $x \gets$ \Call{convoluteKernelX}{$I,\mathit{start.x},\mathit{start.y},\mathit{green},\mathit{imageWidth},\mathit{imageHeight}$}
		\State $y \gets$ \Call{convoluteKernelY}{$I,\mathit{start.x},\mathit{start.y},\mathit{green},\mathit{imageWidth},\mathit{imageHeight}$}
		\If{$x < \mathit{threshold}/2 \land y < \mathit{threshold}/2$}
		\label{alg:extendline1-threshold}
			\State $\mathit{merge} \gets$ \textbf{false}
			\label{alg:extendline1-break}
			\State \textbf{break}
		\EndIf
		\algstore{alg:extendline1-first}
\end{algorithmic}
\end{algorithm}


In Zeile \ref{alg:extendline1-ortho-start}--\ref{alg:extendline1-ortho-end} wird die Orthogonale der Richtung berechnet
 und gespeichert. Die boolsche Variable wird in Zeile \ref{alg:extendline1-boolean} initialisiert. In der Schleife in
 Zeile \ref{alg:extendline1-loop-start}--\ref{alg:extendline2-loop-end} wird die Länge der Linie Punkt für Punkt
 untersucht.

Die Addition von $\mathit{start}$ und $\mathit{slope}$ rückt die Position der Linie um einen Punkt weiter (Zeile
 \ref{alg:extendline1-next}). Die Faltung für die vertikale und horizontale Richtung wird in den Variablen $x$ und $y$
 gespeichert. In Zeile \ref{alg:extendline1-threshold} wird überprüft, ob der Gradient kleiner ist als die Hälfte des
 festgelegten Schwellwerts. Dadurch kann festgestellt werden, ob an der Positon $\mathit{start}$ sich tatsächlich ein
 \gls{edgel} befindet. Falls dem nicht so ist, wird in Zeile \ref{alg:extendline1-break} die boolsche Variable
 aktualisiert und die Schleife abgebrochen. Andernfalls wird in Zeile
 \ref{alg:extendline2-orientation-start}--\ref{alg:extendline2-orientation-end} (\autoref{alg:extendline2}) die
 Orientierung an der Position von $\mathit{start}$ berechnet und der Winkel zum Gradienten ermittelt.

\begin{algorithm}[!ht]
\caption{\textproc{extendLine} (Fortsetzung)}
\label{alg:extendline2}
\begin{algorithmic}[1]
	\algrestore{alg:extendline1-first}
	\State $\mathit{intensity} \gets$ \Call{gradientIntensity}{$I, \mathit{imageWidth}, \mathit{imageHeight}, \mathit{start.x}, \mathit{start.y}$}
	\label{alg:extendline2-orientation-start}
	\State $\mathit{tmp} \gets$ \Call{dotProduct}{$\mathit{intensity},\mathit{gradient}$}
	\label{alg:extendline2-orientation-end}
	\If{$\mathit{tmp} > 0.38$}
	\label{alg:extendline2-iscompatible}
		\State \textbf{continue}
	\EndIf
	\State $\mathit{intensity} \gets$ \textproc{gradientIntensity}
	$\left(
	\begin{aligned}
		& I, \mathit{imageWidth}, \mathit{imageHeight},\\
		& \mathit{start.x} + \mathit{normal.x},\\
		& \mathit{start.y} + normal.y
	\end{aligned}\right)$
	\label{alg:extendline2-pointabove}
	\State $\mathit{tmp} \gets$ \Call{dotProduct}{$\mathit{intensity},\mathit{gradient}$}
	\If{$\mathit{tmp} > 0.38$}
		\State \textbf{continue}
	\EndIf
	\State $\mathit{intensity} \gets$ \textproc{gradientIntensity}
	$\left(
	\begin{aligned}
		& I, \mathit{imageWidth}, \mathit{imageHeight},\\
		& \mathit{start.x} - \mathit{normal.x},\\
		& \mathit{start.y} - normal.y
	\end{aligned}\right)$
	\label{alg:extendline2-pointbelow}
	\State $\mathit{tmp} \gets$ \Call{dotProduct}{$\mathit{intensity},\mathit{gradient}$}
	\If{$\mathit{tmp} > 0.38$}
		\State \textbf{continue}
	\EndIf
	\State $\mathit{merge} \gets$ \textbf{false}
	\label{alg:extendline2-false}
	\State \textbf{break}
	\State $i \gets i + 1$
	\EndFor
	\label{alg:extendline2-loop-end}
	\State $\mathit{end} \gets$ \Call{vectorSubtract}{$\mathit{start},\mathit{slope}$}
	\label{alg:extendline2-endofline}
	\State \textbf{return} $\mathit{merge}$
	\label{alg:extendline2-return}
\end{algorithmic}
\end{algorithm}


In Zeile \ref{alg:extendline2-iscompatible} wird überprüft, ob der Winkel sich innerhalb des festgelegten Grenzwertes
 befinden. Falls dem so ist, wird ein neuer Schleifendurchlauf ausgeführt. Doch falls der Winkel nicht dem vorgegeben
 Grenzwert entspricht, wird der Gradient in Zeile \ref{alg:extendline2-pointabove} an der Position untersucht, die
 einen Punkt oberhalb der Linie liegt. Dadurch muss eine Linie im Bildsignal nicht ganz gerade sein und gilt dennnoch
 als eine Linie. Falls diese Untersuchung auch fehlschlägt, wird der Gradient in Zeile \ref{alg:extendline2-pointbelow}
 eine Position unterhalb der Linie untersuchen. Auch hier liegt der Grund für diese Vorgehnsweise darin begründet, dass
 eine Linie nicht zwangsweise gerade sein muss. Falls dieser Vergleich wieder fehlschlägt, wird die boolsche Variable
 in Zeile \ref{alg:extendline2-false} gesetzt und die Schleife abgebrochen. In Zeile
 \ref{alg:extendline2-endofline} wird das Ende der Linie aktualisiert und der Wert der boolschen Variable in Zeile
 \ref{alg:extendline2-return} als Ergebnis zurückgegeben.

Wie in \autoref{sub:line_detection} bereits erläutert, wird das Zusammenführen von Linien einmal für jede Region und
 einmal auf dem gesamten Bildsignal durchgeführt. Der Ablauf ist in \autoref{alg:linedetection-hirzermerging}
 dargestellt.

\begin{algorithm}[!ht]\small
\caption{\textproc{lineDetection} mit zusammengeführten Linien}
\label{alg:linedetection-hirzermerging}
	\begin{algorithmic}[1]
		\Require $I$
		\For{$y \gets 0$ \textbf{to} $y < \mathit{imageHeight}$}
		\Cost{$c_{1}$}{$(\frac{h}{r} + 1)$}
			\For{$x \gets 0$ \textbf{to} $x < \mathit{imageWidth}$}
			\Cost{$c_{2}$}{$\frac{h}{r}(\frac{w}{r} + 1)$}
				\State \Call{findEdgels}{$I,E,x,y$}
				\Cost{$c_{3}$}{$\frac{hw}{r^2}\cdot\Theta(r^2)$}
				\State \Call{findLineSegments}{$E,L$}
				\Cost{$c_{4}$}{$\frac{hw}{r^2}\cdot\Theta(mn^2)$}
				\State \Call{mergeLines}{$L$}
				\Cost{$c_{5}$}{$\frac{hw}{r^2}\cdot\Theta(l^2\cdot\mathit{length})$}
				\label{alg:linedetection-hirzermerging-mergeregion}
				\State \ldots \Comment Speichern der Liniensegmente zur weiteren Verarbeitung
				\State \Call{resetMemoryPool}{$E$}
				\Cost{$c_{7}$}{$\frac{hw}{r^2}\cdot\Theta(1)$}
				\State \Call{resetMemoryPool}{$L$}
				\Cost{$c_{8}$}{$\frac{hw}{r^2}\cdot\Theta(1)$}
				\State $ x \gets x + 40$
				\Cost{$c_{9}$}{$2\frac{hw}{r^2}$}
			\EndFor
			\State $y \gets y + 40$
			\Cost{$c_{11}$}{$2\frac{h}{r}$}
		\EndFor
		\State \Call{mergeLines}{$L$}
		\Cost{$c_{13}$}{$\Theta(l^2\cdot\mathit{length})$}
		\label{alg:linedetection-hirzermerging-mergeall}
	\end{algorithmic}
\end{algorithm}


In Zeile \ref{alg:linedetection-hirzermerging-mergeregion} werden Liniensegmente für die aktuelle Region
 zusammengeführt. Danach wird in Zeile \ref{alg:linedetection-hirzermerging-mergeall} alle Linien und Liniensegemente zusammengeführt.

% subsection linienerkennung_nach_hirzer08 (end)

\subsection{Line Extension} % (fold)
\label{sub:analyse_line_extension}
$\mathit{extendLinesInPool}$ erweitert, wie in \autoref{sub:line_extension} beschrieben, die Linien am Anfang und am
 Ende. Das Verfahren ist in \autoref{alg:extendlinesinpool} dargestellt und benötigt als Parameter den Speicherblock
 $L$ mit Linien.
\begin{algorithm}[!ht]
\caption{\textproc{extendLinesInPool}}
\label{alg:extendlinesinpool}
\begin{algorithmic}[1]
	\Require $L$
	\State $\mathit{lineCount} \gets$ \Call{getLineCount}{$L$}
	\Cost{$c_{1}$}{$1 + \Theta(1)$}
	\For{$i \gets 0$ \textbf{to} $i < \mathit{lineCount}$}
	\Cost{$c_{2}$}{$(l + 1)$}
	\label{alg:extendlinesinpool-loop-start}
		\State $l \gets \mathit{L.data}[i]$
		\Cost{$c_{3}$}{$3l$}
		\label{alg:extendlinesinpool-line}
		\State $\mathit{slope} \gets \mathit{l.slope}$
		\Cost{$c_{4}$}{$2l$}
		\label{alg:extendlinesinpool-slope}
		\State \Call{extendLine}{$\mathit{l.end.coordinate}, \mathit{slope}, \mathit{l.end.slope},
		 \mathit{l.end.coordinate},999,I$}
		\Cost{$c_{5}$}{$l\cdot\Theta(999)$}
		\State $\mathit{slope.x} \gets \mathit{slope.x} \cdot - 1$
		\Cost{$c_{6}$}{$4l$}
		\label{alg:extendlinesinpool-slopex-invert}
		\State $\mathit{slope.y} \gets \mathit{slope.y} \cdot - 1$
		\Cost{$c_{7}$}{$4l$}
		\label{alg:extendlinesinpool-slopey-invert}
		\State \Call{extendLine}{$\mathit{l.start.coordinate}, \mathit{slope}, \mathit{l.end.slope},
		 \mathit{l.start.coordinate},999,I$}
		\Cost{$c_{8}$}{$l\cdot\Theta(999)$}
		\label{alg:extendlinesinpool-extend-start}
		\State $i \gets i + 1$
		\Cost{$c_{9}$}{$l$}
	\EndFor
	\label{alg:extendlinesinpool-loop-end}
\end{algorithmic}
\end{algorithm}

In der Schleife in Zeile \ref{alg:extendlinesinpool-loop-start}--\ref{alg:extendlinesinpool-loop-end} wird jede Linie im
 Speicherblock $L$ erweitert. Dazu wird zuerst in Zeile \ref{alg:extendlinesinpool-line} und
 \ref{alg:extendlinesinpool-slope} die Linie an Position $i$ ausgewählt und ihre Richtung ausgelesen. Danach wird durch
 \textproc{extendLine} die Linie erweitert. Die Methode \textproc{extendLine} wurde in
 \autoref{sub:linienerkennung_nach_hirzer08} beschrieben. Der Parameter $\mathit{maxLength}$ wird mit einem großen Wert
 verwendet, um die Linie soweit wie möglich zu erweitern. Die Wachstumsrate $\Theta(\mathit{length})$ von
 \textproc{extendLine} wird in diesem Verfahren mit dem konstanten Wert $\mathit{maxLength} = 999$ verwendet. Danach
 wird die Orientierung der Linie in Zeile
 \ref{alg:extendlinesinpool-slopex-invert}--\ref{alg:extendlinesinpool-slopey-invert} umgekehrt und in Zeile
 \ref{alg:extendlinesinpool-extend-start} der Anfang der Linie erweitert. Die Laufzeitfunktion des Verfahrens ist für
 den schlechtesten Fall in \autoref{eq:extendlinesinpool-1} angegeben. Die Wachstumsrate ist $2013l + 3 = \Theta(l)$,
 für $c_{1} = 2012$, $c_{2} = 2013$ und $l_{0} = 1$.
\begin{subequations}
\label{eq:extendlinesinpool}
\begin{align}
\label{eq:extendlinesinpool-1}
T_{worst}(l)& =
c_{1}\bigl(1 + \Theta(1)\bigr)
+ c_{2}(l+1)
+ c_{3}3l
+ c_{4}2l
+ c_{5}\bigl(l \cdot \Theta(999)\bigr)
\\
& \quad
+ c_{6}4l
+ c_{7}4l
+ c_{8}\bigl(l \cdot \Theta(999)\bigr)
+ c_{9}l
\nonumber \\
\label{eq:extendlinesinpool-2}
T_{worst}(l)& =
\Bigl(c_{1}\bigl(1 + \Theta(1)\bigr) + c_{2}\Bigr)
+ l(c_{2} + c_{3}3 + c_{4}2 + c_{6}4 + c_{7}4 + c_{9})
\\
& \quad
+ l\bigl(c_{5}\Theta(999) + c_{8}\Theta(999)\bigr)
\nonumber
\end{align}
\end{subequations}


Ob eine erweiterte Linie zur Markenerkennung geeignet ist, wird dadurch bestimmt, ob über das Linienende hinaus ein
 heller Pixel liegt. Dazu wird das Verfahren $\mathit{findLinesWithCornersInLinePool}$ verwedent
 (\autoref{alg:findlineswithcornersinlinepool1}--\autoref{alg:findlineswithcornersinlinepool2}).

\begin{algorithm}[!ht]
\caption{\textproc{findLinesWithCornersInLinePool}}
\label{alg:findlineswithcornersinlinepool1}
\begin{algorithmic}[1]
	\Require $L,C$
	\State \Call{resetMemoryPool}{$C$}
	\label{alg:findlineswithcornersinlinepool1-clear}
	\State $w \gets \mathit{imageWidth}$ 
	\State $h \gets \mathit{imageHeight}$
	\State $n \gets$ \Call{getLineCount}{L}
	\For{$i \gets 0$ \textbf{to} $i < n$}
	\label{alg:findlineswithcornersinlinepool1-loop-start}
		\State $l \gets \mathit{pool \to data}[i]$
		\label{alg:findlineswithcornersinlinepool1-line}
		\State $\mathit{dx} \gets l\mathit{.slope.x} \cdot 4$
		\label{alg:findlineswithcornersinlinepool1-dx}
		\State $\mathit{dy} \gets l\mathit{.slope.y} \cdot 4$
		\label{alg:findlineswithcornersinlinepool1-dy}
		\State $x \gets l\mathit{.start.coordinate.x} - dx$
		\label{alg:findlineswithcornersinlinepool1-x}
		\State $y \gets l\mathit{.start.coordinate.y} - dy$
		\label{alg:findlineswithcornersinlinepool1-y}
		\State $r \gets$ \Call{getRGBValue}{$I,x,y,red,w,h$}
		\State $g \gets$ \Call{getRGBValue}{$I,x,y,green,w,h$}
		\State $b \gets$ \Call{getRGBValue}{$I,x,y,blue,w,h$}
	\algstore{brk-findlineswithcornersinlinepool1}
\end{algorithmic}
\end{algorithm}

\begin{algorithm}[!ht]
\caption{\textproc{findLinesWithCornersInLinePool} (Fortsetzung)}
\label{alg:findlineswithcornersinlinepool2}
\begin{algorithmic}[1]
	\algrestore{brk-findlineswithcornersinlinepool1}
		\If{$r > 10 \land g > 10 \land b > 10$}
		\label{alg:findlineswithcornersinlinepool2-iswhite-start}
			\State $l\mathit{.startCorner} \gets$ \textbf{true}
		\EndIf
		\State $x \gets l\mathit{.end.coordinate.x} + dx$
		\label{alg:findlineswithcornersinlinepool2-lineend-start}
		\State $y \gets l\mathit{.end.coordinate.y} + dy$
		\State $r \gets$ \Call{getRGBValue}{$I,x,y,red,w,h$}
		\State $g \gets$ \Call{getRGBValue}{$I,x,y,green,w,h$}
		\State $b \gets$ \Call{getRGBValue}{$I,x,y,blue,w,h$}
		\label{alg:findlineswithcornersinlinepool2-lineend-end}
		\If{$r > 10 \land g > 10 \land b > 10$}
		\label{alg:findlineswithcornersinlinepool2-iswhite-end}
			\State $l\mathit{.endCorner} \gets$ \textbf{true}
		\EndIf
		\If{$l\mathit{.startCorner} \lor l\mathit{.endCorner}$}
		\label{alg:findlineswithcornersinlinepool2-hascorner}
			\State \Call{addLineSegment}{$C,l$}
			\label{alg:findlineswithcornersinlinepool2-addline}
		\EndIf
		\State $i \gets i + 1$
	\EndFor
	\label{alg:findlineswithcornersinlinepool2-loop-end}
\end{algorithmic}
\end{algorithm}


Das Verfahren untersucht alle Linien in Speicherblock $L$ und speichert Linien, die sich zur Erkennung eignen, in Block
 $C$. Zu Beginn des Verfahrens wird in Zeile \ref{alg:findlineswithcornersinlinepool1-clear} der Speicherblock $C$
 gelöscht. Danach werden die lokalen Variablen initialisiert. Die Bildbreite und -höhe wird in Variable $w$ und $h$
 hinterlegt. Die Anzahl der Linien in $L$ wird in Variable $n$ gespeichert. In der Schleife in Zeile
 \ref{alg:findlineswithcornersinlinepool1-loop-start}--\ref{alg:findlineswithcornersinlinepool2-loop-end} wird jede
 Linie untersucht, indem in Zeile \ref{alg:findlineswithcornersinlinepool1-line} eine Linie an Position $i$ zuerst in
 $l$ gespeichert wird. Danach wird die Richtung der Linie in Zeile \ref{alg:findlineswithcornersinlinepool1-dx} und
 \ref{alg:findlineswithcornersinlinepool1-dy} verlängert. Die Variable $dx$ und $dy$ sind dienen als Abstand der
 Linienenden. In Zeile \ref{alg:findlineswithcornersinlinepool1-x}--\ref{alg:findlineswithcornersinlinepool1-y} wird
 die Position vor dem Startpunkt der Linie berechent. Danach werden die Farbkomponenten an dieser Position ausgelesen.

In Zeile \ref{alg:findlineswithcornersinlinepool2-iswhite-start} wird dann jeder Farbwert mit einem Schwellwert
 verglichen, um festzustellen ob an der Position ein heller gls{pixel} vorliegt. Falls dem so ist, eignet sich die
 Startposition zur Erkennung einer Linie. In Zeile
 \ref{alg:findlineswithcornersinlinepool2-lineend-start}--\ref{alg:findlineswithcornersinlinepool2-lineend-end} wird
 dann für das Linienende die Koordinate und die Farbkomponenten an dieser Position berechnet. In Zeile
 \ref{alg:findlineswithcornersinlinepool2-iswhite-end} werden dann die Farbwerte am Linienende untersucht. Zuletzt wird
 in Zeile \ref{alg:findlineswithcornersinlinepool2-hascorner} untersucht, ob der Startpunkt oder der Endpunkt zur
 Markenerkennung geeignet ist. Wenn die Untersuchung positiv ausfällt, wird in Zeile
 \ref{alg:findlineswithcornersinlinepool2-addline} die Linie $l$ in den Speicherblock $C$ hinterlegt.

Das Verfahren der Linienerweiterung wird in \textproc{lineDetection} integriert und ist in
 \autoref{alg:linedetection-hirzerextending} dargestellt.

\begin{algorithm}[ht]
\caption{\textproc{lineDetection} mit Linienerweiterung}
\label{alg:linedetection-hirzerextending}
	\begin{algorithmic}[1]
		\Require $I$
		\For{$y \gets 0$ \textbf{to} $y < \mathit{imageHeight}$}
			\For{$x \gets 0$ \textbf{to} $x < \mathit{imageWidth}$}
				\State \Call{findEdgels}{$I,E,x,y$}
				\State \Call{findLineSegments}{$E,L$}
				\State \Call{mergeLines}{$L$}
				\State \ldots \Comment Speichern der Liniensegmente zur weiteren Verarbeitung
				\State \Call{resetMemoryPool}{$E$}
				\State \Call{resetMemoryPool}{$L$}
				\State $ x \gets x + 40$
			\EndFor
			\State $y \gets y + 40$
		\EndFor
		\State \Call{mergeLines}{$L$}
		\State \Call{extendLinesInPool}{$L$}
	\end{algorithmic}
\end{algorithm}


In Zeile \ref{alg:linedetection-hirzerextending-extend} werden zuerst die Linien in $M$ erweitert. Im Anschluß daran
 werden in Zeile \ref{alg:linedetection-hirzerextending-corner} die Linien auf ihre Tauglichkeit zur Markenerkennung
 untersucht. Alle Linien die zur Erkennung geeignet sind, werden in Speicherblock $L$ gespeichert.

% subsection line_extension (end)

\subsection{Quadrangle Detection} % (fold)
\label{sub:analyse_quadrangle_detection}
Bei dem letzten Schritt im Verfahren von \citeauthor{hirzer08}, der Quadrangle Detection, werden die gefundenen Linien
 untersucht, um daraus eine quadratische Marke zu erkennen. Dazu wird das Verfahren \textproc{findChainsOfLines}
 verwendet, dass in \autoref{alg:findchainoflines1}--\autoref{alg:findchainoflines4} dargestellt ist.

\begin{algorithm}[ht]
\caption{\textproc{findChainOfLines}}
\label{alg:findchainoflines1}
\begin{algorithmic}[1]
	\Require $B,\mathit{start},\mathit{fromStart},\mathit{chainLength},C$
	\If{$\mathit{fromStart}$}
		\State $\mathit{startPoint} \gets \mathit{start \to start.coordinate}$
	\Else
		\State $\mathit{startPoint} \gets \mathit{start \to end.coordinate}$
	\EndIf
	\For{$i \gets 0$ \textbf{to} $i < \mathit{B \to count}$}
		\If{\Call{isOrientationCompatible}{$\mathit{start},\mathit{B \to data}[i]$}}
			\State \textbf{continue}
		\EndIf
		\algstore{brk-findchainoflines1}
\end{algorithmic}
\end{algorithm}


Das Verfahren benötigt einen Speicherblock $B$ mit Linien, eine Startlinie $\mathit{start}$, sowie eine boolsche
 Aussage $\mathit{fromStart}$, ob der Anfang der Linie oder das Ende der Linie untersucht werden soll. Zusätzlich wird
 die zu untersuchende Länge einer Linie benötigt und ein Speicherblock $C$, in dem eine Linienkette gespeichert wird.

In Zeile \ref{alg:findchainoflines1-fromstart}--\ref{alg:findchainoflines1-fromend} wird der Startpunkt der zu
 untersuchenden Linie festgelegt und ist von der boolschen Variable $\mathit{fromStart}$ abhängig. Danach wird in Zeile
 \ref{alg:findchainoflines1-loop-start}--\ref{alg:findchainoflines4-loop-end} jede Linie aus $B$ untersucht. In Zeile
 \ref{alg:findchainoflines1-iscompatible} wird überprüft, ob Linie $\mathit{start}$ und die Linie an Position $i$ des
 Speicherblocks fast parallel zueinander stehen. Falls dem so ist wird in Zeile \ref{alg:findchainoflines1-parallel}
 die Untersuchung der nächsten Linie eingeleitet. Nur wenn die Linien nicht paralle ausgerichtet sind, wird das
 Verfahren in \autoref{alg:findchainoflines2} fortgesetzt.

\begin{algorithm}[!ht]\small
\caption{\textproc{findChainOfLines} (Abstand der Linienenden)}
\label{alg:findchainoflines2}
\begin{algorithmic}[1]
	\algrestore{brk-findchainoflines1}
		\If{$\mathit{fromStart}$}
		\Cost{$c_{10}$}{$n$}
		\label{alg:findchainoflines2-endpointend}
			\State $\mathit{endpoint} \gets \mathit{B.data}[i]\mathit{.end.coordinate}$
			\Cost{$c_{11}$}{$5n$}
		\Else
			\State $\mathit{endpoint} \gets \mathit{B.data}[i]\mathit{.start.coordinate}$
			\Cost{$c_{13}$}{$5n$}
		\EndIf
		\label{alg:findchainoflines2-endpointstart}
		\State $\mathit{distance} \gets$ \Call{vectorSubtract}{$\mathit{startPoint},\mathit{endpoint}$}
		\Cost{$c_{15}$}{$n\bigl(1 + \Theta(1)\bigr)$}
		\label{alg:findchainoflines2-distance}
		\State $\mathit{squaredLength} \gets$ \Call{squaredLength}{$\mathit{distance}$}
		\Cost{$c_{16}$}{$n\bigl(1 + \Theta(1)\bigr)$}
		\label{alg:findchainoflines2-length}
		\If{$\mathit{squaredLength} > 16$}
		\Cost{$c_{17}$}{$n$}
			\State \textbf{continue}
			\Cost{$c_{18}$}{$n$}
			\label{alg:findchainoflines2-toofar}
		\EndIf
	\algstore{brk-findchainoflines2}
\end{algorithmic}
\end{algorithm}


In Zeile \ref{alg:findchainoflines2-endpointend}--\ref{alg:findchainoflines2-endpointstart} wird nun der Endpunkt der
 Linie $i$ festgelegt. Wie beim Startpunkt auch, ist der Endpunkt von der Variablen $\mathit{fromStart}$ abhängig. In
 Zeile \ref{alg:findchainoflines2-distance} und Zeile \ref{alg:findchainoflines2-length} wird dann der Abstand der
 Linienenden von $\mathit{start}$ und $i$ berechnet. Wenn der Abstand zu groß ist, wird mit der nächsten Linie die
 Untersuchung wiederholt (Zeile \ref{alg:findchainoflines2-toofar}). Andernfalls wird das Verfahren in
 \autoref{alg:findchainoflines3} fortgesetzt.

\begin{algorithm}[ht]
\caption{\textproc{findChainOfLines} (Fortsetzung)}
\label{alg:findchainoflines3}
\begin{algorithmic}[1]
	\algrestore{brk-findchainoflines2}
		\State $\mathit{test} \gets (\mathit{start \to slope.x} \cdot L\mathit{\to data}[i]\mathit{.slope.y})$
		\State $\mathit{test} \gets \mathit{test} - (\mathit{start \to slope.y} \cdot L\mathit{\to data}[i]\mathit{.slope.x})$
		\If{$(\mathit{fromStart} \land \mathit{test} \leq 0) \lor (\lnot \mathit{fromStart} \land \mathit{test} \geq 0)$}
			\State \textbf{continue}
		\EndIf
		\State $\mathit{chainLength} \gets \mathit{chainLength} + 1$
		\State $\mathit{chainSegment} \gets L\mathit{\to data}[i]$
		\State \Call{removeLine}{$L,i$}
		\If{$\mathit{chainLength} == 4$}
			\State \Call{addLine}{$C,\mathit{chainSegment}$}
			\State \textbf{return}
		\EndIf
	\algstore{brk-findchainoflines3}
\end{algorithmic}
\end{algorithm}


In Zeile \ref{alg:findchainoflines3-vectorproduct-start}--\ref{alg:findchainoflines3-vectorproduct-end} wird das
 Kreuzprodukt der Linienorientierung berechnet um danach in Zeile \ref{alg:findchainoflines3-checkorientation} die
 Orientierung zu überprüfen. Je nachdem ob $\mathit{fromStart}$ gesetzt ist oder nicht, wird die Orientierung
 unterschiedlich betrachtet. Wenn $\mathit{fromStart}$ wahr ist, muss die Orientierung der beiden Linien kleiner oder
 gleich $0$ sein. Damit wird die Orientierung als negativer Winkel betrachtet und der Wertebereich entspricht
 $\left[-1,0\right]$. Andernfalls, wenn $\mathit{fromStart}$ falsch ist, muss die Orientierung größer oder gleich $0$
 sein. In diesem Fall wird die Orientierung als positiver Winkel betrachtet und der Wertebereich entspricht
 $\left[0,1\right]$. Werden diese Bedingungen erfüllt, eignet sich diese Linienkombination nicht zur Markenerkennung
 und es wird eine andere Linie untersucht. Falls die Bedingungen nicht erfüllt werden, eigent sich diese
 Linienkombination zur Erkennung einer Marke. In diesem Fall wird in Zeile \ref{alg:findchainoflines3-incchain} die
 Anzahl der Linienkette erhöht und die Linie an Position $i$ in der Variable $\mathit{chainSegment}$ hinterlegt. In
 Zeile \ref{alg:findchainoflines3-removeline} wird danach die Linie an Position $i$ aus dem Speicherblock $B$ entfernt.
 Danach wird in Zeile \ref{alg:findchainoflines3-has4lines} die Anzahl der Linienketten untersucht. Wenn genau vier
 Linienketten vorhanden sind, wird in Zeile \ref{alg:findchainoflines3-saveline} die Linie in $\mathit{chainSegment}$
 zum Speicherblock $C$ hinzugefügt und danach das Verfahren beendet. Anonsten wird in Zeile
 \ref{alg:findchainoflines4-savelineifnotfromstart} (\autoref{alg:findchainoflines4}) die Linie zum Speicherblock
 hinzugefügt, wenn $\mathit{fromStart}$ falsch ist.

\begin{algorithm}[!ht]\small
\caption{\textproc{findChainOfLines} (Fortsetzung)}
\label{alg:findchainoflines4}
\begin{algorithmic}[1]
	\algrestore{brk-findchainoflines3}
		\If{$\lnot \mathit{fromStart}$}
		\Cost{$c_{32}$}{$n$}
			\State \Call{addLine}{$C,chainSegment$}
			\Cost{$c_{33}$}{$n\cdot\Theta(1)$}
			\label{alg:findchainoflines4-savelineifnotfromstart}
		\EndIf
		\State \Call{findChainOfLines}{$B,\mathit{chainSegment},\mathit{fromStart},\mathit{chainLength},C$}
		\label{alg:findchainoflines4-callmethod}
		\If{$\mathit{fromStart}$}
		\Cost{$c_{36}$}{$n$}
		\label{alg:findchainoflines4-isfromstart}
			\State \Call{addLine}{$C,\mathit{chainSegment}$}
			\Cost{$c_{37}$}{$n\cdot\Theta(1)$}
		\EndIf
		\State \textbf{return}
		\Cost{$c_{33}$}{$n$}
		\State $i \gets i + 1$
		\Cost{$c_{33}$}{$n$}
	\EndFor
	\label{alg:findchainoflines4-loop-end}
\end{algorithmic}
\end{algorithm}


Danach wird in Zeile \ref{alg:findchainoflines4-callmethod} $\mathit{findChainsOfLines}$ rekursiv aufgerufen, um
 weitere passende Linien zur Markenerkennung zu identifizieren. Abschliessend wird in Zeile
 \ref{alg:findchainoflines4-isfromstart} überprüft, ob $\mathit{fromStart}$ wahr ist, was dazu führt, dass die Linie in
 $\mathit{chainSegment}$ dem Speicherblock $C$ hinzugefügt wird. Danach wird die Ausführung abgebrochen.

Das Verfahren \textproc{findChainOfLines} ist in \textproc{lineDetection}
 (\autoref{alg:linedetection-hirzerquaddetection}) integriert, um aus den gefundenen Linien eine Marke zu konstruieren.

\algblockdefx[DOWHILE]{DO}{WHILE}
   {\textbf{do}}
   [1]{\textbf{while} #1}
\begin{algorithm}[ht]
\caption{\textproc{lineDetection} mit Quadraterkennung}
\label{alg:linedetection-hirzerquaddetection}
	\begin{algorithmic}[1]
		\Require $I$
		\For{$y \gets 0$ \textbf{to} $y < \mathit{imageHeight}$}
			\For{$x \gets 0$ \textbf{to} $x < \mathit{imageWidth}$}
				\State \Call{findEdgels}{$I,E,x,y$}
				\State \Call{findLineSegments}{$E,L$}
				\State \Call{mergeLines}{$L$}
				\State $n \gets$ \Call{getLineCount}{$L$}
				\For{$i \gets 0$ \textbf{to} $i < n$}
					\State \Call{addLineSegment}{$M,$\textproc{getLineSegment}$(L,i)$}
					\State $i \gets i + 1$
				\EndFor
				\State \Call{resetMemoryPool}{$E$}
				\State \Call{resetMemoryPool}{$L$}
				\State $ x \gets x + 40$
			\EndFor
			\State $y \gets y + 40$
		\EndFor
		\State \Call{mergeLines}{$M$}
		\State \Call{extendLinesInPool}{$M$}
		\State \Call{findLinesWithCornersInPool}{$M,L$}
		\DO
		\label{alg:linedetection-hirzerquaddetection-do}
			\State \Call{resetMemoryPool}{$C$}
			\State $\mathit{chain} \gets L\mathit{\to data}[0]$
			\State \Call{removeLine}{$L,0$}
			\label{alg:linedetection-hirzerquaddetection-removeline}
			\State $\mathit{length} \gets 1$
			\State \Call{findChainOfLines}{$L,\mathit{chain},$\textbf{true}$,\mathit{length},C$}
			\label{alg:linedetection-hirzerquaddetection-find1}
			\State \Call{addLine}{$C,\mathit{chain}$}
			\label{alg:linedetection-hirzerquaddetection-addline}
			\If{$\mathit{length} < 4$}
			\label{alg:linedetection-hirzerquaddetection-find2-start}
				\State \Call{findChainOfLines}{$L,\mathit{chain},$\textbf{false}$,\mathit{length},C$}
			\EndIf
			\label{alg:linedetection-hirzerquaddetection-find2-end}
			\If{$\mathit{length} > 2$}
			\label{alg:linedetection-hirzerquaddetection-twolines}
				\State $\mathit{marker} \gets$ \Call{reconstructCorners}{$C$}
				\label{alg:linedetection-hirzerquaddetection-reconstruct}
				\State \Call{addMarker}{$Q,\mathit{marker}$}
				\label{alg:linedetection-hirzerquaddetection-marker}
			\EndIf
		\WHILE{\Call{getLineCount}{$L$}}
		\label{alg:linedetection-hirzerquaddetection-while}
	\end{algorithmic}
\end{algorithm}


Dazu werden alle Linien in der Schleife von Zeile
 \ref{alg:linedetection-hirzerquaddetection-do}--\ref{alg:linedetection-hirzerquaddetection-while} untersucht. Der
 Speicherblock $C$ wir zubeginn jeder Iteration gelöscht, um nur Linien einer Marke zu speichern. Danach wird aus dem
 Speicherblock $L$ die erste Linie ausgelesen und in $\mathit{chain}$ gespeichert. In Zeile
 \ref{alg:linedetection-hirzerquaddetection-removeline} wird die Linie aus dem Speicherblock $L$ entfernt. Die Variable
 $\mathit{length}$ wird anschliessend initialisiert und enthält die Anzahl der Linien einer potentiellen Marke. Mit
 \textproc{findChainOfLines} wird in Zeile \ref{alg:linedetection-hirzerquaddetection-find1}, von der Linie
 $\mathit{chain}$ ausgehend, im Uhrzeigersinn nach Linien gesucht, die zu einem Quadrat zusammengefügt werden können.
 Danach wird $\mathit{chain}$ in Zeile \ref{alg:linedetection-hirzerquaddetection-addline} zu dem Speicherblock $C$
 hinzugefügt. Wenn weniger als vier Linien durch \textproc{findChainOfLines} gefunden wurden, wird das Verfahren
 wiederholt und sucht gegen den Uhrzerigersinn nach weiterten Linien (Zeile
 \ref{alg:linedetection-hirzerquaddetection-find2-start}--\ref{alg:linedetection-hirzerquaddetection-find2-end}). In
 Zeile \ref{alg:linedetection-hirzerquaddetection-twolines} wird überprüft, ob mehr als zwei Linien gefunden wurden.
 Nur wenn mehr als zwei Linien vorhanden sind, lassen sich die Eckpunkte einer Marke in Zeile
 \ref{alg:linedetection-hirzerquaddetection-reconstruct} durch \textproc{reconstructCorners} wiederherstellen. Danach
 wird in Zeile \ref{alg:linedetection-hirzerquaddetection-marker} die Marke $\mathit{marker}$ dem Speicherblock $Q$
 hinzugefügt. Das Verfahren wird solange wiederholt, bis keine Linie mehr im Speicherblock vorhanden ist. Danach sind
 erkannte Marken in Speicherblock $Q$ hinterlegt.

Mit \textproc{reconstructCorners} (\autoref{alg:reconstructcorners}) werden die Eckpunkte einer Marke aufbereitet. Dazu
 benötigt das Verfahren den Speicherblock $C$, mit den Linien der Marke.

\begin{algorithm}[ht]
\caption{\textproc{reconstructCorners}}
\label{alg:reconstructcorners}
\begin{algorithmic}[1]
	\Require $C$
	\State $\mathit{marker} \gets \infty$
	\State $\mathit{c1} \gets$ \Call{intersection}{$\mathit{C.data}[0],\mathit{C.data}[1]$}
	\label{alg:reconstructcorners-c1}
	\State $\mathit{c2} \gets$ \Call{intersection}{$\mathit{C.data}[1],\mathit{C.data}[2]$}
	\label{alg:reconstructcorners-c2}
	\State $\mathit{c3} \gets \infty$
	\State $\mathit{c4} \gets \infty$
	\If{\Call{getLineCount}{$C$}$ == 4$}
	\label{alg:reconstructcorners-4lines-start}
		\State $\mathit{c3} \gets$ \Call{intersection}{$\mathit{C.data}[2],\mathit{C.data}[3]$}
		\State $\mathit{c4} \gets$ \Call{intersection}{$\mathit{C.data}[3],\mathit{C.data}[0]$}
	\label{alg:reconstructcorners-4lines-end}
	\Else
	\label{alg:reconstructcorners-else-start}
		\State $\mathit{c3} \gets \mathit{C.data}[2]\mathit{.end.coordinate}$
		\State $\mathit{c4} \gets \mathit{C.data}[0]\mathit{.start.coordinate}$
	\EndIf
	\label{alg:reconstructcorners-else-end}
	\State $\mathit{marker.c1} \gets \mathit{c1}$
	\label{alg:reconstructcorners-return-start}
	\State $\mathit{marker.c2} \gets \mathit{c2}$
	\State $\mathit{marker.c3} \gets \mathit{c3}$
	\State $\mathit{marker.c4} \gets \mathit{c4}$
	\State \textbf{return} $\mathit{marker}$
	\label{alg:reconstructcorners-return-end}
\end{algorithmic}
\end{algorithm}


In Zeile \ref{alg:reconstructcorners-c1} wird der erste Eckpunkt erstellt, indem der Schnittpunkt der Linien an
 Position $1$ und Position $2$ ermittelt wird. Für den zweiten Eckpunkt in Zeile \ref{alg:reconstructcorners-c2} werden
 die Linien $2$ und $3$ verwendet. Wenn exakt vier Linien gefunden wurden, werden die Eckpunkte für $\mathit{c3}$ und
 $\mathit{c4}$ durch den Schnittpunkt der Linien ermittelt (Zeile
 \ref{alg:reconstructcorners-4lines-start}--\ref{alg:reconstructcorners-4lines-end}). Ansonsten werden die Eckpunkte in
 Zeile \ref{alg:reconstructcorners-else-start}--\ref{alg:reconstructcorners-else-end} nur durch die Linienenden
 dargestellt werden, was möglicherweise zu ungenauen Ergebnissen führt (Vgl. \autoref{sub:quadrangle_detection}). In
 Zeile \ref{alg:reconstructcorners-return-start}--\ref{alg:reconstructcorners-return-end} werden die Eckpunkte in
 $\mathit{marker}$ gespeichert und als Ergebnis an die aufrufende Methode zurückgeliefert.

% section quadrangle_detection (end)

% section hirzer (end)

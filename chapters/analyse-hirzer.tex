\section{Verfahren nach Hirzer} % (fold)
\label{sec:hirzer}

In diesem Kapitel wird das Verfahren von \citeauthor{hirzer08} untersucht. Die im Rahmen dieser Arbeit entstandene
 Implementierung verwendet eigene Datentypen, die in den ersten Abschnitten beschrieben werden. Danch wird das
 Verfahren von \citeauthor{clarke96} beschrieben, auf dem das Verfahren von \citeauthor{hirzer08} aufgebaut ist. Danach
 werden die Optimierungen von \citeauthor{hirzer08} in \autoref{sub:linienerkennung_nach_hirzer08} untersucht. In den
 restlichen Abschnitten des Kapitels werden die Verfahren zur Linien Erweiterung (Line Extension) und Quadraterkennung
 (Quadrangle Detection) beschrieben.

\subsection{Datenstrukturen} % (fold)
\label{sub:datenstrukturen}

In diesem Abschnitt werden die verwendeten Datenstrukturen für Vektoren, \gls{edgel}, Liniensegmente und Marken
 beschrieben und analysiert.

\subsubsection{Vektor} % (fold)
\label{sub:vektor}

Die meisten Datenstrukturen in der Implementierung des Verfahrens nach \citeauthor{hirzer08}, benötigen als Basis die
 Datenstruktur $\textproc{vector}$. Bei der Datenstruktur handelt es sich um eine Vektor in $\mathbb{R}^2$ und benötigt
 die beiden Variablen $x$ und $y$ (Vgl. \autoref{alg:vector}).
\begin{algorithm}[!ht]
\caption{\textproc{vector}}
\label{alg:vector}
	\begin{algorithmic}[1]
		\State $x$
		\State $y$
	\end{algorithmic}
\end{algorithm}

Operationen mit der Datenstruktur \textproc{vector} umfassen

\begin{itemize}
	\item Addition,
	\item Subtraktion,
	\item Skalarprodukt,
	\item Quadratische Länge eines Vektors,
	\item Länge eines Vektors und
	\item Normalisieren von Vektoren.
\end{itemize}

Auf die Variablen $x$ und $y$ der Datenstruktur kann direkt zugegriffen werden, sowohl lesend als auch schreibend. Zum
 schreiben der Variablen kann auch \autoref{alg:vectorsetcoordinate} verwendet werden.
\begin{algorithm}[ht]
\caption{\textproc{vectorSetCoordinate}}
\label{alg:vectorsetcoordinate}
	\begin{algorithmic}[1]
		\Require $\mathit{vector}, x, y$
		\State $\mathit{vector \to x} \gets x$
		\State $\mathit{vector \to y} \gets y$
	\end{algorithmic}
\end{algorithm}

\textproc{vectorSetCoordinate} benötigt einen Vektor, dem die Werte $x$ und $y$ hinzugefügt werden. Der Zugriff auf die
 Datenstruktur \textproc{vector} ist konstant. Die Laufzeitfunktion ist $T(n) = 4$.

\paragraph{Addition:} % (fold)
\label{par:addition}

Die Addition von zwei Vektoren ist in \autoref{alg:vectoradd} aufgeführt.
\begin{algorithm}[!ht]
\caption{\textproc{vectorAddition}}
\label{alg:vectoraddition}
	\begin{algorithmic}[1]
		\Require $\mathit{left}, \mathit{right}$
		\State $\mathit{vector} \gets \infty$
		\State \Call{vectorSetCoordinate}{$\mathit{vector},\mathit{left.x} + \mathit{right.x}, \mathit{left.y} + \mathit{right.y}$}
		\State \textbf{return} $\mathit{vector}$
	\end{algorithmic}
\end{algorithm}

Das Verfahren benötigt zwei Vektoren $\mathit{left}$ und $\mathit{right}$, die addiert werden sollen. Dazu wird ein
 neuer Vektor $\mathit{vector}$ initialisiert, der mit Hilfe von \textproc{vectorSetCoordinate} die addierten Werte von
 $\mathit{left}$ und $\mathit{right}$ zugewiesen bekommt. Danach wird der Vektor an die aufrufenden Funktion
 zurückgegeben. Die Laufzeitfunktion ist $T(n) = 12$.

% paragraph addition (end)

\paragraph{Subtraktion:} % (fold)
\label{par:subtraktion}

Die Subtraktion von zwei Vektoren (\autoref{alg:vectorsubtract}) ähnelt dem Verfahren der Addition.
\begin{algorithm}[!ht]
\caption{\textproc{vectorSubtract}}
\label{alg:vectorsubtract}
	\begin{algorithmic}[1]
		\Require $\mathit{left}, \mathit{right}$
		\State $\mathit{vector} \gets \infty$
		\State \Call{vectorSetCoordinate}{$\mathit{vector},\mathit{left.x} - \mathit{right.x}, \mathit{left.y} - \mathit{right.y}$}
		\State \textbf{return} $\mathit{vector}$
	\end{algorithmic}
\end{algorithm}

Auch hier wird durch die Parameter $\mathit{left}$ und $\mathit{right}$ ein Vektor initialisiert und an die
 aufrufende Funktion zurückgeliefert. Die Laufzeitfunktion ist ebenfalls $T(n) = 12$.

% paragraph subtraktion (end)

\paragraph{Skalarprodukt:} % (fold)
\label{par:skalarprodukt}

Das Skalarprodukt der Vektoren $\mathit{left}$ und $\mathit{right}$ wird durch \autoref{alg:vectordotproduct} direkt
 zurückgegeben.
\begin{algorithm}[!ht]
\caption{\textproc{dotProduct}}
\label{alg:vectordotproduct}
	\begin{algorithmic}[1]
		\Require $\mathit{left}, \mathit{right}$
		\State \textbf{return} $\left(\mathit{left.x} \cdot \mathit{right.x}\right) + \left(\mathit{left.y} \cdot \mathit{right.y}\right)$
	\end{algorithmic}
\end{algorithm}

Die Laufzeitfunktion ist $T(n) = 7$.

% paragraph skalarprodukt (end)

\paragraph{Länge eines Vektors:} % (fold)
\label{par:länge_eines_vektors}

Die quadratische Länge eines Vektors (\autoref{alg:vectorsquaredlength}) benötigt als Parameter einen Vektor
 $\mathit{vector}$ und berechnet durch direkten Zugriff auf die Datenstruktur die quadratische Länge. Die
 Laufzeitfunktion ist $T(n) = 7$.
\begin{algorithm}[ht]
\caption{\textproc{squaredLength}}
\label{alg:vectorsquaredlength}
	\begin{algorithmic}[1]
		\Require $\mathit{vector}$
		\State \textbf{return} $\left(\mathit{vector \to x} \cdot \mathit{vector \to x}\right) + \left(\mathit{vector \to y} \cdot \mathit{vector \to y}\right)$
	\end{algorithmic}
\end{algorithm}

Das Verfahren \textproc{length} (\autoref{alg:vectorlength}) nutzt \autoref{alg:vectorsquaredlength} zur Berechnung der
 Länge eines Vektors.
\begin{algorithm}[!ht]
\caption{\textproc{length}}
\label{alg:vectorlength}
	\begin{algorithmic}[1]
		\Require $\mathit{vector}$
		\State \textbf{return} \Call{sqrt}{\textproc{squaredlength}$(\mathit{vector})$}
	\end{algorithmic}
\end{algorithm}

Die Laufzeit von \autoref{alg:vectorlength} ist abhängig von der Funktion \textproc{sqrt}\footcite[Vgl.][]{sqrtf},
 deren Laufzeit durch ein Testprogramm ermittlet wurde. Dazu wurden $300$ Datenpunkte erfasst und grafisch dargestellt.
 In \autoref{fig:regression-sqrtf} ist zu erkennen, dass die gemessenen Werte keinen linearen Zusammenhang
\begin{figure}[!ht]
	\centering
	\input{resources/Regression-sqrtf.pdf_tex}
	\caption{Regressionsanalyse von \textproc{sqrt}. $300$ Datenpunkte in logarithmischer Darstellung ($x$-Achse). Der
	 Mittelwert der Daten ist als grüne Linie eingezeichnet.}
	\label{fig:regression-sqrtf}
\end{figure}
 aufweisen. Der Median und Mittelwert liegt jeweils bei $0$. Damit ist die Laufzeit von \textproc{sqrt}, und somit von
 \textproc{length}, konstant. Die Laufzeitfunktion ist $T(n) = 9$.

% paragraph länge_eines_vektors (end)

\paragraph{Normalisieren von Vektoren:} % (fold)
\label{par:normalisieren_von_vektoren}

Das Verfahren zum normalisieren von Vektoren ist in \autoref{alg:vectornormalized} beschrieben und benötigt als
 Parameter einen Vektor.
\begin{algorithm}[!ht]
\caption{\textproc{normalized}}
\label{alg:vectornormalized}
	\begin{algorithmic}[1]
		\Require $\mathit{vector}$
		\State $\mathit{invertedLength} \gets 1 /$ \Call{length}{$\mathit{vector}$}
		\Cost{$c_{1}$}{$2 + 9$}
		\State $\mathit{vector.x} \gets \mathit{vector.x} \cdot \mathit{invertedLength}$
		\Cost{$c_{2}$}{$4$}
		\State $\mathit{vector.y} \gets \mathit{vector.y} \cdot \mathit{invertedLength}$
		\Cost{$c_{3}$}{$4$}
	\end{algorithmic}
\end{algorithm}

Durch \textproc{length} (\autoref{alg:vectorlength}) wird die Länge des Vektors berechnet und direkt im Parameter
 gespeichert. Die Laufzeit von \autoref{alg:vectornormalized} ist konstant. Die Laufzeitfunktion ist $T(n) = 19$.

% paragraph normalisieren_von_vektoren (end)

Die vorgestellten Operationen (\autoref{alg:vectorsetcoordinate}--\autoref{alg:vectornormalized}) arbeiten in konstanter
 Laufzeit durch den direkten Zugriff auf die Variablen der Datenstruktur \textproc{vector}.

% subsubsection vektor (end)


\subsubsection{Edgels} % (fold)
\label{sub:datenstruktur-edgels}

Die Datenstruktur eines \glspl{edgel} besteht aus den beiden Variablen $\mathit{coordinate}$ und $\mathit{slope}$
 (\autoref{alg:datastructure-edgel}). Beide Variablen sind vom Datentyp \textproc{vector} (Vgl.
 \autoref{alg:vector}), sodass Lese- und Schreibzugriffe auf die Elemente eines \glspl{edgel} konstant sind.
\begin{algorithm}[ht]
\caption{Datenstruktur eines Edgels}
\label{alg:datastructure-edgel}
	\begin{algorithmic}[1]
		\State $x$
		\State $y$
		\State $o$
		\Comment Orientierung
	\end{algorithmic}
\end{algorithm}


Wie in \autoref{sub:line_detection} beschrieben, muss im Verfahren von \citeauthor{hirzer08} die Orientierung von zwei
 \glspl{edgel} überprüft werden, was mit \autoref{alg:edgeliscompatible} bewerkstelligt wird. Als Parameter werden
 zwei zu vergleichende \gls{edgel} $\mathit{left}$ und $\mathit{right}$ übergeben.
\begin{algorithm}[ht]
\caption{\textproc{isCompatible}}
\label{alg:edgeliscompatible}
\begin{algorithmic}[1]
	\Require $\mathit{left}, \mathit{right}$
	\State \textbf{return} \Call{dotProduct}{$\mathit{left}, \mathit{right}$} $> 0.38$
\end{algorithmic}
\end{algorithm}

Mithilfe von \textproc{dotProduct} (\autoref{alg:vectordotproduct}) kann die Orientierung der beiden \gls{edgel}
 überprüft werden. Zwei \gls{edgel} sind dann kompatibel, wenn der Winkel zwischen den Vektoren nicht größer als
 $67.5^\circ$ ist\footcite[Vgl.][S.~417]{clarke96}. Dies wird durch \autoref{eq:edgeliscompatible} in
 \autoref{alg:edgeliscompatible} sichergestellt. Es ist dabei darauf zu achten, dass die Berechnung in Bogenmaß
 erfolgt. Die Laufzeitfunktion ist $T(n) = 8$.

\begin{equation}
	\label{eq:edgeliscompatible}
	\cos \left(67.5\right) \approx 0.38
\end{equation}

Der Edgelspeicher in \autoref{alg:datastructure-edgelpool} verwendet ein Array von \glspl{edgel}
 (Vgl. \autoref{alg:datastructure-edgel}) mit fester Größe $N$ und einer Zählvariable, um die nächste freie Position im
 Array zu markieren.
\begin{algorithm}[!ht]
\caption{\textproc{edgelPool} (Datenstruktur)}
\label{alg:datastructure-edgelpool}
\begin{algorithmic}[1]
	\State $\mathit{data}[N]$
	\Comment Anzahl der Einträge
	\State $\mathit{count}$
\end{algorithmic}
\end{algorithm}

Der Speichervorrat aus \autoref{alg:datastructure-poolimplementation} ist ein Array vom Typ
 \textproc{edgelPool} mit der Größe $S$, dessen Adresse im Zeiger $\mathit{pool}$ gespeichert wird.
\begin{algorithm}[!ht]\small
\caption{\textproc{edgelPool} (Speichervorrat)}
\label{alg:datastructure-poolimplementation}
\begin{algorithmic}[1]
	\State $\mathit{data}[S]$
	\Comment Anzahl der Pools
	\State $\mathit{pool}$
\end{algorithmic}
\end{algorithm}

\autoref{alg:edgelpool-getmemorypools} basiert auf einem einfachen Stack Allocator von
 \citeauthor{kr}\footcite[Vgl.][S.~100--104]{kr}.
\input{alg/analyse-hirzer/datastructure-edgelpool-getmemorypools}
Die Variable $n$ gibt die Anzahl der angeforderten Blöcke aus dem Speichervorrat an. In Zeile
 \ref{alg:edgelpool-getmemorypools-checkpoolsize} wird überprüft, ob genügend Blöcke zur Verfügung stehen und liefert
 im Erfolgsfall die Adresse zu einem Block (\autoref{alg:datastructure-edgelpool}) zurück. Falls keine Blöcke mehr zur
 Verfügung stehen, wird $\mathit{NULL}$ zurückgegeben. Die Laufzeitfunktion ist im schlechtesten Fall $T(n)=7$.
 \autoref{alg:edgelpool-getmemorypool}
\begin{algorithm}[ht]
\caption{\textproc{getMemoryPool}}
\label{alg:edgelpool-getmemorypool}
\begin{algorithmic}[1]
	\State $p \gets$ \Call{getmemorypools}{1}
	\State \textbf{return} $p$
\end{algorithmic}
\end{algorithm}
 vereinfacht die Anforderung von Speicherblöcken, da in den meisten Fällen nur ein Block benötigt wird. Bei einem
 Aufruf kann somit auf einen Parameter verzichtet werden. Die Laufzeitfunktion ist im schlechtesten Fall $T(n)=9$.
 Sowohl \autoref{alg:edgelpool-getmemorypools} als auch \autoref{alg:edgelpool-getmemorypool} haben eine konstante
 Laufzeit.

Um \glspl{edgel} in einem Block zu speichern, wird \autoref{alg:edgelpool-addedgel} verwendet.
\begin{algorithm}[!ht]
\caption{\textproc{addEdgel}}
\label{alg:edgelpool-addedgel}
\begin{algorithmic}[1]
	\Require $p,e$
	\If{$\lnot p$}
	\label{alg:edgelpool-addedgel-validpointer-start}
		\State \textbf{return}
	\EndIf
	\label{alg:edgelpool-addedgel-validpointer-end}
	\If{$\lnot \left(\mathit{p.count} < N\right)$}
	\label{alg:edgelpool-addedgel-checkspace-start}
		\State \textbf{return} \Comment Speicher voll
	\EndIf
	\label{alg:edgelpool-addedgel-checkspace-end}
	\State $c \gets \mathit{p.count}$
	\label{alg:edgelpool-addedgel-add-start}
	\State $\mathit{p.data}[c] \gets e$
	\State $\mathit{p.count} \gets c + 1$
	\label{alg:edgelpool-addedgel-add-end}
\end{algorithmic}
\end{algorithm}

Der Algorithmus benötigt einen Zeiger $p$ auf einen Speicherblock und einen \gls{edgel} $e$. In Zeile
 \ref{alg:edgelpool-addedgel-validpointer-start}--\ref{alg:edgelpool-addedgel-validpointer-end} wird geprüft, ob der
 Zeiger auf eine Adresse verweist. Falls $p$ null ist, wird der Algorithmus verlassen. In Zeile
 \ref{alg:edgelpool-addedgel-checkspace-start}--\ref{alg:edgelpool-addedgel-checkspace-end} wird geprüft, ob im Array
 genügend Platz für einen weiteren Eintrag vorhanden ist. Die Größe von $N$ Einträgen richtet sich nach der in
 \autoref{alg:datastructure-edgelpool} festgelegten Arraygröße $N$. Wenn genügend Platz vorhanden ist, wird in Zeile
 \ref{alg:edgelpool-addedgel-add-start}--\ref{alg:edgelpool-addedgel-add-end} der \gls{edgel} $e$ an die freie
 Position $c$ geschrieben. Danach wird $\mathit{count}$ inkrementiert. Das Hinzufügen eines \glspl{edgel} ist konstant
 und die Laufzeitfunktion entspricht im schlechtesten Fall $T(n) = 12$.

Die Position eines \glspl{edgel} kann mit \autoref{alg:edgelpool-edgelposition} gesucht werden.
\begin{algorithm}[!ht]
\caption{\textproc{edgelPosition}}
\label{alg:edgelpool-edgelposition}
\begin{algorithmic}[1]
	\Require $p,e$
	\If{$\lnot p$}
	\label{alg:edgelpool-edgelposition-validpointer-start}
		\State \textbf{return} $- 1$
	\EndIf
	\label{alg:edgelpool-edgelposition-validpointer-end}
	\State $n \gets \mathit{p.count}$
	\label{alg:edgelpool-edgelposition-count}
	\State $ \mathit{sx} \gets \mathit{e.slope.x}$
	\label{alg:edgelpool-edgelposition-e-start}
	\State $ \mathit{sy} \gets \mathit{e.slope.y}$
	\State $ x \gets \mathit{e.coordinate.x}$
	\State $ y \gets \mathit{e.coordinate.y}$
	\label{alg:edgelpool-edgelposition-e-end}
	\For{$i \gets 0$ \textbf{to} $i < n$}
	\label{alg:edgelpool-edgelposition-search-start}
		\State $c \gets \mathit{p.data}[i]$
		\If{$\mathit{sx} = \mathit{c.slope.x} \land \mathit{sy} = \mathit{c.slope.y} \land x =
		 \mathit{c.coordinate.x} \land y = \mathit{c.coordinate.y}$}
		\label{alg:edgelpool-edgelposition-isequal-start}
			\State \textbf{return} $i$
			\label{alg:edgelpool-edgelposition-returni}
		\EndIf
		\label{alg:edgelpool-edgelposition-isequal-end}
		\State $i \gets i + 1$
	\EndFor
	\label{alg:edgelpool-edgelposition-search-end}
	\State \textbf{return} $- 1$
	\label{alg:edgelpool-edgelposition-returnerror}
\end{algorithmic}
\end{algorithm}

Dazu wird der Zeiger $p$ auf den Speicherblock und das zu suchende \gls{edgel} übergeben. In Zeile
 \ref{alg:edgelpool-edgelposition-validpointer-start}--\ref{alg:edgelpool-edgelposition-validpointer-end} wird
 überprüft, ob der Zeiger $p$ auf einen gültigen Speicherblock verweist. In Zeile
 \ref{alg:edgelpool-edgelposition-count} wird die Anzahl der eingetragenen \gls{edgel} ausgelesen. In Zeile
 \ref{alg:edgelpool-edgelposition-e-start}--\ref{alg:edgelpool-edgelposition-e-end} werden die Daten des \gls{edgel}
 ausgelesen. Die Suche des \gls{edgel} erfolgt in Zeile
 \ref{alg:edgelpool-edgelposition-search-start}--\ref{alg:edgelpool-edgelposition-search-end}. Dazu wird in Zeile
 \ref{alg:edgelpool-edgelposition-isequal-start}--\ref{alg:edgelpool-edgelposition-isequal-end} ein \gls{edgel} an der
 aktuellen Position $i$ mit den lokalen Variablen verglichen. Stimmen die Werte überein, wird die Position $i$
 zurückgegeben (Zeile \ref{alg:edgelpool-edgelposition-returni}). Andernfalls wird $i$ inkrementiert und die Suche
 fortgesetzt. Wenn das \gls{edgel} nicht im Speicherblock hinterlegt ist, wird in Zeile
 \ref{alg:edgelpool-edgelposition-returnerror} ein Fehlerwert zurückgegeben. Die Laufzeit des Algorithmus ist im besten
 Fall konstant, wenn das zu suchende \gls{edgel} an der ersten Position gespeichert ist. Im schlimmsten Fall, wenn ein
 \gls{edgel} nicht gefunden wird, entspricht die Laufzeitfunktion $T(n) = 21n + 18 $ für
 $n = \text{Anzahl der \gls{edgel}}$. Die Wachstumsrate ist $21n + 18 = \Theta(n)$ für $c_{1}=20$, $c_{2}=21$ und
 $n_{0} = 18$.

\glspl{edgel} werden mittels \autoref{alg:edgelpool-getedgel} aus einem Speicherblock gelesen.
\begin{algorithm}[!ht]
\caption{\textproc{getEdgel}}
\label{alg:edgelpool-getedgel}
\begin{algorithmic}[1]
	\Require $p,i$
	\If{$\lnot p$}
	\Cost{$c_{1}$}{$1$}
	\label{alg:edgelpool-getedgel-validpointer-start}
		\State \textbf{return}
		\Cost{$c_{2}$}{$1$}
	\EndIf
	\label{alg:edgelpool-getedgel-validpointer-end}
	\State $c \gets \mathit{p.count}$
	\Cost{$c_{4}$}{$2$}
	\If{$\lnot \left(c > i\right)$}
	\Cost{$c_{5}$}{$2$}
	\label{alg:edgelpool-getedgel-validrange-start}
		\State \textbf{return}
		\Cost{$c_{6}$}{$1$}
	\EndIf
	\label{alg:edgelpool-getedgel-validrange-end}
	\State \textbf{return} $\mathit{p.data}[i]$
	\Cost{$c_{8}$}{$2$}
	\label{alg:edgelpool-getedgel-returnedgel}
\end{algorithmic}
\end{algorithm}
Dazu wird der Zeiger $p$ auf den Speicherblock und der Index $i$ übergeben. In Zeile
 \ref{alg:edgelpool-getedgel-validpointer-start}--\ref{alg:edgelpool-getedgel-validpointer-end} wird geprüft, ob es
 sich um einen gesetzten Zeiger handelt. Anschließend wird in Zeile
 \ref{alg:edgelpool-getedgel-validrange-start}--\ref{alg:edgelpool-getedgel-validrange-end} geprüft, ob der Index $i$
 innerhalb des gespeicherten Bereichs der \glspl{edgel} liegt. Danach wird in Zeile
 \ref{alg:edgelpool-getedgel-returnedgel} der Wert des \glspl{edgel} an Position $i$ zurückgegeben. Der Zugriff auf
 einen \gls{edgel} ist konstant. Die Laufzeitfunktion ist im schlechtesten Fall $T(n) = 7$.

Damit \glspl{edgel} aus einem Array entfernt werden können, wird \autoref{alg:edgelpool-removeedgel} verwendet.
\begin{algorithm}[!ht]\small
\caption{\textproc{removeEdgel}}
\label{alg:edgelpool-removeedgel}
\begin{algorithmic}[1]
	\Require $p,e$
	\State $\mathit{position} \gets$ \Call{edgelPosition}{$p,e$}
	\Cost{$c_{1}$}{$1 + 21n + 18$}
	\label{alg:edgelpool-removeedgel-position}
	\If{$\mathit{position} < 0$}
	\Cost{$c_{2}$}{$1$}
		\State \textbf{return}
		\Cost{$c_{3}$}{$1$}
		\label{alg:edgelpool-removeedgel-error}
	\EndIf
	\label{alg:edgelpool-removeedgel-best-end}
	\State $c \gets \mathit{p.count}$
	\Cost{$c_{5}$}{$2$}
	\If{$c > \mathit{position} + 1$}
	\Cost{$c_{6}$}{$2$}
	\label{alg:edgelpool-removeedgel-isvalid}
		\State \Call{memmove}{$\mathit{p.data}[\mathit{position}], \mathit{p.data}[\mathit{position} + 1],
		 \left(c - \mathit{position} + 1\right) \cdot \textproc{sizeof}(e)$}
		\Cost{$c_{7}$}{$9$}
		\label{alg:edgelpool-removeedgel-memmove}
	\EndIf
	\State $\mathit{p.count} \gets \mathit{p.count} - 1$
	\Cost{$c_{9}$}{$4$}
\end{algorithmic}
\end{algorithm}
Es wird der Zeiger $p$ auf einen Speicherblock und das zu löschenden \glspl{edgel} übergeben. In Zeile
 \ref{alg:edgelpool-removeedgel-position} wird mit \textproc{edgelPosition} (\autoref{alg:edgelpool-edgelposition}) die
 Position des \gls{edgel} gesucht. Falls das \gls{edgel} nicht gefunden wurde, wird in Zeile
 \ref{alg:edgelpool-removeedgel-error} das Verfahren abgebrochen. Ansonsten gibt es zwei zu behandelnde Fälle um ein
 \gls{edgel} zu löschen. Das \gls{edgel} liegt
\begin{enumerate}
	\item nicht am Ende des Arrays oder \label{removeedgel-worst}
	\item liegt am Ende des Arrays. \label{removeedgel-best}
\end{enumerate}
Bei \autoref{removeedgel-best} muss lediglich $\mathit{count}$ dekrementiert werden um auf den vorigen Wert zu
 verweisen (Vgl. \autoref{fig:decrementcounter}). Das dekrementieren der Zählvariable $\mathit{p.count}$ ist eine
 Zuweisung in konstanter Zeit.
\begin{figure}[!ht]
	\centering
	\subfigure[]{
		\input{resources/Memory-Decrement-Before.pdf_tex}
		\label{fig:decrementcounter-before}
	}
	\subfigure[]{
		\input{resources/Memory-Decrement-After.pdf_tex}
		\label{fig:decrementcounter-after}
	}
	\caption{Dekrementieren von $\mathit{count}$. In \subref{fig:decrementcounter-before} soll Position $i$ gelöscht
	 werden. $c$ verweist auf die nächste freie Speicherstelle. In \subref{fig:decrementcounter-after} wird $c$
	 dekrementiert und verweist auf die neue freie Speicherstelle.}
	\label{fig:decrementcounter}
\end{figure}
Bei \autoref{removeedgel-worst} wird das Array an der Position $\mathit{position}$ geteilt und der Wertebereich von
 $[\mathit{position}+1 \dotsc \mathit{position}-n]$ wird an die Position $\mathit{position}$ verschoben
 (Vgl. \autoref{fig:memmove}).
\begin{figure}[!ht]
	\centering
	\subfigure[]{
		\input{resources/Memory-Move-Before.pdf_tex}
		\label{fig:memmove-before}
	}
	\subfigure[]{
		\input{resources/Memory-Move-After.pdf_tex}
		\label{fig:memmove-after}
	}
	\caption{Verschieben des Speicherinhalts. In \subref{fig:memmove-before} soll Position $i$ gelöscht werden. Die
	 grau schattierten Einträge werden von ihrer Position an Position $i$ verschoben \subref{fig:memmove-after}.}
	\label{fig:memmove}
\end{figure}
In Zeile \ref{alg:edgelpool-removeedgel-memmove} gibt der Operator \textproc{sizeof}($e$) die Speichergröße eines
 \glspl{edgel} an, welche zum verschieben der Daten notwendig ist. Mit $c - \mathit{position} + 1$ wird die Anzahl der
 zu verschiebenden Einträge ermittelt. Im worst-case werden $N-1$ Einträge an Position $0$ des Arrays verschoben.

Um die Laufzeit der Funktion \textproc{memmove} zu bestimmen, wurde ein Testprogramm geschrieben, dass die Zeit misst,
\label{sub:datenstruktur-edgels-memmove}
 die benötigt wird, um Einträge zu verschieben. Anhand der Daten wurde mittels einer Regressionsanalyse untersucht, ob
 die gemessenen Daten einen linearen Zusammenhang aufweisen. Die erfassten $2200$ Datenpunkte wurden nach dem Vorbild
 von \textproc{time}\footcite{time-1} ermittelt um Real-, User- und Sys-Zeit zu bestimmen. Dabei wurde ein Bereich von
 $4$ Byte bis $8388608$ Byte $= 8$ MByte als Eingabe für \textproc{memmove} verwendet. Aus User- und Sys-Zeit wurde
 die CPU-Zeit bestimmt, die zur Analyse benutzt wurde. Der Korrelationskoeffizient für $X = \mathit{BYTES}$ und
 $Y = \mathit{CPU}$ beträgt $r = 0.9754288$ und das Bestimmungsmaß $r^2 = 0.9515$. Der Interzept beträgt
 $\beta_0 = -37.77\e{-06}$ (Abweichung von $58.55\e{-06}$) und die Steigung $\beta_1 = 5.885\e{-09}$
 (Abweichung von $28.35\e{-12}$). Daraus ergibt sich eine Laufzeit von $T(n) =\Theta(n)$
 (Vgl.~\autoref{eq:analyse-removeedgel-worst}).
\begin{equation}
	\label{eq:analyse-removeedgel-worst}
	\begin{split}
		y& = \beta_0 + \beta_1 \cdot n\\
		 & = -37.77\e{-06} + 5.885\e{-09} \cdot n\\
		T(n)& = -37.77\e{-06} + 5.885\e{-09} \cdot n\\
		 & = \Theta(n)
	\end{split}
\end{equation}
In \autoref{fig:regression-memmove} sind die Daten grafisch dargestellt.
\begin{figure}[!ht]
	\centering
	\input{resources/Regression-memmove.pdf_tex}
	\caption{Lineares Modell der CPU Zeit mit $2200$ Datenpunkten. In der Darstellung sind die Konfidenzintervalle für
	 $95\%$ (grüne Linie) und für $99\%$ (rote Linie) für die vorhergesagten Werte enthalten. Die Regressionsgerade ist
	 als blaue Linie eingezeichnet.}
	\label{fig:regression-memmove}
\end{figure}

In dem Verfahren nach \citeauthor{hirzer08} ist die Menge der \gls{edgel}, und somit der Speicher, begrenzt
 (Vgl. \autoref{alg:datastructure-edgelpool}). In der Implementierung des Verfahrens nach \citeauthor{hirzer08} werden
 maximal $N = 8192$ \gls{edgel} gespeichert. Die Regressionsanalyse wurde mit einer Eingabemenge von $4$ \gls{edgel}
 bis $8192$ \gls{edgel} wiederholt. Dies entspricht einer Speichergröße von $64$ Byte bis $131072$ Byte. Der
 Korrelationskoeffizient $r = 0.1332313$ und das Bestimmungsmaß $r^2 = 0.01775$ zeigen, dass eine lineare Abhängigkeit
 in diesem Bereich unwahrscheinlich ist. Wie in \autoref{fig:regression-memmove2} zusehen ist, sind Mittelwert und
 Median parallel zur $x$-Achse. Somit ist die Laufzeit von \textproc{memmove}, für die Untersuchung des Verfahrens nach
 Hirzer mit maximal $8192$ Einträgen, konstant.
\begin{figure}[!ht]
	\centering
	\input{resources/Regression-memmove2.pdf_tex}
	\caption{Regressionsanalyse mit $1200$ Datenpunkten. Der Mittelwert ist als grüne Linie eingezeichnet und der
	 Median als rote Linie.}
	\label{fig:regression-memmove2}
\end{figure}
Im schlechtesten Fall wird ein zu löschender \gls{edgel} in \autoref{alg:edgelpool-removeedgel} am Ende der Liste
 gefunden. Dadurch beträgt die Laufzeit $T(n) = 21n + 37$. Die Wachstumsrate ist, für $c_{1} = 20$, $c_{2} = 21$ und
 $n_{0} = 37$, $21n + 37= \Theta(n)$.

Um einen Speicherblock für einen neuen Durchlauf zu löschen, kommt \autoref{alg:edgelpool-resetmemorypool} zum Einsatz.
\begin{algorithm}[!ht]
\caption{\textproc{resetMemoryPool}}
\label{alg:edgelpool-resetmemorypool}
\begin{algorithmic}[1]
	\Require $p$
	\If{$\lnot p$}
	\label{alg:edgelpool-resetmemorypool-validpointer-start}
		\State \textbf{return}
	\EndIf
	\label{alg:edgelpool-resetmemorypool-validpointer-end}
	\State $\mathit{p.count} \gets 0$
	\label{alg:edgelpool-resetmemorypool-reset}
\end{algorithmic}
\end{algorithm}

 Als Parameter wird der Zeiger $p$ übergeben und in Zeile
 \ref{alg:edgelpool-resetmemorypool-validpointer-start}--\ref{alg:edgelpool-resetmemorypool-validpointer-end}
 überprüft. Um alle Daten als gelöscht zu markieren, wird lediglich die Zählvariable in Zeile
 \ref{alg:edgelpool-resetmemorypool-reset} auf $0$ gesetzt. Die Zuweisung erfolgt in konstanter Zeit. Die
 Laufzeitfunktion ist im schlechtesten Fall $T(n) = 3$.

Wenn ein Speicherblock nicht mehr benötigt wird, kann er mit \autoref{alg:edgelpool-freememorypool} freigegeben werden.
\begin{algorithm}[!ht]
\caption{\textproc{freeMemoryPool}}
\label{alg:edgelpool-freememorypool}
\begin{algorithmic}[1]
	\Require $p$
	\If{$\lnot p$}
	\label{alg:edgelpool-freememorypool-validpointer-start}
		\State \textbf{return}
	\EndIf
	\label{alg:edgelpool-freememorypool-validpointer-end}
	\State \Call{resetMemoryPool}{$p$}
	\label{alg:edgelpool-freememorypool-resetmemory}
	\If{$p \geq \mathit{data} \land p \leq \mathit{data} + S$}
	\label{alg:edgelpool-freememorypool-checkpointer}
		\State $\mathit{pool} \gets p$
	\EndIf
\end{algorithmic}
\end{algorithm}

 Der Zeiger $p$ wird in Zeile
 \ref{alg:edgelpool-freememorypool-validpointer-start}--\ref{alg:edgelpool-freememorypool-validpointer-end} überprüft.
 In Zeile \ref{alg:edgelpool-freememorypool-resetmemory} werden die Daten des Blocks als gelöscht markiert.
 (Vgl. \autoref{alg:edgelpool-resetmemorypool}). Im Anschluss wird in Zeile
 \ref{alg:edgelpool-freememorypool-checkpointer} überprüft, ob $p$ zu dem Array $\mathit{data}$ gehört und nicht größer
 als die definierte Speichergröße ist. Wenn der Test positiv ausfällt, wird der Zeiger $p$ zur weiteren Verwendung in
 $\mathit{pool}$ gespeichert. Das Freigeben eines Speicherblocks erfolgt in konstanter Zeit. Im schlechtesten Fall ist
 die Laufzeitfunktion $T(n) = 9$.

Die Anzahl der \glspl{edgel} in einem Speicherblock werden durch \autoref{alg:edgelpool-count} ermittelt.
\begin{algorithm}[!ht]
\caption{\textproc{getEdgelCount}}
\label{alg:edgelpool-count}
\begin{algorithmic}[1]
	\Require $p$
	\If{$\lnot p$}
	\Cost{$c_{1}$}{$1$}
	\label{alg:edgelpool-count-validpointer-start}
		\State \textbf{return}
		\Cost{$c_{2}$}{$1$}
	\EndIf
	\label{alg:edgelpool-count-validpointer-end}
	\State \textbf{return} $\mathit{p.count}$
	\Cost{$c_{4}$}{$2$}
	\label{alg:edgelpool-count-counter}
\end{algorithmic}
\end{algorithm}

Als Parameter wird der Zeiger $p$ übergeben und in Zeile
 \ref{alg:edgelpool-count-validpointer-start}--\ref{alg:edgelpool-count-validpointer-end} überprüft. Die Anzahl der
 Einträge wird in Zeile \ref{alg:edgelpool-count-counter} über die Zählvariable $\mathit{p.count}$ ermittelt. Der
 Zugriff auf die Variable, und somit die Laufzeit des Algorithmus, erfolgt in konstanter Zeit. Die Laufzeitfunktion ist
 im schlechtesten Fall $T(n) = 3$.

% subsection datenstruktur-edgels (end)


\subsubsection{Liniensegmente} % (fold)
\label{sub:datenstruktur-liniensegmente}

Die Datenstruktur eines Liniensegments und die Methoden zum hinzufügen, löschen und freigeben des Speichers sind nach
 dem Vorbild des Edgelspeichers aufgebaut. Die Datenstruktur eines Liniensegments ist in
 \autoref{alg:datastructure-linesegment} definiert.
\begin{algorithm}[!ht]
\caption{\textproc{lineSegment}}
\label{alg:datastructure-linesegment}
	\begin{algorithmic}[1]
		\State $\mathit{start}$
		\State $\mathit{end}$
		\State $\mathit{slope}$
		\State $\mathit{supportCount}$
		\State $\mathit{remove}$
		\State $\mathit{startCorner}$
		\State $\mathit{endCorner}$
		\State $\mathit{support}[\mathit{MAXEDGELS}]$
	\end{algorithmic}
\end{algorithm}

Eine Linie besteht aus den \glspl{edgel} $\mathit{start}$ und $\mathit{end}$, die den Start- und Endpunkt der Linie
 darstellen. Die Variable $\mathit{slope}$ enthält die Orientierung des Liniensegments, während die Variable
 $\mathit{supportCount}$ die Anzahl der unterstützenden \glspl{edgel} der Linie speichert. $\mathit{remove}$,
 $\mathit{startCorner}$ und $\mathit{endCorner}$ sind boolesche Variablen. $\mathit{remove}$ dient im späteren Verlauf
 zur Erkennung, ob ein Liniensegment gelöscht werden muss. Wenn eine Linie einen Eckpunkt am Anfang oder am Ende
 besitzt, wird dies in den Variablen $\mathit{startCorner}$ und $\mathit{endCorner}$ festgehalten.Die letzte Variable
 $\mathit{support}$ dient zur Speicherung von \glspl{edgel}, die eine Linienhypothese unterstützen. Die Lese- und
 Schreibzugriffe auf die Datenstruktur ist konstant.

Mit \autoref{alg:linesegmentaddedgel} wird ein Unterstüzungsedgel zu einem Liniensegment hinzugefügt. Das Verfahren
 benötigt dazu das Liniensegment $l$, den \gls{edgel} $e$ und die Position $\mathit{pos}$, an die der \gls{edgel}
 gespeichert wird.
\begin{algorithm}[!ht]
\caption{\textproc{addEdgel}}
\label{alg:linesegmentaddedgel}
\begin{algorithmic}[1]
	\Require $l, e, \mathit{pos}$
	\If{$\mathit{pos} > \mathit{MAXEDGELS} - 1$}
	\label{alg:linesegmentaddedgel-hasvalidrange}
		\State \textbf{return}
		\label{alg:linesegmentaddedgel-notvalidrange}
	\EndIf
	\State $\mathit{l.support}[\mathit{pos}] \gets e$
	\label{alg:linesegmentaddedgel-storeedgel}
\end{algorithmic}
\end{algorithm}

In Zeile \ref{alg:linesegmentaddedgel-hasvalidrange} wird überprüft, ob genügend Speicherplatz für ein \gls{edgel} zur
 Verfügung steht. Falls dem nicht so ist, wird in Zeile \ref{alg:linesegmentaddedgel-notvalidrange} das Verfahren
 beendet. Andernfalls, wenn genügend Speicherplatz vorhanden ist, wird in Zeile
 \ref{alg:linesegmentaddedgel-storeedgel} der \gls{edgel} im Liniensegment gespeichert. Die Laufzeit des Verfahrens ist
 konstant.

Die Methode \textproc{isOrientationCompatible} untersucht, ob zwei Liniensegmente $\mathit{left}$ und $\mathit{right}$
 fast parallel zueinander stehen (\autoref{alg:linesegmentisorientationcompatible}).
\begin{algorithm}[ht]
\caption{\textproc{isOrientationCompatible}}
\label{alg:linesegmentisorientationcompatible}
\begin{algorithmic}[1]
	\Require $\mathit{left}, \mathit{right}$
	\State \textbf{return} \Call{dotProduct}{$\mathit{left.slope}, \mathit{right.slope}$} $> 0.92$
\end{algorithmic}
\end{algorithm}

Dazu wird mithilfe von \textproc{dotProduct} die Orientierung berechnet. Wenn die Orientierung im Bereich von
 $(0.92,1]$ liegt, wird als Ergebnis wahr zurückgeliefert. Das bedeutet, dass die Orientierung der Linien im Bereich
 von $0^\circ$ bis $\sim 23^\circ$ liegt und die Linien als parallel betrachtet werden. Ansonsten wird als Ergebnis
 falsch zurückgegeben, was bedeutet, dass die Linien nicht parallel sind. Die Laufzeit von
 \autoref{alg:linesegmentisorientationcompatible} ist konstant.

Mit \autoref{alg:isedgelnearline} wird der Abstand eines \gls{edgel} zu einem Liniensegment berechnet.
\begin{algorithm}[!ht]\small
\caption{\textproc{isEdgelNearLine}}
\label{alg:isedgelnearline}
\begin{algorithmic}[1]
	\Require $l,e$
	\If{$\lnot$ \Call{isCompatible}{$\mathit{l.start,e}$}}
	\Cost{$c_{1}$}{$1+8$}
	\label{alg:isedgelnearline-iscompatible}
		\State \textbf{return FALSE}
		\Cost{$c_{2}$}{$1$}
		\label{alg:isedgelnearline-notcompatible}
	\EndIf
	\State $a \gets \mathit{l.end.coordinate.x} - \mathit{l.start.coordinate.x}$
	\Cost{$c_{4}$}{$8$}
	\label{alg:isedgelnearline-distance-start}
	\State $b \gets \mathit{l.end.coordinate.y} - \mathit{l.start.coordinate.y}$
	\Cost{$c_{5}$}{$8$}
	\State $c \gets$ \Call{sqrt}{$(a \cdot a)+(b \cdot b)$}
	\Cost{$c_{6}$}{$3+1$}
	\label{alg:isedgelnearline-distance-end}
	\State $\mathit{AB1} \gets \mathit{l.end.coordinate.x} - \mathit{l.start.coordinate.x}$
	\Cost{$c_{7}$}{$8$}
	\label{alg:isedgelnearline-pointline-start}
	\State $\mathit{AC2} \gets \mathit{e.coordinate.y} - \mathit{l.start.coordinate.y}$
	\Cost{$c_{8}$}{$7$}
	\State $\mathit{AC1} \gets \mathit{e.coordinate.x} - \mathit{l.start.coordinate.x}$
	\Cost{$c_{9}$}{$7$}
	\State $\mathit{AB2} \gets \mathit{l.end.coordinate.y} - \mathit{l.start.coordinate.y}$
	\Cost{$c_{10}$}{$8$}
	\State $\mathit{crossproduct} \gets (\mathit{AB1} \cdot \mathit{AC2}) - (\mathit{AC1} \cdot \mathit{AB2})$
	\Cost{$c_{11}$}{$4$}
	\State $\mathit{distance} \gets$ \Call{ABS}{$\mathit{crossproduct}/c$}
	\Cost{$c_{12}$}{$3+2$}
	\label{alg:isedgelnearline-pointline-end}
	\State \textbf{return} $\mathit{distance} < 0.75$
	\Cost{$c_{13}$}{$2$}
	\label{alg:isedgelnearline-return}
\end{algorithmic}
\end{algorithm}

Das Verfahren benötigt dazu ein Liniensegment $l$ und ein \gls{edgel} $e$. In Zeile
 \ref{alg:isedgelnearline-iscompatible} wird überprüft, ob die Orientierung des \gls{edgel} kompatibel mit der
 Orientierung des Liniensegments ist. Wenn dies der Fall ist, wird das Verfahren fortgesetz. Andrenfalls wird das
 Verfahren in Zeile \ref{alg:isedgelnearline-notcompatible} abgebrochen. In Zeile
 \ref{alg:isedgelnearline-distance-start}--\ref{alg:isedgelnearline-distance-end} wird die Länge des Abstands der
 Endpunkte der Linie berechnet und in lokalen Variablen gespeichert. Im Anschluß wird in Zeile
 \ref{alg:isedgelnearline-pointline-start}--\ref{alg:isedgelnearline-pointline-end} der Abstand des \gls{edgel} zur
 Linie berechnet. Das Präprozessor Makro \textproc{ABS} berechnet den absoluten Betrag der Distanz in konstanter Zeit
 (Vgl. \autoref{alg:abs}).
\begin{algorithm}[!ht]\small
\caption{\textproc{ABS}}
\label{alg:abs}
\begin{algorithmic}[1]
	\Require $a$
	\If{$a < 0$}
	\Cost{$c_{1}$}{$1$}
		\State \textbf{return} $-a$
		\Cost{$c_{2}$}{$1$}
	\Else
		\State \textbf{return} $a$
		\Cost{$c_{4}$}{$1$}
	\EndIf
\end{algorithmic}
\end{algorithm}

In Zeile \ref{alg:isedgelnearline-return} wird als Rückgabewert, abhängig vom Vergleich der Distanz, wahr oder falsch
 zurückgegebn. Bleibt der Abstand des \gls{edgel} zur Linie unter $0.75$ wird wahr and die aufrufende Methode
 zurückgeben. Ansonsten, wenn der Abstand größer ist, wird falsch zurückgegeben. Die Laufzeit von
 \autoref{alg:isedgelnearline} ist konstant.

Mit \textproc{intersection} wird der Schnittpunkt zweier Linien berechnet. Dazu benötigt das Verfahren in
 \autoref{alg:linesegmenintersection} eine linke und eine rechte Linie.
\begin{algorithm}[!ht]
\caption{\textproc{intersection}}
\label{alg:linesegmenintersection}
\begin{algorithmic}[1]
	\Require $\mathit{left}, \mathit{right}$
	\State $\mathit{intersection} \gets \infty$
	\Cost{$c_{1}$}{$1$}
	\State $\mathit{x1} \gets \mathit{left.start.coordinate.x}$
	\Cost{$c_{2}$}{$4$}
	\label{alg:linesegmenintersection-var-start}
	\State $\mathit{y1} \gets \mathit{left.start.coordinate.y}$
	\Cost{$c_{3}$}{$4$}
	\State $\mathit{x2} \gets \mathit{left.end.coordinate.x}$
	\Cost{$c_{4}$}{$4$}
	\State $\mathit{y2} \gets \mathit{left.end.coordinate.y}$
	\Cost{$c_{5}$}{$4$}
	\State $\mathit{x3} \gets \mathit{right.start.coordinate.x}$
	\Cost{$c_{6}$}{$4$}
	\State $\mathit{y3} \gets \mathit{right.start.coordinate.y}$
	\Cost{$c_{7}$}{$4$}
	\State $\mathit{x4} \gets \mathit{right.end.coordinate.x}$
	\Cost{$c_{8}$}{$4$}
	\State $\mathit{y4} \gets \mathit{right.end.coordinate.y}$
	\Cost{$c_{9}$}{$4$}
	\label{alg:linesegmenintersection-var-end}
	\State $\mathit{numerator} \gets \bigl((\mathit{x4} - \mathit{x3}) \cdot (\mathit{y1} - \mathit{y3})\bigr)
	 - \bigl((\mathit{y4} - \mathit{y3}) \cdot (\mathit{x1} - \mathit{x3})\bigr)$
	\Cost{$c_{10}$}{$8$}
	\label{alg:linesegmenintersection-intersect-start}
	\State $\mathit{denumerator} \gets \bigl((\mathit{y4} - \mathit{y3}) \cdot (\mathit{x2} - \mathit{x1})\bigr)
	 - \bigl((\mathit{x4} - \mathit{x3}) \cdot (\mathit{y2} - \mathit{y1})\bigr)$
	\Cost{$c_{11}$}{$8$}
	\State $\mathit{u\_a} \gets \mathit{numerator} / \mathit{denumerator}$
	\Cost{$c_{12}$}{$2$}
	\State $\mathit{intersection.x} \gets \mathit{x1} + \mathit{u\_a} \cdot (\mathit{x2} - \mathit{x1})$
	\Cost{$c_{13}$}{$5$}
	\State $\mathit{intersection.y} \gets \mathit{y1} + \mathit{u\_a} \cdot (\mathit{y2} - \mathit{y1})$
	\Cost{$c_{14}$}{$5$}
	\label{alg:linesegmenintersection-intersect-end}
	\State \textbf{return} $\mathit{intersection}$
	\Cost{$c_{15}$}{$1$}
	\label{alg:linesegmenintersection-return}
\end{algorithmic}
\end{algorithm}

 In Zeile \ref{alg:linesegmenintersection-var-start}--\ref{alg:linesegmenintersection-var-end} werden die Punkte der
 Linienkoordinaten in lokalen Variablen gespeichert. Danach wird in Zeile
 \ref{alg:linesegmenintersection-intersect-start}--\ref{alg:linesegmenintersection-intersect-end} der Schnittpunkt der
 beiden Linie berechnet und in Zeile \ref{alg:linesegmenintersection-return} an die aufrufende Methode zurückgegeben.
 Die Berechnung des Schnittpunktes zweier Linien erfolgt in konstanter Zeit.

Die Datenstruktur eines Speichervorrats für Linien in \autoref{alg:datastructure-linesegmentpool} besteht aus einem
 Array $\mathit{data}$ mit der festen Größe $N$ und einer Zählvariablen $\mathit{count}$.
\begin{algorithm}[!ht]\small
\caption{\textproc{lineSegmentPool} (Datenstruktur)}
\label{alg:datastructure-linesegmentpool}
\begin{algorithmic}[1]
	\State $\mathit{data}[N]$
	\Comment Anzahl der Einträge
	\State $\mathit{count}$
\end{algorithmic}
\end{algorithm}

Der Speichervorrat für Linien in \autoref{alg:datastructure-linesegmentpoolimplementation} besteht wiederum aus einem
 Array $\mathit{data}$ mit der Anzahl $S$ der zur Verfügung stehenden Speicherblöcke.
\begin{algorithm}[ht]
\caption{\textproc{lineSegmentPool} (Implementierung)}
\label{alg:datastructure-linesegmentpoolimplementation}
\begin{algorithmic}[1]
	\State $\mathit{data}[S]$
	\Comment Anzahl der Pools
	\State $\mathit{pool}$
\end{algorithmic}
\end{algorithm}

Der Zeiger von $\mathit{data}$ wird in der Variablen $\mathit{pool}$ gespeichert. Der Zugriff auf die Datenstruktur
 erfolgt in konstanter Zeit.

Mehrere Speicherblöcke können mit \autoref{alg:linepool-getmemorypools} angefordert werden und mit
\begin{algorithm}[!ht]
\caption{\textproc{getMemoryPools}}
\label{alg:linepool-getmemorypools}
\begin{algorithmic}[1]
	\Require $n$
	\If{$\mathit{data} + S - \mathit{pool} \geq n$}
	\Cost{$c_{1}$}{$3$}
	\label{alg:linepool-getmemorypools-checkpoolsize}
		\State $\mathit{pool} \gets \mathit{pool} + n$
		\Cost{$c_{2}$}{$2$}
		\State \textbf{return} $\mathit{pool} - n$
		\Cost{$c_{3}$}{$2$}
	\Else
		\State \textbf{return} $\mathit{NULL}$
		\Cost{$c_{5}$}{$1$}
	\EndIf
\end{algorithmic}
\end{algorithm}

 \autoref{alg:linepool-getmemorypool} wird ein Speicherblock angefordert.
\input{alg/analyse-hirzer/datastructure-linesegmentpool-getmemorypool}
Der Aufbau der Verfahren entspricht den Verfahren des Speichervorrats für \glspl{edgel}
 (Vgl. \autoref{alg:edgelpool-getmemorypools} und \autoref{alg:edgelpool-getmemorypool}). Der Zugriff erfolgt in
 konstanter Zeit.

Um eine Linie zu einem Speicherblock hinzuzufügen, wird \autoref{alg:linesegmentpool-addline} verwendet.
\begin{algorithm}[ht]
\caption{\textproc{addLineSegment}}
\label{alg:linesegmentpool-addline}
\begin{algorithmic}[1]
	\Require $p,l$
	\If{$\lnot p$}
	\label{alg:linesegmentpool-addline-validpointer-start}
		\State \textbf{return}
	\EndIf
	\label{alg:linesegmentpool-addline-validpointer-end}
	\If{$\lnot \left(p(\mathit{count}) < N\right)$}
	\label{alg:linepool-addline-checkspace-start}
		\State \textbf{return} \Comment Speicher voll
	\EndIf
	\label{alg:linesegmentpool-addline-checkspace-end}
	\State $c \gets p(\mathit{count})$
	\label{alg:linesegmentpool-addline-add-start}
	\State $p(\mathit{data}[c]) \gets l$
	\State $p(\mathit{count}) \gets c + 1$
	\label{alg:linesegmentpool-addline-add-end}
\end{algorithmic}
\end{algorithm}

Es wird ein Zeiger $p$ auf den Speicherblock, sowie eine Linie $l$ übergeben. Wenn es sich um einen gültigen Zeiger $p$
 handelt, und genügend freier Speicherplatz für eine weitere Linie vorhanden ist, wird in Zeile
 \ref{alg:linesegmentpool-addline-add-start}--\ref{alg:linesegmentpool-addline-add-end} die Linie hinzugefügt und die
 Zählvariable inkrementiert. Das Hinzufügen einer Linie ist konstant.

Zum auslesen einer Linie aus dem Speicherblock, wird \autoref{alg:linepool-getline} verwendet.
\begin{algorithm}[!ht]
\caption{\textproc{getLineSegment}}
\label{alg:linepool-getline}
\begin{algorithmic}[1]
	\Require $p,i$
	\If{$\lnot p$}
	\label{alg:linepool-getline-validpointer-start}
		\State \textbf{return}
	\EndIf
	\label{alg:linepool-getline-validpointer-end}
	\State $c \gets \mathit{p.count}$
	\If{$\lnot \left(c > i\right)$}
	\label{alg:linepool-getline-validrange-start}
		\State \textbf{return}
	\EndIf
	\label{alg:linepool-getline-validrange-end}
	\State \textbf{return} $\mathit{p.data}[i]$
	\label{alg:linepool-getline-returnline}
\end{algorithmic}
\end{algorithm}

Als Parameter werden ein Zeiger $p$ und ein Index $i$ übergeben. Der Index gibt an, welche Linie aus dem Block
 ausgelesen werden soll. In Zeile \ref{alg:linepool-getline-validrange-start} wird geprüft, ob der Index sich innerhalb
 der Grenzen der gespeicherten Linien befindet. Wenn dies der Fall ist, wird in Zeile
 \ref{alg:linepool-getline-returnline} die Linie in konstanter Zeit zurückgegeben.

Mit \autoref{alg:linepool-resetmemorypool} werden die Einträge im Speicherblock gelöscht.
\begin{algorithm}[!ht]\small
\caption{\textproc{resetMemoryPool}}
\label{alg:linepool-resetmemorypool}
\begin{algorithmic}[1]
	\Require $p$
	\If{$\lnot p$}
	\Cost{$c_{1}$}{$1$}
	\label{alg:linepool-resetmemorypool-validpointer-start}
		\State \textbf{return}
		\Cost{$c_{2}$}{$1$}
	\EndIf
	\label{alg:linepool-resetmemorypool-validpointer-end}
	\State $\mathit{p.count} \gets 0$
	\Cost{$c_{4}$}{$2$}
	\label{alg:linepool-resetmemorypool-reset}
\end{algorithmic}
\end{algorithm}

Dazu wird der Zeiger $p$ auf den Speicherblock übergeben und in Zeile
 \ref{alg:linepool-resetmemorypool-validpointer-start}--\ref{alg:linepool-resetmemorypool-validpointer-end} überprüft.
 Wenn es sich um einen gültigen Zeiger handelt, wird die Zählvariable auf $0$ gesetzt. Da es sich um einen direkten
 Zugriff handelt, erfolgt das Löschen in konstanter Zeit.

Durch \autoref{alg:linepool-freememorypool} kann ein Speicherblock wieder freigegeben werden.
\begin{algorithm}[!ht]\small
\caption{\textproc{freeMemoryPool}}
\label{alg:linepool-freememorypool}
\begin{algorithmic}[1]
	\Require $p$
	\If{$\lnot p$}
	\Cost{$c_{1}$}{$1$}
	\label{alg:linepool-freememorypool-validpointer-start}
		\State \textbf{return}
		\Cost{$c_{2}$}{$1$}
	\EndIf
	\label{alg:linepool-freememorypool-validpointer-end}
	\State \Call{resetmemorypool}{$p$}
	\Cost{$c_{4}$}{$3$}
	\label{alg:linepool-freememorypool-resetmemory}
	\If{$p \geq \mathit{data} \land p \leq \mathit{data} + S$}
	\Cost{$c_{5}$}{$4$}
	\label{alg:linepool-freememorypool-checkpointer}
		\State $\mathit{pool} \gets p$
		\Cost{$c_{6}$}{$1$}
		\label{alg:linepool-freememorypool-savepointer}
	\EndIf
\end{algorithmic}
\end{algorithm}

Dazu wird der Zeiger $p$ auf Gültigkeit geprüft. Danach wird der Speicher durch \textproc{resetMemoryPool}
 (\autoref{alg:linepool-resetmemorypool}) gelöscht. In Zeile \ref{alg:linepool-freememorypool-checkpointer} wird
 überprüft, ob der Zeiger $p$ zu dem entsprechenden Block gehört, um danach die Adresse in Zeile
 \ref{alg:linepool-freememorypool-savepointer} in $\mathit{pool}$ zu speichern. Auch hier erfolgt das Freigeben des
 Speichers wieder in konstanter Zeit.

Die Anzahl der Einträge in einem Pool werden durch \autoref{alg:linepool-count} bestimmt, indem die Zählvariable
 $\mathit{count}$ zurückgegeben wird.
\begin{algorithm}[ht]
\caption{\textproc{getLineCount}}
\label{alg:linepool-count}
\begin{algorithmic}[1]
	\Require $p$
	\If{$\lnot p$}
	\label{alg:linepool-count-validpointer-start}
		\State \textbf{return}
	\EndIf
	\label{alg:linepool-count-validpointer-end}
	\State \textbf{return} $p(\mathit{count})$
	\label{alg:linepool-count-counter}
\end{algorithmic}
\end{algorithm}

Der Zugriff auf die Variable erfolgt in konstanter Zeit.

Im Verfahren nach \citeauthor{clarke96} werden Liniensegmente nicht aus dem Speicherpool gelöscht. Darum kann auf
 einen Algorithmus zum löschen der Einträge, wie \autoref{alg:edgelpool-removeedgel} bei \glspl{edgel}, verzichtet
 werden. Alle Operationen für Linien erfolgen somit in konstanter Zeit $T(n) = \Theta(1)$.

% subsection datenstruktur-liniensegmente (end)


\subsubsection{Distanz} % (fold)
\label{sub:distanz}

Die Datenstruktur \textproc{distance} dient zur Speicherung der Länge von Liniensegmenten und der Speicherung eines
 Index, um eine Linie in einem Speicherblock anzusprechen (\autoref{alg:distance}).
\begin{algorithm}[ht]
\caption{\textproc{distance}}
\label{alg:distance}
	\begin{algorithmic}[1]
		\State $\mathit{distance}$
		\State $\mathit{index}$
	\end{algorithmic}
\end{algorithm}


Wie bei \gls{edgel} oder Liniensegmenten, kann \textproc{distance} in einem Speicherbereich hinterlegt werden und ist
 in \autoref{alg:datastructure-distancepool} dargestellt.
\begin{algorithm}[ht]
\caption{\textproc{distancePool} (Datenstruktur)}
\label{alg:datastructure-distancepool}
\begin{algorithmic}[1]
	\State $\mathit{data}[N]$
	\Comment Anzahl der Einträge
	\State $\mathit{count}$
\end{algorithmic}
\end{algorithm}

Auch bei diesem Verfahren wird ein Array mit festert Größe $N$ und einer Zählvariable verwendet. Im Gegensatz zu den
 anderen Speicherbereichen wird bei \textproc{distance} auf einen Speichervorrat verzichtet. Die Verwaltung eines
 Speichervorrat wird manuell durchgeführt.

% Mit \autoref{alg:distancepool-adddistance}
% \begin{algorithm}[!ht]
\caption{\textproc{addDistance}}
\label{alg:distancepool-adddistance}
\begin{algorithmic}[1]
	\Require $p,d$
	\If{$\lnot p$}
		\State \textbf{return}
	\EndIf
	\If{$\lnot \left(\mathit{p.count} < N\right)$}
		\State \textbf{return} \Comment Speicher voll
	\EndIf
	\State $c \gets \mathit{p.count}$
	\State $\mathit{p.data}[c] \gets d$
	\State $\mathit{p.count} \gets c + 1$
\end{algorithmic}
\end{algorithm}

%  und \autoref{alg:distancepool-removedistance}
% \begin{algorithm}[!ht]
\caption{\textproc{removeLine}}
\label{alg:distancepool-removeline}
\begin{algorithmic}[1]
	\Require $p,i$
	\If{$\lnot p$}
		\State \textbf{return}
	\EndIf
	\State $c \gets \mathit{p.count}$
	\If{$i < 0 \lor i > c$}
	\label{alg:distancepool-isvalid-start}
		\State \textbf{return}
	\EndIf
	\label{alg:distancepool-isvalid-end}
	\If{$c > i + 1$}
		\State \Call{memmove}{$\mathit{p.data}[\mathit{i}], \mathit{p.data}[\mathit{i} + 1], \left(c - \mathit{i}
		 + 1\right) \cdot \textproc{sizeof}(d)$}
	\EndIf
\end{algorithmic}
\end{algorithm}

%  können Distanzwerte dem Speicherbereich hinzugefügt und entfernt werden. Die Laufzeit zum hinzufügen ist dabei
%  konstant. Bei \autoref{alg:distancepool-removedistance} ist im besten Fall der Index $i$ ausserhalb des Bereichs der
%  gespeicherten Daten (Zeile \ref{alg:distancepool-isvalid-start}--\ref{alg:distancepool-isvalid-end}), und die Laufzeit
%  somit $T_{best}=\Theta(1)$. Im schlechtesten Fall müssen mit \textproc{memmove} $n-1$ Einträge an Position $0$
%  verschoben werden. Die Laufzeit beträgt dann
%  $T_{worst}=\Theta(n)$ (Vgl. \autoref{sub:datenstruktur-edgels}, S. \pageref{sub:datenstruktur-edgels-memmove}).

Mit \autoref{alg:distancepool-adddistance}
\begin{algorithm}[!ht]
\caption{\textproc{addDistance}}
\label{alg:distancepool-adddistance}
\begin{algorithmic}[1]
	\Require $p,d$
	\If{$\lnot p$}
		\State \textbf{return}
	\EndIf
	\If{$\lnot \left(\mathit{p.count} < N\right)$}
		\State \textbf{return} \Comment Speicher voll
	\EndIf
	\State $c \gets \mathit{p.count}$
	\State $\mathit{p.data}[c] \gets d$
	\State $\mathit{p.count} \gets c + 1$
\end{algorithmic}
\end{algorithm}

können Distanzwerte dem Speicherbereich hinzugefügt werden. Die Laufzeit zum hinzufügen ist dabei konstant und die
 Laufzeitfunktion ist im schlechtesten Fall $T(n) = 12$.

Durch \autoref{alg:distancepool-distancecount}
\begin{algorithm}[!ht]
\caption{\textproc{getDistanceCount}}
\label{alg:distancepool-distancecount}
\begin{algorithmic}[1]
	\Require $p$
	\If{$\lnot p$}
		\State \textbf{return}
	\EndIf
	\State \textbf{return} $\mathit{p.count}$
\end{algorithmic}
\end{algorithm}

 kann die Anzahl der gespeicherten Distanzwerte in konstanter Zeit ausgelesen werden. Die Laufzeitfunktion ist im schlechtesten Fall $T(n) = 3$.

\textproc{freePool} (\autoref{alg:distancepool-freepool}) löscht den Speicherbereich in konstanter Zeit.
\begin{algorithm}[!ht]
\caption{\textproc{freePool}}
\label{alg:distancepool-freepool}
\begin{algorithmic}[1]
	\Require $p$
	\If{$\lnot p$}
	\Cost{$c_{1}$}{$1$}
		\State \textbf{return}
		\Cost{$c_{2}$}{$1$}
	\EndIf
	\State $\mathit{p.count} \gets 0$
	\Cost{$c_{4}$}{$2$}
\end{algorithmic}
\end{algorithm}

Die Laufzeitfunktion ist im schlechsten Fall ebenfalls $T(n) = 3$.

Die Laufzeit der Methoden für \textproc{distance} sind alle konstant und somit $T(n)=\Theta(1)$.

% subsubsection distanz (end)


\subsubsection{Marken} % (fold)
\label{sub:marken}

Die Datenstruktur \textproc{marker}, der Aufbau des Speicherblocks und die Operationen zum hinzufügen und freigeben von
 Daten ist, wie bei der Datenstruktur der Liniensegmente auch, nach dem Vorbild des Edgelspeichers aufgebaut. Die
 Datenstruktur in \autoref{alg:marker} verwendet vier Variablen vom Typ \textproc{vector}, um die Eckpunkte einer Marke
 zu speichern.
\begin{algorithm}[!ht]
\caption{\textproc{marker}}
\label{alg:marker}
	\begin{algorithmic}[1]
		\State $c1$
		\State $c2$
		\State $c3$
		\State $c4$
	\end{algorithmic}
\end{algorithm}


Die Datenstruktur des Speichervorrats (\autoref{alg:datastructure-markerpool})
\begin{algorithm}[!ht]\small
\caption{\textproc{markerPool} (Datenstruktur)}
\label{alg:datastructure-markerpool}
\begin{algorithmic}[1]
	\State $\mathit{data}[N]$
	\Comment Anzahl der Einträge
	\State $\mathit{count}$
\end{algorithmic}
\end{algorithm}

und des Speicherblocks (\autoref{alg:datastructure-markerpoolimpl}) verwenden eine festgelegte Anzahl von Einträgen zum
\begin{algorithm}[!ht]
\caption{\textproc{markerPool} (Speichervorrat)}
\label{alg:datastructure-markerpoolimpl}
\begin{algorithmic}[1]
	\State $\mathit{data}[S]$
	\Comment Anzahl der Pools
	\State $\mathit{pool}$
\end{algorithmic}
\end{algorithm}

speichern der Einträge.

Mit \textproc{getMemoryPools} (\autoref{alg:markerpool-getmemorypools})
\begin{algorithm}[!ht]
\caption{\textproc{getMemoryPools}}
\label{alg:markerpool-getmemorypools}
\begin{algorithmic}[1]
	\Require $n$
	\If{$\mathit{data} + S - \mathit{pool} \geq n$}
	\Cost{$c_{1}$}{$3$}
		\State $\mathit{pool} \gets \mathit{pool} + n$
		\Cost{$c_{2}$}{$2$}
		\State \textbf{return} $\mathit{pool} - n$
		\Cost{$c_{3}$}{$2$}
	\Else
		\State \textbf{return} $\mathit{NULL}$
		\Cost{$c_{5}$}{$1$}
	\EndIf
\end{algorithmic}
\end{algorithm}

und \textproc{getMemoryPool} (\autoref{alg:markerpool-getmemorypool})
\begin{algorithm}[!ht]
\caption{\textproc{getMemoryPool}}
\label{alg:markerpool-getmemorypool}
\begin{algorithmic}[1]
	\State $p \gets$ \Call{getmemorypools}{1}
	\Cost{$c_{1}$}{$1 + 7$}
	\State \textbf{return} $p$
	\Cost{$c_{2}$}{$1$}
\end{algorithmic}
\end{algorithm}

werden Speicherblöcke in konstanter Zeit angefordert.

Um die Einträge in einem Speicherblock zu löschen, wird \autoref{alg:markerpool-resetmemorypool} verwendet. Die
 Laufzeit ist dabei konstant.
\begin{algorithm}[!ht]\small
\caption{\textproc{resetMemoryPool}}
\label{alg:markerpool-resetmemorypool}
\begin{algorithmic}[1]
	\Require $p$
	\If{$\lnot p$}
	\Cost{$c_{1}$}{$1$}
		\State \textbf{return}
		\Cost{$c_{2}$}{$1$}
	\EndIf
	\State $\mathit{p.count} \gets 0$
	\Cost{$c_{4}$}{$2$}
\end{algorithmic}
\end{algorithm}


Mit \textproc{freeMemoryPool} (\autoref{alg:markerpool-freememorypool}) kann der Speicherblock an den Vorrat
 zurückgegeben werden. Auch hier ist die Laufzeit konstant.
\begin{algorithm}[!ht]\small
\caption{\textproc{freeMemoryPool}}
\label{alg:markerpool-freememorypool}
\begin{algorithmic}[1]
	\Require $p$
	\If{$\lnot p$}
	\Cost{$c_{1}$}{$1$}
		\State \textbf{return}
		\Cost{$c_{2}$}{$1$}
	\EndIf
	\State \Call{resetMemoryPool}{$p$}
	\Cost{$c_{4}$}{$3$}
	\If{$p \geq \mathit{data} \land p \leq \mathit{data} + S$}
	\Cost{$c_{5}$}{$4$}
		\State $\mathit{pool} \gets p$
		\Cost{$c_{6}$}{$1$}
	\EndIf
\end{algorithmic}
\end{algorithm}


Die Anzahl der gespeicherten Marken in einem Speicherblock können mit \autoref{alg:markerpool-count} ermittelt werden.
 Durch den direkten Zugriff auf die Zählvariable ist die Laufzeit konstant.
\begin{algorithm}[!ht]
\caption{\textproc{getMarkerCount}}
\label{alg:markerpool-count}
\begin{algorithmic}[1]
	\Require $p$
	\If{$\lnot p$}
		\State \textbf{return}
	\EndIf
	\State \textbf{return} $p(\mathit{count})$
\end{algorithmic}
\end{algorithm}


Um eine Marke zu einem Speicherblock hinzuzufügen, wird \autoref{alg:markerpool-addmarker} verwendet. Auch hier wird durch den direkten Zugriff auf die Variablen eine konstante Laufzeit erreicht.
\begin{algorithm}[!ht]
\caption{\textproc{addMarker}}
\label{alg:markerpool-addmarker}
\begin{algorithmic}[1]
	\Require $p,m$
	\If{$\lnot p$}
		\State \textbf{return}
	\EndIf
	\If{$\lnot \left(\mathit{p.count} < N\right)$}
		\State \textbf{return} \Comment Speicher voll
	\EndIf
	\State $c \gets \mathit{p.count}$
	\State $\mathit{p.data}[c] \gets m$
	\State $\mathit{p.count} \gets c + 1$
\end{algorithmic}
\end{algorithm}


Alle Methoden der Datenstruktur \textproc{marker} haben eine konstante Laufzeit $T(n)=\Theta(1)$.
% subsubsection marken (end)


% subsection datenstrukturen (end)

\subsection{Linienerkennung nach \texorpdfstring{\citeauthor{clarke96}}{Clarke, Carlsson und Zisserman}} % (fold)
\label{sub:linienerkennung_nach_clarke96}
\citeauthor{clarke96} verwenden in ihrem Verfahren ein monochromes Bildsignal $I_m$\footcite[Vgl.][S.~417]{clarke96}.
 Die Konvertierung des Bildsignals $I$ von YCbCr in $I_m$ erfolgt durch \autoref{alg:convertmonochrome}. Wie in
 \autoref{sub:farbräume} beschrieben, besteht ein YCbCr Signal aus einem Luminanz Kanal $Y$ und den Chroma Abweichungen
 $Cb$ und $Cr$. Um ein monochromes Signal $I_m$ zu erstellen, muss der Luminanz Kanal ausgelesen und in einen Puffer
 kopiert werden.
\begin{algorithm}[!ht]\small
\caption{Konvertierung zu monochromen Bildsignal}
\label{alg:convertmonochrome}
	\begin{algorithmic}[1]
		\Require $I, I_m$
		\State $Y \gets$ \Call{baseAddress}{$I$}
		\label{alg:convertmonochrome-baseaddress}
		\State $w \gets$ \Call{width}{$I$}
		\State $h \gets$ \Call{height}{$I$}
		\State $l \gets w \cdot h$
		\State $I_m \gets$ \Call{copy}{$I, Y, l$}
	\end{algorithmic}
\end{algorithm}

Der Algorithmus verwendet als Parameter das Bildsignal $I$ und einen Zeiger $I_m$ auf einen Puffer für das monochrome
 Signal. Der Monochrompuffer $I_m$ ist ein Array mit fester Größe, das beim initialisieren einmalig angelegt wird und
 danach wiederverwendet werden kann. In Zeile~\ref{alg:convertmonochrome-baseaddress} wird die Adresse des
 Luminanz-Kanals $Y$ ausgelesen. Die Funktionen \textproc{width} und \textproc{height} liefern die Breite und Höhe des
 Signals in Pixeln, mit denen die Länge der Daten berechnet wird. Anschließend werden die Daten in den Puffer kopiert.
 Die Konvertierung des Bildsignals ist Verarbeitungsschritt der vor dem Verfahren von \citeauthor{clarke96} durchgeführt
 wird und wird nur der Vollständigkeit erwähnt.

Um auf \gls{pixel} zugreifen zu können, wird \autoref{alg:getpixel} verwendet.
\begin{algorithm}[!ht]\small
\caption{\textproc{getPixel}}
\label{alg:getpixel}
	\begin{algorithmic}[1]
		\Require $I_m, x, y, w, h$
		\If{$x < 0$}
		\Cost{$c_{1}$}{$1$}
		\label{alg:getpixel-startcheck}
			\State $x \gets 0$
			\Cost{$c_{2}$}{$1$}
		\EndIf
		\If{$y < 0$}
		\Cost{$c_{4}$}{$1$}
			\State $y \gets 0$
			\Cost{$c_{5}$}{$1$}
		\EndIf
		\If{$x \geq w$}
		\Cost{$c_{7}$}{$1$}
			\State $x \gets w - 1$
			\Cost{$c_{8}$}{$2$}
		\EndIf
		\If{$y \geq h$}
		\Cost{$c_{10}$}{$1$}
			\State $y \gets h -1$
			\Cost{$c_{11}$}{$2$}
		\EndIf
		\label{alg:getpixel-stopcheck}
		\State $i \gets x + \left(y \cdot w\right)$
		\Cost{$c_{13}$}{$3$}
		\State \textbf{return} $I_m[i]$
		\Cost{$c_{14}$}{$2$}
	\end{algorithmic}
\end{algorithm}

Es wird der Puffer $I_m$ als Zeiger übergeben und die Position $x$ und $y$ des gewünschten \gls{pixel}. $w$ und $h$
 entsprechen der Breite und Höhe von $I_m$. Zeile~\ref{alg:getpixel-startcheck} bis Zeile~\ref{alg:getpixel-stopcheck}
 sorgen dafür, dass keine Werte außerhalb des Puffers gelesen werden können. Dies ist für die Randbehandlung bei
 Faltungsoperationen (Vgl. \autoref{sub:filter}) wichtig und wiederholt den \gls{pixel}.Die Laufzeit von
 \autoref{alg:getpixel} ist konstant und somit $T(n)=\Theta(1)$.

Der Algorithmus von \citeauthor{clarke96} ist in \autoref{alg:linedetection-clarke} aufgeführt und benötigt das
 monochrome Bildsignal $I_m$.
\begin{algorithm}[!ht]\small
\caption{\textproc{lineDetection}}
\label{alg:linedetection-clarke}
	\begin{algorithmic}[1]
		\Require $I_m$
		\For{$y \gets 0$ \textbf{to} $y < \mathit{imageHeight}$}
		\Cost{$c_{1}$}{$\frac{h}{r} + 1$}
		\label{alg:linedetection-clarke-start}
			\For{$x \gets 0$ \textbf{to} $x < \mathit{imageWidth}$}
			\Cost{$c_{2}$}{$\frac{h}{r}(\frac{w}{r} + 1)$}
				\State \Call{findEdgels}{$I_m,E,x,y$}
				\Cost{$c_{3}$}{$\frac{h \cdot w}{r^2} t_{1}$}
				\label{alg:linedetection-clarke-call-start}
				\State \Call{findLineSegments}{$E,L$}
				\Cost{$c_{4}$}{$\frac{h \cdot w}{r^2} t_{2}$}
				\label{alg:linedetection-clarke-call-end}
				\State \ldots \Comment{Speichern der Liniensegmente zur weiteren Verarbeitung}
				\State \Call{resetMemoryPool}{$E$}
				\Cost{$c_{6}$}{$\frac{h \cdot w}{r^2} \Theta(1)$}
				\label{alg:linedetection-clarke-reset-start}
				\State \Call{resetMemoryPool}{$L$}
				\Cost{$c_{7}$}{$\frac{h \cdot w}{r^2} \Theta(1)$}
				\label{alg:linedetection-clarke-reset-end}
				\State $ x \gets x + \mathit{regionSize}$
				\Cost{$c_{8}$}{$\frac{h \cdot w}{r^2} 2$}
			\EndFor
			\State $y \gets y + \mathit{regionSize}$
			\Cost{$c_{10}$}{$\frac{h}{r} 2$}
		\EndFor
		\label{alg:linedetection-clarke-end}
	\end{algorithmic}
\end{algorithm}

In der doppelten Schleife in Zeile \ref{alg:linedetection-clarke-start} bis \ref{alg:linedetection-clarke-end} wird
 $I_m$ in Regionen der Größe $40 \times 40$ \gls{pixel} unterteilt. Die globalen Variablen $\mathit{imageWidth}$ und
 $\mathit{imageHeight}$ enthalten die Breite und Höhe des Bildsignals $I_m$. Die Regionengröße von $40$ \gls{pixel} ist
 in der globalen Variable $\mathit{regionSize}$ gespeichert. In \citeauthor{clarke96}\footcite{clarke96} sind keine
 Hintergrundinformationen zu der Größe einer Region angegeben. Betrachtet man $640 \bmod 40 = 0$ und $480 \bmod 40 = 0$
 ist ersichtlich, dass die Größe der Region und der Aufteilung des Bildsignals in Zusammenhang steht. In Zeile
 \ref{alg:linedetection-clarke-call-start}--\ref{alg:linedetection-clarke-call-end} werden zuerst \glspl{edgel}
 ermittelt, um im Anschluss daraus Liniensegmente zu erstellen. Wenn für eine Region Liniensegmente erstellt und
 gespeichert wurden, wird der Speicherblock der \gls{edgel} und Liniensegmente in Zeile
 \ref{alg:linedetection-clarke-reset-start}--\ref{alg:linedetection-clarke-reset-end} gelöscht.

% \begin{subequations}
% \begin{align}
% \label{eq:linedetection-analyze1}
% T(I)& =
% c_1
% + c_2
% + c_3 \left(\frac{h}{40} + 1\right)
% + c_4 \sum \limits_{y = 0}^{\frac{h}{40}} t_y \left(\frac{w}{40} + 1 \right)
% + c_5 \sum \limits_{y = 0}^{\frac{h}{40}} \sum \limits_{x = 0}^{\frac{w}{40}} t_y t_x\\
% & \quad + c_6 \sum \limits_{y = 0}^{\frac{h}{40}} \sum \limits_{x = 0}^{\frac{w}{40}} t_y t_x
% + c_7 \sum \limits_{y = 0}^{\frac{h}{40}} \sum \limits_{x = 0}^{\frac{w}{40}} t_y t_x
% + c_9 \sum \limits_{y = 0}^{\frac{h}{40}} t_y \nonumber \\
% \label{eq:linedetection-analyze2}
% T(I)& =
% c_1
% + c_2
% + c_3 \left(n + 1\right)
% + c_4 \sum \limits_{y = 0}^{n} t_y \left(k + 1 \right)
% + c_5 \sum \limits_{y = 0}^{n} \sum \limits_{x = 0}^{k} t_y t_x\\
% & \quad + c_6 \sum \limits_{y = 0}^{n} \sum \limits_{x = 0}^{k} t_y t_x
% + c_7 \sum \limits_{y = 0}^{n} \sum \limits_{x = 0}^{k} t_y t_x
% + c_9 \sum \limits_{y = 0}^{n} t_y \nonumber \\
% \label{eq:linedetection-analyze3}
% T(I)& =
% c_1
% + c_2
% + c_3 \left(n + 1\right)
% + c_4 \left[n \left(k + 1 \right)\right]
% + c_5 n k
% + c_6 n k
% + c_7 n k
% + c_9 n\\
% \label{eq:linedetection-analyze4}
% T(I)& = c_1 + c_2 + c_3 + \left(c_3 + c_4 + c_9\right) n + \left(c_4 + c_5 + c_6 + c_7\right) n k\\
% \label{eq:linedetection-analyze5}
% T(I)& = \Theta(nk)
% \end{align}
% \end{subequations}

Das Verfahren zur Bestimmung der Edgels (\autoref{alg:findedgels-horizontal} und \autoref{alg:findedgels-vertical})
 benötigt das monochrome Bildsignal $I_m$, sowie die Position der oberen linken Ecke der Region, die durch oben $t$ und
 links $l$ definiert ist.
\begin{algorithm}[!ht]
\caption{\textproc{findEdgels} (Horizontale Scanlines)}
\label{alg:findedgels-horizontal}
	\begin{algorithmic}[1]
		\Require $I_m, E, t, l$
		\For{$y \gets t$ \textbf{to} $y < t + \mathit{regionSize}$}
		\label{alg:findedgels-horizontal-scanlinestart}
			\State $p_1 \gets 0$
			\State $p_2 \gets 0$
			\For{$x \gets l$ \textbf{to} $x < l + \mathit{regionSize}$}
			\label{alg:findedgels-horizontal-loopstart}
				\State $\mathit{currentEdgel} \gets$ \Call{convoluteKernelX}{$I_m,x,y,\mathit{imageWidth},\mathit{imageHeight}$}
				\label{alg:findedgels-horizontal-convolute}
				\If{$\mathit{currentEdgel} > \mathit{threshold}$}
				\label{alg:findedgels-horizontal-foundedgel}
					\Comment Möglicherweise ein Egdel
				\Else
					\State $\mathit{currentEdgel} \gets 0$
				\EndIf
				\If{$p_1 > 0 \land p_1 > p_2 \land p_1 > \mathit{currentEdgel}$}
				\label{alg:findedgels-horizontal-maxima}
					\Comment $p_1$ ist lokales Maximum
					\State $\mathit{edgel} \gets \infty$
					\State \Call{vectorSetCoordinate}{$\mathit{edgel},x - 1,y$}
					\State $\mathit{edgel.slope} \gets$ \Call{gradientIntensity}{$I_m, \mathit{imageWidth}, \mathit{imageHeight}, x - 1,y$}
					\State \Call{addEdgel}{$E,\mathit{edgel}$}
				\EndIf
				\label{alg:findedgels-horizontal-maxima-end}
				\State $p_2 \gets p_1$
				\label{alg:findedgels-horizontal-copy-prev1}
				\State $p_1 \gets currentEdgel$
				\label{alg:findedgels-horizontal-copy-edgel}
				\State $x \gets x + 1$
			\EndFor
			\label{alg:findedgels-horizontal-loopend}
			\State $y \gets y + 5$
			\label{alg:findedgels-horizontal-increment}
		\EndFor
		\label{alg:findedgels-horizontal-scanlineend}
	\algstore{brkfindedgels}
	\end{algorithmic}
\end{algorithm}

Der Zeiger $E$ wird zur Speicherung der gefundenen \glspl{edgel} verwendet. Zeile
 \ref{alg:findedgels-horizontal-scanlinestart}--\ref{alg:findedgels-horizontal-scanlineend} ist für den Aufbau der
 horizontalen Scanlines verantwortlich. Die Überprüfung sorgt dafür, dass die Scanlines bis zum Ende der Region im
 Abstand von $5$ Pixeln untersucht werden. Nach der Initialisierung der Variablen wird in der Schleife von
 Zeile~\ref{alg:findedgels-horizontal-loopstart}--\ref{alg:findedgels-horizontal-loopend} jeder Pixel auf der Scanline
 untersucht. Zuerst wird in Zeile \ref{alg:findedgels-horizontal-convolute} die Faltung mit einem Gauß-Kernel
 vorgenommen (Vgl. \autoref{alg:convolutekernel-horizontal}, S. \pageref{alg:convolutekernel-horizontal}). Der Test
 in Zeile \ref{alg:findedgels-horizontal-foundedgel} überprüft anschließend das Ergebnis der Faltung. Wenn der
 Schwellwert nicht überschritten wird, gibt es keinen genügend großen Anstieg des Gradienten und das Ergebnis wird auf
 $0$ gesetzt. Wird der Schwellwert überschritten, handelt es sich um einen Edgel und das Ergebnis wird in den
 Bedingungen von Zeile \ref{alg:findedgels-horizontal-maxima} weiter untersucht, ob es sich um ein lokales Maximum
 handelt. Ein lokales Maximum bedeutet, dass ein Edgel einen größeren Gradienten besitzt als seine beiden Nachbarn. Die
 Bedingung in Zeile \ref{alg:findedgels-horizontal-maxima} wird bei der ersten Überprüfung immer fehlschlagen.
 Dadurch wird sichergestellt, dass kein Maximum an den Rändern existiert, da hier nicht genügend Nachbarn vorhanden sind
 um eine verlässliche Aussage zu treffen. Zeile \ref{alg:findedgels-horizontal-copy-prev1} und
 Zeile \ref{alg:findedgels-horizontal-copy-edgel} kopieren die Werte für den nächsten Durchlauf. Durch das kopieren der
 Werte werden die Nachbarn für den nächsten Durchlauf um eine Position weiterverschoben. Nur bei einem lokalen Maximum
 (Zeile \ref{alg:findedgels-horizontal-maxima}--\ref{alg:findedgels-horizontal-maxima-end}) wird die Position des
 Edgels gespeichert (Vgl. \autoref{alg:vectorsetcoordinate}, S. \pageref{alg:vectorsetcoordinate}), und seine
 Orientierung berechnet (Vgl. \autoref{alg:gradientintensity}, S. \pageref{alg:gradientintensity}). Der Edgel wird mit
 \textproc{addEdgel} (\autoref{alg:edgelpool-addedgel}, S. \pageref{alg:edgelpool-addedgel}) in einem Speicherblock zu
 weiteren Verarbeitung gespeichert. Sind alle Pixel auf einer Scanline untersucht, wird in Zeile
 \ref{alg:findedgels-horizontal-increment} die nächste Scanline ausgewählt. Das Verfahren wird solange wiederholt, bis
 alle Scanlines innerhalb der Region untersucht wurden. \autoref{alg:findedgels-vertical} untersucht die vertikalen
 Scanlines in Zeile \ref{alg:findedgels-vertical-scanlinestart}--\ref{alg:findedgels-vertical-scanlineend} analog zu
 \autoref{alg:findedgels-horizontal} Zeile
 \ref{alg:findedgels-horizontal-scanlinestart}--\ref{alg:findedgels-horizontal-scanlineend}.
\begin{algorithm}[!ht]
\caption{\textproc{findEdgels} (Vertikale Scanlines)}
\label{alg:findedgels-vertical}
\begin{algorithmic}[1]
\algrestore{brkfindedgels}
	\For{$x \gets l$ \textbf{to} $x < l + \mathit{regionSize}$}
	\Cost{$c_{21}$}{$\frac{40}{5} + 1$}
	\label{alg:findedgels-vertical-scanlinestart}
		\State $p_1, p_2 \gets 0$
		\Cost{$c_{22}$}{$8(2)$}
		\For{$y \gets t$ \textbf{to} $y < t + \mathit{regionSize}$}
		\Cost{$c_{23}$}{$8(40 + 1)$}
			\State $currentEdgel \gets$ \Call{convoluteKernelY}{$I_m,x,y,\mathit{imageWidth},\mathit{imageHeight}$}
			\Cost{$c_{24}$}{$320t_{3}$}
			\If{$currentEdgel > threshold$}
			\Comment Möglicherweise ein Egdel
			\Cost{$c_{25}$}{$320$}
			\Else
				\State $currentEdgel \gets 0$
				\Cost{$c_{27}$}{$320$}
			\EndIf
			\If{$p_1 > 0 \land p_1 > p_2 \land p_1 > currentEdgel$}
			\Comment $p_1$ ist lokales Maximum
			\Cost{$c_{29}$}{$320(5)$}
			\label{alg:findedgels-vertical-maxima}
				\State $edgel \gets \infty$
				\Cost{$c_{30}$}{$320$}
				\State \Call{vectorSetCoordinate}{$\mathit{edgel},x,y - 1$}
				\Cost{$c_{31}$}{$320 \cdot \Theta(1)$}
				\State $\mathit{edgel.slope} \gets$ \Call{gradientIntensity}{$I_m, \mathit{imageWidth},
				 \mathit{imageHeight}, x,y - 1$}
				\Cost{$c_{32}$}{$320t_{4}$}
				\State \Call{addEdgel}{$E,\mathit{edgel}$}
				\Cost{$c_{33}$}{$320 \cdot \Theta(1)$}
			\EndIf
			\State $p_2 \gets p_1$
			\Cost{$c_{35}$}{$320$}
			\State $p_1 \gets currentEdgel$
			\Cost{$c_{36}$}{$320$}
			\State $y \gets y + 1$
			\Cost{$c_{37}$}{$320$}
		\EndFor
		\State $x \gets x + 5$
		\Cost{$c_{39}$}{$8(2)$}
	\EndFor
	\label{alg:findedgels-vertical-scanlineend}
\end{algorithmic}
\end{algorithm}


In \autoref{alg:findedgels-horizontal-analyse} und \autoref{alg:findedgels-vertical-analyse} sind die Kosten von
 \textproc{findEdgels} aufgeführt.
\begin{algorithm}[!ht]
\caption{\textproc{findEdgels} (Analyse)}
\label{alg:findedgels-horizontal-analyse}
	\begin{algorithmic}[1]
		\Require $I_m, E, t, l$
		\For{$y \gets t$ \textbf{to} $y < t + \mathit{regionSize}$}
		\Cost{$c_1$}{$\tfrac{n}{5} + 1$}
			\State $p_1 \gets 0$
			\Cost{$c_2$}{$\sum_{y=0}^{\tfrac{n}{5}} 1$}
			\State $p_2 \gets 0$
			\Cost{$c_3$}{$\sum_{y=0}^{\tfrac{n}{5}} 1$}
			\For{$x \gets l$ \textbf{to} $x < l + \mathit{regionSize}$}
			\Cost{$c_4$}{$\sum_{y=0}^{\tfrac{n}{5}} (n + 1)$}
				\State Gradienten bestimmen
				\Cost{$c_5$}{$\sum_{y=0}^{\tfrac{n}{5}} \sum_{x=0}^{n} \Theta(1)$}
				\If{$\mathit{currentEdgel} > \mathit{threshold}$}
				\Cost{$c_6$}{$\sum_{y=0}^{\tfrac{n}{5}} \sum_{x=0}^{n} 1$}
				\Else
					\State $\mathit{currentEdgel} \gets 0$
					\Cost{$c_7$}{$\sum_{y=0}^{\tfrac{n}{5}} \sum_{x=0}^{n} 1$}
				\EndIf
				\If{$p_1 > 0 \land p_1 > p_2 \land p_1 > \mathit{currentEdgel}$}
				\Cost{$c_8$}{$\sum_{y=0}^{\tfrac{n}{5}} \sum_{x=0}^{n} 1$}
					\State $\mathit{edgel} \gets \infty$
					\Cost{$c_9$}{$\sum_{y=0}^{\tfrac{n}{5}} \sum_{x=0}^{n} 1$}
					\State Koordinaten speichern
					\Cost{$c_{10}$}{$\sum_{y=0}^{\tfrac{n}{5}} \sum_{x=0}^{n} \Theta(1)$}
					\State Orientierung berechnen
					\Cost{$c_{11}$}{$\sum_{y=0}^{\tfrac{n}{5}} \sum_{x=0}^{n} \Theta(1)$}
					\State \Call{addEdgel}{$E,\mathit{edgel}$}
					\Cost{$c_{12}$}{$\sum_{y=0}^{\tfrac{n}{5}} \sum_{x=0}^{n} \Theta(1)$}
				\EndIf
				\State $p_2 \gets p_1$
				\Cost{$c_{13}$}{$\sum_{y=0}^{\tfrac{n}{5}} \sum_{x=0}^{n} 1$}
				\State $p_1 \gets currentEdgel$
				\Cost{$c_{14}$}{$\sum_{y=0}^{\tfrac{n}{5}} \sum_{x=0}^{n} 1$}
				\State $x \gets x + 1$
				\Cost{$c_{15}$}{$\sum_{y=0}^{\tfrac{n}{5}} \sum_{x=0}^{n} 1$}
			\EndFor
			\State $y \gets y + 5$
			\Cost{$c_{16}$}{$\sum_{y=0}^{\tfrac{n}{5}} 1$}
		\EndFor
	\algstore{brk-findedgelsanalyse}
	\end{algorithmic}
\end{algorithm}

Die bereits vorgestellten Methoden \textproc{vectorSetCoordinate} (\autoref{alg:vectorsetcoordinate}) und
 \textproc{addEdgel} (\autoref{alg:edgelpool-addedgel}) haben eine konstante Laufzeit. Die Methoden
 \textproc{convoluteKernelX}, \textproc{convoluteKernelY} und \textproc{gradientIntensity} haben ebenfalls eine
 konstante Laufzeit, die zu einem späteren Zeitpunkt bewiesen wird.  Die Laufzeit von
 \autoref{alg:findedgels-horizontal-analyse} lässt sich wie in \autoref{eq:findedgels1} zusammenfassen.
\begin{subequations}
\begin{align}
\label{eq:findedgels1}
T(n)& =
c_1 \cdot (\frac{n}{5} + 1) + (c_2 + c_3 + c_{16}) \cdot \sum_{y=0}^{\frac{n}{5}} 1
 + c_4 \cdot \sum_{y=0}^{\frac{n}{5}} (n+1) \\
& \quad + (c_5 + c_6 + c_7 + c_8 + c_9 + c_{10} + c_{11} + c_{12} + c_{13} + c_{14} + c_{15})
 \cdot \sum_{y=0}^{\frac{n}{5}} \sum_{x=0}^{n} 1 \nonumber \\
\label{eq:findedgels2}
T(n)& = c_1 + (c_1 + c_2 + c_3 + c_4 + c_{16}) \cdot \frac{n}{5} + (c_4 + \ldots + c_{15})
\cdot (\frac{n}{5} \cdot n) \\
\label{eq:findedgels3}
T(n)& = \frac{1}{5}n^2 \\
\label{eq:findedgels1-end}
T(n)& = n^2
\end{align}
\end{subequations}

Der Algorithmus ist von den Variablen $t$, $l$ und $\mathit{regionSize}$ abhängig. Der Bereich
 $[t,t+\mathit{regionSize})$ und $[l,l+\mathit{regionSize})$ wird in der Analyse als $[t,t+\mathit{regionSize}) = n$
 und $[l,l+\mathit{regionSize}) = m$ bezeichnet. Durch Umformung in \autoref{eq:findedgels2} werden die Konstanten
 isoliert, was zu einer Laufzeit von $T(n,m) = \Theta(\tfrac{nm}{5})$ für \autoref{alg:findedgels-horizontal-analyse}
 führt (\autoref{eq:findedgels3}). Die Kosten für \textproc{findEdgels} zur Untersuchung der vertikalen Scanline sind in
 \autoref{alg:findedgels-vertical-analyse} aufgeführt und entsprechen den Kosten von
 \autoref{alg:findedgels-horizontal-analyse}.
\begin{algorithm}[!ht]\small
	\caption{\textproc{findEdgels} (Fortsetzung der Analyse)}
	\label{alg:findedgels-vertical-analyse}
	\begin{algorithmic}[1]
	\algrestore{brk-findedgelsanalyse}
		\For{$x \gets l$ \textbf{to} $x < l + \mathit{regionSize}$}
		\Cost{$c_{17}$}{$\tfrac{n}{5} + 1$}
			\State $p_1 \gets 0$
			\Cost{$c_{18}$}{$\sum_{x=0}^{\tfrac{n}{5}} 1$}
			\State $p_2 \gets 0$
			\Cost{$c_{19}$}{$\sum_{x=0}^{\tfrac{n}{5}} 1$}
			\For{$y \gets t$ \textbf{to} $y < t + \mathit{regionSize}$}
			\Cost{$c_{20}$}{$\sum_{x=0}^{\tfrac{n}{5}} (n + 1)$}
				\State Gradient bestimmen
				\Cost{$c_{21}$}{$\sum_{x=0}^{\tfrac{n}{5}} \sum_{y=0}^{n} \Theta(1)$}
				\If{$currentEdgel > threshold$}
				\Cost{$c_{22}$}{$\sum_{x=0}^{\tfrac{n}{5}} \sum_{y=0}^{n} 1$}
				\Else
					\State $currentEdgel \gets 0$
					\Cost{$c_{23}$}{$\sum_{x=0}^{\tfrac{n}{5}} \sum_{y=0}^{n} 1$}
				\EndIf
				\If{$p_1 > 0 \land p_1 > p_2 \land p_1 > currentEdgel$}
				\Cost{$c_{24}$}{$\sum_{x=0}^{\tfrac{n}{5}} \sum_{y=0}^{n} 1$}
					\State $edgel \gets \infty$
					\Cost{$c_{25}$}{$\sum_{x=0}^{\tfrac{n}{5}} \sum_{y=0}^{n} 1$}
					\State Koordinaten speichern
					\Cost{$c_{26}$}{$\sum_{x=0}^{\tfrac{n}{5}} \sum_{y=0}^{n} \Theta(1)$}
					\State Orientierung berechnen
					\Cost{$c_{27}$}{$\sum_{x=0}^{\tfrac{n}{5}} \sum_{y=0}^{n} \Theta(1)$}
					\State \Call{addEdgel}{$E,\mathit{edgel}$}
					\Cost{$c_{28}$}{$\sum_{x=0}^{\tfrac{n}{5}} \sum_{y=0}^{n} \Theta(1)$}
				\EndIf
				\State $p_2 \gets p_1$
				\Cost{$c_{29}$}{$\sum_{x=0}^{\tfrac{n}{5}} \sum_{y=0}^{n} 1$}
				\State $p_1 \gets currentEdgel$
				\Cost{$c_{30}$}{$\sum_{x=0}^{\tfrac{n}{5}} \sum_{y=0}^{n} 1$}
				\State $y \gets y + 1$
				\Cost{$c_{31}$}{$\sum_{x=0}^{\tfrac{n}{5}} \sum_{y=0}^{n} 1$}
			\EndFor
			\State $x \gets x + 5$
			\Cost{$c_{32}$}{$\sum_{x=0}^{\tfrac{n}{5}} 1$}
		\EndFor
	\end{algorithmic}
\end{algorithm}

Die Kosten des Algorithmus sind in \autoref{eq:findedgels4} aufgeführt und durch Umformung in \autoref{eq:findedgels5}
 werden die Konstanten isoliert.
\begin{subequations}
\begin{align}
\label{eq:findedgels4}
T(n,m)& =
c_{17} \cdot (\frac{m}{5} + 1) + (c_{18} + c_{19} + c_{32}) \cdot \sum_{x=0}^{\frac{m}{5}} 1
 + c_{20} \cdot \sum_{x=0}^{\frac{m}{5}} (n+1) \\
& \quad + (c_{21} + c_{22} + c_{23} + c_{24} + c_{25} + c_{26} + c_{27} + c_{28} + c_{29} + c_{30} + c_{31})
 \cdot \sum_{x=0}^{\frac{m}{5}} \sum_{y=0}^{n} 1 \nonumber \\
\label{eq:findedgels5}
T(n,m)& = c_{17} + (c_{17} + c_{18} + c_{19} + c_{20} + c_{32}) \cdot \frac{m}{5} + (c_{20} + \ldots + c_{31})
\cdot (\frac{m}{5} \cdot n) \\
\label{eq:findedgels6}
T(n,m)& = \frac{m \cdot n}{5}
\end{align}
\end{subequations}

Dies führt zu einer Laufzeit von $T(n,m) = \Theta(\tfrac{nm}{5})$ für \autoref{alg:findedgels-vertical-analyse}
 (\autoref{eq:findedgels6}).
Um die Laufzeit von \textproc{findEdgels} zu bestimmen, werden die Laufzeiten von
 \autoref{alg:findedgels-horizontal-analyse} und \autoref{alg:findedgels-vertical-analyse} in \autoref{eq:findedgels7}
 zusammengefasst.
\begin{subequations}
\begin{align}
\label{eq:findedgels7}
T(n,m)& = \frac{n \cdot m}{5} + \frac{n \cdot m}{5}\\
\label{eq:findedgels8}
T(n,m)& = 2 \cdot \frac{n \cdot m}{5}\\
\label{eq:findedgels9}
T(n,m)& = \frac{n \cdot m}{5}
\end{align}
\end{subequations}

Durch Umformung in \autoref{eq:findedgels8} kann die Laufzeit des Algorithmus in \autoref{eq:findedgels9} ermittelt
 werden. Die Laufzeit von \textproc{findEdgels} entspricht demnach $T(n,m) = \Theta(\tfrac{nm}{5})$.

\autoref{alg:convolutekernel-horizontal} und \autoref{alg:convolutekernel-vertical} berechnen den Gradienten durch Faltung mit dem Gauß-Kernel
$\left( \begin{smallmatrix}
-3& -5& 0& 5& 3
\end{smallmatrix} \right)$
 auf der horizontalen und vertikalen Scanline. Als Parameter benötigt der Algorithmus den Zeiger des monochromen
 Bildsignals $I_m$, die Position des Pixels ($x$ und $y$), sowie die Breite $w$ und Höhe $h$ von $I_m$. In Zeile
 \ref{alg:convolutekernel-horizontal-readstart}--\ref{alg:convolutekernel-horizontal-readend} werden durch die
 Funktion \textproc{getpixel} (Vgl. \autoref{alg:getpixel}, S. \pageref{alg:getpixel}) die benötigten Pixelwerte
 ausgelesen und den Variablen zugewiesen. Im Anschluss werden die Werte mit dem Gauß-Kernel
$\left( \begin{smallmatrix}
-3& -5& 0& 5& 3
\end{smallmatrix} \right)$
berechnet um den Gradienten zu bestimmen.
\begin{algorithm}[!ht]\small
\caption{\textproc{convoluteKernelX} (horizontale Scanline)}
\label{alg:convolutekernel-horizontal}
\begin{algorithmic}[1]
	\Require $I_m,x,y,w,h$
	\State $p_1 \gets$ \Call{getpixel}{$I_m, x - 2, y, w, h$}
	\Cost{$c_{1}$}{$2 + \Theta(1)$}
	\label{alg:convolutekernel-horizontal-readstart}
	\State $p_2 \gets$ \Call{getpixel}{$I_m, x - 1, y, w, h$}
	\Cost{$c_{2}$}{$2 + \Theta(1)$}
	% \State $p_3 \gets$ \Call{getpixel}{$I_m, x, y, w, h$}
	\State $p_4 \gets$ \Call{getpixel}{$I_m, x + 1, y, w, h$}
	\Cost{$c_{3}$}{$2 + \Theta(1)$}
	\State $p_5 \gets$ \Call{getpixel}{$I_m, x + 2, y, w, h$}
	\Cost{$c_{4}$}{$2 + \Theta(1)$}
	\label{alg:convolutekernel-horizontal-readend}
	\State $v \gets 0$
	\Cost{$c_{5}$}{$1$}
	\State $v \gets v + \left( -3 \cdot p_1 \right)$
	\Cost{$c_{6}$}{$3$}
	\State $v \gets v + \left( -5 \cdot p_2 \right)$
	\Cost{$c_{7}$}{$3$}
	% \State $v \gets v + \left( 0 \cdot p_3 \right)$
	\State $v \gets v + \left( 5 \cdot p_4 \right)$
	\Cost{$c_{8}$}{$3$}
	\State $v \gets v + \left( 3 \cdot p_5 \right)$
	\Cost{$c_{9}$}{$3$}
	\State \textbf{return} $v$
	\Cost{$c_{10}$}{$1$}
\end{algorithmic}
\end{algorithm}

Bei genauer Betrachtung von \autoref{alg:convolutekernel-horizontal} und \autoref{alg:convolutekernel-vertical}
 fällt auf, dass der Wert $p_3$ in der Berechnung nicht vorkommt.
\input{alg/analyse-hirzer/convolutekernely}
Dies ist darauf zurückzuführen, dass der Gauß-Kernel an der dritten Stelle mit $0$ definiert ist. Somit kann die
 Multiplikation vernachlässigt werden. Die Laufzeit von \autoref{alg:convolutekernel-horizontal} und
 \autoref{alg:convolutekernel-vertical} ist konstant.

In \autoref{alg:gradientintensity} wird mittels Faltung die Orientierung eines \glspl{edgel} bestimmt. Als
 Eingabeparameter wird das monochrome Bildsignal $I_m$, dessen Breite $w$ und Höhe $h$, sowie die Position des
 \glspl{edgel} ($x,y$) benötigt.
\begin{algorithm}[!ht]\small
\caption{\textproc{gradientIntensity}}
\label{alg:gradientintensity}
\begin{algorithmic}[1]
	\Require $I_m, x, y, w, h$
	\State $p_{x_1} \gets$ \Call{getPixel}{$I_m, x -1, y - 1, w, h$}
	\Cost{$c_{1}$}{$3 + \Theta(1)$}
	\label{alg:gradientintensity-readstart}
	\State $p_{x_2} \gets$ \Call{getPixel}{$I_m, x, y - 1, w, h$}
	\Cost{$c_{2}$}{$2 + \Theta(1)$}
	\State $p_{x_3} \gets$ \Call{getPixel}{$I_m, x + 1, y - 1, w, h$}
	\Cost{$c_{3}$}{$3 + \Theta(1)$}
	\State $p_{x_4} \gets$ \Call{getPixel}{$I_m, x - 1, y + 1, w, h$}
	\Cost{$c_{4}$}{$3 + \Theta(1)$}
	\State $p_{x_5} \gets$ \Call{getPixel}{$I_m, x, y + 1, w, h$}
	\Cost{$c_{5}$}{$2 + \Theta(1)$}
	\State $p_{x_6} \gets$ \Call{getPixel}{$I_m, x + 1, y + 1, w, h$}
	\Cost{$c_{6}$}{$3 + \Theta(1)$}
	\State $p_{y_1} \gets$ \Call{getPixel}{$I_m, x - 1, y - 1, w, h$}
	\Cost{$c_{7}$}{$3 + \Theta(1)$}
	\State $p_{y_2} \gets$ \Call{getPixel}{$I_m, x - 1, y, w, h$}
	\Cost{$c_{8}$}{$2 + \Theta(1)$}
	\State $p_{y_3} \gets$ \Call{getPixel}{$I_m, x - 1, y + 1, w, h$}
	\Cost{$c_{9}$}{$3 + \Theta(1)$}
	\State $p_{y_4} \gets$ \Call{getPixel}{$I_m, x + 1, y - 1, w, h$}
	\Cost{$c_{10}$}{$3 + \Theta(1)$}
	\State $p_{y_5} \gets$ \Call{getPixel}{$I_m, x + 1, y, w, h$}
	\Cost{$c_{11}$}{$2 + \Theta(1)$}
	\State $p_{y_6} \gets$ \Call{getPixel}{$I_m, x + 1, y + 1, w, h$}
	\Cost{$c_{12}$}{$2 + \Theta(1)$}
	\label{alg:gradientintensity-readend}
	% \State $g_x \gets 0$
	% \State $g_y \gets 0$
	\State $g_x \gets g_x + p_{x_1}$
	\Cost{$c_{13}$}{$2$}
	\label{alg:gradientintensity-convolutestart}
	\State $g_x \gets g_x + \left(p_{x_2} \cdot 2\right)$
	\Cost{$c_{14}$}{$3$}
	\State $g_x \gets g_x + p_{x_3}$
	\Cost{$c_{15}$}{$2$}
	\State $g_x \gets g_x - p_{x_4}$
	\Cost{$c_{16}$}{$2$}
	\State $g_x \gets g_x - \left(p_{x_5} \cdot 2\right)$
	\Cost{$c_{17}$}{$3$}
	\State $g_x \gets g_x - p_{x_6}$
	\Cost{$c_{18}$}{$2$}
	\State $g_y \gets g_y + p_{y_1}$
	\Cost{$c_{19}$}{$2$}
	\State $g_y \gets g_y + \left(p_{y_2} \cdot 2\right)$
	\Cost{$c_{20}$}{$3$}
	\State $g_y \gets g_y + p_{y_3}$
	\Cost{$c_{21}$}{$2$}
	\State $g_y \gets g_y - p_{y_4}$
	\Cost{$c_{22}$}{$2$}
	\State $g_y \gets g_y - \left(p_{y_5} \cdot 2\right)$
	\Cost{$c_{23}$}{$3$}
	\State $g_y \gets g_y - p_{y_6}$
	\Cost{$c_{24}$}{$2$}
	\label{alg:gradientintensity-convoluteend}
	\State $\mathit{slope} \gets \infty$
	\Cost{$c_{25}$}{$1$}
	\State \Call{vectorSetCoordinate}{$\mathit{slope},\mathit{gy},\mathit{gx}$}
	\Cost{$c_{26}$}{$\Theta(1)$}
	\label{alg:gradientintensity-vector-start}
	\State \Call{normalized}{$\mathit{slope}$}
	\Cost{$c_{27}$}{$\Theta(1)$}
	\label{alg:gradientintensity-vector-end}
	\State \textbf{return} $\mathit{slope}$
	\Cost{$c_{28}$}{$1$}
\end{algorithmic}
\end{algorithm}

In Zeile \ref{alg:gradientintensity-readstart}--\ref{alg:gradientintensity-readend} werden die Pixelwerte ausgelesen
 und den Variablen zugewiesen. In Zeile
 \ref{alg:gradientintensity-convolutestart}--\ref{alg:gradientintensity-convoluteend} erfolgt die Faltung mit dem
 Sobel-Operator\footcite[Vgl.][S.~120--123]{burger05}, dessen Filter
\begin{subequations}
\begin{align}
	H_x =&
	\begin{pmatrix}
		1& 0& -1\\
		2& 0& -2\\
		1& 0& -1
	\end{pmatrix}
\end{align}
\begin{align}
	H_y =&
	\begin{pmatrix}
		1& 2& 1\\
		0& 0& 0\\
		-1& -2& -1
	\end{pmatrix}
\end{align}
\end{subequations}
den Gradienten $G_x$ und $G_y$ bestimmen. Wie in \autoref{alg:convolutekernel-horizontal} werden Multiplikationen von
 $0$-Werten des Filters vernachlässigt. Mit
\begin{equation}
	\label{eq:orientation}
	\Phi(x,y) = \arctan{\left(\tfrac{G_y}{G_x}\right)}
\end{equation}
kann die Orientierung berechnet werden. $G_x$ und $G_y$ werden mit \autoref{alg:vectorsetcoordinate} als
 \textproc{vector} gespeichert und normalisiert
 (Zeile \ref{alg:gradientintensity-vector-start}--\ref{alg:gradientintensity-vector-end}). Die Laufzeit von
 \autoref{alg:gradientintensity} ist konstant.

Um aus den gefundenen \glspl{edgel} Liniensegmente zu erzeugen, wird \autoref{alg:findlinesegments1} verwendet. Das
 Verfahren benötigt den Speicherblock $E$, in dem die \glspl{edgel} vorliegen, und den Speicherblock $L$, der zur
 Speicherung der gefundenen Liniensegmente dient.

\algblockdefx[DOWHILE]{DO}{WHILE}
   {\textbf{do}}
   [1]{\textbf{while} #1}
\begin{algorithm}[ht]
\caption{\textproc{findLineSegments}}
\label{alg:findlinesegments1}
\begin{algorithmic}[1]
	\Require $E, L$
	\State $\mathit{presentLine} \gets \infty$
	\State $\mathit{edgelsInRegion} \gets 0$
	\DO
		\State $\mathit{presentLine.supportCount} \gets 0$
		\For{$i \gets 0$ \textbf{to} $i < 25$}
			\State \ldots \Comment 1. Teil
			\State $i \gets i + 1$
		\EndFor
		\If{$\mathit{presentLine.supportCount} > \mathit{minEdgels}$}
			\State \ldots \Comment 2. Teil
		\EndIf
		\State $\mathit{edgelsInRegion} \gets$ \Call{getEdgelCount}{$E$}
	\WHILE{$\mathit{presentLine.supportCount} > \mathit{minEdgels} \land \mathit{edgelsInRegion} > \mathit{minEdgels}$}
\end{algorithmic}
\end{algorithm}


In Zeile \ref{alg:findlinesegments1-init-start}--\ref{alg:findlinesegments1-init-end} wird die Variable
 $\mathit{presentLine}$ initialisiert. Diese Variable enthält das zu speichernde Liniensegment im Verfahren.
 Die Zählvariable $\mathit{edgelsInRegion}$ wird mit $0$ initialisiert und speichert die Anzahl der \gls{edgel} in
 einer Region. In der Schleife von Zeile \ref{alg:findlinesegments1-loop-start} bis Zeile
 \ref{alg:findlinesegments1-loop-end} wird das RANSAC-Verfahren wiederholt, solange das zu untersuchende Liniensegment
 genügend Unterstützung durch \gls{edgel} besitzt und genügend \gls{edgel} in der Region vorhanden sind. Die Anzahl der
 unterstützenden \gls{edgel} wird in Zeile \ref{alg:findlinesegments1-clearsupport} für jeden Durchlauf gelöscht.
 Danach wird in Zeile \ref{alg:findlinesegments1-line-start}--\ref{alg:findlinesegments1-line-end} ein zufälliges
 Liniensegment in der Region erstellt und untersucht
 (Vgl. \autoref{alg:findlinesegments2}--\autoref{alg:findlinesegments4}). Die Erstellung eines Liniensegments wird
 mehrmahls wiederholt, um das Liniensegment mit den meisten Unterstützungsedgels zu finden. In Zeile
 \ref{alg:findlinesegments1-hasenoughsupport} wird die Anzahl der der unterstützenden \gls{edgel} mit der Anzahl der
 benötigten \gls{edgel} verglichen. Nur wenn genügend \gls{edgel} das Liniensegment unterstützen, wird in
 \autoref{alg:findlinesegments5}--\autoref{alg:findlinesegments7} das Verfahren fortgesetzt und die
 Unterstützungsedgels aus dem Speicherblock $E$ entfernt.

In \autoref{alg:findlinesegments2} wird die Initialisierung der Variablen vorgenommen, die zur Erstellung eines
 Liniensegments benötigt werden. Die Variablen $\mathit{start}$ und $\mathit{end}$ werden als Indizes für \gls{edgel}
 benutzt. Die Variable $\mathit{maxIteration}$ gibt die maximale Anzahl der Interationen an, die $\mathit{iterator}$
 durchlaufen kann. $\mathit{first}$ ist der Startedgel und $\mathit{last}$ der Endedgel des Liniensegments. In
 $\mathit{numberOfEdgels}$ wird die Anzahl der \gls{edgel} in $E$ gespeichert.

\algblockdefx[DOWHILE]{DO}{WHILE}
   {\textbf{do}}
   [1]{\textbf{while} #1}
\begin{algorithm}[ht]
\caption{\textproc{findLineSegments} (Initialisierung der Liniensegmentvariablen)}
\label{alg:findlinesegments2}
\begin{algorithmic}[1]
	\State $\mathit{start} \gets \infty$
	\State $\mathit{end} \gets \infty$
	\State $\mathit{maxIteration} \gets 64$
	\State $\mathit{iterator} \gets 0$
	\State $\mathit{first} \gets \infty$
	\State $\mathit{last} \gets \infty$
	\State $\mathit{numberOfEdgels} \gets$ \Call{getEdgelCount}{$E$}
	\algstore{brk-findlinesegments2-init}
\end{algorithmic}
\end{algorithm}


Die Auswahl eines Liniensegments erfolgt mit \autoref{alg:findlinesegments3}. Zeile
 \ref{alg:findlinesegments3-loop-start}--\ref{alg:findlinesegments3-loop-end} ist dafür verantwortlich, solange nach
 einem Liniensegment zu suchen, bis der erste und letzte \gls{edgel} sich unterscheiden oder ihre Orientierung
 zueinander kompatibel ist. Als letzte Bedingung muss die Anzahl der Iterationen unterhalb des festgelegten
 Schwellwerts $\mathit{maxIteration}$ bleiben.

\algblockdefx[DOWHILE]{DO}{WHILE}
   {\textbf{do}}
   [1]{\textbf{while} #1}
\begin{algorithm}[!ht]
\caption{\textproc{findLineSegments} (Liniensegment suchen)}
\label{alg:findlinesegments3}
\begin{algorithmic}[1]
	\algrestore{brk-findlinesegments2-init}
	\DO
	\label{alg:findlinesegments3-loop-start}
		\State $\mathit{first} \gets$ \Call{rand} $ \bmod \mathit{numberOfEdgels}$
		\label{alg:findlinesegments3-first}
		\State $\mathit{last} \gets$ \Call{rand} $ \bmod \mathit{numberOfEdgels}$
		\label{alg:findlinesegments3-last}
		\State $\mathit{start} \gets$ \Call{getEdgel}{$E,\mathit{first}$}
		\label{alg:findlinesegments3-start}
		\State $\mathit{end} \gets$ \Call{getEdgel}{$E,\mathit{last}$}
		\label{alg:findlinesegments3-end}
		\State $\mathit{iterator} \gets \mathit{iterator} + 1$
		\label{alg:findlinesegments3-inc}
	\WHILE{$\bigl(\mathit{first} = \mathit{last} \lor \lnot \textproc{isCompatible}(\mathit{start},\mathit{end})\bigr)
	 \land \mathit{iterator} < \mathit{maxIteration}$}
	\label{alg:findlinesegments3-loop-end}
	\algstore{brk-findlinesegments3-dowhile}
\end{algorithmic}
\end{algorithm}


In Zeile \ref{alg:findlinesegments3-first} und Zeile \ref{alg:findlinesegments3-last} werden zufällig zwei Indizes aus
 der Menge der vorhandenen \gls{edgel} errechnet. Diese werden in Zeile
 \ref{alg:findlinesegments3-start}--\ref{alg:findlinesegments3-end} verwendet, um die beiden \gls{edgel}
 $\mathit{start}$ und $\mathit{end}$ auszuwählen. Der Iterator wird anschliessend in Zeile
 \ref{alg:findlinesegments3-inc} inkrementiert.

Wenn in \autoref{alg:findlinesegments3} ein Liniensegment erstellt wurde, werden in \autoref{alg:findlinesegments4} die
 \gls{edgel} zur unterstüztung der Hypothese hinzugefügt. Dazu wird in Zeile \ref{alg:findlinesegments4-isbelowmax}
 überprüft, ob die maximale Anzahl der Iterationen nicht überschritten wurde. Falls dem so ist wird
 \autoref{alg:findlinesegments4} nicht weiter ausgeführt. Andernfalls wird in Zeile
 \ref{alg:findlinesegments4-init-start}--\ref{alg:findlinesegments4-init-end} die Variable $\mathit{segment}$ zur
 Speicherung des Liniensegments vorbereitet und der Start- und Endedgel, sowie die Orientierung zugewiesen. Die Anzahl
 der Unterstützungedgels beträgt zu diesem Zeitpunkt noch $0$.

\algblockdefx[DOWHILE]{DO}{WHILE}
   {\textbf{do}}
   [1]{\textbf{while} #1}
\begin{algorithm}[!ht]
\caption{\textproc{findLineSegments} (Unterstützungsedgel bestimmen)}
\label{alg:findlinesegments4}
\begin{algorithmic}[1]
	\algrestore{brk-findlinesegments3-dowhile}
	\If{$\mathit{iterator} < \mathit{maxIteration}$}
	\Cost{$c_{12}$}{$1$}
	\label{alg:findlinesegments4-isbelowmax}
		\State $\mathit{segment} \gets \infty$
		\Cost{$c_{13}$}{$1$}
		\label{alg:findlinesegments4-init-start}
		\State $\mathit{segment.start} \gets \mathit{start}$
		\Cost{$c_{14}$}{$2$}
		\State $\mathit{segment.end} \gets \mathit{end}$
		\Cost{$c_{15}$}{$2$}
		\State $\mathit{segment.slope} \gets \mathit{start.slope}$
		\Cost{$c_{16}$}{$3$}
		\State $\mathit{segment.supportCount} \gets 0$
		\Cost{$c_{17}$}{$2$}
		\State $\mathit{segment.remove} \gets$ \textbf{FALSE}
		\Cost{$c_{18}$}{$2$}
		\State $\mathit{segment.startCorner} \gets$ \textbf{FALSE}
		\Cost{$c_{19}$}{$2$}
		\State $\mathit{segment.endCorner} \gets$ \textbf{FALSE}
		\Cost{$c_{20}$}{$2$}
		\label{alg:findlinesegments4-init-end}
		\For{$j \gets 0$ \textbf{to} $j < \mathit{numberOfEdgels}$}
		\Cost{$c_{21}$}{$n + 1$}
		\label{alg:findlinesegments4-loop-start}
			\State $\mathit{supportEdgel} \gets$ \Call{getEdgel}{$E,j$}
			\Cost{$c_{22}$}{$n \cdot \bigl(1 + \Theta(1)\bigr)$}
			\label{alg:findlinesegments4-edgel}
			\If{\Call{isEdgelNearLine}{$\mathit{segment}, \mathit{supportEdgel}$}}
			\Cost{$c_{23}$}{$n \cdot \bigl(1 + \Theta(1)\bigr)$}
			\label{alg:findlinesegments4-isedgelnearline}
				\State \Call{addEdgel}{$\mathit{segment},\mathit{supportCount},\mathit{segment.supportCount}$}
				\Cost{$c_{24}$}{$n \cdot \Theta(1)$}
				\label{alg:findlinesegments4-addedgel}
				\State $\mathit{segment.supportCount} \gets \mathit{segment.supportCount} + 1$
				\Cost{$c_{25}$}{$4n$}
				\label{alg:findlinesegments4-count}
			\EndIf
			\State $j \gets j + 1$
			\Cost{$c_{27}$}{$n$}
			\label{alg:findlinesegments4-inc}
		\EndFor
		\label{alg:findlinesegments4-loop-end}
		\If{$\mathit{segment.supportCount} > \mathit{presentLine.supportCount}$}
		\Cost{$c_{29}$}{$3$}
		\label{alg:findlinesegments4-hasmoresupport}
			\State $\mathit{presentLine} \gets \mathit{segment}$
			\Cost{$c_{30}$}{$1$}
		\EndIf
	\EndIf
\end{algorithmic}
\end{algorithm}


In der Schleife in Zeile \ref{alg:findlinesegments4-loop-start}--\ref{alg:findlinesegments4-loop-end} werden die
 \gls{edgel} gezählt, die die Linienhypothese unterstützen. Dazu wird in Zeile \ref{alg:findlinesegments4-edgel} ein
 \gls{edgel} ausgewählt und in Zeile \ref{alg:findlinesegments4-isedgelnearline} untersucht, ob der Abstand zur Linie
 klein genug ist. Wenn nicht, wird in Zeile \ref{alg:findlinesegments4-inc} die Laufvariable $j$ inkrementiert und das
 nächste \gls{edgel} ausgewählt. Wenn ein \gls{edgel} nahe genug an dem Liniensegment liegt, wird es in Zeile
 \ref{alg:findlinesegments4-addedgel} dem Liniensegment hinzugefügt und die Anzahl der Unterstützungsedgel wird in
 Zeile \ref{alg:findlinesegments4-count} erhöht. Nachdem alle \gls{edgel} untersucht wurden, wird in Zeile
 \ref{alg:findlinesegments4-hasmoresupport} überprüft, ob die Anzahl der \gls{edgel} des Liniensegments
 $\mathit{segment}$ größer ist als die Anzahl der \gls{edgel} in $\mathit{presentLine}$. Im Falle, dass
 $\mathit{segment}$ mehr Unterstützungsedgel besitzt, wird das Liniensegment in $\mathit{presentLine}$ gespeichert.
 Durch die Wiederholung in \autoref{alg:findlinesegments1} wird sichergestellt, dass das Liniensegment mit der
 größten Unterstützung ausgewählt wird.

Nachdem ein Liniensegment mit genügend Unterstützung ausgewählt wurde, kann mit \autoref{alg:findlinesegments5} die
 Start- und Endposition des Liniensegments bestimmt werden. Da ein Liniensegment aus zwei zufällig ausgewählten
 \glspl{edgel} besteht, können diese \gls{edgel} und die tatsächlichen Start- und Endpunkte voneinander abweichen
 (Vgl. \autoref{fig:})

\algblockdefx[DOWHILE]{DO}{WHILE}
   {\textbf{do}}
   [1]{\textbf{while} #1}
\begin{algorithm}[ht]
\caption{\textproc{findLineSegments} (Fortsetzung 2. Teil)}
\label{alg:findlinesegments5}
\begin{algorithmic}[1]
	\State $\mathit{start} \gets 0$
	\State $\mathit{end} \gets \infty$
	\State $\mathit{slope} \gets$ \Call{vectorSubtract}{$\mathit{presentLine.start.coordinate},\mathit{presentLine.end.coordinate}$}
	\If{$\mathit{slope.x} \leq \mathit{slope.y}$}
		\For{$k \gets 0$ \textbf{to} $k < \mathit{presentLine.supportCount}$}
			\State $\mathit{edgel} \gets \mathit{presentLine.support}[k]$
			\If{$\mathit{edgel.coordinate} > start$}
				\State $\mathit{start} \gets \mathit{edgel.coordinate.y}$
				\State $\mathit{presentLine.start} \gets \mathit{edgel}$
			\EndIf
			\If{$\mathit{edgel.coordinate.y} < \mathit{end}$}
				\State $\mathit{end} \gets \mathit{edgel.coordinate.y}$
				\State $\mathit{presentLine.end} \gets \mathit{edgel}$
			\EndIf
			\State $k \gets k + 1$
		\EndFor
	\algstore{brk-findlinesegments5-if}
\end{algorithmic}
\end{algorithm}


In Zeile \ref{alg:findlinesegments5-start}--\ref{alg:findlinesegments5-end} wird dazu die Variable $\mathit{start}$ mit
 einem kleinen Wert, und die Variable $\mathit{end}$ mit einem großen Wert, initialisiert. Die Steigung des
 Liniensegments und die Orhogonale werden in Zeile
 \ref{alg:findlinesegments5-slope-start}--\ref{alg:findlinesegments5-slope-end} berechnet. In Zeile
 \ref{alg:findlinesegments5-isxsmaller} wird geprüft, ob der Absolutwert der Steigung an Punkt $x$ kleiner ist als der
 Punkt $y$. Falls dem so ist, wird in Zeile
 \ref{alg:findlinesegments5-newstart-start}--\ref{alg:findlinesegments5-newstart-end} ein neuer Start- und Endpunkt
 gesucht, indem die $y$-Koordinate aller Unterstützungsedgels des Liniensegments verglichen werden. Andernfalls wird in
 \autoref{alg:findlinesegments6} in Zeile
 \ref{alg:findlinesegments6-newstart-start}--\ref{alg:findlinesegments6-newstart-end} ein neuer Start- und Endpunkt
 gesucht, indem die $x$-Koordiante aller \glspl{edgel} des Liniensegments untersucht und verglichen werden. Am Ende von
 \autoref{alg:findlinesegments6} ist in $\mathit{presentLine}$ ein neuer Start- und Endpunkt gespeichert.

\algblockdefx[DOWHILE]{DO}{WHILE}
   {\textbf{do}}
   [1]{\textbf{while} #1}
\begin{algorithm}[!ht]\small
\caption{\textproc{findLineSegments} (Neuer Start- und Endpunkt für x)}
\label{alg:findlinesegments6}
\begin{algorithmic}[1]
	\algrestore{brk-findlinesegments5-if}
	\Else
		\For{$k \gets 0$ \textbf{to} $k < \mathit{presentLine.supportCount}$}
		\Cost{$c_{20}$}{$(n + 1)$}
		\label{alg:findlinesegments6-newstart-start}
			\State $\mathit{edgel} \gets \mathit{presentLine.support}[k]$
			\Cost{$c_{21}$}{$3n$}
			\If{$\mathit{edgel.coordinate.x} > \mathit{start}$}
			\Cost{$c_{22}$}{$3n$}
				\State $\mathit{start} \gets \mathit{edgel.coordinate.x}$
				\Cost{$c_{23}$}{$3n$}
				\State $\mathit{presentLine.start} \gets \mathit{edgel}$
				\Cost{$c_{24}$}{$2n$}
			\EndIf
			\If{$\mathit{edgel.coordinate.x} < end$}
			\Cost{$c_{26}$}{$3n$}
				\State $\mathit{end} \gets \mathit{edgel.coordinate.x}$
				\Cost{$c_{27}$}{$3n$}
				\State $\mathit{presentLine.end} \gets \mathit{edgel}$
				\Cost{$c_{28}$}{$2n$}
			\EndIf
			\State $\mathit{k} \gets k + 1$
			\Cost{$c_{30}$}{$n$}
		\EndFor
		\label{alg:findlinesegments6-newstart-end}
	\EndIf
	\algstore{brk-findlinesegments6-else}
\end{algorithmic}
\end{algorithm}


In \autoref{alg:findlinesegments7} wird nun geprüft, ob der Start- und Endpunkt vertauscht ist. Dazu wird der Winkel
 zwischen dem Liniensegment und seiner Orthogonalen gebildet. Wenn der Winkel kleiner als $0$ ist, werden Start- und
 Endpunkt in Zeile \ref{alg:findlinesegments7-newstart-start}--\ref{alg:findlinesegments7-newstart-end} getauscht. Im
 Anschluss daran, wird in Zeile \ref{alg:findlinesegments7-save-start}--\ref{alg:findlinesegments7-save-end} die
 Orientierung des Liniensegments berechnet und gespeichert. Danach wird das Liniensegment in Zeile
 \ref{alg:findlinesegments7-addtomempool} in den Speicherblock $L$ hinterlegt. Jetzt müssen alle Unterstützungsedgel
 des Liniensegments in Zeile \ref{alg:findlinesegments7-removeedgel-start}--\ref{alg:findlinesegments7-removeedgel-end}
 aus dem Speicherblock $E$ entfernt werden. Das entfernen der Unterstützungsedgel bewirkt, dass entweder das
 RANSAC-Verfahren wiederholt werden kann ohne die gleichen \gls{edgel} erneut zu betrachten, oder, wenn nicht mehr
 genügend \gls{edgel} in der Region vorhanden sind, das Verfahren abzubrechen und die Linienerkennung zu beenden.

\algblockdefx[DOWHILE]{DO}{WHILE}
   {\textbf{do}}
   [1]{\textbf{while} #1}
\begin{algorithm}[ht]
\caption{\textproc{findLineSegments} (Fortsetzung 2. Teil)}
\label{alg:findlinesegments7}
\begin{algorithmic}[1]
	\algrestore{brk-findlinesegments6-else}
	\State $\mathit{magnitude} \gets$ \Call{vectorSubtract}{$\mathit{presentLine.end.coordinate},\mathit{presentLine.start.coordinate}$}
	\State $\mathit{angle} \gets$ \Call{dotProduct}{$\mathit{magnitude},\mathit{orientation}$}
	\If{$\mathit{angle} < 0$}
		\State $\mathit{newEnd} \gets \mathit{presentLine.start}$
		\State $\mathit{presentLine.start} \gets \mathit{presentLine.end}$
		\State $\mathit{presentLine.end} \gets \mathit{newEnd}$
	\EndIf
	\State $\mathit{mathit} \gets$ \Call{vectorSubtract}{$\mathit{presentLine.end.coordinate},\mathit{presentLine.start.coordinate}$}
	\State \Call{normalized}{$\mathit{magnitude}$}
	\State $\mathit{presentLine.slope} \gets \mathit{magnitude}$
	\State \Call{addLineSegment}{$L,\mathit{presentLine}$}
	\State $\mathit{supportCount} \gets \mathit{presentLine.supportCount}$
	\For{$i \gets 0$ \textbf{0} $i < \mathit{supportCount}$}
		\State $\mathit{remove} \gets \mathit{presentLine.support}[i]$
		\State \Call{removeEdgel}{$E,\mathit{remove}$}
		\State $i \gets i + 1$
	\EndFor
\end{algorithmic}
\end{algorithm}


% subsection linienerkennung_nach_clarke96 (end)

\subsection{Linienerkennung nach \texorpdfstring{\citeauthor{hirzer08}}{Hirzer}} % (fold)
\label{sub:linienerkennung_nach_hirzer08}
Das Verfahren zur Linienerkennung nach \citeauthor{hirzer08} basiert, wie in \autoref{sub:line_detection} bereits
 erwähnt, auf dem Verfahren von \citeauthor{clarke96}. \citeauthor{hirzer08} verwendet in seinem Verfahren anstatt
 eines monochromen Bildsignals ein RGB-Signal. Dadurch ändern sich einige Algorithmen geringfügig und werden im
 folgenden Abschnitt erläutert.

Das Verfahren in \autoref{alg:linedetection-hirzer} unterscheidet sich von \autoref{alg:linedetection-clarke}, dass
 anstatt dem monochromen Signal $I_m$ das RGB-Signal $I$ verwendet wird. Dadurch verändern sich auch die Verfahren
 \textproc{findEdgels} und \textproc{findLineSegments}.

\begin{algorithm}[ht]
\caption{\textproc{lineDetection} nach Hirzer}
\label{alg:linedetection-hirzer}
	\begin{algorithmic}[1]
		\Require $I$
		\For{$y \gets 0$ \textbf{to} $y < \mathit{imageHeight}$}
			\For{$x \gets 0$ \textbf{to} $x < \mathit{imageWidth}$}
				\State \Call{findEdgels}{$I,E,x,y$}
				\State \Call{findLineSegments}{$E,L$}
				\State \ldots \Comment Speichern der Liniensegmente zur weiteren Verarbeitung
				\State \Call{resetMemoryPool}{$E$}
				\State \Call{resetMemoryPool}{$L$}
				\State $ x \gets x + 40$
			\EndFor
			\State $y \gets y + 40$
		\EndFor
	\end{algorithmic}
\end{algorithm}


In dem Verfahren zur Erstellung von \gls{edgel} in \autoref{alg:findedgelshirzer-1}--\autoref{alg:findedgelshirzer-2}
 wird im Gegensatz zu \autoref{alg:findedgels-horizontal} die Faltung für alle drei Farbkanäle durchgeführt (Vgl. Zeile
 \ref{alg:findedgelshirzer-1-color-start}--\ref{alg:findedgelshirzer-1-color-end} in \autoref{alg:findedgelshirzer-1}
 und Zeile \ref{alg:findedgelshirzer-2-color-start}--\ref{alg:findedgelshirzer-2-color-end} in
 \autoref{alg:findedgelshirzer-2}).

\begin{algorithm}[ht]
\caption{\textproc{findEdgels} (Hirzer)}
\label{alg:findedgelshirzer-1}
	\begin{algorithmic}[1]
		\Require $I, E,\mathit{left},\mathit{top}$
		\For{$y \gets \mathit{top}$ \textbf{to} $y < \mathit{top} + \mathit{regionSize}$}
			\State $\mathit{prev1} \gets 0$
			\State $\mathit{prev2} \gets 0$
			\For{$x \gets \mathit{left}$ \textbf{to} $x < \mathit{left} + \mathit{regionSize}$}
				\State $\mathit{currentEdgel} \gets$ \Call{convoluteKernelX}{$I,x,y,\mathit{blue},\mathit{imageWidth}, \mathit{imageHeight}$}
				\State $\mathit{red} \gets$ \Call{convoluteKernelX}{$I,x,y,\mathit{red},\mathit{imageWidth}, \mathit{imageHeight}$}
				\State $\mathit{green} \gets$ \Call{convoluteKernelX}{$I,x,y,\mathit{green},\mathit{imageWidth}, \mathit{imageHeight}$}
				\If{$\mathit{currentEdgel} > \mathit{threshold} \land \mathit{red} > \mathit{threshold} \land \mathit{green} > \mathit{threshold}$}
				\Comment Möglicherweise ein Edgel
				\Else
					\State $\mathit{currentEdgel} \gets 0$
				\EndIf
				\If{$\mathit{prev1} > 0 \land \mathit{prev2} > \mathit{prev1} \land \mathit{prev1} > \mathit{currentEdgel}$}
					\State $\mathit{edgel} \gets \infty$
					\State \Call{vectorSetCoordinate}{$\mathit{edgel},x - 1,y$}
					\State $\mathit{edgel.slope} \gets$ \Call{gradientIntensity}{$I_m, \mathit{imageWidth}, \mathit{imageHeight}, x - 1,y$}
					\State \Call{addEdgel}{$E,\mathit{edgel}$}
				\EndIf
				\State $\mathit{prev2} \gets \mathit{prev1}$
				\State $\mathit{prev1} \gets \mathit{currentEdgel}$
				\State $x \gets x + 1$
			\EndFor
			\State $y \gets y + 5$
		\EndFor
		\algstore{brk-findedgels-x}
	\end{algorithmic}
\end{algorithm}

\begin{algorithm}[!ht]
\caption{\textproc{findEdgels} (Fortsetzung)}
\label{alg:findedgelshirzer-2}
	\begin{algorithmic}[1]
		\algrestore{brk-findedgels-x}
		\For{$x \gets \mathit{left}$ \textbf{to} $x < \mathit{left} + \mathit{regionSize}$}
			\State $\mathit{prev1} \gets 0$
			\State $\mathit{prev2} \gets 0$
			\For{$y \gets \mathit{top}$ \textbf{to} $y < \mathit{top} + \mathit{regionSize}$}
				\State $\mathit{currentEdgel} \gets$ \Call{convoluteKernelY}{$I,x,y,\mathit{blue},\mathit{imageWidth},
				 \mathit{imageHeight}$}
				\label{alg:findedgelshirzer-2-color-start}
				\State $\mathit{red} \gets$ \Call{convoluteKernelY}{$I,x,y,\mathit{red},\mathit{imageWidth},
				 \mathit{imageHeight}$}
				\State $\mathit{green} \gets$ \Call{convoluteKernelY}{$I,x,y,\mathit{green},\mathit{imageWidth},
				 \mathit{imageHeight}$}
				\label{alg:findedgelshirzer-2-color-end}
				\If{$\mathit{currentEdgel} > \mathit{threshold} \land \mathit{red} > \mathit{threshold} \land
				 \mathit{green} > \mathit{threshold}$}
				\Comment Edgel gefunden
				\Else
					\State $\mathit{currentEdgel} \gets 0$
				\EndIf
				\If{$\mathit{prev1} > 0 \land \mathit{prev2} > \mathit{prev1} \land \mathit{prev1} >
				 \mathit{currentEdgel}$}
				\State $\mathit{edgel} \gets \infty$
				\State \Call{vectorSetCoordinate}{$\mathit{edgel},x,y - 1$}
				\State $\mathit{edgel.slope} \gets$ \Call{gradientIntensity}{$I_m, \mathit{imageWidth},
				 \mathit{imageHeight}, x,y - 1$}
				\State \Call{addEdgel}{$E,\mathit{edgel}$}
				\EndIf
				\State $\mathit{prev2} \gets \mathit{prev1}$
				\State $\mathit{prev1} \gets \mathit{currentEdgel}$
				\State $y \gets y + 1$
			\EndFor
			\State $y \gets y + 5$
		\EndFor
	\end{algorithmic}
\end{algorithm}


Betrachtet man die Methode \textproc{convoluteKernelX} und \textproc{convoluteKernelY} stellt man fest, dass hier eine
 Angabe zum Farbkanal erfolgt. In \autoref{alg:convolutekernelhirzer-horizontal} und
 \autoref{alg:convolutekernelhirzer-vertical} wird durch den Aufruf der Methode \textproc{getRGBValue} ein bestimmter
 Farbkanal betrachtet.

\input{alg/analyse-hirzer/convolutekernelhirzer}

Auch das Verfahren \textproc{gradientIntensity} in \autoref{alg:gradientintensityhirzer} verwendet die Methode
 \textproc{getRGBValue}.

\input{alg/analyse-hirzer/gradientIntensityhirzer}

\textproc{getRGBValue} dient dem auslesen eines \gls{pixel} für einen angegeben Farbkanal. In \autoref{alg:getrgbvalue} ist das Verfahren beschrieben.

\begin{algorithm}[!ht]
\caption{\textproc{getRGBValue}}
\label{alg:getrgbvalue}
\begin{algorithmic}[1]
	\Require $I,x,y,\mathit{color},w,h$
	\If{$x < 0$}
	\Cost{$c_{1}$}{$1$}
	\label{alg:getrgbvalue-sanity-start}
		\State $x \gets 0$
		\Cost{$c_{2}$}{$1$}
	\EndIf
	\If{$y < 0$}
	\Cost{$c_{4}$}{$1$}
		\State $y \gets 0$
		\Cost{$c_{5}$}{$1$}
	\EndIf
	\If{$x \geq w$}
	\Cost{$c_{7}$}{$1$}
		\State $x \gets w - 1$
		\Cost{$c_{8}$}{$2$}
	\EndIf
	\If{$y \geq h$}
	\Cost{$c_{10}$}{$1$}
		\State $y \gets h -1$
		\Cost{$c_{11}$}{$2$}
	\EndIf
	\label{alg:getrgbvalue-sanity-end}
	\State $\mathit{offset} \gets (x + (y \cdot w)) \cdot \left(\textproc{sizeof}(\mathit{char}) \cdot 4\right)$
	\Cost{$c_{13}$}{$6$}
	\label{alg:getrgbvalue-offset}
	\State $\mathit{address} \gets I + \mathit{offset}$
	\Cost{$c_{14}$}{$2$}
	\label{alg:getrgbvalue-address}
	\State \textbf{return} $\mathit{address} + \mathit{color}$
	\Cost{$c_{13}$}{$2$}
	\label{alg:getrgbvalue-returncolor}
\end{algorithmic}
\end{algorithm}


Das Verfahren benötigt das RGB-Signal $I$, die Position des \gls{pixel} ($x$, $y$), den Farbkanal, sowie die Breite $w$
 und Höhe $h$ von $I$. In den Zeilen \ref{alg:getrgbvalue-sanity-start}--\ref{alg:getrgbvalue-sanity-end} wird
 sichergestellt, dass kein \gls{pixel} ausserhalb der Bildgrenzen gelesen werden können. In Zeile
 \ref{alg:getrgbvalue-offset} wird der Adressabstand für ein \gls{pixel} berechnet. Die Adresse des \gls{pixel} wird in
 Zeile \ref{alg:getrgbvalue-address} aus der Adresse von $I$ und dem Adressabstand berechnet. Zuletzt wird der Wert des
 \gls{pixel} in Zeile \ref{alg:getrgbvalue-returncolor} für den angegeben Farbkanal zurückgegeben.
% subsection linienerkennung_nach_hirzer08 (end)

\subsection{Line Extension} % (fold)
\label{sub:analyse_line_extension}
$\mathit{extendLinesInPool}$ erweitert, wie in \autoref{sub:line_extension} beschrieben, die Linien am Anfang und am
 Ende. Das Verfahren ist in \autoref{alg:extendlinesinpool} dargestellt und benötigt als Parameter den Speicherblock
 $L$ mit Linien.
\begin{algorithm}[!ht]\small
\caption{\textproc{extendLinesInPool}}
\label{alg:extendlinesinpool}
\begin{algorithmic}[1]
	\Require $L$
	\State $\mathit{lineCount} \gets$ \Call{getLineCount}{$L$}
	\Cost{$c_{1}$}{$1 + \Theta(1)$}
	\For{$i \gets 0$ \textbf{to} $i < \mathit{lineCount}$}
	\Cost{$c_{2}$}{$(l + 1)$}
	\label{alg:extendlinesinpool-loop-start}
		\State $l \gets \mathit{L.data}[i]$
		\Cost{$c_{3}$}{$3l$}
		\label{alg:extendlinesinpool-line}
		\State $\mathit{slope} \gets \mathit{l.slope}$
		\Cost{$c_{4}$}{$2l$}
		\label{alg:extendlinesinpool-slope}
		\State \Call{extendLine}{$\mathit{l.end.coordinate}, \mathit{slope}, \mathit{l.end.slope},
		 \mathit{l.end.coordinate},999,I$}
		\Cost{$c_{5}$}{$l\cdot\Theta(999)$}
		\State $\mathit{slope.x} \gets \mathit{slope.x} \cdot - 1$
		\Cost{$c_{6}$}{$4l$}
		\label{alg:extendlinesinpool-slopex-invert}
		\State $\mathit{slope.y} \gets \mathit{slope.y} \cdot - 1$
		\Cost{$c_{7}$}{$4l$}
		\label{alg:extendlinesinpool-slopey-invert}
		\State \Call{extendLine}{$\mathit{l.start.coordinate}, \mathit{slope}, \mathit{l.end.slope},
		 \mathit{l.start.coordinate},999,I$}
		\Cost{$c_{8}$}{$l\cdot\Theta(999)$}
		\label{alg:extendlinesinpool-extend-start}
		\State $i \gets i + 1$
		\Cost{$c_{9}$}{$l$}
	\EndFor
	\label{alg:extendlinesinpool-loop-end}
\end{algorithmic}
\end{algorithm}

In der Schleife in Zeile \ref{alg:extendlinesinpool-loop-start}--\ref{alg:extendlinesinpool-loop-end} wird jede Linie im
 Speicherblock $L$ erweitert. Dazu wird zuerst in Zeile \ref{alg:extendlinesinpool-line} und
 \ref{alg:extendlinesinpool-slope} die Linie an Position $i$ ausgewählt und ihre Richtung ausgelesen. Danach wird durch
 \textproc{extendLine} die Linie erweitert. Die Methode \textproc{extendLine} wurde in
 \autoref{sub:linienerkennung_nach_hirzer08} beschrieben. Der Parameter $\mathit{maxLength}$ wird mit einem großen Wert
 verwendet, um die Linie soweit wie möglich zu erweitern. Die Wachstumsrate $\Theta(\mathit{length})$ von
 \textproc{extendLine} wird in diesem Verfahren mit dem konstanten Wert $\mathit{maxLength} = 999$ verwendet. Danach
 wird die Orientierung der Linie in Zeile
 \ref{alg:extendlinesinpool-slopex-invert}--\ref{alg:extendlinesinpool-slopey-invert} umgekehrt und in Zeile
 \ref{alg:extendlinesinpool-extend-start} der Anfang der Linie erweitert. Die Laufzeitfunktion des Verfahrens ist für
 den schlechtesten Fall in \autoref{eq:extendlinesinpool-1} angegeben. Die Wachstumsrate ist $2013l + 3 = \Theta(l)$,
 für $c_{1} = 2012$, $c_{2} = 2013$ und $l_{0} = 1$.
\begin{subequations}
\label{eq:extendlinesinpool}
\begin{align}
\label{eq:extendlinesinpool-1}
T_{worst}(l)& =
c_{1}\bigl(1 + \Theta(1)\bigr)
+ c_{2}(l+1)
+ c_{3}3l
+ c_{4}2l
+ c_{5}\bigl(l \cdot \Theta(999)\bigr)
\\
& \quad
+ c_{6}4l
+ c_{7}4l
+ c_{8}\bigl(l \cdot \Theta(999)\bigr)
+ c_{9}l
\nonumber \\
\label{eq:extendlinesinpool-2}
T_{worst}(l)& =
\Bigl(c_{1}\bigl(1 + \Theta(1)\bigr) + c_{2}\Bigr)
+ l(c_{2} + c_{3}3 + c_{4}2 + c_{6}4 + c_{7}4 + c_{9})
\\
& \quad
+ l\bigl(c_{5}\Theta(999) + c_{8}\Theta(999)\bigr)
\nonumber
\end{align}
\end{subequations}


Ob eine erweiterte Linie zur Markenerkennung geeignet ist, wird dadurch bestimmt, ob über das Linienende hinaus ein
 heller Pixel liegt. Dazu wird das Verfahren $\mathit{findLinesWithCornersInLinePool}$ verwedent
 (\autoref{alg:findlineswithcornersinlinepool1}--\autoref{alg:findlineswithcornersinlinepool2}).

\begin{algorithm}[!ht]
\caption{\textproc{findLinesWithCornersInLinePool}}
\label{alg:findlineswithcornersinlinepool1}
\begin{algorithmic}[1]
	\Require $L,C$
	\State \Call{resetMemoryPool}{$C$}
	\label{alg:findlineswithcornersinlinepool1-clear}
	\State $w \gets \mathit{imageWidth}$ 
	\State $h \gets \mathit{imageHeight}$
	\State $n \gets$ \Call{getLineCount}{L}
	\For{$i \gets 0$ \textbf{to} $i < n$}
	\label{alg:findlineswithcornersinlinepool1-loop-start}
		\State $l \gets \mathit{pool.data}[i]$
		\label{alg:findlineswithcornersinlinepool1-line}
		\State $\mathit{dx} \gets l\mathit{.slope.x} \cdot 4$
		\label{alg:findlineswithcornersinlinepool1-dx}
		\State $\mathit{dy} \gets l\mathit{.slope.y} \cdot 4$
		\label{alg:findlineswithcornersinlinepool1-dy}
		\State $x \gets l\mathit{.start.coordinate.x} - dx$
		\label{alg:findlineswithcornersinlinepool1-x}
		\State $y \gets l\mathit{.start.coordinate.y} - dy$
		\label{alg:findlineswithcornersinlinepool1-y}
		\State $r \gets$ \Call{getRGBValue}{$I,x,y,red,w,h$}
		\State $g \gets$ \Call{getRGBValue}{$I,x,y,green,w,h$}
		\State $b \gets$ \Call{getRGBValue}{$I,x,y,blue,w,h$}
	\algstore{brk-findlineswithcornersinlinepool1}
\end{algorithmic}
\end{algorithm}

\begin{algorithm}[!ht]\small
\caption{\textproc{findLinesWithCornersInLinePool} (Fortsetzung)}
\label{alg:findlineswithcornersinlinepool2}
\begin{algorithmic}[1]
	\algrestore{brk-findlineswithcornersinlinepool1}
		\If{$r > 10 \land g > 10 \land b > 10$}
		\Cost{$c_{14}$}{$5l$}
		\label{alg:findlineswithcornersinlinepool2-iswhite-start}
			\State $l\mathit{.startCorner} \gets$ \textbf{true}
			\Cost{$c_{15}$}{$2l$}
		\EndIf
		\State $x \gets l\mathit{.end.coordinate.x} + dx$
		\Cost{$c_{17}$}{$5l$}
		\label{alg:findlineswithcornersinlinepool2-lineend-start}
		\State $y \gets l\mathit{.end.coordinate.y} + dy$
		\Cost{$c_{18}$}{$5l$}
		\State $r \gets$ \Call{getRGBValue}{$I,x,y,red,w,h$}
		\Cost{$c_{19}$}{$l\bigl(1 + \Theta(1)\bigr)$}
		\State $g \gets$ \Call{getRGBValue}{$I,x,y,green,w,h$}
		\Cost{$c_{20}$}{$l\bigl(1 + \Theta(1)\bigr)$}
		\State $b \gets$ \Call{getRGBValue}{$I,x,y,blue,w,h$}
		\Cost{$c_{21}$}{$l\bigl(1 + \Theta(1)\bigr)$}
		\label{alg:findlineswithcornersinlinepool2-lineend-end}
		\If{$r > 10 \land g > 10 \land b > 10$}
		\Cost{$c_{22}$}{$5l$}
		\label{alg:findlineswithcornersinlinepool2-iswhite-end}
			\State $l\mathit{.endCorner} \gets$ \textbf{true}
			\Cost{$c_{23}$}{$2l$}
		\EndIf
		\If{$l\mathit{.startCorner} \lor l\mathit{.endCorner}$}
		\Cost{$c_{25}$}{$3l$}
		\label{alg:findlineswithcornersinlinepool2-hascorner}
			\State \Call{addLineSegment}{$C,l$}
			\Cost{$c_{26}$}{$l\bigl(\Theta(1)\bigr)$}
			\label{alg:findlineswithcornersinlinepool2-addline}
		\EndIf
		\State $i \gets i + 1$
		\Cost{$c_{28}$}{$l$}
	\EndFor
	\label{alg:findlineswithcornersinlinepool2-loop-end}
\end{algorithmic}
\end{algorithm}


Das Verfahren untersucht alle Linien in Speicherblock $L$ und speichert Linien, die sich zur Erkennung eignen, in Block
 $C$. Zu Beginn des Verfahrens wird in Zeile \ref{alg:findlineswithcornersinlinepool1-clear} der Speicherblock $C$
 gelöscht. Danach werden die lokalen Variablen initialisiert. Die Bildbreite und -höhe wird in Variable $w$ und $h$
 hinterlegt. Die Anzahl der Linien in $L$ wird in Variable $n$ gespeichert. In der Schleife in Zeile
 \ref{alg:findlineswithcornersinlinepool1-loop-start}--\ref{alg:findlineswithcornersinlinepool2-loop-end} wird jede
 Linie untersucht, indem in Zeile \ref{alg:findlineswithcornersinlinepool1-line} eine Linie an Position $i$ zuerst in
 $l$ gespeichert wird. Danach wird die Richtung der Linie in Zeile \ref{alg:findlineswithcornersinlinepool1-dx} und
 \ref{alg:findlineswithcornersinlinepool1-dy} verlängert. Die Variable $dx$ und $dy$ sind dienen als Abstand der
 Linienenden. In Zeile \ref{alg:findlineswithcornersinlinepool1-x}--\ref{alg:findlineswithcornersinlinepool1-y} wird
 die Position vor dem Startpunkt der Linie berechent. Danach werden die Farbkomponenten an dieser Position ausgelesen.

In Zeile \ref{alg:findlineswithcornersinlinepool2-iswhite-start} wird dann jeder Farbwert mit einem Schwellwert
 verglichen, um festzustellen ob an der Position ein heller gls{pixel} vorliegt. Falls dem so ist, eignet sich die
 Startposition zur Erkennung einer Linie. In Zeile
 \ref{alg:findlineswithcornersinlinepool2-lineend-start}--\ref{alg:findlineswithcornersinlinepool2-lineend-end} wird
 dann für das Linienende die Koordinate und die Farbkomponenten an dieser Position berechnet. In Zeile
 \ref{alg:findlineswithcornersinlinepool2-iswhite-end} werden dann die Farbwerte am Linienende untersucht. Zuletzt wird
 in Zeile \ref{alg:findlineswithcornersinlinepool2-hascorner} untersucht, ob der Startpunkt oder der Endpunkt zur
 Markenerkennung geeignet ist. Wenn die Untersuchung positiv ausfällt, wird in Zeile
 \ref{alg:findlineswithcornersinlinepool2-addline} die Linie $l$ in den Speicherblock $C$ hinterlegt.

Das Verfahren der Linienerweiterung wird in \textproc{lineDetection} integriert und ist in
 \autoref{alg:linedetection-hirzerextending} dargestellt.

\begin{algorithm}[!ht]\small
\caption{\textproc{lineDetection} mit Linienerweiterung}
\label{alg:linedetection-hirzerextending}
	\begin{algorithmic}[1]
		\Require $I$
		\For{$y \gets 0$ \textbf{to} $y < \mathit{imageHeight}$}
		\Cost{$c_{1}$}{$(\frac{h}{r} + 1)$}
			\For{$x \gets 0$ \textbf{to} $x < \mathit{imageWidth}$}
			\Cost{$c_{2}$}{$\frac{h}{r}(\frac{w}{r} + 1)$}
				\State \Call{findEdgels}{$I,E,x,y$}
				\Cost{$c_{3}$}{$\frac{hw}{r^2}\cdot\Theta(r^2)$}
				\State \Call{findLineSegments}{$E,L$}
				\Cost{$c_{4}$}{$\frac{hw}{r^2}\cdot\Theta(mn^2)$}
				\State \Call{mergeLines}{$L$}
				\Cost{$c_{5}$}{$\frac{hw}{r^2}\cdot\Theta(l^2\cdot\mathit{length})$}
				\State $n \gets$ \Call{getLineCount}{$L$}
				\Cost{$c_{6}$}{$\frac{hw}{r^2}\bigl(1 + \Theta(1)\bigr)$}
				\For{$i \gets 0$ \textbf{to} $i < n$}
				\Cost{$c_{7}$}{$\frac{hw}{r^2}(l + 1)$}
					\State \Call{addLineSegment}{$M,$\textproc{getLineSegment}$(L,i)$}
					\Cost{$c_{8}$}{$\frac{hw}{r^2}l\bigl(\Theta(1) + \Theta(1)\bigr)$}
					\State $i \gets i + 1$
					\Cost{$c_{9}$}{$\frac{hw}{r^2}l$}
				\EndFor
				\State \Call{resetMemoryPool}{$E$}
				\Cost{$c_{11}$}{$\frac{hw}{r^2}\cdot\Theta(1)$}
				\State \Call{resetMemoryPool}{$L$}
				\Cost{$c_{12}$}{$\frac{hw}{r^2}\cdot\Theta(1)$}
				\State $ x \gets x + 40$
				\Cost{$c_{13}$}{$2\frac{hw}{r^2}$}
			\EndFor
			\State $y \gets y + 40$
			\Cost{$c_{15}$}{$2\frac{h}{r}$}
		\EndFor
		\State \Call{mergeLines}{$M$}
		\Cost{$c_{17}$}{$\Theta(l^2\cdot\mathit{length})$}
		\State \Call{extendLinesInPool}{$M$}
		\Cost{$c_{18}$}{$\Theta(l)$}
		\label{alg:linedetection-hirzerextending-extend}
		\State \Call{findLinesWithCornersInPool}{$M,L$}
		\Cost{$c_{19}$}{$\Theta(l)$}
		\label{alg:linedetection-hirzerextending-corner}
	\end{algorithmic}
\end{algorithm}


In Zeile \ref{alg:linedetection-hirzerextending-extend} werden zuerst die Linien in $M$ erweitert. Im Anschluß daran
 werden in Zeile \ref{alg:linedetection-hirzerextending-corner} die Linien auf ihre Tauglichkeit zur Markenerkennung
 untersucht. Alle Linien die zur Erkennung geeignet sind, werden in Speicherblock $L$ gespeichert.

% subsection line_extension (end)

\subsection{Quadrangle Detection} % (fold)
\label{sub:analyse_quadrangle_detection}
Bei dem letzten Schritt im Verfahren von \citeauthor{hirzer08}, der Quadrangle Detection, werden die gefundenen Linien
 untersucht, um daraus eine quadratische Marke zu erkennen. Dazu wird das Verfahren \textproc{findChainsOfLines}
 verwendet, dass in \autoref{alg:findchainoflines1}--\autoref{alg:findchainoflines4} dargestellt ist.

\begin{algorithm}[!ht]
\caption{\textproc{findChainOfLines}}
\label{alg:findchainoflines1}
\begin{algorithmic}[1]
	\Require $B,\mathit{start},\mathit{fromStart},\mathit{chainLength},C$
	\If{$\mathit{fromStart}$}
	\Cost{$c_{1}$}{$1$}
	\label{alg:findchainoflines1-fromstart}
		\State $\mathit{startPoint} \gets \mathit{start.start.coordinate}$
		\Cost{$c_{2}$}{$3$}
	\Else
		\State $\mathit{startPoint} \gets \mathit{start.end.coordinate}$
		\Cost{$c_{4}$}{$3$}
	\EndIf
	\label{alg:findchainoflines1-fromend}
	\For{$i \gets 0$ \textbf{to} $i < \mathit{B.count}$}
	\Cost{$c_{6}$}{$2(n + 1)$}
	\label{alg:findchainoflines1-loop-start}
		\If{\Call{isOrientationCompatible}{$\mathit{start},\mathit{B.data}[i]$}}
		\Cost{$c_{7}$}{$n\bigl(2 + \Theta(1)\bigr)$}
		\label{alg:findchainoflines1-iscompatible}
			\State \textbf{continue}
			\Cost{$c_{8}$}{$n$}
			\label{alg:findchainoflines1-parallel}
		\EndIf
		\algstore{brk-findchainoflines1}
\end{algorithmic}
\end{algorithm}


Das Verfahren benötigt einen Speicherblock $B$ mit Linien, eine Startlinie $\mathit{start}$, sowie eine boolsche
 Aussage $\mathit{fromStart}$, ob der Anfang der Linie oder das Ende der Linie untersucht werden soll. Zusätzlich wird
 die zu untersuchende Länge einer Linie benötigt und ein Speicherblock $C$, in dem eine Linienkette gespeichert wird.

In Zeile \ref{alg:findchainoflines1-fromstart}--\ref{alg:findchainoflines1-fromend} wird der Startpunkt der zu
 untersuchenden Linie festgelegt und ist von der boolschen Variable $\mathit{fromStart}$ abhängig. Danach wird in Zeile
 \ref{alg:findchainoflines1-loop-start}--\ref{alg:findchainoflines4-loop-end} jede Linie aus $B$ untersucht. In Zeile
 \ref{alg:findchainoflines1-iscompatible} wird überprüft, ob Linie $\mathit{start}$ und die Linie an Position $i$ des
 Speicherblocks fast parallel zueinander stehen. Falls dem so ist wird in Zeile \ref{alg:findchainoflines1-parallel}
 die Untersuchung der nächsten Linie eingeleitet. Nur wenn die Linien nicht paralle ausgerichtet sind, wird das
 Verfahren in \autoref{alg:findchainoflines2} fortgesetzt.

\begin{algorithm}[ht]
\caption{\textproc{findChainOfLines} (Fortsetzung)}
\label{alg:findchainoflines2}
\begin{algorithmic}[1]
	\algrestore{brk-findchainoflines1}
		\If{$\mathit{fromStart}$}
			\State $\mathit{endpoint} \gets L\mathit{\to data}[i]\mathit{.end.coordinate}$
		\Else
			\State $\mathit{endpoint} \gets L\mathit{\to data}[i]\mathit{.start.coordinate}$
		\EndIf
		\State $\mathit{distance} \gets$ \Call{vectorSubtract}{$\mathit{startPoint},\mathit{endpoint}$}
		\State $\mathit{squaredLength} \gets$ \Call{getSquaredLength}{$\mathit{distance}$}
		\If{$\mathit{squaredLength} > 16$}
			\State \textbf{continue}
		\EndIf
	\algstore{brk-findchainoflines2}
\end{algorithmic}
\end{algorithm}


In Zeile \ref{alg:findchainoflines2-endpointend}--\ref{alg:findchainoflines2-endpointstart} wird nun der Endpunkt der
 Linie $i$ festgelegt. Wie beim Startpunkt auch, ist der Endpunkt von der Variablen $\mathit{fromStart}$ abhängig. In
 Zeile \ref{alg:findchainoflines2-distance} und Zeile \ref{alg:findchainoflines2-length} wird dann der Abstand der
 Linienenden von $\mathit{start}$ und $i$ berechnet. Wenn der Abstand zu groß ist
 (Zeile \ref{alg:findchainoflines2-toofar}), wird mit der nächsten Linie die Untersuchung wiederholt. Andernfalls wird
 das Verfahren in \autoref{alg:findchainoflines3} fortgesetzt.

\begin{algorithm}[ht]
\caption{\textproc{findChainOfLines} (Fortsetzung)}
\label{alg:findchainoflines3}
\begin{algorithmic}[1]
	\algrestore{brk-findchainoflines2}
		\State $\mathit{test} \gets (\mathit{start \to slope.x} \cdot L\mathit{\to data}[i]\mathit{.slope.y})$
		\State $\mathit{test} \gets \mathit{test} - (\mathit{start \to slope.y} \cdot L\mathit{\to data}[i]\mathit{.slope.x})$
		\If{$(\mathit{fromStart} \land \mathit{test} \leq 0) \lor (\lnot \mathit{fromStart} \land \mathit{test} \geq 0)$}
			\State \textbf{continue}
		\EndIf
		\State $\mathit{chainLength} \gets \mathit{chainLength} + 1$
		\State $\mathit{chainSegment} \gets L\mathit{\to data}[i]$
		\State \Call{removeLine}{$L,i$}
		\If{$\mathit{chainLength} == 4$}
			\State \Call{addLine}{$C,\mathit{chainSegment}$}
			\State \textbf{return}
		\EndIf
	\algstore{brk-findchainoflines3}
\end{algorithmic}
\end{algorithm}


In Zeile \ref{alg:findchainoflines3-vectorproduct-start}--\ref{alg:findchainoflines3-vectorproduct-end} wird das
 Kreuzprodukt der Linienorientierung berechnet um danach in Zeile \ref{alg:findchainoflines3-checkorientation} die
 Orientierung zu überprüfen. Je nachdem ob $\mathit{fromStart}$ gesetzt ist oder nicht, wird die Orientierung
 unterschiedlich betrachtet. Wenn $\mathit{fromStart}$ wahr ist, muss die Orientierung der beiden Linien kleiner oder
 gleich $0$ sein. Damit wird die Orientierung als negativer Winkel betrachtet und der Wertebereich entspricht
 $\left[-1,0\right]$. Andernfalls, wenn $\mathit{fromStart}$ falsch ist, muss die Orientierung größer oder gleich $0$
 sein. In diesem Fall wird die Orientierung als positiver Winkel betrachtet und der Wertebereich entspricht
 $\left[0,1\right]$. Werden diese Bedingungen erfüllt, eignet sich diese Linienkombination nicht zur Markenerkennung
 und es wird eine andere Linie untersucht. Falls die Bedingungen nicht erfüllt werden, eigent sich diese
 Linienkombination zur Erkennung einer Marke. In diesem Fall wird in Zeile \ref{alg:findchainoflines3-incchain} die
 Anzahl der Linienkette erhöht und die Linie an Position $i$ in der Variable $\mathit{chainSegment}$ hinterlegt. In
 Zeile \ref{alg:findchainoflines3-removeline} wird danach die Linie an Position $i$ aus dem Speicherblock $B$ entfernt.
 Danach wird in Zeile \ref{alg:findchainoflines3-has4lines} die Anzahl der Linienketten untersucht. Wenn genau vier
 Linienketten vorhanden sind, wird in Zeile \ref{alg:findchainoflines3-saveline} die Linie in $\mathit{chainSegment}$
 zum Speicherblock $C$ hinzugefügt und danach das Verfahren beendet. Anonsten wird in Zeile
 \ref{alg:findchainoflines4-savelineifnotfromstart} (\autoref{alg:findchainoflines4}) die Linie zum Speicherblock
 hinzugefügt, wenn $\mathit{fromStart}$ falsch ist.

\begin{algorithm}[ht]
\caption{\textproc{findChainOfLines} (Fortsetzung)}
\label{alg:findchainoflines4}
\begin{algorithmic}[1]
	\algrestore{brk-findchainoflines3}
		\If{$\lnot \mathit{fromStart}$}
			\State \Call{addLine}{$C,chainSegment$}
		\EndIf
		\State \Call{findChainOfLines}{$B,\mathit{chainSegment},\mathit{fromStart},\mathit{chainLength},C$}
		\If{$\mathit{fromStart}$}
			\State \Call{addLine}{$C,\mathit{chainSegment}$}
		\EndIf
		\State \textbf{return}
		\State $i \gets i + 1$
	\EndFor
\end{algorithmic}
\end{algorithm}


Danach wird in Zeile \ref{alg:findchainoflines4-callmethod} $\mathit{findChainsOfLines}$ rekursiv aufgerufen, um
 weitere passende Linine zur Markenerkennung zu identifizieren. Abschliessend wird in Zeile
 \ref{alg:findchainoflines4-isfromstart} überprüft, ob $\mathit{fromStart}$ wahr ist, was dazu führt, dass die Linie in
 $\mathit{chainSegment}$ dem Speicherblock $C$ hinzugefügt wird. Danach wird die Ausführung abgebrochen.

% section quadrangle_detection (end)

% section hirzer (end)

\chapter{Fazit} % (fold)
\label{cha:fazit}

Wie in \autoref{cha:ergebnisse} gezeigt, erfüllen beide Verfahren die Echtzeitbedingung von \citeauthor{azuma97}. Die
 Verwendung der Verfahren unter iOS auf einem iPod touch ist durch die experimentelle Analyse bewiesen worden.

Wie die Analyse in \autoref{cha:analyse} gezeigt hat, sind beide Verfahren von der Eingabemenge des Bildsignals
 abhängig. Da ARToolKitPlus und das Verfahren von \citeauthor{hirzer08} unterschiedliche Eingabemengen zur Erkennung
 einer Marke verwenden, ist ein Vergleich mit dem asymptotischen Wachstum nur bedingt möglich. Die von beiden Verfahren
 verwendeten Optimierungsschritte, um nicht alle \gls{pixel} in einem Bildsignal verarbeiten zu müssen, werden durch
 die Natur der asymptotischen Analyse nicht berücksichtigt. Die Leistungsfähigkeit der vorgestellten Tracking Verfahren
 wurde erst durch eine experimentelle Analyse deutlich.

Die Stärke von ARToolKitPlus ist die Verarbeitungsgeschwindigkeit eines Bildsignals. Diesen Vorteil gegenüber dem
 Verfahren von \citeauthor{hirzer08} erlangt ARToolKitPlus auf Kosten der Erkennungsrate. Da ARToolKitPlus eine Marke
 durch eine Region in einem Binärbild erkennt, wird eine Marke schon bei geringer Verdeckung nicht mehr erkannt. Das
 Problem wird in \autoref{fig:artoolkit-failure} an einem Beispiel verdeutlicht.
\begin{figure}[!ht]
	\centering
	\subfigure[Erfolgreiche Erkennung]{
		\includegraphics[width=.35\textwidth]{resources/ARToolKitPlus1.pdf}
		% \label{fig:}
	}
	\subfigure[Fehlgeschlagene Erkennung]{
		\includegraphics[width=.35\textwidth]{resources/ARToolKitPlus2.pdf}
		% \label{fig:}
	}
	\caption{Fehlgeschlagene Erkennung einer Marke bei geringfügiger Verdeckung durch ARToolKitPlus. Aus
	 \cite{fiala2004artagb}}
	\label{fig:artoolkit-failure}
\end{figure}

Die Schwäche von ARToolKitPlus wird durch die Betrachtung von Liniensegmenten im Verfahren von \citeauthor{hirzer08}
 vermieden. Auch wenn beispielsweise eine Ecke einer Marke verdeckt ist, erkennt das Verfahren eine Marke, wie in
 \autoref{fig:arvideo} dargestellt.
\begin{figure}[!ht]
	\centering
	\includegraphics[width=.35\textwidth]{resources/ARVideo2.png}
	\caption{Erkennung einer verdeckten Marke durch das Verfahren von \citeauthor{hirzer08}.}
	\label{fig:arvideo}
\end{figure}
Der Schwachpunkt des Verfahrens nach \citeauthor{hirzer08} tritt bei vielen dunklen Objekten auf, die auf einem hellen
 Untergrund liegen. Ein Schachbrettmuster beeinflusst die Verarbeitunszeit des Verfahrens erheblich durch die hohe
 Anzahl von Edgeln und Linien.

Beide Verfahren erreichen unter guten Lichtverhältnissen mit wenigen Störelemente gute Resultate. Sollen die Verfahren
 jedoch in Anwendungen verwendet werden, die unter Umständen auch bei schlechten Lichtverhältnissen eingesetzt werden,
 kann nicht die Geschwindigkeit der Verfahren als alleiniges Kriterium verwendet werden. Welches Verfahren eingesetzt
 wird, muss sorgfältig evaluiert werden, da beide Systeme Schwächen besitzen, die in einer praktischen Anwendung zu
 Problemen führen kann.

% chapter fazit (end)
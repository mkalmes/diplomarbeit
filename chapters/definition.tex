\section{Definition} % (fold)
\label{sec:definition}
\begin{comment}
	Definiere Begriffe der Augmented Reality und Bildverarbeitung, die dem Leser nicht geläufig sind. Denke dabei an Prof. Klocke als Leser ohne besonderen Kenntnisstand in der AR/Bildverarbeitung.

	Azuma Definition der AR
	Milgram + Kishino MR oder nur Milgram Definition MR
\end{comment}

Bis heute ist \gls{AR} nicht eindeutig definiert, obwohl die ersten Untersuchung von \citeauthor{sutherland} bereits
 1968 vorgestellt wurden\footcite{sutherland}. \citeauthor{milgram94b} beschrieben in
 \citetitle{milgram94b}\footcite{milgram94b} \gls{AR} als einen Punkt des Reality-Virtuality (RV) Continuum
 (Vgl. \autoref{fig:mixedreality}), welches reale und virtuelle Objekte gemeinsam auf einem Bildschirm darstellt.
\begin{figure}[!ht]
	\centering
	\input{resources/mixedreality.pdf_tex}
	\caption{Reality-Virtual Continuum nach \citeauthor{milgram94b}}
	\label{fig:mixedreality}
\end{figure}

Nach der neueren Definition von \citeauthor{azuma97}\footcite{azuma97} muss ein \gls{AR}-System drei Kriterien
 erfüllen:
\begin{itemize}
	\item Reale und virtuelle Objekte werden kombiniert
	\item Bezug von realen und virtuellen Objekten im 3-dimensionalen Raum
	\item Interaktion in Echtzeit
\end{itemize}
Das erste Kriterium bedeutet, dass, im Gegensatz zu Computerspielen, virtuelle Gegenstände mit der realen Welt in
 Verbindung stehen. Durch die zweite Bedinung muss ein virtuelles Objekt in einem räumlichen Zusammenhang stehen. Als
 Beispiel muss eine virtuelle Tasse auf dem realen Schreibtisch plaziert sein und nicht im Raum schweben. Zuletzt muss
 die Interaktion mit einem \gls{AR}-System in Echtzeit erfolgen. Dies ist eine Abgrenzung zu einer Computeranimation in
 einem Film.

Nach \citeauthor{moeller2008}\footcite[Vgl.][S.~1]{moeller2008} ist Interaktion in Echtzeit mit $15$ FPS erfüllt, wobei
 eine Geschwindigkeit von $6$ FPS noch als interaktiv gilt. Diese Definition wird von \citeauthor{wagner09b} in ihrer
 Untersuchung\footcite[Vgl.][S.~8--9]{wagner09b} unterstützt.

\subsection{Tracking Verfahren} % (fold)
\label{sec:tracking_verfahren}
Tracking Verfahren bezeichnet bei \gls{AR}-Systemen die Fähigkeit, die reale Position eines Benutzers oder eines
 Displays in Relation zur Position einer \gls{AR}-Umgebung zu setzen. Tracking Verfahren müssen, wie in der Definition
 von \citeauthor{azuma97} schon erwähnt, in Echtzeit agieren. Zusätzlich müssen Tracking Verfahren, je nach
 Anwendungsgebiet, unter unterschiedlichen Lichtverhältnissen arbeiten können.

Ein Bereich der Tracking Verfahren ist das Fiducial Tracking, auch Markenbasiertes Tracking genannt. Bei diesem
 Tracking Verfahren wird durch Hilfe einer Marke die Position im realen und virtuellen Raum berechnet. Die bereits
 erwähnten Systeme ARToolKit, ARToolKitPlus und Studierstube, sind Markenbasierte Verfahren und suchen in einem
 Bildsignal nach einer bitonalen Marke. Das erste Markenbasierte \gls{AR}-System war
 Matrix\footcite{rekimoto1998matrix} von \citeauthor{rekimoto1998matrix}. \citeauthor{fiala2004artaga} entwickelte mit
 ARTag\footcite{fiala2004artaga} ein \gls{AR}-System, dass die Idee eines 2D-Codes zur Identifizierung von Matrix
 übernahm. In ARToolKitPlus wurde die Identifizierung von 2D-Codes ebenfalls implementiert.
% subsection tracking_verfahren (end)

\subsection{Bitonale Marken} % (fold)
\label{sub:bitonalemarken}
Bei Markenbasierten \gls{AR}-Verfahren werden bitonale Marken zur Erkennung und Verfolgung von realen Objekten
 eingesetzt. Eine Marke besteht aus einem schwarzen quadratischen Rahmen auf weißem Untergrund und trägt in der Mitte
 Informationen zur Identifizierung. ARToolKit verwendet simple Muster oder Schriftzeichen zur Identifizierung, wie in
 \autoref{fig:marke-artoolkit} gezeigt. ARTag hingegen verwendet eine binär kodierte Zahl als Identifizierungsmerkmal,
 die als Anordnung von $6 \times 6$ Pixel dargestellt wird (s.~\autoref{fig:marke-artag}). ARToolKitPlus, der
 Nachfolger von ARToolKit nutzt ein ähnliches Verfahren wie ARTag\footcite[Vgl.][S.~142]{wagner07b} und unterscheidet
 sich in der Rahmenbreite (s.~\autoref{fig:marke-artoolkitplus}). Wenn in den folgenden Kapiteln von einer bitonalen
 Marke, oder einfach Marke, gesprochen wird, beziehe ich mich auf einen schwarzen quadratischen Rahmen auf weißem
 Untergrund.

\begin{figure}[!ht]
	\centering
	\subfigure[ARToolKit]{
		\label{fig:marke-artoolkit}
		\includegraphics[scale=1]{resources/Marker-ART.pdf}
	}
	\subfigure[ARTag]{
		\label{fig:marke-artag}
		\includegraphics[scale=1]{resources/Marker-ARTag.pdf}
	}
	\subfigure[ARToolKitPlus]{
		\label{fig:marke-artoolkitplus}
		\includegraphics[scale=1]{resources/Marker-ART+.pdf}
	}
	\caption{Verschiedene bitonale Marken.
	}
	\label{fig:bitonale-marken}
\end{figure}
% subsection bitonale_marken (end)

\begin{comment}
	(wagner/schmalstieg ARToolKitPlus fpr Pose Tracking on Mobile Devices S.4)

	Beim Fiducial Marker Tracking werden künstliche Marken zur Erkennung und Verfolgung von realen Objekten eingesetzt. Häufig werden für diese Marken quadratische schwarze Rahmen verwendet die innerhalb des Rahmens Schriftzeichen, Bilder oder 2D Codes enthalten. Diese Marken sind einfach herzustellen und können mit geringem Aufwand an Objekte angebracht werden.

	Diese bitonale Marken haben den Vorteil, dass sie Robust gegen Helligkeitsveränderung sind und die Entscheidung eines Pixels auf eine Schwellwert-Entscheidung reduziert werden kann. Marken für \gls{AR} müssen in einem großen Blickfeld erkannt werden können, was wiederum bei industriellen Anwendung nicht der Fall ist, da hier Marken den größten Teil des Bildes einnehmen können.
\end{comment}

% section definition (end)
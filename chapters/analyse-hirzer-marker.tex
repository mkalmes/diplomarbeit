\subsubsection{Marken} % (fold)
\label{sub:marken}

Die Datenstruktur $\mathit{marker}$, der Aufbau des Speicherblocks und die Operationen zum hinzufügen und freigeben von
 Daten ist, wie bei der Datenstruktur der Liniensegmente auch, nach dem Vorbild des Edgelspeichers aufgebaut. Die
 Datenstruktur in \autoref{alg:marker} verwendet vier Variablen vom Typ \textproc{vector}, um die Eckpunkte einer Marke
 zu speichern.

\begin{algorithm}[!ht]\small
\caption{\textproc{marker}}
\label{alg:marker}
	\begin{algorithmic}[1]
		\State $c1$
		\State $c2$
		\State $c3$
		\State $c4$
	\end{algorithmic}
\end{algorithm}


Die Datenstruktur des Speichervorrats (\autoref{alg:datastructure-markerpool}) und des Speicherblocks
 (\autoref{alg:datastructure-markerpoolimpl}) verwenden eine festgelegte Anzahl von Einträgen zum speichern der
 Einträge.

\begin{algorithm}[!ht]\small
\caption{\textproc{markerPool} (Datenstruktur)}
\label{alg:datastructure-markerpool}
\begin{algorithmic}[1]
	\State $\mathit{data}[N]$
	\Comment Anzahl der Einträge
	\State $\mathit{count}$
\end{algorithmic}
\end{algorithm}


Mit \textproc{getMemoryPools} (\autoref{alg:markerpool-getmemorypools}) und \textproc{getMemoryPool}
 (\autoref{alg:markerpool-getmemorypool}) werden Speicherblöcke angefordert.

\begin{algorithm}[!ht]\small
\caption{\textproc{getMemoryPools}}
\label{alg:markerpool-getmemorypools}
\begin{algorithmic}[1]
	\Require $n$
	\If{$\mathit{data} + S - \mathit{pool} \geq n$}
	\Cost{$c_{1}$}{$3$}
		\State $\mathit{pool} \gets \mathit{pool} + n$
		\Cost{$c_{2}$}{$2$}
		\State \textbf{return} $\mathit{pool} - n$
		\Cost{$c_{3}$}{$2$}
	\Else
		\State \textbf{return} $\mathit{NULL}$
		\Cost{$c_{5}$}{$1$}
	\EndIf
\end{algorithmic}
\end{algorithm}

\begin{algorithm}[!ht]
\caption{\textproc{getMemoryPool}}
\label{alg:markerpool-getmemorypool}
\begin{algorithmic}[1]
	\State $p \gets$ \Call{getmemorypools}{1}
	\State \textbf{return} $p$
\end{algorithmic}
\end{algorithm}


Um die Einträge in einem Speicherblock zu löschen, wird \autoref{alg:markerpool-resetmemorypool} verwendet.

\begin{algorithm}[!ht]
\caption{\textproc{resetMemoryPool}}
\label{alg:markerpool-resetmemorypool}
\begin{algorithmic}[1]
	\Require $p$
	\If{$\lnot p$}
		\State \textbf{return}
	\EndIf
	\State $p(\mathit{count}) \gets 0$
\end{algorithmic}
\end{algorithm}


Mit \textproc{freeMemoryPool} (\autoref{alg:markerpool-freememorypool}) kann der Speicherblock an den Vorrat
 zurückgegeben werden.

\begin{algorithm}[!ht]
\caption{\textproc{freeMemoryPool}}
\label{alg:markerpool-freememorypool}
\begin{algorithmic}[1]
	\Require $p$
	\If{$\lnot p$}
		\State \textbf{return}
	\EndIf
	\State \Call{resetMemoryPool}{$p$}
	\If{$p \geq \mathit{data} \land p \leq \mathit{data} + S$}
		\State $\mathit{pool} \gets p$
	\EndIf
\end{algorithmic}
\end{algorithm}


Die Anzahl der gespeicherten Marken in einem Speicherblock können mit \autoref{alg:markerpool-count} ermittelt werden.

\begin{algorithm}[!ht]
\caption{\textproc{getMarkerCount}}
\label{alg:markerpool-count}
\begin{algorithmic}[1]
	\Require $p$
	\If{$\lnot p$}
	\Cost{$c_{1}$}{$1$}
		\State \textbf{return}
		\Cost{$c_{2}$}{$1$}
	\EndIf
	\State \textbf{return} $\mathit{p.count}$
	\Cost{$c_{4}$}{$2$}
\end{algorithmic}
\end{algorithm}


Um eine Marke zu einem Speicherblock hinzuzufügen, wird \autoref{alg:markerpool-addmarker} verwendet.

\begin{algorithm}[!ht]
\caption{\textproc{addMarker}}
\label{alg:markerpool-addmarker}
\begin{algorithmic}[1]
	\Require $p,m$
	\If{$\lnot p$}
	\Cost{$c_{1}$}{$1$}
		\State \textbf{return}
		\Cost{$c_{2}$}{$1$}
	\EndIf
	\If{$\lnot \left(\mathit{p.count} < N\right)$}
	\Cost{$c_{4}$}{$3$}
		\State \textbf{return} \Comment Speicher voll
		\Cost{$c_{5}$}{$1$}
	\EndIf
	\State $c \gets \mathit{p.count}$
	\Cost{$c_{7}$}{$2$}
	\State $\mathit{p.data}[c] \gets m$
	\Cost{$c_{8}$}{$3$}
	\State $\mathit{p.count} \gets c + 1$
	\Cost{$c_{9}$}{$3$}
\end{algorithmic}
\end{algorithm}


% subsubsection marken (end)

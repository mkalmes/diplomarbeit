\subsubsection{Konturerzeugung} % (fold)
\label{sec:konturerzeugung}

Das zweite Verfahren der Fiducial Detection ermittelt aus einer Regionenmarkierung mit dem Verfahren
 \textproc{arGetContour} eine Kontur. Nachdem \textproc{arLabeling} (\autoref{alg:arlabeling-overview}) beendet ist,
 wird das Regionenbild an die aufrufende Methode \textproc{arDetectMarker} übergeben und in Zeile
 \ref{alg:detectmarker-1-call-labeling} von \autoref{alg:detectmarker-1} gespeichert.
\begin{algorithm}[ht]
\caption{\textproc{arDetectMarker} (Fortsetzung)}
\label{alg:detectmarker-1}
\begin{algorithmic}[1]
	\Require $I,\mathit{thresh},\mathit{marker\_info},\mathit{marker\_num}$
	\State $I_l \gets$ \textproc{NULL}
	\State $\mathit{label\_num},\mathit{area},\mathit{clip},\mathit{label\_ref},\mathit{pos} \gets \infty$
	\State \Call{autoThreshold.reset}{}
	\State \Call{checkImageBuffer}{}
	\State $\mathit{marker\_num} \gets 0$
	\State $I_l \gets$ \Call{arLabeling}{$I,\mathit{thresh},\mathit{label\_num},\mathit{area},\mathit{pos},\mathit{clip},\mathit{label\_ref}$}
	\label{alg:detectmarker-1-call-labeling}
	% AR_AREA_MAX (100000) und AR_AREA_MIN (70) sind defines
	% wmarker_num globale variable
	\If{$I_l$}
	\label{alg:detectmarker-1-check-il-start}
		\State $\mathit{marker\_info2} \gets$ \textproc{arDetectMarker2}$\left(
		\begin{aligned}
				& I_l,\mathit{thresh},\mathit{label\_ref},\mathit{area},\mathit{pos},\mathit{clip},\\
				& \mathit{AR\_AREA\_MAX},\\
				& \mathit{AR\_AREA\_MIN},\\
				& 1.0, \mathit{wmarker\_num}
		\end{aligned}\right)$
		\label{alg:detectmarker-1-call-method}
		\If{$\mathit{marker\_info2}$}
			\State \ldots \Comment{Weitere Anweisungen zur Identifikation einer Marke.}
		\EndIf
	\EndIf
	\label{alg:detectmarker-1-check-il-end}
	\State \ldots \Comment{Weitere Anweisungen zur Identifikation einer Marke.}
\end{algorithmic}
\end{algorithm}

In Zeile \ref{alg:detectmarker-1-check-il-start} wird überprüft, ob der Speicher des Regionenbildes gültig ist um
 danach in Zeile \ref{alg:detectmarker-1-call-method} die Methode \textproc{arDetectMarker2}
 (\autoref{alg:detectmarker2-1}) aufzurufen.

In \autoref{alg:detectmarker2-1} werden die lokalen Variablen initialisiert, deren Bedeutung bei ihrem ersten Einsatz
 erläutert wird.
\begin{algorithm}[!ht]\small
\caption{\textproc{arDetectMarker2} (Initialisierung)}
\label{alg:detectmarker2-1}
\begin{algorithmic}[1]
	\Require $
	\begin{aligned}
		& \mathit{limage}, \mathit{label\_num}, \mathit{label\_ref}, \mathit{warea}, \mathit{wpos}, \mathit{wclip}, \mathit{area\_max}, \mathit{area\_min},\\
		& \mathit{factor}, \mathit{marker\_num}
	\end{aligned}$
	\State $\mathit{pm}, i, j, d, \mathit{ret} \gets \infty$
	\State $\mathit{area\_min} \gets \mathit{area\_min} / 4$
	\State $\mathit{area\_max} \gets \mathit{area\_max} / 4$
	\State $\mathit{xsize} \gets \mathit{arImXsize} / 2$
	\State $\mathit{ysize} \gets \mathit{arImYsize} / 2$
	\algstore{brk-detectmarker2-init}
\end{algorithmic}
\end{algorithm}

Die Laufzeitfunktion von \autoref{alg:detectmarker2-1} ist $T(n) = 13$. \autoref{alg:detectmarker2-2} sorgt mit seiner
 Schleife in Zeile \ref{alg:detectmarker2-2-loop-start}--\ref{alg:detectmarker2-2-loop-end} für die Konturenermittlung.
\begin{algorithm}[ht]
\caption{\textproc{arDetectMarker2} (Fortsetzung)}
\label{alg:detectmarker2-2}
\begin{algorithmic}[1]
	\algrestore{brk-detectmarker2-init}
	\State $\mathit{marker\_num2} \gets 0$
	\label{alg:detectmarker2-2-markernum2}
	\For{$i \gets 0$ \textbf{to} $i < \mathit{label\_num}$}
	\label{alg:detectmarker2-2-loop-start}
		\If{$\mathit{warea}[i] < \mathit{area\_min} \lor \mathit{warea}[i] > \mathit{area\_max}$}
		\label{alg:detectmarker2-2-area-start}
			\State \textbf{continue}
		\EndIf
		\label{alg:detectmarker2-2-area-end}
		\If{$\mathit{wclip}[i \cdot 4 + 0] == 1 \lor \mathit{wclip}[i \cdot 4 + 1] == \mathit{xsize} - 2$}
		\label{alg:detectmarker2-2-checkx-start}
			\State \textbf{continue}
			\label{alg:detectmarker2-2-checkx-continue}
		\EndIf
		\label{alg:detectmarker2-2-checkx-end}
		\If{$\mathit{wclip}[i \cdot 4 + 2] == 1 \lor \mathit{wclip}[i \cdot 4 + 3] == \mathit{ysize} - 2$}
		\label{alg:detectmarker2-2-checky-start}
			\State \textbf{continue}
		\EndIf
		\label{alg:detectmarker2-2-checky-end}
		\State $\mathit{ret} \gets$ \textproc{arGetContour}$\left(
		\begin{aligned}
			& \mathit{limage}, \mathit{label\_ref}, i + 1,\\
			& \mathit{wclip}[i \cdot 4], \mathit{marker\_infoTWO}[\mathit{marker\_num2}]
		\end{aligned}\right)$
		\label{alg:detectmarker2-2-callcontour}
		\State \ldots
		\State $i \gets i + 1$
	\EndFor
	\label{alg:detectmarker2-2-loop-end}
\end{algorithmic}
\end{algorithm}

Dazu wird in Zeile \ref{alg:detectmarker2-2-markernum2} die Variable $\mathit{marker\_num2}$ initialisiert, die zur
 Identifizierung einer Region dient. In Zeile \ref{alg:detectmarker2-2-loop-start}--\ref{alg:detectmarker2-2-loop-end}
 wird jede Regionenmarkierung untersucht. Dazu wird in Zeile
 \ref{alg:detectmarker2-2-area-start}--\ref{alg:detectmarker2-2-area-end} der Fläche der Region $i$ mit dem festgelegten
 minimalen und maximalen Wert verglichen. Nur wenn die Fläche der Region sich innerhalb dieser Grenzen befindet, wird
 die Region $i$ weiter untersucht. Andernfalls wird mit der nächsten Region das Verfahren wiederholt. Zeile
 \ref{alg:detectmarker2-2-checkx-start}--\ref{alg:detectmarker2-2-checkx-end} überprüft die $x$-Koordinaten der
 Start- und Endposition, die in $\mathit{wclip}$ gespeichert sind. Wenn die Koordinaten an der Rändern des
 Regionenbildes $\mathit{limage}$ liegen, wird die weitere Untersuchung in Zeile
 \ref{alg:detectmarker2-2-checkx-continue} abgebrochen. Nur Regionen die nicht an den Rändern liegen, eignen sich zur
 Kontuerenermittlung. In Zeile \ref{alg:detectmarker2-2-checky-start}--\ref{alg:detectmarker2-2-checky-end} wird das
 Verfahren für die $y$-Koordinaten wiederholt. Nur wenn eine Region eine festgelegte Größe nicht unter- oder
 überschreitet, und die Region nicht an den Rändern liegt,
 eignet sie sich zur Konturenermittlung und wird mit der Methode \textproc{arGetContour} in Zeile
 \ref{alg:detectmarker2-2-callcontour} aufgerufen.

\textproc{arGetContour} (\autoref{alg:argetcontour-1}--\autoref{alg:argetcontour-3}) ermittelt für eine Region eine
 Kontur. In \autoref{alg:argetcontour-1} werden die Variablen in Zeile
 \ref{alg:argetcontour-1-initvar-start}--\ref{alg:argetcontour-1-initvar-end} initialisiert.
\begin{algorithm}[ht]
\caption{\textproc{arGetContour} (Initialisierung)}
\label{alg:argetcontour-1}
\begin{algorithmic}[1]
	\Require $\mathit{limage},\mathit{label\_ref},\mathit{label},\mathit{wclip}\left[4\right],\mathit{marker\_infoTWO}$ 
	% static const int      xdir[8] = { 0, 1, 1, 1, 0,-1,-1,-1};
	% static const int      ydir[8] = {-1,-1, 0, 1, 1, 1, 0,-1};
	\State $\mathit{xdir} \gets \left[0, 1, 1, 1, 0,-1,-1,-1\right]$
	\label{alg:argetcontour-1-initvar-start}
	\State $\mathit{ydir} \gets \left[-1,-1, 0, 1, 1, 1, 0,-1\right]$
	\label{alg:argetcontour-1-initvar-neighbour}
	% ARInt16         *p1;
	% int             xsize, ysize;
	% int             sx, sy, dir;
	% int             dmax, d, v1 = 0;
	% int             i, j;
	\State $\mathit{p1} \gets \infty$
	\State $\mathit{sx} \gets \infty$
	\State $\mathit{sy} \gets \infty$
	\State $\mathit{dir} \gets \infty$
	\State $\mathit{dmax} \gets \infty$
	\State $d \gets \infty$
	\State $\mathit{v1} \gets 0$
	\State $i \gets \infty$
	\State $j \gets \infty$
	%     xsize = arImXsize / 2;
	%     ysize = arImYsize / 2;
	\State $\mathit{xsize} \gets \mathit{arImXsize} / 2$
	\label{alg:argetcontour-1-initsizex}
	\State $\mathit{ysize} \gets \mathit{arImYsize} / 2$
	\label{alg:argetcontour-1-initvar-end}
	% j = clip[2];
	% p1 = &(limage[j*xsize+clip[0]]);
	\State $j \gets \mathit{clip}\left[2\right]$
	\State $\mathit{offset} \gets j \cdot \mathit{xsize} + \mathit{clip}\left[0\right]$
	\label{alg:argetcontour-1-offset}
	\State $\mathit{p1} \gets \mathit{limage}\left[\mathit{offset}\right]$
	\label{alg:argetcontour-1-p1}
	% for( i = clip[0]; i <= clip[1]; i++, p1++ ) {
	%     if( *p1 > 0 && label_ref[(*p1)-1] == label ) {
	%         sx = i; sy = j; break;
	%     }
	% }
	\For{$i \gets \mathit{clip}\left[0\right]$ \textbf{to} $i \leq \mathit{clip}\left[1\right]$}
	\label{alg:argetcontour-1-loop-start}
		\If{$\mathit{p1} > 0 \land \mathit{label\_ref}\left[\mathit{p1 - 1}\right] == \mathit{label}$}
		\label{alg:argetcontour-1-haslabel-start}
			\State $\mathit{sx} \gets i$
			\label{alg:argetcontour-1-savex}
			\State $\mathit{sy} \gets j$
			\label{alg:argetcontour-1-savey}
			\State \textbf{break}
		\EndIf
		\label{alg:argetcontour-1-haslabel-end}
		\State $i \gets i + 1$
		\label{alg:argetcontour-1-inci}
		\State Inkrementiere $\mathit{p1}$
		\label{alg:argetcontour-1-incp1}
	\EndFor
	\label{alg:argetcontour-1-loop-end}
	\algstore{brk-argetcontour-init}
\end{algorithmic}
\end{algorithm}

Die 8er-Nachbarschaft zur Verfolgung einer Kontur wird in Zeile
 \ref{alg:argetcontour-1-initvar-start}--\ref{alg:argetcontour-1-initvar-neighbour} initialisiert. Die Bildbreite und
 Bildhöhe werden in Zeile \ref{alg:argetcontour-1-initsizex}--\ref{alg:argetcontour-1-initvar-end} in $\mathit{xsize}$
 und $\mathit{ysize}$ gespeichert. Danach wird in Zeile \ref{alg:argetcontour-1-offset} ein Adressabstand berechnet um
 den Startpunkt der ersten Regionenmarkierung in $\mathit{l\_image}$ in $\mathit{p1}$ zu speichern (Zeile
 \ref{alg:argetcontour-1-p1}). In der Schleife in Zeile
 \ref{alg:argetcontour-1-loop-start}--\ref{alg:argetcontour-1-loop-end} wird der Startpunkt der Region gesucht. Dazu
 wird in Zeile \ref{alg:argetcontour-1-haslabel-start}--\ref{alg:argetcontour-1-haslabel-end} geprüft, ob an der
 Adresse $\mathit{p1}$ eine Regionenmarkierung gespeichert ist und ob der Wert der Regionenmarkierung in
 $\mathit{label\_ref}$ mit der Markierung in $\mathit{label}$ übereinstimmt. Falls die Werte übereinstimmen, werden die
 Koordinaten in Zeile \ref{alg:argetcontour-1-savex}--\ref{alg:argetcontour-1-savey} in den Variablen $\mathit{sx}$ und
 $\mathit{sy}$ gespeichert und die Schleife wird abgebrochen. Ansonsten wird in Zeile
 \ref{alg:argetcontour-1-inci}--\ref{alg:argetcontour-1-incp1} die Laufvariable $i$ um $1$ erhöht und die Adresse
 $\mathit{p1}$ inkrementiert. Die Laufzeitfunktion ist in \autoref{eq:analyse-argetconout-init} aufgeführt.
\begin{subequations}
\label{eq:analyse-argetconout-init}
\begin{align}
\label{eq:analyse-argetconout-init-1}
T_{worst}(\mathit{xsize})& = (c_{1} + c_{2} + c_{3}6 + c_{4} + c_{5}2 + c_{6}2 + c_{7}2 + c_{8}2 + c_{9}4 + c_{10}2)
\\
& \quad
+ c_{11}(\mathit{xsize} - 3)
+ (c_{12}5 + c_{17} + c_{18})(\mathit{xsize} - 4)
\nonumber \\
\label{eq:analyse-argetconout-init-2}
T_{worst}(\mathit{xsize})& =
(c_{11} + c_{12}5 + c_{17} + c_{18})\mathit{xsize}
- (3c_{11} + 20c_{12} + 4c_{17} + 4c_{18})
\\
& \quad
+  (c_{1} + c_{2} + c_{3}6 + c_{4} + c_{5}2 + c_{6}2 + c_{7}2 + c_{8}2 + c_{9}4 + c_{10}2)
\nonumber
\end{align}
\end{subequations}

Die Wachstumsrate von \autoref{eq:analyse-argetconout-init-2} ist, für $c_{1} = 7$, $c_{2} = 8$ und $\mathit{xsize}_0
 \geq 8$, $8\mathit{xsize} - 8 = \Theta(\mathit{xsize})$.

\autoref{alg:argetcontour-2} verfolgt nun die Kontur der Region. Dazu wird zuerst in der Datenstruktur
 $\mathit{marker\_infoTWO}$ die Anzahl der Koordinaten initialisiert und die $x$- und $y$-Werte der ersten
 Regionenmarkierung gespeichert (Zeile
 \ref{alg:argetcontour-2-markerinit-start}--\ref{alg:argetcontour-2-markerinit-end}). Danach wird die Orientiertung der
 8er-Nachbarschaft in Zeile \ref{alg:argetcontour-2-orientation} festgelegt.

\begin{algorithm}[!ht]
\caption{\textproc{arGetContour} (Verfolge Kontur)}
\label{alg:argetcontour-2}
\begin{algorithmic}[1]
	\algrestore{brk-argetcontour-init}
	% marker_infoTWO->coord_num = 1;
	% marker_infoTWO->x_coord[0] = sx;
	% marker_infoTWO->y_coord[0] = sy;
	\State $\mathit{marker\_infoTWO.coord\_num} \gets 1$
	\Cost{$c_{20}$}{$2$}
	\label{alg:argetcontour-2-markerinit-start}
	\State $\mathit{marker\_infoTWO.xcoord}\left[0\right] \gets \mathit{sx}$
	\Cost{$c_{21}$}{$3$}
	\State $\mathit{marker\_infoTWO.ycoord}\left[0\right] \gets \mathit{sy}$
	\Cost{$c_{22}$}{$3$}
	\label{alg:argetcontour-2-markerinit-end}
	% dir = 5;
	\State $\mathit{dir} \gets 5$
	\Cost{$c_{23}$}{$1$}
	\label{alg:argetcontour-2-orientation}
	% for(;;) {
	\For{\textbf{true}}
	\label{alg:argetcontour-2-contourloop-start}
	%     p1 = &(limage[marker_infoTWO->y_coord[marker_infoTWO->coord_num-1] * xsize
	%                 + marker_infoTWO->x_coord[marker_infoTWO->coord_num-1]]);
		\State $\mathit{tmp\_coord\_num} \gets \mathit{marker\_infoTWO.coord\_num} - 1$
		\Cost{$c_{25}$}{$(n-3) \cdot 3$}
		\label{alg:argetcontour-2-offset-start}
		\State $\mathit{offset} \gets \mathit{marker\_infoTWO.y\_coord}\left[\mathit{tmp\_coord\_num}\right]
		 \cdot \mathit{xsize}$
		\Cost{$c_{26}$}{$(n-3) \cdot 5$}
		\State $\mathit{offset} \gets \mathit{offset} +
		 \mathit{marker\_infoTWO.x\_coord}\left[\mathit{tmp\_coord\_num}\right]$
		\Cost{$c_{27}$}{$(n-3) \cdot 5$}
		\label{alg:argetcontour-2-offset-end}
		\State $\mathit{p1} \gets \mathit{l\_image}[\mathit{offset}]$
		\Cost{$c_{28}$}{$(n-3) \cdot 2$}
		\label{alg:argetcontour-2-address}
	%     dir = (dir+5)%8;
		\State $\mathit{dir} \gets \left(\mathit{dir} + 5\right)\mod 8$
		\Cost{$c_{29}$}{$(n-3) \cdot 3$}
		\label{alg:argetcontour-2-nextorientation}
	%     for(i=0;i<8;i++) {
	%         if( p1[ydir[dir]*xsize+xdir[dir]] > 0 ) break;
	%         dir = (dir+1)%8;
	%     }
		\For{$i \gets 0$ \textbf{to} $i < 8$}
		\Cost{$c_{30}$}{$(n-3) \cdot \bigl((8 - 1) + 1\bigr)$}
		\label{alg:argetcontour-2-neighbourloop-start}
			\State $\mathit{offset} \gets \mathit{ydir}\left[dir\right] \cdot \mathit{xsize} +
			 \mathit{xdir}\left[\mathit{dir}\right]$
			\Cost{$c_{31}$}{$(n-3) \cdot 5(8 - 1)$}
			\label{alg:argetcontour-2-offset}
			\If{$\mathit{p1}\left[\mathit{offset}\right] > 0$}
			\Cost{$c_{32}$}{$(n-3) \cdot 2(8 - 1)$}
			\label{alg:argetcontour-2-haslabel-start}
				\State \textbf{break}
				\Cost{$c_{33}$}{$n-3$}
			\EndIf
			\label{alg:argetcontour-2-haslabel-end}
			\State $\mathit{dir} \gets \left(\mathit{dir} + 1\right)\mod 8$
			\Cost{$c_{35}$}{$(n-3) \cdot 3(8 - 1)$}
			\label{alg:argetcontour-2-incdir}
			\State $i \gets i + 1$
			\Cost{$c_{36}$}{$(n-3)(8 - 1)$}
			\label{alg:argetcontour-2-inci}
		\EndFor
		\label{alg:argetcontour-2-neighbourloop-end}
		% if( i == 8 ) {
		%             printf("??? 2\n"); return(-1);
		%         }
		\If{$i = 8$}
		\Cost{$c_{38}$}{$(n-3) \cdot 1$}
		\label{alg:argetcontour-2-hasnolabel-start}
			\State \textbf{return}$(-1)$
			% \Cost{$c_{39}$}{$9997 \cdot 1$}
		\EndIf
		\label{alg:argetcontour-2-hasnolabel-end}
	%     marker_infoTWO->x_coord[marker_infoTWO->coord_num]
	%         = marker_infoTWO->x_coord[marker_infoTWO->coord_num-1] + xdir[dir];
		\State $\mathit{mkx} \gets \mathit{marker\_infoTWO.x\_coord}\left[\mathit{marker\_infoTwo.coord\_num}\right]$
		\Cost{$c_{41}$}{$(n-3) \cdot 4$}
		\label{alg:argetcontour-2-savex}
		\State $\mathit{mkx} \gets \mathit{marker\_infoTWO.x\_coord}\left[\mathit{marker\_infoTwo.coord\_num}
		 - 1\right] + \mathit{xdir}\left[\mathit{dir}\right]$
		\Cost{$c_{42}$}{$(n-3) \cdot 7$}
	%     marker_infoTWO->y_coord[marker_infoTWO->coord_num]
	%         = marker_infoTWO->y_coord[marker_infoTWO->coord_num-1] + ydir[dir];
		\State $\mathit{mky} \gets \mathit{marker\_infoTWO.y\_coord}
		\left[\mathit{marker\_infoTwo.coord\_num}\right]$
		\Cost{$c_{43}$}{$(n-3) \cdot 4$}
		\State $\mathit{mky} \gets \mathit{marker\_infoTWO.y\_coord}\left[\mathit{marker\_infoTwo.coord\_num} - 1\right]
		 + \mathit{ydir}\left[\mathit{dir}\right]$
		\Cost{$c_{44}$}{$(n-3) \cdot 7$}
		\label{alg:argetcontour-2-savey}
	%     if( marker_infoTWO->x_coord[marker_infoTWO->coord_num] == sx
	%      && marker_infoTWO->y_coord[marker_infoTWO->coord_num] == sy ) break;
		\If{$\mathit{mkx} = sx \land \mathit{mky} = sy$}
		\Cost{$c_{45}$}{$(n-3) \cdot 3$}
		\label{alg:argetcontour-2-iscontourclosed-start}
			\State \textbf{break}
			\Cost{$c_{46}$}{$1$}
		\EndIf
		\label{alg:argetcontour-2-iscontourclosed-end}
	%     marker_infoTWO->coord_num++;
		\State $\mathit{marker\_infoTWO.coord\_num} \gets \mathit{marker\_infoTWO.coord\_num} + 1$
		\Cost{$c_{48}$}{$(n-3) \cdot 4$}
		\label{alg:argetcontour-2-inccoordnum}
		% if( marker_infoTWO->coord_num == AR_CHAIN_MAX-1 ) {
		%             printf("??? 3\n"); return(-1);
		%         }
		\If{$\mathit{marker\_infoTWO.coord\_num} = \mathit{AR\_CHAIN\_MAX} - 1}$
		\Cost{$c_{49}$}{$(n-3) \cdot 3$}
		\label{alg:argetcontour-2-ismaxchainreached-start}
			\State \textbf{return}$(-1)$
			% \Cost{$c_{50}$}{$9997 \cdot 1$}
			\label{alg:argetcontour-2-error}
		\EndIf
		\label{alg:argetcontour-2-ismaxchainreached-end}
	% }
	\EndFor
	\label{alg:argetcontour-2-contourloop-end}
	\algstore{brk-argetcontour-contour}
\end{algorithmic}
\end{algorithm}


Die Kontur wird in der Schleife in Zeile
 \ref{alg:argetcontour-2-contourloop-start}--\ref{alg:argetcontour-2-contourloop-end} verfolgt. Zuerst muss der
 Adressabstand der letzten Regionenmarkierung in $\mathit{l\_image}$ berechnet werden (Zeile
 \ref{alg:argetcontour-2-offset-start}--\ref{alg:argetcontour-2-offset-end}). Danach wird die Adresse der
 Regionenmarkierung in $\mathit{p1}$ hinterlegt (Zeile \ref{alg:argetcontour-2-address}). In Zeile
 \ref{alg:argetcontour-2-nextorientation} wird die Orientierung der 8er-Nachbarschaft auf die nächste zu untersuchende
 Richtung gedreht. Das drehen der 8er-Nachbarschaft erfolgt, indem von der letzten Position einer Kontur im
 Uhrzeigersinn auf den nächsten Nachbarn weitergerückt wird (Vgl. \autoref{fig:}).

In Zeile \ref{alg:argetcontour-2-neighbourloop-start}--\ref{alg:argetcontour-2-neighbourloop-end} wird in der Schleife
 die Nachbarn im Uhrzeigersinn untersucht. Dazu wird mit Hilfe der 8er-Nachbarschaft in Zeile
 \ref{alg:argetcontour-2-offset} ein Adressabstand berechnet. Durch die Überprüfung einer Regionenmarkierung an der
 Adresse von $\mathit{p1}$ in Zeile \ref{alg:argetcontour-2-haslabel-start}--\ref{alg:argetcontour-2-haslabel-end} wird
 entschieden, ob die Schleife von Zeile
 \ref{alg:argetcontour-2-neighbourloop-start}--\ref{alg:argetcontour-2-neighbourloop-end} abgebrochen wird weil eine
 Markierung gefunden wurde, oder ob ein weiterer Nachbar untersucht werden muss. Falls ein weiterer Nachbar untersucht
 werden muss, wird in Zeile \ref{alg:argetcontour-2-incdir} die Position der 8er-Nachbarschaft um eine Position
 weitergedreht und in Zeile \ref{alg:argetcontour-2-inci} die Laufvariable $i$ erhöht. Falls alle Nachbarn der
 Regionenmarkierung untersucht worden sind ohne eine weiter Markierung zu finden, wird das Verfahren in Zeile
 \ref{alg:argetcontour-2-hasnolabel-start}--\ref{alg:argetcontour-2-hasnolabel-end} mit einem Fehlerwert abgebrochen.

Wenn eine weitere Markierung in Zeile
 \ref{alg:argetcontour-2-neighbourloop-start}--\ref{alg:argetcontour-2-neighbourloop-end} gefunden und die Schleife
 abgebrochen wurde, wird in Zeile \ref{alg:argetcontour-2-savex}--\ref{alg:argetcontour-2-savey} die Markierung
 gespeichert. Dazu wird die $x$- und $y$-Koordinate der letzten Markierung ausgelesen und die Richtung der
 8er-Nachbarschaft addiert. Danach wird in Zeile
 \ref{alg:argetcontour-2-iscontourclosed-start}--\ref{alg:argetcontour-2-iscontourclosed-end} überprüft, ob die
 gespeicherte Markierung mit den Anfangskoordinaten übereinstimmt und es sich somit um eine geschlossene Kontur
 handelt. Falls das Verfahren eine geschlossene Kontur gefunden hat, wird die Schleife von Zeile
 \ref{alg:argetcontour-2-contourloop-start}--\ref{alg:argetcontour-2-contourloop-end} abgebrochen. Falls es sich nicht
 um eine geschlossene Kontur handelt, wird die Kontur weiterverfolgt. In Zeile \ref{alg:argetcontour-2-inccoordnum}
 wird die Anzahl der Koordinaten in $\mathit{marker\_infoTWO}$ erhöht. Bevor die Kontur weiterverfolgt wird, wird in
 Zeile \ref{alg:argetcontour-2-ismaxchainreached-start}--\ref{alg:argetcontour-2-ismaxchainreached-end} überprüft, ob
 die Anzahl der Koordinaten einer festgelegten Größe entspricht. Falls dem so ist, wird das Verfahren mit einem
 Fehlerwert in Zeile \ref{alg:argetcontour-2-error} abgebrochen. Dadurch wird sichergestellt, dass die Schleife in
 Zeile \ref{alg:argetcontour-2-contourloop-start}--\ref{alg:argetcontour-2-contourloop-end} auch bei einer nicht
 geschlossenen Kontur terminiert.

In \autoref{alg:argetcontour-3} wird, nachdem eine geschlossene Kontur vorliegt, die Koordinate mit dem größten Abstand
 zum Startpunkt der Region gesucht (Zeile
 \ref{alg:argetcontour-3-finddmax-start}--\ref{alg:argetcontour-3-finddmax-end}).

\begin{algorithm}[!ht]\small
\caption{\textproc{arGetContour} (Finde größten Abstand zu Punkt $(sx,sy)$)}
\label{alg:argetcontour-3}
\begin{algorithmic}[1]
	\algrestore{brk-argetcontour-contour}
	% dmax = 0;
	\State $\mathit{dmax} \gets 0$
	\Cost{$c_{53}$}{$1$}
	\label{alg:argetcontour-3-initdmax}
	% for(i=1;i<marker_infoTWO->coord_num;i++) {
	\For{$i \gets 1$ \textbf{to} $i < \mathit{marker\_infoTWO.coord\_num}$}
	\Cost{$c_{54}$}{$n - 2$}
	% \Cost{$c_{54}$}{$1$}
	\label{alg:argetcontour-3-finddmax-start}
	%     d = (marker_infoTWO->x_coord[i]-sx)*(marker_infoTWO->x_coord[i]-sx)
	%       + (marker_infoTWO->y_coord[i]-sy)*(marker_infoTWO->y_coord[i]-sy);
	%     if( d > dmax ) {
	%         dmax = d;
	%         v1 = i;
	%     }
	% }
		\State $a \gets (\mathit{marker\_infoTWO.x\_coord}\left[i\right] - sx)$
		\Cost{$c_{55}$}{$4(n-3)$}
		\label{alg:argetcontour-3-calcdistance-start}
		\State $b \gets (\mathit{marker\_infoTWO.y\_coord}\left[i\right] - sy)$
		\Cost{$c_{56}$}{$4(n-3)$}
		\State $d \gets (a \cdot a) + (b \cdot b)$
		\Cost{$c_{57}$}{$4(n-3)$}
		\label{alg:argetcontour-3-calcdistance-end}
		\If{$d > \mathit{dmax}$}
		\Cost{$c_{58}$}{$n-3$}
		\label{alg:argetcontour-3-isdmaxbigger}
			\State $\mathit{dmax} \gets d$
			\Cost{$c_{59}$}{$n-3$}
			\label{alg:argetcontour-3-savedmax}
			\State $\mathit{v1} \gets i$
			\Cost{$c_{60}$}{$n-3$}
		\EndIf
		\State $i \gets i + 1$
		\Cost{$c_{62}$}{$n-3$}
	\EndFor
	\label{alg:argetcontour-3-finddmax-end}
	% for(i=0;i<v1;i++) {
	%     arGetContour_wx[i] = marker_infoTWO->x_coord[i];
	%     arGetContour_wy[i] = marker_infoTWO->y_coord[i];
	% }
	\For{$i \gets 0$ \textbf{to} $i < \mathit{v1}$}
	\Cost{$c_{64}$}{$n-2$}
	\label{alg:argetcontour-3-dividev1-start}
		\State $\mathit{arGetContour\_wx}[i] \gets \mathit{marker\_infoTWO.x\_coord}\left[i\right]$
		\Cost{$c_{65}$}{$4(n-3)$}
		\State $\mathit{arGetContour\_wy}[i] \gets \mathit{marker\_infoTWO.y\_coord}\left[i\right]$
		\Cost{$c_{66}$}{$4(n-3)$}
		\State $i \gets i + 1$
		\Cost{$c_{67}$}{$(n-3)$}
	\EndFor
	\label{alg:argetcontour-3-dividev1-end}
	% for(i=v1;i<marker_infoTWO->coord_num;i++) {
	\For{$i \gets \mathit{v1}$ \textbf{to} $i < \mathit{marker\_infoTWO.coord\_num}$}
	\Cost{$c_{69}$}{$2$}
	\label{alg:argetcontour-3-dividecoordnum-start}
	%     marker_infoTWO->x_coord[i-v1] = marker_infoTWO->x_coord[i];
	%     marker_infoTWO->y_coord[i-v1] = marker_infoTWO->y_coord[i];
		\State $\mathit{marker\_infoTWO.x\_coord}\left[i - \mathit{v1}\right] \gets
		 \mathit{marker\_infoTWO.x\_coord}\left[i\right]$
		\Cost{$c_{70}$}{$6$}
		\State $\mathit{marker\_infoTWO.y\_coord}\left[i - \mathit{v1}\right] \gets
		 \mathit{marker\_infoTWO.y\_coord}\left[i\right]$
		\Cost{$c_{71}$}{$6$}
		\State $i \gets i + 1$
		\Cost{$c_{72}$}{$1$}
	% }
	\EndFor
	\label{alg:argetcontour-3-dividecoordnum-end}
	% for(i=0;i<v1;i++) {
	\For{$i \gets 0$ \textbf{to} $i < \mathit{v1}$}
	\Cost{$c_{74}$}{$(n-2)$}
	\label{alg:argetcontour-3-merge-start}
	%     marker_infoTWO->x_coord[i-v1+marker_infoTWO->coord_num] = arGetContour_wx[i];
	%     marker_infoTWO->y_coord[i-v1+marker_infoTWO->coord_num] = arGetContour_wy[i];
		\State $num \gets \mathit{marker\_infoTWO.coord\_num}$
		\Cost{$c_{75}$}{$2(n-3)$}
		\State $\mathit{marker\_infoTWO.x\_coord}\left[i - \mathit{v1} + num\right] \gets
		 \mathit{arGetContour\_wx}[i]$
		\Cost{$c_{76}$}{$6(n-3)$}
		\State $\mathit{marker\_infoTWO.y\_coord}\left[i - \mathit{v1} + num\right] \gets
		 \mathit{arGetContour\_wy}[i]$
		\Cost{$c_{77}$}{$6(n-3)$}
		\State $i \gets i + 1$
		\Cost{$c_{78}$}{$(n-3)$}
	% }
	\EndFor
	\label{alg:argetcontour-3-merge-end}
	% marker_infoTWO->x_coord[marker_infoTWO->coord_num] = marker_infoTWO->x_coord[0];
	% marker_infoTWO->y_coord[marker_infoTWO->coord_num] = marker_infoTWO->y_coord[0];
	% marker_infoTWO->coord_num++;
	% 
	% return 0;
	\State $num \gets \mathit{marker\_infoTWO.coord\_num}$
	\Cost{$c_{80}$}{$2$}
	\label{alg:argetcontour-3-savev1-start}
	\State $\mathit{marker\_infoTWO.x\_coord}[num] \gets \mathit{marker\_infoTWO.x\_coord}[0]$
	\Cost{$c_{81}$}{$5$}
	\State $\mathit{marker\_infoTWO.y\_coord}[num] \gets \mathit{marker\_infoTWO.y\_coord}[0]$
	\Cost{$c_{82}$}{$5$}
	\label{alg:argetcontour-3-savev1-end}
	\State $\mathit{marker\_infoTWO.coord\_num} \gets \mathit{marker\_infoTWO.coord\_num} + 1$
	\Cost{$c_{83}$}{$4$}
	\label{alg:argetcontour-3-inccoordnum}
	\State \textbf{return} $0$
	\Cost{$c_{84}$}{$1$}
\end{algorithmic}
\end{algorithm}


Dazu wird in Zeile \ref{alg:argetcontour-3-initdmax} die Variable $\mathit{dmax}$ initialisiert. Danach wird in Zeile
 \ref{alg:argetcontour-3-calcdistance-start}--\ref{alg:argetcontour-3-calcdistance-end} der Abstand zwischen dem
 aktuellen Punkt $i$ und der Koordinaten $\mathit{sx}$ und $\mathit{sy}$ berechnet. Falls der Abstandswert größer als
 $\mathit{dmax}$ ist (Zeile \ref{alg:argetcontour-3-isdmaxbigger}), wird der Abstandswert in $\mathit{dmax}$
 gespeichert (Zeile \ref{alg:argetcontour-3-savedmax}) und die Position $i$ in $\mathit{v1}$ hinterlegt.

Nachdem die Schleife in Zeile \ref{alg:argetcontour-3-calcdistance-start}--\ref{alg:argetcontour-3-calcdistance-end}
 beendet ist, ist der größte Abstand in $\mathit{dmax}$ gespeichert und die Position der Koordinate in $\mathit{v1}$
 gespeichert. Nun wird in Zeile \ref{alg:argetcontour-3-dividev1-start}--\ref{alg:argetcontour-3-dividev1-end} von der
 ersten Position bis zur Position $\mathit{v1}$ alle Koordinaten in einer temporären Liste gespeichert. In Zeile
 \ref{alg:argetcontour-3-dividecoordnum-start}--\ref{alg:argetcontour-3-dividecoordnum-end} werden alle Koordinaten ab
 Position $\mathit{v1}$ an den Anfang von $\mathit{marker\_infoTWO}$ verschoben. Die temporär gespeicherten Koordinaten
 werden in Zeile \ref{alg:argetcontour-3-merge-start}--\ref{alg:argetcontour-3-merge-end} an das Ende von
 $\mathit{marker\_infoTWO}$ angehängt.

Zum Schluss wird in Zeile \ref{alg:argetcontour-3-savev1-start}--\ref{alg:argetcontour-3-savev1-end} die Koordinaten
 des Punktes mit dem größten Abstandswert an die letzte Stelle von marker\_infoTWO kopiert, so dass die Koordinaten am
 Anfang und am Ende gespeichert sind. Die Anzahl der gespeicherten Koordinaten wird in Zeile
 \ref{alg:argetcontour-3-inccoordnum} erhöht und die Konturverfolgung beendet.

% subsubsection konturerzeugung (end)

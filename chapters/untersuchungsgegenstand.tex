\section{Untersuchungsgegenstand: Tracking Verfahren und Tracking Algorithmen} % (fold)
\label{sec:untersuchungsgegenstand}
\begin{comment}
	Untersuchungsgegenstand: Verfahren und Algorithmen präzise vorstellen und ihre Unterschiede hervorheben.
	Notwendige Kriterien der Algorithmen bestimmen

	Grober Ablauf der Verfahren:
	* Wer hats erfunden?
	* Wie ist das Verfahren aufgebaut (Algo in grob)
	* Welche Kriterien müssen erfüllt sein (monochrom, rgb eingabe)?
\end{comment}

\subsection{Verfahren nach Hirzer} % (fold)
\label{sub:verfahren_nach_hirzer}

Der Algorithmus von \citeauthor{hirzer08}\footcite{hirzer08} ist nach dem Vorbild der \textit{pixel connectivity edge
 linking based algorithms} entworfen und ist in drei Hauptteile aufgebaut. Zuerst werden Liniensegmente erstellt,
 indem \glspl{edgel} auf einem Suchraster gefunden und zusammengeführt werden. Die kurzen Liniensegmente werden dann zu
 längeren Linien zusammengeführt. Anschließend werden im zweiten Schritt alle gefundenen Linien erweitert um die
 Gesamtlänge einer Linie zu erhalten. Im letzten Schritt werden die Linien zu Vierecken verbunden. Im weiteren Verlauf
 werden diese Schritte als Line Detection, Line Extension und Quadrangle Detection bezeichnet.

\subsubsection{Line Detection} % (fold)
\label{sub:line_detection}
Die Linienerkennung basiert auf dem Verfahren von \citeauthor{clarke96}\footcite{clarke96} und besteht aus zwei
 Schritten. Im ersten Schritt wird das Bildsignal grob abgetastet um im zweiten Schritt durch das RANSAC Verfahren eine
 Linienhypothese zu erstellen und zu bewerten.

Im ersten Schritt wird zuerst das monochrome Eingabesignal $I_m$ in $40 \times 40$ \gls{pixel} große Regionen
 unterteilt. Jede nachfolgende Operation erfolgt innerhalb einer Region. Eine Region wird wiederum unterteilt in
 horizontale und vertikale Scanlines, die jeweils $5$ \gls{pixel} Abstand zueinander haben. Jedes \gls{pixel} auf den
 Scanlines wird mit einem Gauß-Kernel gefaltet um die Komponente des Gradienten zu bestimmen. Ein lokales Maximum des
 Gradienten, dass größer als ein festgelegter Schwellwert ist, wird als \gls{edgel} betrachtet und seine Orientierung
 berechnet.

Im zweiten Schritt wird das RANSAC Verfahren verwendet, um aus der Menge der \glspl{edgel} Liniensegmente zu bestimmen.
 Eine Linienhypothese wird durch die zufällige Auswahl zweier \glspl{edgel} erstellt, deren Orientierung innerhalb
 eines Grenzwert liegen müssen. Ein \gls{edgel} dient als Startpunkt und das andere \gls{edgel} als Endpunkt der Linie.
 Im Anschluss wird die Anzahl der \glspl{edgel} betrachtet, die in der Nähe dieser Linie liegen und eine kompatible
 Orientierung mit der Linie aufweisen. Diese \glspl{edgel} unterstützen die Hypothese einer Linie im Bildsignal, wenn
 die Anzahl größer ist als die minimal geforderte Anzahl von Unterstützungsedgels. Die zufällige Auswahl zweier
 \glspl{edgel} um eine Linie zu erstellen und deren Edgelunterstützung zu ermitteln wird mehrmals wiederholt um die
 Linie mit der meisten Edgelunterstützung zu finden. Wenn eine solche dominante Linie gefunden wurde, gilt die
 Hypothese als bestätigt und die Linie wird als vorhanden betrachtet. Die Edgels die zur Unterstüztung der Hypothese
 der Linie galten, werden aus der Menge der Edgels entfernt und das Verfahren wird solange wiederholt, bis alle
 Liniensegmente gefunden wurden oder nicht mehr genügend Edgels vorhanden sind.

Das Verfahren ist in \autoref{alg:linedetection-clarke} dargestellt.

\begin{algorithm}
	\caption{Line Detection nach \citeauthor{clarke96}}
	\label{src:lineDetection}
	\begin{algorithmic}[1]
	\Procedure{LineDetection}{$I$}
		\State $I$ in $40 \times 40$ Pixel große Regionen unterteilen
		\For{$i \gets 0$, alle Regionen}
			\State $Liste \gets \infty$
			\State $i$ in horizontale und vertikale Scanlines unterteilen mit $5$ Pixel Abstand
			\For{$j \gets 0$, alle Pixel auf allen Scanlines}
				\State $x \gets$ Falte $I_m\left(j\right)$ mit Gauß-Kernel
				\If{$x > Schwellwert$}
					\Comment Edgel gefunden
					\State Orientierung des Edgels berechnen
					\State $Liste \gets edgel$
				\EndIf
			\EndFor
			\While{$Liste \not=0 \lor Liste <$ min. Support-Edgels}
				\State $erkannteLinie \gets 0$
				\For{$j \gets 1, 25$}
					\State $Linie \gets$ zwei zufällige Edgels mit kompatibler Orientierung aus $Liste$ wählen
					\State Anzahl von Support-Edgels in der Nähe der Linie bestimmen
					\If{$Linie > erkannteLinie$}
						\State $erkannteLinie \gets Linie$
					\EndIf
				\EndFor
				\If{Anzahl von Support-Edgels > min. Support-Edgels}
					\Comment Linie wurde erkannt
					\State $Liste \gets Liste -$ Support-Edgels der Linie
				\EndIf
			\EndWhile
		\EndFor
	\EndProcedure
	\end{algorithmic}
\end{algorithm}



\citeauthor{hirzer08} hat das Verfahren von \citeauthor{clarke96} abgewandelt, um es zur Erkennung einer bitonalen
 Marke zu nutzen. Dazu verwendet \citeauthor{hirzer08}, anstatt eines monochromen Bildsignals, ein farbiges Bildsignal
 und untersucht zuerst einen der drei Farbkanäle. Wenn ein \gls{edgel} in einem Kanal gefunden wird, werden die
 verbleibenden Kanäle untersucht, um sicherzustellen, dass auch hier ein \gls{edgel} vorliegt. Ist der Gradient in
 allen drei Kanälen höher als der festgelegte Schwellwert, handelt es sich um einen Übergang  von Schwarz nach Weiß.
 Ist dies nicht der Fall, handelt es sich um einen farbigen Übergang und ist somit zur Erkennung einer Marke
 uninteressant. Da ein monochromes Signal wie in \autoref{sub:bildtypen} beschrieben nur ein Kanal besitzt, kann hier
 diese Unterscheidung nicht getroffen werden, was zu einer größeren Anzahl von \glspl{edgel}
 führt\footcite[Vgl.][S.~6--7]{hirzer08}.

Das vorgestellte Verfahren von \citeauthor{clarke96} liefert als Ergebnis nur kurze Liniensegmente. Um eine Kante
 entlang einer Marke zu erkennen, müssen die kurzen Liniensegmente zusammengeführt werden. Dazu werden alle
 Liniensegmente miteinander verglichen um jede Kombinattionsmöglichkeit zu testen. Ob zwei Liniensegmente zu einer
 Linie zusammengeführt werden können, ist von drei Kriterien abhängig. Zuerst müssen zwei Liniensegmente eine
 kompatible Orientierung aufweisen, die nur geringfügig abweichen darf um als Ergebnis eine gerade Linien zu erhalten.
 Als zweites Kriterium muss eine Verbindungslinie zwischen den Liniensegmenten ebenfalls eine kompatible Orientierung
 aufweisen. Dadurch wird sichergestellt, dass keine Liniensegmente zusammengeführt werden, die zwar eine kompatible
 Orientierung besitzen aber parallel zueinander liegen. Als letztes Kriterium wird der Gradient der Verbindungslinie
 Punkt für Punkt untersucht. Dieses Kriterium dient dazu nebeneinanderliegende Marken zu unterscheiden. Würde man dies
 Unterscheidung vernachlässigen, würden mehrere Kanten unterschiedlicher Marken zusammengeführt (Vgl. \autoref{fig:}).

\citeauthor{hirzer08} verwendet keinen Schwellwert um einen minimalen oder maximalen Abstand zwischen zwei
 Liniensegmenten festzulegen. \citeauthor{hirzer08} begründet dies Entscheidung damit, dass bei einem zu kleinen
 Schwellwert Liniensegmente, die zuweit auseinander liegen, nicht zusammengeführt werden, obwohl sie zusammengehören.
 Wird der Abstand des Schwellwerts aber zu groß gewählt, werden Liniensegmente zusammengeführt, die nicht
 zusammengehören. Um auf einen Distanzschwellwert verzichten zukönnen, werden die Liniensegmente sortiert, sodaß
 Liniensegmente mit kurzen Verbindungslinien zuerst zusammengeführt werden. Dadurch kann sichergestellt werden, dass
 Liniensegmente zusammengeführt werden die nah beieinander liegen.

Das Zusammenführen der Liniensegmente wird zweimal durchgeführt. Zuerst innerhalb einer Region um alle kurzen
 Liniensegmente zu verbinden. Nachdem innerhalb aller Regionen Liniensegmente zusammengeführt wurden, wird der Vorgang
 auf dem gesammten Bildsignal wiederholt um größere Liniensegmente zuvereinen. Dadurch müssen nicht alle
 Liniensegmente-Kombinationen im gesammten Bildsignal verglichen werden, was die Laufzeit
 reduziert\footcite[Vgl.][S.~10]{hirzer08}.

% subsubsection line_detection (end)

\subsubsection{Line Extension} % (fold)
\label{sub:line_extension}
% subsubsection line_extension (end)

\subsubsection{Quadrangle Detection} % (fold)
\label{sub:quadrangle_detection}
% subsubsection quadrangle_detection (end)

\begin{comment}
	Der RANSAC Grouper wird verwendet um gerade Liniensegmente zu finden. Dazu werden zwei zufällige edgels ausgewählt
	 und ihre kompatible Orientierung überprüft. Die Anzahl der supporting edgels wird durch die Distanz des
	 Liniensegments und der Orientierung bestimmt. Durch wiederholung dieses Prozesses wird die dominante Linie in der
	 Region bestimmt. Die supporting edgels werden entfernt und der Prozess wird wiederholt bis keine edgels mehr
	 vorhanden sind oder eine maximale Anzahl von wiederholungen erreicht wurde. Durch dieses Wiederholung wird
	 sichergestellt, dass alle dominanten Linien in einer Region erkannt werden.

	Vorteil: Der Algorithmus ist sehr schnell und lässt sich für gewünschte Liniensegmente anpassen.
	Nachteil: Durch sein antisotropic detection verhalten diskriminiert das verfahren diagonale liniensegmente. Dies
	 ist durch ein rechteckiges samplingrid bedingt.

	Hirzer hat das Verfahren um zwei Punkte erweitert und angepasst.
	Wird in einem RGB Bild ein Kanal untersucht und ein Edgel gefunden, werden in den restlichen zwei Kanälen an der
	 gleichen Position nach einem Edgel gesucht. Falls in allen drei Kanälen ein Edgel gefunden wird, handelt es sich
	 um ein Linie (Schwarz/Weiß) und keine Farblinie.

	Nur das erste Frame wird vollständig untersucht und die Position der gefundenen Marken notiert. In den folgenden
	 frames wird nur in den Regionen der gefundenen Marke das Verfahren benutzt. Erst nach einer festgelegten Anzahl
	 von frames wird wieder ein vollständiger Durchlauf des Verfahrens durchgeführt.
\end{comment}

% subsection verfahren_nach_hirzer (end)

% section section_name (end)
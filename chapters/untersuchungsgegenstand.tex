\section{Untersuchungsgegenstand: Tracking Verfahren und Tracking Algorithmen} % (fold)
\label{sec:untersuchungsgegenstand}
\begin{comment}
	Untersuchungsgegenstand: Verfahren und Algorithmen präzise vorstellen und ihre Unterschiede hervorheben.
	Notwendige Kriterien der Algorithmen bestimmen

	Grober Ablauf der Verfahren:
	* Wer hats erfunden?
	* Wie ist das Verfahren aufgebaut (Algo in grob)
	* Welche Kriterien müssen erfüllt sein (monochrome, rgb eingabe)?
\end{comment}

\subsection{Verfahren nach Hirzer} % (fold)
\label{sub:verfahren_nach_hirzer}

Der Algorithmus von \citeauthor{hirzer08}\footcite{hirzer08} ist nach dem Vorbild der pixel connectivity edge linking based algorithms entworfen und ist in drei Hauptteile aufgebaut. Zuerst werden Liniensegmente erstellt, indem \gls{edgels} auf einem Suchraster gefunden und zusammengefürht werden. Die kurzen Liniensegmente werden dann zu längeren Linien zusammengeführt. Anschliessend werden im zweiten Schritt alle gefundenen Linien erweitert um die Gesamtlänage einer Linie zu erhalten. Im letzten Schritt werden die Linien zu Vierecken verbunden. Im weiteren Verlauf werden diese Schritte als Line Detection, Line Extension und Quadrangle Detection bezeichnet.

\subsubsection{Line Detection} % (fold)
\label{sub:line_detection}
Die Linienerkennung basiert auf dem Verfahren von \citeauthor{clarke96}\footcite{clarke96} und besteht aus zwei Schritten. Im ersten Schritt wird das Bildsignal grob abgetastet um im zweiten Schritt durch das RANSAC Verfaren eine Linienhypothese zu erstellen und zu bewerten.

Im ersten Schritt wird zuerst das Eingabesignal in $40 \times 40$ \gls{pixel} große Regionen unterteilt. Jede nachfolgende Operation erfolgt innerhalb einer Region. Eine Region wird wiederum unterteilt in horizontale und vertikale Scanlines, die jeweils $5$ \gls{pixel} Abstand zueinander haben. Jedes \gls{pixel} auf den Scanlines wird mit einem Gauß-Kernel gefaltet um die Komponente des Gradienten zu bestimmen. Ein lokales Maximum des Gradienten, dass größer als ein festgelegter Schwellwert ist, wird als \gls{edgels} betrachtet und seine Orientierung berechnet.

Im zweiten Schritt wird das RANSAC Verfahren verwendet, um aus der Menge der \gls{edgels} Liniensegmente zu bestimmen. Eine Linienhypothese wird durch die zufällige Auswahl zweier \gls{edgels} erstellt, deren Orientierung innerhalb eines Grenzwert liegen müssen.

Zwischen den \gls{edgels} wird eine Verbindung eingezogen.

Im Anschluss wird die Anzahl der \gls{edgels} betrachtet, die in der Nähe dieser Linie liegen und eine kompatible Orientierung mit der Linie aufweisen. Diese \gls{edgels} unterstützen die Hypothese einer Linie im Bildsignal. Die zufällige Auswahl zweier \gls{edgels} um eine Linie zu erstellen und deren Edgelunterstützung zu ermitteln wird mehrmals wiederholt um eine Linie mit der meisten Edgelunterstützung zu finden. Wenn eine solche dominante Linie gefunden wurde, gilt die Hypothese als bestätigt und die Linie ist wohl(? - deemed present) im Bild vorhanden. Die Edgels die zur Unterstüztung der Hyptohese der Linie galten, werden aus der Menge der Edgels entfernt und das Verfahren wird solange wiederholt, bis alle Liniensegmente gefunden wurden oder eine maximale Anzahl von Iterationen erreicht wurde.

\begin{algorithm}
	\caption{Line Detection nach \citeauthor{clarke96}}
	\label{src:lineDetection}
	\begin{algorithmic}[1]
	\Procedure{LineDetection}{$I_m$}
		\State $I_m$ in $40 \times 40$ Pixel große Regionen unterteilen
		\For{$i \gets 0$, alle Regionen}
			\State $Liste \gets \infty$
			\State $i$ in horizontale und vertikale Scanlines unterteilen mit $5$ Pixel Abstand
			\For{$j \gets 0$, alle Pixel auf allen Scanlines}
				\State $x \gets$ Falte $I_m\left(j\right)$ mit Kernel
				\If{$x > Schwellwert$}
					\Comment Edgel gefunden
					\State Orientierung des Edgels berechnen
					\State $Liste \gets edgel$
				\EndIf
			\EndFor
			\State $k \gets 0$
			\While{$Liste \not=0 \lor k <$ maximale Iterationen}
				\For{$j \gets 1, 25$}
					\State $Linie \gets$ zwei zufällige Edgels mit kompatibler Orientierung aus $Liste$ wählen
					\State Anzahl von Support-Edgels in der Nähe der Linie bestimmen
				\EndFor
				\If{Anzahl von Support-Edgels > Schwellwert}
					\Comment Linie wurde erkannt
					\State $Liste \gets Liste -$ Support-Edgels der Linie
				\EndIf
				\State $k \gets k + 1$
			\EndWhile
		\EndFor
	\EndProcedure
	\end{algorithmic}
\end{algorithm}

% subsubsection line_detection (end)

\subsubsection{Line Extension} % (fold)
\label{sub:line_extension}
% subsubsection line_extension (end)

\subsubsection{Quadrangle Detection} % (fold)
\label{sub:quadrangle_detection}
% subsubsection quadrangle_detection (end)

\begin{comment}
	Der RANSAC Grouper wird verwendet um gerade Liniensegmente zu finden. Dazu werden zwei zufällige edgels ausgewählt und ihre kompatible Orientierung überprüft. Die Anzahl der supporting edgels wird durch die Distanz des Liniensegments und der Orientierung bestimmt. Durch wiederholung dieses Prozesses wird die dominante Linie in der Region bestimmt. Die supporting edgels werden entfernt und der Prozess wird wiederholt bis keine edgels mehr vorhanden sind oder eine maximale Anzahl von wiederholungen erreicht wurde. Durch dieses Wiederholung wird sichergestellt, daß alle dominanten Linien in einer Region erkannt werden.

	Vorteil: Der Algorithmus ist sehr schnell und lässt sich für gewünschte Liniensegmente anpassen.
	Nachteil: Durch sein antisotropic detection verhalten diskriminiert das verfahren diagonale liniensegmente. Dies ist durch ein rechteckiges samplingrid bedingt.

	Hirzer hat das Verfahren um zwei Punkte erweitert und angepasst.
	Wird in einem RGB Bild ein Kanal untersucht und ein Edgel gefunden, werden in den restlichen zwei Kanälen an der gleichen Position nach einem Edgel gesucht. Falls in allen drei Kanälen ein Edgel gefunden wird, handelt es sich um ein Linie (schwarz/weiss) und keine Farblinie.

	Nur das erste Frame wird vollständig untersucht und die Position der gefundenen Marken notiert. In den folgenden frames wird nur in den Regionen der gefundenen Marke das Verfahren benutzt. Erst nach einer festgelegten Anzahl von frames wird wieder ein vollständiger durchlauf des Verfahrens durchgeführt.
\end{comment}

% subsection verfahren_nach_hirzer (end)

% section section_name (end)
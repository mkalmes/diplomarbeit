\section{Forschungsstand} % (fold)
\label{sec:forschungsstand}
\begin{comment}
	Forschungsstand: Alle untersuchten Arbeiten aufführen und kurz erklären.
\end{comment}

ARToolKit\footcite{artoolkit} wurde 1999 von Kato entwickelt und als Open Source Software einer breiten
 Entwicklergmeinde zur Verfügung gestellt. ARToolKit galt als Referenz der Forschung im Bereich der \gls{AR}.
 Allerdings war der Systementwurf nicht auf das Aufkommen von mobilen Computern und Smartphones vorbereitet.

Im Jahre 2004 wurde von \citeauthor{wagner04} ARToolKit für die Windows CE Plattform portiert\footcite{wagner04}. Diese
 Portierung führt zur Entwicklung von ARToolKitPlus\footcite{artoolkitplus}, das nicht nur für Workstations, sondern
 auch für mobile Geräte, entworfen wurde\footcite{wagner07b}. \citeauthor{wagner07a} machte diese Entwicklung zum Thema
 seiner Dissertation\footcite{wagner07a}. Die Schwächen von ARToolKit wurden überarbeitet und das System auf den
 aktuellen Stand der Forschung gebracht. Neue Verfahren zur robusten Mar\-ken\-er\-kennung in ARToolKitPlus waren aber
 nicht in der Lage mit der begrenzten Prozessorgeschwindigkeit mobiler Geräte ausgeführt zu werden.

Die Erfahrung von \citeauthor{wagner04} bei der Entwicklung von ARToolKitPlus wurde genutzt um
 Studierstube\footcite{studierstube} zu entwickeln. Studierstube ist das erste System, dass auf die Bedürfnisse von
 Smartphones und mobilen Geräten konzipiert wurde und von Grund auf neu entwickelt wurde. Anders als ARToolKit und
 ARToolKitPlus ist Studierstube nicht im Quellcode verfügbar.
\citeauthor{wagner09a} erläutern in \citetitle{wagner09a}\footcite{wagner09a} und
 \citetitle{wagner09b}\footcite{wagner09b} einen Überblick über die Strategien und Entwicklung von Studierstube.

In \citetitle{hirzer08}\footcite{hirzer08} wird von \citeauthor{hirzer08} ein Verfahren zur robosten Markenerkennung
 vorgestellt und basiert auf der Arbeit\footcite{clarke96} von \citeauthor{clarke96}. Die robuste Erkennung von Linien
 und die hohe Fehlertoleranz gegenüber schwankenden Lichtverhältnissen zeichnen dieses Verfahren aus um als Grundlage
 zur Markenerkennung verwendet zu werden.

% section forschungsstand (end)
\section{Forschungsstand} % (fold)
\label{sec:forschungsstand}
\begin{comment}
	Forschungsstand: Alle untersuchten Arbeiten aufführen und kurz erklären.
\end{comment}

ARToolKit\footcite{artoolkit} wurde 1999 von Kato entwickelt und war das erste \gls{AR}-System, dass als Open Source einer breiten Entwicklergemeinde zur Verfügung stand. ARToolKit galt als die Referenz für Forschung im Bereich der \gls{AR}. Allerdings war der Systementwurf nicht auf das Aufkommen von mobilen Computern und Smartphones vorbereitet, was dafür sorgte, dass das System in dieser Form für weitere Forschung uninteressant wurde.

ARToolKitPlus\footcite{artoolkitplus} ist eine Optimierung von ARToolKit und wurde für mobile Geräte angepasst. Die Schwächen von ARToolKit wurden entfernt und das System auf den neusten Stand der Forschung aktualisiert. Besondere Eigenschaften zur robusten Mar\-ken\-er\-kennung waren aber nicht in der Lage auf limitierten mobilen Geräten ausgeführt zu werden.

Die Erfahrung aus der Entwicklung von ARToolKitPlus wurde genutzt um Studierstube\footcite{studierstube} zu entwickeln. Studierstube ist das erste System, dass auf die Bedürfnisse von Smartphones und mobilen Geräten konzipiert wurde und von Grund auf neu entwickelt wurde. Studierstube ist nicht wie ARToolKit und ARToolKitPlus im Quellcode verfügbar.

Wagner und Schmalstieg waren an der Weiterentwicklung von ARToolKitPlus beteiligt und haben aus dieser Erfahrung heraus Studierstube entwickelt. Mit ihrer Ver\-öf\-fent\-li\-chung von \citetitle{wagner03}\footcite{wagner03} wird die frühe Entwicklung von mobilen \gls{AR}-Systemen beleuchtet. In \citetitle{wagner09a}\footcite{wagner09a} und \citetitle{wagner09b}\footcite{wagner09b} geben sie einen Einblick in ihre Arbeit, die dieser Untersuchung zu Grunde liegt.

In \citetitle{clarke96}\footcite{clarke96} wird ein Verfahren zur schnellen Linienerkennung vorgestellt. Die robuste Erkennung von Linien und die hohe Fehlertoleranz gegenüber schwankenden Lichtverhältnissen zeichnen dieses Verfahren aus um als Grundlage zur Markenerkennung verwendet zu werden.

% section forschungsstand (end)
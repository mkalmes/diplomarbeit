\subsubsection{Bildspeicher} % (fold)
\label{sub:artoolkitplus-bildspeicher}

Die Algorithmen zur Verwaltung des Bildspeichers werden zur Vollständigkeit erwähnt, da sie einmalig vor den Verfahren
 der Markenerkennung ausgeführt werden. Die Verfahren des Bildspeichers haben auf die Analyse keinen Einfluss.

% \autoref{alg:checkimagebuffer} ist beim Initialisieren der \textit{Runtime Tracking Pipeline} für die Bereitstellung
% des Bildspeichers zuständig.
% In Zeile \ref{alg:checkimagebuffer-size} wird die Größe des Eingangssignals mit Hilfe der globalen Variablen errechnet.
%  Danach wird in Zeile \ref{alg:checkimagebuffer-checksize-start}--\ref{alg:checkimagebuffer-checksize-end} die
%  errechnete Größe mit der Größe der globalen Variable verglichen. Wenn die Größe nicht übereinstimmt, wird in Zeile
%  \ref{alg:checkimagebuffer-checkmem-start}--\ref{alg:checkimagebuffer-checkmem-end} überprüft, ob der Speicher für ein
%  Bild bereits gesetzt ist. Falls der Speicherbereich schon gesetzt ist, muss er mit \autoref{alg:artkpfree} gelöscht
%  werden. In Zeile \ref{alg:checkimagebuffer-sizeglobal} wird der globalen Variable die berechnete Größe des
%  Bildspeichers zugewiesen. Zuletzt wird der globalen Variable des Bildspeichers ein neuer Speicherbereich in Zeile
%  \ref{alg:checkimagebuffer-newmem} zugewiesen.
\textproc{checkImageBuffer} (\autoref{alg:checkimagebuffer}) wird beim ersten Aufruf Speicher für das Bildsignal anlegen
 und den Algorithmus vollständig durchlaufen.
\begin{algorithm}[!ht]
\caption{\textproc{checkImageBuffer}}
\label{alg:checkimagebuffer}
\begin{algorithmic}[1]
	\State $\mathit{newSize} \gets \mathit{screenWidth} \cdot \mathit{screenHeight}$
	\label{alg:checkimagebuffer-size}
	\If{$\mathit{newSize} = \mathit{l\_imageL\_size}$}
	\label{alg:checkimagebuffer-checksize-start}
		\State \textbf{return}
	\EndIf
	\label{alg:checkimagebuffer-checksize-end}
	\If{$\mathit{l\_imageL}$}
	\label{alg:checkimagebuffer-checkmem-start}
		\State \Call{artkpFree}{$\mathit{l\_imageL}$}
	\EndIf
	\label{alg:checkimagebuffer-checkmem-end}
	\State $\mathit{l\_imageL\_size} \gets \mathit{newSize}$
	\label{alg:checkimagebuffer-sizeglobal}
	\State $\mathit{l\_imageL} \gets$ \Call{artkpAlloc}{$\mathit{newSize}$}
	\label{alg:checkimagebuffer-newmem}
\end{algorithmic}
\end{algorithm}

In allen weiteren Schritten, wenn der Bildspeicher angelegt ist, wird der Algorithmus lediglich die Größe des
 Bildspeichers berechnen und vergleichen
(Zeile \ref{alg:checkimagebuffer-size}--\ref{alg:checkimagebuffer-checksize-end}), was in konstanter Zeit erfolgt.
\autoref{alg:artkpfree} überprüft in Zeile \ref{alg:artkpfree-checkmem-start}--\ref{alg:artkpfree-checkmem-end} ob der
 Speicherbereich gültig ist.
\begin{algorithm}[ht]
\caption{\textproc{artkpFree}}
\label{alg:artkpfree}
\begin{algorithmic}[1]
	\Require $\mathit{rawMemory}$
	\If{$\neg\mathit{rawMemory}$}
	\label{alg:artkpfree-checkmem-start}
		\State \textbf{return}
	\EndIf
	\label{alg:artkpfree-checkmem-end}
	\State \Call{free}{$\mathit{rawMemory}$}
	\label{alg:artkpfree-deletemem-start}
	\State $\mathit{rawMemory} \gets \mathit{NULL}$
	\label{alg:artkpfree-deletemem-end}
\end{algorithmic}
\end{algorithm}

% Falls nicht wird die weitere Ausführung abgebrochen.
Nur im Falle, dass es sich um einen gültigen Speicherbereich handelt, wird in Zeile
 \ref{alg:artkpfree-deletemem-start}--\ref{alg:artkpfree-deletemem-end} der Speicher gelöscht und $\mathit{NULL}$
 zugewiesen. \autoref{alg:artkpalloc} alloziert den Speicherbereich für die benötigte Größe, die in Zeile
 \ref{alg:checkimagebuffer-size} von \autoref{alg:checkimagebuffer} berechnet wurde.
\begin{algorithm}[!ht]
\caption{\textproc{artkpAlloc}}
\label{alg:artkpalloc}
\begin{algorithmic}[1]
	\Require $\mathit{size}$
	\State $\mathit{rawMemory} \gets \mathit{size} \cdot$ \Call{sizeof}{$\mathit{short}$}
	\State \textbf{return} \Call{malloc}{$\mathit{rawMemory}$}
\end{algorithmic}
\end{algorithm}

Nachdem der Speicher das erste Mal angelegt wurde, werden weder \autoref{alg:artkpfree} noch \autoref{alg:artkpalloc}
 aufgerufen. Somit ist die Laufzeit, die zur Überprüfung der Größe des Bildspeichers verwendet wird, konstant.
% subsubsection bildspeicher (end)

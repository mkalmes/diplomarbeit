\subsubsection{Edgels} % (fold)
\label{sub:datenstruktur-edgels}

Die Datenstruktur eines \glspl{edgel} besteht aus der $x$- und $y$-Koordinate und der Orientierung
 (Vgl. \autoref{alg:datastructure-edgel}). Lese- und Schreibzugriffe auf die Elemente eines \glspl{edgel} sind konstant.

\begin{algorithm}[ht]
\caption{Datenstruktur eines edgels}
\label{alg:datastructure-edgel}
	\begin{algorithmic}[1]
		\State $x$
		\State $y$
		\State $o$
		\Comment Orientierung
	\end{algorithmic}
\end{algorithm}


Der Vergleich, ob \glspl{edgel} kompatibel sind, wird mit \autoref{alg:compatibleedgel} bewerkstelligt. Als Parameter
 werden zwei zu vergleichende \gls{edgel} $e_1$ und $e_2$ übergeben. In Zeile~$6$ und $13$ wird sichergestellt, dass
 $e_1.o$ und $e_2.o$ innerhalb von $67.5^\circ$\footcite[Vgl.][S.~417]{clarke96} liegen und damit kompatibel wären.
 Dies wird durch

\begin{equation}
	d = 2 \pi \left( \frac{ \frac{67.5}{2} }{360} \right) = 0.589
\end{equation}

überprüft. Es muss sichergestellt werden, dass die Orientierung in Bogenmaß erfolgt.

\begin{algorithm}[ht]
\caption{Orientierung von zwei edgels überprüfen}
\label{alg:compatibleedgel}
\begin{algorithmic}[1]
	\Require $e_1, e_2$
	\State $c \gets 0$
	\If{$e_1.o == e_2.o$}
		\State \textbf{return TRUE}
	\ElsIf{$e_1.o < e_2.o$}
		\State $c \gets e_2.o - e_1.o$
		\If{$c < d$}
			\State \textbf{return TRUE}
		\Else
			\State \textbf{return FALSE}
		\EndIf
	\Else
		\Comment $e_1.o > e_2.o$
		\State $c \gets e_1.o - e_2.o$
		\If{$c < d$}
			\State \textbf{return TRUE}
		\Else
			\State \textbf{return FALSE}
		\EndIf
	\EndIf
\end{algorithmic}
\end{algorithm}


Der Edgelspeicher in \autoref{alg:datastructure-edgelpool} verwendet ein Array von \glspl{edgel}
 (Vgl. \autoref{alg:datastructure-edgel}) mit fester Größe $N$ und einer Zählvariable um die nächste freie Position im
 Array zu markieren. Der Speicherpool aus \autoref{alg:datastructure-poolimplementation} ist ein Array vom Typ des
 Edgelspeichers mit der Größe $S$, dessen Adresse im Pointer $\mathit{pool}$ gespeichert wird.

\begin{algorithm}[ht]
\caption{Datenstruktur des Edgel-Speichers}
\label{alg:datastructure-edgelpool}
\begin{algorithmic}[1]
	\State $\mathit{data}[512]$
	\Comment Anzahl der Einträge
	\State $\mathit{count}$
\end{algorithmic}
\end{algorithm}

\autoref{alg:edgelpool-getmemorypools} basiert auf einem einfachen Stack Allocator von
 \citeauthor{kr}\footcite[Vgl.][S.~100--104]{kr}. Die Variable $n$ gibt die Anzahl der angeforderten Pools an. In Zeile
 \ref{alg:edgelpool-getmemorypools-checkpoolsize} wird überprüft, ob genügend Pools zur Verfügung stehen und liefert im
 Erfolgsfall die Adresse zu einem Speicher (\autoref{alg:datastructure-edgelpool}) zurück. Falls kein Pool mehr zur
 Verfügung steht, wird $\mathit{NULL}$ zurückgegeben. \autoref{alg:edgelpool-getmemorypool} vereinfacht die Anforderung
 eines Pools, da in den meisten Fällen nur ein Pool benötigt wird. Bei einem Aufruf kann somit auf einen Parameter
 verzichtet werden. Sowohl \autoref{alg:edgelpool-getmemorypools} als auch \autoref{alg:edgelpool-getmemorypool} haben
 eine konstante Laufzeit.

\begin{algorithm}[ht]
\caption{Hole Edgelpools}
\label{alg:edgelpool-getmemorypools}
\begin{algorithmic}[1]
	\Require $n$
	\If{$\mathit{data} + S - {pool} \geq n$}
		\State $pool = pool + n$
		\State \textbf{return} $\mathit{pool} - n$
	\Else
		\State \textbf{return} $\mathit{NULL}$
	\EndIf
\end{algorithmic}
\end{algorithm}

\begin{algorithm}[ht]
\caption{Hole Edgelpool}
\label{alg:edgelpool-getmemorypool}
\begin{algorithmic}[1]
	\State $p \gets$ \Call{getmemorypools}{1}
	\State \textbf{return} $p$
\end{algorithmic}
\end{algorithm}

Um \glspl{edgel} in einem Pool zu speichern, verwende ich \autoref{alg:edgelpool-addedgel}. Der Algorithmus benötigt
 einen Pointer $p$ auf einen Pool und einen \gls{edgel} $e$. In Zeile
 \ref{alg:edgelpool-addedgel-validpointer-start}--\ref{alg:edgelpool-addedgel-validpointer-end} wird geprüft, ob der
 Pointer auf eine Adresse verweist. Falls $p$ null ist, wird der Algorithmus verlassen. In Zeile
 \ref{alg:edgelpool-addedgel-checkspace-start}--\ref{alg:edgelpool-addedgel-checkspace-end} wird geprüft, ob im Array
 genügend Platz für einen weiteren Eintrag vorhanden ist. Die Größe von $N$ Einträgen richtet sich nach der in \autoref{alg:datastructure-edgelpool} festgelegten Arraygröße $N$. Wenn genügend Platz vorhanden ist, wird in Zeile
 \ref{alg:edgelpool-addedgel-add-start}--\ref{alg:edgelpool-addedgel-add-end} der \gls{edgel} $e$ an die freie
 Position $c$ geschrieben. Danach wird $\mathit{count}$ inkrementiert. Das Hinzufügen eines \glspl{edgel} ist konstant.

\begin{algorithm}[ht]
\caption{Edgel hinzufügen}
\label{alg:edgelpool-addedgel}
\begin{algorithmic}[1]
	\Require $p,e$
	\If{$!p$}
		\State \textbf{return}
	\EndIf
	\If{$!p(\mathit{count}) < N$}
		\State \textbf{return} \Comment Speicher voll
	\EndIf
	\State $c \gets p(\mathit{count})$
	\State $p(\mathit{data}[c]) \gets e$
	\State $p(\mathit{count}) \gets c + 1$
\end{algorithmic}
\end{algorithm}

\glspl{edgel} werden mittels \autoref{alg:edgelpool-getedgel} gelesen. Dazu wird der Pointer $p$ auf den Pool und der
 Index $i$ übergeben. In Zeile
 \ref{alg:edgelpool-getedgel-validpointer-start}--\ref{alg:edgelpool-getedgel-validpointer-end} wird geprüft, ob es
 sich um einen gesetzten Pointer handelt. Anschließend wird in Zeile
 \ref{alg:edgelpool-getedgel-validrange-start}--\ref{alg:edgelpool-getedgel-validrange-end} geprüft, ob der Index $i$
 innerhalb des gespeicherten Bereichs der \glspl{edgel} liegt. Danach wird in Zeile
 \ref{alg:edgelpool-getedgel-returnedgel} der Wert des \glspl{edgel} an Position $i$ zurückgegeben. Der Zugriff auf
 einen \gls{edgel} ist konstant.

\begin{algorithm}[ht]
\caption{Edgel lesen}
\label{alg:edgelpool-getedgel}
\begin{algorithmic}[1]
	\Require $p,i$
	\If{$!p$}
	\label{alg:edgelpool-getedgel-validpointer-start}
		\State \textbf{return}
	\EndIf
	\label{alg:edgelpool-getedgel-validpointer-end}
	\State $c \gets p(\mathit{count})$
	\If{$! c > i$}
	\label{alg:edgelpool-getedgel-validrange-start}
		\State \textbf{return}
	\EndIf
	\label{alg:edgelpool-getedgel-validrange-end}
	\State \textbf{return} $p(\mathit{data}[i])$
	\label{alg:edgelpool-getedgel-returnedgel}
\end{algorithmic}
\end{algorithm}

Damit \glspl{edgel} aus dem Array entfernt werden können, wird \autoref{alg:edgelpool-removeedgel} verwendet. Es wird
 der Pool-Pointer $p$ und die Position $i$ des zu löschenden \glspl{edgel} übergeben. Nach der Überprüfung des Pointers
 $p$ in Zeile~\ref{alg:edgelpool-removeedgel-validpointer-start}--\ref{alg:edgelpool-removeedgel-validpointer-end} und
 der Überprüfung des zulässigen Bereichs in Zeile
 \ref{alg:edgelpool-removeedgel-validrange-start}--\ref{alg:edgelpool-removeedgel-validrange-end}, gibt es zwei zu
 behandelnde Fälle um einen \gls{edgel} zu löschen.

\begin{algorithm}[ht]
\caption{Edgel löschen}
\label{alg:edgelpool-removeedgel}
\begin{algorithmic}[1]
	\Require $p,i$
	\If{$!p$}
		\State \textbf{return}
	\EndIf
	\State $c \gets p(\mathit{count})$
	\If{$! c > i$}
		\State \textbf{return}
	\EndIf
	\If{$c > i + 1$}
		\State \Call{memmove}{$p(\mathit{data}[i]), p(\mathit{data}[i + 1]), c - (i + 1) * \textproc{sizeof}(e)$}
	\EndIf
	\State $p(\mathit{count}) = c - 1$
\end{algorithmic}
\end{algorithm}

Der Edgel liegt
\begin{enumerate}
	\item nicht am Ende des Arrays oder \label{removeedgel-worst}
	\item liegt am Ende des Arrays. \label{removeedgel-best}
\end{enumerate}

Bei \autoref{removeedgel-best} muss lediglich $\mathit{count}$ dekrementiert werden um auf den vorigen Wert zu verweisen
 (Vgl. \autoref{fig:decrementcounter}). Das dekrementieren der Zählvariable $p(\mathit{count})$ ist eine Zuweisung in
 konstanter Zeit.

\begin{figure}[!ht]
	\centering
	\subfigure[]{
		\input{resources/Memory-Decrement-Before.pdf_tex}
		\label{fig:decrementcounter-before}
	}
	\subfigure[]{
		\input{resources/Memory-Decrement-After.pdf_tex}
		\label{fig:decrementcounter-after}
	}
	\caption{Dekrementieren von $\mathit{count}$. In \subref{fig:decrementcounter-before} soll Position $i$ gelöscht
	 werden. $c$ verweist auf die nächste freie Speicherstelle. In \subref{fig:decrementcounter-after} wird $c$
	 dekrementiert und verweist auf die neue freie Speicherstelle.}
	\label{fig:decrementcounter}
\end{figure}

Bei \autoref{removeedgel-worst} wird das Array an der Position $i$ geteilt und der Wertebereich von $[i+1 \dotsc i-n]$
 wird an die Position $i$ verschoben (Vgl. \autoref{fig:memmove}). % TODO: Grafik [1][2]..[i-1][i][i+1]..[i-n]
In Zeile \ref{alg:edgelpool-removeedgel-memmove} gibt die Funktion \textproc{sizeof}($e$) die Speichergröße eines
 \glspl{edgel} an, welche zum verschieben der Daten notwendig ist. Mit $c - (i + 1)$ wird die Anzahl der zu
 verschiebenden Einträge ermittelt. Im worst-case werden $N-1$ Einträge an Position $0$ des Arrays verschoben.

Um die Laufzeit der Funktion \textproc{memmove} zu bestimmen, wurde ein Testprogramm geschrieben, dass die Zeit misst,
 die benötigt wird, um Einträge zu verschieben. Anhand der Daten wurde mittels einer Regressionsanalyse untersucht, ob
 die gemessenen Daten einen linearen Zusammenhang aufweisen. Die erfassten $2000$ Datenpunkte wurden nach dem Vorbild von
 \textproc{time} % TODO: Referenz auf den Katalog für /usr/bin/time und man-page
ermittelt um Real-, User- und Sys-Zeit zu bestimmen. Aus User- und Sys-Zeit wurde die CPU-Zeit bestimmt, die zur
 Analyse benutzt wurde. Die Kovarianz für $X = \mathit{BYTES}$ und $Y = \mathit{CPU}$ beträgt $r = 0.9937538$ und
 $r^2 = 0.9875$. An dieser Stelle sei darauf hingewiesen, dass die Kovarianz für $X = \mathit{BYTES}$ und
 $Y = \mathit{REAL}$ mit $r = 0.9981969$ zwar größer ist, aber nicht die tatsächlichen Operationen des Testprogramms
 untersucht. Aus diesem Grund wurde $Y = \mathit{CPU}$ untersucht. Der Interzept beträgt $\beta_0 = -71.89\e{-06}$
 (Abweichung von $71.98\e{-06}$) und die Steigung $\beta_1 = 4.410\e{-09}$ (Abweichung von $11.08\e{-12}$), sodass

\begin{subequations}
\begin{multline}
	y =\\ \beta_0 + \beta_1n
\end{multline}
\begin{multline}
	y =\\ -71.89\e{-06} + 4.410\e{-09}n
\end{multline}
\end{subequations} % TODO: Sauber formatieren

Daraus ergibt sich eine Laufzeit von $\Theta(n)$. In \autoref{fig:regression-memmove} ist der Plot der Daten angegeben.

\begin{figure}[!ht]
	\centering
	\includegraphics[width=.8\textwidth]{resources/Regression-memmove.png}
	\caption{Regressionsanalyse von $2000$ Datenpunkten.}
	\label{fig:regression-memmove}
\end{figure}

Um einen Pool für einen neuen Durchlauf zu löschen, kommt \autoref{alg:edgelpool-resetmemorypool} zum Einsatz. Als
 Parameter wird der Pointer $p$ übergeben und in Zeile
 \ref{alg:edgelpool-resetmemorypool-validpointer-start}--\ref{alg:edgelpool-resetmemorypool-validpointer-end}
 überprüft. Um alle Daten als gelöscht zu markieren, wird lediglich die Zählvariable in Zeile
 \ref{alg:edgelpool-resetmemorypool-reset} auf $0$ gesetzt. Die Zuweisung erfolgt in konstanter Zeit.

\begin{algorithm}[ht]
\caption{Lösche Daten in Edgelpool}
\label{alg:edgelpool-resetmemorypool}
\begin{algorithmic}[1]
	\Require $p$
	\If{$!p$}
	\label{alg:edgelpool-resetmemorypool-validpointer-start}
		\State \textbf{return}
	\EndIf
	\label{alg:edgelpool-resetmemorypool-validpointer-end}
	\State $p(\mathit{count}) \gets 0$
	\label{alg:edgelpool-resetmemorypool-reset}
\end{algorithmic}
\end{algorithm}

Wenn ein Speicherpool nicht mehr benötigt wird, kann er mit \autoref{alg:edgelpool-freememorypool} freigegeben werden.
 Der Pointer $p$ wird in Zeile
 \ref{alg:edgelpool-freememorypool-validpointer-start}--\ref{alg:edgelpool-freememorypool-validpointer-end} überprüft.
 Zeile~\ref{alg:edgelpool-freememorypool-resetmemory} werden die Daten des Pools gelöscht
 (Vgl. \autoref{alg:edgelpool-resetmemorypool}). Im Anschluss wird in Zeile
 \ref{alg:edgelpool-freememorypool-checkpointer} überprüft, ob $p$ zu dem Array $\mathit{data}$ gehört und nicht größer
 als die definierte Speichergröße ist. Wenn der Test positiv ausfällt, wird der Pointer $p$ zur weiteren Verwendung in
 $\mathit{pool}$ gespeichert. Das Freigeben eines Pools erfolgt in konstanter Zeit.

\begin{algorithm}[ht]
\caption{Speicher des Edgelpool freigeben}
\label{alg:edgelpool-freememorypool}
\begin{algorithmic}[1]
	\Require $p$
	\If{$!p$}
	\label{alg:edgelpool-freememorypool-validpointer-start}
		\State \textbf{return}
	\EndIf
	\label{alg:edgelpool-freememorypool-validpointer-end}
	\State \Call{resetmemorypool}{$p$}
	\label{alg:edgelpool-freememorypool-resetmemory}
	\If{$p \geq \mathit{data} \land p \leq \mathit{data} + S$}
	\label{alg:edgelpool-freememorypool-checkpointer}
		\State $\mathit{pool} \gets p$
	\EndIf
\end{algorithmic}
\end{algorithm}

Die Anzahl der \glspl{edgel} in einem Pool werden durch \autoref{alg:edgelpool-count} ermittelt. Als Parameter wird
 Pointer $p$ übergeben und in Zeile
 \ref{alg:edgelpool-count-validpointer-start}--\ref{alg:edgelpool-count-validpointer-end} überprüft. Die Anzahl der
 Einträge wird in Zeile~\ref{alg:edgelpool-count-counter} über die Zählvariable $p(\mathit{count})$ ermittelt. Der
 Zugriff auf die Variable, und somit die Laufzeit des Algorithmus, erfolgt in konstanter Zeit.

\begin{algorithm}[ht]
\caption{Anzahl der Einträge}
\label{alg:edgelpool-count}
\begin{algorithmic}[1]
	\Require $p$
	\If{$!p$}
		\State \textbf{return}
	\EndIf
	\State \textbf{return} $p(\mathit{count})$
\end{algorithmic}
\end{algorithm}

% subsection datenstruktur-edgels (end)

Bei dem letzten Schritt im Verfahren von \citeauthor{hirzer08}, der Quadrangle Detection, werden die gefundenen Linien
 untersucht, um daraus eine quadratische Marke zu erkennen. Dazu wird das Verfahren \textproc{findChainsOfLines}
 verwendet, dass in \autoref{alg:findchainoflines1}--\autoref{alg:findchainoflines4} dargestellt ist.

\begin{algorithm}[ht]
\caption{\textproc{findChainOfLines}}
\label{alg:findchainoflines1}
\begin{algorithmic}[1]
	\Require $B,\mathit{start},\mathit{fromStart},\mathit{chainLength},C$
	\If{$\mathit{fromStart}$}
		\State $\mathit{startPoint} \gets \mathit{start \to start.coordinate}$
	\Else
		\State $\mathit{startPoint} \gets \mathit{start \to end.coordinate}$
	\EndIf
	\For{$i \gets 0$ \textbf{to} $i < \mathit{B \to count}$}
		\If{\Call{isOrientationCompatible}{$\mathit{start},\mathit{B \to data}[i]$}}
			\State \textbf{continue}
		\EndIf
		\algstore{brk-findchainoflines1}
\end{algorithmic}
\end{algorithm}


Das Verfahren benötigt einen Speicherblock $B$ mit Linien, eine Startlinie $\mathit{start}$, sowie eine boolsche
 Aussage $\mathit{fromStart}$, ob der Anfang der Linie oder das Ende der Linie untersucht werden soll. Zusätzlich wird
 die zu untersuchende Länge einer Linie benötigt und ein Speicherblock $C$, in dem eine Linienkette gespeichert wird.

In Zeile \ref{alg:findchainoflines1-fromstart}--\ref{alg:findchainoflines1-fromend} wird der Startpunkt der zu
 untersuchenden Linie festgelegt und ist von der boolschen Variable $\mathit{fromStart}$ abhängig. Danach wird in Zeile
 \ref{alg:findchainoflines1-loop-start}--\ref{alg:findchainoflines4-loop-end} jede Linie aus $B$ untersucht. In Zeile
 \ref{alg:findchainoflines1-iscompatible} wird überprüft, ob Linie $\mathit{start}$ und die Linie an Position $i$ des
 Speicherblocks fast parallel zueinander stehen. Falls dem so ist wird in Zeile \ref{alg:findchainoflines1-parallel}
 die Untersuchung der nächsten Linie eingeleitet. Nur wenn die Linien nicht paralle ausgerichtet sind, wird das
 Verfahren in \autoref{alg:findchainoflines2} fortgesetzt.

\begin{algorithm}[!ht]\small
\caption{\textproc{findChainOfLines} (Abstand der Linienenden)}
\label{alg:findchainoflines2}
\begin{algorithmic}[1]
	\algrestore{brk-findchainoflines1}
		\If{$\mathit{fromStart}$}
		\Cost{$c_{10}$}{$n$}
		\label{alg:findchainoflines2-endpointend}
			\State $\mathit{endpoint} \gets \mathit{B.data}[i]\mathit{.end.coordinate}$
			\Cost{$c_{11}$}{$5n$}
		\Else
			\State $\mathit{endpoint} \gets \mathit{B.data}[i]\mathit{.start.coordinate}$
			\Cost{$c_{13}$}{$5n$}
		\EndIf
		\label{alg:findchainoflines2-endpointstart}
		\State $\mathit{distance} \gets$ \Call{vectorSubtract}{$\mathit{startPoint},\mathit{endpoint}$}
		\Cost{$c_{15}$}{$n\bigl(1 + \Theta(1)\bigr)$}
		\label{alg:findchainoflines2-distance}
		\State $\mathit{squaredLength} \gets$ \Call{squaredLength}{$\mathit{distance}$}
		\Cost{$c_{16}$}{$n\bigl(1 + \Theta(1)\bigr)$}
		\label{alg:findchainoflines2-length}
		\If{$\mathit{squaredLength} > 16$}
		\Cost{$c_{17}$}{$n$}
			\State \textbf{continue}
			\Cost{$c_{18}$}{$n$}
			\label{alg:findchainoflines2-toofar}
		\EndIf
	\algstore{brk-findchainoflines2}
\end{algorithmic}
\end{algorithm}


In Zeile \ref{alg:findchainoflines2-endpointend}--\ref{alg:findchainoflines2-endpointstart} wird nun der Endpunkt der
 Linie $i$ festgelegt. Wie beim Startpunkt auch, ist der Endpunkt von der Variablen $\mathit{fromStart}$ abhängig. In
 Zeile \ref{alg:findchainoflines2-distance} und Zeile \ref{alg:findchainoflines2-length} wird dann der Abstand der
 Linienenden von $\mathit{start}$ und $i$ berechnet. Wenn der Abstand zu groß ist, wird mit der nächsten Linie die
 Untersuchung wiederholt (Zeile \ref{alg:findchainoflines2-toofar}). Andernfalls wird das Verfahren in
 \autoref{alg:findchainoflines3} fortgesetzt.

\begin{algorithm}[ht]
\caption{\textproc{findChainOfLines} (Fortsetzung)}
\label{alg:findchainoflines3}
\begin{algorithmic}[1]
	\algrestore{brk-findchainoflines2}
		\State $\mathit{test} \gets (\mathit{start \to slope.x} \cdot L\mathit{\to data}[i]\mathit{.slope.y})$
		\State $\mathit{test} \gets \mathit{test} - (\mathit{start \to slope.y} \cdot L\mathit{\to data}[i]\mathit{.slope.x})$
		\If{$(\mathit{fromStart} \land \mathit{test} \leq 0) \lor (\lnot \mathit{fromStart} \land \mathit{test} \geq 0)$}
			\State \textbf{continue}
		\EndIf
		\State $\mathit{chainLength} \gets \mathit{chainLength} + 1$
		\State $\mathit{chainSegment} \gets L\mathit{\to data}[i]$
		\State \Call{removeLine}{$L,i$}
		\If{$\mathit{chainLength} == 4$}
			\State \Call{addLine}{$C,\mathit{chainSegment}$}
			\State \textbf{return}
		\EndIf
	\algstore{brk-findchainoflines3}
\end{algorithmic}
\end{algorithm}


In Zeile \ref{alg:findchainoflines3-vectorproduct-start}--\ref{alg:findchainoflines3-vectorproduct-end} wird das
 Kreuzprodukt der Linienorientierung berechnet um danach in Zeile \ref{alg:findchainoflines3-checkorientation} die
 Orientierung zu überprüfen. Je nachdem ob $\mathit{fromStart}$ gesetzt ist oder nicht, wird die Orientierung
 unterschiedlich betrachtet. Wenn $\mathit{fromStart}$ wahr ist, muss die Orientierung der beiden Linien kleiner oder
 gleich $0$ sein. Damit wird die Orientierung als negativer Winkel betrachtet und der Wertebereich entspricht
 $\left[-1,0\right]$. Andernfalls, wenn $\mathit{fromStart}$ falsch ist, muss die Orientierung größer oder gleich $0$
 sein. In diesem Fall wird die Orientierung als positiver Winkel betrachtet und der Wertebereich entspricht
 $\left[0,1\right]$. Werden diese Bedingungen erfüllt, eignet sich diese Linienkombination nicht zur Markenerkennung
 und es wird eine andere Linie untersucht. Falls die Bedingungen nicht erfüllt werden, eigent sich diese
 Linienkombination zur Erkennung einer Marke. In diesem Fall wird in Zeile \ref{alg:findchainoflines3-incchain} die
 Anzahl der Linienkette erhöht und die Linie an Position $i$ in der Variable $\mathit{chainSegment}$ hinterlegt. In
 Zeile \ref{alg:findchainoflines3-removeline} wird danach die Linie an Position $i$ aus dem Speicherblock $B$ entfernt.
 Danach wird in Zeile \ref{alg:findchainoflines3-has4lines} die Anzahl der Linienketten untersucht. Wenn genau vier
 Linienketten vorhanden sind, wird in Zeile \ref{alg:findchainoflines3-saveline} die Linie in $\mathit{chainSegment}$
 zum Speicherblock $C$ hinzugefügt und danach das Verfahren beendet. Anonsten wird in Zeile
 \ref{alg:findchainoflines4-savelineifnotfromstart} (\autoref{alg:findchainoflines4}) die Linie zum Speicherblock
 hinzugefügt, wenn $\mathit{fromStart}$ falsch ist.

\begin{algorithm}[!ht]\small
\caption{\textproc{findChainOfLines} (Fortsetzung)}
\label{alg:findchainoflines4}
\begin{algorithmic}[1]
	\algrestore{brk-findchainoflines3}
		\If{$\lnot \mathit{fromStart}$}
		\Cost{$c_{32}$}{$n$}
			\State \Call{addLine}{$C,chainSegment$}
			\Cost{$c_{33}$}{$n\cdot\Theta(1)$}
			\label{alg:findchainoflines4-savelineifnotfromstart}
		\EndIf
		\State \Call{findChainOfLines}{$B,\mathit{chainSegment},\mathit{fromStart},\mathit{chainLength},C$}
		\label{alg:findchainoflines4-callmethod}
		\If{$\mathit{fromStart}$}
		\Cost{$c_{36}$}{$n$}
		\label{alg:findchainoflines4-isfromstart}
			\State \Call{addLine}{$C,\mathit{chainSegment}$}
			\Cost{$c_{37}$}{$n\cdot\Theta(1)$}
		\EndIf
		\State \textbf{return}
		\Cost{$c_{33}$}{$n$}
		\State $i \gets i + 1$
		\Cost{$c_{33}$}{$n$}
	\EndFor
	\label{alg:findchainoflines4-loop-end}
\end{algorithmic}
\end{algorithm}


Danach wird in Zeile \ref{alg:findchainoflines4-callmethod} $\mathit{findChainsOfLines}$ rekursiv aufgerufen, um
 weitere passende Linien zur Markenerkennung zu identifizieren. Abschliessend wird in Zeile
 \ref{alg:findchainoflines4-isfromstart} überprüft, ob $\mathit{fromStart}$ wahr ist, was dazu führt, dass die Linie in
 $\mathit{chainSegment}$ dem Speicherblock $C$ hinzugefügt wird. Danach wird die Ausführung abgebrochen.

Das Verfahren \textproc{findChainOfLines} ist in \textproc{lineDetection}
 (\autoref{alg:linedetection-hirzerquaddetection}) integriert, um aus den gefundenen Linien eine Marke zu konstruieren.

\algblockdefx[DOWHILE]{DO}{WHILE}
   {\textbf{do}}
   [1]{\textbf{while} #1}
\begin{algorithm}[ht]
\caption{\textproc{lineDetection} mit Quadraterkennung}
\label{alg:linedetection-hirzerquaddetection}
	\begin{algorithmic}[1]
		\Require $I$
		\For{$y \gets 0$ \textbf{to} $y < \mathit{imageHeight}$}
			\For{$x \gets 0$ \textbf{to} $x < \mathit{imageWidth}$}
				\State \Call{findEdgels}{$I,E,x,y$}
				\State \Call{findLineSegments}{$E,L$}
				\State \Call{mergeLines}{$L$}
				\State $n \gets$ \Call{getLineCount}{$L$}
				\For{$i \gets 0$ \textbf{to} $i < n$}
					\State \Call{addLineSegment}{$M,$\textproc{getLineSegment}$(L,i)$}
					\State $i \gets i + 1$
				\EndFor
				\State \Call{resetMemoryPool}{$E$}
				\State \Call{resetMemoryPool}{$L$}
				\State $ x \gets x + 40$
			\EndFor
			\State $y \gets y + 40$
		\EndFor
		\State \Call{mergeLines}{$M$}
		\State \Call{extendLinesInPool}{$M$}
		\State \Call{findLinesWithCornersInPool}{$M,L$}
		\DO
		\label{alg:linedetection-hirzerquaddetection-do}
			\State \Call{resetMemoryPool}{$C$}
			\State $\mathit{chain} \gets L\mathit{\to data}[0]$
			\State \Call{removeLine}{$L,0$}
			\label{alg:linedetection-hirzerquaddetection-removeline}
			\State $\mathit{length} \gets 1$
			\State \Call{findChainOfLines}{$L,\mathit{chain},$\textbf{true}$,\mathit{length},C$}
			\label{alg:linedetection-hirzerquaddetection-find1}
			\State \Call{addLine}{$C,\mathit{chain}$}
			\label{alg:linedetection-hirzerquaddetection-addline}
			\If{$\mathit{length} < 4$}
			\label{alg:linedetection-hirzerquaddetection-find2-start}
				\State \Call{findChainOfLines}{$L,\mathit{chain},$\textbf{false}$,\mathit{length},C$}
			\EndIf
			\label{alg:linedetection-hirzerquaddetection-find2-end}
			\If{$\mathit{length} > 2$}
			\label{alg:linedetection-hirzerquaddetection-twolines}
				\State $\mathit{marker} \gets$ \Call{reconstructCorners}{$C$}
				\label{alg:linedetection-hirzerquaddetection-reconstruct}
				\State \Call{addMarker}{$Q,\mathit{marker}$}
				\label{alg:linedetection-hirzerquaddetection-marker}
			\EndIf
		\WHILE{\Call{getLineCount}{$L$}}
		\label{alg:linedetection-hirzerquaddetection-while}
	\end{algorithmic}
\end{algorithm}


Dazu werden alle Linien in der Schleife von Zeile
 \ref{alg:linedetection-hirzerquaddetection-do}--\ref{alg:linedetection-hirzerquaddetection-while} untersucht. Der
 Speicherblock $C$ wir zubeginn jeder Iteration gelöscht, um nur Linien einer Marke zu speichern. Danach wird aus dem
 Speicherblock $L$ die erste Linie ausgelesen und in $\mathit{chain}$ gespeichert. In Zeile
 \ref{alg:linedetection-hirzerquaddetection-removeline} wird die Linie aus dem Speicherblock $L$ entfernt. Die Variable
 $\mathit{length}$ wird anschliessend initialisiert und enthält die Anzahl der Linien einer potentiellen Marke. Mit
 \textproc{findChainOfLines} wird in Zeile \ref{alg:linedetection-hirzerquaddetection-find1}, von der Linie
 $\mathit{chain}$ ausgehend, im Uhrzeigersinn nach Linien gesucht, die zu einem Quadrat zusammengefügt werden können.
 Danach wird $\mathit{chain}$ in Zeile \ref{alg:linedetection-hirzerquaddetection-addline} zu dem Speicherblock $C$
 hinzugefügt. Wenn weniger als vier Linien durch \textproc{findChainOfLines} gefunden wurden, wird das Verfahren
 wiederholt und sucht gegen den Uhrzerigersinn nach weiterten Linien (Zeile
 \ref{alg:linedetection-hirzerquaddetection-find2-start}--\ref{alg:linedetection-hirzerquaddetection-find2-end}). In
 Zeile \ref{alg:linedetection-hirzerquaddetection-twolines} wird überprüft, ob mehr als zwei Linien gefunden wurden.
 Nur wenn mehr als zwei Linien vorhanden sind, lassen sich die Eckpunkte einer Marke in Zeile
 \ref{alg:linedetection-hirzerquaddetection-reconstruct} durch \textproc{reconstructCorners} wiederherstellen. Danach
 wird in Zeile \ref{alg:linedetection-hirzerquaddetection-marker} die Marke $\mathit{marker}$ dem Speicherblock $Q$
 hinzugefügt. Das Verfahren wird solange wiederholt, bis keine Linie mehr im Speicherblock vorhanden ist. Danach sind
 erkannte Marken in Speicherblock $Q$ hinterlegt.

Mit \textproc{reconstructCorners} (\autoref{alg:reconstructcorners}) werden die Eckpunkte einer Marke aufbereitet. Dazu
 benötigt das Verfahren den Speicherblock $C$, mit den Linien der Marke.

\begin{algorithm}[ht]
\caption{\textproc{reconstructCorners}}
\label{alg:reconstructcorners}
\begin{algorithmic}[1]
	\Require $C$
	\State $\mathit{marker} \gets \infty$
	\State $\mathit{c1} \gets$ \Call{intersection}{$\mathit{C.data}[0],\mathit{C.data}[1]$}
	\label{alg:reconstructcorners-c1}
	\State $\mathit{c2} \gets$ \Call{intersection}{$\mathit{C.data}[1],\mathit{C.data}[2]$}
	\label{alg:reconstructcorners-c2}
	\State $\mathit{c3} \gets \infty$
	\State $\mathit{c4} \gets \infty$
	\If{\Call{getLineCount}{$C$}$ == 4$}
	\label{alg:reconstructcorners-4lines-start}
		\State $\mathit{c3} \gets$ \Call{intersection}{$\mathit{C.data}[2],\mathit{C.data}[3]$}
		\State $\mathit{c4} \gets$ \Call{intersection}{$\mathit{C.data}[3],\mathit{C.data}[0]$}
	\label{alg:reconstructcorners-4lines-end}
	\Else
	\label{alg:reconstructcorners-else-start}
		\State $\mathit{c3} \gets \mathit{C.data}[2]\mathit{.end.coordinate}$
		\State $\mathit{c4} \gets \mathit{C.data}[0]\mathit{.start.coordinate}$
	\EndIf
	\label{alg:reconstructcorners-else-end}
	\State $\mathit{marker.c1} \gets \mathit{c1}$
	\label{alg:reconstructcorners-return-start}
	\State $\mathit{marker.c2} \gets \mathit{c2}$
	\State $\mathit{marker.c3} \gets \mathit{c3}$
	\State $\mathit{marker.c4} \gets \mathit{c4}$
	\State \textbf{return} $\mathit{marker}$
	\label{alg:reconstructcorners-return-end}
\end{algorithmic}
\end{algorithm}


In Zeile \ref{alg:reconstructcorners-c1} wird der erste Eckpunkt erstellt, indem der Schnittpunkt der Linien an
 Position $1$ und Position $2$ ermittelt wird. Für den zweiten Eckpunkt in Zeile \ref{alg:reconstructcorners-c2} werden
 die Linien $2$ und $3$ verwendet. Wenn exakt vier Linien gefunden wurden, werden die Eckpunkte für $\mathit{c3}$ und
 $\mathit{c4}$ durch den Schnittpunkt der Linien ermittelt (Zeile
 \ref{alg:reconstructcorners-4lines-start}--\ref{alg:reconstructcorners-4lines-end}). Ansonsten werden die Eckpunkte in
 Zeile \ref{alg:reconstructcorners-else-start}--\ref{alg:reconstructcorners-else-end} nur durch die Linienenden
 dargestellt werden, was möglicherweise zu ungenauen Ergebnissen führt (Vgl. \autoref{sub:quadrangle_detection}). In
 Zeile \ref{alg:reconstructcorners-return-start}--\ref{alg:reconstructcorners-return-end} werden die Eckpunkte in
 $\mathit{marker}$ gespeichert und als Ergebnis an die aufrufende Methode zurückgeliefert.

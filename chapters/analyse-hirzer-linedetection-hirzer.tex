Das Verfahren zur Linienerkennung nach \citeauthor{hirzer08} basiert, wie in \autoref{sub:line_detection} bereits
 erwähnt, auf dem Verfahren von \citeauthor{clarke96}. \citeauthor{hirzer08} verwendet in seinem Verfahren anstatt
 eines monochromen Bildsignals ein RGB-Signal. Dadurch ändern sich einige Algorithmen geringfügig und werden im
 folgenden Abschnitt erläutert.

Das Verfahren in \autoref{alg:linedetection-hirzer} unterscheidet sich von \autoref{alg:linedetection-clarke}, dass
 anstatt dem monochromen Signal $I_m$ das RGB-Signal $I$ verwendet wird. Dadurch verändern sich auch die Verfahren
 \textproc{findEdgels} und \textproc{findLineSegments}.

\begin{algorithm}[!ht]\small
\caption{\textproc{lineDetection} nach Hirzer}
\label{alg:linedetection-hirzer}
	\begin{algorithmic}[1]
		\Require $I$
		\For{$y \gets 0$ \textbf{to} $y < \mathit{imageHeight}$}
			\For{$x \gets 0$ \textbf{to} $x < \mathit{imageWidth}$}
				\State \Call{findEdgels}{$I,E,x,y$}
				\State \Call{findLineSegments}{$E,L$}
				\State \ldots \Comment Speichern der Liniensegmente zur weiteren Verarbeitung
				\State \Call{resetMemoryPool}{$E$}
				\State \Call{resetMemoryPool}{$L$}
				\State $ x \gets x + 40$
			\EndFor
			\State $y \gets y + 40$
		\EndFor
	\end{algorithmic}
\end{algorithm}


In dem Verfahren zur Erstellung von \gls{edgel} in \autoref{alg:findedgelshirzer-1}--\autoref{alg:findedgelshirzer-2}
 wird im Gegensatz zu \autoref{alg:findedgels-horizontal} die Faltung für alle drei Farbkanäle durchgeführt (Vgl. Zeile
 \ref{alg:findedgelshirzer-1-color-start}--\ref{alg:findedgelshirzer-1-color-end} in \autoref{alg:findedgelshirzer-1}
 und Zeile \ref{alg:findedgelshirzer-2-color-start}--\ref{alg:findedgelshirzer-2-color-end} in
 \autoref{alg:findedgelshirzer-2}).

\begin{algorithm}[ht]
\caption{\textproc{findEdgels} (Hirzer)}
\label{alg:findedgelshirzer-1}
	\begin{algorithmic}[1]
		\Require $I, E,\mathit{left},\mathit{top}$
		\For{$y \gets \mathit{top}$ \textbf{to} $y < \mathit{top} + \mathit{regionSize}$}
			\State $\mathit{prev1} \gets 0$
			\State $\mathit{prev2} \gets 0$
			\For{$x \gets \mathit{left}$ \textbf{to} $x < \mathit{left} + \mathit{regionSize}$}
				\State $\mathit{currentEdgel} \gets$ \Call{convoluteKernelX}{$I,x,y,\mathit{blue},\mathit{imageWidth}, \mathit{imageHeight}$}
				\State $\mathit{red} \gets$ \Call{convoluteKernelX}{$I,x,y,\mathit{red},\mathit{imageWidth}, \mathit{imageHeight}$}
				\State $\mathit{green} \gets$ \Call{convoluteKernelX}{$I,x,y,\mathit{green},\mathit{imageWidth}, \mathit{imageHeight}$}
				\If{$\mathit{currentEdgel} > \mathit{threshold} \land \mathit{red} > \mathit{threshold} \land \mathit{green} > \mathit{threshold}$}
				\Comment Möglicherweise ein Edgel
				\Else
					\State $\mathit{currentEdgel} \gets 0$
				\EndIf
				\If{$\mathit{prev1} > 0 \land \mathit{prev2} > \mathit{prev1} \land \mathit{prev1} > \mathit{currentEdgel}$}
					\State $\mathit{edgel} \gets \infty$
					\State \Call{vectorSetCoordinate}{$\mathit{edgel},x - 1,y$}
					\State $\mathit{edgel.slope} \gets$ \Call{gradientIntensity}{$I_m, \mathit{imageWidth}, \mathit{imageHeight}, x - 1,y$}
					\State \Call{addEdgel}{$E,\mathit{edgel}$}
				\EndIf
				\State $\mathit{prev2} \gets \mathit{prev1}$
				\State $\mathit{prev1} \gets \mathit{currentEdgel}$
				\State $x \gets x + 1$
			\EndFor
			\State $y \gets y + 5$
		\EndFor
		\algstore{brk-findedgels-x}
	\end{algorithmic}
\end{algorithm}

\begin{algorithm}[!ht]
\caption{\textproc{findEdgels} (Fortsetzung)}
\label{alg:findedgelshirzer-2}
	\begin{algorithmic}[1]
		\algrestore{brk-findedgels-x}
		\For{$x \gets \mathit{left}$ \textbf{to} $x < \mathit{left} + \mathit{regionSize}$}
			\State $\mathit{prev1} \gets 0$
			\State $\mathit{prev2} \gets 0$
			\For{$y \gets \mathit{top}$ \textbf{to} $y < \mathit{top} + \mathit{regionSize}$}
				\State $\mathit{currentEdgel} \gets$ \textproc{convoluteKernelY}
				$\left(
				\begin{aligned}
					& I,x,y,\mathit{blue},\\
					& \mathit{imageWidth},\mathit{imageHeight}
				\end{aligned}\right)$
				\label{alg:findedgelshirzer-2-color-start}
				\State $\mathit{red} \gets$ \Call{convoluteKernelY}{$I,x,y,\mathit{red},\mathit{imageWidth},
				 \mathit{imageHeight}$}
				\State $\mathit{green} \gets$ \Call{convoluteKernelY}{$I,x,y,\mathit{green},\mathit{imageWidth},
				 \mathit{imageHeight}$}
				\label{alg:findedgelshirzer-2-color-end}
				\If{$\mathit{currentEdgel} > \mathit{threshold} \land \mathit{red} > \mathit{threshold} \land
				 \mathit{green} > \mathit{threshold}$}
					\State Möglicherweise ein Edgel
				\Else
					\State $\mathit{currentEdgel} \gets 0$
				\EndIf
				\If{$\mathit{prev1} > 0 \land \mathit{prev2} > \mathit{prev1} \land \mathit{prev1} >
				 \mathit{currentEdgel}$}
				\State $\mathit{edgel} \gets \infty$
				\State \Call{vectorSetCoordinate}{$\mathit{edgel},x,y - 1$}
				\State $\mathit{edgel.slope} \gets$ \textproc{gradientIntensity}
					$\left(
					\begin{aligned}
						& I_m, \mathit{imageWidth},\\
						& \mathit{imageHeight}, x,y - 1
					\end{aligned}\right)$
				\State \Call{addEdgel}{$E,\mathit{edgel}$}
				\EndIf
				\State $\mathit{prev2} \gets \mathit{prev1}$
				\State $\mathit{prev1} \gets \mathit{currentEdgel}$
				\State $y \gets y + 1$
			\EndFor
			\State $y \gets y + 5$
		\EndFor
	\end{algorithmic}
\end{algorithm}


Betrachtet man die Methode \textproc{convoluteKernelX} und \textproc{convoluteKernelY} stellt man fest, dass hier eine
 Angabe zum Farbkanal erfolgt. In \autoref{alg:convolutekernelhirzer-horizontal} und
 \autoref{alg:convolutekernelhirzer-vertical} wird durch den Aufruf der Methode \textproc{getRGBValue} ein bestimmter
 Farbkanal betrachtet.

\begin{algorithm}[!ht]
\caption{\textproc{convoluteKernelX} (horizontale Scanline nach Hirzer)}
\label{alg:convolutekernelhirzer-horizontal}
\begin{algorithmic}[1]
	\Require $I,x,y,\mathit{color},w,h$
	\State $p_1 \gets$ \Call{getRGBValue}{$I,x-2,y,\mathit{color},w,h$}
	\label{alg:convolutekernel-horizontal-readstart}
	\State $p_2 \gets$ \Call{getRGBValue}{$I,x-1,y,\mathit{color},w,h$}
	\State $p_4 \gets$ \Call{getRGBValue}{$I,x+1,y,\mathit{color},w,h$}
	\State $p_5 \gets$ \Call{getRGBValue}{$I,x+2,y,\mathit{color},w,h$}
	\label{alg:convolutekernel-horizontal-readend}
	\State $v \gets 0$
	\State $v \gets v + \left( -3 \cdot p_1 \right)$
	\State $v \gets v + \left( -5 \cdot p_2 \right)$
	% \State $v \gets v + \left( 0 \cdot p_3 \right)$
	\State $v \gets v + \left( 5 \cdot p_4 \right)$
	\State $v \gets v + \left( 3 \cdot p_5 \right)$
	\State \textbf{return} $v$
\end{algorithmic}
\end{algorithm}

\begin{algorithm}[!ht]
\caption{\textproc{convoluteKernelY} (vertikale Scanline nach Hirzer)}
\label{alg:convolutekernelhirzer-vertical}
\begin{algorithmic}[1]
	\Require $I,x,y,\mathit{color},w,h$
	\State $p_1 \gets$ \Call{getRGBValue}{$I,x,y-2,\mathit{color},w,h$}
	\State $p_2 \gets$ \Call{getRGBValue}{$I,x,y-1,\mathit{color},w,h$}
	\State $p_4 \gets$ \Call{getRGBValue}{$I,x,y+1,\mathit{color},w,h$}
	\State $p_5 \gets$ \Call{getRGBValue}{$I,x,y+2,\mathit{color},w,h$}
	\State $v \gets 0$
	\State $v \gets v + \left( -3 \cdot p_1 \right)$
	\State $v \gets v + \left( -5 \cdot p_2 \right)$
	% \State $v \gets v + \left( 0 \cdot p_3 \right)$
	\State $v \gets v + \left( 5 \cdot p_4 \right)$
	\State $v \gets v + \left( 3 \cdot p_5 \right)$
	\State \textbf{return} $v$
\end{algorithmic}
\end{algorithm}


Auch das Verfahren \textproc{gradientIntensity} in \autoref{alg:gradientintensityhirzer} verwendet die Methode
 \textproc{getRGBValue}.

\input{alg/analyse-hirzer/gradientIntensityhirzer}

\textproc{getRGBValue} dient dem auslesen eines \gls{pixel} für einen angegeben Farbkanal. In \autoref{alg:getrgbvalue} ist das Verfahren beschrieben.

\begin{algorithm}[!ht]
\caption{\textproc{getRGBValue}}
\label{alg:getrgbvalue}
\begin{algorithmic}[1]
	\Require $I,x,y,\mathit{color},w,h$
	\If{$x < 0$}
	\Cost{$c_{1}$}{$1$}
	\label{alg:getrgbvalue-sanity-start}
		\State $x \gets 0$
		\Cost{$c_{2}$}{$1$}
	\EndIf
	\If{$y < 0$}
	\Cost{$c_{4}$}{$1$}
		\State $y \gets 0$
		\Cost{$c_{5}$}{$1$}
	\EndIf
	\If{$x \geq w$}
	\Cost{$c_{7}$}{$1$}
		\State $x \gets w - 1$
		\Cost{$c_{8}$}{$2$}
	\EndIf
	\If{$y \geq h$}
	\Cost{$c_{10}$}{$1$}
		\State $y \gets h -1$
		\Cost{$c_{11}$}{$2$}
	\EndIf
	\label{alg:getrgbvalue-sanity-end}
	\State $\mathit{offset} \gets (x + (y \cdot w)) \cdot \left(\textproc{sizeof}(\mathit{char}) \cdot 4\right)$
	\Cost{$c_{13}$}{$6$}
	\label{alg:getrgbvalue-offset}
	\State $\mathit{address} \gets I + \mathit{offset}$
	\Cost{$c_{14}$}{$2$}
	\label{alg:getrgbvalue-address}
	\State \textbf{return} $\mathit{address} + \mathit{color}$
	\Cost{$c_{13}$}{$2$}
	\label{alg:getrgbvalue-returncolor}
\end{algorithmic}
\end{algorithm}


Das Verfahren benötigt das RGB-Signal $I$, die Position des \gls{pixel} ($x$, $y$), den Farbkanal, sowie die Breite $w$
 und Höhe $h$ von $I$. In den Zeilen \ref{alg:getrgbvalue-sanity-start}--\ref{alg:getrgbvalue-sanity-end} wird
 sichergestellt, dass kein \gls{pixel} ausserhalb der Bildgrenzen gelesen werden können. In Zeile
 \ref{alg:getrgbvalue-offset} wird der Adressabstand für ein \gls{pixel} berechnet. Die Adresse des \gls{pixel} wird in
 Zeile \ref{alg:getrgbvalue-address} aus der Adresse von $I$ und dem Adressabstand berechnet. Zuletzt wird der Wert des
 \gls{pixel} in Zeile \ref{alg:getrgbvalue-returncolor} für den angegeben Farbkanal zurückgegeben.

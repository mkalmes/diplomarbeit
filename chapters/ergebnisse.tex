\chapter{Ergebnisse} % (fold)
\label{cha:ergebnisse}
\begin{comment}
	Ergebnisse: Die gewonnen Daten aus Kap. Analyse bewerten.
\end{comment}

In diesem Kapitel werden die gewonnen Daten aus \autoref{cha:analyse} bewertet. Bei beiden Verfahren ist die Eingabemenge ein Bildsignal der Abmessung $640 \times 480$ \gls{pixel}.

Das asymptotische Wachstum von ARToolKitPlus besteht aus den Verfahren \textproc{regionLabeling} und
 \textproc{arDetectMarker2} und kann als $O(\mathit{lxsize}\cdot\mathit{lysize})
 + \Theta(\mathit{label\_num}\cdot n \log n)$ beschrieben werden, wobei $\mathit{lxsize}\cdot\mathit{lysize}$ die Größe
 des reduzierten Bildsignals darstellt (Vgl. \autoref{par:artoolkitplus}). Bei dem Verfahren von
 \citeauthor{hirzer08} ist das asymptotische Wachstum für das Verfahren \textproc{lineDetection}
 $\Theta(hwmn^2+hwl^2\cdot\mathit{length})$. Auch hier ist $h \cdot w$ die Größe des Bildsignals.
Beide Verfahren sind somit abhängig von der Eingabemenge des Bildsignals.

Die tatsächliche Laufzeit der beiden Verfahren wurde durch Messung der Verarbeitungszeit ermittelt. Die Ergebnisse der Untersuchung sind in \autoref{fig:runtime} grafisch dargestellt.
\begin{figure}[!ht]
	\centering
	\input{resources/runtime.pdf_tex}
	\caption{Messergebnis der Verarbeitungszeit. Die Verarbeitungszeit von ARToolKitPlus ist als blaue Linie
	 eingezeichnet und die Zeit von dem Verfahren nach \citeauthor{hirzer08} als rote Linie.}
	\label{fig:runtime}
\end{figure}
 Bei der Untersuchung wurde zuerst nur die Zeit gemessen, die benötigt wurde um das Videosignal anzufordern. Im zweiten
 Schritt wurde das Videosignal durch die erste Stufe der Verfahren verarbeitet. Bei ARToolKitPlus wurde die Zeit
 ermittelt, die das Verfahren \textproc{regionLabeling} benötigt, um eine Regionenmarkierung zu erstellen. Bei dem
 Verfahren von \citeauthor{hirzer08} wurde die Verarbeitungszeit der Linienerkennung gemessen. Bei der letzten
 Untersuchung wurde für beide Verfahren die gesammte Verarbeitungszeit für ein Bildsignal gemessen. Für jeden
 Untersuchungsschritt wurden $300$ Frames prozessiert und der Median der Verarbeitungszeit ermittelt. Bei beiden
 Verfahren liegt der Median der gemessenen Zeit bei der Bildsignalbeschaffung bei $30$ FPS. In der zweiten
 Untersuchung, der Vorverarbeitungsschritt der Verfahren, fällt die Zeit bei beiden Verfahren ab. Bei ARToolKitPlus
 liegt die Verabeitungszeit bei $19.1$ FPS, während bei dem Verfahren nach \citeauthor{hirzer08} die Verarbeitungszeit
 auf $15.8$ FPS absinkt. Bei der letzten Untersuchung, der vollständigen Verarbeitung des Bildsignals und der Erkennung
 einer Marke, liegt die Verarbeitungszeit von ARToolKitPlus bei $19.1$ FPS. Dies ist die gleiche Geschwindigkeit, wie
 der Untersuchung der Regionenmarkierung. Bei dem Verfahren nach \citeauthor{hirzer08} beträgt die Geschwindigkeit
 $15.7$ FPS, und ist nur geringfügig langsamer als in der vorherigen Untersuchung.

% chapter ergebnisse (end)
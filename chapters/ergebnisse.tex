\chapter{Ergebnisse} % (fold)
\label{cha:ergebnisse}
\begin{comment}
	Ergebnisse: Die gewonnen Daten aus Kap. Analyse bewerten.
\end{comment}

In diesem Kapitel werden die gewonnen Daten aus \autoref{cha:analyse} bewertet.

ARToolKitPlus benötigt zur Erkennung einer Marke das Verfahren \textproc{regionLabeling} und
 \textproc{arDetectMarker2}. In \autoref{sec:artoolkitplus} wurde das asymptotische Wachstum des Verfahrens für den
 schlechtesten Fall auf
\begin{equation*}
O(\mathit{lxsize}\cdot\mathit{lysize})  + \Theta(\mathit{label\_num}\cdot n \log n)
\end{equation*}
bestimmt. Die Eingabemenge für das Verfahren ist das Bildsignal $I$, dass in der Untersuchung aus $640 \times 480$
 Pixeln besteht. Als Optimierung des Verfahrens werden nur die Hälfte der horizontalen \gls{pixel} ($\mathit{lxsize}$)
 und die Hälfte der vertikalen \gls{pixel} ($\mathit{lysize}$) prozessiert (Vgl. \autoref{par:artoolkitplus}).
 $\mathit{label\_num}$ ist durch $\tfrac{\mathit{lxsize}\cdot\mathit{lysize}}{4}$ begrenzt und beschreibt die maximal
 mögliche Anzahl von Regionen in einem Bildsignal mit der Größe $\mathit{lxsize}\cdot\mathit{lysize}$. Variable $n$ in
 $n \log n$ beschreibt die Anzahl der Verarbeitungsschritte in der Konturverfolgung. Die Konstante
 $\mathit{AR\_CHAIN\_MAX} = 10000$ schränkt $n$ ein, sodass eine Konturverfolgung abgebrochen wird, falls nach $10000$
 untersuchten \gls{pixel} keine geschlossene Kontur gefunden werden konnte.

Das Verfahren von \citeauthor{hirzer08} besteht aus dem Algorithmus \textproc{lineDetection}
 (\autoref{alg:linedetection-hirzerquaddetection}), um eine Marke zu erkennen. Das asymptotische Wachstum des
 Verfahrens wurde, für den schlechtesten Fall, in \autoref{sec:hirzer} auf
\begin{equation*}
\Theta(hwmn^2+hwl^2\cdot\mathit{length})
\end{equation*}
bestimmt. Auch hier ist die Eingabemenge des Verfahrens das Bildsignal $I$, mit einer Bildgröße von $640 \times 480$
 \gls{pixel}. Die Anzahl der horizontalen \gls{pixel} wird durch die Variable $w$ und die Anzahl der vertikalen
 \gls{pixel} durch die Variable $h$ beschrieben. Variable $n$ bezeichnet die Anzahl der maximal möglichen Edgel und ist
 begrenzt auf $\tfrac{1}{5}(\mathit{regionSize}^2 - 2\mathit{regionSize})$, mit $\mathit{regionSize} = 40$. Die Anzahl
 der maximal möglichen Wiederholungen des RANSAC Verfahrens (\autoref{sub:linienerkennung_nach_clarke96}) ist durch
 $m = \tfrac{1}{2}(\sqrt{8n + 25} - 5)$ begrenzt. Die Variable $l$ bezeichnet die Menge der Liniensegmente und
 $\mathit{length}$ bezeichnet die Länge des zu verarbeitenden Liniensegments.

Beide Tracking Verfahren verarbeiten die gleiche Eingabemenge $I$ eines Bildsignals und berechnen für jede im
 Bildsignal vorhandene Marke vier Eckpunkte. Beide Verfahren sind von der Höhe und der Breite des Bildsignals abhängig.
 Durch die unterschiedlichen Arbeitsweisen der Verfahren wird das Bildsignal $I$ in nicht vergleichbare Eingabemengen
 aufgeteilt. Dadurch bedingt ist eine Analyse anhand des asymptotischen Wachstums nicht eindeutig. Da beide Verfahren
 ein RGB Signal mit $640 \times 480$ \gls{pixel} zu vier Eckpunkten einer Marke verarbeiten, kann durch die Messung
 der Verarbeitungszeit die Verfahren verglichen werden.

Für die Untersuchung der Laufzeit wurden zuerst $300$ Bilder vom Kamerasensor angefordert ohne sie von den Verfahren
 prozessieren zu lassen. Die Ergebnisse sind in \autoref{tab:image} angegeben.
\begin{table}[!ht]
	\begin{center}
	\begin{tabular}[]{r..}
	\toprule
	& \multicolumn{1} {>{\centering\arraybackslash}m{4cm}}{ARToolKitPlus}
	& \multicolumn{1} {>{\centering\arraybackslash}m{4cm}}{\citeauthor{hirzer08}} \\
	\midrule
	Median		& 30.00  & 30.00  \\
	Mittelwert	& 30.12  & 30.09  \\
	Max.		& 252.60 & 218.00 \\
	Min.		& 5.10   & 9.60   \\
	\bottomrule
	\end{tabular}
	\caption{Messergebnisse der Bilddaten im Überblick.}
	\label{tab:image}
	\end{center}
\end{table}
Danach wurde in einer zweiten Messung $300$ Bilder durch die Verfahren verarbeitet. Die Ergebnisse der Markenerkennung
 sind in \autoref{tab:marker} aufgeführt. Für die Untersuchung der Markenerkennung wurde eine Marke auf einem hellen
 Untergrund platziert und mittig ausgerichtet. Die Erstellung der Messpunkte wurde für beide Verfahren unter gleichen
 Bedingungen durchgeführt.
\begin{table}[!ht]
	\begin{center}
	\begin{tabular}[]{r..}
	\toprule
	& \multicolumn{1} {>{\centering\arraybackslash}m{4cm}}{ARToolKitPlus}
	& \multicolumn{1} {>{\centering\arraybackslash}m{4cm}}{\citeauthor{hirzer08}} \\
	\midrule
	Median		& 19.10  & 15.70 \\
	Mittelwert	& 18.29  & 15.81 \\
	Max.		& 20.50  & 20.10 \\
	Min.		& 1.40   & 5.10  \\
	\bottomrule
	\end{tabular}
	\caption{Messergebnisse der Markenerkennung im Überblick.}
	\label{tab:marker}
	\end{center}
\end{table}


% Wie die Ergebnisse zeigen,
%  Untersuchung wurde für beide Verfahren die gesamte Verarbeitungszeit für ein Bildsignal gemessen. Für jeden
%  Untersuchungsschritt wurden $300$ Frames prozessiert und der Median der Verarbeitungszeit ermittelt. Bei beiden
%  Verfahren liegt der Median der gemessenen Zeit bei der Bildsignalbeschaffung bei $30$ FPS. In der zweiten
%  Untersuchung, der Vorverarbeitungsschritt der Verfahren, fällt die Zeit bei beiden Verfahren ab. Bei ARToolKitPlus
%  liegt die Verarbeitungszeit bei $19.1$ FPS, während bei dem Verfahren nach \citeauthor{hirzer08} die Verarbeitungszeit
%  auf $15.8$ FPS absinkt. Bei der letzten Untersuchung, der vollständigen Verarbeitung des Bildsignals und der Erkennung
%  einer Marke, liegt die Verarbeitungszeit von ARToolKitPlus bei $19.1$ FPS. Dies ist die gleiche Geschwindigkeit, wie
%  der Untersuchung der Regionenmarkierung. Bei dem Verfahren nach \citeauthor{hirzer08} beträgt die Geschwindigkeit
%  $15.7$ FPS, und ist nur geringfügig langsamer als in der vorherigen Untersuchung.

% chapter ergebnisse (end)
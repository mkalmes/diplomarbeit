\section{ARToolKitPlus} % (fold)
\label{sec:artoolkitplus}

\subsection{Datenstrukturen} % (fold)
\label{sec:datenstrukturen}

ARToolKitPlus speichert die Informationen einer Marke in zwei einfachen Datenstrukturen. Informationen zur
 Identifizierung einer Marke werden in \textproc{MarkerInfo} gespeichert (\autoref{alg:datastructure-markerinfo}) und
 Informationen zur Erkennung einer Marke werden in \textproc{MarkerInfo2} gespeichert
 (\autoref{alg:datastructure-markerinfo2}). Da \textproc{MarkerInfo} zur Identifizierung einer Marke verwendet wird,
 werde ich diese Datenstruktur nur zur Vollständigkeit erwähnen. Das Identifikationsmerkmal $\mathit{id}$ wird in
 \autoref{alg:calc} als Rückgabewert verwendet.

\begin{algorithm}[ht]
\caption{MarkerInfo}
\label{alg:datastructure-markerinfo}
\begin{algorithmic}[1]
	\State $\mathit{area}$
	\State $\mathit{id}$
	\State $\mathit{dir}$
	\State $\mathit{cf}$
	\State $\mathit{pos}[2]$
	\State $\mathit{line}[4][3]$
	\State $\mathit{vertex}[4][2]$
\end{algorithmic}
\end{algorithm}


\subsubsection{MakerInfo2} % (fold)
\label{sec:makerinfo2}

Die Variable $\mathit{area}$ speichert den Flächeninhalt einer Marke, während $\mathit{pos}[2]$ die Position des
 Zentrums der Marke enthält. $\mathit{coord\_num}$ enthält die Anzahl der gefundenen Konturpixel, die in
 $\mathit{x\_coord}$ als X-Koordinate und in $\mathit{y\_coord}$ als Y-Koordinate gespeichert sind. Die konstante Größe
 $\mathit{AR\_CHAIN\_MAX}$ des Speichers für die Koordinaten wird zur Laufzeit nicht verändert. ARToolKitPlus erlaubt
 maximal $10000$ Einträge für Koordinaten pro Marke. Die Eckpunkte einer Marke sind in $\mathit{x\_coord}$ und
 $\mathit{y\_coord}$ enthalten und werden durch einen Index in $\mathit{vertex}$ referenziert. Hierbei ist zu beachten,
 dass $\mathit{vertex}$ fünf Einträge speichert, wobei der erste und letzte Eintrag auf die gleiche Koordinate
 verweisen. Dadurch kann bei der Grafikprogrammierung mit OpenGL sehr einfach ein Rahmen um eine Marke gezeichnet
 werden.

\begin{algorithm}[!ht]
\caption{MarkerInfo2}
\label{alg:datastructure-markerinfo2}
\begin{algorithmic}[1]
	\State $\mathit{area}$
	\State $\mathit{pos}[2]$
	\State $\mathit{coord\_num}$
	\State $\mathit{x\_coord}[\mathit{AR\_CHAIN\_MAX}]$
	\State $\mathit{y\_coord}[\mathit{AR\_CHAIN\_MAX}]$
	\State $\mathit{vertex}[5]$
\end{algorithmic}
\end{algorithm}


Der Zugriff auf die Variablen der Datenstruktur, sowohl lesend als auch schreibend, ist konstant.

% subsubsection makerinfo2 (end)

% subsection datenstrukturen (end)

\subsection{Fiducial Detection} % (fold)
\label{sec:fiducial_detection}

% subsection fiducial_detection (end)

\subsection{Rectangle Fitting} % (fold)
\label{sec:rectangle_fitting}

% subsection rectangle_fitting (end)

% section artoolkitplus (end)


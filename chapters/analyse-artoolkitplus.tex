\section{ARToolKitPlus} % (fold)
\label{sec:artoolkitplus}

Die \textit{Runtime Tracking Pipeline} (Vgl. \autoref{sub:artoolkitplus}, S.~\pageref{sub:artoolkitplus}) von
 ARToolKitPlus wird mit der Methode \textproc{calc} (Vgl. \autoref{alg:calc}) gestartet und benötigt das Bildsignal
 $I$, sowie den Schwellwert $\mathit{tresh}$. Die Methode \textproc{calc} dient in erster Linie als Einstiegspunkt der
 \textit{Runtime Tracking Pipeline}. Verarbeitungsschritte die nicht zur Fiducial Detection oder Rectangle
 Fitting gehören werden nicht aufgeführt.

\begin{algorithm}[ht]
\caption{\textproc{calc}}
\label{alg:calc}
\begin{algorithmic}[1]
	\Require $I, \mathit{thresh}$
	\State $\mathit{marker\_info} \gets \infty$
	\label{alg:calc-init-start}
	\State $\mathit{marker\_num} \gets \infty$
	\label{alg:calc-init-end}
	\If{$I = 0$}
	\label{alg:calc-check-image-start}
		\State \textbf{return} $0$
	\EndIf
	\label{alg:calc-check-image-end}
	\If{\Call{arDetectMarker}{$I,\mathit{thresh},\mathit{marker\_info},\mathit{marker\_num}$} $<0$}
	\label{alg:calc-start-detection-start}
		\State \textbf{return} $-1$
	\EndIf
	\label{alg:calc-start-detection-end}
	\State \ldots \Comment{Weitere Anweisungen zur Identifikation einer Marke.}
	\State \textbf{return} $\mathit{marker\_info}.\mathit{id}$
	\label{alg:calc-identified-marker}
\end{algorithmic}
\end{algorithm}


Die Variablen $\mathit{marker\_info}$ und $\mathit{marker\_num}$ werden in Zeile
 \ref{alg:calc-init-start}--\ref{alg:calc-init-end} initialisiert. Bei $\mathit{marker\_info}$ handelt es sich um die
 Datenstruktur \textproc{MarkerInfo} (Vgl. \autoref{alg:datastructure-markerinfo}), die der Identifikation einer Marke
 dient. In Zeile \ref{alg:calc-check-image-start}--\ref{alg:calc-check-image-end} wird überprüft, ob das Bildsignal
 nicht $0$ ist und es sich somit um einen gesetzten Zeiger auf einen Speicherbereich handelt. Falls dies nicht der Fall
 ist, wird das Verfahren abgebrochen und ein Fehlerwert von $0$ zurückgeliefert. Anschließend wird in Zeile
 \ref{alg:calc-start-detection-start}--\ref{alg:calc-start-detection-end} die Methode \textproc{arDetectMarker}
 aufgerufen und ihr Rückgabewert überprüft. Falls während der Ausführung von \textproc{arDetectMarker} einen Fehler
 aufgetreten ist, wird die weitere Verarbeitung gestoppt und ein Fehlerwert von $-1$ zurückgegeben. Wenn
 \textproc{arDetectMarker} erfolgreich war und eine Marke identifizieren konnte, wird die Verarbeitung fortgesetzt und
 letztendlich in Zeile \ref{alg:calc-identified-marker} eine Markenidentifizierung zurückgegeben. Die Laufzeit des
 Algorithmus ist konstant.

\subsection{Datenstrukturen} % (fold)
\label{sec:datenstrukturen}

ARToolKitPlus speichert die Informationen einer Marke in zwei einfachen Datenstrukturen. Informationen zur
 Identifizierung einer Marke werden in \textproc{MarkerInfo} gespeichert (\autoref{alg:datastructure-markerinfo}) und
 Informationen zur Erkennung einer Marke werden in \textproc{MarkerInfo2} gespeichert
 (\autoref{alg:datastructure-markerinfo2}). Da \textproc{MarkerInfo} zur Identifizierung einer Marke verwendet wird,
 werde ich diese Datenstruktur nur zur Vollständigkeit erwähnen. Das Identifikationsmerkmal $\mathit{id}$ wird in
 \autoref{alg:calc} als Rückgabewert verwendet.

\begin{algorithm}[ht]
\caption{MarkerInfo}
\label{alg:datastructure-markerinfo}
\begin{algorithmic}[1]
	\State $\mathit{area}$
	\State $\mathit{id}$
	\State $\mathit{dir}$
	\State $\mathit{cf}$
	\State $\mathit{pos}[2]$
	\State $\mathit{line}[4][3]$
	\State $\mathit{vertex}[4][2]$
\end{algorithmic}
\end{algorithm}


\subsubsection{MakerInfo2} % (fold)
\label{sec:makerinfo2}

Die Variable $\mathit{area}$ speichert den Flächeninhalt einer Marke, während $\mathit{pos}[2]$ die Position des
 Zentrums der Marke enthält. $\mathit{coord\_num}$ enthält die Anzahl der gefundenen Konturpixel, die in
 $\mathit{x\_coord}$ als $x$-Koordinate und in $\mathit{y\_coord}$ als $y$-Koordinate gespeichert sind. Die konstante
 Größe $\mathit{AR\_CHAIN\_MAX}$ des Speichers für die Koordinaten wird zur Laufzeit nicht verändert. ARToolKitPlus
 erlaubt maximal $10000$ Einträge für Koordinaten pro Marke. Die Eckpunkte einer Marke sind in $\mathit{x\_coord}$ und
 $\mathit{y\_coord}$ enthalten und werden durch einen Index in $\mathit{vertex}$ referenziert. Hierbei ist zu beachten,
 dass $\mathit{vertex}$ fünf Einträge speichert, wobei der erste und letzte Eintrag auf die gleiche Koordinate
 verweisen. Dadurch kann bei der Grafikprogrammierung mit OpenGL sehr einfach ein Rahmen um eine Marke gezeichnet
 werden.

\begin{algorithm}[!ht]\small
\caption{MarkerInfo2}
\label{alg:datastructure-markerinfo2}
\begin{algorithmic}[1]
	\State $\mathit{area}$
	\State $\mathit{pos}[2]$
	\State $\mathit{coord\_num}$
	\State $\mathit{x\_coord}[\mathit{AR\_CHAIN\_MAX}]$
	\State $\mathit{y\_coord}[\mathit{AR\_CHAIN\_MAX}]$
	\State $\mathit{vertex}[5]$
\end{algorithmic}
\end{algorithm}


Der Zugriff auf die Variablen der Datenstruktur, sowohl lesend als auch schreibend, ist konstant.

% subsubsection makerinfo2 (end)

\subsubsection{Automatischer Schwellwert} % (fold)
\label{sec:automatischer_schwellwert}

Die automatische Schwellwertanalyse von ARToolKitPlus wird durch \autoref{alg:autothresholdreset} vor ihrer Verwendung
 auf bekannte Startwerte gesetzt. Dabei wird in Zeile \ref{alg:autothresholdreset-minLum} die globale Variable
 $\mathit{minLum}$ gesetzt und in Zeile \ref{alg:autothresholdreset-maxLum} die globale Variable $\mathit{maxLum}$
 gesetzt. Die statischen Größen $\mathit{MINLUM0}$ und $\mathit{MAXLUM0}$ geben dabei die Werte an. Die Laufzeit von
 \autoref{alg:autothresholdreset} ist konstant.

\begin{algorithm}[ht]
\caption{\textproc{autothreshold.reset}}
\label{alg:autothresholdreset}
\begin{algorithmic}[1]
	\State $\mathit{minLum} \gets \mathit{MINLUM0}$
	\label{alg:autothresholdreset-minLum}
	\State $\mathit{maxLum} \gets \mathit{MAXLUM0}$
	\label{alg:autothresholdreset-maxLum}
\end{algorithmic}
\end{algorithm}


% subsubsection automatischer_schwellwert (end)

\subsubsection{Bildspeicher} % (fold)
\label{sec:bildspeicher}

Die Algorithmen zur Verwaltung des Bildspeichers werden zur Vollständigkeit erwähnt. Da die folgenden Algorithmen
 einmalig vor den Verfahren der Markenerkennung ausgeführt werden, haben sie auf die Analyse keinen Einfluß.

\autoref{alg:checkimagebuffer} ist beim Initialisieren der \textit{Runtime Tracking Pipeline} für die Bereitstellung
 des Bildspeichers zuständig. In Zeile \ref{alg:checkimagebuffer-size} wird die Größe des Eingangssignals mit Hilfe der
 globalen Variablen errechnet. Danach wird in Zeile
 \ref{alg:checkimagebuffer-checksize-start}--\ref{alg:checkimagebuffer-checksize-end} die errechnete Größe mit der
 Größe der globalen Variable verglichen. Wenn die Größe nicht übereinstimmt, wird in Zeile
 \ref{alg:checkimagebuffer-checkmem-start}--\ref{alg:checkimagebuffer-checkmem-end} überprüft, ob der Speicher für ein
 Bild bereits gesetzt ist. Falls der Speicherbereich schon gesetzt ist, muss er mit \autoref{alg:artkpfree} gelöscht
 werden. In Zeile \ref{alg:checkimagebuffer-sizeglobal} wird der globalen Variable die berechnete Größe des
 Bildspeichers zugewiesen. Zuletzt wird der globalen Variable des Bildspeichers ein neuer Speicherbereich in Zeile
 \ref{alg:checkimagebuffer-newmem} zugewiesen.

\begin{algorithm}[!ht]
\caption{\textproc{checkImageBuffer}}
\label{alg:checkimagebuffer}
\begin{algorithmic}[1]
	\State $\mathit{newSize} \gets \mathit{screenWidth} \cdot \mathit{screenHeight}$
	\label{alg:checkimagebuffer-size}
	\If{$\mathit{newSize} = \mathit{l\_imageL\_size}$}
	\label{alg:checkimagebuffer-checksize-start}
		\State \textbf{return}
	\EndIf
	\label{alg:checkimagebuffer-checksize-end}
	\If{$\mathit{l\_imageL}$}
	\label{alg:checkimagebuffer-checkmem-start}
		\State \Call{artkpFree}{$\mathit{l\_imageL}$}
	\EndIf
	\label{alg:checkimagebuffer-checkmem-end}
	\State $\mathit{l\_imageL\_size} \gets \mathit{newSize}$
	\label{alg:checkimagebuffer-sizeglobal}
	\State $\mathit{l\_imageL} \gets$ \Call{artkpAlloc}{$\mathit{newSize}$}
	\label{alg:checkimagebuffer-newmem}
\end{algorithmic}
\end{algorithm}


\autoref{alg:checkimagebuffer} wird beim ersten Aufruf Speicher für das Bildsignal anlegen müssen und den Algorithmus
 vollständig durchlaufen. In allen weiteren Schritten, wenn der Bildspeicher angelegt ist, wird der Algorithmus
 lediglich die Größe des Bildspeichers berechnen und vergleichen
 (Zeile \ref{alg:checkimagebuffer-size}--\ref{alg:checkimagebuffer-checksize-end}), was in konstanter Zeit erfolgt.

\autoref{alg:artkpfree} überprüft in Zeile \ref{alg:artkpfree-checkmem-start}--\ref{alg:artkpfree-checkmem-end} ob der
 Speicherbereich gültig ist. Falls nicht wird die weitere Ausführung abgebrochen. Nur im Falle, dass es sich um einen
 gültigen Speicherbereich handelt, wird in Zeile \ref{alg:artkpfree-deletemem-start}--\ref{alg:artkpfree-deletemem-end}
 der Speicher gelöscht und $\mathit{NULL}$ zugewiesen.

\begin{algorithm}[ht]
\caption{\textproc{artkpFree}}
\label{alg:artkpfree}
\begin{algorithmic}[1]
	\Require $\mathit{rawMemory}$
	\If{$\neg\mathit{rawMemory}$}
	\label{alg:artkpfree-checkmem-start}
		\State \textbf{return}
	\EndIf
	\label{alg:artkpfree-checkmem-end}
	\State \Call{free}{$\mathit{rawMemory}$}
	\label{alg:artkpfree-deletemem-start}
	\State $\mathit{rawMemory} \gets \mathit{NULL}$
	\label{alg:artkpfree-deletemem-end}
\end{algorithmic}
\end{algorithm}


\autoref{alg:artkpalloc} alloziert den Speicherbereich für die benötigte Größe, die in Zeile
 \ref{alg:checkimagebuffer-size} von \autoref{alg:checkimagebuffer} berechnet wurde.

\begin{algorithm}[!ht]
\caption{\textproc{artkpAlloc}}
\label{alg:artkpalloc}
\begin{algorithmic}[1]
	\Require $\mathit{size}$
	\State $\mathit{rawMemory} \gets \mathit{size} \cdot$ \Call{sizeof}{$\mathit{short}$}
	\State \textbf{return} \Call{malloc}{$\mathit{rawMemory}$}
\end{algorithmic}
\end{algorithm}


Nachdem der Speicher das erste Mal angelegt wurde, werden weder \autoref{alg:artkpfree} noch \autoref{alg:artkpalloc}
 aufgerufen. Somit ist die Laufzeit, die zur Überprüfung der Größe des Bildspeichers verwendet wird, konstant.

% subsubsection bildspeicher (end)

% subsection datenstrukturen (end)

\subsection{Fiducial Detection} % (fold)
\label{sec:fiducial_detection}

\textproc{arDetectMarker} (Vgl. \autoref{alg:detectmarker}) wird von \textproc{calc} aufgerufen und benötigt das
 Bildsignal $I$, den Schwellwert $\mathit{tresh}$ sowie $\mathit{marker\_info}$ und $\mathit{marker\_num}$. In Zeile
 \ref{alg:detectmarker-init-start}--\ref{alg:detectmarker-init-end} werden die lokalen Variablen initialisiert. Die
 Variablen werden als Parameter für den Aufruf der Methode \textproc{arDetectMarker2} in Zeile
 \ref{alg:detectmarker-call-method} verwendet.

\begin{algorithm}[!ht]
\caption{\textproc{arDetectMarker}}
\label{alg:detectmarker}
\begin{algorithmic}[1]
	\Require $I,\mathit{thresh},\mathit{marker\_info},\mathit{marker\_num}$
	\State $I_l \gets$ \textproc{NULL}
	\label{alg:detectmarker-init-start}
	\State $\mathit{label\_num},\mathit{area},\mathit{clip},\mathit{label\_ref},\mathit{pos} \gets \infty$
	\label{alg:detectmarker-init-end}
	\State \Call{autoThreshold.reset}{}
	\label{alg:detectmarker-call-autothreshold}
	\State \Call{checkImageBuffer}{}
	\label{alg:detectmarker-call-imagebuffer}
	\State $\mathit{marker\_num} \gets 0$
	\State $I_l \gets$ \Call{arLabeling}{$I,\mathit{thresh},\mathit{label\_num},\mathit{area},\mathit{pos},\mathit{clip},\mathit{label\_ref}$}
	\label{alg:detectmarker-call-labeling}
	% AR_AREA_MAX (100000) und AR_AREA_MIN (70) sind defines
	% wmarker_num globale variable
	\If{$I_l$}
	\label{alg:detectmarker-check-il-start}
		\State $\mathit{marker\_info2} \gets$ \textproc{arDetectMarker2}$\left(
		\begin{aligned}
				& I_l,\mathit{thresh},\mathit{label\_ref},\mathit{area},\mathit{pos},\mathit{clip},\\
				& \mathit{AR\_AREA\_MAX},\\
				& \mathit{AR\_AREA\_MIN},\\
				& 1.0, \mathit{wmarker\_num}
		\end{aligned}\right)$
		\label{alg:detectmarker-call-method}
		\If{$\mathit{marker\_info2}$}
		\label{alg:detectmarker-check-marker-start}
			\State \ldots \Comment{Weitere Anweisungen zur Identifikation einer Marke.}
		\EndIf
		\label{alg:detectmarker-check-marker-end}
	\EndIf
	\label{alg:detectmarker-check-il-end}
	\State \ldots \Comment{Weitere Anweisungen zur Identifikation einer Marke.}
\end{algorithmic}
\end{algorithm}


In Zeile \ref{alg:detectmarker-call-autothreshold} wird der Schwellwert auf seine Startwerte zurückgesetzt
 (Vgl. \autoref{alg:autothresholdreset}) und der Bildspeicher in Zeile \ref{alg:detectmarker-call-imagebuffer}
 überprüft (Vgl. \autoref{alg:checkimagebuffer}). Die Regionenmarkierung $I_l$ wird durch den Rückgabewert von
 \textproc{arLabeling} in Zeile \ref{alg:detectmarker-call-labeling} gesetzt. Im Anschluss wird in Zeile
 \ref{alg:detectmarker-check-il-start}--\ref{alg:detectmarker-check-il-end} geprüft, ob der Speicher der
 Regionenmarkierung erfolgreich gesetzt wurde. Andernfalls wird die Untersuchung für das aktuelle Bildsignal $I$
 beendet. Nur wenn die Regionenmarkierung erfolgreich war, wird in Zeile \ref{alg:detectmarker-call-method} die Methode
 \textproc{arDetectMarker2} aufgerufen. Der Rückgabewert von \textproc{arDetectMarker2} wird in der Membervariable
 $\mathit{marker\_info2}$ gespeichert. Es wird anschließend in Zeile
 \ref{alg:detectmarker-check-marker-start}--\ref{alg:detectmarker-check-marker-end} geprüft, ob der Zeiger von
 $\mathit{marker\_info2}$ auf einen gültigen Speicherbereich verweist. Falls die Überprüfung erfolgreich war, sind in
 $\mathit{marker\_info2}$ die Koordinaten der Eckpunkte der Marke gespeichert und das Verfahren beendet.
 \autoref{alg:detectmarker} verarbeitet die Anweisungen in konstanter Zeit.

\clearpage

\subsubsection{Regionenmarkierung} % (fold)
\label{sec:regionenmarkierung}

Die Methode \textproc{arLabeling} (\autoref{alg:arlabeling-init1}--\autoref{alg:arlabeling-calcregiondata}) wird zur
 Markierung der Regionen in einem Bildsignal $I$ verwendet. Wie in \autoref{sub:fiducial_detection} beschrieben,
 verzichet ARToolKitPlus beim Aufruf des Verfahrens auf ein Binärbild. Stattdessen wird ein Bildsignal $I$ während der
 Verarbeitung durch eine Schwellwertanalyse untersucht. \textproc{arLabeling} kann in drei Abschnitte unterteilt werden
 (Vgl. \autoref{alg:arlabeling-overview}):

\begin{enumerate}
	\item Initialisierung der Variablen und des Speichers, \label{label-init}
	\item Regionenmarkierung und Auflösen von Kollisionen und \label{label-region}
	\item Aufbereiten der Regionenmarkierung zur Speicherung. \label{label-cleaning}
\end{enumerate}

\begin{algorithm}[!ht]
\caption{\textproc{arLabeling} (Übersicht)}
\label{alg:arlabeling-overview}
\begin{algorithmic}[1]
	\Require $I,\mathit{thresh},\mathit{label\_num},\mathit{area},\mathit{pos},\mathit{clip},\mathit{label\_ref}$

	\State Initialisieren der Variablen und des Speichers
	\State Regionenmarkierung
	\State Regionenmarkierung aufbereiten und speichern

\end{algorithmic}
\end{algorithm}


\autoref{label-init} ist in \autoref{alg:arlabeling-init1} aufgeführt. In Zeile
 \ref{alg:arlabeling-init1-local-start}--\ref{alg:arlabeling-init1-local-end} werden die lokalen Variablen deklariert,
 deren Bedeutung bei ihrem ersten Auftreten erklärt werden. Speicheradressen werden in Zeile
 \ref{alg:arlabeling-init1-address-start}--\ref{alg:arlabeling-init1-address-end} initialisiert. Der
 Schwellwertparameter wird verdreifacht und in lokal gespeichert (Zeile \ref{alg:arlabeling-init1-threshold}). Die
 Schwellwertanalyse in \autoref{alg:arlabeling-regionlabeling} benutzt einen dreifachen Wert als Optimierung
 (Vgl. S.~\pageref{sub:arlabel-threshold}). Zum Schluss wird in Zeile
 \ref{alg:arlabeling-init1-size-start}--\ref{alg:arlabeling-init1-size-end} die Größe des Bildsignals festgelegt. Die
 Variablen $\mathit{arImXsize}$ und $\mathit{arImYsize}$ enthalten die Breite und Höhe des Bildsignals $I$. Das
 halbieren der Bildhöhe und -breite ist ebenfalls eine Optimierung. Da auf alle Variablen und Adressen direkt
 zugegriffen wird, ist die Laufzeit des Algorithmus konstant.

\begin{algorithm}[!hb]\small
\caption{\textproc{arLabeling} (Initialisierung)}
\label{alg:arlabeling-init1}
\begin{algorithmic}[1]
	\Require $I,\mathit{thresh},\mathit{label\_num},\mathit{area},\mathit{pos},\mathit{clip},\mathit{label\_ref}$

	\State $\mathit{pnt}, \mathit{pnt1}, \mathit{pnt2} \gets \infty$
	\label{alg:arlabeling-init1-local-start}
	\State $\mathit{wk}, \mathit{wk\_max}, m, n, i, j, k, \mathit{lxsize}, \mathit{lysize}, \mathit{poff} \gets \infty$
	\State $\mathit{l\_image}, \mathit{work}, \mathit{work2}, \mathit{wlabel\_num}, \mathit{warea}, \mathit{wclip},
	 \mathit{wpos} \gets \infty$
	\label{alg:arlabeling-init1-local-end}

	\State $\mathit{l\_image} \gets \mathit{l\_imageL}[0]$
	\label{alg:arlabeling-init1-address-start}
	\State $\mathit{work} \gets \mathit{workL}[0]$
	\State $\mathit{work2} \gets \mathit{work2L}[0]$
	\State $\mathit{wlabel\_num} \gets \mathit{wlabel\_numL}$
	\State $\mathit{warea} \gets \mathit{wareaL}[0]$
	\State $\mathit{wclip} \gets \mathit{wclipL}[0]$
	\State $\mathit{wpos} \gets \mathit{wposL}[0]$
	\label{alg:arlabeling-init1-address-end}

	\State $\mathit{thresh} \gets \mathit{thresh} \cdot 3$
	\label{alg:arlabeling-init1-threshold}
	\State $\mathit{lxsize} \gets \tfrac{arImXsize}{2}$
	\label{alg:arlabeling-init1-size-start}
	\State $\mathit{lysize} \gets \tfrac{arImYsize}{2}$
	\label{alg:arlabeling-init1-size-end}

	\algstore{brk-arkabelinginit}
\end{algorithmic}
\end{algorithm}


Die Initialisierung von \textproc{arLabeling} wird in \autoref{alg:arlabeling-init2} fortgesetzt. $\mathit{pnt1}$ wird
 in Zeile \ref{alg:arlabeling-init2-address1-start} auf die erste Speicherstelle der ersten Zeile des Regionenbildes
 gesetzt. $\mathit{pnt2}$ erhält in Zeile \ref{alg:arlabeling-init2-address1-end} die erste Speicherstelle der
 letzten Zeile des Regionenbildes. In der Schleife in Zeile
 \ref{alg:arlabeling-init2-loop1-start}--\ref{alg:arlabeling-init2-loop1-end} wird über die Breite des Regionenbildes
 iteriert, um die erste und die letzte Zeile des Regionenbildes zu löschen. Dazu wird an den Adressen von
 $\mathit{pnt1}$ und $\mathit{pnt2}$ der Wert $0$ gespeichert. Danach werden die Adressen von $\mathit{pnt1}$ und
 $\mathit{pnt2}$ inkrementiert. In Zeile \ref{alg:arlabeling-init2-inc-1} wird die Laufvariable $i$ inkrementiert.

\begin{algorithm}[!ht]
\caption{\textproc{arLabeling} (Fortsetzung der Initialisierung)}
\label{alg:arlabeling-init2}
\begin{algorithmic}[1]
	\algrestore{brk-arkabelinginit}

	\State $\mathit{pnt1} \gets \mathit{l\_image}[0]$
	\Cost{$c_{14}$}{$2$}
	\label{alg:arlabeling-init2-address1-start}
	\State $\mathit{pnt2} \gets \mathit{l\_image}[(\mathit{lysize} - 1) \cdot \mathit{lxsize}]$
	\Cost{$c_{15}$}{$4$}
	\label{alg:arlabeling-init2-address1-end}
	\For{$i \gets 1$ \textbf{to} $i < \mathit{lxsize}$}
	\Cost{$c_{16}$}{$(\mathit{lxsize} - 1) + 1$}
	\label{alg:arlabeling-init2-loop1-start}
		\State $\mathit{pnt1} \gets 0$
		\Cost{$c_{17}$}{$(\mathit{lxsize} - 1 )$}
		\State $\mathit{pnt2} \gets 0$
		\Cost{$c_{18}$}{$(\mathit{lxsize} - 1 )$}
		\State Inkrementiere ${pnt1}$
		\Cost{$c_{19}$}{$(\mathit{lxsize} - 1 )$}
		\State Inkrementiere ${pnt2}$
		\Cost{$c_{20}$}{$(\mathit{lxsize} - 1 )$}
		\State $i \gets i + 1$
		\Cost{$c_{21}$}{$(\mathit{lxsize} - 1 )$}
		\label{alg:arlabeling-init2-inc-1}
	\EndFor
	\label{alg:arlabeling-init2-loop1-end}

	\State $\mathit{pnt1} \gets \mathit{l\_image}[0]$
	\Cost{$c_{23}$}{$2$}
	\label{alg:arlabeling-init2-address2-start}
	\State $\mathit{pnt2} \gets \mathit{l\_image}[\mathit{lxsize} - 1]$
	\Cost{$c_{24}$}{$3$}
	\label{alg:arlabeling-init2-address2-end}
	\For{$i \gets 1$ \textbf{to} $i < \mathit{lysize}$}
	\Cost{$c_{25}$}{$(\mathit{lysize} - 1) + 1$}
	\label{alg:arlabeling-init2-loop2-start}
		\State $\mathit{pnt1} \gets 0$
		\Cost{$c_{26}$}{$\mathit{lysize} - 1$}
		\label{alg:arlabeling-init2-clearfirstrow}
		\State $\mathit{pnt2} \gets 0$
		\Cost{$c_{27}$}{$\mathit{lysize} - 1$}
		\label{alg:arlabeling-init2-clearlastrow}
		\State $\mathit{pnt1} \gets \mathit{pnt1} + \mathit{lxsize}$
		\Cost{$c_{28}$}{$(\mathit{lysize} - 1) 2$}
		\State $\mathit{pnt2} \gets \mathit{pnt2} + \mathit{lxsize}$
		\Cost{$c_{29}$}{$(\mathit{lysize} - 1) 2$}
		\State $i \gets i + 1$
		\Cost{$c_{30}$}{$\mathit{lysize} - 1$}
	\EndFor
	\label{alg:arlabeling-init2-loop2-end}

	\State $\mathit{wk\_max} \gets 0$
	\Cost{$c_{32}$}{$1$}
	\label{alg:arlabeling-init2-label}
	\State $\mathit{pnt2} \gets \mathit{l\_image}[\mathit{lxsize} + 1]$
	\Cost{$c_{33}$}{$3$}
	\State $\mathit{pnt} \gets I[\left(\mathit{arImXsize} \cdot 2 + 2 \right) \cdot \mathit{pixelSize}]$
	\Cost{$c_{34}$}{$5$}
	\State $\mathit{poff} \gets \mathit{pixelSize} \cdot 2$
	\Cost{$c_{35}$}{$2$}
\end{algorithmic}
\end{algorithm}


In Zeile \ref{alg:arlabeling-init2-address2-start}--\ref{alg:arlabeling-init2-address2-end} werden die Adressen
 von $\mathit{pnt1}$ und $\mathit{pnt2}$ erneut festgelegt. $\mathit{pnt1}$ wird die erste Speicherstelle des
 Regionenbildes zugewiesen. In $\mathit{pnt2}$ wird die Adresse der ersten Speicherstelle der letzten Zeile hinterlegt.
 In der Schleife von Zeile \ref{alg:arlabeling-init2-loop2-start}--\ref{alg:arlabeling-init2-loop2-end} wird die erste
 und letzte Spalte des Regionenbildes gelöscht, indem der Wert $0$ an die Speicherstelle von $\mathit{pnt1}$ und
 $\mathit{pnt2}$ geschrieben wird
 (Zeile \ref{alg:arlabeling-init2-clearfirstrow}--\ref{alg:arlabeling-init2-clearlastrow}). Im Anschluss daran werden
 die Adressen $\mathit{pnt1}$ und $\mathit{pnt2}$, sowie die Laufvariable $i$, inkrementiert.

In Zeile \ref{alg:arlabeling-init2-label} wird die Markierungsvariable $\mathit{wk\_max}$ mit dem Wert $0$
 initialisiert. Danach wird die Adresse der Startposition des Regionenbildes in $\mathit{pnt2}$ gespeichert. Dabei ist
 zu beachten, dass die Adresse auf den zweiten \gls{pixel} der zweiten Zeile verweist. $\mathit{pnt}$ wird daraufhin die
 Adresse des Bildsignals $I$ zugewiesen. Der Adresse von $I$ ist der vierte \gls{pixel} der zweiten Zeile. Die
 Variable $\mathit{poff}$ ist der Adressabstand der \gls{pixel} in $I$ und dient der Adressierung des nächsten
 \glspl{pixel}. $\mathit{poff}$ wird in der letzten Zeile von \autoref{alg:arlabeling-init2} gesetzt.

Die Kosten des Algorithmus sind in \autoref{alg:arlabeling-init2} angegeben und in \autoref{eq:analyse-arlabeling}
 aufgeführt. Da die Breite $\mathit{lxsize}$ des Regionenbildes größer ist als die Höhe $\mathit{lysize}$, ist die
 Laufzeit des Algorithmus abhängig von $\mathit{lxsize}$ und beträgt somit $\Theta{(\mathit{lxsize})}$.

\begin{equation}
	\label{eq:analyse-arlabeling}
	\begin{split}
		T(I) = &
		c_1
		+ c_2
		+ c_3 \left(\mathit{lxsize} + 1\right)\\
		& + c_4 \sum_{i=1}^{\mathit{lxsize}} 1
		+ c_5 \sum_{i=1}^{\mathit{lxsize}} 1
		+ c_6 \sum_{i=1}^{\mathit{lxsize}} 1
		+ c_7 \sum_{i=1}^{\mathit{lxsize}} 1
		+ c_8 \sum_{i=1}^{\mathit{lxsize}} 1\\
		& + c_9
		+ c_{10}
		+ c_{11} \left(\mathit{lysize} + 1\right)\\
		& + c_{12} \sum_{i=1}^{\mathit{lysize}} 1
		+ c_{13} \sum_{i=1}^{\mathit{lysize}} 1
		+ c_{14} \sum_{i=1}^{\mathit{lysize}} 1
		+ c_{15} \sum_{i=1}^{\mathit{lysize}} 1
		+ c_{16} \sum_{i=1}^{\mathit{lysize}} 1\\
		& + c_{17}
		+ c_{18}
		+ c_{19}
		+ c_{20}\\
		T(I) = &
		c_1
		+ c_2
		+ c_3
		+ \left(c_3 \mathit{lxsize}\right)\\
		& + c_4 \left[ \mathit{lxsize} \left(1\right) \right]
		+ c_5 \left[ \mathit{lxsize} \left(1\right) \right]
		+ c_6 \left[ \mathit{lxsize} \left(1\right) \right]
		+ c_7 \left[ \mathit{lxsize} \left(1\right) \right]
		+ c_8 \left[ \mathit{lxsize} \left(1\right) \right]\\
		& + c_9
		+ c_{10}
		+ c_{11}
		+ c_{11} \left[ \mathit{lysize} \left(1\right) \right]\\
		& + c_{12} \left[ \mathit{lysize} \left(1\right) \right]
		+ c_{13} \left[ \mathit{lysize} \left(1\right) \right]
		+ c_{14} \left[ \mathit{lysize} \left(1\right) \right]
		+ c_{15} \left[ \mathit{lysize} \left(1\right) \right]
		+ c_{16} \left[ \mathit{lysize} \left(1\right) \right]\\
		& + c_{17}
		+ c_{18}
		+ c_{19}
		+ c_{20}\\
		T(I) = &
		c_1
		+ c_2
		+ c_3
		+ \left( c_3 + c_4 + c_5 + c_6 + c_7 + c_8 \right) \mathit{lxsize}\\
		& + c_9
		+ c_{10}
		+ c_{11}
		+ \left( c_{11} + c_{12} + c_{13} + c_{14} + c_{15} + c_{16} \right) \mathit{lysize}\\
		& + c_{17}
		+ c_{18}
		+ c_{19}
		+ c_{20}\\
		T(I) = &
		\mathit{lxsize} + \mathit{lysize}\\
		T(I) = &
		\Theta\left( \mathit{lxsize} \right)
	\end{split}
\end{equation}

\autoref{label-region} von \autoref{alg:arlabeling-overview} ist für die Regionenmarkierung und das Auflösen von
 Kollisionen verantwortlich. Eine Übersich des Verfahrens ist in \autoref{alg:arlabeling-regionlabeling} dargestellt.
 \textproc{arLabeling} untersucht das Bildsignal $I$ zeilenweise von links oben nach rechts unten. Durch eine
 Schwellwertanalyse wird entschieden, ob ein \gls{pixel} an Position $I(u,v)$ ein Vordergrund- oder Hintergrundpixel
 ist. Die Regionenmarkierung wird dann in $\mathit{l\_image}$ gespeichert. Das Verfahren wird solange wiederholt, bis
 das Bildsignal $I$ vollständig prozessiert wurde und zusammenhängende Bildregionen in $\mathit{l\_image}$ markiert
 sind. Die Variable $\mathit{pnt}$ enthält die Adresse des zu untersuchenden \gls{pixel} aus $I$. Die nächste freie
 Speicherstelle in $\mathit{l\_image}$ ist in $\mathit{pnt2}$ hinterlegt.

\begin{algorithm}[!ht]\small
\caption{\textproc{arLabeling} (Regionenmarkierung)}
\label{alg:arlabeling-regionlabeling}
\begin{algorithmic}[1]
	\For{$j \gets 1$ \textbf{to} $j < \mathit{lysize} - 1$}
	\Cost{$c_{1}$}{$\mathit{lysize} - 1$}
	\label{alg:arlabeling-regionlabeling-loop1-start}

		\For{$i \gets 1$ \textbf{to} $i < \mathit{lxsize} - 1$}
		\Cost{$c_{2}$}{$(\mathit{lysize} - 2)(\mathit{lxsize} - 1)$}
		\label{alg:arlabeling-regionlabeling-loop2-start}

			\State $\mathit{coorTresh} \gets \mathit{tresh}$
			\Cost{$c_{3}$}{$(\mathit{lysize} - 2)(\mathit{lxsize} - 2)$}
			\label{alg:arlabeling-regionlabeling-threshold-start}
			\State $\mathit{isBlack} \gets \textbf{false}$
			\Cost{$c_{4}$}{$(\mathit{lysize} - 2)(\mathit{lxsize} - 2)$}
			\State $\mathit{isBlack} \gets \left(
			\begin{aligned}
				& \quad (\mathit{pnt} + 0) \\
				& \quad + (\mathit{pnt} + 1) \\
				& \quad + (\mathit{pnt} + 2) \\
				& \leq \mathit{coorTresh}
			\end{aligned}\right)$
			\Cost{$c_{5}$}{$(\mathit{lysize} - 2)(\mathit{lxsize} - 2)7$}
			\label{alg:arlabeling-regionlabeling-calcblack}

			\If{$\mathit{isBlack}$}
			\Cost{$c_{6}$}{$(\mathit{lysize} - 2)(\mathit{lxsize} - 2)$}
			\label{alg:arlabeling-regionlabeling-isblack?}
				\State Untersuche 8er-Nachbarschaft
				\Cost{$c_{7}$}{$(\mathit{lysize} - 2)(\mathit{lxsize} - 2)t_{1}$}
				\label{alg:arlabeling-regionlabeling-black}
			\Else
				\State $\mathit{pnt2} \gets 0$
				\Cost{$c_{9}$}{$(\mathit{lysize} - 2)(\mathit{lxsize} - 2)$}
				\label{alg:arlabeling-regionlabeling-notblack}
			\EndIf
			\label{alg:arlabeling-regionlabeling-threshold-end}

			\State $i \gets i + 1$
			\Cost{$c_{11}$}{$(\mathit{lysize} - 2)(\mathit{lxsize} - 2)$}
			\label{alg:arlabeling-regionlabeling-inc1-start}
			\State $\mathit{pnt} \gets \mathit{pnt} + \mathit{poff}$
			\Cost{$c_{12}$}{$(\mathit{lysize} - 2)(\mathit{lxsize} - 2)2$}
			\State Inkrementiere $\mathit{pnt2}$
			\Cost{$c_{13}$}{$(\mathit{lysize} - 2)(\mathit{lxsize} - 2)$}
			\label{alg:arlabeling-regionlabeling-inc1-end}
		\EndFor
		\label{alg:arlabeling-regionlabeling-loop2-end}

		\State $\mathit{pnt} \gets \mathit{pnt} + \mathit{arImXsize} \cdot \mathit{pixelSize}$
		\Cost{$c_{15}$}{$(\mathit{lysize} - 2)3$}
		\label{alg:arlabeling-regionlabeling-inc2-start}
		\State $j \gets j + 1$
		\Cost{$c_{16}$}{$(\mathit{lysize} - 2)$}
		\State $\mathit{pnt} \gets \mathit{pnt} + \mathit{poff} \cdot 2$
		\Cost{$c_{17}$}{$(\mathit{lysize} - 2)3$}
		\State $\mathit{pnt2} \gets \mathit{pnt2} + 2$
		\Cost{$c_{18}$}{$(\mathit{lysize} - 2)2$}
		\label{alg:arlabeling-regionlabeling-inc2-end}
	\EndFor
	\label{alg:arlabeling-regionlabeling-loop1-end}
\end{algorithmic}
\end{algorithm}

In den beiden Schleifen in Zeile
 \ref{alg:arlabeling-regionlabeling-loop1-start}--\ref{alg:arlabeling-regionlabeling-loop1-end} und Zeile
 \ref{alg:arlabeling-regionlabeling-loop2-start}--\ref{alg:arlabeling-regionlabeling-loop2-end} wird das Bildsignal
 zeilenweise, von oben links nach unten rechts, verarbeitet. Das Inkrementieren der Variablen in Zeile
 \ref{alg:arlabeling-regionlabeling-inc1-start}--\ref{alg:arlabeling-regionlabeling-inc1-end} und Zeile
 \ref{alg:arlabeling-regionlabeling-inc2-start}--\ref{alg:arlabeling-regionlabeling-inc2-end} sorgt dafür, dass nur die
 Hälfte der \gls{pixel} prozessiert werden.

Die Schwellwertanalyse wird in Zeile
 \ref{alg:arlabeling-regionlabeling-threshold-start}--\ref{alg:arlabeling-regionlabeling-threshold-end} durchgeführt.
 Dazu wird in Zeile \ref{alg:arlabeling-regionlabeling-calcblack} die RGB-Komponente des Bildsignals $I$ ausgelesen und
 addiert. Normalerweise würde man an dieser Stelle den Schwellwert mit jeder Komponente $R$, $G$ und $B$ einzeln
 vergleichen. Die Verdreifachung des Schwellwerts in \autoref{alg:arlabeling-init1} und die Addition der
 RGB-Komponenten ermöglichen hingegen eine Schwellwertanalyse mit nur einem Vergleich.\label{sub:arlabel-threshold}

In Zeile \ref{alg:arlabeling-regionlabeling-isblack?}--\ref{alg:arlabeling-regionlabeling-threshold-end} wird
 untersucht, ob ein Vordergrundpixel gefunden wurde. Falls nicht, wird in das Regionenbild $\mathit{l\_image}$ eine $0$
 geschrieben. Wenn die Schwellertanalyse ein Vordergrundpixel bestimmt hat, müssen in Zeile
 \ref{alg:arlabeling-regionlabeling-black} die Nachbarn des Vordergrundpixels mit einer 8er-Nachbarschaft untersucht
 werden (Vgl. \autoref{alg:arlabeling-neighbour}).

\begin{algorithm}[!ht]
\caption{\textproc{arLabeling} (Untersuchung der 8er-Nachbarschaft)}
\label{alg:arlabeling-neighbour}
\begin{algorithmic}[1]
	\State $\mathit{pnt1} \gets \mathit{pnt2}[-\mathit{lxsize}]$ \Comment Adresse zuweisen
	\Cost{$c_{1}$}{$3$}
	\label{alg:arlabeling-neighbour-n3}
	\If{$\mathit{pnt1} > 0$}
	\Cost{$c_{2}$}{$2$}
		\State \ldots \Comment 1. Fall
	\ElsIf{$\left(\mathit{pnt1} + 1\right) > 0$}
	\Cost{$c_{4}$}{$2$}
		\State \ldots \Comment 2. Fall
	\ElsIf{$\left(\mathit{pnt1} - 1\right) > 0$}
	\Cost{$c_{6}$}{$2$}
		\State \ldots \Comment 3. Fall
	\ElsIf{$\left(\mathit{pnt2} - 1\right) > 0$}
	\Cost{$c_{8}$}{$2$}
		\State \ldots \Comment 4. Fall
	\Else
		\State \ldots \Comment 5. Fall
	\EndIf
\end{algorithmic}
\end{algorithm}


Die Nachbarschafsuntersuchung in ARToolKitPlus untersucht die vier Nachbarn der Markierung an Position $I(u,v)$. Wie in
 \autoref{sec:vorläufige_makierung} bereits beschrieben, ist eine Regionenmarkierung davon abhängig ob alle Nachbarn
 Hintergrundpixel sind, genau ein Nachbar eine Markierung hat oder mehrere Nachbarn eine Markierung haben. Die Fälle
 1, 3 und 4 in \autoref{alg:arlabeling-neighbour} untersuchen, ob genau ein Nachbar eine Markierung hat. Bei Fall 5 sind
 alle Nachbarn Hintergrundpixel. Nur bei Fall 2 können  mehrere Nachbarn eine Markierung haben. In Zeile
 \ref{alg:arlabeling-neighbour-n3} wird der Variablen $\mathit{pnt1}$ der Nachbar $N_3 = (u,v+1)$ zugewiesen. Die
 Nachbarn sind in \autoref{fig:analyse-nachbarschaftsbeziehung} illustriert.

\begin{figure}[!ht]
	\centering
	\input{resources/8er-Nachbarschaft.pdf_tex}
	\caption{8er-Nachbarschaft mit $N_1 = I(u-1,v)$, $N_2 = I(u-1,v+1)$, $N_3 = (u,v+1)$ und $N_4 = I(u+1,v+1)$.}
	\label{fig:analyse-nachbarschaftsbeziehung}
\end{figure}

\paragraph{1. Fall:} % (fold)
\label{par:fall_1_}
Wir wissen durch \autoref{alg:arlabeling-regionlabeling}, dass $I(u,v)$ ein Vordergrundpixel ist, dem wir an dieser
 Stelle eine Markierung zuweisen wollen. Im ersten Fall wird die Markierung $N_3$ untersucht. Wenn $N_3$ ein
 Vordergrundpixel ist kann die Markierung für $I(u,v)$ übernommen werden. Falls es sich bei den Nachbarn
 $N_1 = I(u-1,v)$, $N_2 = I(u-1,v+1)$ und $N_4 = I(u+1,v+1)$ um Vordergrundpixel handelt, haben sie die gleiche
 Markierung wie $N_3$ und müssen nicht weiter untersucht werden. Das Verfahren ist in
 \autoref{alg:arlabeling-neighbour-case1} beschrieben.

% Fall 1: *pnt1 > 0
\begin{algorithm}[ht]
\caption{\textproc{arLabeling} (8er-Nachbarschaft: 1. Fall)}
\label{alg:arlabeling-neighbour-case1}
\begin{algorithmic}[1]

	\If{$\mathit{pnt1} > 0$}
		\State $\mathit{pnt2} \gets \mathit{pnt1}$
		\label{alg:arlabeling-neighbour-case1-save-label}
		\State $\mathit{pnt2\_index} \gets \left(\mathit{pnt2} - 1\right) \cdot 7$
		\label{alg:arlabeling-neighbour-case1-calc-offset}
		\State $\mathit{work2}\left[\mathit{pnt2\_index} + 0\right] \gets \mathit{work2}\left[\mathit{pnt2\_index} + 0\right] + 1$
		\label{alg:arlabeling-neighbour-case1-inc-region}
		\State $\mathit{work2}\left[\mathit{pnt2\_index} + 1\right] \gets \mathit{work2}\left[\mathit{pnt2\_index} + 1\right] + i$
		\State $\mathit{work2}\left[\mathit{pnt2\_index} + 2\right] \gets \mathit{work2}\left[\mathit{pnt2\_index} + 2\right] + j$
		\State $\mathit{work2}\left[\mathit{pnt2\_index} + 6\right] \gets j$
		\label{alg:arlabeling-neighbour-case1-save-j}
		\algstore{brk-case1}
\end{algorithmic}
\end{algorithm}


In Zeile \ref{alg:arlabeling-neighbour-case1-save-label} wird die Markierung von $N_3$ übernommen und in
 $\mathit{l\_image}$ an Position $(x,y)$ gespeichert. In der Variablen $\mathit{work2}$ werden Informationen der
 Regionenmarkierung gespeichert. Dazu wird zuerst in Zeile \ref{alg:arlabeling-neighbour-case1-calc-offset} aus
 $\mathit{pnt2}$ der Wert der Regionenmarkierung gelesen. Der Wert der Markierung wird zur Berechnung des
 Adressabstands benutzt, um die Werte für die Region an die richtige Stelle zu schreiben. An der Position $0$ von
 $\mathit{work2}$ (Zeile \ref{alg:arlabeling-neighbour-case1-inc-region}) wird die Anzahl der Vordergrundpixel der
 Region erhöht. Position 1 und Position 2 von $\mathit{work2}$ enthalten die akummulierten Werte von $i$ und $j$ für
 die u- und v-Koordinaten aller Vordergrundpixel der Region. An Position 6 von $\mathit{work2}$ in Zeile
 \ref{alg:arlabeling-neighbour-case1-save-j} wird die y-Koordinate gespeichert. Alle Anweisungen in
 \autoref{alg:arlabeling-neighbour-case1} werden in konstanter Zeit ausgeführt.

% paragraph fall_1_ (end)

\paragraph{2. Fall:} % (fold)
\label{par:fall_2_}
Beim zweiten Fall wird die Markierung $N_4$ betrachtet. Da $N_3$ keine Markierung aufweist, können nur $N_1$ und $N_2$
 Markierungen haben. Da die Markierungen nur durch $I(u,v)$ verbunden sind, kann es sich hier um eine Kollision
 handeln, die durch \autoref{alg:arlabeling-neighbour-case2-1} oder \autoref{alg:arlabeling-neighbour-case2-2}
 besonders behandelt wird. Das Verfahren ist in \autoref{alg:arlabeling-neighbour-case2} aufgeführt.

% Fall 2: *(pnt1+1) > 0
\begin{algorithm}[!ht]\small
\caption{\textproc{arLabeling} (8er-Nachbarschaft: 2. Fall)}
\label{alg:arlabeling-neighbour-case2}
\begin{algorithmic}[1]
	\algrestore{brk-case1}

	\ElsIf{$\left(\mathit{pnt1} + 1\right) > 0$}
	\Cost{$c_{8}$}{$2$}
		\If{$\left(\mathit{pnt1} - 1\right) > 0$}
		\Cost{$c_{9}$}{$2$}
		\Comment Ist in $N_2$ eine Markierung?
		\label{alg:arlabeling-neighbour-case2-n2}
			\State \ldots

		\ElsIf{$\left(\mathit{pnt2} - 1\right) > 0$}
		\Cost{$c_{11}$}{$2$}
		\Comment Ist in $N_1$ eine Markierung?
		\label{alg:arlabeling-neighbour-case2-n1}
			\State \ldots

		\Else \Comment Nur $N_4$ hat eine Markierung.
		\label{alg:arlabeling-neighbour-case2-4}
			\State \ldots
		\EndIf
		\algstore{brk-case2}
\end{algorithmic}
\end{algorithm}


In Zeile \ref{alg:arlabeling-neighbour-case2-n2} wird überprüft, ob $N_2$ eine Markierung enthält. Falls ja, wird
 \autoref{alg:arlabeling-neighbour-case2-1} untersuchen, ob eine Kollision vorliegt und sie gegebenfalls auflösen.
 Zeile \ref{alg:arlabeling-neighbour-case2-n1} überprüft $N_1$ auf eine vorhandene Markierung und fährt mit der
 Untersuchung einer evtl. Kollision in \autoref{alg:arlabeling-neighbour-case2-2} fort. Wenn weder $N_2$ noch $N_1$
 eine Markierung haben, ist nur $N_4$ ein Vordergrundpixel und wird mit \autoref{alg:arlabeling-neighbour-case2-3}
 markiert.

% Fall 2: *(pnt1+1) > 0
\begin{algorithm}[!ht]\small
\caption{\textproc{arLabeling} (8er-Nachbarschaft: 2. Fall, $N_4$ und $N_2$)}
\label{alg:arlabeling-neighbour-case2-1}
\begin{algorithmic}[1]
	% \State \Comment $N_4$ und $N_2$ haben eine Markierung.
	\State $m \gets \mathit{work}\left[\left(\mathit{pnt1} + 1\right) - 1\right]$
	\Cost{$c_{1}$}{$4$}
	\label{alg:arlabeling-neighbour-case2-1-m}
	\State $n \gets \mathit{work}\left[\left(\mathit{pnt1} - 1\right) - 1\right]$
	\Cost{$c_{2}$}{$4$}
	\label{alg:arlabeling-neighbour-case2-1-n}

	\If{$m > n$}
	\Cost{$c_{3}$}{$1$}
	\label{alg:arlabeling-neighbour-case2-mn-start}
		\State $\mathit{pnt2} \gets n$
		\Cost{$c_{4}$}{$1$}
		\label{alg:arlabeling-neighbour-case2-saven}
		\State $\mathit{wk} \gets \left(\mathit{work}\left[0\right]\right)$
		\Cost{$c_{5}$}{$2$}
		\label{alg:arlabeling-neighbour-case2-worklist}
		\For{$k \gets 0$ \textbf{to} $k < \mathit{wk\_max}$}
		\Cost{$c_{6}$}{$\mathit{wk\_max} + 1$}
		\label{alg:arlabeling-neighbour-case2-loop-start}
			\If{$\mathit{wk} = m$}
			\Cost{$c_{7}$}{$\mathit{wk\_max}$}
				\State $\mathit{wk} \gets n$
				\Cost{$c_{8}$}{$\mathit{wk\_max}$}
			\EndIf
			\State Inkrementiere $\mathit{wk}$
			\Cost{$c_{10}$}{$\mathit{wk\_max}$}
			\State $k \gets k + 1$
			\Cost{$c_{11}$}{$\mathit{wk\_max}$}
		\EndFor
		\label{alg:arlabeling-neighbour-case2-loop-end}
	\label{alg:arlabeling-neighbour-case2-mn-end}
	\ElsIf{$ m < n$}
	\Cost{$c_{13}$}{$1$}
	\label{alg:arlabeling-neighbour-case2-nm-start}
		\State $\mathit{pnt2} \gets m$
		\Cost{$c_{14}$}{$1$}
		\State $\mathit{wk} \gets \left(\mathit{work}\left[0\right]\right)$
		\Cost{$c_{15}$}{$2$}
		\For{$k \gets 0$ \textbf{to} $k < \mathit{wk\_max}$}
		\Cost{$c_{16}$}{$\mathit{wk\_max} + 1$}
			\If{$\mathit{wk} = n$}
			\Cost{$c_{17}$}{$\mathit{wk\_max}$}
				\State $\mathit{wk} \gets m$
				\Cost{$c_{18}$}{$\mathit{wk\_max}$}
			\EndIf
			\State Inkrementiere $\mathit{wk}$
			\Cost{$c_{20}$}{$\mathit{wk\_max}$}
			\State $k \gets k + 1$
			\Cost{$c_{21}$}{$\mathit{wk\_max}$}
		\EndFor
	\label{alg:arlabeling-neighbour-case2-nm-stop}
	\Else
		\State $\mathit{pnt2} \gets m$
		\Cost{$c_{24}$}{$1$}
		\label{alg:arlabeling-neighbour-case2-savem}
	\EndIf
\end{algorithmic}
\end{algorithm}


In \autoref{alg:arlabeling-neighbour-case2-1} wird in Zeile \ref{alg:arlabeling-neighbour-case2-1-m} der Wert der
 Markierung $N_4$ in Variable $m$ gespeichert. In Zeile \ref{alg:arlabeling-neighbour-case2-1-n} wird der Wert $N_2$
 in $n$ hinterlegt. In Zeile \ref{alg:arlabeling-neighbour-case2-mn-start}--\ref{alg:arlabeling-neighbour-case2-mn-end}
 wird überprüft, ob $m$ größer als $n$ ist. Falls dem so ist, wird der Wert $n$ in $\mathit{l\_image}$ gespeichert
 (Zeile \ref{alg:arlabeling-neighbour-case2-saven}). Danach wird in Zeile \ref{alg:arlabeling-neighbour-case2-worklist}
 die Adresse der ersten Stelle der Regionenmarkierungsliste $\mathit{work}$ in $wk$ hinterlegt. In der Schleife von
 Zeile \ref{alg:arlabeling-neighbour-case2-loop-start} bis Zeile \ref{alg:arlabeling-neighbour-case2-loop-end} wird die
 Liste durchlaufen und alle Werte von $m$ durch den Wert $n$ ersetzt. Falls der Wert $m$ kleiner als $n$ ist
 (Zeile \ref{alg:arlabeling-neighbour-case2-nm-start}--\ref{alg:arlabeling-neighbour-case2-nm-stop}) wird das gleiche
 Verfahren angewendet. Lediglich $m$ und $n$ werden getauscht. Wenn es sich bei $m$ und $n$ um den gleichen Wert
 handet, und somit $m$ und $n$ zur gleichen Region gehören, wird der Wert $m$ in $\mathit{l\_image}$ gespeichert
 (Zeile \ref{alg:arlabeling-neighbour-case2-savem}).

% Fall 2: *(pnt1+1) > 0
\begin{algorithm}[ht]
\caption{\textproc{arLabeling} (8er-Nachbarschaft: 2. Fall, $N_4$ und $N_1$)}
\label{alg:arlabeling-neighbour-case2-2}
\begin{algorithmic}[1]
	\State \Comment $N_4$ und $N_1$ haben eine Markierung.
	\State $m \gets \mathit{work}\left[\left(\mathit{pnt1} + 1\right) - 1\right]$ \Cost{$c_{1}$}{$1$}
	\State $n \gets \mathit{work}\left[\left(\mathit{pnt2} - 1\right) - 1\right]$ \Cost{$c_{2}$}{$1$}

	\If{$m > n$} \Cost{$c_{3}$}{$1$}
		\State $\mathit{pnt2} \gets n$ \Cost{$c_{4}$}{$1$}
		\State $\mathit{wk} \gets \left(\mathit{work}\left[0\right]\right)$ \Cost{$c_{5}$}{$1$}
		\For{$k \gets 0$ \textbf{to} $k < \mathit{wk\_max}$} \Cost{$c_{6}$}{$\mathit{wk\_max} + 1$}
			\If{$\mathit{wk} == m$} \Cost{$c_{7}$}{$\mathit{wk\_max}$}
				\State $\mathit{wk} \gets n$ \Cost{$c_{8}$}{$\mathit{wk\_max}$}
			\EndIf
			\State Inkrementiere $\mathit{wk}$ \Cost{$c_{9}$}{$\mathit{wk\_max}$}
			\State $k \gets k + 1$ \Cost{$c_{9}$}{$\mathit{wk\_max}$}
		\EndFor

	\ElsIf{$ m < n$}
		\State $\mathit{pnt2} \gets m$
		\State $\mathit{wk} \gets \left(\mathit{work}\left[0\right]\right)$
		\For{$k \gets 0$ \textbf{to} $k < \mathit{wk\_max}$}
			\If{$\mathit{wk} == n$}
				\State $\mathit{wk} \gets m$
			\EndIf
			\State Inkrementiere $\mathit{wk}$
			\State $k \gets k + 1$
		\EndFor

	\Else
		\State $\mathit{pnt2} \gets m$
	\EndIf
\end{algorithmic}
\end{algorithm}


\autoref{alg:arlabeling-neighbour-case2-2} behandelt den Fall, dass $N_4$ und $N_1$ eine Markierung aufweisen. Das Verfahren entspricht dem Verfahren in \autoref{alg:arlabeling-neighbour-case2-1}.

In \autoref{alg:arlabeling-neighbour-case2-3} ist der Fall beschrieben, dass $N_4$ der einzige Nachbar des
 Vordergrundpixels $(u,v)$ ist. In diesem Fall wird der Wert der Markierung von $N_4$ in $\mathit{l\_image}$
 gespeichert (Zeile \ref{alg:arlabeling-neighbour-case2-3-n4}). Danach wird der Adressabstand berechnet, um die
 Informationen der Regionenmarkierung in $\mathit{work2}$ zu aktualisieren. Zuerst wird in Zeile
 \ref{alg:arlabeling-neighbour-case2-3-incregion} die Anzahl der Vordergrundpixel der Region erhöht. An Position 1 und
 Position 2 von $\mathit{work2}$ werden die Werte von $i$ und $j$ aufaddiert. Falls in Zeile
 \ref{alg:arlabeling-neighbour-case2-3-isismaller} die Position $x$ des ersten Vordergrundpixels der Region größer ist,
 als der aktuelle Wert in $i$, wird die Position in Zeile \ref{alg:arlabeling-neighbour-case2-3-newi} aktualisiert. Zum
 Schluss wird die Position $y$ des letzten Vordergrundpixels der Region durch den aktuellen Wert $j$ ersetzt.

% Fall 2: *(pnt1+1) > 0
\begin{algorithm}[!ht]
\caption{\textproc{arLabeling} (8er-Nachbarschaft: 2. Fall, $N_4$)}
\label{alg:arlabeling-neighbour-case2-3}
\begin{algorithmic}[1]
	\State $\mathit{pnt2} \gets \left(\mathit{pnt1} + 1\right)$
	\Cost{$c_{1}$}{$2$}
	\label{alg:arlabeling-neighbour-case2-3-n4}
	\State $\mathit{pnt2\_index} \gets \left(\left(\mathit{pnt2}\right) - 1\right) \cdot 7$
	\Cost{$c_{2}$}{$4$}
	\State $\mathit{work2}\left[\mathit{pnt2\_index} + 0\right] \gets
	 \mathit{work2}\left[\mathit{pnt2\_index} + 0\right] + 1$
	\Cost{$c_{3}$}{$6$}
	\label{alg:arlabeling-neighbour-case2-3-incregion}
	\State $\mathit{work2}\left[\mathit{pnt2\_index} + 1\right] \gets
	 \mathit{work2}\left[\mathit{pnt2\_index} + 1\right] + i$
	\Cost{$c_{4}$}{$6$}
	\State $\mathit{work2}\left[\mathit{pnt2\_index} + 2\right] \gets
	 \mathit{work2}\left[\mathit{pnt2\_index} + 2\right] + j$
	\Cost{$c_{5}$}{$6$}
	\If{$\mathit{work2}\left[\mathit{pnt2\_index} + 3\right] > i$}
	\Cost{$c_{6}$}{$3$}
	\label{alg:arlabeling-neighbour-case2-3-isismaller}
		\State $\mathit{work2}\left[\mathit{pnt2\_index} + 3\right] \gets i$
		\Cost{$c_{7}$}{$3$}
		\label{alg:arlabeling-neighbour-case2-3-newi}
	\EndIf
	\State $\mathit{work2}\left[\mathit{pnt2\_index} + 6\right] \gets j$
	\Cost{$c_{9}$}{$3$}
\end{algorithmic}
\end{algorithm}


% paragraph fall_2_ (end)

\paragraph{3. Fall:} % (fold)
\label{par:fall_3_}
Beim dritten Fall wird die Markierung $N_2$ untersucht und wir wissen, dass $N_3$ und $N_4$ keine Markierungen haben
 können. Demnach kann es nur den Nachbarn $N_2$ geben, dessen Markierung an $I(u,v)$ weitergereicht wird. Falls $N_1$
 ebenfalls ein Vordergrundpixel ist, handelt es sich um die gleiche Markierung wie in $N_2$, da beide Vordergrundpixel
 direkt miteinander verbunden sind. Das Verfahren ist in \autoref{alg:arlabeling-neighbour-case3} dargestellt.

% Fall 3: *(pnt1-1) > 0
\begin{algorithm}[ht]
\caption{\textproc{arLabeling} (8er-Nachbarschaft: 3. Fall)}
\label{alg:arlabeling-neighbour-case3}
\begin{algorithmic}[1]
	\algrestore{brk-case2}
	\ElsIf{$\left(\mathit{pnt1} - 1\right) > 0$}
		\State $\mathit{pnt2} \gets \left(\mathit{pnt1} - 1\right)$
		\State $\mathit{pnt2\_index} \gets \left(\mathit{pnt2} - 1\right) \cdot 7$
		\label{alg:arlabeling-neighbour-case3-offset}
		\State $\mathit{work2}\left[\mathit{pnt2\_index} + 0\right] \gets \mathit{work2}\left[\mathit{pnt2\_index} + 0\right] + 1$
		\State $\mathit{work2}\left[\mathit{pnt2\_index} + 1\right] \gets \mathit{work2}\left[\mathit{pnt2\_index} + 1\right] + i$
		\State $\mathit{work2}\left[\mathit{pnt2\_index} + 2\right] \gets \mathit{work2}\left[\mathit{pnt2\_index} + 2\right] + j$
		\If{$\mathit{work2}\left[\mathit{pnt2\_index} + 4\right] < i$}
			\State $\mathit{work2}\left[\mathit{pnt2\_index} + 4\right] \gets i$
			\label{alg:arlabeling-neighbour-case3-newi}
		\EndIf
		\State $\mathit{work2}\left[\mathit{pnt2\_index} + 6\right] \gets j$
		\algstore{brk-case3}
\end{algorithmic}
\end{algorithm}


Für diesen Fall müssen nur die Daten in $\mathit{work2}$ gespeichert werden. Der Wert von $N_2$ wird im Regionenbild
 $\mathit{l\_image}$ gespeichert. Danach wird in Zeile \ref{alg:arlabeling-neighbour-case3-offset} der Adressabstand
 zur Speicherung der Daten in $\mathit{work2}$ berechnet. Danach wird die Anzahl der Vordergrundpixel der Region erhöht
 und $i$ und $j$ auf die bestehenden Werte in $\mathit{work2}$ addiert. Falls die $x$-Koordinate des letzten
 Vordergrundpixels der Region kleiner als der aktuelle Wert $i$ ist, wird die $x$-Koordinate in Zeile
 \ref{alg:arlabeling-neighbour-case3-newi} ersetzt. Zum Schluss wird die $y$-Koordinate des letzten Vordergrundpixels
 der Region mit dem Wert $j$ aktualisiert.

% paragraph fall_3_ (end)

\paragraph{4. Fall:} % (fold)
\label{par:fall_4_}
Der vierte Fall untersucht den letzten, und einzigen, Nachbarn $N_1$ (\autoref{alg:arlabeling-neighbour-case4}). Alle
 anderen Nachbarn sind keine Vordergrundpixel und es besteht keine Kollision. Die Markierung von $N_1$ wird für
 $I(u,v)$ übernommen (Zeile \ref{alg:arlabeling-neighbour-case4-n1}). Ansonsten sind
 \autoref{alg:arlabeling-neighbour-case4} und \autoref{alg:arlabeling-neighbour-case3} identisch.

% Fall 4: *(pnt2-1) > 0
\begin{algorithm}[ht]
\caption{\textproc{arLabeling} (8er-Nachbarschaft: 4. Fall)}
\label{alg:arlabeling-neighbour-case4}
\begin{algorithmic}[1]
	\algrestore{brk-case3}
	\ElsIf{$\left(\mathit{pnt2} - 1\right) > 0$}
		\State $\mathit{pnt2} \gets \left(\mathit{pnt2} - 1\right)$
		\label{alg:arlabeling-neighbour-case4-n1}
		\State $\mathit{pnt2\_index} \gets \left(\left(\mathit{pnt2}\right) - 1\right) \cdot 7$
		\State $\mathit{work2}\left[\mathit{pnt2\_index} + 0\right] \gets \mathit{work2}\left[\mathit{pnt2\_index} + 0\right] + 1$
		\State $\mathit{work2}\left[\mathit{pnt2\_index} + 1\right] \gets \mathit{work2}\left[\mathit{pnt2\_index} + 1\right] + i$
		\State $\mathit{work2}\left[\mathit{pnt2\_index} + 2\right] \gets \mathit{work2}\left[\mathit{pnt2\_index} + 2\right] + j$
		\If{$\mathit{work2}\left[\mathit{pnt2\_index} + 4\right] < i$}
			\State $\mathit{work2}\left[\mathit{pnt2\_index} + 4\right] \gets i$
		\EndIf
	\algstore{brk-case4}
\end{algorithmic}
\end{algorithm}


% paragraph fall_4_ (end)

\paragraph{5. Fall:} % (fold)
\label{par:fall_5_}
Im letzten Fall sind alle Nachbarn Hintergrundpixel und $\mathit{l\_image}$ wird eine neue Markierung für $I(u,v)$
 zugewiesen (\autoref{alg:arlabeling-neighbour-case5}).

% Fall 5: else
\begin{algorithm}[!ht]\small
\caption{\textproc{arLabeling} (8er-Nachbarschaft: 5. Fall)}
\label{alg:arlabeling-neighbour-case5}
\begin{algorithmic}[1]
	\algrestore{brk-case4}
	\Else
		\State $\mathit{wk\_max} \gets \mathit{wk\_max} + 1$
		\Cost{$c_{36}$}{$2$}
		\label{alg:arlabeling-neighbour-case5-incwk_max}
		\If{$\mathit{wk\_max} > \mathit{WORK\_SIZE}$}
		\Cost{$c_{37}$}{$1$}
			\State{\textbf{return} $0$}
			\Cost{$c_{38}$}{$1$}
		\EndIf
		\State $\mathit{pnt2} \gets \mathit{wk\_max}$
		\Cost{$c_{40}$}{$1$}
		\label{alg:arlabeling-neighbour-case5-save-uv}
		\State $\mathit{work}\left[\mathit{wk\_max} - 1\right] \gets \mathit{wk\_max}$
		\Cost{$c_{41}$}{$3$}
		\State $\mathit{wmax\_idx} \gets \left(\mathit{wk\_max} - 1\right) \cdot 7$
		\Cost{$c_{42}$}{$3$}
		\label{alg:arlabeling-neighbour-case5-offset}
		\State $\mathit{work2}\left[\mathit{wmax\_idx} + 0 \right] \gets 1$
		\Cost{$c_{43}$}{$3$}
		\label{alg:arlabeling-neighbour-case5-offset-0}
		\State $\mathit{work2}\left[\mathit{wmax\_idx} + 1 \right] \gets i$
		\Cost{$c_{44}$}{$3$}
		\label{alg:arlabeling-neighbour-case5-offset-1}
		\State $\mathit{work2}\left[\mathit{wmax\_idx} + 2 \right] \gets j$
		\Cost{$c_{45}$}{$3$}
		\State $\mathit{work2}\left[\mathit{wmax\_idx} + 3 \right] \gets i$
		\Cost{$c_{46}$}{$3$}
		\State $\mathit{work2}\left[\mathit{wmax\_idx} + 4 \right] \gets i$
		\Cost{$c_{47}$}{$3$}
		\State $\mathit{work2}\left[\mathit{wmax\_idx} + 5 \right] \gets j$
		\Cost{$c_{48}$}{$3$}
		\State $\mathit{work2}\left[\mathit{wmax\_idx} + 6 \right] \gets j$
		\Cost{$c_{49}$}{$3$}
		\label{alg:arlabeling-neighbour-case5-offset-6}
	\EndIf
\end{algorithmic}
\end{algorithm}


In Zeile \ref{alg:arlabeling-neighbour-case5-incwk_max} wird der aktuelle numerische Markierungswert erhöht. Falls der
 Wert in $\mathit{wk\_max}$ größer als ein festgelegter Wert ist, wird das Verfahren abgebrochen, da zuviele Regionen
 im Bildsignal $I$ vorkommen. Die Markierung wird in Zeile \ref{alg:arlabeling-neighbour-case5-save-uv} in
 $\mathit{l\_image}$ gespeichert. Danach wird die der Wert der Markierung in die Liste der Markierungen $\mathit{work}$
 eingetragen. Der Adressabstand für die Region mit der Markierung $\mathit{wk\_max}$ wird in Zeile
 \ref{alg:arlabeling-neighbour-case5-offset} berechnet. Anschliessend wird die neue Region in $\mathit{work2}$
 gespeichert. Der erste Vordergrundpixel wird in Zeile \ref{alg:arlabeling-neighbour-case5-offset-0} an Position $0$
 von $\mathit{work2}$ gespeichert. Danach werden in Zeile
 \ref{alg:arlabeling-neighbour-case5-offset-1}--\ref{alg:arlabeling-neighbour-case5-offset-1} die Position des ersten
 Vordergrundpixels gespeichert.

% paragraph fall_5_ (end)

Nachdem die Regionemarkierung abgeschlossen ist, enthält $\mathit{work}$ die numerischen Werte der Regionen. In
 $\mathit{work2}$ ist für jede Region die Anzahl der Vordergrundpixel, die summierten $x$- und $y$-Koordinaten aller
 Vordergrundpixel, sowie die $x$- und $y$-Koordinate für den ersten und letzten Vordergrundpixel, hinterlegt. Das
 Regionenbild ist in $\mathit{l\_image}$ gespeichert. Die erfassten Daten müssen nun in Schritt \ref{label-cleaning} von
 \autoref{alg:arlabeling-overview} aufbereitet werden. Zuerst werden mit \autoref{alg:sortlabels} die Werte
 in $\mathit{work}$ sortiert.

Die aufsteigenden Markierungswerte in $\mathit{work}$ sind duch Kollisionen in
 \autoref{alg:arlabeling-neighbour-case2-1} durch kleinere Werte ersetzt worden. Dadurch ist eine Lücke im Interval
 $\left[1,2,3..n\right]$ der numerischen Werte entstanden. Diese Lücken werden durch \autoref{alg:sortlabels}
 entfernt, um wieder ein aufsteigendes Interval zu erhalten.

\begin{algorithm}[ht]
\caption{\textproc{arLabeling} (Sortiere Markierungen)}
\label{alg:sortlabels}
\begin{algorithmic}[1]
	\State $j \gets 1$
	\label{alg:sortlabels-j}
	\State $\mathit{wk} \gets \mathit{work}\left[0\right]$
	\label{alg:sortlabels-address}
	\For{$i \gets 1$ \textbf{to} $i \leq \mathit{wk\_max}$}
	\label{alg:sortlabels-loop-start}
		\If{$\mathit{wk} == i$}
			\State $\mathit{wk} \gets j$
			\State $j \gets j + 1$
			\label{alg:sortlabels-savelabel}
		\Else
		\label{alg:sortlabels-collision-start}
			\State $\mathit{wk} \gets \mathit{work}\left[\mathit{wk} - 1\right]$ \Comment Letzten Markierungswert $\mathit{work}$ zuweisen.
		\EndIf
		\label{alg:sortlabels-collision-end}
		\State $i \gets i + 1$
		\label{alg:sortlabels-inci}
		\State Inkrementiere $\mathit{wk}$
		\label{alg:sortlabels-incwk}
	\EndFor
	\label{alg:sortlabels-loop-end}
\end{algorithmic}
\end{algorithm}


In Zeile \ref{alg:sortlabels-j} wird $j$ als Variable für die aktuelle Markierung initialisiert. In Zeile
 \ref{alg:sortlabels-address} wird die Adresse der ersten Markierung aus $\mathit{work}$ in $\mathit{wk}$ gespeichert.
 Die Schleife in Zeile \ref{alg:sortlabels-loop-start}--\ref{alg:sortlabels-loop-end} überprüft, ob die numerischen
 Werte ohne Lücken durch Kollisionen gespeichert sind. Dazu wird jeder Eintrag in $\mathit{work}$ mit der Laufvariable
 $i$ verglichen. Falls keine Kollisionen vorliegen, werden die Werte in $\mathit{work}$ immer mit $i$ übereinstimmen,
 da keine Lücken im Intervall vorhanden sind. In Zeile \ref{alg:sortlabels-savelabel} wird dann der Wert aus $j$ an die
 aktuelle Position von $\mathit{work}$ geschrieben und $j$ danach inkrementiert. Bei einer Kollision stimmt der Wert in
 $\mathit{work}$ nicht mit $i$ überein
 (Zeile \ref{alg:sortlabels-collision-start}--\ref{alg:sortlabels-collision-end}). In diesem Fall wird der Wert an der
 aktuellen Position von $\mathit{work}$ ausgelesen und dekrementiert. Dieser Wert wird als Index verwendet um den Wert
 des zuletzt zugewiesenen Markierungswert an die aktuelle Position von $\mathit{work}$ zu schreiben. Da die Variable
 $j$ in diesem Fall nicht inkrementiert wird, enthält sie den nächsten Wert des Markierungsintervals. In Zeile
 \ref{alg:sortlabels-inci} wird die Laufvariable $i$ um $1$ erhöht und in Zeile \ref{alg:sortlabels-incwk} die Adresse
 von $\mathit{work}$ inkrementiert. Das Beispiel in \autoref{fig:} illustriert das Verfahren.

Das Verfahren in \autoref{alg:arlabeling-initlabelmemory} initialisiert den Speicher zur Berechnung des Flächeninhalts
 einer Region und den Speicher der Koordinaten des Mittelpunkts einer Region. Dazu wird in Zeile
 \ref{alg:arlabeling-initlabelmemory-label-start}--\ref{alg:arlabeling-initlabelmemory-label-end} der größte
 Markierungswert aus $\mathit{work}$ in $\mathit{wlabel\_num}$ und $\mathit{label\_num}$ gespeichert. Falls in der
 Überprüfung in Zeile
 \ref{alg:arlabeling-initlabelmemory-islabel-start}--\ref{alg:arlabeling-initlabelmemory-islabel-end}
 $\mathit{label\_num}$ keine Markierung enthält, wird das Verfahren abgebrochen und das Regionenbild $\mathit{l\_image}$
 zurückgegeben. Andernfalls wird in Zeile
 \ref{alg:arlabeling-initlabelmemory-initwarea-start}--\ref{alg:arlabeling-initlabelmemory-initwarea-end} der Speicher
 des Flächeinhalts $\mathit{warea}$ mit dem Wert $0$ initialisiert, indem über die Größe des Speichers iteriert wird.
 Der Speicher zur Berechnung des Mittelpunkts der Regionen wird in Zeile
 \ref{alg:arlabeling-initlabelmemory-initwpos-start}--\ref{alg:arlabeling-initlabelmemory-initwpos-end} nach dem
 gleichen Prinzip initialisiert.

\begin{algorithm}[ht]
\caption{\textproc{arLabeling} (Regionenspeicher initialisieren)}
\label{alg:arlabeling-initlabelmemory}
\begin{algorithmic}[1]
	\State $\mathit{wlabel\_num} \gets j - 1$
	\label{alg:arlabeling-initlabelmemory-label-start}
	\State $\mathit{label\_num} \gets j - 1$
	\label{alg:arlabeling-initlabelmemory-label-end}
	\If{$\mathit{label\_num} == 0$}
	\label{alg:arlabeling-initlabelmemory-islabel-start}
		\State \textbf{return} $\mathit{l\_image}$
	\EndIf
	\label{alg:arlabeling-initlabelmemory-islabel-end}
	\State $\mathit{size} \gets \mathit{label\_num} \cdot$ \Call{sizeof}{$int$}
	\label{alg:arlabeling-initlabelmemory-initwarea-start}
	\For{$\mathit{size} > 0$}
		\State $\mathit{warea} \gets 0$
		\State Inkrementiere $\mathit{warea}$
		\State $\mathit{size} \gets \mathit{size} - 1$
	\EndFor
	\label{alg:arlabeling-initlabelmemory-initwarea-end}
	\State $\mathit{size} \gets \mathit{label\_num} \cdot 2 \cdot$ \Call{sizeof}{$float$}
	\label{alg:arlabeling-initlabelmemory-initwpos-start}
	\For{$\mathit{size} > 0$}
		\State $\mathit{wpos} \gets 0$
		\State Inkrementiere $\mathit{wpos}$
		\State $\mathit{size} \gets \mathit{size} - 1$
	\EndFor
	\label{alg:arlabeling-initlabelmemory-initwpos-end}
\end{algorithmic}
\end{algorithm}


In \autoref{alg:arlabeling-calcregiondata} werden die Daten aller Regionenmarkierungen aufbereitet. Das Verfahren
 berechnet den Flächeninhalt, den Mittelpunkt und die Start- und Endkoordinaten für alle Regionen.

\begin{algorithm}[ht]
\caption{\textproc{arLabeling} (Berechne Regionendaten)}
\label{alg:arlabeling-calcregiondata}
\begin{algorithmic}[1]
    % for(i = 0; i < *label_num; i++) {
    %     wclip[i*4+0] = lxsize;
    %     wclip[i*4+1] = 0;
    %     wclip[i*4+2] = lysize;
    %     wclip[i*4+3] = 0;
    % }
	\For{$i \gets 0$ \textbf{to} $i < \mathit{label\_num}$}
	\label{alg:arlabeling-calcregiondata-wclip-start}
		\State $\mathit{wclip}\left[i \cdot 4 + 0\right] \gets \mathit{lxsize}$
		\State $\mathit{wclip}\left[i \cdot 4 + 1\right] \gets 0$
		\State $\mathit{wclip}\left[i \cdot 4 + 2\right] \gets \mathit{lysize}$
		\State $\mathit{wclip}\left[i \cdot 4 + 3\right] \gets 0$
		\State $i \gets i + 1$
	\EndFor
	\label{alg:arlabeling-calcregiondata-wclip-end}
    % for(i = 0; i < wk_max; i++) {
    %     j = work[i] - 1;
    %     warea[j]    += work2[i*7+0];
    %     wpos[j*2+0] += work2[i*7+1];
    %     wpos[j*2+1] += work2[i*7+2];
    %     if( wclip[j*4+0] > work2[i*7+3] ) wclip[j*4+0] = work2[i*7+3];
    %     if( wclip[j*4+1] < work2[i*7+4] ) wclip[j*4+1] = work2[i*7+4];
    %     if( wclip[j*4+2] > work2[i*7+5] ) wclip[j*4+2] = work2[i*7+5];
    %     if( wclip[j*4+3] < work2[i*7+6] ) wclip[j*4+3] = work2[i*7+6];
    % }
	\For{$i \gets 0$ \textbf{to} $i < \mathit{wk\_max}$}
	\label{alg:arlabeling-calcregiondata-work-start}
		\State $j \gets \mathit{work}\left[i\right] - 1$
		\label{alg:arlabeling-calcregiondata-label}
		\State $\mathit{warea}\left[j\right] \gets \mathit{warea}\left[j\right] + \mathit{work2}\left[i \cdot 7 + 0\right]$
		\State $\mathit{wpos}\left[j \cdot 2 + 0\right] \gets \mathit{wpos}\left[j \cdot 2 + 0\right] + \mathit{work2}\left[i \cdot 7 + 1\right]$
		\label{alg:arlabeling-calcregiondata-sumx}
		\State $\mathit{wpos}\left[j \cdot 2 + 1\right] \gets \mathit{wpos}\left[j \cdot 2 + 1\right] + \mathit{work2}\left[i \cdot 7 + 2\right]$
		\label{alg:arlabeling-calcregiondata-sumy}
		\If{$\mathit{wclip}\left[i \cdot 4 + 0\right] > \mathit{work2}\left[i \cdot 7 + 3\right]$}
		\label{alg:arlabeling-calcregiondata-startx-start}
			\State $\mathit{wclip}\left[i \cdot 4 + 0\right] \gets \mathit{work2}\left[i \cdot 7 + 3\right]$
		\EndIf
		\label{alg:arlabeling-calcregiondata-startx-end}

		\If{$\mathit{wclip}\left[i \cdot 4 + 1\right] < \mathit{work2}\left[i \cdot 7 + 4\right]$}
		\label{alg:arlabeling-calcregiondata-endx-start}
			\State $\mathit{wclip}\left[i \cdot 4 + 1\right] \gets \mathit{work2}\left[i \cdot 7 + 4\right]$
		\EndIf
		\label{alg:arlabeling-calcregiondata-endx-end}

		\If{$\mathit{wclip}\left[i \cdot 4 + 2\right] > \mathit{work2}\left[i \cdot 7 + 5\right]$}
		\label{alg:arlabeling-calcregiondata-y-start}
			\State $\mathit{wclip}\left[i \cdot 4 + 2\right] \gets \mathit{work2}\left[i \cdot 7 + 5\right]$
		\EndIf

		\If{$\mathit{wclip}\left[i \cdot 4 + 3\right] < \mathit{work2}\left[i \cdot 7 + 6\right]$}
			\State $\mathit{wclip}\left[i \cdot 4 + 3\right] \gets \mathit{work2}\left[i \cdot 7 + 6\right]$
		\EndIf
		\label{alg:arlabeling-calcregiondata-y-end}

		\State $i \gets i + 1$
	\EndFor
	\label{alg:arlabeling-calcregiondata-work-end}
    % for( i = 0; i < *label_num; i++ ) {
    %     wpos[i*2+0] /= warea[i];
    %     wpos[i*2+1] /= warea[i];
    % }
	\For{$i \gets 0$ \textbf{to} $\mathit{label\_num}$}
	\label{alg:arlabeling-calcregiondata-pos-start}
		\State $\mathit{wpos}\left[i \cdot 2 + 0\right] \gets \mathit{wpos}\left[i \cdot 2 + 0\right] / \mathit{warea}\left[i\right]$
		\State $\mathit{wpos}\left[i \cdot 2 + 1\right] \gets \mathit{wpos}\left[i \cdot 2 + 1\right] / \mathit{warea}\left[i\right]$
		\State $i \gets i + 1$
	\EndFor
	\label{alg:arlabeling-calcregiondata-pos-end}
	% *label_ref = work;
	% *area      = warea;
	% *pos       = wpos;
	% *clip      = wclip;
	% return( l_image );
	\State $\mathit{label\_ref} \gets \mathit{work}$
	\State $\mathit{area} \gets \mathit{warea}$
	\State $\mathit{pos} \gets \mathit{wpos}$
	\State $\mathit{clip} \gets \mathit{wclip}$
	\State \textbf{return} $\mathit{l\_image}$
\end{algorithmic}
\end{algorithm}


Dazu wird in Zeile \ref{alg:arlabeling-calcregiondata-wclip-start}--\ref{alg:arlabeling-calcregiondata-wclip-end} der
 Speicher der Start- und Endkoordinaten initialisiert, indem über die Menge der Regionen itertiert wird und die
 Initialwerte, $\mathit{lxsize}$ und $0$ für die Startkoordiante und $\mathit{lysize}$ und $0$ für die Endkoordinate,
 gespeichert werden. In Zeile
 \ref{alg:arlabeling-calcregiondata-work-start}--\ref{alg:arlabeling-calcregiondata-work-end} werden die Werte aus
 $\mathit{work2}$ aufbereitet, indem über die Anzahl der Regionen iteriert wird. In Zeile
 \ref{alg:arlabeling-calcregiondata-label} wird die Regionemarkierung ausgelesen und in $j$ gespeichert. Danach wird
 die Anzahl der Vordergrundpixel für Region $j$ aus $\mathit{work2}$ ausgelesen und in $\mathit{warea}$ gespeichert. In
 Zeile \ref{alg:arlabeling-calcregiondata-sumx}--\ref{alg:arlabeling-calcregiondata-sumy} werden die summierten $x$- und
 $y$-Koordinaten aus $\mathit{work2}$ ausgelsen und in $\mathit{wpos}$ gespeichert. Die Überprüfung des $x$-Werts in
 Zeile \ref{alg:arlabeling-calcregiondata-startx-start}--\ref{alg:arlabeling-calcregiondata-startx-end} sorgt dafür,
 dass die Startkoordinate soweit links wie möglich beginnt. Der $y$-Wert der Startkoordinate wird in Zeile
 \ref{alg:arlabeling-calcregiondata-starty-start}--\ref{alg:arlabeling-calcregiondata-starty-end} überprüft. Bei dieser
 Überprüfung soll der $y$-Wert soweit oben wie möglich liegen. Zeile
 \ref{alg:arlabeling-calcregiondata-endcoord-start}--\ref{alg:arlabeling-calcregiondata-endcoord-end} wiederholen das
 Verfahren für den $x$- und $y$-Wert der Endkoordinate. Danach wird in Zeile
 \ref{alg:arlabeling-calcregiondata-pos-start}--\ref{alg:arlabeling-calcregiondata-pos-end} die Position
 des Mittelpunkts einer Region berechnet, indem die aufsummierten Koordinaten einer Region durch die Anzahl der
 Vordergrundpixel in $\mathit{warea}$ geteilt wird. Zuletzt werden die lokalen Variablen in den Übergabeparametern
 gespeichert und das Regionenbild $\mathit{l\_image}$ an die aufrufende Methode zurückgegeben. An dieser Stelle ist
 \autoref{alg:arlabeling-overview} beendet.

% subsubsection regionenmarkierung (end)


\clearpage

\subsubsection{Konturerzeugung} % (fold)
\label{sec:konturerzeugung}

Das zweite Verfahren der Fiducial Detection ermittelt aus einer Regionenmarkierung mit dem Verfahren
 \textproc{arGetContour} eine Kontur. Nachdem \textproc{arLabeling} (\autoref{alg:arlabeling-overview}) beendet ist,
 wird das Regionenbild an die Methode \textproc{arDetectMarker2} übergeben.

In \autoref{alg:detectmarker2-1} werden die lokalen Variablen initialisiert, deren Bedeutung bei ihrem ersten Einsatz
 erläutert wird.
\begin{algorithm}[!ht]\small
\caption{\textproc{arDetectMarker2} (Initialisierung)}
\label{alg:detectmarker2-1}
\begin{algorithmic}[1]
	\Require $
	\begin{aligned}
		& \mathit{limage}, \mathit{label\_num}, \mathit{label\_ref}, \mathit{warea}, \mathit{wpos}, \mathit{wclip}, \mathit{area\_max}, \mathit{area\_min},\\
		& \mathit{factor}, \mathit{marker\_num}
	\end{aligned}$
	\State $\mathit{pm}, i, j, d, \mathit{ret} \gets \infty$
	\State $\mathit{area\_min} \gets \mathit{area\_min} / 4$
	\State $\mathit{area\_max} \gets \mathit{area\_max} / 4$
	\State $\mathit{xsize} \gets \mathit{arImXsize} / 2$
	\State $\mathit{ysize} \gets \mathit{arImYsize} / 2$
	\algstore{brk-detectmarker2-init}
\end{algorithmic}
\end{algorithm}

Die Laufzeitfunktion von \autoref{alg:detectmarker2-1} ist $T(n) = 13$. \autoref{alg:detectmarker2-2} sorgt mit seiner
 Schleife in Zeile \ref{alg:detectmarker2-2-loop-start}--\ref{alg:detectmarker2-2-loop-end} für die Konturenermittlung.
\begin{algorithm}[ht]
\caption{\textproc{arDetectMarker2} (Fortsetzung)}
\label{alg:detectmarker2-2}
\begin{algorithmic}[1]
	\algrestore{brk-detectmarker2-init}
	\State $\mathit{marker\_num2} \gets 0$
	\label{alg:detectmarker2-2-markernum2}
	\For{$i \gets 0$ \textbf{to} $i < \mathit{label\_num}$}
	\label{alg:detectmarker2-2-loop-start}
		\If{$\mathit{warea}[i] < \mathit{area\_min} \lor \mathit{warea}[i] > \mathit{area\_max}$}
		\label{alg:detectmarker2-2-area-start}
			\State \textbf{continue}
		\EndIf
		\label{alg:detectmarker2-2-area-end}
		\If{$\mathit{wclip}[i \cdot 4 + 0] == 1 \lor \mathit{wclip}[i \cdot 4 + 1] == \mathit{xsize} - 2$}
		\label{alg:detectmarker2-2-checkx-start}
			\State \textbf{continue}
			\label{alg:detectmarker2-2-checkx-continue}
		\EndIf
		\label{alg:detectmarker2-2-checkx-end}
		\If{$\mathit{wclip}[i \cdot 4 + 2] == 1 \lor \mathit{wclip}[i \cdot 4 + 3] == \mathit{ysize} - 2$}
		\label{alg:detectmarker2-2-checky-start}
			\State \textbf{continue}
		\EndIf
		\label{alg:detectmarker2-2-checky-end}
		\State $\mathit{ret} \gets$ \textproc{arGetContour}$\left(
		\begin{aligned}
			& \mathit{limage}, \mathit{label\_ref}, i + 1,\\
			& \mathit{wclip}[i \cdot 4], \mathit{marker\_infoTWO}[\mathit{marker\_num2}]
		\end{aligned}\right)$
		\label{alg:detectmarker2-2-callcontour}
		\State \ldots
		\State $i \gets i + 1$
	\EndFor
	\label{alg:detectmarker2-2-loop-end}
\end{algorithmic}
\end{algorithm}

Dazu wird in Zeile \ref{alg:detectmarker2-2-markernum2} die Variable $\mathit{marker\_num2}$ initialisiert, die zur
 Identifizierung einer Region dient. In Zeile \ref{alg:detectmarker2-2-loop-start}--\ref{alg:detectmarker2-2-loop-end}
 wird jede Regionenmarkierung untersucht. Dazu wird in Zeile
 \ref{alg:detectmarker2-2-area-start}--\ref{alg:detectmarker2-2-area-end} die Fläche der Region $i$ mit dem festgelegten
 minimalen und maximalen Wert verglichen. Nur wenn die Fläche der Region sich innerhalb dieser Grenzen befindet, wird
 die Region $i$ weiter untersucht. Andernfalls wird mit der nächsten Region das Verfahren wiederholt. Zeile
 \ref{alg:detectmarker2-2-checkx-start}--\ref{alg:detectmarker2-2-checkx-end} überprüft die $x$-Koordinaten der
 Start- und Endposition, die in $\mathit{wclip}$ gespeichert sind. Wenn die Koordinaten an den Rändern des
 Regionenbildes $\mathit{limage}$ liegen, wird die weitere Untersuchung in Zeile
 \ref{alg:detectmarker2-2-checkx-continue} abgebrochen. Nur Regionen, die nicht an den Rändern liegen, eignen sich zur
 Konturenermittlung. In Zeile \ref{alg:detectmarker2-2-checky-start}--\ref{alg:detectmarker2-2-checky-end} wird das
 Verfahren für die $y$-Koordinaten wiederholt. Nur wenn eine Region eine festgelegte Größe nicht unter- oder
 überschreitet und die Region nicht an den Rändern liegt,
 eignet sie sich zur Konturenermittlung und wird mit der Methode \textproc{arGetContour} in Zeile
 \ref{alg:detectmarker2-2-callcontour} aufgerufen. Die Laufzeitfunktion ist in \autoref{eq:analyse-detectmarker2-1}
 angegeben.
\begin{subequations}
\label{eq:analyse-detectmarker2-1}
\begin{align}
\label{eq:analyse-detectmarker2-1-1}
T(\mathit{label\_num})& =
c_{6}
+ c_{7}(\mathit{label\_num} + 1)
+ 5c_{8}(\mathit{label\_num})
+ c_{9}(\mathit{label\_num})
\\
& \quad
+ 10c_{11}(\mathit{label\_num})
+ 10c_{14}(\mathit{label\_num})
+ tc_{17}(\mathit{label\_num})
\nonumber \\
& \quad
+ c_{19}(\mathit{label\_num})
\nonumber \\
\label{eq:analyse-detectmarker2-1-2}
T(\mathit{label\_num})& =
tc_{17}(\mathit{label\_num})
\\
& \quad
+ (c_{7} + 5c_{8} + c_{9} + 10c_{11} + 10c_{14} + c_{19}) (\mathit{label\_num})
+ (c_{6} + c_{7})
\nonumber
\end{align}
\end{subequations}


\textproc{arGetContour} (\autoref{alg:argetcontour-1}--\autoref{alg:argetcontour-3}) ermittelt für eine Region eine
 Kontur. In \autoref{alg:argetcontour-1} werden die Variablen in Zeile
 \ref{alg:argetcontour-1-initvar-start}--\ref{alg:argetcontour-1-initvar-end} initialisiert.
\begin{algorithm}[ht]
\caption{\textproc{arGetContour} (Initialisierung)}
\label{alg:argetcontour-1}
\begin{algorithmic}[1]
	\Require $\mathit{limage},\mathit{label\_ref},\mathit{label},\mathit{wclip}\left[4\right],\mathit{marker\_infoTWO}$ 
	% static const int      xdir[8] = { 0, 1, 1, 1, 0,-1,-1,-1};
	% static const int      ydir[8] = {-1,-1, 0, 1, 1, 1, 0,-1};
	\State $\mathit{xdir} \gets \left[0, 1, 1, 1, 0,-1,-1,-1\right]$
	\label{alg:argetcontour-1-initvar-start}
	\State $\mathit{ydir} \gets \left[-1,-1, 0, 1, 1, 1, 0,-1\right]$
	\label{alg:argetcontour-1-initvar-neighbour}
	% ARInt16         *p1;
	% int             xsize, ysize;
	% int             sx, sy, dir;
	% int             dmax, d, v1 = 0;
	% int             i, j;
	\State $\mathit{p1} \gets \infty$
	\State $\mathit{sx} \gets \infty$
	\State $\mathit{sy} \gets \infty$
	\State $\mathit{dir} \gets \infty$
	\State $\mathit{dmax} \gets \infty$
	\State $d \gets \infty$
	\State $\mathit{v1} \gets 0$
	\State $i \gets \infty$
	\State $j \gets \infty$
	%     xsize = arImXsize / 2;
	%     ysize = arImYsize / 2;
	\State $\mathit{xsize} \gets \mathit{arImXsize} / 2$
	\label{alg:argetcontour-1-initsizex}
	\State $\mathit{ysize} \gets \mathit{arImYsize} / 2$
	\label{alg:argetcontour-1-initvar-end}
	% j = clip[2];
	% p1 = &(limage[j*xsize+clip[0]]);
	\State $j \gets \mathit{clip}\left[2\right]$
	\State $\mathit{offset} \gets j \cdot \mathit{xsize} + \mathit{clip}\left[0\right]$
	\label{alg:argetcontour-1-offset}
	\State $\mathit{p1} \gets \mathit{limage}\left[\mathit{offset}\right]$
	\label{alg:argetcontour-1-p1}
	% for( i = clip[0]; i <= clip[1]; i++, p1++ ) {
	%     if( *p1 > 0 && label_ref[(*p1)-1] == label ) {
	%         sx = i; sy = j; break;
	%     }
	% }
	\For{$i \gets \mathit{clip}\left[0\right]$ \textbf{to} $i \leq \mathit{clip}\left[1\right]$}
	\label{alg:argetcontour-1-loop-start}
		\If{$\mathit{p1} > 0 \land \mathit{label\_ref}\left[\mathit{p1 - 1}\right] == \mathit{label}$}
		\label{alg:argetcontour-1-haslabel-start}
			\State $\mathit{sx} \gets i$
			\label{alg:argetcontour-1-savex}
			\State $\mathit{sy} \gets j$
			\label{alg:argetcontour-1-savey}
			\State \textbf{break}
		\EndIf
		\label{alg:argetcontour-1-haslabel-end}
		\State $i \gets i + 1$
		\label{alg:argetcontour-1-inci}
		\State Inkrementiere $\mathit{p1}$
		\label{alg:argetcontour-1-incp1}
	\EndFor
	\label{alg:argetcontour-1-loop-end}
	\algstore{brk-argetcontour-init}
\end{algorithmic}
\end{algorithm}

Die 8er-Nachbarschaft zur Verfolgung einer Kontur wird in Zeile
 \ref{alg:argetcontour-1-initvar-start}--\ref{alg:argetcontour-1-initvar-neighbour} initialisiert. Die Bildbreite und
 Bildhöhe werden in Zeile \ref{alg:argetcontour-1-initsizex}--\ref{alg:argetcontour-1-initvar-end} in $\mathit{xsize}$
 und $\mathit{ysize}$ gespeichert. Danach wird in Zeile \ref{alg:argetcontour-1-offset} ein Adressabstand berechnet, um
 den Startpunkt der ersten Regionenmarkierung in $\mathit{l\_image}$ in $\mathit{p1}$ zu speichern (Zeile
 \ref{alg:argetcontour-1-p1}). In der Schleife in Zeile
 \ref{alg:argetcontour-1-loop-start}--\ref{alg:argetcontour-1-loop-end} wird der Startpunkt der Region gesucht. Dazu
 wird in Zeile \ref{alg:argetcontour-1-haslabel-start}--\ref{alg:argetcontour-1-haslabel-end} geprüft, ob an der
 Adresse $\mathit{p1}$ eine Regionenmarkierung gespeichert ist und ob der Wert der Regionenmarkierung in
 $\mathit{label\_ref}$ mit der Markierung in $\mathit{label}$ übereinstimmt. Falls die Werte übereinstimmen, werden die
 Koordinaten in Zeile \ref{alg:argetcontour-1-savex}--\ref{alg:argetcontour-1-savey} in den Variablen $\mathit{sx}$ und
 $\mathit{sy}$ gespeichert und die Schleife wird abgebrochen. Ansonsten wird in Zeile
 \ref{alg:argetcontour-1-inci}--\ref{alg:argetcontour-1-incp1} die Laufvariable $i$ um $1$ erhöht und die Adresse
 $\mathit{p1}$ inkrementiert. Der Bereich $[\mathit{clip}[0],\mathit{clip}[1])$ hängt von $\mathit{xsize}$ ab und
 umfasst maximal $\mathit{xsize} - 4$ Einträge (Vgl. \autoref{alg:detectmarker2-2}, Zeile
 \ref{alg:detectmarker2-2-checkx-start}). Die Laufzeitfunktion ist in \autoref{eq:analyse-argetconout-init} aufgeführt.
\begin{subequations}
\label{eq:analyse-argetconout-init}
\begin{align}
\label{eq:analyse-argetconout-init-1}
T_{worst}(\mathit{xsize})& = (c_{1} + c_{2} + c_{3}6 + c_{4} + c_{5}2 + c_{6}2 + c_{7}2 + c_{8}2 + c_{9}4 + c_{10}2)
\\
& \quad
+ c_{11}(\mathit{xsize} - 3)
+ (c_{12}5 + c_{17} + c_{18})(\mathit{xsize} - 4)
\nonumber \\
\label{eq:analyse-argetconout-init-2}
T_{worst}(\mathit{xsize})& =
(c_{11} + c_{12}5 + c_{17} + c_{18})\mathit{xsize}
- (3c_{11} + 20c_{12} + 4c_{17} + 4c_{18})
\\
& \quad
+  (c_{1} + c_{2} + c_{3}6 + c_{4} + c_{5}2 + c_{6}2 + c_{7}2 + c_{8}2 + c_{9}4 + c_{10}2)
\nonumber
\end{align}
\end{subequations}

Die Wachstumsrate von \autoref{eq:analyse-argetconout-init-2} ist, für $c_{1} = 7$, $c_{2} = 8$ und
 $\mathit{xsize}_0 = 8$, $8\mathit{xsize} - 8 = \Theta(\mathit{xsize})$.

\autoref{alg:argetcontour-2} verfolgt nun die Kontur der Region. Dazu wird zuerst in der Datenstruktur
 $\mathit{marker\_infoTWO}$ die Anzahl der Koordinaten initialisiert und die $x$- und $y$-Werte der ersten
 Regionenmarkierung gespeichert (Zeile
 \ref{alg:argetcontour-2-markerinit-start}--\ref{alg:argetcontour-2-markerinit-end}).
\begin{algorithm}[!ht]
\caption{\textproc{arGetContour} (Verfolge Kontur)}
\label{alg:argetcontour-2}
\begin{algorithmic}[1]
	\algrestore{brk-argetcontour-init}
	% marker_infoTWO->coord_num = 1;
	% marker_infoTWO->x_coord[0] = sx;
	% marker_infoTWO->y_coord[0] = sy;
	\State $\mathit{marker\_infoTWO.coord\_num} \gets 1$
	\Cost{$c_{20}$}{$2$}
	\label{alg:argetcontour-2-markerinit-start}
	\State $\mathit{marker\_infoTWO.xcoord}\left[0\right] \gets \mathit{sx}$
	\Cost{$c_{21}$}{$3$}
	\State $\mathit{marker\_infoTWO.ycoord}\left[0\right] \gets \mathit{sy}$
	\Cost{$c_{22}$}{$3$}
	\label{alg:argetcontour-2-markerinit-end}
	% dir = 5;
	\State $\mathit{dir} \gets 5$
	\Cost{$c_{23}$}{$1$}
	\label{alg:argetcontour-2-orientation}
	% for(;;) {
	\For{\textbf{true}}
	\label{alg:argetcontour-2-contourloop-start}
	%     p1 = &(limage[marker_infoTWO->y_coord[marker_infoTWO->coord_num-1] * xsize
	%                 + marker_infoTWO->x_coord[marker_infoTWO->coord_num-1]]);
		\State $\mathit{tmp\_coord\_num} \gets \mathit{marker\_infoTWO.coord\_num} - 1$
		\Cost{$c_{25}$}{$(n-3) \cdot 3$}
		\label{alg:argetcontour-2-offset-start}
		\State $\mathit{offset} \gets \mathit{marker\_infoTWO.y\_coord}\left[\mathit{tmp\_coord\_num}\right]
		 \cdot \mathit{xsize}$
		\Cost{$c_{26}$}{$(n-3) \cdot 5$}
		\State $\mathit{offset} \gets \mathit{offset} +
		 \mathit{marker\_infoTWO.x\_coord}\left[\mathit{tmp\_coord\_num}\right]$
		\Cost{$c_{27}$}{$(n-3) \cdot 5$}
		\label{alg:argetcontour-2-offset-end}
		\State $\mathit{p1} \gets \mathit{l\_image}[\mathit{offset}]$
		\Cost{$c_{28}$}{$(n-3) \cdot 2$}
		\label{alg:argetcontour-2-address}
	%     dir = (dir+5)%8;
		\State $\mathit{dir} \gets \left(\mathit{dir} + 5\right)\mod 8$
		\Cost{$c_{29}$}{$(n-3) \cdot 3$}
		\label{alg:argetcontour-2-nextorientation}
	%     for(i=0;i<8;i++) {
	%         if( p1[ydir[dir]*xsize+xdir[dir]] > 0 ) break;
	%         dir = (dir+1)%8;
	%     }
		\For{$i \gets 0$ \textbf{to} $i < 8$}
		\Cost{$c_{30}$}{$(n-3) \cdot \bigl((8 - 1) + 1\bigr)$}
		\label{alg:argetcontour-2-neighbourloop-start}
			\State $\mathit{offset} \gets \mathit{ydir}\left[dir\right] \cdot \mathit{xsize} +
			 \mathit{xdir}\left[\mathit{dir}\right]$
			\Cost{$c_{31}$}{$(n-3) \cdot 5(8 - 1)$}
			\label{alg:argetcontour-2-offset}
			\If{$\mathit{p1}\left[\mathit{offset}\right] > 0$}
			\Cost{$c_{32}$}{$(n-3) \cdot 2(8 - 1)$}
			\label{alg:argetcontour-2-haslabel-start}
				\State \textbf{break}
				\Cost{$c_{33}$}{$n-3$}
			\EndIf
			\label{alg:argetcontour-2-haslabel-end}
			\State $\mathit{dir} \gets \left(\mathit{dir} + 1\right)\mod 8$
			\Cost{$c_{35}$}{$(n-3) \cdot 3(8 - 1)$}
			\label{alg:argetcontour-2-incdir}
			\State $i \gets i + 1$
			\Cost{$c_{36}$}{$(n-3)(8 - 1)$}
			\label{alg:argetcontour-2-inci}
		\EndFor
		\label{alg:argetcontour-2-neighbourloop-end}
		% if( i == 8 ) {
		%             printf("??? 2\n"); return(-1);
		%         }
		\If{$i = 8$}
		\Cost{$c_{38}$}{$(n-3) \cdot 1$}
		\label{alg:argetcontour-2-hasnolabel-start}
			\State \textbf{return}$(-1)$
			% \Cost{$c_{39}$}{$9997 \cdot 1$}
		\EndIf
		\label{alg:argetcontour-2-hasnolabel-end}
	%     marker_infoTWO->x_coord[marker_infoTWO->coord_num]
	%         = marker_infoTWO->x_coord[marker_infoTWO->coord_num-1] + xdir[dir];
		\State $\mathit{mkx} \gets \mathit{marker\_infoTWO.x\_coord}\left[\mathit{marker\_infoTwo.coord\_num}\right]$
		\Cost{$c_{41}$}{$(n-3) \cdot 4$}
		\label{alg:argetcontour-2-savex}
		\State $\mathit{mkx} \gets \mathit{marker\_infoTWO.x\_coord}\left[\mathit{marker\_infoTwo.coord\_num}
		 - 1\right] + \mathit{xdir}\left[\mathit{dir}\right]$
		\Cost{$c_{42}$}{$(n-3) \cdot 7$}
	%     marker_infoTWO->y_coord[marker_infoTWO->coord_num]
	%         = marker_infoTWO->y_coord[marker_infoTWO->coord_num-1] + ydir[dir];
		\State $\mathit{mky} \gets \mathit{marker\_infoTWO.y\_coord}
		\left[\mathit{marker\_infoTwo.coord\_num}\right]$
		\Cost{$c_{43}$}{$(n-3) \cdot 4$}
		\State $\mathit{mky} \gets \mathit{marker\_infoTWO.y\_coord}\left[\mathit{marker\_infoTwo.coord\_num} - 1\right]
		 + \mathit{ydir}\left[\mathit{dir}\right]$
		\Cost{$c_{44}$}{$(n-3) \cdot 7$}
		\label{alg:argetcontour-2-savey}
	%     if( marker_infoTWO->x_coord[marker_infoTWO->coord_num] == sx
	%      && marker_infoTWO->y_coord[marker_infoTWO->coord_num] == sy ) break;
		\If{$\mathit{mkx} = sx \land \mathit{mky} = sy$}
		\Cost{$c_{45}$}{$(n-3) \cdot 3$}
		\label{alg:argetcontour-2-iscontourclosed-start}
			\State \textbf{break}
			\Cost{$c_{46}$}{$1$}
		\EndIf
		\label{alg:argetcontour-2-iscontourclosed-end}
	%     marker_infoTWO->coord_num++;
		\State $\mathit{marker\_infoTWO.coord\_num} \gets \mathit{marker\_infoTWO.coord\_num} + 1$
		\Cost{$c_{48}$}{$(n-3) \cdot 4$}
		\label{alg:argetcontour-2-inccoordnum}
		% if( marker_infoTWO->coord_num == AR_CHAIN_MAX-1 ) {
		%             printf("??? 3\n"); return(-1);
		%         }
		\If{$\mathit{marker\_infoTWO.coord\_num} = \mathit{AR\_CHAIN\_MAX} - 1}$
		\Cost{$c_{49}$}{$(n-3) \cdot 3$}
		\label{alg:argetcontour-2-ismaxchainreached-start}
			\State \textbf{return}$(-1)$
			% \Cost{$c_{50}$}{$9997 \cdot 1$}
			\label{alg:argetcontour-2-error}
		\EndIf
		\label{alg:argetcontour-2-ismaxchainreached-end}
	% }
	\EndFor
	\label{alg:argetcontour-2-contourloop-end}
	\algstore{brk-argetcontour-contour}
\end{algorithmic}
\end{algorithm}

Danach wird die Orientiertung der 8er-Nachbarschaft in Zeile \ref{alg:argetcontour-2-orientation} festgelegt. Die
 Kontur wird in der Schleife in Zeile
 \ref{alg:argetcontour-2-contourloop-start}--\ref{alg:argetcontour-2-contourloop-end} verfolgt. Zuerst muss der
 Adressabstand der letzten Regionenmarkierung in $\mathit{l\_image}$ berechnet werden (Zeile
 \ref{alg:argetcontour-2-offset-start}--\ref{alg:argetcontour-2-offset-end}). Danach wird die Adresse der
 Regionenmarkierung in $\mathit{p1}$ hinterlegt (Zeile \ref{alg:argetcontour-2-address}). In Zeile
 \ref{alg:argetcontour-2-nextorientation} wird die Orientierung der 8er-Nachbarschaft auf die nächste zu untersuchende
 Richtung gedreht. Das drehen der 8er-Nachbarschaft erfolgt, indem von der letzten Position einer Kontur im
 Uhrzeigersinn auf den nächsten Nachbarn weitergerückt wird (Vgl. \autoref{fig:}). In Zeile
 \ref{alg:argetcontour-2-neighbourloop-start}--\ref{alg:argetcontour-2-neighbourloop-end} werden in der Schleife die
 Nachbarn im Uhrzeigersinn untersucht. Dazu wird mit Hilfe der 8er-Nachbarschaft in Zeile
 \ref{alg:argetcontour-2-offset} ein Adressabstand berechnet. Durch die Überprüfung einer Regionenmarkierung an der
 Adresse von $\mathit{p1}$ in Zeile \ref{alg:argetcontour-2-haslabel-start}--\ref{alg:argetcontour-2-haslabel-end} wird
 entschieden, ob die Schleife von Zeile
 \ref{alg:argetcontour-2-neighbourloop-start}--\ref{alg:argetcontour-2-neighbourloop-end} abgebrochen wird, weil eine
 Markierung gefunden wurde, oder ob ein weiterer Nachbar untersucht werden muss. Falls ein weiterer Nachbar untersucht
 werden muss, wird in Zeile \ref{alg:argetcontour-2-incdir} die Position der 8er-Nachbarschaft um eine Position
 weitergedreht und in Zeile \ref{alg:argetcontour-2-inci} die Laufvariable $i$ erhöht. Falls alle Nachbarn der
 Regionenmarkierung untersucht worden sind ohne eine weiter Markierung zu finden, wird das Verfahren in Zeile
 \ref{alg:argetcontour-2-hasnolabel-start}--\ref{alg:argetcontour-2-hasnolabel-end} mit einem Fehlerwert abgebrochen.
 Um eine erfolgreiche Iteration zu erlangen, wird die Schleife im schlechtesten Fall $8-1$-mal wiederholt, ohne in Zeile
 \ref{alg:argetcontour-2-hasnolabel-start} mit einem Fehlerwert beendet zu werden. Wenn eine weitere Markierung in
 Zeile \ref{alg:argetcontour-2-neighbourloop-start}--\ref{alg:argetcontour-2-neighbourloop-end} gefunden und die
 Schleife abgebrochen wurde, wird in Zeile \ref{alg:argetcontour-2-savex}--\ref{alg:argetcontour-2-savey} die
 Markierung gespeichert. Dazu werden die $x$- und $y$-Koordinate der letzten Markierung ausgelesen und die Richtung der
 8er-Nachbarschaft addiert. Danach wird in Zeile
 \ref{alg:argetcontour-2-iscontourclosed-start}--\ref{alg:argetcontour-2-iscontourclosed-end} überprüft, ob die
 gespeicherte Markierung mit den Anfangskoordinaten übereinstimmt und es sich somit um eine geschlossene Kontur
 handelt. Falls das Verfahren eine geschlossene Kontur gefunden hat, wird die Schleife von Zeile
 \ref{alg:argetcontour-2-contourloop-start}--\ref{alg:argetcontour-2-contourloop-end} abgebrochen. Falls es sich nicht
 um eine geschlossene Kontur handelt, wird die Kontur weiterverfolgt. In Zeile \ref{alg:argetcontour-2-inccoordnum}
 wird die Anzahl der Koordinaten in $\mathit{marker\_infoTWO}$ erhöht. Bevor die Kontur weiterverfolgt wird, wird in
 Zeile \ref{alg:argetcontour-2-ismaxchainreached-start}--\ref{alg:argetcontour-2-ismaxchainreached-end} überprüft, ob
 die Anzahl der Koordinaten $\mathit{AR\_CHAIN\_MAX} - 1 = 9999$ entspricht. Falls dem so ist, wird das Verfahren mit
 einem Fehlerwert in Zeile \ref{alg:argetcontour-2-error} abgebrochen. Dadurch wird sichergestellt, dass die Schleife
 in Zeile \ref{alg:argetcontour-2-contourloop-start}--\ref{alg:argetcontour-2-contourloop-end} auch bei einer nicht
 geschlossenen Kontur terminiert. Somit wird im schlechtesten Fall die Schleife $9997$-mal wiederholt, ohne das
 Verfahren mit einem Fehlerwert zu beenden. Die Laufzeitfunktion ist in \autoref{eq:analyse-argetcontour-contour},
 für $\mathit{AR\_CHAIN\_MAX} = n$, angegeben.
\begin{subequations}
\label{eq:analyse-argetconout-contour}
\begin{align}
\label{eq:analyse-argetconout-contour-1}
T_{worst}(n)& =
(c_{20}2 + c_{21}3 + c_{22}3 + c_{23})
\\
& \quad
+ (n-3)(c_{25}3 + c_{26}5 + c_{27}5 + c_{28}2 + c_{29}3)
\nonumber \\
& \quad
+ (n-3)(c_{30}8 + c_{31}35 + c_{32}14 + c_{33} + c_{35}21 + c_{36}7)
\nonumber \\
& \quad
+ (n-3)(c_{38} + c_{41}4 + c_{42}7 + c_{43}4 + c_{44}7 + c_{45}3)
+ c_{46}
\nonumber \\
& \quad
+ (n-3)(c_{48}4 + c_{49}3)
\nonumber \\
\label{eq:analyse-argetconout-contour-1}
T_{worst}(n)& =
+ (n-3)(c_{25}3 + c_{26}5 + c_{27}5 + c_{28}2 + c_{29}3)
\nonumber \\
& \quad
+ (n-3)(c_{30}8 + c_{31}35 + c_{32}14 + c_{33} + c_{35}21 + c_{36}7)
\nonumber \\
& \quad
+ (n-3)(c_{38} + c_{41}4 + c_{42}7 + c_{43}4 + c_{44}7 + c_{45}3)
\nonumber \\
& \quad
+ (n-3)(c_{48}4 + c_{49}3)
+ (c_{20}2 + c_{21}3 + c_{22}3 + c_{23} + c_{46})
\nonumber
\end{align}
\end{subequations}

Das Wachstum der Laufzeitfunktion ist $137n -401 = \Theta(n)$ für $c_{1} = 136$, $c_{2} = 137$ und $n_{0} = 401$.

In \autoref{alg:argetcontour-3} wird, wenn eine geschlossene Kontur vorliegt, die Koordinate mit dem größten Abstand
 zum Startpunkt der Region gesucht (Zeile
 \ref{alg:argetcontour-3-finddmax-start}--\ref{alg:argetcontour-3-finddmax-end}).
\begin{algorithm}[!ht]\small
\caption{\textproc{arGetContour} (Finde größten Abstand zu Punkt $(sx,sy)$)}
\label{alg:argetcontour-3}
\begin{algorithmic}[1]
	\algrestore{brk-argetcontour-contour}
	% dmax = 0;
	\State $\mathit{dmax} \gets 0$
	\Cost{$c_{53}$}{$1$}
	\label{alg:argetcontour-3-initdmax}
	% for(i=1;i<marker_infoTWO->coord_num;i++) {
	\For{$i \gets 1$ \textbf{to} $i < \mathit{marker\_infoTWO.coord\_num}$}
	\Cost{$c_{54}$}{$n - 2$}
	% \Cost{$c_{54}$}{$1$}
	\label{alg:argetcontour-3-finddmax-start}
	%     d = (marker_infoTWO->x_coord[i]-sx)*(marker_infoTWO->x_coord[i]-sx)
	%       + (marker_infoTWO->y_coord[i]-sy)*(marker_infoTWO->y_coord[i]-sy);
	%     if( d > dmax ) {
	%         dmax = d;
	%         v1 = i;
	%     }
	% }
		\State $a \gets (\mathit{marker\_infoTWO.x\_coord}\left[i\right] - sx)$
		\Cost{$c_{55}$}{$4(n-3)$}
		\label{alg:argetcontour-3-calcdistance-start}
		\State $b \gets (\mathit{marker\_infoTWO.y\_coord}\left[i\right] - sy)$
		\Cost{$c_{56}$}{$4(n-3)$}
		\State $d \gets (a \cdot a) + (b \cdot b)$
		\Cost{$c_{57}$}{$4(n-3)$}
		\label{alg:argetcontour-3-calcdistance-end}
		\If{$d > \mathit{dmax}$}
		\Cost{$c_{58}$}{$n-3$}
		\label{alg:argetcontour-3-isdmaxbigger}
			\State $\mathit{dmax} \gets d$
			\Cost{$c_{59}$}{$n-3$}
			\label{alg:argetcontour-3-savedmax}
			\State $\mathit{v1} \gets i$
			\Cost{$c_{60}$}{$n-3$}
		\EndIf
		\State $i \gets i + 1$
		\Cost{$c_{62}$}{$n-3$}
	\EndFor
	\label{alg:argetcontour-3-finddmax-end}
	% for(i=0;i<v1;i++) {
	%     arGetContour_wx[i] = marker_infoTWO->x_coord[i];
	%     arGetContour_wy[i] = marker_infoTWO->y_coord[i];
	% }
	\For{$i \gets 0$ \textbf{to} $i < \mathit{v1}$}
	\Cost{$c_{64}$}{$n-2$}
	\label{alg:argetcontour-3-dividev1-start}
		\State $\mathit{arGetContour\_wx}[i] \gets \mathit{marker\_infoTWO.x\_coord}\left[i\right]$
		\Cost{$c_{65}$}{$4(n-3)$}
		\State $\mathit{arGetContour\_wy}[i] \gets \mathit{marker\_infoTWO.y\_coord}\left[i\right]$
		\Cost{$c_{66}$}{$4(n-3)$}
		\State $i \gets i + 1$
		\Cost{$c_{67}$}{$(n-3)$}
	\EndFor
	\label{alg:argetcontour-3-dividev1-end}
	% for(i=v1;i<marker_infoTWO->coord_num;i++) {
	\For{$i \gets \mathit{v1}$ \textbf{to} $i < \mathit{marker\_infoTWO.coord\_num}$}
	\Cost{$c_{69}$}{$2$}
	\label{alg:argetcontour-3-dividecoordnum-start}
	%     marker_infoTWO->x_coord[i-v1] = marker_infoTWO->x_coord[i];
	%     marker_infoTWO->y_coord[i-v1] = marker_infoTWO->y_coord[i];
		\State $\mathit{marker\_infoTWO.x\_coord}\left[i - \mathit{v1}\right] \gets
		 \mathit{marker\_infoTWO.x\_coord}\left[i\right]$
		\Cost{$c_{70}$}{$6$}
		\State $\mathit{marker\_infoTWO.y\_coord}\left[i - \mathit{v1}\right] \gets
		 \mathit{marker\_infoTWO.y\_coord}\left[i\right]$
		\Cost{$c_{71}$}{$6$}
		\State $i \gets i + 1$
		\Cost{$c_{72}$}{$1$}
	% }
	\EndFor
	\label{alg:argetcontour-3-dividecoordnum-end}
	% for(i=0;i<v1;i++) {
	\For{$i \gets 0$ \textbf{to} $i < \mathit{v1}$}
	\Cost{$c_{74}$}{$(n-2)$}
	\label{alg:argetcontour-3-merge-start}
	%     marker_infoTWO->x_coord[i-v1+marker_infoTWO->coord_num] = arGetContour_wx[i];
	%     marker_infoTWO->y_coord[i-v1+marker_infoTWO->coord_num] = arGetContour_wy[i];
		\State $num \gets \mathit{marker\_infoTWO.coord\_num}$
		\Cost{$c_{75}$}{$2(n-3)$}
		\State $\mathit{marker\_infoTWO.x\_coord}\left[i - \mathit{v1} + num\right] \gets
		 \mathit{arGetContour\_wx}[i]$
		\Cost{$c_{76}$}{$6(n-3)$}
		\State $\mathit{marker\_infoTWO.y\_coord}\left[i - \mathit{v1} + num\right] \gets
		 \mathit{arGetContour\_wy}[i]$
		\Cost{$c_{77}$}{$6(n-3)$}
		\State $i \gets i + 1$
		\Cost{$c_{78}$}{$(n-3)$}
	% }
	\EndFor
	\label{alg:argetcontour-3-merge-end}
	% marker_infoTWO->x_coord[marker_infoTWO->coord_num] = marker_infoTWO->x_coord[0];
	% marker_infoTWO->y_coord[marker_infoTWO->coord_num] = marker_infoTWO->y_coord[0];
	% marker_infoTWO->coord_num++;
	% 
	% return 0;
	\State $num \gets \mathit{marker\_infoTWO.coord\_num}$
	\Cost{$c_{80}$}{$2$}
	\label{alg:argetcontour-3-savev1-start}
	\State $\mathit{marker\_infoTWO.x\_coord}[num] \gets \mathit{marker\_infoTWO.x\_coord}[0]$
	\Cost{$c_{81}$}{$5$}
	\State $\mathit{marker\_infoTWO.y\_coord}[num] \gets \mathit{marker\_infoTWO.y\_coord}[0]$
	\Cost{$c_{82}$}{$5$}
	\label{alg:argetcontour-3-savev1-end}
	\State $\mathit{marker\_infoTWO.coord\_num} \gets \mathit{marker\_infoTWO.coord\_num} + 1$
	\Cost{$c_{83}$}{$4$}
	\label{alg:argetcontour-3-inccoordnum}
	\State \textbf{return} $0$
	\Cost{$c_{84}$}{$1$}
\end{algorithmic}
\end{algorithm}

Dazu wird in Zeile \ref{alg:argetcontour-3-initdmax} die Variable $\mathit{dmax}$ initialisiert. Danach wird in Zeile
 \ref{alg:argetcontour-3-calcdistance-start}--\ref{alg:argetcontour-3-calcdistance-end} der Abstand zwischen dem
 aktuellen Punkt $i$ und der Koordinaten $\mathit{sx}$ und $\mathit{sy}$ berechnet. Falls der Abstandswert größer als
 $\mathit{dmax}$ ist (Zeile \ref{alg:argetcontour-3-isdmaxbigger}), wird der Abstandswert in $\mathit{dmax}$
 gespeichert (Zeile \ref{alg:argetcontour-3-savedmax}) und die Position $i$ in $\mathit{v1}$ hinterlegt. Nachdem die
 Schleife in Zeile \ref{alg:argetcontour-3-calcdistance-start}--\ref{alg:argetcontour-3-calcdistance-end} beendet ist,
 ist der größte Abstand in $\mathit{dmax}$ gespeichert und die Position der Koordinate in $\mathit{v1}$ gespeichert.
 Nun wird in Zeile \ref{alg:argetcontour-3-dividev1-start}--\ref{alg:argetcontour-3-dividev1-end} von der ersten
 Position bis zur Position $\mathit{v1}$ alle Koordinaten in einer temporären Liste gespeichert. In Zeile
 \ref{alg:argetcontour-3-dividecoordnum-start}--\ref{alg:argetcontour-3-dividecoordnum-end} werden alle Koordinaten ab
 Position $\mathit{v1}$ an den Anfang von $\mathit{marker\_infoTWO}$ verschoben. Die temporär gespeicherten Koordinaten
 werden in Zeile \ref{alg:argetcontour-3-merge-start}--\ref{alg:argetcontour-3-merge-end} an das Ende von
 $\mathit{marker\_infoTWO}$ angehängt. Zum Schluss wird in Zeile
 \ref{alg:argetcontour-3-savev1-start}--\ref{alg:argetcontour-3-savev1-end} die Koordinaten des Punktes mit dem größten
 Abstandswert an die letzte Stelle von marker\_infoTWO kopiert, sodass die Koordinaten am Anfang und am Ende
 gespeichert sind. Die Anzahl der gespeicherten Koordinaten wird in Zeile \ref{alg:argetcontour-3-inccoordnum} erhöht
 und die Konturverfolgung beendet. $\mathit{marker\_infoTWO.coord\_num}$ ist von $\mathit{AR\_CHAIN\_MAX}$ abhängig und
 kann maximal $\mathit{AR\_CHAIN\_MAX} - 3$ Einträge enthalten. Im schlechtesten Fall liegt der Punkt mit dem größten
 Abstand am Ende der Liste von $\mathit{marker\_infoTWO.coord\_num}$. Für diesen Fall sind die Kosten des Verfahrens
 in \autoref{alg:argetcontour-3} aufgelistet. Für die Kosten der Laufzeitfunktion in
 \autoref{eq:analyse-argetcontour-distance} ist $\mathit{AR\_CHAIN\_MAX} = n$.
\begin{subequations}
\label{eq:analyse-argetcontour-distance}
\begin{align}
\label{eq:analyse-argetcontour-distance-1}
T_{worst}(n)& =
c_{53}
 + c_{54}(n-2)
 + 4(c_{55} + c_{56} + c_{57})(n-3)
\\
& \quad
 + (c_{58} + c_{59} + c_{60} + c_{62})(n-3)
 + c_{64}(n-2)
 + 4(c_{65} + c_{66})(n-3)
\nonumber \\
& \quad
 + c_{67}(n-3)
 + (c_{69}2 + c_{70}6 + c_{71}6 + c_{72})
 + c_{74}(n-2)
\nonumber \\
& \quad
 + (c_{75}2 + c_{76}6 + c_{77}6 + c_{78})(n-3)
 + (c_{80}2 + c_{81}5 + c_{82}5 + c_{83}4 + c_{84})
\nonumber \\
\label{eq:analyse-argetcontour-distance-2}
T_{worst}(n)& =
n(c_{54} + c_{64} + c_{74} + c_{75}8 + c_{76}24 + c_{77}24 + c_{78}4
\\
& \quad \quad
 + c_{58} + c_{59} + c_{60} + c_{62} + c_{67})
\nonumber \\
& \quad
- 2(c_{54} + c_{64} + c_{74})
- (c_{75}24 + c_{76}72 + c_{77}72 + c_{78}12)
\nonumber \\
& \quad
- (c_{58}3 + c_{59}3 + c_{60}3 + c_{62}3 + c_{67}3)
+ 4(c_{55} + c_{56} + c_{57} + c_{65} + c_{66})
\nonumber \\
& \quad
+ (c_{53} + c_{69}2 + c_{70}6 + c_{71}6 + c_{72} + c_{80}2 + c_{81}5 + c_{82}5 + c_{83}4 + c_{84})
\nonumber
\end{align}
\end{subequations}

Für $c_{1} = 67$, $c_{2} = 68$ und $n_{0} = 163$ ist $68n - 163 = \Theta(n)$.

Die Laufzeitfunktion der Methode \textproc{arGetContour}, bestehend aus \autoref{alg:argetcontour-1},
 \autoref{alg:argetcontour-2} und \autoref{alg:argetcontour-3}, ist in \autoref{eq:analyse-argetcontour-all} angegeben.
 Die dominante Eingabegröße des Verfahrens ist $\mathit{AR\_CHAIN\_MAX} = n$ (Vgl. \autoref{alg:argetcontour-2},
 Zeile \ref{alg:argetcontour-2-contourloop-start}--\ref{alg:argetcontour-2-contourloop-end}). Die Variable
 $\mathit{xsize}$ kann durch die Breite des Bildsignals ersetzt werden und entspricht $\mathit{xsize} = \tfrac{640}{2}$.
\begin{align}
\label{eq:analyse-argetcontour-all}
T_{worst}(n)& =
8(320) - 8 + 137n - 401 + 68n - 163
\\
& =
205n - 3132
\nonumber
\end{align}

Die Wachstumsrate der Funktion aus \autoref{eq:analyse-argetcontour-all} ist im schlechtesten Fall
 $205n - 3132 = \Theta(n)$, für $c_{1} = 204$, $c_{2} = 205$ und $n_{0} = 3132$.
% subsubsection konturerzeugung (end)


\clearpage

% subsection fiducial_detection (end)

\clearpage

\subsection{Rectangle Fitting} % (fold)
\label{sec:rectangle_fitting}

In der Rectangle Fitting Phase wird überprüft, ob eine Kontur ein Rechteck ist. Dazu wird in Rectangle Fitting die
 Methode \textproc{checkSquare} verwendet (\autoref{alg:checksquare-1}--\autoref{alg:checksquare-7}). Doch zuerst
 müssen die Schritte zum Aufruf der Methode in \textproc{arDetectMarker2} erläutert werden
 (\autoref{alg:detectmarker2-3}).

\begin{algorithm}[ht]
\caption{\textproc{arDetectMarker2} (Rechtecküberprüfung)}
\label{alg:detectmarker2-3}
\begin{algorithmic}[1]
	\State $\mathit{marker\_num2} \gets 0$
	\For{$i \gets 0$ \textbf{to} $i < \mathit{label\_num}$}
	\label{alg:detectmarker2-3-loop-start}
		\State \ldots
		\State $\mathit{ret} \gets$ \textproc{arGetContour}$\left(
		\begin{aligned}
			& \mathit{limage}, \mathit{label\_ref}, i + 1,\\
			& \mathit{wclip}[i \cdot 4], \mathit{marker\_infoTWO}[\mathit{marker\_num2}]
		\end{aligned}\right)$
		\If{$\mathit{ret} < 0$}
		\label{alg:detectmarker2-3-retvalue}
			\State \textbf{continue}
			\label{alg:detectmarker2-3-continue}
		\EndIf
		\State $\mathit{ret} \gets$ \Call{checkSquare}{$\mathit{warea}[i],\mathit{marker\_infoTWO}[\mathit{marker\_num2}], \mathit{factor}$}
		\label{alg:detectmarker2-3-checksquare}
		\If{$\mathit{ret} < 0$}
			\State \textbf{continue}
			\label{alg:detectmarker2-3-continue2}
		\EndIf
		\State $\mathit{marker\_infoTWO}[marker\_num2].area \gets \mathit{warea}[i]$
		\label{alg:detectmarker2-3-save-start}
		\State $\mathit{marker\_infoTWO}[marker\_num2].pos[0] \gets \mathit{wpos}[i \cdot 2 + 0]$
		\State $\mathit{marker\_infoTWO}[marker\_num2].pos[1] \gets \mathit{wpos}[i \cdot 2 + 1]$
		\State $\mathit{marker\_num2} \gets \mathit{marker\_num2} + 1$
		\label{alg:detectmarker2-3-save-end}
		\If{$\mathit{marker\_num2} == \mathit{MAX\_IMAGE\_PATTERNS}$}
		\label{alg:detectmarker2-3-maxpatterns}
			\State \textbf{break}
		\EndIf
		\State $i \gets i + 1$
	\EndFor
	\label{alg:detectmarker2-3-loop-end}
	\algstore{brk-detectmarker2-process}
\end{algorithmic}
\end{algorithm}


Nachdem die Kontur in \textproc{arGetContour} ermittelt wurde, wird das Ergebnis in $\mathit{ret}$ gespeichtert und
 anschließend in Zeile \ref{alg:detectmarker2-3-retvalue} geprüft, ob \textproc{arGetContour} nicht mit einem
 Fehlerwert beendet wurde. Sonst wird in Zeile \ref{alg:detectmarker2-3-continue} die nächste Iteration der Schleife
 eingeleitet. Danach wird in Zeile \ref{alg:detectmarker2-3-checksquare} die Methode \textproc{checkSquare} aufgerufen.
 Auch hier wird der Rückgabewert überprüft, um im Fehlerfall einen neuen Durchlauf der Schleife in Zeile
 \ref{alg:detectmarker2-3-continue2} einzuleiten. Wenn \textproc{checkSquare} erfolgreich beendet wurde, wird in Zeile
 \ref{alg:detectmarker2-3-save-start}--\ref{alg:detectmarker2-3-save-end} der Flächeninhalt und die Position des
 Zentrums der Marke aktualisiert, und die Anzahl der gefundenen Marken wird erhöht. In Zeile
 \ref{alg:detectmarker2-3-maxpatterns} wird die Anzahl der gefundenen Marken mit einem festgelegten Wert verglichen, der
 die maximale Anzahl gleichzeitiger Marken festlegt. Wenn die Werte übereinstimmen, wird die Schleife von Zeile
 \ref{alg:detectmarker2-3-loop-start} bis Zeile \ref{alg:detectmarker2-3-loop-end} abgebrochen. Andernfalls wird die
 Laufvariable $i$ erhöht und ein neuer Schleifendurchgang begonnen.

Das Verfahren \textproc{checksquare} benötigt als Eingabeparameter die Fläche der Region ($\mathit{area}$), die
 Datenstruktur $\mathit{marker\_infoTwo}$ mit Informationen der Region und Kontur und einen Faktor ($\mathit{factor}$)
 zur Berechnung eines Distanzschwellwerts.

\begin{algorithm}[!ht]\small
\caption{\textproc{checkSquare} (Initialisierung)}
\label{alg:checksquare-1}
\begin{algorithmic}[1]
	\Require $\mathit{area}, \mathit{marker\_infoTWO}, \mathit{factor}$
	\State $\mathit{sx}, \mathit{sy}, d, \mathit{vertex}[10] \gets \infty$
	\Cost{$c_{1}$}{$4$}
	\State $\mathit{vnum}, \mathit{wv1}[10], \mathit{wvnum1}, \mathit{wv2}[10], \mathit{wvnum2} \gets \infty$
	\Cost{$c_{2}$}{$5$}
	\State $\mathit{v2}, \mathit{thresh}, i \gets \infty$
	\Cost{$c_{3}$}{$3$}
	\State $\mathit{dmax},\mathit{v1} \gets 0$
	\Cost{$c_{4}$}{$2$}
	\algstore{brk-checksquare-1-init}
\end{algorithmic}
\end{algorithm}


Die lokalen Variablen werden in \autoref{alg:checksquare-1} initialisiert und ihre Bedeutung bei der ersten Verwendung erläutert.

\begin{algorithm}[ht]
\caption{\textproc{checkSquare} (Fortsetzung)}
\label{alg:checksquare-2}
\begin{algorithmic}[1]
	\algrestore{brk-checksquare-1-init}
	\State $\mathit{sx} \gets \mathit{marker\_infoTWO \to x\_coord}[0]$
	\label{alg:checksquare-2-x}
	\State $\mathit{sy} \gets \mathit{marker\_infoTWO \to y\_coord}[0]$
	\label{alg:checksquare-2-y}
	\For{$i \gets 1$ \textbf{to} $\mathit{marker\_infoTWO \to coord\_num} - 1$}
	\label{alg:checksquare-2-loop-start}
		\State $a \gets \mathit{marker\_infoTWO \to x\_coord}[i] - \mathit{sx}$
		\State $b \gets \mathit{marker\_infoTWO \to y\_coord}[i] - \mathit{sy}$
		\State $d \gets \left(a \cdot a\right) + \left(b \cdot b\right)$
		\If{$d > \mathit{dmax}$}
		\label{alg:checksquare-2-isdbigger}
			\State $\mathit{dmax} \gets d$
			\State $\mathit{v1} \gets i$
		\EndIf
		\State $i \gets i + 1$
	\EndFor
	\label{alg:checksquare-2-loop-end}
	\algstore{brk-checksquare-2-distance}
\end{algorithmic}
\end{algorithm}


In \autoref{alg:checksquare-2} wird der größte Abstand von dem Punkt $(\mathit{sx},\mathit{sy})$ für alle anderen
 Konturpunkte berechnet. In Zeile \ref{alg:checksquare-2-x}--\ref{alg:checksquare-2-y} wird aus der Koordinatenliste
 von $\mathit{marker\_infoTWO}$ die $x$-Koordinate in $\mathit{sx}$ und die $y$-Koordinate in $\mathit{sy}$
 gespeichert. In \textproc{arGetContour} (\autoref{alg:argetcontour-3}) wurde an der ersten Position von
 $\mathit{marker\_infoTWO}$ eine Koordinate mit dem größten Abstand zum Anfang der Kontur gespeichert, die hier als
 Startpunkt der Rechteckerkennung dient. In der Schleife in Zeile
 \ref{alg:checksquare-2-loop-start}--\ref{alg:checksquare-2-loop-end} wird für alle Konturpunkte in
 $\mathit{marker\_infoTWO}$ der Abstand $d$ berechnet. Wenn der Abstand $d$ größer als $\mathit{dmax}$ ist (Zeile
 \ref{alg:checksquare-2-isdbigger}), wird $\mathit{dmax}$ mit dem neuen Abstandswert $d$ überschrieben und der Index
 $i$ in $\mathit{v1}$ notiert. Am Ende von \autoref{alg:checksquare-2} enthält $\mathit{v1}$ den Index der Koordinaten
 mit dem größten Abstand zum Punkt $(\mathit{sx},\mathit{sy})$ und ist der Index zu einem Eckpunkt der Kontur. Die
 Menge der Konturpixel in $\mathit{marker\_infoTWO}$ wird mit $\mathit{v1}$ in \autoref{alg:checksquare-3} geteilt und
 die Teilmengen einzeln untersucht.

\begin{algorithm}[!ht]\small
\caption{\textproc{checkSquare} (Eckpunkte bestimmen)}
\label{alg:checksquare-3}
\begin{algorithmic}[1]
	\algrestore{brk-checksquare-2-distance}
	\State $\mathit{thresh} \gets \left(\mathit{area}/0.75\right) \cdot 0.01 \cdot \mathit{factor}$
	\Cost{$c_{17}$}{$4$}
	\label{alg:checksquare-3-thresh}
	\State $\mathit{vnum} \gets 1$
	\Cost{$c_{18}$}{$1$}
	\State $\mathit{vertex}[0] \gets 0$
	\Cost{$c_{19}$}{$2$}
	\State $\mathit{wvnum1} \gets 0$
	\Cost{$c_{20}$}{$1$}
	\label{alg:checksquare-3-wvnum1}
	\State $\mathit{wvnum2} \gets 0$
	\Cost{$c_{21}$}{$1$}
	\label{alg:checksquare-3-wvnum2}
	\State $x \gets \mathit{marker\_infoTWO.x\_coord}$
	\Cost{$c_{22}$}{$2$}
	\State $y \gets \mathit{marker\_infoTWO.y\_coord}$
	\Cost{$c_{23}$}{$2$}
	\If{\Call{getVertex}{$x, y, 0, \mathit{v1}, \mathit{thresh}, \mathit{wv1}, \mathit{wvnum1}$} $< 0$}
	\Cost{$c_{24}$}{$\Theta(\frac{n}{2} \log \frac{n}{2})$}
	\label{alg:checksquare-3-getvertex1}
		\State \textbf{return}$(-1)$
	\EndIf
	\State $x \gets \mathit{marker\_infoTWO.x\_coord}$
	\Cost{$c_{27}$}{$2$}
	\State $y \gets \mathit{marker\_infoTWO.y\_coord}$
	\Cost{$c_{28}$}{$2$}
	\State $\mathit{num} \gets \mathit{marker\_infoTWO.coord\_num} - 1$
	\Cost{$c_{29}$}{$3$}
	\If{\Call{getVertex}{$x, y, \mathit{v1}, \mathit{num}, \mathit{thresh}, \mathit{wv2}, \mathit{wvnum2}$} $<$}
	\Cost{$c_{30}$}{$\Theta(\frac{n}{2} \log \frac{n}{2})$}
	\label{alg:checksquare-3-getvertex2}
		\State \textbf{return}$(-1)$
	\EndIf
	\algstore{brk-checksquare-3-getvertex}
\end{algorithmic}
\end{algorithm}


Bevor die Konturpixel untersucht werden, wird in Zeile \ref{alg:checksquare-3-thresh} der Distanzschwellwert
 $\mathit{thresh}$ mit Hilfe der Fläche der Region und $\mathit{factor}$ erstellt. Danach wird die Anzahl der Vektoren
 $\mathit{vnum}$ initialisiert und der Index zum Punkt $(\mathit{sx}, \mathit{sy})$ in der Liste der Vektoren
 ($\mathit{vertex}$) gespeichert. $wvnum1$ und $wvnum2$ speichern die Anzahl der gefundenen Eckpunkte und werden in
 Zeile \ref{alg:checksquare-3-wvnum1}--\ref{alg:checksquare-3-wvnum2} initialisiert. In Zeile
 \ref{alg:checksquare-3-getvertex1} wird die Methode \textproc{getVertex} aufgerufen, die einen Eckpunkt in der Menge
 der Konturpixel zwischen Position $0$ und $\mathit{v1}$ in $\mathit{marker\_infoTWO}$ finden soll. Danach wird
 \textproc{getVertex} in Zeile \ref{alg:checksquare-3-getvertex2} aufgerufen, um einen Eckpunkt in der Menge zwischen
 $\mathit{v1}$ und dem letzten Eintrag in $\mathit{marker\_infoTWO}$ zu finden. Die Methode \textproc{getVertex} wird
 am Ende des Abschnitts beschrieben.

Die Ergebnisse aus \autoref{alg:checksquare-3} können in drei Fälle kategorisiert werden. Entweder wurde

\begin{enumerate}
	\item ein Eckpunkt in jeder Teilmenge gefunden, \label{enum:checksquare-case1}
	\item in der ersten Teilmenge mehr als ein Eckpunkt gefunden oder \label{enum:checksquare-case2}
	\item in der zweiten Teilmenge mehr als ein Eckpunkt gefunden. \label{enum:checksquare-case3}
\end{enumerate}

Jeder dieser Fälle wird in \autoref{alg:checksquare-4}--\autoref{alg:checksquare-6} untersucht.

\paragraph{1. Fall:} % (fold)
\label{par:1_fall}

Das Verfahren \autoref{alg:checksquare-4} behandelt den Fall, dass die Methode \textproc{getVertex} jeweils einen
 Eckpunkt für jede Teilmenge gefunden hat. Somit sind in diesem Fall die Indizes der beiden Eckpunkte $\mathit{wv1}$
 und $\mathit{wv2}$ bekannt. Der Index $\mathit{v1}$ eines Eckpunktes wurde in \autoref{alg:checksquare-2} berechnet.
 Die Indizes werden in Zeile \ref{alg:checksquare-4-save-start}--\ref{alg:checksquare-4-save-end} in die Liste der
 Vektoren gespeichert.

\begin{algorithm}[!ht]\small
\caption{\textproc{checkSquare} (1. Fall)}
\label{alg:checksquare-4}
\begin{algorithmic}[1]
	\algrestore{brk-checksquare-3-getvertex}
	\If{$\mathit{wvnum1} = 1 \land \mathit{wvnum2} = 1$}
	\Cost{$c_{33}$}{$3$}
		\State $\mathit{vertex}[1] \gets \mathit{wv1}[0]$
		\Cost{$c_{34}$}{$3$}
		\label{alg:checksquare-4-save-start}
		\State $\mathit{vertex}[2] \gets \mathit{v1}$
		\Cost{$c_{35}$}{$2$}
		\State $\mathit{vertex}[3] \gets \mathit{wv2}[0]$
		\Cost{$c_{36}$}{$3$}
		\label{alg:checksquare-4-save-end}
	\algstore{brk-checksquare-4-case1}
\end{algorithmic}
\end{algorithm}


% paragraph  (end)

\paragraph{2. Fall} % (fold)
\label{par:2_fall}

In diesem Fall wurde in der ersten Teilmenge mehr als ein Eckpunkt gefunden und in der zweiten Teilmenge kein Eckpunkt.
 \autoref{alg:checksquare-5} versucht in diesem Fall die Eckpunkte zu korrigieren. Dazu wird ein neuer Index
 $\mathit{v2}$ in Zeile \ref{alg:checksquare-5-v2} generiert, indem der Index $\mathit{v1}$ verwendet wird. Danach wird
 die Anzahl der Eckpunkte in Zeile \ref{alg:checksquare-5-delwvnum1}--\ref{alg:checksquare-5-delwvnum2} gelöscht. In
 Zeile \ref{alg:checksquare-5-vertex1-start}--\ref{alg:checksquare-5-vertex1-end} wird mit \textproc{getVertex} die
 erste Teilmenge von $0$--$\mathit{v2}$ untersucht, und in Zeile
 \ref{alg:checksquare-5-vertex2-start}--\ref{alg:checksquare-5-vertex2-end} die zweite Teilmenge von
 $\mathit{v2}$--$\mathit{v1}$. Danach wird in Zeile \ref{alg:checksquare-5-hascorners} überprüft, ob ein Eckpunkt in
 jeder Teilmenge gefunden wurde. Falls dem so ist, werden die Indizes in Zeile
 \ref{alg:checksquare-5-save-start}--\ref{alg:checksquare-5-save-end} in die Liste der Vektoren geschrieben. Andernfalls
 handelt es sich bei der Kontur nicht um ein Rechteck und das Verfahren wird in Zeile \ref{alg:checksquare-5-error}
 abgebrochen.

\begin{algorithm}[ht]
\caption{\textproc{checkSquare} (Fortsetzung)}
\label{alg:checksquare-5}
\begin{algorithmic}[1]
	\algrestore{brk-checksquare-4-case1}
	\ElsIf{$\mathit{wvnum1} > 1 \land \mathit{wvnum2} == 0$}
		\State $\mathit{v2} \gets \mathit{v1} / 2$
		\label{alg:checksquare-5-v2}
		\State $\mathit{wvnum1} \gets 0$
		\label{alg:checksquare-5-delwvnum1}
		\State $\mathit{wvnum2} \gets 0$
		\label{alg:checksquare-5-delwvnum2}
		\State $x \gets \mathit{marker\_infoTWO \to x\_coord}$
		\State $y \gets \mathit{marker\_infoTWO \to y\_coord}$
		\If{\Call{getVertex}{$x, y, 0, \mathit{v2}, \mathit{thresh}, \mathit{wv1}, \mathit{wvnum1}$} $< 0$}
		\label{alg:checksquare-5-vertex1-start}
			\State \textbf{return}$(-1)$
		\EndIf
		\label{alg:checksquare-5-vertex1-end}
		\State $x \gets \mathit{marker\_infoTWO \to x\_coord}$
		\State $y \gets \mathit{marker\_infoTWO \to y\_coord}$
		\If{\Call{getVertex}{$x, y, \mathit{v2}, \mathit{v1}, \mathit{thresh}, \mathit{wv2}, \mathit{wvnum2}$} $< 0$}
		\label{alg:checksquare-5-vertex2-start}
			\State \textbf{return}$(-1)$
		\EndIf
		\label{alg:checksquare-5-vertex2-end}
		\If{$\mathit{wvnum1} == 1 \land \mathit{wvnum2} == 1$}
		\label{alg:checksquare-5-hascorners}
			\State $\mathit{vertex}[1] = \mathit{wv1}[0]$
			\label{alg:checksquare-5-save-start}
			\State $\mathit{vertex}[2] = \mathit{wv2}[0]$
			\State $\mathit{vertex}[3] = \mathit{v1}$
			\label{alg:checksquare-5-save-end}
		\Else
			\State \textbf{return}$(-1)$
			\label{alg:checksquare-5-error}
		\EndIf
	\algstore{brk-checksquare-5-case2}
\end{algorithmic}
\end{algorithm}


% paragraph 2_fall (end)

\paragraph{3. Fall} % (fold)
\label{par:3_fall}

Der dritte Fall bedeutet, dass in der ersten Teilmenge kein Eckpunkt und in der zweiten Teilmenge mehr als ein Eckpunkt
 gefunden wurde. Das Verfahren \autoref{alg:checksquare-6} ähnelt dem Verfahren \autoref{alg:checksquare-5} (2. Fall).
 Auch hier wird ein neuer Index $\mathit{v2}$ erstellt (Zeile \ref{alg:checksquare-6-v2}). In diesem Fall wird der
 Index $\mathit{v2}$ erstellt, indem der Index $\mathit{v1}$ und die Anzahl der gespeicherten Konturpunkte verwendet
 werden. Der Rest des Verfahrens entspricht dem Verfahren \autoref{alg:checksquare-5}. Lediglich der Aufruf von
 \textproc{getVertex} in Zeile \ref{alg:checksquare-6-vertex1} und Zeile \ref{alg:checksquare-6-vertex2} verwendet
 andere Teilmengen der Konturpunkte aus $\mathit{marker\_infoTWO}$.

\begin{algorithm}[ht]
\caption{\textproc{checkSquare} (Fortsetzung)}
\label{alg:checksquare-6}
\begin{algorithmic}[1]
	\algrestore{brk-checksquare-5-case2}
	\ElsIf{$\mathit{wvnum1} == 0 \land \mathit{wvnum2} > 1$}
		\State $\mathit{v2} \gets (\mathit{v1} + \mathit{marker\_infoTWO \to coord\_num} - 1) / 2$
		\label{alg:checksquare-6-v2}
		\State $\mathit{wvnum1} \gets 0$
		\State $\mathit{wvnum2} \gets 0$
		\State $x \gets \mathit{marker\_infoTWO \to x\_coord}$
		\State $y \gets \mathit{marker\_infoTWO \to y\_coord}$
		\If{\Call{getVertex}{$x, y, \mathit{v1}, \mathit{v2}, \mathit{thresh}, \mathit{wv1}, \mathit{wvnum1}$} $< 0$}
		\label{alg:checksquare-6-vertex1}
			\State \textbf{return}$(-1)$
		\EndIf
		\State $x \gets \mathit{marker\_infoTWO \to x\_coord}$
		\State $y \gets \mathit{marker\_infoTWO \to y\_coord}$
		\State $\mathit{num} \gets \mathit{marker\_infoTWO \to coord\_num} - 1$
		\If{\Call{getVertex}{$x, y, \mathit{v2}, \mathit{num}, \mathit{thresh}, \mathit{wv2}, \mathit{wvnum2}$} $< 0$}
		\label{alg:checksquare-6-vertex2}
			\State \textbf{return}$(-1)$
		\EndIf
		\If{$\mathit{wvnum1} == 1 \land \mathit{wvnum2} == 1$}
			\State $\mathit{vertex}[1] \gets \mathit{v1}$
			\State $\mathit{vertex}[2] \gets \mathit{wv1}[0]$
			\State $\mathit{vertex}[3] \gets \mathit{wv2}[0]$
		\Else
			\State \textbf{return}$(-1)$
		\EndIf
	\Else
		\State \textbf{return}$(-1)$
	\EndIf
	\algstore{brk-checksquare-6-case3}
\end{algorithmic}
\end{algorithm}


% paragraph 3_fall (end)

Wenn \autoref{alg:checksquare-4}--\autoref{alg:checksquare-6} erfolgreich waren, und Eckpunkte in die Liste der
 Vektoren schreiben konnten, wird mit \autoref{alg:checksquare-7} die Daten in $\mathit{marker\_infoTWO}$ gespeichert.

\begin{algorithm}[ht]
\caption{\textproc{checkSquare} (Fortsetzung)}
\label{alg:checksquare-7}
\begin{algorithmic}[1]
	\algrestore{brk-checksquare-6-case3}
	\State $\mathit{marker\_infoTWO \to vertex}[0] \gets \mathit{vertex}[0]$
	\label{alg:checksquare-7-savesxsy}
	\State $\mathit{marker\_infoTWO \to vertex}[1] \gets \mathit{vertex}[1]$
	\label{alg:checksquare-7-save1}
	\State $\mathit{marker\_infoTWO \to vertex}[2] \gets \mathit{vertex}[2]$
	\State $\mathit{marker\_infoTWO \to vertex}[3] \gets \mathit{vertex}[3]$
	\label{alg:checksquare-7-save3}
	\State $\mathit{marker\_infoTWO \to vertex}[4] \gets \mathit{marker\_infoTWO \to coord\_num} - 1$
	\label{alg:checksquare-7-savelast}
	\State \textbf{return} $0$
\end{algorithmic}
\end{algorithm}


In Zeile \ref{alg:checksquare-7-savesxsy} wird der Punkt $(\mathit{sx},\mathit{sy})$ als erster Index in die Liste der
 Vektoren von $\mathit{marker\_infoTWO}$ gespeichert. Je nachdem welcher der drei Fälle behandelt wurde, werden die
 Indizes $\mathit{v1}$, $\mathit{wv1}$ und $\mathit{wv2}$ in Zeile
 \ref{alg:checksquare-7-save1}--\ref{alg:checksquare-7-save3} gespeichert. Die Indizes sind danach im Uhrzeigersinn in
 $\mathit{marker\_infoTWO}$ hinterlegt\footcite[Vgl.][S.~44]{wagner07a}. In Zeile \ref{alg:checksquare-7-savelast} wird
 der letzte Index aus $\mathit{marker\_infoTWO}$ in die Liste der Vektoren geschrieben. Da der erste und der letzte
 Index identisch ist (Vgl. \autoref{alg:argetcontour-3}), sind die Eckpunkte so angeordnet, dass sie grafisch einfach
 dargestellt werden können, indem mit nur einer Linie, vom Startpunkt aus über jeden Eckpunkt zurück zum Startpunkt,
 ein Rahmen gezeichnet wird.

\textproc{getvertex} (\autoref{alg:getvertex-1}--\autoref{alg:getvertex-2}) sucht in der Menge der Koordinaten den
 Punkt, der von der Linie zwischen der Start- und Endkoordinate am weitesten entfernt liegt. Dazu benötigt
 \autoref{alg:getvertex-1} die Liste der $x$- und $y$-Koordinaten, den Startpunkt $\mathit{st}$ und Endpunkt
 $\mathit{ed}$, sowie den Schwellwert zur Überprüfung der Distanz. In der Liste $\mathit{vertex}$ werden die Indizes
 der Koordinaten gespeichert. Die Variable $\mathit{vnum}$ enthält die Anzahl der gefundenen Eckpunkte.

\begin{algorithm}[!ht]
\caption{\textproc{getVertex} (Abstand Punkt-Gerade)}
\label{alg:getvertex-1}
\begin{algorithmic}[1]
	\Require $\mathit{x\_coord}[], \mathit{y\_coord}[], \mathit{st}, \mathit{ed}, \mathit{thresh}, \mathit{vertex}[],
	 \mathit{vnum}$
	\State $d, \mathit{dmax}, i \gets \infty$
	\Cost{$c_{1}$}{$3$}
	\label{alg:getvertex-1-init-start}
	\State $\mathit{v1} \gets 0$
	\Cost{$c_{2}$}{$1$}
	\label{alg:getvertex-1-init-end}
	\State $a \gets \mathit{y\_coord}[ed] - \mathit{y\_coord}[st]$
	\Cost{$c_{3}$}{$4$}
	\label{alg:getvertex-1-line-start}
	\State $b \gets \mathit{x\_coord}[st] - \mathit{x\_coord}[ed]$
	\Cost{$c_{4}$}{$4$}
	\State $c \gets \left(\mathit{x\_coord}[ed] \cdot \mathit{y\_coord}[st]\right) - \left(\mathit{y\_coord}[ed] \cdot
	 \mathit{x\_coord}[st]\right)$
	\Cost{$c_{5}$}{$8$}
	\label{alg:getvertex-1-line-end}
	\State $\mathit{dmax} \gets 0$
	\Cost{$c_{6}$}{$1$}
	\For{$\mathit{i} \gets \mathit{st} + 1$ \textbf{to} $i < \mathit{ed}$}
	\Cost{$c_{7}$}{$n$}
	\label{alg:getvertex-1-loop-start}
		\State $d \gets \left(a \cdot \mathit{x\_coord}[i]\right) + \left(b \cdot \mathit{y\_coord}[i]\right) + c$
		\Cost{$c_{8}$}{$7(n - 1)$}
		\If{$d \cdot d > \mathit{dmax}$}
		\Cost{$c_{9}$}{$2(n - 1)$}
			\State $\mathit{dmax} \gets d \cdot d$
			\Cost{$c_{10}$}{$2(n - 1)$}
			\label{alg:getvertex-1-savedmax}
			\State $\mathit{v1} \gets i$
			\Cost{$c_{11}$}{$n - 1$}
		\EndIf
		\State $i \gets i + 1$
		\Cost{$c_{13}$}{$n - 1$}
	\EndFor
	\label{alg:getvertex-1-loop-end}
	\algstore{brk-getvertex1-distance}
\end{algorithmic}
\end{algorithm}


In Zeile \ref{alg:getvertex-1-init-start}--\ref{alg:getvertex-1-init-end} werden die lokalen Variablen initialisiert.
 Die Variable $a$, $b$ und $c$ werden in Zeile \ref{alg:getvertex-1-line-start}--\ref{alg:getvertex-1-line-end}
 initialisiert und werden zur Berechnung des Abstand zwischen Punkt $i$ und der Geraden $(\mathit{st},\mathit{ed})$
 verwendet. Dazu wird in der Schleife in Zeile \ref{alg:getvertex-1-loop-start}--\ref{alg:getvertex-1-loop-end} jede
 Koordinate $i$ in die Geradengleichung \autoref{eq:punktgeradenabstand} eingesetzt, um den Abstand des Punktes zur
 Geraden zu berechnen.

\begin{equation}
	\label{eq:punktgeradenabstand}
	d = \frac{\mathit{ax} + \mathit{by} + c}{\sqrt{(a^2 + b^2)}}
\end{equation}

Wenn der Abstand größer als $\mathit{dmax}$ ist, wird $d$ in Zeile \ref{alg:getvertex-1-savedmax} in $\mathit{dmax}$
 gespeichert und die Position $i$ in $\mathit{v1}$ hinterlegt. Somit ist am Ende von \autoref{alg:getvertex-1} die
 Position $i$ des Punktes mit dem größten Abstand zur Linie gespeichert.

In \autoref{alg:getvertex-2} wird in Zeile \ref{alg:getvertex-2-isdmaxgreater} der Abstand mit dem Distanzschwellwert
 verglichen. Nur wenn der Abstand größer als der Distanzschwellwert ist, wird das Verfahren fortgesetzt. Ansonsten ist
 das Verfahren in Zeile \ref{alg:getvertex-2-done} beendet.

\begin{algorithm}[!ht]
\caption{\textproc{getVertex} (Fortsetzung)}
\label{alg:getvertex-2}
\begin{algorithmic}[1]
	\algrestore{brk-getvertex1-distance}
	\If{$\mathit{dmax}/\left((a \cdot a) + (b \cdot b)\right) > \mathit{thresh}$}
	\Cost{$c_{15}$}{$5$}
	\label{alg:getvertex-2-isdmaxgreater}
			\If{\Call{getVertex}{$\mathit{x\_coord}, \mathit{y\_coord}, \mathit{st}, \mathit{v1}, \mathit{thresh},
			 \mathit{vertex}, \mathit{vnum}$} $< 0$}
			\Cost{$c_{16}$}{$1$}
			\label{alg:getvertex-2-recursiv1}
				\State \textbf{return}$(-1)$
			\EndIf
			\If{$\mathit{vnum} > 5$}
			\Cost{$c_{19}$}{$1$}
			\label{alg:getvertex-2-isvnumgreater}
				\State \textbf{return}$(-1)$
				\label{alg:getvertex-2-error}
			\EndIf
			\State $\mathit{vertex}[\mathit{vnum}] \gets \mathit{v1}$
			\Cost{$c_{22}$}{$2$}
			\State $\mathit{vnum} \gets \mathit{vnum} + 1$
			\Cost{$c_{23}$}{$1$}
			\If{\Call{getVertex}{$\mathit{x\_coord}, \mathit{y\_coord}, \mathit{v1}, \mathit{ed}, \mathit{thresh},
			 \mathit{vertex}, \mathit{vnum}$} $< 0$}
			\Cost{$c_{24}$}{$1$}
			\label{alg:getvertex-2-recursiv2-start}
				\State \textbf{return}$(-1)$
			\EndIf
			\label{alg:getvertex-2-recursiv2-end}
	\EndIf
	\State \textbf{return} $0$
	\Cost{$c_{28}$}{$1$}
	\label{alg:getvertex-2-done}
\end{algorithmic}
\end{algorithm}


In Zeile \ref{alg:getvertex-2-recursiv1} wird die Methode \textproc{getVertex} rekursiv aufgerufen, um die Teilmenge
 von der Startposition $\mathit{st}$ bis zur Position $\mathit{v1}$ zu untersuchen. Da \textproc{getVertex} pro
 Teilmenge nur den größten Abstand als Eckpunkt betrachet, kann durch den rekursiven Aufruf für eine kleinere Menge von
 Koordinaten ein weitere Eckpunkt gefunden werden, solange der Abstand größer als der Distanzschwellert ist.

In Zeile \ref{alg:getvertex-2-isvnumgreater} wird überprüft, ob nicht mehr als fünf Eckpunkte gefunden wurde. Falls
 mehr als fünf Eckpunkte gefunden wurden, wird das Verfahren in Zeile \ref{alg:getvertex-2-error} mit einem Fehlerwert
 beendet. Falls jedoch weniger als fünf Eckpunkte gefunden wurden, wird der Index $\mathit{v1}$ in der Liste
 $\mathit{vertex}$ gespeichert und die Anzahl der gefundenen Eckpunkte erhöht. Danach wird \textproc{getVertex} für die
 zweite Teilmenge, von der Position $\mathit{v1}$ bis zum Endpunkt $\mathit{ed}$, in Zeile
 \ref{alg:getvertex-2-recursiv2-start}--\ref{alg:getvertex-2-recursiv2-end} rekursiv aufgerufen.

Nachdem \textproc{arGetCountour} abgeschlossen ist, wird in \textproc{arDetectMarker2} die Marke mit der größten Fläche
 gesucht. Dazu wird in \autoref{alg:detectmarker2-4} in der Schleife von Zeile \ref{alg:detectmarker2-4-loop1-start}
 bis Zeile \ref{alg:detectmarker2-4-loop1-end} jede Marke mit allen anderen Marken in der Schleife von Zeile
 \ref{alg:detectmarker2-4-loop2-start}--\ref{alg:detectmarker2-4-loop2-end} verglichen.

\begin{algorithm}[!ht]\small
\caption{\textproc{arDetectMarker2} (Größte Marke)}
\label{alg:detectmarker2-4}
\begin{algorithmic}[1]
	\algrestore{brk-detectmarker2-process}
	\For{$i \gets 0$ \textbf{to} $i < \mathit{marker\_num2}$}
	\label{alg:detectmarker2-4-loop1-start}
		\For{$j \gets i + 1$ \textbf{to} $j < \mathit{marker\_num2}$}
		\label{alg:detectmarker2-4-loop2-start}
			\State $a \gets \mathit{marker\_infoTWO}[i].pos[0] - \mathit{marker\_infoTWO}[j].pos[0]$
			\State $b \gets \mathit{marker\_infoTWO}[i].pos[1] - \mathit{marker\_infoTWO}[j].pos[1]$
			\State $d \gets \left(a \cdot a\right) + \left(b \cdot b\right)$
			\label{alg:detectmarker2-4-length}
			\If{$\mathit{marker\_infoTWO}[i].area > \mathit{marker\_infoTWO}[j].area$}
				\If{$d < \mathit{marker\_infoTWO}[i].area / 4$}
				\label{alg:detectmarker2-4-di}
					\State $\mathit{marker\_infoTWO}[j].area \gets 0$
				\EndIf
			\Else
				\If{$d < \mathit{marker\_infoTWO}[j].area / 4$}
				\label{alg:detectmarker2-4-dj}
					\State $\mathit{marker\_infoTWO}[i].area \gets 0$
				\EndIf
			\EndIf
		\State $j \gets j + 1$
		\EndFor
		\label{alg:detectmarker2-4-loop2-end}
	\State $i \gets i + 1$
	\EndFor
	\label{alg:detectmarker2-4-loop1-end}
	\algstore{brk-detectmarker2-sort}
\end{algorithmic}
\end{algorithm}


In Zeile \ref{alg:detectmarker2-4-length} wird die quadratische Länge des Abstands zwischen dem Mittelpunkt der beiden
 Marken berechnet. Anschließend wird die Anzahl der Konturpixel in $\mathit{area}$ der Marke $i$ und $j$ verglichen.
 Wenn $i$ größer als $j$ ist, wird in Zeile \ref{alg:detectmarker2-4-di} der Abstand $d$ mit $\frac{\mathit{area}}{4}$
 von Marke $i$ verglichen. Wenn der Abstand kleiner ist, wird die Anzahl der Konturpixel der Marke $j$ auf $0$ gesetzt.
 Falls jedoch die Anzahl der Konturpixel von Marke $j$ größer sein sollte als die von $i$, wird in Zeile
 \ref{alg:detectmarker2-4-dj} der Abstand $d$ mit $\frac{\mathit{area}}{4}$ von Marke $j$ verglichen. Ist der Abstand
 kleiner, wird die Anzahl von Marke $i$ auf $0$ gesetzt.

In \autoref{alg:detectmarker2-5} werden nun alle Marken gelöscht, deren Variable $\mathit{area}$ in
 \autoref{alg:detectmarker2-4} auf $0$ gesetzt worden sind. Dazu werden alle Marken in Zeile
 \ref{alg:detectmarker2-5-loop-start}--\ref{alg:detectmarker2-5-loop-end} untersucht. Wenn die Überprüfung in Zeile
 \ref{alg:detectmarker2-5-shoulddelete} eine zu löschende Marke findet, wird in der Schleife von Zeile
 \ref{alg:detectmarker2-5-move-start} bis Zeile \ref{alg:detectmarker2-5-move-end} alle nachfolgende Marken um eine
 Position verschoben. Danach wird in Zeile \ref{alg:detectmarker2-5-decrease} die Anzahl der Marken verringert.

\begin{algorithm}[!ht]\small
\caption{\textproc{arDetectMarker2} (Lösche Marken)}
\label{alg:detectmarker2-5}
\begin{algorithmic}[1]
	\algrestore{brk-detectmarker2-sort}
	\For{$i \gets 0$ \textbf{to} $\mathit{marker\_num2}$}
	\label{alg:detectmarker2-5-loop-start}
		\If{$\mathit{marker\_infoTWO}[i].area = 0$}
		\label{alg:detectmarker2-5-shoulddelete}
			\For{$j \gets i + 1$ \textbf{to} $\mathit{marker\_num2}$}
			\label{alg:detectmarker2-5-move-start}
				\State $\mathit{marker\_infoTWO}[j - 1] \gets \mathit{marker\_infoTWO}[j]$
				\State $j \gets j + 1$
			\EndFor
			\label{alg:detectmarker2-5-move-end}
			\State $\mathit{marker\_num2} \gets \mathit{marker\_num2} - 1$
			\label{alg:detectmarker2-5-decrease}
		\EndIf
		\State $i \gets i + 1$
	\EndFor
	\label{alg:detectmarker2-5-loop-end}
	\algstore{brk-detectmarker2-decrease}
\end{algorithmic}
\end{algorithm}


In \autoref{sub:regionenmarkierung} wurde zur Optimierung des Verfahrens nur ein Teil des Bildsignals analysiert.
 Dadurch sind Daten entstanden, die nicht mit den tatsächlichen Daten im Bildsignal übereinstimmen.
 \autoref{alg:detectmarker2-6} sorgt dafür, dass diese Daten wieder aufbereitet werden.

\begin{algorithm}[ht]
\caption{\textproc{arDetectMarker2} (Koordinaten aufbereiten)}
\label{alg:detectmarker2-6}
\begin{algorithmic}[1]
	\algrestore{brk-detectmarker2-decrease}
	\State $\mathit{pm} \gets \mathit{marker\_infoTWO}[0]$
	\label{alg:detectmarker2-6-address}
	\For{$i \gets 0$ \textbf{to} $\mathit{marker\_num2}$}
	\label{alg:detectmarker2-6-loop-start}
		\State $\mathit{pm.area} \gets \mathit{pm.area} \cdot 4$
		\label{alg:detectmarker2-6-area}
		\State $\mathit{pm.pos}[0] \gets \mathit{pm.pos}[0] \cdot 2$
		\State $\mathit{pm.pos}[1] \gets \mathit{pm.pos}[1] \cdot 2$
		\label{alg:detectmarker2-6-pos}
		\For{$j \gets 0$ \textbf{to} $\mathit{pm.coord\_num}$}
		\label{alg:detectmarker2-6-coord-start}
			\State $\mathit{pm.x\_coord}[j] \gets \mathit{pm.x\_coord}[j] \cdot 2$
			\State $\mathit{pm.y\_coord}[j] \gets \mathit{pm.y\_coord}[j] \cdot 2$
			\State $j \gets j + 1$
		\EndFor
		\label{alg:detectmarker2-6-coord-end}
		\State Inkrementiere $\mathit{pm}$
		\label{alg:detectmarker2-6-incpm}
		\State $i \gets i + 1$
	\EndFor
	\label{alg:detectmarker2-6-loop-end}
	\State $\mathit{marker\_num} \gets \mathit{marker\_num2}$
	\State \textbf{return} $\mathit{marker\_infoTWO}[0]$
\end{algorithmic}
\end{algorithm}


In Zeile \ref{alg:detectmarker2-6-address} wird die Adresse der ersten Speicherstelle der Markeninformationen in der
 Variable $\mathit{pm}$ hinterlegt. In der Schleife in Zeile
 \ref{alg:detectmarker2-6-loop-start}--\ref{alg:detectmarker2-6-loop-end} werden nun alle verbliebenen Marken
 aufbereitet. Dazu wird in Zeile \ref{alg:detectmarker2-6-area} bis Zeile \ref{alg:detectmarker2-6-pos} die Anzahl der
 Konturpixel erhöht und die Koordinaten des Zentrums der Marke korrigiert. Danach wird in der Schleife von Zeile
 \ref{alg:detectmarker2-6-coord-start}--\ref{alg:detectmarker2-6-coord-end} die Koordinaten aller Konturpixel
 korrigiert. Anschließend wird in Zeile \ref{alg:detectmarker2-6-incpm} die Adresse inkrementiert und das Verfahren mit
 der nächsten Marke wiederholt. Abschließend wird die Anzahl der Marken in $\mathit{marker\_num}$ gespeichert und die
 Speicheradresse der ersten Markeninformation an \textproc{arDetectMarker} (\autoref{alg:detectmarker},
 S.~\pageref{alg:detectmarker}) zurückgegeben.

% subsection rectangle_fitting (end)

% section artoolkitplus (end)

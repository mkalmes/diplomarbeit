\subsubsection{Regionenmarkierung} % (fold)
\label{sub:regionenmarkierung}

Die Methode \textproc{arLabeling} (\autoref{alg:arlabeling-init1}--\autoref{alg:arlabeling-calcregiondata}) wird zur
 Markierung der Regionen in einem Bildsignal $I$ verwendet. Wie in \autoref{sub:fiducial_detection} beschrieben,
 verzichet ARToolKitPlus beim Aufruf des Verfahrens auf ein Binärbild. Stattdessen wird ein Bildsignal $I$ während der
 Verarbeitung durch eine Schwellwertanalyse untersucht. \textproc{arLabeling} kann in drei Abschnitte unterteilt werden
 (Vgl. \autoref{alg:arlabeling-overview}):
\begin{enumerate}
	\item Initialisierung der Variablen und des Speichers, \label{label-init}
	\item Regionenmarkierung und Auflösen von Kollisionen und \label{label-region}
	\item Aufbereiten der Regionenmarkierung zur Speicherung. \label{label-cleaning}
\end{enumerate}
\begin{algorithm}[!ht]
\caption{\textproc{arLabeling} (Übersicht)}
\label{alg:arlabeling-overview}
\begin{algorithmic}[1]
	\Require $I,\mathit{thresh},\mathit{label\_num},\mathit{area},\mathit{pos},\mathit{clip},\mathit{label\_ref}$

	\State Initialisieren der Variablen und des Speichers
	\State Regionenmarkierung
	\State Regionenmarkierung aufbereiten und speichern

\end{algorithmic}
\end{algorithm}


\autoref{label-init} ist in \autoref{alg:arlabeling-init1} aufgeführt.
\begin{algorithm}[!hb]\small
\caption{\textproc{arLabeling} (Initialisierung)}
\label{alg:arlabeling-init1}
\begin{algorithmic}[1]
	\Require $I,\mathit{thresh},\mathit{label\_num},\mathit{area},\mathit{pos},\mathit{clip},\mathit{label\_ref}$

	\State $\mathit{pnt}, \mathit{pnt1}, \mathit{pnt2} \gets \infty$
	\label{alg:arlabeling-init1-local-start}
	\State $\mathit{wk}, \mathit{wk\_max}, m, n, i, j, k, \mathit{lxsize}, \mathit{lysize}, \mathit{poff} \gets \infty$
	\State $\mathit{l\_image}, \mathit{work}, \mathit{work2}, \mathit{wlabel\_num}, \mathit{warea}, \mathit{wclip},
	 \mathit{wpos} \gets \infty$
	\label{alg:arlabeling-init1-local-end}

	\State $\mathit{l\_image} \gets \mathit{l\_imageL}[0]$
	\label{alg:arlabeling-init1-address-start}
	\State $\mathit{work} \gets \mathit{workL}[0]$
	\State $\mathit{work2} \gets \mathit{work2L}[0]$
	\State $\mathit{wlabel\_num} \gets \mathit{wlabel\_numL}$
	\State $\mathit{warea} \gets \mathit{wareaL}[0]$
	\State $\mathit{wclip} \gets \mathit{wclipL}[0]$
	\State $\mathit{wpos} \gets \mathit{wposL}[0]$
	\label{alg:arlabeling-init1-address-end}

	\State $\mathit{thresh} \gets \mathit{thresh} \cdot 3$
	\label{alg:arlabeling-init1-threshold}
	\State $\mathit{lxsize} \gets \tfrac{arImXsize}{2}$
	\label{alg:arlabeling-init1-size-start}
	\State $\mathit{lysize} \gets \tfrac{arImYsize}{2}$
	\label{alg:arlabeling-init1-size-end}

	\algstore{brk-arkabelinginit}
\end{algorithmic}
\end{algorithm}

In Zeile \ref{alg:arlabeling-init1-local-start}--\ref{alg:arlabeling-init1-local-end} werden die lokalen Variablen
 deklariert, deren Bedeutung bei ihrem ersten Auftreten erklärt werden. Speicheradressen werden in Zeile
 \ref{alg:arlabeling-init1-address-start}--\ref{alg:arlabeling-init1-address-end} initialisiert. Der
 Schwellwertparameter wird verdreifacht und lokal gespeichert (Zeile \ref{alg:arlabeling-init1-threshold}). Die
 Schwellwertanalyse in \autoref{alg:arlabeling-regionlabeling} benutzt einen dreifachen Wert als Optimierung
 (Vgl. S.~\pageref{sub:arlabel-threshold}). Zum Schluss wird in Zeile
 \ref{alg:arlabeling-init1-size-start}--\ref{alg:arlabeling-init1-size-end} die Bildbreite durch die Konstante
 $\mathit{arImXsize}$ und die Bildhöhe durch die Konstante $\mathit{arImYsize}$ festgelegt. Das halbieren der Bildhöhe
 und -breite ist eine Optimierung (Vgl. \autoref{sec:vorgehen}). Das initialisieren der Variablen erfolgt in konstanter
 Laufzeit $T(n) = 39$.

Die Initialisierung von \textproc{arLabeling} wird in \autoref{alg:arlabeling-init2} fortgesetzt.
\begin{algorithm}[!ht]
\caption{\textproc{arLabeling} (Fortsetzung der Initialisierung)}
\label{alg:arlabeling-init2}
\begin{algorithmic}[1]
	\algrestore{brk-arkabelinginit}

	\State $\mathit{pnt1} \gets \mathit{l\_image}[0]$
	\Cost{$c_{14}$}{$2$}
	\label{alg:arlabeling-init2-address1-start}
	\State $\mathit{pnt2} \gets \mathit{l\_image}[(\mathit{lysize} - 1) \cdot \mathit{lxsize}]$
	\Cost{$c_{15}$}{$4$}
	\label{alg:arlabeling-init2-address1-end}
	\For{$i \gets 1$ \textbf{to} $i < \mathit{lxsize}$}
	\Cost{$c_{16}$}{$(\mathit{lxsize} - 1) + 1$}
	\label{alg:arlabeling-init2-loop1-start}
		\State $\mathit{pnt1} \gets 0$
		\Cost{$c_{17}$}{$(\mathit{lxsize} - 1 )$}
		\State $\mathit{pnt2} \gets 0$
		\Cost{$c_{18}$}{$(\mathit{lxsize} - 1 )$}
		\State Inkrementiere ${pnt1}$
		\Cost{$c_{19}$}{$(\mathit{lxsize} - 1 )$}
		\State Inkrementiere ${pnt2}$
		\Cost{$c_{20}$}{$(\mathit{lxsize} - 1 )$}
		\State $i \gets i + 1$
		\Cost{$c_{21}$}{$(\mathit{lxsize} - 1 )$}
		\label{alg:arlabeling-init2-inc-1}
	\EndFor
	\label{alg:arlabeling-init2-loop1-end}

	\State $\mathit{pnt1} \gets \mathit{l\_image}[0]$
	\Cost{$c_{23}$}{$2$}
	\label{alg:arlabeling-init2-address2-start}
	\State $\mathit{pnt2} \gets \mathit{l\_image}[\mathit{lxsize} - 1]$
	\Cost{$c_{24}$}{$3$}
	\label{alg:arlabeling-init2-address2-end}
	\For{$i \gets 1$ \textbf{to} $i < \mathit{lysize}$}
	\Cost{$c_{25}$}{$(\mathit{lysize} - 1) + 1$}
	\label{alg:arlabeling-init2-loop2-start}
		\State $\mathit{pnt1} \gets 0$
		\Cost{$c_{26}$}{$\mathit{lysize} - 1$}
		\label{alg:arlabeling-init2-clearfirstrow}
		\State $\mathit{pnt2} \gets 0$
		\Cost{$c_{27}$}{$\mathit{lysize} - 1$}
		\label{alg:arlabeling-init2-clearlastrow}
		\State $\mathit{pnt1} \gets \mathit{pnt1} + \mathit{lxsize}$
		\Cost{$c_{28}$}{$(\mathit{lysize} - 1) 2$}
		\State $\mathit{pnt2} \gets \mathit{pnt2} + \mathit{lxsize}$
		\Cost{$c_{29}$}{$(\mathit{lysize} - 1) 2$}
		\State $i \gets i + 1$
		\Cost{$c_{30}$}{$\mathit{lysize} - 1$}
	\EndFor
	\label{alg:arlabeling-init2-loop2-end}

	\State $\mathit{wk\_max} \gets 0$
	\Cost{$c_{32}$}{$1$}
	\label{alg:arlabeling-init2-label}
	\State $\mathit{pnt2} \gets \mathit{l\_image}[\mathit{lxsize} + 1]$
	\Cost{$c_{33}$}{$3$}
	\State $\mathit{pnt} \gets I[\left(\mathit{arImXsize} \cdot 2 + 2 \right) \cdot \mathit{pixelSize}]$
	\Cost{$c_{34}$}{$5$}
	\State $\mathit{poff} \gets \mathit{pixelSize} \cdot 2$
	\Cost{$c_{35}$}{$2$}
\end{algorithmic}
\end{algorithm}

$\mathit{pnt1}$ wird in Zeile \ref{alg:arlabeling-init2-address1-start} auf die erste Speicherstelle der ersten Zeile
 des Regionenbildes gesetzt. $\mathit{pnt2}$ erhält in Zeile \ref{alg:arlabeling-init2-address1-end} die erste
 Speicherstelle der letzten Zeile des Regionenbildes.
% In der Schleife in Zeile \ref{alg:arlabeling-init2-loop1-start}--\ref{alg:arlabeling-init2-loop1-end} wird
%  über die Breite des Regionenbildes iteriert, um die erste und die letzte Zeile des Regionenbildes zu löschen.
In der Schleife in Zeile \ref{alg:arlabeling-init2-loop1-start}--\ref{alg:arlabeling-init2-loop1-end} wird die erste
 und die letzte Zeile des Regionenbildes gelöscht. Dazu wird an den Adressen von
 $\mathit{pnt1}$ und $\mathit{pnt2}$ der Wert $0$ gespeichert. Danach werden die Adressen von $\mathit{pnt1}$ und
 $\mathit{pnt2}$ inkrementiert. In Zeile \ref{alg:arlabeling-init2-inc-1} wird die Laufvariable $i$ inkrementiert. In
 Zeile \ref{alg:arlabeling-init2-address2-start}--\ref{alg:arlabeling-init2-address2-end} werden die Adressen
 von $\mathit{pnt1}$ und $\mathit{pnt2}$ erneut festgelegt. $\mathit{pnt1}$ wird die erste Speicherstelle des
 Regionenbildes zugewiesen. In $\mathit{pnt2}$ wird die Adresse der ersten Speicherstelle der letzten Zeile hinterlegt.
 In der Schleife von Zeile \ref{alg:arlabeling-init2-loop2-start}--\ref{alg:arlabeling-init2-loop2-end} wird die erste
 und letzte Spalte des Regionenbildes gelöscht, indem der Wert $0$ an die Speicherstelle von $\mathit{pnt1}$ und
 $\mathit{pnt2}$ geschrieben wird
 (Zeile \ref{alg:arlabeling-init2-clearfirstrow}--\ref{alg:arlabeling-init2-clearlastrow}). Im Anschluss daran werden
 die Adressen $\mathit{pnt1}$ und $\mathit{pnt2}$ auf die nächste Position der Bildspalte inkrementiert, indem die
 Breite $\mathit{lxsize}$ auf die Adressen addiert wird. Danach wird $i$ inkrementiert. In Zeile
 \ref{alg:arlabeling-init2-label} wird die Markierungsvariable $\mathit{wk\_max}$ mit dem Wert $0$ initialisiert.
 Danach wird die Adresse der Startposition des Regionenbildes in $\mathit{pnt2}$ gespeichert. Dabei ist zu beachten,
 dass die Adresse auf den zweiten \gls{pixel} der zweiten Zeile verweist. $\mathit{pnt}$ wird daraufhin die Adresse des
 Bildsignals $I$ zugewiesen. Der Adresse von $I$ ist der vierte \gls{pixel} der zweiten Zeile. Die Variable
 $\mathit{poff}$ ist der Adressabstand der \gls{pixel} in $I$ und dient der Adressierung des nächsten \glspl{pixel}.
 $\mathit{poff}$ wird in der letzten Zeile von \autoref{alg:arlabeling-init2} gesetzt. Die Kosten des Algorithmus sind
 in \autoref{alg:arlabeling-init2} angegeben. Durch Umformung der Kosten von \autoref{eq:analyse-arlabeling-init-1}
 erhalten wir die Laufzeitfunktion \autoref{eq:analyse-arlabeling-init-2}.
\begin{subequations}
\label{eq:analyse-arlabeling-init}
\begin{align}
\label{eq:analyse-arlabeling-init-1}
T(\mathit{lxsize},\mathit{lysize})& = c_{16}\bigl(\mathit{lxsize}\bigr) + (c_{17} + c_{18} + c_{19} + c_{20} + c_{21})
(\mathit{lxsize} - 1) \\
& \quad  + c_{25}\bigl(\mathit{lysize}\bigr) + (c_{26} + c_{27} +  c_{30})(\mathit{lysize} - 1) \nonumber \\
& \quad + (c_{28} + c_{29})(2\mathit{lysize} - 2) + c_{32} + (c_{14} + c_{23} + c_{35})2 \nonumber \\
& \quad + (c_{33} + c_{24})3 + c_{15}4 + c_{34}5 \nonumber \\
\label{eq:analyse-arlabeling-init-2}
T(\mathit{lxsize},\mathit{lysize})& = (c_{16} + c_{17} + c_{18} + c_{19} + c_{20} + c_{21})\mathit{lxsize} \\
& \quad + (c_{25} + c_{26} + c_{27} + c_{30})\mathit{lysize} \nonumber \\
& \quad + (c_{28} + c_{29})(2\mathit{lysize}) - (c_{28} + c_{29})2 \nonumber \\
& \quad - (c_{17} + c_{18} + c_{19} + c_{20} + c_{21})  - (c_{26} + c_{27} + c_{28} + c_{29} + c_{30})  \nonumber \\
& \quad + c_{32} + (c_{14} + c_{23} + c_{35})2 + (c_{33} + c_{24})3 + c_{15}4 + c_{34}5 \nonumber
\end{align}
\end{subequations}

Das asymptotische Wachstum für $T(\mathit{lxsize}) = 6\mathit{lxsize} + 1$ ist, für die Werte $c_{1} = 6$,
 $c_{2} = 7$ und $\mathit{lxsize}_{0} = 1$, $6\mathit{lxsize} + 1 = \Theta(lxsize)$. Für
 $T(\mathit{lysize}) = 8\mathit{lysize} + 9$, mit den Werten $c_{1} = 7$, $c_{2} = 9$ und $\mathit{lysize}_{0} = 9$,
 ist das Wachstum $8\mathit{lysize} + 9 = \Theta(lysize)$.

\autoref{label-region} von \autoref{alg:arlabeling-overview} ist für die Regionenmarkierung und das Auflösen von
 Kollisionen verantwortlich. Eine Übersich des Verfahrens ist in \autoref{alg:arlabeling-regionlabeling} dargestellt.
\begin{algorithm}[!ht]\small
\caption{\textproc{arLabeling} (Regionenmarkierung)}
\label{alg:arlabeling-regionlabeling}
\begin{algorithmic}[1]
	\For{$j \gets 1$ \textbf{to} $j < \mathit{lysize} - 1$}
	\Cost{$c_{1}$}{$\mathit{lysize} - 1$}
	\label{alg:arlabeling-regionlabeling-loop1-start}

		\For{$i \gets 1$ \textbf{to} $i < \mathit{lxsize} - 1$}
		\Cost{$c_{2}$}{$(\mathit{lysize} - 2)(\mathit{lxsize} - 1)$}
		\label{alg:arlabeling-regionlabeling-loop2-start}

			\State $\mathit{coorTresh} \gets \mathit{tresh}$
			\Cost{$c_{3}$}{$(\mathit{lysize} - 2)(\mathit{lxsize} - 2)$}
			\label{alg:arlabeling-regionlabeling-threshold-start}
			\State $\mathit{isBlack} \gets \textbf{false}$
			\Cost{$c_{4}$}{$(\mathit{lysize} - 2)(\mathit{lxsize} - 2)$}
			\State $\mathit{isBlack} \gets \left(
			\begin{aligned}
				& \quad (\mathit{pnt} + 0) \\
				& \quad + (\mathit{pnt} + 1) \\
				& \quad + (\mathit{pnt} + 2) \\
				& \leq \mathit{coorTresh}
			\end{aligned}\right)$
			\Cost{$c_{5}$}{$(\mathit{lysize} - 2)(\mathit{lxsize} - 2)7$}
			\label{alg:arlabeling-regionlabeling-calcblack}

			\If{$\mathit{isBlack}$}
			\Cost{$c_{6}$}{$(\mathit{lysize} - 2)(\mathit{lxsize} - 2)$}
			\label{alg:arlabeling-regionlabeling-isblack?}
				\State Untersuche 8er-Nachbarschaft
				\Cost{$c_{7}$}{$(\mathit{lysize} - 2)(\mathit{lxsize} - 2)t_{1}$}
				\label{alg:arlabeling-regionlabeling-black}
			\Else
				\State $\mathit{pnt2} \gets 0$
				\Cost{$c_{9}$}{$(\mathit{lysize} - 2)(\mathit{lxsize} - 2)$}
				\label{alg:arlabeling-regionlabeling-notblack}
			\EndIf
			\label{alg:arlabeling-regionlabeling-threshold-end}

			\State $i \gets i + 1$
			\Cost{$c_{11}$}{$(\mathit{lysize} - 2)(\mathit{lxsize} - 2)$}
			\label{alg:arlabeling-regionlabeling-inc1-start}
			\State $\mathit{pnt} \gets \mathit{pnt} + \mathit{poff}$
			\Cost{$c_{12}$}{$(\mathit{lysize} - 2)(\mathit{lxsize} - 2)2$}
			\State Inkrementiere $\mathit{pnt2}$
			\Cost{$c_{13}$}{$(\mathit{lysize} - 2)(\mathit{lxsize} - 2)$}
			\label{alg:arlabeling-regionlabeling-inc1-end}
		\EndFor
		\label{alg:arlabeling-regionlabeling-loop2-end}

		\State $\mathit{pnt} \gets \mathit{pnt} + \mathit{arImXsize} \cdot \mathit{pixelSize}$
		\Cost{$c_{15}$}{$(\mathit{lysize} - 2)3$}
		\label{alg:arlabeling-regionlabeling-inc2-start}
		\State $j \gets j + 1$
		\Cost{$c_{16}$}{$(\mathit{lysize} - 2)$}
		\State $\mathit{pnt} \gets \mathit{pnt} + \mathit{poff} \cdot 2$
		\Cost{$c_{17}$}{$(\mathit{lysize} - 2)3$}
		\State $\mathit{pnt2} \gets \mathit{pnt2} + 2$
		\Cost{$c_{18}$}{$(\mathit{lysize} - 2)2$}
		\label{alg:arlabeling-regionlabeling-inc2-end}
	\EndFor
	\label{alg:arlabeling-regionlabeling-loop1-end}
\end{algorithmic}
\end{algorithm}
 \textproc{arLabeling} untersucht das Bildsignal $I$ zeilenweise von links oben nach rechts unten. Durch eine
 Schwellwertanalyse wird entschieden, ob ein \gls{pixel} an Position $I(u,v)$ ein Vordergrund- oder Hintergrundpixel
 ist. Die Regionenmarkierung wird dann in $\mathit{l\_image}$ gespeichert. Das Verfahren wird solange wiederholt, bis
 das Bildsignal $I$ vollständig prozessiert wurde und zusammenhängende Bildregionen in $\mathit{l\_image}$ markiert
 sind. Die Variable $\mathit{pnt}$ enthält die Adresse des zu untersuchenden \gls{pixel} aus $I$. Die nächste freie
 Speicherstelle in $\mathit{l\_image}$ ist in $\mathit{pnt2}$ hinterlegt. In den beiden Schleifen in Zeile
 \ref{alg:arlabeling-regionlabeling-loop1-start}--\ref{alg:arlabeling-regionlabeling-loop1-end} und Zeile
 \ref{alg:arlabeling-regionlabeling-loop2-start}--\ref{alg:arlabeling-regionlabeling-loop2-end} wird das Bildsignal
 zeilenweise, von oben links nach unten rechts, verarbeitet. Das Inkrementieren der Variablen in Zeile
 \ref{alg:arlabeling-regionlabeling-inc1-start}--\ref{alg:arlabeling-regionlabeling-inc1-end} und Zeile
 \ref{alg:arlabeling-regionlabeling-inc2-start}--\ref{alg:arlabeling-regionlabeling-inc2-end} sorgt dafür, dass nur die
 Hälfte der \gls{pixel} prozessiert werden (Vgl. \autoref{sec:vorgehen}). Die Schwellwertanalyse wird in Zeile
 \ref{alg:arlabeling-regionlabeling-threshold-start}--\ref{alg:arlabeling-regionlabeling-threshold-end} durchgeführt.
 Dazu wird in Zeile \ref{alg:arlabeling-regionlabeling-calcblack} die RGB-Komponente des Bildsignals $I$ ausgelesen und
 addiert. Normalerweise würde man an dieser Stelle den Schwellwert mit jeder Komponente $R$, $G$ und $B$ einzeln
 vergleichen. Die Verdreifachung des Schwellwerts in \autoref{alg:arlabeling-init1} und die Addition der
 RGB-Komponenten ermöglichen hingegen eine Schwellwertanalyse mit nur einem Vergleich.\label{sub:arlabel-threshold} In
 Zeile \ref{alg:arlabeling-regionlabeling-isblack?}--\ref{alg:arlabeling-regionlabeling-threshold-end} wird
 untersucht, ob ein Vordergrundpixel gefunden wurde. Falls nicht, wird in das Regionenbild $\mathit{l\_image}$ eine $0$
 geschrieben. Wenn die Schwellertanalyse ein Vordergrundpixel bestimmt hat, müssen in Zeile
 \ref{alg:arlabeling-regionlabeling-black} die Nachbarn des Vordergrundpixels mit einer 8er-Nachbarschaft untersucht
 werden (Vgl. \autoref{alg:arlabeling-neighbour}). Die Laufzeitfunktion für den besten Fall ist in
 \autoref{eq:analyse-arlabeling-regionlabeling-1} angegeben.
\begin{subequations}
\label{eq:analyse-arlabeling-regionlabeling-1}
\begin{align}
\label{eq:analyse-arlabeling-regionlabeling-1-1}
T_{best}(\mathit{lysize},\mathit{lxsize})& = c_{1}(\mathit{lysize} - 1) + (\mathit{lysize} - 2) \\
& \quad \cdot \bigl(c_2(\mathit{lxsize} - 1) + (c_{3} + c_{4} + c_{6} + c_{9} + c_{11} + c_{13})(\mathit{lxsize} - 2)
 \nonumber \\
& \quad \quad + c_{5}7(\mathit{lxsize} - 2) + c_{12}2(\mathit{lxsize} - 2) \bigr) \nonumber \\
\label{eq:analyse-arlabeling-regionlabeling-1-2}
T_{best}(\mathit{lysize},\mathit{lxsize})& = c_{1}(\mathit{lysize} - 1) + (\mathit{lysize} - 2) \\
& \quad \cdot \bigl((c_{2} + c_{3} + c_{4} + c_{5} \cdot 7 + c_{6} + c_{9} + c_{11} + c_{12} \cdot 2 + c_{13})\mathit{lxsize}
 \nonumber \\
& \quad \quad - 2(\frac{c_2}{2} + c_{3} + c_{4} + c_{5} \cdot 7 + c_{6} + c_{9} + c_{11} + c_{12} \cdot 2 + c_{13})\bigr)
 \nonumber \\
\label{eq:analyse-arlabeling-regionlabeling-1-3}
T_{best}(\mathit{lysize},\mathit{lxsize})& = c_{1}\mathit{lysize} - c_{1}
 \\
& \quad + (c_{2} + c_{3} + c_{4} + c_{5} \cdot 7 + c_{6} + c_{9} + c_{11} + c_{12} \cdot 2 + c_{13})\mathit{lxsize}\cdot\mathit{lysize}
 \nonumber \\
& \quad - 2(c_{2} + c_{3} + c_{4} + c_{5} \cdot 7 + c_{6} + c_{9} + c_{11} + c_{12} \cdot 2 + c_{13})\mathit{lxsize}
 \nonumber \\
& \quad - 2(\frac{c_2}{2} + c_{3} + c_{4} + c_{5} \cdot 7 + c_{6} + c_{9} + c_{11} + c_{12} \cdot 2 + c_{13})\mathit{lysize}
 \nonumber \\
& \quad + 4(\frac{c_2}{2} + c_{3} + c_{4} + c_{5} \cdot 7 + c_{6} + c_{9} + c_{11} + c_{12} \cdot 2 + c_{13})
 \nonumber
\end{align}
\end{subequations}

Durch Substition der Konstanten $c$ mit $1$ folgt
 $T_{best}(\mathit{lxsize},\mathit{lysize}) = 16 \cdot \mathit{lxsize} \cdot \mathit{lysize} - 32 \cdot \mathit{lxsize}
 - 31 \cdot \mathit{lysize} + 61$. Für $c_{1} = 2$, $c_{2} = 16$, $\mathit{lxsize}_{0} = 9$ und
 $\mathit{lysize}_{0} = 3$ ist $16 \cdot \mathit{lxsize} \cdot \mathit{lysize} - 32 \cdot \mathit{lxsize}
 - 31 \cdot \mathit{lysize} = \Theta(\mathit{lxsize} \cdot \mathit{lysize})$. Die Laufzeitfunktion für den schlechtesten
 Fall wird nach Analyse der Nachbarschaftsuntersuchung angegeben.

Die Nachbarschafsuntersuchung in ARToolKitPlus untersucht die vier Nachbarn der Markierung an Position $I(u,v)$. Wie in
 \autoref{sec:vorläufige_makierung} bereits beschrieben, ist eine Regionenmarkierung davon abhängig ob alle Nachbarn
 Hintergrundpixel sind, genau ein Nachbar eine Markierung hat oder mehrere Nachbarn eine Markierung haben. Die Fälle
 1, 3 und 4 in \autoref{alg:arlabeling-neighbour} untersuchen, ob genau ein Nachbar eine Markierung hat.
\begin{algorithm}[!ht]
\caption{\textproc{arLabeling} (Untersuchung der 8er-Nachbarschaft)}
\label{alg:arlabeling-neighbour}
\begin{algorithmic}[1]
	\State $\mathit{pnt1} \gets \mathit{pnt2}[-\mathit{lxsize}]$ \Comment Adresse zuweisen
	\Cost{$c_{1}$}{$3$}
	\label{alg:arlabeling-neighbour-n3}
	\If{$\mathit{pnt1} > 0$}
	\Cost{$c_{2}$}{$2$}
		\State \ldots \Comment 1. Fall
	\ElsIf{$\left(\mathit{pnt1} + 1\right) > 0$}
	\Cost{$c_{4}$}{$2$}
		\State \ldots \Comment 2. Fall
	\ElsIf{$\left(\mathit{pnt1} - 1\right) > 0$}
	\Cost{$c_{6}$}{$2$}
		\State \ldots \Comment 3. Fall
	\ElsIf{$\left(\mathit{pnt2} - 1\right) > 0$}
	\Cost{$c_{8}$}{$2$}
		\State \ldots \Comment 4. Fall
	\Else
		\State \ldots \Comment 5. Fall
	\EndIf
\end{algorithmic}
\end{algorithm}

Bei Fall 5 sind alle Nachbarn Hintergrundpixel. Nur bei Fall 2 können  mehrere Nachbarn eine Markierung haben. In Zeile
 \ref{alg:arlabeling-neighbour-n3} wird der Variablen $\mathit{pnt1}$ der Nachbar $N_3 = (u,v+1)$ zugewiesen. Die
 Nachbarn sind in \autoref{fig:analyse-nachbarschaftsbeziehung} illustriert.
\begin{figure}[!ht]
	\centering
	\input{resources/8er-Nachbarschaft.pdf_tex}
	\caption{8er-Nachbarschaft mit $N_1 = I(u-1,v)$, $N_2 = I(u-1,v+1)$, $N_3 = (u,v+1)$ und $N_4 = I(u+1,v+1)$.}
	\label{fig:analyse-nachbarschaftsbeziehung}
\end{figure}
\paragraph{1. Fall:} % (fold)
\label{par:fall_1_}
Wir wissen durch \autoref{alg:arlabeling-regionlabeling}, dass $I(u,v)$ ein Vordergrundpixel ist, dem wir an dieser
 Stelle eine Markierung zuweisen wollen. Im ersten Fall wird die Markierung $N_3$ untersucht. Wenn $N_3$ ein
 Vordergrundpixel ist kann die Markierung für $I(u,v)$ übernommen werden. Falls es sich bei den Nachbarn
 $N_1 = I(u-1,v)$, $N_2 = I(u-1,v+1)$ und $N_4 = I(u+1,v+1)$ um Vordergrundpixel handelt, haben sie die gleiche
 Markierung wie $N_3$ und müssen nicht weiter untersucht werden. Das Verfahren ist in
 \autoref{alg:arlabeling-neighbour-case1} beschrieben.
% Fall 1: *pnt1 > 0
\begin{algorithm}[ht]
\caption{\textproc{arLabeling} (8er-Nachbarschaft: 1. Fall)}
\label{alg:arlabeling-neighbour-case1}
\begin{algorithmic}[1]

	\If{$\mathit{pnt1} > 0$}
		\State $\mathit{pnt2} \gets \mathit{pnt1}$
		\label{alg:arlabeling-neighbour-case1-save-label}
		\State $\mathit{pnt2\_index} \gets \left(\mathit{pnt2} - 1\right) \cdot 7$
		\label{alg:arlabeling-neighbour-case1-calc-offset}
		\State $\mathit{work2}\left[\mathit{pnt2\_index} + 0\right] \gets \mathit{work2}\left[\mathit{pnt2\_index} + 0\right] + 1$
		\label{alg:arlabeling-neighbour-case1-inc-region}
		\State $\mathit{work2}\left[\mathit{pnt2\_index} + 1\right] \gets \mathit{work2}\left[\mathit{pnt2\_index} + 1\right] + i$
		\State $\mathit{work2}\left[\mathit{pnt2\_index} + 2\right] \gets \mathit{work2}\left[\mathit{pnt2\_index} + 2\right] + j$
		\State $\mathit{work2}\left[\mathit{pnt2\_index} + 6\right] \gets j$
		\label{alg:arlabeling-neighbour-case1-save-j}
		\algstore{brk-case1}
\end{algorithmic}
\end{algorithm}

In Zeile \ref{alg:arlabeling-neighbour-case1-save-label} wird die Markierung von $N_3$ übernommen und in
 $\mathit{l\_image}$ an Position $(x,y)$ gespeichert. In der Variablen $\mathit{work2}$ werden Informationen der
 Regionenmarkierung gespeichert. Dazu wird zuerst in Zeile \ref{alg:arlabeling-neighbour-case1-calc-offset} aus
 $\mathit{pnt2}$ der Wert der Regionenmarkierung gelesen. Der Wert der Markierung wird zur Berechnung des
 Adressabstands benutzt, um die Werte für die Region an die richtige Stelle zu schreiben. An der Position $0$ von
 $\mathit{work2}$ (Zeile \ref{alg:arlabeling-neighbour-case1-inc-region}) wird die Anzahl der Vordergrundpixel der
 Region erhöht. Position 1 und Position 2 von $\mathit{work2}$ enthalten die akummulierten Werte von $i$ und $j$ für
 die $x$- und $y$-Koordinaten aller Vordergrundpixel der Region. An Position 6 von $\mathit{work2}$ in Zeile
 \ref{alg:arlabeling-neighbour-case1-save-j} wird die $y$-Koordinate gespeichert. Die Kosten des Algorithmus sind in
 \autoref{alg:arlabeling-neighbour-case1} angegeben. Die Laufzeitfunktion $T(n) = 26$ ist konstant.
% paragraph fall_1_ (end)

\paragraph{2. Fall:} % (fold)
\label{par:fall_2_}
Beim zweiten Fall wird die Markierung $N_4$ betrachtet. Da $N_3$ keine Markierung aufweist, können nur $N_1$ und $N_2$
 Markierungen haben. Da die Markierungen nur durch $I(u,v)$ verbunden sind, kann es sich hier um eine Kollision
 handeln, die durch \autoref{alg:arlabeling-neighbour-case2-1} oder \autoref{alg:arlabeling-neighbour-case2-2}
 besonders behandelt wird. Das Verfahren ist in \autoref{alg:arlabeling-neighbour-case2} aufgeführt.
% Fall 2: *(pnt1+1) > 0
\begin{algorithm}[!ht]\small
\caption{\textproc{arLabeling} (8er-Nachbarschaft: 2. Fall)}
\label{alg:arlabeling-neighbour-case2}
\begin{algorithmic}[1]
	\algrestore{brk-case1}

	\ElsIf{$\left(\mathit{pnt1} + 1\right) > 0$}
	\Cost{$c_{8}$}{$2$}
		\If{$\left(\mathit{pnt1} - 1\right) > 0$}
		\Cost{$c_{9}$}{$2$}
		\Comment Ist in $N_2$ eine Markierung?
		\label{alg:arlabeling-neighbour-case2-n2}
			\State \ldots

		\ElsIf{$\left(\mathit{pnt2} - 1\right) > 0$}
		\Cost{$c_{11}$}{$2$}
		\Comment Ist in $N_1$ eine Markierung?
		\label{alg:arlabeling-neighbour-case2-n1}
			\State \ldots

		\Else \Comment Nur $N_4$ hat eine Markierung.
		\label{alg:arlabeling-neighbour-case2-4}
			\State \ldots
		\EndIf
		\algstore{brk-case2}
\end{algorithmic}
\end{algorithm}

In Zeile \ref{alg:arlabeling-neighbour-case2-n2} wird überprüft, ob $N_2$ eine Markierung enthält. Falls ja, wird
 \autoref{alg:arlabeling-neighbour-case2-1} untersuchen, ob eine Kollision vorliegt und sie gegebenfalls auflösen.
 Zeile \ref{alg:arlabeling-neighbour-case2-n1} überprüft $N_1$ auf eine vorhandene Markierung und fährt mit der
 Untersuchung einer evtl. Kollision in \autoref{alg:arlabeling-neighbour-case2-2} fort. Wenn weder $N_2$ noch $N_1$
 eine Markierung haben, ist nur $N_4$ ein Vordergrundpixel und wird mit \autoref{alg:arlabeling-neighbour-case2-3}
 markiert.

In \autoref{alg:arlabeling-neighbour-case2-1} wird in Zeile \ref{alg:arlabeling-neighbour-case2-1-m} der Wert der
 Markierung $N_4$ in Variable $m$ gespeichert.
% Fall 2: *(pnt1+1) > 0
\begin{algorithm}[!ht]\small
\caption{\textproc{arLabeling} (8er-Nachbarschaft: 2. Fall, $N_4$ und $N_2$)}
\label{alg:arlabeling-neighbour-case2-1}
\begin{algorithmic}[1]
	% \State \Comment $N_4$ und $N_2$ haben eine Markierung.
	\State $m \gets \mathit{work}\left[\left(\mathit{pnt1} + 1\right) - 1\right]$
	\Cost{$c_{1}$}{$4$}
	\label{alg:arlabeling-neighbour-case2-1-m}
	\State $n \gets \mathit{work}\left[\left(\mathit{pnt1} - 1\right) - 1\right]$
	\Cost{$c_{2}$}{$4$}
	\label{alg:arlabeling-neighbour-case2-1-n}

	\If{$m > n$}
	\Cost{$c_{3}$}{$1$}
	\label{alg:arlabeling-neighbour-case2-mn-start}
		\State $\mathit{pnt2} \gets n$
		\Cost{$c_{4}$}{$1$}
		\label{alg:arlabeling-neighbour-case2-saven}
		\State $\mathit{wk} \gets \left(\mathit{work}\left[0\right]\right)$
		\Cost{$c_{5}$}{$2$}
		\label{alg:arlabeling-neighbour-case2-worklist}
		\For{$k \gets 0$ \textbf{to} $k < \mathit{wk\_max}$}
		\Cost{$c_{6}$}{$\mathit{wk\_max} + 1$}
		\label{alg:arlabeling-neighbour-case2-loop-start}
			\If{$\mathit{wk} = m$}
			\Cost{$c_{7}$}{$\mathit{wk\_max}$}
				\State $\mathit{wk} \gets n$
				\Cost{$c_{8}$}{$\mathit{wk\_max}$}
			\EndIf
			\State Inkrementiere $\mathit{wk}$
			\Cost{$c_{10}$}{$\mathit{wk\_max}$}
			\State $k \gets k + 1$
			\Cost{$c_{11}$}{$\mathit{wk\_max}$}
		\EndFor
		\label{alg:arlabeling-neighbour-case2-loop-end}
	\label{alg:arlabeling-neighbour-case2-mn-end}
	\ElsIf{$ m < n$}
	\Cost{$c_{13}$}{$1$}
	\label{alg:arlabeling-neighbour-case2-nm-start}
		\State $\mathit{pnt2} \gets m$
		\Cost{$c_{14}$}{$1$}
		\State $\mathit{wk} \gets \left(\mathit{work}\left[0\right]\right)$
		\Cost{$c_{15}$}{$2$}
		\For{$k \gets 0$ \textbf{to} $k < \mathit{wk\_max}$}
		\Cost{$c_{16}$}{$\mathit{wk\_max} + 1$}
			\If{$\mathit{wk} = n$}
			\Cost{$c_{17}$}{$\mathit{wk\_max}$}
				\State $\mathit{wk} \gets m$
				\Cost{$c_{18}$}{$\mathit{wk\_max}$}
			\EndIf
			\State Inkrementiere $\mathit{wk}$
			\Cost{$c_{20}$}{$\mathit{wk\_max}$}
			\State $k \gets k + 1$
			\Cost{$c_{21}$}{$\mathit{wk\_max}$}
		\EndFor
	\label{alg:arlabeling-neighbour-case2-nm-stop}
	\Else
		\State $\mathit{pnt2} \gets m$
		\Cost{$c_{24}$}{$1$}
		\label{alg:arlabeling-neighbour-case2-savem}
	\EndIf
\end{algorithmic}
\end{algorithm}

In Zeile \ref{alg:arlabeling-neighbour-case2-1-n} wird der Wert $N_2$ in $n$ hinterlegt. In Zeile
 \ref{alg:arlabeling-neighbour-case2-mn-start}--\ref{alg:arlabeling-neighbour-case2-mn-end} wird überprüft, ob $m$
 größer als $n$ ist. Falls dem so ist, wird der Wert $n$ in $\mathit{l\_image}$ gespeichert
 (Zeile \ref{alg:arlabeling-neighbour-case2-saven}). Danach wird in Zeile \ref{alg:arlabeling-neighbour-case2-worklist}
 die Adresse der ersten Stelle der Regionenmarkierungsliste $\mathit{work}$ in $wk$ hinterlegt. In der Schleife von
 Zeile \ref{alg:arlabeling-neighbour-case2-loop-start} bis Zeile \ref{alg:arlabeling-neighbour-case2-loop-end} wird die
 Liste durchlaufen und alle Werte von $m$ durch den Wert $n$ ersetzt. Falls der Wert $m$ kleiner als $n$ ist
 (Zeile \ref{alg:arlabeling-neighbour-case2-nm-start}--\ref{alg:arlabeling-neighbour-case2-nm-stop}) wird das gleiche
 Verfahren angewendet. Lediglich $m$ und $n$ werden getauscht. Wenn es sich bei $m$ und $n$ um den gleichen Wert
 handet, und somit $m$ und $n$ zur gleichen Region gehören, wird der Wert $m$ in $\mathit{l\_image}$ gespeichert
 (Zeile \ref{alg:arlabeling-neighbour-case2-savem}). Die Laufzeitfunktion für den besten Fall ist in
 \autoref{eq:analyse-arlabeling-neighbour-case2-1-1} angegben und ist konstant. Die Laufzeitfunktion für den
 schlechtesten Fall ist in \autoref{eq:analyse-arlabeling-neighbour-case2-1-2} angegeben und durchläuft Zeile
 \ref{alg:arlabeling-neighbour-case2-nm-start}--\ref{alg:arlabeling-neighbour-case2-nm-stop}. Für
 $T_{worst}(\mathit{wk\_max})=5 \mathit{wk\_max} + 14$ und den Werten $c =5$ und
 $\mathit{wk\_max}_{0} = 1$ ist $5 \mathit{wk\_max} = O(n)$.
\begin{subequations}
\label{eq:analyse-arlabeling-neighbour-case2-1}
\begin{align}
\label{eq:analyse-arlabeling-neighbour-case2-1-1}
T_{best}(\mathit{wk\_max})& = (c_{3} + c_{13} + c_{24}) + (c_{1} + c_{2})4 \\
\label{eq:analyse-arlabeling-neighbour-case2-1-2}
T_{worst}(\mathit{wk\_max})& = (c_{16} + c_{17} + c_{18} + c_{20} + c_{21})\mathit{wk\_max} \\
& \quad + (c_{3} + c_{13} + c_{14} + c_{16}) + c_{15}2 + (c_{1} + c_{2})4 \nonumber
\end{align}
\end{subequations}


\autoref{alg:arlabeling-neighbour-case2-2} behandelt den Fall, dass $N_4$ und $N_1$ eine Markierung aufweisen. Das Verfahren entspricht dem Verfahren in \autoref{alg:arlabeling-neighbour-case2-1}.
% Fall 2: *(pnt1+1) > 0
\begin{algorithm}[ht]
\caption{\textproc{arLabeling} (8er-Nachbarschaft: 2. Fall, $N_4$ und $N_1$)}
\label{alg:arlabeling-neighbour-case2-2}
\begin{algorithmic}[1]
	\State \Comment $N_4$ und $N_1$ haben eine Markierung.
	\State $m \gets \mathit{work}\left[\left(\mathit{pnt1} + 1\right) - 1\right]$ \Cost{$c_{1}$}{$1$}
	\State $n \gets \mathit{work}\left[\left(\mathit{pnt2} - 1\right) - 1\right]$ \Cost{$c_{2}$}{$1$}

	\If{$m > n$} \Cost{$c_{3}$}{$1$}
		\State $\mathit{pnt2} \gets n$ \Cost{$c_{4}$}{$1$}
		\State $\mathit{wk} \gets \left(\mathit{work}\left[0\right]\right)$ \Cost{$c_{5}$}{$1$}
		\For{$k \gets 0$ \textbf{to} $k < \mathit{wk\_max}$} \Cost{$c_{6}$}{$\mathit{wk\_max} + 1$}
			\If{$\mathit{wk} == m$} \Cost{$c_{7}$}{$\mathit{wk\_max}$}
				\State $\mathit{wk} \gets n$ \Cost{$c_{8}$}{$\mathit{wk\_max}$}
			\EndIf
			\State Inkrementiere $\mathit{wk}$ \Cost{$c_{9}$}{$\mathit{wk\_max}$}
			\State $k \gets k + 1$ \Cost{$c_{9}$}{$\mathit{wk\_max}$}
		\EndFor

	\ElsIf{$ m < n$}
		\State $\mathit{pnt2} \gets m$
		\State $\mathit{wk} \gets \left(\mathit{work}\left[0\right]\right)$
		\For{$k \gets 0$ \textbf{to} $k < \mathit{wk\_max}$}
			\If{$\mathit{wk} == n$}
				\State $\mathit{wk} \gets m$
			\EndIf
			\State Inkrementiere $\mathit{wk}$
			\State $k \gets k + 1$
		\EndFor

	\Else
		\State $\mathit{pnt2} \gets m$
	\EndIf
\end{algorithmic}
\end{algorithm}


In \autoref{alg:arlabeling-neighbour-case2-3} ist der Fall beschrieben, dass $N_4$ der einzige Nachbar des
 Vordergrundpixels $(u,v)$ ist.
% Fall 2: *(pnt1+1) > 0
\begin{algorithm}[!ht]
\caption{\textproc{arLabeling} (8er-Nachbarschaft: 2. Fall, $N_4$)}
\label{alg:arlabeling-neighbour-case2-3}
\begin{algorithmic}[1]
	\State $\mathit{pnt2} \gets \left(\mathit{pnt1} + 1\right)$
	\Cost{$c_{1}$}{$2$}
	\label{alg:arlabeling-neighbour-case2-3-n4}
	\State $\mathit{pnt2\_index} \gets \left(\left(\mathit{pnt2}\right) - 1\right) \cdot 7$
	\Cost{$c_{2}$}{$4$}
	\State $\mathit{work2}\left[\mathit{pnt2\_index} + 0\right] \gets
	 \mathit{work2}\left[\mathit{pnt2\_index} + 0\right] + 1$
	\Cost{$c_{3}$}{$6$}
	\label{alg:arlabeling-neighbour-case2-3-incregion}
	\State $\mathit{work2}\left[\mathit{pnt2\_index} + 1\right] \gets
	 \mathit{work2}\left[\mathit{pnt2\_index} + 1\right] + i$
	\Cost{$c_{4}$}{$6$}
	\State $\mathit{work2}\left[\mathit{pnt2\_index} + 2\right] \gets
	 \mathit{work2}\left[\mathit{pnt2\_index} + 2\right] + j$
	\Cost{$c_{5}$}{$6$}
	\If{$\mathit{work2}\left[\mathit{pnt2\_index} + 3\right] > i$}
	\Cost{$c_{6}$}{$3$}
	\label{alg:arlabeling-neighbour-case2-3-isismaller}
		\State $\mathit{work2}\left[\mathit{pnt2\_index} + 3\right] \gets i$
		\Cost{$c_{7}$}{$3$}
		\label{alg:arlabeling-neighbour-case2-3-newi}
	\EndIf
	\State $\mathit{work2}\left[\mathit{pnt2\_index} + 6\right] \gets j$
	\Cost{$c_{9}$}{$3$}
\end{algorithmic}
\end{algorithm}

In diesem Fall wird der Wert der Markierung von $N_4$ in $\mathit{l\_image}$ gespeichert
 (Zeile \ref{alg:arlabeling-neighbour-case2-3-n4}). Danach wird der Adressabstand berechnet, um die Informationen der
 Regionenmarkierung in $\mathit{work2}$ zu aktualisieren. Zuerst wird in Zeile
 \ref{alg:arlabeling-neighbour-case2-3-incregion} die Anzahl der Vordergrundpixel der Region erhöht. An Position 1 und
 Position 2 von $\mathit{work2}$ werden die Werte von $i$ und $j$ aufaddiert. Falls in Zeile
 \ref{alg:arlabeling-neighbour-case2-3-isismaller} die Position $x$ des ersten Vordergrundpixels der Region größer ist,
 als der aktuelle Wert in $i$, wird die Position in Zeile \ref{alg:arlabeling-neighbour-case2-3-newi} aktualisiert. Zum
 Schluss wird die Position $y$ des letzten Vordergrundpixels der Region durch den aktuellen Wert $j$ ersetzt. Die
 Laufzeitfunktion ist in \autoref{eq:analyse-arlabeling-neighbour-case2-3} angegeben. Die Wachstumsrate ist konstant.
\begin{align}
\label{eq:analyse-arlabeling-neighbour-case2-3}
T_{worst}(n)& = (c_{3} + c_{4} + c_{5})6 + c_{2}4 + (c_{6} + c_{7} + c_{9})3 + c_{1}2
\end{align}

% paragraph fall_2_ (end)

\paragraph{3. Fall:} % (fold)
\label{par:fall_3_}
Beim dritten Fall wird die Markierung $N_2$ untersucht und wir wissen, dass $N_3$ und $N_4$ keine Markierungen haben
 können. Demnach kann es nur den Nachbarn $N_2$ geben, dessen Markierung an $I(u,v)$ weitergereicht wird. Falls $N_1$
 ebenfalls ein Vordergrundpixel ist, handelt es sich um die gleiche Markierung wie in $N_2$, da beide Vordergrundpixel
 direkt miteinander verbunden sind. Das Verfahren ist in \autoref{alg:arlabeling-neighbour-case3} dargestellt.
% Fall 3: *(pnt1-1) > 0
\begin{algorithm}[ht]
\caption{\textproc{arLabeling} (8er-Nachbarschaft: 3. Fall)}
\label{alg:arlabeling-neighbour-case3}
\begin{algorithmic}[1]
	\algrestore{brk-case2}
	\ElsIf{$\left(\mathit{pnt1} - 1\right) > 0$}
		\State $\mathit{pnt2} \gets \left(\mathit{pnt1} - 1\right)$
		\State $\mathit{pnt2\_index} \gets \left(\mathit{pnt2} - 1\right) \cdot 7$
		\label{alg:arlabeling-neighbour-case3-offset}
		\State $\mathit{work2}\left[\mathit{pnt2\_index} + 0\right] \gets \mathit{work2}\left[\mathit{pnt2\_index} + 0\right] + 1$
		\State $\mathit{work2}\left[\mathit{pnt2\_index} + 1\right] \gets \mathit{work2}\left[\mathit{pnt2\_index} + 1\right] + i$
		\State $\mathit{work2}\left[\mathit{pnt2\_index} + 2\right] \gets \mathit{work2}\left[\mathit{pnt2\_index} + 2\right] + j$
		\If{$\mathit{work2}\left[\mathit{pnt2\_index} + 4\right] < i$}
			\State $\mathit{work2}\left[\mathit{pnt2\_index} + 4\right] \gets i$
			\label{alg:arlabeling-neighbour-case3-newi}
		\EndIf
		\State $\mathit{work2}\left[\mathit{pnt2\_index} + 6\right] \gets j$
		\algstore{brk-case3}
\end{algorithmic}
\end{algorithm}

Für diesen Fall müssen nur die Daten in $\mathit{work2}$ gespeichert werden. Der Wert von $N_2$ wird im Regionenbild
 $\mathit{l\_image}$ gespeichert. Danach wird in Zeile \ref{alg:arlabeling-neighbour-case3-offset} der Adressabstand
 zur Speicherung der Daten in $\mathit{work2}$ berechnet. Danach wird die Anzahl der Vordergrundpixel der Region erhöht
 und $i$ und $j$ auf die bestehenden Werte in $\mathit{work2}$ addiert. Falls die $x$-Koordinate des letzten
 Vordergrundpixels der Region kleiner als der aktuelle Wert $i$ ist, wird die $x$-Koordinate in Zeile
 \ref{alg:arlabeling-neighbour-case3-newi} ersetzt. Zum Schluss wird die $y$-Koordinate des letzten Vordergrundpixels
 der Region mit dem Wert $j$ aktualisiert. Die Laufzeitfunktion ist in \autoref{eq:analyse-arlabeling-neighbour-case3}
 angegeben. Die Wachstumsrate ist konstant.
\begin{align}
\label{eq:analyse-arlabeling-neighbour-case3}
T_{worst}(n)& = (c_{19} + c_{20} + c_{21})6 + (c_{18} + c_{22} + c_{23} + c_{25})3 + (c_{16} + c_{17})2
\end{align}

% paragraph fall_3_ (end)

\paragraph{4. Fall:} % (fold)
\label{par:fall_4_}
Der vierte Fall untersucht den letzten, und einzigen, Nachbarn $N_1$ (\autoref{alg:arlabeling-neighbour-case4}).
% Fall 4: *(pnt2-1) > 0
\begin{algorithm}[ht]
\caption{\textproc{arLabeling} (8er-Nachbarschaft: 4. Fall)}
\label{alg:arlabeling-neighbour-case4}
\begin{algorithmic}[1]
	\algrestore{brk-case3}
	\ElsIf{$\left(\mathit{pnt2} - 1\right) > 0$}
		\State $\mathit{pnt2} \gets \left(\mathit{pnt2} - 1\right)$
		\label{alg:arlabeling-neighbour-case4-n1}
		\State $\mathit{pnt2\_index} \gets \left(\left(\mathit{pnt2}\right) - 1\right) \cdot 7$
		\State $\mathit{work2}\left[\mathit{pnt2\_index} + 0\right] \gets \mathit{work2}\left[\mathit{pnt2\_index} + 0\right] + 1$
		\State $\mathit{work2}\left[\mathit{pnt2\_index} + 1\right] \gets \mathit{work2}\left[\mathit{pnt2\_index} + 1\right] + i$
		\State $\mathit{work2}\left[\mathit{pnt2\_index} + 2\right] \gets \mathit{work2}\left[\mathit{pnt2\_index} + 2\right] + j$
		\If{$\mathit{work2}\left[\mathit{pnt2\_index} + 4\right] < i$}
			\State $\mathit{work2}\left[\mathit{pnt2\_index} + 4\right] \gets i$
		\EndIf
	\algstore{brk-case4}
\end{algorithmic}
\end{algorithm}

Alle anderen Nachbarn sind keine Vordergrundpixel und es besteht keine Kollision. Die Markierung von $N_1$ wird für
 $I(u,v)$ übernommen (Zeile \ref{alg:arlabeling-neighbour-case4-n1}). Ansonsten sind
 \autoref{alg:arlabeling-neighbour-case4} und \autoref{alg:arlabeling-neighbour-case3} identisch. Die Laufzeitfunktion
 in \autoref{eq:analyse-arlabeling-neighbour-case4} hat eine konstante Wachstumsrate.
\begin{align}
\label{eq:analyse-arlabeling-neighbour-case4}
T_{worst}(n)& = (c_{29} + c_{30} + c_{31})6 + c_{28}4 + (c_{32} + c_{33})3 + (c_{26} + c_{27})2
\end{align}

% paragraph fall_4_ (end)

\paragraph{5. Fall:} % (fold)
\label{par:fall_5_}
Im letzten Fall sind alle Nachbarn Hintergrundpixel und $\mathit{l\_image}$ wird eine neue Markierung für $I(u,v)$
 zugewiesen (\autoref{alg:arlabeling-neighbour-case5}).
% Fall 5: else
\begin{algorithm}[!ht]\small
\caption{\textproc{arLabeling} (8er-Nachbarschaft: 5. Fall)}
\label{alg:arlabeling-neighbour-case5}
\begin{algorithmic}[1]
	\algrestore{brk-case4}
	\Else
		\State $\mathit{wk\_max} \gets \mathit{wk\_max} + 1$
		\Cost{$c_{36}$}{$2$}
		\label{alg:arlabeling-neighbour-case5-incwk_max}
		\If{$\mathit{wk\_max} > \mathit{WORK\_SIZE}$}
		\Cost{$c_{37}$}{$1$}
			\State{\textbf{return} $0$}
			\Cost{$c_{38}$}{$1$}
		\EndIf
		\State $\mathit{pnt2} \gets \mathit{wk\_max}$
		\Cost{$c_{40}$}{$1$}
		\label{alg:arlabeling-neighbour-case5-save-uv}
		\State $\mathit{work}\left[\mathit{wk\_max} - 1\right] \gets \mathit{wk\_max}$
		\Cost{$c_{41}$}{$3$}
		\State $\mathit{wmax\_idx} \gets \left(\mathit{wk\_max} - 1\right) \cdot 7$
		\Cost{$c_{42}$}{$3$}
		\label{alg:arlabeling-neighbour-case5-offset}
		\State $\mathit{work2}\left[\mathit{wmax\_idx} + 0 \right] \gets 1$
		\Cost{$c_{43}$}{$3$}
		\label{alg:arlabeling-neighbour-case5-offset-0}
		\State $\mathit{work2}\left[\mathit{wmax\_idx} + 1 \right] \gets i$
		\Cost{$c_{44}$}{$3$}
		\label{alg:arlabeling-neighbour-case5-offset-1}
		\State $\mathit{work2}\left[\mathit{wmax\_idx} + 2 \right] \gets j$
		\Cost{$c_{45}$}{$3$}
		\State $\mathit{work2}\left[\mathit{wmax\_idx} + 3 \right] \gets i$
		\Cost{$c_{46}$}{$3$}
		\State $\mathit{work2}\left[\mathit{wmax\_idx} + 4 \right] \gets i$
		\Cost{$c_{47}$}{$3$}
		\State $\mathit{work2}\left[\mathit{wmax\_idx} + 5 \right] \gets j$
		\Cost{$c_{48}$}{$3$}
		\State $\mathit{work2}\left[\mathit{wmax\_idx} + 6 \right] \gets j$
		\Cost{$c_{49}$}{$3$}
		\label{alg:arlabeling-neighbour-case5-offset-6}
	\EndIf
\end{algorithmic}
\end{algorithm}

In Zeile \ref{alg:arlabeling-neighbour-case5-incwk_max} wird der aktuelle numerische Markierungswert erhöht. Falls der
 Wert in $\mathit{wk\_max}$ größer als ein festgelegter Wert ist, wird das Verfahren abgebrochen, da zuviele Regionen
 im Bildsignal $I$ vorkommen. Die Markierung wird in Zeile \ref{alg:arlabeling-neighbour-case5-save-uv} in
 $\mathit{l\_image}$ gespeichert. Danach wird der Wert der Markierung in die Liste der Markierungen $\mathit{work}$
 eingetragen. Der Adressabstand für die Region mit der Markierung $\mathit{wk\_max}$ wird in Zeile
 \ref{alg:arlabeling-neighbour-case5-offset} berechnet. Anschließend wird die neue Region in $\mathit{work2}$
 gespeichert. Der erste Vordergrundpixel wird in Zeile \ref{alg:arlabeling-neighbour-case5-offset-0} an Position $0$
 von $\mathit{work2}$ gespeichert. Danach werden in Zeile
 \ref{alg:arlabeling-neighbour-case5-offset-1}--\ref{alg:arlabeling-neighbour-case5-offset-1} die Position des ersten
 Vordergrundpixels gespeichert. Die Laufzeitfunktion ist in \autoref{eq:analyse-arlabeling-neighbour-case5} angegeben.
 Die Wachstumsrate der Funktion ist konstant.
\begin{align}
\label{eq:analyse-arlabeling-neighbour-case5}
T_{worst}(n)& = (c_{41} + c_{42} + c_{43} + c_{44} + c_{45} + c_{46} + c_{47} + c_{48} + c_{49})3 \\
& \quad + c_{36}2 + (c_{37} + c_{38} + c_{40}) \nonumber
\end{align}

% paragraph fall_5_ (end)

Nachdem die Regionemarkierung abgeschlossen ist, enthält $\mathit{work}$ die numerischen Werte der Regionen. In
 $\mathit{work2}$ ist für jede Region die Anzahl der Vordergrundpixel, die summierten $x$- und $y$-Koordinaten aller
 Vordergrundpixel, sowie die $x$- und $y$-Koordinate für den ersten und letzten Vordergrundpixel, hinterlegt. Das
 Regionenbild ist in $\mathit{l\_image}$ gespeichert. Der schlechteste Fall der Nachbarschaftsuntersuchung ist der 2.
 Fall. Die Laufzeitfunktion (\autoref{eq:analyse-arlabeling-neighbour}) setzt sich aus den Kosten der Verzweigung in
 \autoref{alg:arlabeling-neighbour} und \autoref{alg:arlabeling-neighbour-case2}, sowie der Laufzeifunktion aus
 \autoref{alg:arlabeling-neighbour-case2-1}, zusammen. Für $c = 6$ und $\mathit{wk\_max}_{0} = 1$ ist
 $5 \mathit(wk\_max) = O(\mathit{wk\_max})$.
\begin{align}
\label{eq:analyse-arlabeling-neighbour}
T_{worst}(\mathit{wk\_max})& = 6 + 4 + 5 \mathit(wk\_max) + 14 \\
& = 5 \mathit(wk\_max) + 24 \nonumber
\end{align}


Nachdem die Laufzeitfunktion der einzelnen Fälle der Regionenmarkierung bekannt sind, kann die Laufzeitfunktion von
 \textproc{arLabeling} (\autoref{alg:arlabeling-regionlabeling}) für den schlechtesten Fall untersucht werden. Die
 Laufzeitfunktion ist in \autoref{eq:analyse-arlabeling-regionlabeling-2} angegeben.
\begin{subequations}
\label{eq:analyse-arlabeling-regionlabeling-2}
\begin{align}
\label{eq:analyse-arlabeling-regionlabeling-2-1}
T_{worst}(\mathit{lysize},\mathit{lxsize})& = c_{1}(\mathit{lysize} - 1) + (\mathit{lysize} - 2)
 \\
& \quad \cdot \bigl(c_2(\mathit{lxsize} - 1) + (c_{3} + c_{4} + c_{6} + c_{11} + c_{13})(\mathit{lxsize} - 2)
 \nonumber \\
& \quad \quad + c_{5}7(\mathit{lxsize} - 2) + c_{12}2(\mathit{lxsize} - 2) + c_{7}t(\mathit{lxsize} - 2) \bigr) 
 \nonumber \\
\label{eq:analyse-arlabeling-regionlabeling-2-2}
T_{worst}(\mathit{lysize},\mathit{lxsize})& = c_{1}(\mathit{lysize} - 1) + (\mathit{lysize} - 2) \\
& \quad \cdot \bigl((c_{2} + c_{3} + c_{4} + c_{5} \cdot 7 + c_{6} + c_{7} \cdot t + c_{11} + c_{12} \cdot 2 + c_{13})\mathit{lxsize} \nonumber \\
& \quad \quad - 2(\frac{c_2}{2} + c_{3} + c_{4} + c_{5} \cdot 7 + c_{6} + c_{7} \cdot t + c_{9} + c_{11} + c_{12} \cdot 2 + c_{13}) \bigr) \nonumber \\
\label{eq:analyse-arlabeling-regionlabeling-2-3}
T_{worst}(\mathit{lysize},\mathit{lxsize})& = c_{1}\mathit{lysize} - c_{1}
 \\
& \quad + (c_{2} + c_{3} + c_{4} + c_{5}7 + c_{6} + c_{7}t + c_{11} + c_{12}2 + c_{13})\mathit{lxsize}
 \cdot \mathit{lysize}
 \nonumber \\
& \quad -2(c_{2} + c_{3} + c_{4} + c_{5}7 + c_{6} + c_{7}t + c_{11} + c_{12}2 + c_{13})\mathit{lxsize}
 \nonumber \\
& \quad - 2(\frac{c_2}{2} + c_{3} + c_{4} + c_{5}7 + c_{6} + c_{7}t + c_{9} + c_{11} + c_{12}2 + c_{13})\mathit{lysize}
 \nonumber \\
& \quad + 4(\frac{c_2}{2} + c_{3} + c_{4} + c_{5}7 + c_{6} + c_{7}t + c_{9} + c_{11} + c_{12}2 + c_{13})
 \nonumber
\end{align}
\end{subequations}

Durch Substitution der Konstanten $c$ mit $1$ folgt $T_{worst}(\mathit{lxsize},\mathit{lysize}) = 15 \mathit{lxsize}
 \cdot \mathit{lysize} + \mathit{lxsize} \cdot \mathit{lysize} \cdot t - 30 \mathit{lxsize} - 2 \mathit{lxsize} \cdot t
 -30 \mathit{lysize} - 2 \mathit{lysize} \cdot t + 61 + 4t$. Die Variable $t$ steht für die Laufzeitfunktion in
 \autoref{eq:analyse-arlabeling-neighbour}. Die Eingabemenge $\mathit{wk\_max}$ ist von $\mathit{lxsize}$ und
 $\mathit{lysize}$ abhänging und kann nicht mehr als $8192$ Einträge enthalten. Wird $t$ durch $5 \cdot 8192 + 24$
 substituiert, folgt die Laufzeitfunktion in \autoref{eq:analyse-arlabeling-regionlabeling-3}.
\begin{subequations}
\label{eq:analyse-arlabeling-regionlabeling-3}
\begin{align}
\label{eq:analyse-arlabeling-regionlabeling-3-1}
T_{worst}(\mathit{lxsize},\mathit{lysize})& =
 15\mathit{lxsize}\cdot\mathit{lysize} + \mathit{lxsize}\cdot\mathit{lysize}\cdot 40984 - 30\mathit{lxsize} \\
 & \quad - 2\mathit{lxsize}\cdot 40984 -30\mathit{lysize} - 2\mathit{lysize}\cdot 40984 + 61 + 4(40984) \nonumber \\
\label{eq:analyse-arlabeling-regionlabeling-3-2}
T_{worst}(\mathit{lxsize},\mathit{lysize})& =
 40999\mathit{lxsize}\cdot\mathit{lysize} -81998\mathit{lxsize} -81998\mathit{lysize} + 163997
\end{align}
\end{subequations}

Für $c = 41000$, $\mathit{lxsize}_{0} = 1$ und $\mathit{lysize}_{0} = 1$ ist die Wachstumsrate von
 $40999\mathit{lxsize}\cdot\mathit{lysize} -81998\mathit{lxsize} -81998\mathit{lysize}
 = O(\mathit{lxsize}\cdot\mathit{lysize})$.

Die erfassten Daten müssen nach der Regionenmarkierung in Schritt \ref{label-cleaning} von
 \autoref{alg:arlabeling-overview} aufbereitet werden. Zuerst werden mit \autoref{alg:sortlabels} die Werte in
 $\mathit{work}$ sortiert.
\begin{algorithm}[ht]
\caption{\textproc{arLabeling} (Sortiere Markierungen)}
\label{alg:sortlabels}
\begin{algorithmic}[1]
	\State $j \gets 1$
	\label{alg:sortlabels-j}
	\State $\mathit{wk} \gets \mathit{work}\left[0\right]$
	\label{alg:sortlabels-address}
	\For{$i \gets 1$ \textbf{to} $i \leq \mathit{wk\_max}$}
	\label{alg:sortlabels-loop-start}
		\If{$\mathit{wk} == i$}
			\State $\mathit{wk} \gets j$
			\State $j \gets j + 1$
			\label{alg:sortlabels-savelabel}
		\Else
		\label{alg:sortlabels-collision-start}
			\State $\mathit{wk} \gets \mathit{work}\left[\mathit{wk} - 1\right]$ \Comment Letzten Markierungswert $\mathit{work}$ zuweisen.
		\EndIf
		\label{alg:sortlabels-collision-end}
		\State $i \gets i + 1$
		\label{alg:sortlabels-inci}
		\State Inkrementiere $\mathit{wk}$
		\label{alg:sortlabels-incwk}
	\EndFor
	\label{alg:sortlabels-loop-end}
\end{algorithmic}
\end{algorithm}

Die aufsteigenden Markierungswerte in $\mathit{work}$ sind duch Kollisionen in
 \autoref{alg:arlabeling-neighbour-case2-1} durch kleinere Werte ersetzt worden. Dadurch ist eine Lücke im Interval
 $\left[1,2,3..n\right]$ der numerischen Werte entstanden. Diese Lücken werden durch \autoref{alg:sortlabels}
 entfernt, um wieder ein aufsteigendes Interval zu erhalten.

In Zeile \ref{alg:sortlabels-j} wird $j$ als Variable für die aktuelle Markierung initialisiert. In Zeile
 \ref{alg:sortlabels-address} wird die Adresse der ersten Markierung aus $\mathit{work}$ in $\mathit{wk}$ gespeichert.
 Die Schleife in Zeile \ref{alg:sortlabels-loop-start}--\ref{alg:sortlabels-loop-end} überprüft, ob die numerischen
 Werte ohne Lücken durch Kollisionen gespeichert sind. Dazu wird jeder Eintrag in $\mathit{work}$ mit der Laufvariable
 $i$ verglichen. Falls keine Kollisionen vorliegen, werden die Werte in $\mathit{work}$ immer mit $i$ übereinstimmen,
 da keine Lücken im Intervall vorhanden sind. In Zeile \ref{alg:sortlabels-savelabel} wird dann der Wert aus $j$ an die
 aktuelle Position von $\mathit{work}$ geschrieben und $j$ danach inkrementiert. Bei einer Kollision stimmt der Wert in
 $\mathit{work}$ nicht mit $i$ überein
 (Zeile \ref{alg:sortlabels-collision-start}--\ref{alg:sortlabels-collision-end}). In diesem Fall wird der Wert an der
 aktuellen Position von $\mathit{work}$ ausgelesen und dekrementiert. Dieser Wert wird als Index verwendet um den Wert
 des zuletzt zugewiesenen Markierungswert an die aktuelle Position von $\mathit{work}$ zu schreiben. Da die Variable
 $j$ in diesem Fall nicht inkrementiert wird, enthält sie den nächsten Wert des Markierungsintervals. In Zeile
 \ref{alg:sortlabels-inci} wird die Laufvariable $i$ um $1$ erhöht und in Zeile \ref{alg:sortlabels-incwk} die Adresse
 von $\mathit{work}$ inkrementiert. Das Beispiel in \autoref{fig:} illustriert das Verfahren. In
 \autoref{eq:analyse-arlabeling-sortlabels} ist die Laufzeitfunktion angegeben. Für $c_{1} = 4$, $c_{2} = 7$ und
 $\mathit{wk\_max}_{0} = 1$ ist die Wachstumsrate $7\mathit{wk\_max} - 3 = \Theta(\mathit{wk\_max})$.
\begin{subequations}
\label{eq:analyse-arlabeling-sortlabels}
\begin{align}
\label{eq:analyse-arlabeling-sortlabels-1}
T_{worst}(\mathit{wk\_max})& = c_{1} + c_{2}2 + c_{3}\mathit{wk\_max} \\
& \quad + (c_{4} + c_{10} + c_{11})(\mathit{wk\_max} - 1) + c_{8}(\mathit{wk\_max} - 1)3 \nonumber \\
\label{eq:analyse-arlabeling-sortlabels-2}
T_{worst}(\mathit{wk\_max})& = c_{8}\mathit{wk\_max}3 + (c_{3} + c_{4} + c_{10} + c_{11})\mathit{wk\_max}\\
& \quad - c_{8}3 - (c_{4} + c_{10} + c_{11}) + c_{1} + c_{2}2\nonumber
\end{align}
\end{subequations}


Das Verfahren in \autoref{alg:arlabeling-initlabelmemory} initialisiert den Speicher zur Berechnung des Flächeninhalts
 einer Region und den Speicher der Koordinaten des Mittelpunkts einer Region.
\begin{algorithm}[ht]
\caption{\textproc{arLabeling} (Regionenspeicher initialisieren)}
\label{alg:arlabeling-initlabelmemory}
\begin{algorithmic}[1]
	\State $\mathit{wlabel\_num} \gets j - 1$
	\label{alg:arlabeling-initlabelmemory-label-start}
	\State $\mathit{label\_num} \gets j - 1$
	\label{alg:arlabeling-initlabelmemory-label-end}
	\If{$\mathit{label\_num} == 0$}
	\label{alg:arlabeling-initlabelmemory-islabel-start}
		\State \textbf{return} $\mathit{l\_image}$
	\EndIf
	\label{alg:arlabeling-initlabelmemory-islabel-end}
	\State $\mathit{size} \gets \mathit{label\_num} \cdot$ \Call{sizeof}{$int$}
	\label{alg:arlabeling-initlabelmemory-initwarea-start}
	\For{$\mathit{size} > 0$}
		\State $\mathit{warea} \gets 0$
		\State Inkrementiere $\mathit{warea}$
		\State $\mathit{size} \gets \mathit{size} - 1$
	\EndFor
	\label{alg:arlabeling-initlabelmemory-initwarea-end}
	\State $\mathit{size} \gets \mathit{label\_num} \cdot 2 \cdot$ \Call{sizeof}{$float$}
	\label{alg:arlabeling-initlabelmemory-initwpos-start}
	\For{$\mathit{size} > 0$}
		\State $\mathit{wpos} \gets 0$
		\State Inkrementiere $\mathit{wpos}$
		\State $\mathit{size} \gets \mathit{size} - 1$
	\EndFor
	\label{alg:arlabeling-initlabelmemory-initwpos-end}
\end{algorithmic}
\end{algorithm}

Dazu wird in Zeile
 \ref{alg:arlabeling-initlabelmemory-label-start}--\ref{alg:arlabeling-initlabelmemory-label-end} der größte
 Markierungswert aus $\mathit{work}$ in $\mathit{wlabel\_num}$ und $\mathit{label\_num}$ gespeichert. Falls in der
 Überprüfung in Zeile
 \ref{alg:arlabeling-initlabelmemory-islabel-start}--\ref{alg:arlabeling-initlabelmemory-islabel-end}
 $\mathit{label\_num}$ keine Markierung enthält, wird das Verfahren abgebrochen und das Regionenbild $\mathit{l\_image}$
 zurückgegeben. Andernfalls wird in Zeile
 \ref{alg:arlabeling-initlabelmemory-initwarea-start}--\ref{alg:arlabeling-initlabelmemory-initwarea-end} der Speicher
 des Flächeinhalts $\mathit{warea}$ mit dem Wert $0$ initialisiert, indem über die Größe des Speichers iteriert wird.
 Der Speicher zur Berechnung des Mittelpunkts der Regionen wird in Zeile
 \ref{alg:arlabeling-initlabelmemory-initwpos-start}--\ref{alg:arlabeling-initlabelmemory-initwpos-end} nach dem
 gleichen Prinzip initialisiert. Die Laufzeitfunktion für Zeile
 \ref{alg:arlabeling-initlabelmemory-label-start}--\ref{alg:arlabeling-initlabelmemory-initwarea-end} von
 \autoref{alg:arlabeling-initlabelmemory} ist in \autoref{eq:analyse-arlabeling-sortlabels-initlabelmemory-1} angegeben.
\begin{subequations}
\label{eq:analyse-arlabeling-sortlabels-initlabelmemory-1}
\begin{align}
\label{eq:analyse-arlabeling-sortlabels-initlabelmemory-1-1}
T_{worst}(\mathit{size})& = c_{1}2 + c_{2}2 + c_{3} + c_{6}2 + c_{7}(\mathit{size} + 1) + (c_{8} + c_{9}
 + c_{10})\mathit{size} \\
\label{eq:analyse-arlabeling-sortlabels-initlabelmemory-1-2}
T_{worst}(\mathit{size})& = (c_{7} + c_{8} + c_{9} + c_{10})\mathit{size} + (c_{7} + c_{3}) + (c_{1} + c_{2} + c_{6})2
\end{align}
\end{subequations}

Für $c_{1} = 4$, $c_{2} = 12$ und $\mathit{size}_{0} = 1$ ist die Wachstumsrate
 $4\mathit{size} + 8 = \Theta(\mathit{size})$. Die Laufzeitfunktion für Zeile
 \ref{alg:arlabeling-initlabelmemory-floatsize}--\ref{alg:arlabeling-initlabelmemory-initwpos-end} ist in
 \autoref{eq:analyse-arlabeling-sortlabels-initlabelmemory-2} aufgeführt.
\begin{subequations}
\label{eq:analyse-arlabeling-sortlabels-initlabelmemory-2}
\begin{align}
\label{eq:analyse-arlabeling-sortlabels-initlabelmemory-2-1}
T_{worst}(\mathit{size})& = c_{12}3 + c_{13}(\mathit{size} + 1) + (c_{14} + c_{15} + c_{16})\mathit{size} \\
\label{eq:analyse-arlabeling-sortlabels-initlabelmemory-2-2}
T_{worst}(\mathit{size})& = (c_{13} + c_{14} + c_{15} + c_{16})\mathit{size} + c_{13} + c_{12}3
\end{align}
\end{subequations}

Die Wachstumsrate ist $4\mathit{size} + 4 = \Theta(\mathit{size})$ für $c_{1} = 4$, $c_{2} = 8$ und
 $\mathit{size}_{0} = 1$.

In \autoref{alg:arlabeling-calcregiondata} werden die Daten aller Regionenmarkierungen aufbereitet. Das Verfahren
 berechnet den Flächeninhalt, den Mittelpunkt und die Start- und Endkoordinaten für alle Regionen.
\begin{algorithm}[ht]
\caption{\textproc{arLabeling} (Berechne Regionendaten)}
\label{alg:arlabeling-calcregiondata}
\begin{algorithmic}[1]
    % for(i = 0; i < *label_num; i++) {
    %     wclip[i*4+0] = lxsize;
    %     wclip[i*4+1] = 0;
    %     wclip[i*4+2] = lysize;
    %     wclip[i*4+3] = 0;
    % }
	\For{$i \gets 0$ \textbf{to} $i < \mathit{label\_num}$}
	\label{alg:arlabeling-calcregiondata-wclip-start}
		\State $\mathit{wclip}\left[i \cdot 4 + 0\right] \gets \mathit{lxsize}$
		\State $\mathit{wclip}\left[i \cdot 4 + 1\right] \gets 0$
		\State $\mathit{wclip}\left[i \cdot 4 + 2\right] \gets \mathit{lysize}$
		\State $\mathit{wclip}\left[i \cdot 4 + 3\right] \gets 0$
		\State $i \gets i + 1$
	\EndFor
	\label{alg:arlabeling-calcregiondata-wclip-end}
    % for(i = 0; i < wk_max; i++) {
    %     j = work[i] - 1;
    %     warea[j]    += work2[i*7+0];
    %     wpos[j*2+0] += work2[i*7+1];
    %     wpos[j*2+1] += work2[i*7+2];
    %     if( wclip[j*4+0] > work2[i*7+3] ) wclip[j*4+0] = work2[i*7+3];
    %     if( wclip[j*4+1] < work2[i*7+4] ) wclip[j*4+1] = work2[i*7+4];
    %     if( wclip[j*4+2] > work2[i*7+5] ) wclip[j*4+2] = work2[i*7+5];
    %     if( wclip[j*4+3] < work2[i*7+6] ) wclip[j*4+3] = work2[i*7+6];
    % }
	\For{$i \gets 0$ \textbf{to} $i < \mathit{wk\_max}$}
	\label{alg:arlabeling-calcregiondata-work-start}
		\State $j \gets \mathit{work}\left[i\right] - 1$
		\label{alg:arlabeling-calcregiondata-label}
		\State $\mathit{warea}\left[j\right] \gets \mathit{warea}\left[j\right] + \mathit{work2}\left[i \cdot 7 + 0\right]$
		\State $\mathit{wpos}\left[j \cdot 2 + 0\right] \gets \mathit{wpos}\left[j \cdot 2 + 0\right] + \mathit{work2}\left[i \cdot 7 + 1\right]$
		\label{alg:arlabeling-calcregiondata-sumx}
		\State $\mathit{wpos}\left[j \cdot 2 + 1\right] \gets \mathit{wpos}\left[j \cdot 2 + 1\right] + \mathit{work2}\left[i \cdot 7 + 2\right]$
		\label{alg:arlabeling-calcregiondata-sumy}
		\If{$\mathit{wclip}\left[i \cdot 4 + 0\right] > \mathit{work2}\left[i \cdot 7 + 3\right]$}
		\label{alg:arlabeling-calcregiondata-startx-start}
			\State $\mathit{wclip}\left[i \cdot 4 + 0\right] \gets \mathit{work2}\left[i \cdot 7 + 3\right]$
		\EndIf
		\label{alg:arlabeling-calcregiondata-startx-end}

		\If{$\mathit{wclip}\left[i \cdot 4 + 1\right] < \mathit{work2}\left[i \cdot 7 + 4\right]$}
		\label{alg:arlabeling-calcregiondata-endx-start}
			\State $\mathit{wclip}\left[i \cdot 4 + 1\right] \gets \mathit{work2}\left[i \cdot 7 + 4\right]$
		\EndIf
		\label{alg:arlabeling-calcregiondata-endx-end}

		\If{$\mathit{wclip}\left[i \cdot 4 + 2\right] > \mathit{work2}\left[i \cdot 7 + 5\right]$}
		\label{alg:arlabeling-calcregiondata-y-start}
			\State $\mathit{wclip}\left[i \cdot 4 + 2\right] \gets \mathit{work2}\left[i \cdot 7 + 5\right]$
		\EndIf

		\If{$\mathit{wclip}\left[i \cdot 4 + 3\right] < \mathit{work2}\left[i \cdot 7 + 6\right]$}
			\State $\mathit{wclip}\left[i \cdot 4 + 3\right] \gets \mathit{work2}\left[i \cdot 7 + 6\right]$
		\EndIf
		\label{alg:arlabeling-calcregiondata-y-end}

		\State $i \gets i + 1$
	\EndFor
	\label{alg:arlabeling-calcregiondata-work-end}
    % for( i = 0; i < *label_num; i++ ) {
    %     wpos[i*2+0] /= warea[i];
    %     wpos[i*2+1] /= warea[i];
    % }
	\For{$i \gets 0$ \textbf{to} $\mathit{label\_num}$}
	\label{alg:arlabeling-calcregiondata-pos-start}
		\State $\mathit{wpos}\left[i \cdot 2 + 0\right] \gets \mathit{wpos}\left[i \cdot 2 + 0\right] / \mathit{warea}\left[i\right]$
		\State $\mathit{wpos}\left[i \cdot 2 + 1\right] \gets \mathit{wpos}\left[i \cdot 2 + 1\right] / \mathit{warea}\left[i\right]$
		\State $i \gets i + 1$
	\EndFor
	\label{alg:arlabeling-calcregiondata-pos-end}
	% *label_ref = work;
	% *area      = warea;
	% *pos       = wpos;
	% *clip      = wclip;
	% return( l_image );
	\State $\mathit{label\_ref} \gets \mathit{work}$
	\State $\mathit{area} \gets \mathit{warea}$
	\State $\mathit{pos} \gets \mathit{wpos}$
	\State $\mathit{clip} \gets \mathit{wclip}$
	\State \textbf{return} $\mathit{l\_image}$
\end{algorithmic}
\end{algorithm}

Dazu wird in Zeile \ref{alg:arlabeling-calcregiondata-wclip-start}--\ref{alg:arlabeling-calcregiondata-wclip-end} der
 Speicher der Start- und Endkoordinaten initialisiert, indem über die Menge der Regionen itertiert wird und die
 Initialwerte, $\mathit{lxsize}$ und $0$ für die Startkoordiante und $\mathit{lysize}$ und $0$ für die Endkoordinate,
 gespeichert werden. In Zeile
 \ref{alg:arlabeling-calcregiondata-work-start}--\ref{alg:arlabeling-calcregiondata-work-end} werden die Werte aus
 $\mathit{work2}$ aufbereitet, indem über die Anzahl der Regionen iteriert wird. In Zeile
 \ref{alg:arlabeling-calcregiondata-label} wird die Regionemarkierung ausgelesen und in $j$ gespeichert. Danach wird
 die Anzahl der Vordergrundpixel für Region $j$ aus $\mathit{work2}$ ausgelesen und in $\mathit{warea}$ gespeichert. In
 Zeile \ref{alg:arlabeling-calcregiondata-sumx}--\ref{alg:arlabeling-calcregiondata-sumy} werden die summierten $x$- und
 $y$-Koordinaten aus $\mathit{work2}$ ausgelsen und in $\mathit{wpos}$ gespeichert. Die Überprüfung des $x$-Werts in
 Zeile \ref{alg:arlabeling-calcregiondata-startx-start}--\ref{alg:arlabeling-calcregiondata-startx-end} sorgt dafür,
 dass die Startkoordinate soweit links wie möglich beginnt. Der $x$-Wert der Endkoordinate wird in Zeile
 \ref{alg:arlabeling-calcregiondata-endx-start}--\ref{alg:arlabeling-calcregiondata-endx-end} überprüft. Bei dieser
 Überprüfung soll der $x$-Wert der Endkoordinate soweit rechts wie möglich liegen. Zeile
 \ref{alg:arlabeling-calcregiondata-y-start}--\ref{alg:arlabeling-calcregiondata-y-end} wiederholen das Verfahren für
 den $y$-Wert der Start- und Endkoordinate. Danach wird in Zeile
 \ref{alg:arlabeling-calcregiondata-pos-start}--\ref{alg:arlabeling-calcregiondata-pos-end} die Position des
 Mittelpunkts einer Region berechnet, indem die aufsummierten Koordinaten einer Region durch die Anzahl der
  Vordergrundpixel in $\mathit{warea}$ geteilt wird. Zuletzt werden die lokalen Variablen in den Übergabeparametern
 gespeichert und das Regionenbild $\mathit{l\_image}$ an die aufrufende Methode zurückgegeben. An dieser Stelle ist
 \textproc{arLabeling} (\autoref{alg:arlabeling-overview}) abgeschlossen.

Die Laufzeitfunktion für Zeile
 \ref{alg:arlabeling-calcregiondata-wclip-start}--\ref{alg:arlabeling-calcregiondata-wclip-end} von
 \autoref{alg:arlabeling-calcregiondata} ist in \autoref{eq:analyse-arlabeling-calcregiondata-1} angegeben.
\begin{align}
\label{eq:analyse-arlabeling-calcregiondata-1}
T(\mathit{labelnum})& = c_{1} + (c_{1} + c_{6}) \mathit{labelnum} + (c_{2} + c_{3} + c_{4} + c_{5})(\mathit{labelnum} 4)
\end{align}

Die Wachstumsrate, für die Werte $c_{1} = 18$, $c_{2} = 19$ und $\mathit{labelnum}_{0} = 1$, ist
 $18\mathit{labelnum} + 1 = \Theta(\mathit{labelnum})$. Die Laufzeitfunktion von Zeile
 \ref{alg:arlabeling-calcregiondata-work-start}--\ref{alg:arlabeling-calcregiondata-work-end} ist in
 \autoref{eq:analyse-arlabeling-calcregiondata-2} aufgeführt. Für $c_{1} = 90$, $c_{2} = 91$ und
 $\mathit{wk\_max}_{0} = 1$ ist die Wachstumsrate $90\mathit{wk\_max} + 1 = \Theta(wk\_max)$.
\begin{subequations}
\label{eq:analyse-arlabeling-calcregiondata-2}
\begin{align}
\label{eq:analyse-arlabeling-calcregiondata-2-1}
T_{worst}(\mathit{wk\_max})& = c_{8} (\mathit{wk\_max} + 1) + c_{9}(\mathit{wk\_max}3) + c_{10}(\mathit{wk\_max} 7) \\
& \quad + (c_{11} + c_{12})(\mathit{wk\_max}11) + c_{25}\mathit{wk\_max} \nonumber \\
& \quad + (c_{13} + c_{14} + c_{16} + c_{17} + c_{19} + c_{20} + c_{22} + c_{23})(\mathit{wk\_max}7) \nonumber \\
\label{eq:analyse-arlabeling-calcregiondata-2-2}
T_{worst}(\mathit{wk\_max})& = c_{8} + (c_{8} + c_{25})\mathit{wk\_max} + c_{9}(\mathit{wk\_max}3)
 + c_{10}(\mathit{wk\_max} 7) \\
& \quad + (c_{11} + c_{12})(\mathit{wk\_max}11) \nonumber \\
& \quad + (c_{13} + c_{14} + c_{16} + c_{17} + c_{19} + c_{20} + c_{22} + c_{23})(\mathit{wk\_max}7) \nonumber
\end{align}	
\end{subequations}

Die letzte Laufzeitfunktion von \autoref{alg:arlabeling-calcregiondata} ist in
 \autoref{eq:analyse-arlabeling-calcregiondata-2} angegeben.
\begin{subequations}
\label{eq:analyse-arlabeling-calcregiondata-3}
\begin{align}
\label{eq:analyse-arlabeling-calcregiondata-3-1}
T(\mathit{labelnum})& = c_{27}(\mathit{labelnum} + 1) + (c_{28} + c_{29})(\mathit{labelnum} 9) \\
& \quad + c_{30}\mathit{labelnum} + (c_{32} + c_{33} + c_{34} + c_{35} + c_{36}) \nonumber \\
\label{eq:analyse-arlabeling-calcregiondata-3-2}
T(\mathit{labelnum})& = (c_{27} + c_{30})\mathit{labelnum} + (c_{28} + c_{29})(\mathit{labelnum} 9) \\
& \quad + (c_{27} + c_{32} + c_{33} + c_{34} + c_{35} + c_{36}) \nonumber
\end{align}	
\end{subequations}

Die Wachstumsrate entspricht $20\mathit{labelnum} + 6 = \Theta(\mathit{labelnum})$, für $c_{1} = 20$, $c_{2} = 21$ und
 $\mathit{labelnum}_{0} = 6$.

Die Laufzeitfunktion von \textproc{arLabeling} (\autoref{alg:arlabeling-overview}) ist in
 \autoref{eq:analyse-arlabeling-all-1} angegeben und besteht aus der Initialisierung (\autoref{alg:arlabeling-init1}
 und \autoref{alg:arlabeling-init2}), der Regionenmarkierung
 (\autoref{alg:arlabeling-regionlabeling}--\autoref{alg:arlabeling-neighbour-case5}) und der Aufbereitung und
 Speicherung der Daten (\autoref{alg:sortlabels}--\autoref{alg:arlabeling-calcregiondata}).
\begin{align}
\label{eq:analyse-arlabeling-all-1}
T_{worst}(\mathit{lxsize},\mathit{lysize})& =
 (39)
 + (6\mathit{lxsize} + 1)
 + (8\mathit{lysize} + 9)
\\
& \quad
 + (40999\mathit{lxsize} \cdot \mathit{lysize} - 81998\mathit{lxsize} - 81998\mathit{lysize} + 163997)
\nonumber \\
& \quad
 + (7\mathit{wk\_max} - 3)
 + (4\mathit{size} + 8)
 + (4\mathit{size}' + 4)
\nonumber \\
& \quad
 + (18\mathit{label\_num} + 1)
 + (90\mathit{wk\_max} + 1)
 + (20\mathit{label\_num} + 6)
\nonumber
\end{align}

Die Eingabemenge des Algorithmus ist das Bildsignal $I$, dass aus $\mathit{lxsize} \cdot \mathit{lysize}$ besteht. Die
 Variable $\mathit{wk\_max}$ ist auf $8192$ Einträge begrenzt. Die Menge $\mathit{size}$ bei
 \autoref{alg:arlabeling-initlabelmemory} ist abhängig von $\mathit{labelnum}$, die bei einer Bildgröße von
 $\mathit{lxsize} \cdot \mathit{lysize}$ maximal $\bigl\lceil\frac{\mathit{lxsize}}{2}\bigr\rceil
 \cdot \bigl\lceil\frac{\mathit{lysize}}{2}\bigr\rceil$ Regionenmarkierungen anlgegen kann.
 $\mathit{int}$ und $\mathit{float}$ sind jeweils $4$ Byte groß, sodaß
 $\mathit{size} = \bigl\lceil\frac{\mathit{lxsize}}{2}\bigr\rceil \cdot \bigl\lceil\frac{\mathit{lysize}}{2}\bigr\rceil
 \cdot 4$ und $\mathit{size}' = \bigl\lceil\frac{\mathit{lxsize}}{2}\bigr\rceil
 \cdot \bigl\lceil\frac{\mathit{lysize}}{2}\bigr\rceil \cdot 2 \cdot 4$.
Dadurch wird \autoref{eq:analyse-arlabeling-all-1} zu \autoref{eq:analyse-arlabeling-all-2} überführt.
\begin{subequations}
\label{eq:analyse-arlabeling-all-2}
\begin{align}
\label{eq:analyse-arlabeling-all-2-1}
T_{worst}(\mathit{lxsize},\mathit{lysize})& =
O(\mathit{lxsize}\cdot\mathit{lysize})
+ \Theta(\mathit{lysize})
+ \Theta(4\frac{\mathit{lxsize}}{2}\cdot\frac{\mathit{lysize}}{2})
\\
& \quad
+ \Theta(2\cdot4\frac{\mathit{lxsize}}{2}\cdot\frac{\mathit{lysize}}{2})
+ 2\Theta(\frac{\mathit{lxsize}}{2}\cdot\frac{\mathit{lysize}}{2})
+ 2\Theta(8192)
+ \Theta(1)
\nonumber \\
\label{eq:analyse-arlabeling-all-2-2}
T_{worst}(\mathit{lxsize},\mathit{lysize})& =
O(\mathit{lxsize}\cdot\mathit{lysize})
+ \Theta(\mathit{lysize})
+ \Theta(\mathit{lxsize}\cdot\mathit{lysize})
\\
& \quad
+ 2\Theta(\mathit{lxsize}\cdot\mathit{lysize})
+ \frac{1}{2}\Theta(\mathit{lxsize}\cdot\mathit{lysize})
+ 2\Theta(8192)
+ \Theta(1)
\nonumber
\end{align}
\end{subequations}

Das Wachstum der Laufzeitfunktion \autoref{eq:analyse-arlabeling-all-2-2} ist $O(\mathit{lxsize\cdot\mathit{lysize}})$.
% \begin{equation*}
% \frac{1}{2}82041\mathit{lxsize} \cdot \mathit{lysize} -81992\mathit{lxsize} -81990\mathit{lysize}
%  = \Theta(\mathit{lxsize}\cdot\mathit{lysize})
% \end{equation*}
% für
% \begin{equation*}
% c_{1} = \frac{82041\mathit{lxsize}\cdot\mathit{lysize} -163984\mathit{lxsize}
%  -163980\mathit{lysize}}{\mathit{lxsize}\cdot\mathit{lysize}} \text{,}
% \end{equation*}
% \begin{equation*}
% c_{2} = \frac{82041\mathit{lxsize}\cdot\mathit{lysize} -163984\mathit{lxsize}
%  -163980\mathit{lysize}}{2\mathit{lxsize}\cdot\mathit{lysize}} \text{,}
% \end{equation*}
% $\mathit{lxsize}_{0} = 1$ und $\mathit{lysize}_{0} = 1$.
% subsubsection regionenmarkierung (end)

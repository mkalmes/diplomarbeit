\subsubsection{Distanz} % (fold)
\label{sub:distanz}

Die Datenstruktur \textproc{distance} dient zur Speicherung der Länge von Liniensegmenten und der Speicherung eines
 Index, um eine Linie in einem Speicherblock anzusprechen (\autoref{alg:distance}).
\begin{algorithm}[!ht]\small
\caption{\textproc{distance}}
\label{alg:distance}
	\begin{algorithmic}[1]
		\State $\mathit{distance}$
		\State $\mathit{index}$
	\end{algorithmic}
\end{algorithm}


Wie bei \gls{edgel} oder Liniensegmenten, kann \textproc{distance} in einem Speicherbereich hinterlegt werden und ist
 in \autoref{alg:datastructure-distancepool} dargestellt.
\begin{algorithm}[!ht]\small
\caption{\textproc{distancePool} (Datenstruktur)}
\label{alg:datastructure-distancepool}
\begin{algorithmic}[1]
	\State $\mathit{data}[N]$
	\Comment Anzahl der Einträge
	\State $\mathit{count}$
\end{algorithmic}
\end{algorithm}

Auch bei diesem Verfahren wird ein Array mit festert Größe $N$ und einer Zählvariable verwendet. Im Gegensatz zu den
 anderen Speicherbereichen wird bei \textproc{distance} auf einen Speichervorrat verzichtet. Die Verwaltung eines
 Speichervorrat wird manuel durchgeführt.

% Mit \autoref{alg:distancepool-adddistance}
% \begin{algorithm}[!ht]
\caption{\textproc{addDistance}}
\label{alg:distancepool-adddistance}
\begin{algorithmic}[1]
	\Require $p,d$
	\If{$\lnot p$}
	\Cost{$c_{1}$}{$1$}
		\State \textbf{return}
		\Cost{$c_{2}$}{$1$}
	\EndIf
	\If{$\lnot \left(\mathit{p.count} < N\right)$}
	\Cost{$c_{4}$}{$3$}
		\State \textbf{return} \Comment Speicher voll
		\Cost{$c_{5}$}{$1$}
	\EndIf
	\State $c \gets \mathit{p.count}$
	\Cost{$c_{7}$}{$2$}
	\State $\mathit{p.data}[c] \gets d$
	\Cost{$c_{8}$}{$3$}
	\State $\mathit{p.count} \gets c + 1$
	\Cost{$c_{9}$}{$3$}
\end{algorithmic}
\end{algorithm}

%  und \autoref{alg:distancepool-removedistance}
% \begin{algorithm}[!ht]
\caption{\textproc{removeDistance}}
\label{alg:distancepool-removedistance}
\begin{algorithmic}[1]
	\Require $p,i$
	\If{$\lnot p$}
	\Cost{$c_{1}$}{$1$}
		\State \textbf{return}
		\Cost{$c_{2}$}{$1$}
	\EndIf
	\State $c \gets \mathit{p.count}$
	\Cost{$c_{4}$}{$2$}
	\If{$i < 0 \lor i > c$}
	\Cost{$c_{5}$}{$3$}
	\label{alg:distancepool-isvalid-start}
		\State \textbf{return}
		\Cost{$c_{6}$}{$1$}
	\EndIf
	\label{alg:distancepool-isvalid-end}
	\If{$c > i + 1$}
	\Cost{$c_{8}$}{$2$}
		\State \Call{memmove}{$\mathit{p.data}[\mathit{i}], \mathit{p.data}[\mathit{i} + 1], \left(c - \mathit{i}
		 + 1\right) \cdot \textproc{sizeof}(d)$}
		\Cost{$c_{9}$}{$8 + \Theta(1)$}
	\EndIf
\end{algorithmic}
\end{algorithm}

%  können Distanzwerte dem Speicherbereich hinzugefügt und entfernt werden. Die Laufzeit zum hinzufügen ist dabei
%  konstant. Bei \autoref{alg:distancepool-removedistance} ist im besten Fall der Index $i$ ausserhalb des Bereichs der
%  gespeicherten Daten (Zeile \ref{alg:distancepool-isvalid-start}--\ref{alg:distancepool-isvalid-end}), und die Laufzeit
%  somit $T_{best}=\Theta(1)$. Im schlechtesten Fall müssen mit \textproc{memmove} $n-1$ Einträge an Position $0$
%  verschoben werden. Die Laufzeit beträgt dann
%  $T_{worst}=\Theta(n)$ (Vgl. \autoref{sub:datenstruktur-edgels}, S. \pageref{sub:datenstruktur-edgels-memmove}).

Mit \autoref{alg:distancepool-adddistance}
\begin{algorithm}[!ht]
\caption{\textproc{addDistance}}
\label{alg:distancepool-adddistance}
\begin{algorithmic}[1]
	\Require $p,d$
	\If{$\lnot p$}
	\Cost{$c_{1}$}{$1$}
		\State \textbf{return}
		\Cost{$c_{2}$}{$1$}
	\EndIf
	\If{$\lnot \left(\mathit{p.count} < N\right)$}
	\Cost{$c_{4}$}{$3$}
		\State \textbf{return} \Comment Speicher voll
		\Cost{$c_{5}$}{$1$}
	\EndIf
	\State $c \gets \mathit{p.count}$
	\Cost{$c_{7}$}{$2$}
	\State $\mathit{p.data}[c] \gets d$
	\Cost{$c_{8}$}{$3$}
	\State $\mathit{p.count} \gets c + 1$
	\Cost{$c_{9}$}{$3$}
\end{algorithmic}
\end{algorithm}

können Distanzwerte dem Speicherbereich hinzugefügt werden. Die Laufzeit zum hinzufügen ist dabei
 konstant.

Durch \autoref{alg:distancepool-distancecount}
\begin{algorithm}[!ht]
\caption{\textproc{getDistanceCount}}
\label{alg:distancepool-distancecount}
\begin{algorithmic}[1]
	\Require $p$
	\If{$\lnot p$}
	\Cost{$c_{1}$}{$1$}
		\State \textbf{return}
		\Cost{$c_{2}$}{$1$}
	\EndIf
	\State \textbf{return} $\mathit{p.count}$
	\Cost{$c_{4}$}{$2$}
\end{algorithmic}
\end{algorithm}

 kann die Anzahl der gespeicherten Distanzwerte in konstanter Zeit ausgelesen werden. \textproc{freePool}
(\autoref{alg:distancepool-freepool}) löscht den Speicherbereich in konstanter Zeit.
\begin{algorithm}[!ht]\small
\caption{\textproc{freePool}}
\label{alg:distancepool-freepool}
\begin{algorithmic}[1]
	\Require $p$
	\If{$\lnot p$}
	\Cost{$c_{1}$}{$1$}
		\State \textbf{return}
		\Cost{$c_{2}$}{$1$}
	\EndIf
	\State $\mathit{p.count} \gets 0$
	\Cost{$c_{4}$}{$2$}
\end{algorithmic}
\end{algorithm}


Die Laufzeit der Methoden für \textproc{distance} sind alle konstant und somit $T(n)=\Theta(n)$.

% subsubsection distanz (end)

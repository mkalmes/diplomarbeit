\subsubsection{Liniensegmente} % (fold)
\label{sub:datenstruktur-liniensegmente}

Die Datenstruktur eines Liniensegments und die Methoden zum hinzufügen, löschen und freigeben des Speichers sind nach
 dem Vorbild des Edgelspeichers aufgebaut. Die Datenstruktur eines Liniensegments ist in
 \autoref{alg:datastructure-linesegment} definiert.
\begin{algorithm}[!ht]\small
\caption{\textproc{lineSegment}}
\label{alg:datastructure-linesegment}
	\begin{algorithmic}[1]
		\State $\mathit{start}$
		\State $\mathit{end}$
		\State $\mathit{slope}$
		\State $\mathit{supportCount}$
		\State $\mathit{remove}$
		\State $\mathit{startCorner}$
		\State $\mathit{endCorner}$
		\State $\mathit{support}[\mathit{MAXEDGELS}]$
	\end{algorithmic}
\end{algorithm}

Eine Linie besteht aus den \glspl{edgel} $\mathit{start}$ und $\mathit{end}$, die den Start- und Endpunkt der Linie
 darstellen. Die Variable $\mathit{slope}$ enthält die Orientierung des Liniensegments, während die Variable
 $\mathit{supportCount}$ die Anzahl der unterstützenden \glspl{edgel} der Linie speichert. $\mathit{remove}$,
 $\mathit{startCorner}$ und $\mathit{endCorner}$ sind boolesche Variablen. $\mathit{remove}$ dient im späteren Verlauf
 zur Erkennung, ob ein Liniensegment gelöscht werden muss. Wenn eine Linie einen Eckpunkt am Anfang oder am Ende
 besitzt, wird dies in den Variablen $\mathit{startCorner}$ und $\mathit{endCorner}$ festgehalten.Die letzte Variable
 $\mathit{support}$ dient zur Speicherung von \glspl{edgel}, die eine Linienhypothese unterstützen. Die Lese- und
 Schreibzugriffe auf die Datenstruktur sind konstant.

Mit \autoref{alg:linesegmentaddedgel} wird ein Unterstützungsedgel zu einem Liniensegment hinzugefügt. Das Verfahren
 benötigt dazu das Liniensegment $l$, den \gls{edgel} $e$ und die Position $\mathit{pos}$, an die der \gls{edgel}
 gespeichert wird.
\begin{algorithm}[ht]
\caption{\textproc{addEdgel}}
\label{alg:linesegmentaddedgel}
\begin{algorithmic}[1]
	\Require $l, e, \mathit{pos}$
	\If{$\mathit{pos} > \mathit{MAXEDGELS} - 1$}
		\State \textbf{return}
	\EndIf
	\State $l(support[\mathit{pos}]) \gets e$
\end{algorithmic}
\end{algorithm}

In Zeile \ref{alg:linesegmentaddedgel-hasvalidrange} wird überprüft, ob genügend Speicherplatz für ein \gls{edgel} zur
 Verfügung steht. Falls dem nicht so ist, wird in Zeile \ref{alg:linesegmentaddedgel-notvalidrange} das Verfahren
 beendet. Andernfalls, wenn genügend Speicherplatz vorhanden ist, wird in Zeile
 \ref{alg:linesegmentaddedgel-storeedgel} der \gls{edgel} im Liniensegment gespeichert. Die Laufzeitfunktion des
 Verfahrens ist im schlechtesten Fall $T(n) = 5$.

Die Methode \textproc{isOrientationCompatible} untersucht, ob zwei Liniensegmente $\mathit{left}$ und $\mathit{right}$
 fast parallel zueinander stehen (\autoref{alg:linesegmentisorientationcompatible}).
\begin{algorithm}[!ht]
\caption{\textproc{isOrientationCompatible}}
\label{alg:linesegmentisorientationcompatible}
\begin{algorithmic}[1]
	\Require $\mathit{left}, \mathit{right}$
	\State \textbf{return} \Call{dotProduct}{$\mathit{left.slope}, \mathit{right.slope}$} $> 0.92$
\end{algorithmic}
\end{algorithm}

Dazu wird mithilfe von \textproc{dotProduct} die Orientierung berechnet. Wenn die Orientierung im Bereich von
 $(0.92,1]$ liegt, wird als Ergebnis wahr zurückgeliefert. Das bedeutet, dass die Orientierung der Linien im Bereich
 von $0^\circ$ bis $\sim 23^\circ$ liegt und die Linien als parallel betrachtet werden. Ansonsten wird als Ergebnis
 falsch zurückgegeben, was bedeutet, dass die Linien nicht parallel sind. Die Laufzeit von
 \autoref{alg:linesegmentisorientationcompatible} ist $T(n) = 10$.

Mit \autoref{alg:isedgelnearline} wird der Abstand eines \gls{edgel} zu einem Liniensegment berechnet.
\begin{algorithm}[ht]
\caption{\textproc{isEdgelNearLine}}
\label{alg:isedgelnearline}
\begin{algorithmic}[1]
	\Require $l,e$
	\If{$\lnot$ \Call{isCompatible}{$\mathit{l.start,e}$}}
		\State \textbf{return FALSE} 
	\EndIf
	\State $a \gets \mathit{l.end.coordinate.x} - \mathit{l.start.coordinate.x}$
	\State $b \gets \mathit{l.end.coordinate.y} - \mathit{l.start.coordinate.y}$
	\State $c \gets$ \Call{sqrtf}{$(a \cdot a)+(b \cdot b)$}
	\State $\mathit{AB1} \gets \mathit{l.end.coordinate.x} - \mathit{l.start.coordinate.x}$
	\State $\mathit{AC2} \gets \mathit{e.coordinate.y} - \mathit{l.start.coordinate.y}$
	\State $\mathit{AC1} \gets \mathit{e.coordinate.x} - \mathit{l.start.coordinate.x}$
	\State $\mathit{AB2} \gets \mathit{l.end.coordinate.y} - \mathit{l.start.coordinate.y}$
	\State $\mathit{crossproduct} \gets (\mathit{AB1} \cdot \mathit{AC2}) - (\mathit{AC1} \cdot \mathit{AB2})$
	\State $\mathit{distance} \gets$ \Call{ABS}{$\mathit{crossproduct}/c$}
	\State \textbf{return} $\mathit{distance} < 0.75$
\end{algorithmic}
\end{algorithm}

Das Verfahren benötigt dazu ein Liniensegment $l$ und ein \gls{edgel} $e$. In Zeile
 \ref{alg:isedgelnearline-iscompatible} wird überprüft, ob die Orientierung des \gls{edgel} kompatibel mit der
 Orientierung des Liniensegments ist. Wenn dies der Fall ist, wird das Verfahren fortgesetzt. Andernfalls wird das
 Verfahren in Zeile \ref{alg:isedgelnearline-notcompatible} abgebrochen. In Zeile
 \ref{alg:isedgelnearline-distance-start}--\ref{alg:isedgelnearline-distance-end} wird die Länge des Abstands der
 Endpunkte der Linie berechnet und in lokalen Variablen gespeichert. Im Anschluss wird in Zeile
 \ref{alg:isedgelnearline-pointline-start}--\ref{alg:isedgelnearline-pointline-end} der Abstand des \gls{edgel} zur
 Linie berechnet. Das Präprozessor Makro \textproc{ABS} berechnet den absoluten Betrag der Distanz in konstanter Zeit
 (Vgl. \autoref{alg:abs}).
\begin{algorithm}[!ht]
\caption{\textproc{ABS}}
\label{alg:abs}
\begin{algorithmic}[1]
	\Require $a$
	\If{$a < 0$}
	\Cost{$c_{1}$}{$1$}
		\State \textbf{return} $-a$
		\Cost{$c_{2}$}{$1$}
	\Else
		\State \textbf{return} $a$
		\Cost{$c_{4}$}{$1$}
	\EndIf
\end{algorithmic}
\end{algorithm}

In Zeile \ref{alg:isedgelnearline-return} wird als Rückgabewert, abhängig vom Vergleich der Distanz, wahr oder falsch
 zurückgegeben. Bleibt der Abstand des \gls{edgel} zur Linie unter $0.75$ wird wahr an die aufrufende Methode
 zurückgegeben. Ansonsten, wenn der Abstand größer ist, wird falsch zurückgegeben. Die Laufzeitfunktion von
 \autoref{alg:isedgelnearline} ist im schlechtesten Fall $T(n) = 71$.

Mit \textproc{intersection} wird der Schnittpunkt zweier Linien berechnet. Dazu benötigt das Verfahren in
 \autoref{alg:linesegmenintersection} eine linke und eine rechte Linie.
\begin{algorithm}[ht]
\caption{\textproc{intersection}}
\label{alg:linesegmenintersection}
\begin{algorithmic}[1]
	\Require $\mathit{left}, \mathit{right}$
	\State $\mathit{intersection} \gets infty$
	% const float x1 = left.start.coordinate.x;
	% const float y1 = left.start.coordinate.y;
	% const float x2 = left.end.coordinate.x;
	% const float y2 = left.end.coordinate.y;
	\State $\mathit{x1} \gets \mathit{left.start.coordinate.x}$
	\State $\mathit{y1} \gets \mathit{left.start.coordinate.y}$
	\State $\mathit{x2} \gets \mathit{left.end.coordinate.x}$
	\State $\mathit{y2} \gets \mathit{left.end.coordinate.y}$
	% const float x3 = right.start.coordinate.x;
	% const float y3 = right.start.coordinate.y;
	% const float x4 = right.end.coordinate.x;
	% const float y4 = right.end.coordinate.y;
	\State $\mathit{x3} \gets \mathit{right.start.coordinate.x}$
	\State $\mathit{y3} \gets \mathit{right.start.coordinate.y}$
	\State $\mathit{x4} \gets \mathit{right.end.coordinate.x}$
	\State $\mathit{y4} \gets \mathit{right.end.coordinate.y}$
	% const float numerator	= ((x4 - x3) * (y1 - y3)) - ((y4 - y3) * (x1 - x3));
	\State $\mathit{numerator} \gets ((\mathit{x4} - \mathit{x3}) \cdot (\mathit{y1} - \mathit{y3})) - ((\mathit{y4} - \mathit{y3}) \cdot (\mathit{x1} - \mathit{x3}))$
	% const float denumerator	= ((y4 - y3) * (x2 - x1)) - ((x4 - x3) * (y2 - y1));
	\State $\mathit{denumerator} \gets ((\mathit{y4} - \mathit{y3}) \cdot (\mathit{x2} - \mathit{x1})) - ((\mathit{x4} - \mathit{x3}) \cdot (\mathit{y2} - \mathit{y1}))$
	% const float u_a			= numerator / denumerator;
	\State $\mathit{u\_a} \gets \mathit{numerator} / \mathit{denumerator}$
	% 
	% intersection.x = x1 + u_a * (x2 - x1);
	% intersection.y = y1 + u_a * (y2 - y1);
	\State $\mathit{intersection.x} \gets \mathit{x1} + \mathit{u\_a} \cdot (\mathit{x2} - \mathit{x1})$
	\State $\mathit{intersection.y} \gets \mathit{y1} + \mathit{u\_a} \cdot (\mathit{y2} - \mathit{y1})$
	% 
	% return intersection;
	\State \textbf{return} $\mathit{intersection}$
\end{algorithmic}
\end{algorithm}

 In Zeile \ref{alg:linesegmenintersection-var-start}--\ref{alg:linesegmenintersection-var-end} werden die Punkte der
 Linienkoordinaten in lokalen Variablen gespeichert. Danach wird in Zeile
 \ref{alg:linesegmenintersection-intersect-start}--\ref{alg:linesegmenintersection-intersect-end} der Schnittpunkt der
 beiden Linie berechnet und in Zeile \ref{alg:linesegmenintersection-return} an die aufrufende Methode zurückgegeben.
 Die Berechnung des Schnittpunktes zweier Linien erfolgt in konstanter Zeit. Die Laufzeitfunktion ist $T(n) = 62$.

Die Datenstruktur eines Speichervorrats für Linien in \autoref{alg:datastructure-linesegmentpool} besteht aus einem
 Array $\mathit{data}$ mit der festen Größe $N$ und einer Zählvariablen $\mathit{count}$.
\begin{algorithm}[!ht]\small
\caption{\textproc{lineSegmentPool} (Datenstruktur)}
\label{alg:datastructure-linesegmentpool}
\begin{algorithmic}[1]
	\State $\mathit{data}[N]$
	\Comment Anzahl der Einträge
	\State $\mathit{count}$
\end{algorithmic}
\end{algorithm}

Der Speichervorrat für Linien in \autoref{alg:datastructure-linesegmentpoolimplementation} besteht wiederum aus einem
 Array $\mathit{data}$ mit der Anzahl $S$ der zur Verfügung stehenden Speicherblöcke.
\begin{algorithm}[ht]
\caption{\textproc{lineSegmentPool} (Implementierung)}
\label{alg:datastructure-linesegmentpoolimplementation}
\begin{algorithmic}[1]
	\State $\mathit{data}[S]$
	\Comment Anzahl der Pools
	\State $\mathit{pool}$
\end{algorithmic}
\end{algorithm}

Der Zeiger von $\mathit{data}$ wird in der Variablen $\mathit{pool}$ gespeichert. Der Zugriff auf die Datenstruktur
 erfolgt in konstanter Zeit.

Mehrere Speicherblöcke können mit \autoref{alg:linepool-getmemorypools} angefordert werden und mit
\begin{algorithm}[!ht]
\caption{\textproc{getMemoryPools}}
\label{alg:linepool-getmemorypools}
\begin{algorithmic}[1]
	\Require $n$
	\If{$\mathit{data} + S - \mathit{pool} \geq n$}
	\label{alg:linepool-getmemorypools-checkpoolsize}
		\State $\mathit{pool} \gets \mathit{pool} + n$
		\State \textbf{return} $\mathit{pool} - n$
	\Else
		\State \textbf{return} $\mathit{NULL}$
	\EndIf
\end{algorithmic}
\end{algorithm}

 \autoref{alg:linepool-getmemorypool} wird ein Speicherblock angefordert.
\begin{algorithm}[!ht]\small
\caption{\textproc{getMemoryPool}}
\label{alg:linepool-getmemorypool}
\begin{algorithmic}[1]
	\State $p \gets$ \Call{getmemorypools}{1}
	\Cost{$c_{1}$}{$1+7$}
	\State \textbf{return} $p$
	\Cost{$c_{2}$}{$1$}
\end{algorithmic}
\end{algorithm}

Der Aufbau der Verfahren entspricht den Verfahren des Speichervorrats für \glspl{edgel}
 (Vgl. \autoref{alg:edgelpool-getmemorypools} und \autoref{alg:edgelpool-getmemorypool}).
Die Laufzeitfunktion von \autoref{alg:linepool-getmemorypools} ist im schlechtesten Fall $T(n) = 7$ und die
 Laufzeitfunktion von \autoref{alg:linepool-getmemorypool} ist $T(n) = 9$.

Um eine Linie zu einem Speicherblock hinzuzufügen, wird \autoref{alg:linesegmentpool-addline} verwendet.
\begin{algorithm}[!ht]\small
\caption{\textproc{addLineSegment}}
\label{alg:linesegmentpool-addline}
\begin{algorithmic}[1]
	\Require $p,l$
	\If{$\lnot p$}
	\Cost{$c_{1}$}{$1$}
	\label{alg:linesegmentpool-addline-validpointer-start}
		\State \textbf{return}
		\Cost{$c_{2}$}{$1$}
	\EndIf
	\label{alg:linesegmentpool-addline-validpointer-end}
	\If{$\lnot \left(\mathit{p.count} < N\right)$}
	\Cost{$c_{4}$}{$2$}
	\label{alg:linepool-addline-checkspace-start}
		\State \textbf{return} \Comment Speicher voll
		\Cost{$c_{5}$}{$1$}
	\EndIf
	\label{alg:linesegmentpool-addline-checkspace-end}
	\State $c \gets \mathit{p.count}$
	\Cost{$c_{7}$}{$2$}
	\label{alg:linesegmentpool-addline-add-start}
	\State $\mathit{p.data}[c] \gets l$
	\Cost{$c_{8}$}{$3$}
	\State $\mathit{p.count} \gets c + 1$
	\Cost{$c_{9}$}{$3$}
	\label{alg:linesegmentpool-addline-add-end}
\end{algorithmic}
\end{algorithm}

Es wird ein Zeiger $p$ auf den Speicherblock, sowie eine Linie $l$ übergeben. Wenn es sich um einen gültigen Zeiger $p$
 handelt, und genügend freier Speicherplatz für eine weitere Linie vorhanden ist, wird in Zeile
 \ref{alg:linesegmentpool-addline-add-start}--\ref{alg:linesegmentpool-addline-add-end} die Linie hinzugefügt und die
 Zählvariable inkrementiert. Die Laufzeitfunktion ist im schlechtesten Fall $T(n) = 12$.

Zum auslesen einer Linie aus dem Speicherblock, wird \autoref{alg:linepool-getline} verwendet.
\begin{algorithm}[!ht]
\caption{\textproc{getLineSegment}}
\label{alg:linepool-getline}
\begin{algorithmic}[1]
	\Require $p,i$
	\If{$\lnot p$}
	\label{alg:linepool-getline-validpointer-start}
		\State \textbf{return}
	\EndIf
	\label{alg:linepool-getline-validpointer-end}
	\State $c \gets \mathit{p.count}$
	\If{$\lnot \left(c > i\right)$}
	\label{alg:linepool-getline-validrange-start}
		\State \textbf{return}
	\EndIf
	\label{alg:linepool-getline-validrange-end}
	\State \textbf{return} $\mathit{p.data}[i]$
	\label{alg:linepool-getline-returnline}
\end{algorithmic}
\end{algorithm}

Als Parameter werden ein Zeiger $p$ und ein Index $i$ übergeben. Der Index gibt an, welche Linie aus dem Block
 ausgelesen werden soll. In Zeile \ref{alg:linepool-getline-validrange-start} wird geprüft, ob der Index sich innerhalb
 der Grenzen der gespeicherten Linien befindet. Wenn dies der Fall ist, wird in Zeile
 \ref{alg:linepool-getline-returnline} die Linie in konstanter Zeit zurückgegeben. Die Laufzeitfunktion ist im schlechtesten Fall $T(n) = 8$.

Mit \autoref{alg:linesegment-removeline}
\begin{algorithm}[!ht]
\caption{\textproc{removeLine}}
\label{alg:linesegment-removeline}
\begin{algorithmic}[1]
	\Require $p,i$
	\If{$\lnot p$}
	\Cost{$c_{1}$}{$1$}
		\State \textbf{return}
		\Cost{$c_{2}$}{$1$}
	\EndIf
	\State $c \gets \mathit{p.count}$
	\Cost{$c_{4}$}{$2$}
	\If{$i < 0 \lor i > c$}
	\Cost{$c_{5}$}{$3$}
	\label{alg:distancepool-isvalid-start}
		\State \textbf{return}
		\Cost{$c_{6}$}{$1$}
	\EndIf
	\label{alg:distancepool-isvalid-end}
	\If{$c > i + 1$}
	\Cost{$c_{8}$}{$2$}
		\State \Call{memmove}{$\mathit{p.data}[\mathit{i}], \mathit{p.data}[\mathit{i} + 1], \left(c - \mathit{i}
		 + 1\right) \cdot \textproc{sizeof}(d)$}
		\Cost{$c_{9}$}{$8 + \Theta(1)$}
	\EndIf
\end{algorithmic}
\end{algorithm}

können Liniensegmente aus dem Speicherbereich wieder gelöscht werden. Die Laufzeitfunktion ist im schlechtesten Fall
 $T(n) = 17$ und somit konstant.

Mit \autoref{alg:linepool-resetmemorypool} werden die Einträge im Speicherblock gelöscht.
\begin{algorithm}[!ht]
\caption{\textproc{resetMemoryPool}}
\label{alg:linepool-resetmemorypool}
\begin{algorithmic}[1]
	\Require $p$
	\If{$\lnot p$}
	\Cost{$c_{1}$}{$1$}
	\label{alg:linepool-resetmemorypool-validpointer-start}
		\State \textbf{return}
		\Cost{$c_{2}$}{$1$}
	\EndIf
	\label{alg:linepool-resetmemorypool-validpointer-end}
	\State $\mathit{p.count} \gets 0$
	\Cost{$c_{4}$}{$2$}
	\label{alg:linepool-resetmemorypool-reset}
\end{algorithmic}
\end{algorithm}

Dazu wird der Zeiger $p$ auf den Speicherblock übergeben und in Zeile
 \ref{alg:linepool-resetmemorypool-validpointer-start}--\ref{alg:linepool-resetmemorypool-validpointer-end} überprüft.
 Wenn es sich um einen gültigen Zeiger handelt, wird die Zählvariable auf $0$ gesetzt. Da es sich um einen direkten
 Zugriff handelt, erfolgt das Löschen in konstanter Zeit. Die Laufzeitfunktion ist im schlechtesten Fall $T(n) = 3$.

Durch \autoref{alg:linepool-freememorypool} kann ein Speicherblock wieder freigegeben werden.
\begin{algorithm}[!ht]\small
\caption{\textproc{freeMemoryPool}}
\label{alg:linepool-freememorypool}
\begin{algorithmic}[1]
	\Require $p$
	\If{$\lnot p$}
	\Cost{$c_{1}$}{$1$}
	\label{alg:linepool-freememorypool-validpointer-start}
		\State \textbf{return}
		\Cost{$c_{2}$}{$1$}
	\EndIf
	\label{alg:linepool-freememorypool-validpointer-end}
	\State \Call{resetmemorypool}{$p$}
	\Cost{$c_{4}$}{$3$}
	\label{alg:linepool-freememorypool-resetmemory}
	\If{$p \geq \mathit{data} \land p \leq \mathit{data} + S$}
	\Cost{$c_{5}$}{$4$}
	\label{alg:linepool-freememorypool-checkpointer}
		\State $\mathit{pool} \gets p$
		\Cost{$c_{6}$}{$1$}
		\label{alg:linepool-freememorypool-savepointer}
	\EndIf
\end{algorithmic}
\end{algorithm}

Dazu wird der Zeiger $p$ auf Gültigkeit geprüft. Danach wird der Speicher durch \textproc{resetMemoryPool}
 (\autoref{alg:linepool-resetmemorypool}) gelöscht. In Zeile \ref{alg:linepool-freememorypool-checkpointer} wird
 überprüft, ob der Zeiger $p$ zu dem entsprechenden Block gehört, um danach die Adresse in Zeile
 \ref{alg:linepool-freememorypool-savepointer} in $\mathit{pool}$ zu speichern. Die Laufzeitfunktion ist im
 schlechtesten Fall $T(n) = 9$. Auch hier erfolgt das Freigeben des Speichers wieder in konstanter Zeit.

Die Anzahl der Einträge in einem Pool werden durch \autoref{alg:linepool-count} bestimmt, indem die Zählvariable
 $\mathit{count}$ zurückgegeben wird.
\begin{algorithm}[!ht]
\caption{\textproc{getLineCount}}
\label{alg:linepool-count}
\begin{algorithmic}[1]
	\Require $p$
	\If{$\lnot p$}
	\Cost{$c_{1}$}{$1$}
	\label{alg:linepool-count-validpointer-start}
		\State \textbf{return}
		\Cost{$c_{2}$}{$1$}
	\EndIf
	\label{alg:linepool-count-validpointer-end}
	\State \textbf{return} $\mathit{p.count}$
	\Cost{$c_{4}$}{$2$}
	\label{alg:linepool-count-counter}
\end{algorithmic}
\end{algorithm}

Die Laufzeitfunktion ist im schlechtesten Fall $T(n) = 3$ und der Zugriff auf die Variable erfolgt in konstanter Zeit.

% Im Verfahren nach \citeauthor{clarke96} werden Liniensegmente nicht aus dem Speicherpool gelöscht. Darum kann auf
%  einen Algorithmus zum löschen der Einträge, wie \autoref{alg:edgelpool-removeedgel} bei \glspl{edgel}, verzichtet
%  werden. Alle Operationen für Linien erfolgen somit in konstanter Zeit $T(n) = \Theta(1)$.

Alle vorgestellten Operationen für Linien erfolgen in konstanter Zeit $T(n) = \Theta(1)$.

% subsection datenstruktur-liniensegmente (end)

\subsubsection{Liniensegmente} % (fold)
\label{sub:datenstruktur-liniensegmente}

Die Datenstruktur eines Liniensegments und die Methoden zum hinzufügen, löschen und freigeben des Speichers sind nach
 dem Vorbild des Edgelspeichers aufgebaut. Die Datenstruktur eines Liniensegments ist in
 \autoref{alg:datastructure-line} definiert. Eine Linie besteht aus den \glspl{edgel} $s$ und $e$, die den Start- und
 End-Punkt der Linie darstellen. Variable $c$ speichert die Anzahl der unterstützenden Edgels der Linie. Die Lese- und
 Schreibzugriffe auf die Datenstruktur ist konstant.

\begin{algorithm}[ht]
\caption{Datenstruktur eines Liniensegments}
\label{alg:datastructure-line}
	\begin{algorithmic}[1]
		\State $s$
		\Comment Start des Liniensegments
		\State $e$
		\Comment Ende des Liniensegments
		\State $c$
		\Comment Anzahl der Support-Edgels
	\end{algorithmic}
\end{algorithm}


Die Datenstruktur eines Speicherpools für Linien in \autoref{alg:datastructure-linepool} besteht aus einem Array
 $\mathit{data}$ mit der festen Größe $N$ und einer Zählvariablen $\mathit{count}$.

\begin{algorithm}[ht]
\caption{Datenstruktur des Linien-Speicher}
\label{alg:datastructure-linepool}
\begin{algorithmic}[1]
	\State $\mathit{data}[N]$
	\Comment Anzahl der Einträge
	\State $\mathit{count}$
\end{algorithmic}
\end{algorithm}

\begin{algorithm}[ht]
\caption{Datenstruktur des Linien-Pools}
\label{alg:datastructure-linepool-implementation}
\begin{algorithmic}[1]
	\State $\mathit{data}[S]$
	\Comment Anzahl der Pools
	\State $\mathit{pool}$
\end{algorithmic}
\end{algorithm}

Der Speicherpool für Linien in \autoref{alg:datastructure-linepool-implementation} besteht wiederum aus einem Array
 $\mathit{data}$ mit der Anzahl $S$ der zur Verfügung stehenden Speicherpools. Der Pointer von $\mathit{data}$ wird in
 der Variablen $\mathit{pool}$ gespeichert. Der Zugriff auf die Datenstruktur erfolgt in konstanter Zeit.

Mehrere Speicherpools können mit \autoref{alg:linepool-getmemorypools} geholt werden und mit
 \autoref{alg:linepool-getmemorypool} wird ein Speicherpool angefordert. Der Aufbau entspricht dem des Speicherpool für
 \glspl{edgel} (Vgl. \autoref{alg:edgelpool-getmemorypools} und \autoref{alg:edgelpool-getmemorypool}). Der Zugriff
 erfolgt in konstanter Zeit.

\begin{algorithm}[ht]
\caption{Hole Linepools}
\label{alg:linepool-getmemorypools}
\begin{algorithmic}[1]
	\Require $n$
	\If{$\mathit{data} + S - \mathit{pool} \geq n$}
	\label{alg:linepool-getmemorypools-checkpoolsize}
		\State $\mathit{pool} \gets \mathit{pool} + n$
		\State \textbf{return} $\mathit{pool} - n$
	\Else
		\State \textbf{return} $\mathit{NULL}$
	\EndIf
\end{algorithmic}
\end{algorithm}


\begin{algorithm}[ht]
\caption{Hole Linepool}
\label{alg:linepool-getmemorypool}
\begin{algorithmic}[1]
	\State $p \gets$ \Call{getmemorypools}{1}
	\State \textbf{return} $p$
\end{algorithmic}
\end{algorithm}


Um eine Linie dem Speicherpool hinzuzufügen, wird \autoref{alg:linepool-addline} verwendet. Es wird ein Pointer $p$ auf
 den Speicherpool, sowie eine Linie $l$ übergeben. Wenn es sich um einen gültigen Pointer $p$ handelt und genügend
 freier Speicherplatz für eine weitere Linie vorhanden ist, wird in Zeile
 \ref{alg:linepool-addline-add-start}--\ref{alg:linepool-addline-add-end} die Linie hinzugefügt und die Zählvariable
 inkrementiert. Das Hinzufügen einer Linie ist konstant.

\begin{algorithm}[ht]
\caption{Linie hinzufügen}
\label{alg:linepool-addline}
\begin{algorithmic}[1]
	\Require $p,l$
	\If{$!p$}
	\label{alg:linepool-addline-validpointer-start}
		\State \textbf{return}
	\EndIf
	\label{alg:linepool-addline-validpointer-end}
	\If{$!p(\mathit{count}) < N$}
	\label{alg:linepool-addline-checkspace-start}
		\State \textbf{return} \Comment Speicher voll
	\EndIf
	\label{alg:linepool-addline-checkspace-end}
	\State $c \gets p(\mathit{count})$
	\label{alg:linepool-addline-add-start}
	\State $p(\mathit{data}[c]) \gets l$
	\State $p(\mathit{count}) \gets c + 1$
	\label{alg:linepool-addline-add-end}
\end{algorithmic}
\end{algorithm}


Zum auslesen einer Linie aus dem Speicherpool, wird \autoref{alg:linepool-getline} verwendet. Als Parameter werden ein
 Pointer $p$ und ein Index $i$ übergeben. Der Index gibt an, welche Linie aus dem Pool ausgelesen werden soll. In Zeile
 \ref{alg:linepool-getline-validrange-start} wird geprüft, ob der Index sich innerhalb der Grenzen der gespeicherten
 Linien befindet. Wenn dies der Fall ist, wird in Zeile~\ref{alg:linepool-getline-returnline} die Linie in konstanter
 Zeit zurückgegeben.

\begin{algorithm}[ht]
\caption{Linie auslesen}
\label{alg:linepool-getline}
\begin{algorithmic}[1]
	\Require $p,i$
	\If{$!p$}
	\label{alg:linepool-getline-validpointer-start}
		\State \textbf{return}
	\EndIf
	\label{alg:linepool-getline-validpointer-end}
	\State $c \gets p(\mathit{count})$
	\If{$! c > i$}
	\label{alg:linepool-getline-validrange-start}
		\State \textbf{return}
	\EndIf
	\label{alg:linepool-getline-validrange-end}
	\State \textbf{return} $p(\mathit{data}[i])$
	\label{alg:linepool-getline-returnline}
\end{algorithmic}
\end{algorithm}


Mit \autoref{alg:linepool-resetmemorypool} werden die Einträge im Speicherpool gelöscht. Dazu wird der Pointer $p$ auf
 den Speicherpool übergeben und in Zeile
 \ref{alg:linepool-resetmemorypool-validpointer-start}--\ref{alg:linepool-resetmemorypool-validpointer-end} überprüft.
 Wenn es sich um einen gültigen Pointer handelt, wird die Zählvariable auf $0$ gesetzt. Da es sich um einen direkten
 Zugriff handelt, erfolgt das Löschen in konstanter Zeit.

\begin{algorithm}[ht]
\caption{Lösche Daten in Linepool}
\label{alg:linepool-resetmemorypool}
\begin{algorithmic}[1]
	\Require $p$
	\If{$!p$}
	\label{alg:linepool-resetmemorypool-validpointer-start}
		\State \textbf{return}
	\EndIf
	\label{alg:linepool-resetmemorypool-validpointer-end}
	\State $p(\mathit{count}) \gets 0$
	\label{alg:linepool-resetmemorypool-reset}
\end{algorithmic}
\end{algorithm}


Durch \autoref{alg:linepool-freememorypool} kann ein Speicherpool wieder freigegeben werden. Dazu wird der Pointer $p$
 auf Gültigkeit geprüft. Danach wird der Speicher durch \textproc{resetmemorypool}
 (\autoref{alg:linepool-resetmemorypool}) gelöscht. In Zeile~\ref{alg:linepool-freememorypool-checkpointer} wird
 überprüft, ob der Pointer $p$ zu dem entsprechenden Pool gehört um danach die Adresse in Zeile
 \ref{alg:linepool-freememorypool-savepointer} im Speicherpool $\mathit{pool}$ zu speichern. Auch hier erfolgt das
 Freigeben des Speichers wieder in konstanter Zeit.

\begin{algorithm}[ht]
\caption{Speicher des Linienpool freigeben}
\label{alg:linepool-freememorypool}
\begin{algorithmic}[1]
	\Require $p$
	\If{$!p$}
	\label{alg:linepool-freememorypool-validpointer-start}
		\State \textbf{return}
	\EndIf
	\label{alg:linepool-freememorypool-validpointer-end}
	\State \Call{resetmemorypool}{$p$}
	\label{alg:linepool-freememorypool-resetmemory}
	\If{$p \geq \mathit{data} \land p \leq \mathit{data} + S$}
	\label{alg:linepool-freememorypool-checkpointer}
		\State $\mathit{pool} \gets p$
		\label{alg:linepool-freememorypool-savepointer}
	\EndIf
\end{algorithmic}
\end{algorithm}


Die Anzahl der Einträge in einem Pool werden durch \autoref{alg:linepool-count} bestimmt, indem die Zählvariable
 $\mathit{count}$ zurückgegeben wird. Der Zugriff auf die Variable erfolgt in konstanter Zeit.

\begin{algorithm}[ht]
\caption{Anzahl der Einträge}
\label{alg:linepool-count}
\begin{algorithmic}[1]
	\Require $p$
	\If{$!p$}
	\label{alg:linepool-count-validpointer-start}
		\State \textbf{return}
	\EndIf
	\label{alg:linepool-count-validpointer-end}
	\State \textbf{return} $p(\mathit{count})$
	\label{alg:linepool-count-counter}
\end{algorithmic}
\end{algorithm}


Im Verfahren nach \citeauthor{clarke96} gibt es keinen Grund Linien aus dem Speicherpool zu löschen. Darum kann auf
 einen Algorithmus wie \autoref{alg:edgelpool-removeedgel} bei \glspl{edgel} verzichtet werden.

Alle Operationen für Linien erfolgen somit in konstanter Zeit $T(n) = \Theta(1)$.

% subsection datenstruktur-liniensegmente (end)

\subsubsection{Liniensegmente} % (fold)
\label{sub:datenstruktur-liniensegmente}

Die Datenstruktur eines Liniensegments und die Methoden zum hinzufügen, löschen und freigeben des Speichers sind nach
 dem Vorbild des Edgelspeichers aufgebaut. Die Datenstruktur eines Liniensegments ist in
 \autoref{alg:datastructure-linesegment} definiert. Eine Linie besteht aus den \glspl{edgel} $\mathit{start}$ und
 $\mathit{end}$, die den Start- und Endpunkt der Linie darstellen. Die Variable $\mathit{slope}$ enthält die
 Orientierung des Liniensegments, während die Variable $\mathit{supportCount}$ die Anzahl der unterstützenden
 \glspl{edgel} der Linie speichert. $\mathit{remove}$, $\mathit{startCorner}$ und $\mathit{endCorner}$ sind boolesche
 Variablen. $\mathit{remove}$ dient im späteren Verlauf zur Erkennung, ob ein Liniensegment gelöscht werden muss,
 während $\mathit{startCorner}$ und $\mathit{endCorner}$ benutzt werden, ob eine Linie einen Eckpunkt am Anfang oder
 Ende besitzt. Die letzte Variable $\mathit{support}$ dient zur Speicherung von \glspl{edgel}, die eine Linienhypothese
 unterstützen. Die Lese- und Schreibzugriffe auf die Datenstruktur ist konstant.

\begin{algorithm}[!ht]
\caption{\textproc{lineSegment}}
\label{alg:datastructure-linesegment}
	\begin{algorithmic}[1]
		\State $\mathit{start}$
		\State $\mathit{end}$
		\State $\mathit{slope}$
		\State $\mathit{supportCount}$
		\State $\mathit{remove}$
		\State $\mathit{startCorner}$
		\State $\mathit{endCorner}$
		\State $\mathit{support}[\mathit{MAXEDGELS}]$
	\end{algorithmic}
\end{algorithm}


Die Methode \textproc{isOrientationCompatible} untersucht, ob zwei Liniensegmente $\mathit{left}$ und $\mathit{right}$
 fast parallel zueinander stehen (\autoref{alg:linesegmentisorientationcompatible}). Dazu wird mithilfe von
 \textproc{dotProduct} die Orientierung berechnet. Wenn die Orientierung im Bereich von (0.92,1] liegt, wird als
 Ergebnis wahr zurückgeliefert. Das bedeutet, dass die Orientierung der Linien im Bereich von $0^\circ$ bis
 $\sim 23^\circ$ liegt und die Linien als parallel betrachtet werden. Ansonsten wird als Ergebnis falsch
 zurückgegeben, was bedeutet, dass die Linien nicht parallel sind.

\begin{algorithm}[ht]
\caption{\textproc{isOrientationCompatible}}
\label{alg:linesegmentisorientationcompatible}
\begin{algorithmic}[1]
	\Require $\mathit{left}, \mathit{right}$
	\State \textbf{return} \Call{dotProduct}{$\mathit{left.slope}, \mathit{right.slope}$} $> 0.92$
\end{algorithmic}
\end{algorithm}


Die Datenstruktur eines Speichervorrats für Linien in \autoref{alg:datastructure-linesegmentpool} besteht aus einem
 Array $\mathit{data}$ mit der festen Größe $N$ und einer Zählvariablen $\mathit{count}$.

\begin{algorithm}[!ht]\small
\caption{\textproc{lineSegmentPool} (Datenstruktur)}
\label{alg:datastructure-linesegmentpool}
\begin{algorithmic}[1]
	\State $\mathit{data}[N]$
	\Comment Anzahl der Einträge
	\State $\mathit{count}$
\end{algorithmic}
\end{algorithm}


Der Speichervorrat für Linien in \autoref{alg:datastructure-linesegmentpool-implementation} besteht wiederum aus einem
 Array $\mathit{data}$ mit der Anzahl $S$ der zur Verfügung stehenden Speicherblöcke. Der Zeiger von $\mathit{data}$
 wird in der Variablen $\mathit{pool}$ gespeichert. Der Zugriff auf die Datenstruktur erfolgt in konstanter Zeit.

Mehrere Speicherblöcke können mit \autoref{alg:linepool-getmemorypools} angefordert werden und mit
 \autoref{alg:linepool-getmemorypool} wird ein Speicherblock angefordert. Der Aufbau der Verfahren entspricht den
 Verfahren des Speichervorrats für \glspl{edgel} (Vgl. \autoref{alg:edgelpool-getmemorypools} und
 \autoref{alg:edgelpool-getmemorypool}). Der Zugriff erfolgt in konstanter Zeit.

\begin{algorithm}[!ht]
\caption{\textproc{getMemoryPools}}
\label{alg:linepool-getmemorypools}
\begin{algorithmic}[1]
	\Require $n$
	\If{$\mathit{data} + S - \mathit{pool} \geq n$}
	\Cost{$c_{1}$}{$3$}
	\label{alg:linepool-getmemorypools-checkpoolsize}
		\State $\mathit{pool} \gets \mathit{pool} + n$
		\Cost{$c_{2}$}{$2$}
		\State \textbf{return} $\mathit{pool} - n$
		\Cost{$c_{3}$}{$2$}
	\Else
		\State \textbf{return} $\mathit{NULL}$
		\Cost{$c_{5}$}{$1$}
	\EndIf
\end{algorithmic}
\end{algorithm}


\input{alg/analyse-hirzer/datastructure-linesegmentpool-getmemorypool}

Um eine Linie zu einem Speicherblock hinzuzufügen, wird \autoref{alg:linesegmentpool-addline} verwendet. Es wird ein Zeiger
 $p$ auf den Speicherblock, sowie eine Linie $l$ übergeben. Wenn es sich um einen gültigen Zeiger $p$ handelt, und
 genügend freier Speicherplatz für eine weitere Linie vorhanden ist, wird in Zeile
 \ref{alg:linesegmentpool-addline-add-start}--\ref{alg:linesegmentpool-addline-add-end} die Linie hinzugefügt und die
 Zählvariable inkrementiert. Das Hinzufügen einer Linie ist konstant.

\begin{algorithm}[ht]
\caption{\textproc{addLineSegment}}
\label{alg:linesegmentpool-addline}
\begin{algorithmic}[1]
	\Require $p,l$
	\If{$\lnot p$}
	\label{alg:linesegmentpool-addline-validpointer-start}
		\State \textbf{return}
	\EndIf
	\label{alg:linesegmentpool-addline-validpointer-end}
	\If{$\lnot \left(p(\mathit{count}) < N\right)$}
	\label{alg:linepool-addline-checkspace-start}
		\State \textbf{return} \Comment Speicher voll
	\EndIf
	\label{alg:linesegmentpool-addline-checkspace-end}
	\State $c \gets p(\mathit{count})$
	\label{alg:linesegmentpool-addline-add-start}
	\State $p(\mathit{data}[c]) \gets l$
	\State $p(\mathit{count}) \gets c + 1$
	\label{alg:linesegmentpool-addline-add-end}
\end{algorithmic}
\end{algorithm}


Zum auslesen einer Linie aus dem Speicherblock, wird \autoref{alg:linepool-getline} verwendet. Als Parameter werden ein
 Zeiger $p$ und ein Index $i$ übergeben. Der Index gibt an, welche Linie aus dem Block ausgelesen werden soll. In Zeile
 \ref{alg:linepool-getline-validrange-start} wird geprüft, ob der Index sich innerhalb der Grenzen der gespeicherten
 Linien befindet. Wenn dies der Fall ist, wird in Zeile \ref{alg:linepool-getline-returnline} die Linie in konstanter
 Zeit zurückgegeben.

\begin{algorithm}[!ht]
\caption{\textproc{getLineSegment}}
\label{alg:linepool-getline}
\begin{algorithmic}[1]
	\Require $p,i$
	\If{$\lnot p$}
	\label{alg:linepool-getline-validpointer-start}
		\State \textbf{return}
	\EndIf
	\label{alg:linepool-getline-validpointer-end}
	\State $c \gets \mathit{p.count}$
	\If{$\lnot \left(c > i\right)$}
	\label{alg:linepool-getline-validrange-start}
		\State \textbf{return}
	\EndIf
	\label{alg:linepool-getline-validrange-end}
	\State \textbf{return} $\mathit{p.data}[i]$
	\label{alg:linepool-getline-returnline}
\end{algorithmic}
\end{algorithm}


Mit \autoref{alg:linepool-resetmemorypool} werden die Einträge im Speicherblock gelöscht. Dazu wird der Zeiger $p$ auf
 den Speicherblock übergeben und in Zeile
 \ref{alg:linepool-resetmemorypool-validpointer-start}--\ref{alg:linepool-resetmemorypool-validpointer-end} überprüft.
 Wenn es sich um einen gültigen Zeiger handelt, wird die Zählvariable auf $0$ gesetzt. Da es sich um einen direkten
 Zugriff handelt, erfolgt das Löschen in konstanter Zeit.

\begin{algorithm}[!ht]\small
\caption{\textproc{resetMemoryPool}}
\label{alg:linepool-resetmemorypool}
\begin{algorithmic}[1]
	\Require $p$
	\If{$\lnot p$}
	\Cost{$c_{1}$}{$1$}
	\label{alg:linepool-resetmemorypool-validpointer-start}
		\State \textbf{return}
		\Cost{$c_{2}$}{$1$}
	\EndIf
	\label{alg:linepool-resetmemorypool-validpointer-end}
	\State $\mathit{p.count} \gets 0$
	\Cost{$c_{4}$}{$2$}
	\label{alg:linepool-resetmemorypool-reset}
\end{algorithmic}
\end{algorithm}


Durch \autoref{alg:linepool-freememorypool} kann ein Speicherblock wieder freigegeben werden. Dazu wird der Zeiger $p$
 auf Gültigkeit geprüft. Danach wird der Speicher durch \textproc{resetMemoryPool}
 (\autoref{alg:linepool-resetmemorypool}) gelöscht. In Zeile \ref{alg:linepool-freememorypool-checkpointer} wird
 überprüft, ob der Zeiger $p$ zu dem entsprechenden Block gehört, um danach die Adresse in Zeile
 \ref{alg:linepool-freememorypool-savepointer} in $\mathit{pool}$ zu speichern. Auch hier erfolgt das Freigeben des
 Speichers wieder in konstanter Zeit.

\begin{algorithm}[!ht]\small
\caption{\textproc{freeMemoryPool}}
\label{alg:linepool-freememorypool}
\begin{algorithmic}[1]
	\Require $p$
	\If{$\lnot p$}
	\Cost{$c_{1}$}{$1$}
	\label{alg:linepool-freememorypool-validpointer-start}
		\State \textbf{return}
		\Cost{$c_{2}$}{$1$}
	\EndIf
	\label{alg:linepool-freememorypool-validpointer-end}
	\State \Call{resetmemorypool}{$p$}
	\Cost{$c_{4}$}{$3$}
	\label{alg:linepool-freememorypool-resetmemory}
	\If{$p \geq \mathit{data} \land p \leq \mathit{data} + S$}
	\Cost{$c_{5}$}{$4$}
	\label{alg:linepool-freememorypool-checkpointer}
		\State $\mathit{pool} \gets p$
		\Cost{$c_{6}$}{$1$}
		\label{alg:linepool-freememorypool-savepointer}
	\EndIf
\end{algorithmic}
\end{algorithm}


Die Anzahl der Einträge in einem Pool werden durch \autoref{alg:linepool-count} bestimmt, indem die Zählvariable
 $\mathit{count}$ zurückgegeben wird. Der Zugriff auf die Variable erfolgt in konstanter Zeit.

\begin{algorithm}[ht]
\caption{\textproc{getLineCount}}
\label{alg:linepool-count}
\begin{algorithmic}[1]
	\Require $p$
	\If{$\lnot p$}
	\label{alg:linepool-count-validpointer-start}
		\State \textbf{return}
	\EndIf
	\label{alg:linepool-count-validpointer-end}
	\State \textbf{return} $p(\mathit{count})$
	\label{alg:linepool-count-counter}
\end{algorithmic}
\end{algorithm}


Im Verfahren nach \citeauthor{clarke96} werden Liniensegmente nicht aus dem Speicherpool gelöscht. Darum kann auf
 einen Algorithmus wie \autoref{alg:edgelpool-removeedgel} bei \glspl{edgel} verzichtet werden.

Alle Operationen für Linien erfolgen somit in konstanter Zeit $T(n) = \Theta(1)$.

% subsection datenstruktur-liniensegmente (end)

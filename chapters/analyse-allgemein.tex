\section{Allgemein} % (fold)
\label{sec:allgemein}

Die in diesem Abschnitt vorgestellten Algorithmen sind sowohl in ARToolKit als auch in dem Verfahren nach Hirzer
 anzutreffen.

\ldots

\citeauthor{clarke96} verwenden in ihrem Verfahren ein monochromes Bildsignal $I_m$\footcite[Vgl.][S.~417]{clarke96}.
 Die Konvertierung des Bildsignals $I$ von YCbCr in $I_m$ erfolgt durch \autoref{alg:convertmonochrome}. Wie in
 \autoref{sub:farbräume} beschrieben, besteht ein YCbCr Signal aus einem Luminaz Kanal $Y$ und den Chroma Abweichungen
 $Cb$ und $Cr$. Um ein monochromes Signal $I_m$ zu erstellen, muss der Luminanz Kanal ausgelesen und in einen Puffer
 kopiert werden.

\begin{algorithm}[ht]
\caption{Konvertierung zu monochromen Bildsignal}
\label{alg:convertmonochrome}
	\begin{algorithmic}[1]
		\Require $I, I_m$
		\State $Y \gets$ \Call{baseAddress}{$I$}
		\label{alg:convertmonochrome-baseaddress}
		\State $w \gets$ \Call{width}{$I$}
		\State $h \gets$ \Call{height}{$I$}
		\State $l \gets w \cdot h$
		\State $I_m \gets$ \Call{copy}{$I, Y, l$}
	\end{algorithmic}
\end{algorithm}


Der Algorithmus verwendet als Parameter das Bildsignal $I$ und einen Pointer $I_m$ auf einen Puffer für das monochrome
 Signal. Der Monochromepuffer $I_m$ ist ein Array mit fester Größe, das beim initialisieren einmalig angelegt wird und
 danach wiederverwendet werden kann. In Zeile \ref{alg:convertmonochrome-baseaddress} wird die Adresse des
 Luminanz-Kanals $Y$ ausgelesen. Die Funktionen \textproc{width} und \textproc{height} liefern die Breite und Höhe des
 Signals in Pixeln, mit denen die Länge der Daten berechnet wird. Anschließend werden die Daten in den Puffer kopiert.
 Die Laufzeit des Algorithmus entspricht $\Theta(1)$. (Vorsicht: Nur Zeile 5 verwendet keine Funktionen, deren Laufzeit
 dir nicht bekannt sind. baseaddress, width und height greifen evtl. auf metadaten zurück und wären damit ein einfacher
 lookup mit $\Theta(1)$. Dann wäre nur noch memcpy zu bestimmen, was im schlimmsten Fall $\Theta(n)$ wäre.)

Um auf \gls{pixel} zugreifen zu können, verwende ich \autoref{alg:getpixel}. Es wird der Puffer $I_m$ als Pointer
 übergeben und die Position $x$ und $y$ des gewünschten \gls{pixel}. $w$ und $h$ entsprechen der Breite und Höhe von
 $I_m$. Zeile \ref{alg:getpixel-startcheck} bis Zeile \ref{alg:getpixel-stopcheck} sorgen dafür, dass keine Werte
 außerhalb des Puffers gelesen werden können. Dies ist für die Randbehandlung bei Faltungsoperationen
 (Vgl. \autoref{sub:filter}) wichtig und wiederholt den \gls{pixel}.

\begin{algorithm}
\caption{Pixelwert auslesen}
\label{alg:getpixel}
	\begin{algorithmic}[1]
		\Require $I_m, x, y, w, h$
		\Ensure $I_m[i]$
		\If{$x < 0$}
		\label{alg:getpixel-startcheck}
			\State $x \gets 0$
		\EndIf
		\If{$y < 0$}
			\State $y \gets 0$
		\EndIf
		\If{$x \geq w$}
			\State $x \gets w - 1$
		\EndIf
		\If{$y \geq h$}
			\State $y \gets h -1$
		\EndIf
		\label{alg:getpixel-stopcheck}
		\State $i \gets x + \left(y \cdot w\right)$
		\State \textbf{return} $I_m[i]$
	\end{algorithmic}	
\end{algorithm}


Die Laufzeit von \autoref{alg:getpixel} ist im worst-case und im best-case konstant und somit $\Theta(1)$.

% section allgemein (end)

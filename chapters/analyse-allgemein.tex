\section{Allgemein} % (fold)
\label{sec:allgemein}

Bei ARToolKitPlus und dem Verfahren nach Hirzer wird das Bildsignal durch AVFoundation zur Verfügung gestellt. Um
 auszuschliessen, dass eine Kombination von Bildformat und Auflösung zu einem Leistungsproblem wird, wurde die
 Geschwindigkeit des Systems untersucht.

AVFoundation bietet auf einem iPod touch die Möglichkeit, Bildsignale in den Auflösungen von $192 \times 144$ (Low),
 $480 \times 360$ (Medium), $640 \times 480$ und $1280 \times 720$ (High) anzufordern. Als Bildformat kann entweder
 YCbCr oder RGBA gewählt werden. In \autoref{fig:videoperformance-rgb} wurde für jede unterstützte
 Auflösung mit dem Bildformat RGBA $300$ Messdaten erfasst.
\begin{figure}[!ht]
	\centering
	\subfigure[RGBA Low]{
		\input{resources/RGB-Low.pdf_tex}
		\label{fig:videoperformance-rgb-low}
	}
	\subfigure[RGBA Medium]{
		\input{resources/RGB-Medium.pdf_tex}
		\label{fig:videoperformance-rgb-medium}
	}
	\subfigure[RGBA $640 \times 480$]{
		\input{resources/RGB-640x480.pdf_tex}
		\label{fig:videoperformance-rgb-640}
	}
	\subfigure[RGBA High]{
		\input{resources/RGB-High.pdf_tex}
		\label{fig:videoperformance-rgb-high}
	}
	\caption{RGBA Analyse für die Auflösung Low (\autoref{fig:videoperformance-rgb-low}),
	 Medium (\autoref{fig:videoperformance-rgb-medium}), $640 \times 480$ (\autoref{fig:videoperformance-rgb-640}) und
	 High (\autoref{fig:videoperformance-rgb-high}). Der Median ist als blaue Linie eingezeichnet.}
	\label{fig:videoperformance-rgb}
\end{figure}
Bei der kleinsten Auflösung (\autoref{fig:videoperformance-rgb-low}) liegt der Median der Bildwiederholfrequenz bei $15$
 FPS. Bei den restlichen Auflösungen beträgt der Median der Bildwiederholfrequenz $30$ FPS. In
 \autoref{fig:videoperformance-ycbcr} wurde die Analyse für das Videoformat YCbCr wiederholt.
\begin{figure}[!ht]
	\centering
	\subfigure[YCbCr Low]{
		\input{resources/YCbCr-Low.pdf_tex}
		\label{fig:videoperformance-ycbcr-low}
	}
	\subfigure[YCbCr Medium]{
		\input{resources/YCbCr-Medium.pdf_tex}
		\label{fig:videoperformance-ycbcr-medium}
	}
	\subfigure[YCbCr $640 \times 480$]{
		\input{resources/YCbCr-640x480.pdf_tex}
		\label{fig:videoperformance-ycbcr-640}
	}
	\subfigure[YCbCr High]{
		\input{resources/YCbCr-High.pdf_tex}
		\label{fig:videoperformance-ycbcr-high}
	}
	\caption{YCbCr Analyse für die Auflösung Low (\autoref{fig:videoperformance-ycbcr-low}),
	 Medium (\autoref{fig:videoperformance-ycbcr-medium}), $640 \times 480$ (\autoref{fig:videoperformance-ycbcr-640})
	 und High (\autoref{fig:videoperformance-ycbcr-high}). Der Median ist als blaue Linie eingezeichnet.}
	\label{fig:videoperformance-ycbcr}
\end{figure}
Auch bei dieser Untersuchung liegt der Median bei der kleinsten Auflösung (\autoref{fig:videoperformance-ycbcr-low})
 bei $15$ FPS. Die restlichen Auflösungen erreichen eine Bildwiederholrate von $30$ FPS.

Bei beiden Bildformaten werden, unabhängig von der Auflösung, gleiche Ergebnisse erzielt. Die wenigsten Abweichung der
 Bildwiederholfrequenz sind in \autoref{fig:videoperformance-rgb-640} und \autoref{fig:videoperformance-ycbcr-640} für die
 Auflösung von $640 \times 480$ \gls{pixel} zu beobachten. Für die Analyse der Verfahren wird die Auflösung
 $640 \times 480$ mit dem Bildformat RGBA verwendet.

% section allgemein (end)

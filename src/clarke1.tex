\begin{algorithm}
	\caption{Line Detection nach \citeauthor{clarke96}}
	\label{src:lineDetection}
	\begin{algorithmic}[1]
	\Procedure{LineDetection}{$I_m$}
		\State $I_m$ in $40 \times 40$ Pixel große Regionen unterteilen
		\For{$i \gets 0$, alle Regionen}
			\State $Liste \gets \infty$
			\State $i$ in horizontale und vertikale Scanlines unterteilen mit $5$ Pixel Abstand
			\For{$j \gets 0$, alle Pixel auf allen Scanlines}
				\State $x \gets$ Falte $I_m\left(j\right)$ mit Kernel
				\If{$x > Schwellwert$}
					\Comment Edgel gefunden
					\State Orientierung des Edgels berechnen
					\State $Liste \gets edgel$
				\EndIf
			\EndFor
			\State $k \gets 0$
			\While{$Liste \not=0 \lor k <$ maximale Iterationen}
				\For{$j \gets 1, 25$}
					\State $Linie \gets$ zwei zufällige Edgels mit kompatibler Orientierung aus $Liste$ wählen
					\State Anzahl von Support-Edgels in der Nähe der Linie bestimmen
				\EndFor
				\If{Anzahl von Support-Edgels > Schwellwert}
					\Comment Linie wurde erkannt
					\State $Liste \gets Liste -$ Support-Edgels der Linie
				\EndIf
				\State $k \gets k + 1$
			\EndWhile
		\EndFor
	\EndProcedure
	\end{algorithmic}
\end{algorithm}

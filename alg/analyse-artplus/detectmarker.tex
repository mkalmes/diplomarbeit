\begin{algorithm}[ht]
\caption{\textproc{arDetectMarker}}
\label{alg:detectmarker}
\begin{algorithmic}[1]
	\Require $I,\mathit{thresh},\mathit{marker\_info},\mathit{marker\_num}$
	\State $I_l \gets$ \textproc{NULL}
	\label{alg:detectmarker-init-start}
	\State $\mathit{label\_num},\mathit{area},\mathit{clip},\mathit{label\_ref},\mathit{pos} \gets \infty$
	\label{alg:detectmarker-init-end}
	\State \Call{autoThreshold.reset}{}
	\State \Call{checkImageBuffer}{}
	\State $\mathit{marker\_num} \gets 0$
	% AR_AREA_MAX (100000) und AR_AREA_MIN (70) sind defines
	% wmarker_num globale variable
	\State $I_l \gets$ \textproc{arDetectMarker2}$\left(
	\begin{aligned}
			& I,\mathit{thresh},\mathit{label\_ref},\mathit{area},\mathit{pos},\mathit{clip},\\
			& \mathit{AR\_AREA\_MAX}, \mathit{AR\_AREA\_MIN},\\
			& 1.0, \mathit{wmarker\_num}
	\end{aligned}\right)$
	\label{alg:detectmarker-call-method}
	\If{$I_l$}
		\label{alg:detectmarker-found-rectangle}
		\State \ldots \Comment{Weitere Anweisungen zur Identifikation einer Marke.}
	\EndIf
	\State \ldots \Comment{Weitere Anweisungen zur Identifikation einer Marke.}
\end{algorithmic}
\end{algorithm}
